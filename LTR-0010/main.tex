\documentclass{ximera}
%%% Begin Laad packages

\makeatletter
\@ifclassloaded{xourse}{%
    \typeout{Start loading preamble.tex (in a XOURSE)}%
    \def\isXourse{true}   % automatically defined; pre 112022 it had to be set 'manually' in a xourse
}{%
    \typeout{Start loading preamble.tex (NOT in a XOURSE)}%
}
\makeatother

\def\isEn\true 

\pgfplotsset{compat=1.16}

\usepackage{currfile}

% 201908/202301: PAS OP: babel en doclicense lijken problemen te veroorzaken in .jax bestand
% (wegens syntax error met toegevoegde \newcommands ...)
\pdfOnly{
    \usepackage[type={CC},modifier={by-nc-sa},version={4.0}]{doclicense}
    %\usepackage[hyperxmp=false,type={CC},modifier={by-nc-sa},version={4.0}]{doclicense}
    %%% \usepackage[dutch]{babel}
}



\usepackage[utf8]{inputenc}
\usepackage{morewrites}   % nav zomercursus (answer...?)
\usepackage{multirow}
\usepackage{multicol}
\usepackage{tikzsymbols}
\usepackage{ifthen}
%\usepackage{animate} BREAKS HTML STRUCTURE USED BY XIMERA
\usepackage{relsize}

\usepackage{eurosym}    % \euro  (€ werkt niet in xake ...?)
\usepackage{fontawesome} % smileys etc

% Nuttig als ook interactieve beamer slides worden voorzien:
\providecommand{\p}{} % default nothing ; potentially usefull for slides: redefine as \pause
%providecommand{\p}{\pause}

    % Layout-parameters voor het onderschrift bij figuren
\usepackage[margin=10pt,font=small,labelfont=bf, labelsep=endash,format=hang]{caption}
%\usepackage{caption} % captionof
%\usepackage{pdflscape}    % landscape environment

% Met "\newcommand\showtodonotes{}" kan je todonotes tonen (in pdf/online)
% 201908: online werkt het niet (goed)
\providecommand\showtodonotes{disable}
\providecommand\todo[1]{\typeout{TODO #1}}
%\usepackage[\showtodonotes]{todonotes}
%\usepackage{todonotes}

%
% Poging tot aanpassen layout
%
\usepackage{tcolorbox}
\tcbuselibrary{theorems}

%%% Einde laad packages

%%% Begin Ximera specifieke zaken

\graphicspath{
	{../../}
	{../}
	{./}
  	{../../pictures/}
   	{../pictures/}
   	{./pictures/}
	{./explog/}    % M05 in groeimodellen       
}

%%% Einde Ximera specifieke zaken

%
% define softer blue/red/green, use KU Leuven base colors for blue (and dark orange for red ?)
%
% todo: rather redefine blue/red/green ...?
%\definecolor{xmblue}{rgb}{0.01, 0.31, 0.59}
%\definecolor{xmred}{rgb}{0.89, 0.02, 0.17}
\definecolor{xmdarkblue}{rgb}{0.122, 0.671, 0.835}   % KU Leuven Blauw
\definecolor{xmblue}{rgb}{0.114, 0.553, 0.69}        % KU Leuven Blauw
\definecolor{xmgreen}{rgb}{0.13, 0.55, 0.13}         % No KULeuven variant for green found ...

\definecolor{xmaccent}{rgb}{0.867, 0.541, 0.18}      % KU Leuven Accent (orange ...)
\definecolor{kuaccent}{rgb}{0.867, 0.541, 0.18}      % KU Leuven Accent (orange ...)

\colorlet{xmred}{xmaccent!50!black}                  % Darker version of KU Leuven Accent

\providecommand{\blue}[1]{{\color{blue}#1}}    
\providecommand{\red}[1]{{\color{red}#1}}

\renewcommand\CancelColor{\color{xmaccent!50!black}}

% werkt in math en text mode om MATH met oranje (of grijze...)  achtergond te tonen (ook \important{\text{blabla}} lijkt te werken)
%\newcommand{\important}[1]{\ensuremath{\colorbox{xmaccent!50!white}{$#1$}}}   % werkt niet in Mathjax
%\newcommand{\important}[1]{\ensuremath{\colorbox{lightgray}{$#1$}}}
\newcommand{\important}[1]{\ensuremath{\colorbox{orange}{$#1$}}}   % TODO: kleur aanpassen voor mathjax; wordt overschreven infra!


% Uitzonderlijk kan met \pdfnl in de PDF een newline worden geforceerd, die online niet nodig/nuttig is omdat daar de regellengte hoe dan ook niet gekend is.
\ifdefined\HCode%
\providecommand{\pdfnl}{}%
\else%
\providecommand{\pdfnl}{%
  \\%
}%
\fi

% Uitzonderlijk kan met \handoutnl in de handout-PDF een newline worden geforceerd, die noch online noch in de PDF-met-antwoorden nuttig is.
\ifdefined\HCode
\providecommand{\handoutnl}{}
\else
\providecommand{\handoutnl}{%
\ifhandout%
  \nl%
\fi%
}
\fi



% \cellcolor IGNORED by tex4ht ?
% \begin{center} seems not to wordk
    % (missing margin-left: auto;   on tabular-inside-center ???)
%\newcommand{\importantcell}[1]{\ensuremath{\cellcolor{lightgray}#1}}  %  in tabular; usablility to be checked
\providecommand{\importantcell}[1]{\ensuremath{#1}}     % no mathjax2 support for colloring array cells

\pdfOnly{
  \renewcommand{\important}[1]{\ensuremath{\colorbox{kuaccent!50!white}{$#1$}}}
  \renewcommand{\importantcell}[1]{\ensuremath{\cellcolor{kuaccent!40!white}#1}}   
}

%%% Tikz styles


\pgfplotsset{compat=1.16}

\usetikzlibrary{trees,positioning,arrows,fit,shapes,math,calc,decorations.markings,through,intersections,patterns,matrix}

\usetikzlibrary{decorations.pathreplacing,backgrounds}    % 5/2023: from experimental


\usetikzlibrary{angles,quotes}

\usepgfplotslibrary{fillbetween} % bepaalde_integraal
\usepgfplotslibrary{polar}    % oa voor poolcoordinaten.tex

\pgfplotsset{ownstyle/.style={axis lines = center, axis equal image, xlabel = $x$, ylabel = $y$, enlargelimits}} 

\pgfplotsset{
	plot/.style={no marks,samples=50}
}

\newcommand{\xmPlotsColor}{
	\pgfplotsset{
		plot1/.style={darkgray,no marks,samples=100},
		plot2/.style={lightgray,no marks,samples=100},
		plotresult/.style={blue,no marks,samples=100},
		plotblue/.style={blue,no marks,samples=100},
		plotred/.style={red,no marks,samples=100},
		plotgreen/.style={green,no marks,samples=100},
		plotpurple/.style={purple,no marks,samples=100}
	}
}
\newcommand{\xmPlotsBlackWhite}{
	\pgfplotsset{
		plot1/.style={black,loosely dashed,no marks,samples=100},
		plot2/.style={black,loosely dotted,no marks,samples=100},
		plotresult/.style={black,no marks,samples=100},
		plotblue/.style={black,no marks,samples=100},
		plotred/.style={black,dotted,no marks,samples=100},
		plotgreen/.style={black,dashed,no marks,samples=100},
		plotpurple/.style={black,dashdotted,no marks,samples=100}
	}
}


\newcommand{\xmPlotsColorAndStyle}{
	\pgfplotsset{
		plot1/.style={darkgray,no marks,samples=100},
		plot2/.style={lightgray,no marks,samples=100},
		plotresult/.style={blue,no marks,samples=100},
		plotblue/.style={xmblue,no marks,samples=100},
		plotred/.style={xmred,dashed,thick,no marks,samples=100},
		plotgreen/.style={xmgreen,dotted,very thick,no marks,samples=100},
		plotpurple/.style={purple,no marks,samples=100}
	}
}


%\iftikzexport
\xmPlotsColorAndStyle
%\else
%\xmPlotsBlackWhite
%\fi
%%%


%
% Om venndiagrammen te arceren ...
%
\makeatletter
\pgfdeclarepatternformonly[\hatchdistance,\hatchthickness]{north east hatch}% name
{\pgfqpoint{-1pt}{-1pt}}% below left
{\pgfqpoint{\hatchdistance}{\hatchdistance}}% above right
{\pgfpoint{\hatchdistance-1pt}{\hatchdistance-1pt}}%
{
	\pgfsetcolor{\tikz@pattern@color}
	\pgfsetlinewidth{\hatchthickness}
	\pgfpathmoveto{\pgfqpoint{0pt}{0pt}}
	\pgfpathlineto{\pgfqpoint{\hatchdistance}{\hatchdistance}}
	\pgfusepath{stroke}
}
\pgfdeclarepatternformonly[\hatchdistance,\hatchthickness]{north west hatch}% name
{\pgfqpoint{-\hatchthickness}{-\hatchthickness}}% below left
{\pgfqpoint{\hatchdistance+\hatchthickness}{\hatchdistance+\hatchthickness}}% above right
{\pgfpoint{\hatchdistance}{\hatchdistance}}%
{
	\pgfsetcolor{\tikz@pattern@color}
	\pgfsetlinewidth{\hatchthickness}
	\pgfpathmoveto{\pgfqpoint{\hatchdistance+\hatchthickness}{-\hatchthickness}}
	\pgfpathlineto{\pgfqpoint{-\hatchthickness}{\hatchdistance+\hatchthickness}}
	\pgfusepath{stroke}
}
%\makeatother

\tikzset{
    hatch distance/.store in=\hatchdistance,
    hatch distance=10pt,
    hatch thickness/.store in=\hatchthickness,
   	hatch thickness=2pt
}

\colorlet{circle edge}{black}
\colorlet{circle area}{blue!20}


\tikzset{
    filled/.style={fill=green!30, draw=circle edge, thick},
    arceerl/.style={pattern=north east hatch, pattern color=blue!50, draw=circle edge},
    arceerr/.style={pattern=north west hatch, pattern color=yellow!50, draw=circle edge},
    outline/.style={draw=circle edge, thick}
}




%%% Updaten commando's
\def\hoofding #1#2#3{\maketitle}     % OBSOLETE ??

% we willen (bijna) altijd \geqslant ipv \geq ...!
\newcommand{\geqnoslant}{\geq}
\renewcommand{\geq}{\geqslant}
\newcommand{\leqnoslant}{\leq}
\renewcommand{\leq}{\leqslant}

% Todo: (201908) waarom komt er (soms) underlined voor emph ...?
\renewcommand{\emph}[1]{\textit{#1}}

% API commando's

\newcommand{\ds}{\displaystyle}
\newcommand{\ts}{\textstyle}  % tegenhanger van \ds   (Ximera zet PER  DEFAULT \ds!)

% uit Zomercursus-macro's: 
\newcommand{\bron}[1]{\begin{scriptsize} \emph{#1} \end{scriptsize}}     % deprecated ...?


%definities nieuwe commando's - afkortingen veel gebruikte symbolen
\newcommand{\R}{\ensuremath{\mathbb{R}}}
\newcommand{\Rnul}{\ensuremath{\mathbb{R}_0}}
\newcommand{\Reen}{\ensuremath{\mathbb{R}\setminus\{1\}}}
\newcommand{\Rnuleen}{\ensuremath{\mathbb{R}\setminus\{0,1\}}}
\newcommand{\Rplus}{\ensuremath{\mathbb{R}^+}}
\newcommand{\Rmin}{\ensuremath{\mathbb{R}^-}}
\newcommand{\Rnulplus}{\ensuremath{\mathbb{R}_0^+}}
\newcommand{\Rnulmin}{\ensuremath{\mathbb{R}_0^-}}
\newcommand{\Rnuleenplus}{\ensuremath{\mathbb{R}^+\setminus\{0,1\}}}
\newcommand{\N}{\ensuremath{\mathbb{N}}}
\newcommand{\Nnul}{\ensuremath{\mathbb{N}_0}}
\newcommand{\Z}{\ensuremath{\mathbb{Z}}}
\newcommand{\Znul}{\ensuremath{\mathbb{Z}_0}}
\newcommand{\Zplus}{\ensuremath{\mathbb{Z}^+}}
\newcommand{\Zmin}{\ensuremath{\mathbb{Z}^-}}
\newcommand{\Znulplus}{\ensuremath{\mathbb{Z}_0^+}}
\newcommand{\Znulmin}{\ensuremath{\mathbb{Z}_0^-}}
\newcommand{\C}{\ensuremath{\mathbb{C}}}
\newcommand{\Cnul}{\ensuremath{\mathbb{C}_0}}
\newcommand{\Cplus}{\ensuremath{\mathbb{C}^+}}
\newcommand{\Cmin}{\ensuremath{\mathbb{C}^-}}
\newcommand{\Cnulplus}{\ensuremath{\mathbb{C}_0^+}}
\newcommand{\Cnulmin}{\ensuremath{\mathbb{C}_0^-}}
\newcommand{\Q}{\ensuremath{\mathbb{Q}}}
\newcommand{\Qnul}{\ensuremath{\mathbb{Q}_0}}
\newcommand{\Qplus}{\ensuremath{\mathbb{Q}^+}}
\newcommand{\Qmin}{\ensuremath{\mathbb{Q}^-}}
\newcommand{\Qnulplus}{\ensuremath{\mathbb{Q}_0^+}}
\newcommand{\Qnulmin}{\ensuremath{\mathbb{Q}_0^-}}

\newcommand{\perdef}{\overset{\mathrm{def}}{=}}
\newcommand{\pernot}{\overset{\mathrm{notatie}}{=}}
\newcommand\perinderdaad{\overset{!}{=}}     % voorlopig gebruikt in limietenrekenregels
\newcommand\perhaps{\overset{?}{=}}          % voorlopig gebruikt in limietenrekenregels

\newcommand{\degree}{^\circ}


\DeclareMathOperator{\dom}{dom}     % domein
\DeclareMathOperator{\codom}{codom} % codomein
\DeclareMathOperator{\bld}{bld}     % beeld
\DeclareMathOperator{\graf}{graf}   % grafiek
\DeclareMathOperator{\rico}{rico}   % richtingcoëfficient
\DeclareMathOperator{\co}{co}       % coordinaat
\DeclareMathOperator{\gr}{gr}       % graad

\newcommand{\func}[5]{\ensuremath{#1: #2 \rightarrow #3: #4 \mapsto #5}} % Easy to write a function


% Operators
\DeclareMathOperator{\bgsin}{bgsin}
\DeclareMathOperator{\bgcos}{bgcos}
\DeclareMathOperator{\bgtan}{bgtan}
\DeclareMathOperator{\bgcot}{bgcot}
\DeclareMathOperator{\bgsinh}{bgsinh}
\DeclareMathOperator{\bgcosh}{bgcosh}
\DeclareMathOperator{\bgtanh}{bgtanh}
\DeclareMathOperator{\bgcoth}{bgcoth}

% Oude \Bgsin etc deprecated: gebruik \bgsin, en herdefinieer dat als je Bgsin wil!
%\DeclareMathOperator{\cosec}{cosec}    % not used? gebruik \csc en herdefinieer

% operatoren voor differentialen: to be verified; 1/2020: inconsequent gebruik bij afgeleiden/integralen
\renewcommand{\d}{\mathrm{d}}
\newcommand{\dx}{\d x}
\newcommand{\dd}[1]{\frac{\mathrm{d}}{\mathrm{d}#1}}
\newcommand{\ddx}{\dd{x}}

% om in voorbeelden/oefeningen de notatie voor afgeleiden te kunnen kiezen
% Usage: \afg{(2\sin(x))}  (en wordt d/dx, of accent, of D )
%\newcommand{\afg}[1]{{#1}'}
\newcommand{\afg}[1]{\left(#1\right)'}
%\renewcommand{\afg}[1]{\frac{\mathrm{d}#1}{\mathrm{d}x}}   % include in relevant exercises ...
%\renewcommand{\afg}[1]{D{#1}}

%
% \xmxxx commands: Extra KU Leuven functionaliteit van, boven of naast Ximera
%   ( Conventie 8/2019: xm+nederlandse omschrijving, maar is niet consequent gevolgd, en misschien ook niet erg handig !)
%
% (Met een minimale ximera.cls en preamble.tex zou een bruikbare .pdf moeten kunnen worden gemaakt van eender welke ximera)
%
% Usage: \xmtitle[Mijn korte abstract]{Mijn titel}{Mijn abstract}
% Eerste command na \begin{document}:
%  -> definieert de \title
%  -> definieert de abstract
%  -> doet \maketitle ( dus: print de hoofding als 'chapter' of 'sectie')
% Optionele parameter geeft eenn kort abstract (die met de globale setting \xmshortabstract{} al dan niet kan worden geprint.
% De optionele korte abstract kan worden gebruikt voor pseudo-grappige abtsarts, dus dus globaal al dan niet kunnen worden gebuikt...
% Globale settings:
%  de (optionele) 'korte abstract' wordt enkele getoond als \xmshortabstract is gezet
\providecommand\xmshortabstract{} % default: print (only!) short abstract if present
\newcommand{\xmtitle}[3][]{
	\title{#2}
	\begin{abstract}
		\ifdefined\xmshortabstract
		\ifstrempty{#1}{%
			#3
		}{%
			#1
		}%
		\else
		#3
		\fi
	\end{abstract}
	\maketitle
}

% 
% Kleine grapjes: moeten zonder verder gevolg kunnen worden verwijderd
%
%\newcommand{\xmopje}[1]{{\small#1{\reversemarginpar\marginpar{\Smiley}}}}   % probleem in floats!!
\newtoggle{showxmopje}
\toggletrue{showxmopje}

\newcommand{\xmopje}[1]{%
   \iftoggle{showxmopje}{#1}{}%
}


% -> geef een abstracte-formule-met-rechts-een-concreet-voorbeeld
% VB:  \formulevb{a^2+b^2=c^2}{3^2+4^2=5^2}
%
\ifdefined\HCode
\NewEnviron{xmdiv}[1]{\HCode{\Hnewline<div class="#1">\Hnewline}\BODY{\HCode{\Hnewline</div>\Hnewline}}}
\else
\NewEnviron{xmdiv}[1]{\BODY}
\fi

\providecommand{\formulevb}[2]{
	{\centering

    \begin{xmdiv}{xmformulevb}    % zie css voor online layout !!!
	\begin{tabular}{lcl}
		\important{#1}
		&  &
		Vb: $#2$
		\end{tabular}
	\end{xmdiv}

	}
}

\ifdefined\HCode
\providecommand{\vb}[1]{%
    \HCode{\Hnewline<span class="xmvb">}#1\HCode{</span>\Hnewline}%
}
\else
\providecommand{\vb}[1]{
    \colorbox{blue!10}{#1}
}
\fi

\ifdefined\HCode
\providecommand{\xmcolorbox}[2]{
	\HCode{\Hnewline<div class="xmcolorbox">\Hnewline}#2\HCode{\Hnewline</div>\Hnewline}
}
\else
\providecommand{\xmcolorbox}[2]{
  \cellcolor{#1}#2
}
\fi


\ifdefined\HCode
\providecommand{\xmopmerking}[1]{
 \HCode{\Hnewline<div class="xmopmerking">\Hnewline}#1\HCode{\Hnewline</div>\Hnewline}
}
\else
\providecommand{\xmopmerking}[1]{
	{\footnotesize #1}
}
\fi
% \providecommand{\voorbeeld}[1]{
% 	\colorbox{blue!10}{$#1$}
% }



% Hernoem Proof naar Bewijs, nodig voor HTML versie
\renewcommand*{\proofname}{Bewijs}

% Om opgave van oefening (wordt niet geprint bij oplossingenblad)
% (to be tested test)
\NewEnviron{statement}{\BODY}

% Environment 'oplossing' en 'uitkomst'
% voor resp. volledige 'uitwerking' dan wel 'enkel eindresultaat'
% geimplementeerd via feedback, omdat er in de ximera-server adhoc feedback-code is toegevoegd
%% Niet tonen indien handout
%% Te gebruiken om volledige oplossingen/uitwerkingen van oefeningen te tonen
%% \begin{oplossing}        De optelling is commutatief \end{oplossing}  : verschijnt online enkel 'op vraag'
%% \begin{oplossing}[toon]  De optelling is commutatief \end{oplossing}  : verschijnt steeds onmiddellijk online (bv te gebruiken bij voorbeelden) 

\ifhandout%
    \NewEnviron{oplossing}[1][onzichtbaar]%
    {%
    \ifthenelse{\equal{\detokenize{#1}}{\detokenize{toon}}}
    {
    \def\PH@Command{#1}% Use PH@Command to hold the content and be a target for "\expandafter" to expand once.

    \begin{trivlist}% Begin the trivlist to use formating of the "Feedback" label.
    \item[\hskip \labelsep\small\slshape\bfseries Oplossing% Format the "Feedback" label. Don't forget the space.
    %(\texttt{\detokenize\expandafter{\PH@Command}}):% Format (and detokenize) the condition for feedback to trigger
    \hspace{2ex}]\small%\slshape% Insert some space before the actual feedback given.
    \BODY
    \end{trivlist}
    }
    {  % \begin{feedback}[solution]   \BODY     \end{feedback}  }
    }
    }    
\else
% ONLY for HTML; xmoplossing is styled with css, and is not, and need not be a LaTeX environment
% THUS: it does NOT use feedback anymore ...
%    \NewEnviron{oplossing}{\begin{expandable}{xmoplossing}{\nlen{Toon uitwerking}{Show solution}}{\BODY}\end{expandable}}
    \newenvironment{oplossing}[1][onzichtbaar]
   {%
       \begin{expandable}{xmoplossing}{}
   }
   {%
   	   \end{expandable}
   } 
%     \newenvironment{oplossing}[1][onzichtbaar]
%    {%
%        \begin{feedback}[solution]   	
%    }
%    {%
%    	   \end{feedback}
%    } 
\fi

\ifhandout%
    \NewEnviron{uitkomst}[1][onzichtbaar]%
    {%
    \ifthenelse{\equal{\detokenize{#1}}{\detokenize{toon}}}
    {
    \def\PH@Command{#1}% Use PH@Command to hold the content and be a target for "\expandafter" to expand once.

    \begin{trivlist}% Begin the trivlist to use formating of the "Feedback" label.
    \item[\hskip \labelsep\small\slshape\bfseries Uitkomst:% Format the "Feedback" label. Don't forget the space.
    %(\texttt{\detokenize\expandafter{\PH@Command}}):% Format (and detokenize) the condition for feedback to trigger
    \hspace{2ex}]\small%\slshape% Insert some space before the actual feedback given.
    \BODY
    \end{trivlist}
    }
    {  % \begin{feedback}[solution]   \BODY     \end{feedback}  }
    }
    }    
\else
\ifdefined\HCode
   \newenvironment{uitkomst}[1][onzichtbaar]
    {%
        \begin{expandable}{xmuitkomst}{}%
    }
    {%
    	\end{expandable}%
    } 
\else
  % Do NOT print 'uitkomst' in non-handout
  %  (presumably, there is also an 'oplossing' ??)
  \newenvironment{uitkomst}[1][onzichtbaar]{}{}
\fi
\fi

%
% Uitweidingen zijn extra's die niet redelijkerwijze tot de leerstof behoren
% Uitbreidingen zijn extra's die wel redelijkerwijze tot de leerstof van bv meer geavanceerde versies kunnen behoren (B-programma/Wiskundestudenten/...?)
% Nog niet voorzien: design voor verschillende versies (A/B programma, BIO, voorkennis/ ...)
% Voor 'uitweidingen' is er een environment die online per default is ingeklapt, en in pdf al dan niet kan worden geincluded  (via \xmnouitweiding) 
%
% in een xourse, per default GEEN uitweidingen, tenzij \xmuitweiding expliciet ergens is gezet ...
\ifdefined\isXourse
   \ifdefined\xmuitweiding
   \else
       \def\xmnouitweiding{true}
   \fi
\fi

\ifdefined\xmnouitweiding
\newcounter{xmuitweiding}  % anders error undefined ...  
\excludecomment{xmuitweiding}
\else
\newtheoremstyle{dotless}{}{}{}{}{}{}{ }{}
\theoremstyle{dotless}
\newtheorem*{xmuitweidingnofrills}{}   % nofrills = no accordion; gebruikt dus de dotless theoremstyle!

\newcounter{xmuitweiding}
\newenvironment{xmuitweiding}[1][ ]%
{% 
	\refstepcounter{xmuitweiding}%
    \begin{expandable}{xmuitweiding}{\nlentext{Uitweiding \arabic{xmuitweiding}: #1}{Digression \arabic{xmuitweiding}: #1}}%
	\begin{xmuitweidingnofrills}%
}
{%
    \end{xmuitweidingnofrills}%
    \end{expandable}%
}   
% \newenvironment{xmuitweiding}[1][ ]%
% {% 
% 	\refstepcounter{xmuitweiding}
% 	\begin{accordion}\begin{accordion-item}[Uitweiding \arabic{xmuitweiding}: #1]%
% 			\begin{xmuitweidingnofrills}%
% 			}
% 			{\end{xmuitweidingnofrills}\end{accordion-item}\end{accordion}}   
\fi


\newenvironment{xmexpandable}[1][]{
	\begin{accordion}\begin{accordion-item}[#1]%
		}{\end{accordion-item}\end{accordion}}


% Command that gives a selection box online, but just prints the right answer in pdf
\newcommand{\xmonlineChoice}[1]{\pdfOnly{\wordchoicegiventrue}\wordChoice{#1}\pdfOnly{\wordchoicegivenfalse}}   % deprecated, gebruik onlineChoice ...
\newcommand{\onlineChoice}[1]{\pdfOnly{\wordchoicegiventrue}\wordChoice{#1}\pdfOnly{\wordchoicegivenfalse}}

\newcommand{\TJa}{\nlentext{ Ja }{ Yes }}
\newcommand{\TNee}{\nlentext{ Nee }{ No }}
\newcommand{\TJuist}{\nlentext{ Juist }{ True }}
\newcommand{\TFout}{\nlentext{ Fout }{ False }}

\newcommand{\choiceTrue }{{\renewcommand{\choiceminimumhorizontalsize}{4em}\wordChoice{\choice[correct]{\TJuist}\choice{\TFout}}}}
\newcommand{\choiceFalse}{{\renewcommand{\choiceminimumhorizontalsize}{4em}\wordChoice{\choice{\TJuist}\choice[correct]{\TFout}}}}

\newcommand{\choiceYes}{{\renewcommand{\choiceminimumhorizontalsize}{3em}\wordChoice{\choice[correct]{\TJa}\choice{\TNee}}}}
\newcommand{\choiceNo }{{\renewcommand{\choiceminimumhorizontalsize}{3em}\wordChoice{\choice{\TJa}\choice[correct]{\TNee}}}}

% Optional nicer formatting for wordChoice in PDF

\let\inlinechoiceorig\inlinechoice

%\makeatletter
%\providecommand{\choiceminimumverticalsize}{\vphantom{$\frac{\sqrt{2}}{2}$}}   % minimum height of boxes (cfr infra)
\providecommand{\choiceminimumverticalsize}{\vphantom{$\tfrac{2}{2}$}}   % minimum height of boxes (cfr infra)
\providecommand{\choiceminimumhorizontalsize}{1em}   % minimum width of boxes (cfr infra)

\newcommand{\inlinechoicesquares}[2][]{%
		\setkeys{choice}{#1}%
		\ifthenelse{\boolean{\choice@correct}}%
		{%
            \ifhandout%
               \fbox{\choiceminimumverticalsize #2}\allowbreak\ignorespaces%
            \else%
               \fcolorbox{blue}{blue!20}{\choiceminimumverticalsize #2}\allowbreak\ignorespaces\setkeys{choice}{correct=false}\ignorespaces%
            \fi%
		}%
		{% else
			\fbox{\choiceminimumverticalsize #2}\allowbreak\ignorespaces%  HACK: wat kleiner, zodat fits on line ... 	
		}%
}

\newcommand{\inlinechoicesquareX}[2][]{%
		\setkeys{choice}{#1}%
		\ifthenelse{\boolean{\choice@correct}}%
		{%
            \ifhandout%
               \framebox[\ifdim\choiceminimumhorizontalsize<\width\width\else\choiceminimumhorizontalsize\fi]{\choiceminimumverticalsize\ #2\ }\allowbreak\ignorespaces\setkeys{choice}{correct=false}\ignorespaces%
            \else%
               \fcolorbox{blue}{blue!20}{\makebox[\ifdim\choiceminimumhorizontalsize<\width\width\else\choiceminimumhorizontalsize\fi]{\choiceminimumverticalsize #2}}\allowbreak\ignorespaces\setkeys{choice}{correct=false}\ignorespaces%
            \fi%
		}%
		{% else
        \ifhandout%
			\framebox[\ifdim\choiceminimumhorizontalsize<\width\width\else\choiceminimumhorizontalsize\fi]{\choiceminimumverticalsize\ #2\ }\allowbreak\ignorespaces%  HACK: wat kleiner, zodat fits on line ... 	
        \fi
		}%
}


\newcommand{\inlinechoicepuntjes}[2][]{%
		\setkeys{choice}{#1}%
		\ifthenelse{\boolean{\choice@correct}}%
		{%
            \ifhandout%
               \dots\ldots\ignorespaces\setkeys{choice}{correct=false}\ignorespaces
            \else%
               \fcolorbox{blue}{blue!20}{\choiceminimumverticalsize #2}\allowbreak\ignorespaces\setkeys{choice}{correct=false}\ignorespaces%
            \fi%
		}%
		{% else
			%\fbox{\choiceminimumverticalsize #2}\allowbreak\ignorespaces%  HACK: wat kleiner, zodat fits on line ... 	
		}%
}

% print niets, maar definieer globale variable \myanswer
%  (gebruikt om oplossingsbladen te printen) 
\newcommand{\inlinechoicedefanswer}[2][]{%
		\setkeys{choice}{#1}%
		\ifthenelse{\boolean{\choice@correct}}%
		{%
               \gdef\myanswer{#2}\setkeys{choice}{correct=false}

		}%
		{% else
			%\fbox{\choiceminimumverticalsize #2}\allowbreak\ignorespaces%  HACK: wat kleiner, zodat fits on line ... 	
		}%
}



%\makeatother

\newcommand{\setchoicedefanswer}{
\ifdefined\HCode
\else
%    \renewenvironment{multipleChoice@}[1][]{}{} % remove trailing ')'
    \let\inlinechoice\inlinechoicedefanswer
\fi
}

\newcommand{\setchoicepuntjes}{
\ifdefined\HCode
\else
    \renewenvironment{multipleChoice@}[1][]{}{} % remove trailing ')'
    \let\inlinechoice\inlinechoicepuntjes
\fi
}
\newcommand{\setchoicesquares}{
\ifdefined\HCode
\else
    \renewenvironment{multipleChoice@}[1][]{}{} % remove trailing ')'
    \let\inlinechoice\inlinechoicesquares
\fi
}
%
\newcommand{\setchoicesquareX}{
\ifdefined\HCode
\else
    \renewenvironment{multipleChoice@}[1][]{}{} % remove trailing ')'
    \let\inlinechoice\inlinechoicesquareX
\fi
}
%
\newcommand{\setchoicelist}{
\ifdefined\HCode
\else
    \renewenvironment{multipleChoice@}[1][]{}{)}% re-add trailing ')'
    \let\inlinechoice\inlinechoiceorig
\fi
}

\setchoicesquareX  % by default list-of-squares with onlineChoice in PDF

% Omdat multicols niet werkt in html: enkel in pdf  (in html zijn langere pagina's misschien ook minder storend)
\newenvironment{xmmulticols}[1][2]{
 \pdfOnly{\begin{multicols}{#1}}%
}{ \pdfOnly{\end{multicols}}}

%
% Te gebruiken in plaats van \section\subsection
%  (in een printstyle kan dan het level worden aangepast
%    naargelang \chapter vs \section style )
% 3/2021: DO NOT USE \xmsubsection !
\newcommand\xmsection\subsection
\newcommand\xmsubsection\subsubsection

% Aanpassen printversie
%  (hier gedefinieerd, zodat ze in xourse kunnen worden gezet/overschreven)
\providebool{parttoc}
\providebool{printpartfrontpage}
\providebool{printactivitytitle}
\providebool{printactivityqrcode}
\providebool{printactivityurl}
\providebool{printcontinuouspagenumbers}
\providebool{numberactivitiesbysubpart}
\providebool{addtitlenumber}
\providebool{addsectiontitlenumber}
\addtitlenumbertrue
\addsectiontitlenumbertrue

% The following three commands are hardcoded in xake, you can't create other commands like these, without adding them to xake as well
%  ( gebruikt in xourses om juiste soort titelpagina te krijgen voor verschillende ximera's )
\newcommand{\activitychapter}[2][]{
    {    
    \ifstrequal{#1}{notnumbered}{
        \addtitlenumberfalse
    }{}
    \typeout{ACTIVITYCHAPTER #2}   % logging
	\chapterstyle
	\activity{#2}
    }
}
\newcommand{\activitysection}[2][]{
    {
    \ifstrequal{#1}{notnumbered}{
        \addsectiontitlenumberfalse
    }{}
	\typeout{ACTIVITYSECTION #2}   % logging
	\sectionstyle
	\activity{#2}
    }
}
% Practices worden als activity getoond om de grote blokken te krijgen online
\newcommand{\practicesection}[2][]{
    {
    \ifstrequal{#1}{notnumbered}{
        \addsectiontitlenumberfalse
    }{}
    \typeout{PRACTICESECTION #2}   % logging
	\sectionstyle
	\activity{#2}
    }
}
\newcommand{\activitychapterlink}[3][]{
    {
    \ifstrequal{#1}{notnumbered}{
        \addtitlenumberfalse
    }{}
    \typeout{ACTIVITYCHAPTERLINK #3}   % logging
	\chapterstyle
	\activitylink[#1]{#2}{#3}
    }
}

\newcommand{\activitysectionlink}[3][]{
    {
    \ifstrequal{#1}{notnumbered}{
        \addsectiontitlenumberfalse
    }{}
    \typeout{ACTIVITYSECTIONLINK #3}   % logging
	\sectionstyle
	\activitylink[#1]{#2}{#3}
    }
}


% Commando om de printstyle toe te voegen in ximera's. Zorgt ervoor dat er geen problemen zijn als je de xourses compileert
% hack om onhandige relative paden in TeX te omzeilen
% should work both in xourse and ximera (pre-112022 only in ximera; thus obsoletes adhoc setup in xourses)
% loads global.sty if present (cfr global.css for online settings!)
% use global.sty to overwrite settings in printstyle.sty ...
\newcommand{\addPrintStyle}[1]{
\iftikzexport\else   % only in PDF
  \makeatletter
  \ifx\@onlypreamble\@notprerr\else   % ONLY if in tex-preamble   (and e.g. not when included from xourse)
    \typeout{Loading printstyle}   % logging
    \usepackage{#1/printstyle} % mag enkel geinclude worden als je die apart compileert
    \IfFileExists{#1/global.sty}{
        \typeout{Loading printstyle-folder #1/global.sty}   % logging
        \usepackage{#1/global}
        }{
        \typeout{Info: No extra #1/global.sty}   % logging
    }   % load global.sty if present
    \IfFileExists{global.sty}{
        \typeout{Loading local-folder global.sty (or TEXINPUTPATH..)}   % logging
        \usepackage{global}
    }{
        \typeout{Info: No folder/global.sty}   % logging
    }   % load global.sty if present
    \IfFileExists{\currfilebase.sty}
    {
        \typeout{Loading \currfilebase.sty}
        \input{\currfilebase.sty}
    }{
        \typeout{Info: No local \currfilebase.sty}
    }
    \fi
  \makeatother
\fi
}

%
%  
% references: Ximera heeft adhoc logica	 om online labels te doen werken over verschillende files heen
% met \hyperref kan de getoonde tekst toch worden opgegeven, in plaats van af te hangen van de label-text
\ifdefined\HCode
% Link to standard \labels, but give your own description
% Usage:  Volg \hyperref[my_very_verbose_label]{deze link} voor wat tijdverlies
%   (01/2020: Ximera-server aangepast om bij class reference-keeptext de link-text NIET te vervangen door de label-text !!!) 
\renewcommand{\hyperref}[2][]{\HCode{<a class="reference reference-keeptext" href="\##1">}#2\HCode{</a>}}
%
%  Link to specific targets  (not tested ?)
\renewcommand{\hypertarget}[1]{\HCode{<a class="ximera-label" id="#1"></a>}}
\renewcommand{\hyperlink}[2]{\HCode{<a class="reference reference-keeptext" href="\##1">}#2\HCode{</a>}}
\fi

% Mmm, quid English ... (for keyword #1 !) ?
\newcommand{\wikilink}[2]{
    \hyperlink{https://nl.wikipedia.org/wiki/#1}{#2}
    \pdfOnly{\footnote{See \url{https://nl.wikipedia.org/wiki/#1}}
    }
}

\renewcommand{\figurename}{Figuur}
\renewcommand{\tablename}{Tabel}

%
% Gedoe om verschillende versies van xourse/ximera te maken afhankelijk van settings
%
% default: versie met antwoorden
% handout: versie voor de studenten, zonder antwoorden/oplossingen
% full: met alles erop en eraan, dus geschikt voor auteurs en/of lesgevers  (bevat in de pdf ook de 'online-only' stukken!)
%
%
% verder kunnen ook opties/variabele worden gezet voor hints/auteurs/uitweidingen/ etc
%
% 'Full' versie
\newtoggle{showonline}
\ifdefined\HCode   % zet default showOnline
    \toggletrue{showonline} 
\else
    \togglefalse{showonline}
\fi

% Full versie   % deprecated: see infra
\newcommand{\printFull}{
    \hintstrue
    \handoutfalse
    \toggletrue{showonline} 
}

\ifdefined\shouldPrintFull   % deprecated: see infra
    \printFull
\fi



% Overschrijf onlineOnly  (zoals gedefinieerd in ximera.cls)
\ifhandout   % in handout: gebruik de oorspronkelijke ximera.cls implementatie  (is dit wel nodig/nuttig?)
\else
    \iftoggle{showonline}{%
        \ifdefined\HCode
          \RenewEnviron{onlineOnly}{\bgroup\BODY\egroup}   % showOnline, en we zijn  online, dus toon de tekst
        \else
          \RenewEnviron{onlineOnly}{\bgroup\color{red!50!black}\BODY\egroup}   % showOnline, maar we zijn toch niet online: kleur de tekst rood 
        \fi
    }{%
      \RenewEnviron{onlineOnly}{}  % geen showOnline
    }
\fi

% hack om na hoofding van definition/proposition/... als dan niet op een nieuwe lijn te starten
% soms is dat goed en mooi, en soms niet; en in HTML is het nu (2/2020) anders dan in pdf
% vandaar suggestie om 
%     \begin{definition}[Nieuw concept] \nl
% te gebruiken als je zeker een newline wil na de hoofdig en titel
% (in het bijzonder itemize zonder \nl is 'lelijk' ...)
\ifdefined\HCode
\newcommand{\nl}{}
\else
\newcommand{\nl}{\ \par} % newline (achter heading van definition etc.)
\fi


% \nl enkel in handoutmode (ihb voor \wordChoice, die dan typisch veeeel langer wordt)
\ifdefined\HCode
\providecommand{\handoutnl}{}
\else
\providecommand{\handoutnl}{%
\ifhandout%
  \nl%
\fi%
}
\fi

% Could potentially replace \pdfOnline/\begin{onlineOnly} : 
% Usage= \ifonline{Hallo surfer}{Hallo PDFlezer}
\providecommand{\ifonline}[2]%
{
\begin{onlineOnly}#1\end{onlineOnly}%
\pdfOnly{#2}
}%


%
% Maak optionele 'basic' en 'extended' versies van een activity
%  met environment basicOnly, basicSkip en extendedOnly
%
%  (
%   Dit werkt ENKEL in de PDF; de online versies tonen (minstens voorklopig) steeds 
%   het default geval met printbasicversion en printextendversion beide FALSE
%  )
%
\providebool{printbasicversion}
\providebool{printextendedversion}   % not properly implemented
\providebool{printfullversion}       % presumably print everything (debug/auteur)
%
% only set these in xourses, and BEFORE loading this preamble
%
%\newif\ifshowbasic     \showbasictrue        % use this line in xourse to show 'basic' sections
%\newif\ifshowextended  \showextendedtrue     % use this line in xourse to show 'extended' sections
%
%
%\ifbool{showbasic}
%      { \NewEnviron{basicOnly}{\BODY} }    % if yes: just print contents
%      { \NewEnviron{basicOnly}{}      }    % if no:  completely ignore contents
%
%\ifbool{showbasic}
%      { \NewEnviron{basicSkip}{}      }
%      { \NewEnviron{basicSkip}{\BODY} }
%

\ifbool{printextendedversion}
      { \NewEnviron{extendedOnly}{\BODY} }
      { \NewEnviron{extendedOnly}{}      }
      


\ifdefined\HCode    % in html: always print
      {\newenvironment*{basicOnly}{}{}}    % if yes: just print contents
      {\newenvironment*{basicSkip}{}{}}    % if yes: just print contents
\else

\ifbool{printbasicversion}
      {\newenvironment*{basicOnly}{}{}}    % if yes: just print contents
      {\NewEnviron{basicOnly}{}      }    % if no:  completely ignore contents

\ifbool{printbasicversion}
      {\NewEnviron{basicSkip}{}      }
      {\newenvironment*{basicSkip}{}{}}

\fi

\usepackage{float}
\usepackage[rightbars,color]{changebar}

% Full versie
\ifbool{printfullversion}{
    \hintstrue
    \handoutfalse
    \toggletrue{showonline}
    \printbasicversionfalse
    \cbcolor{red}
    \renewenvironment*{basicOnly}{\cbstart}{\cbend}
    \renewenvironment*{basicSkip}{\cbstart}{\cbend}
    \def\xmtoonprintopties{FULL}   % will be printed in footer
}
{}
      
%
% Evalueer \ifhints IN de environment
%  
%
%\RenewEnviron{hint}
%{
%\ifhandout
%\ifhints\else\setbox0\vbox\fi%everything in een emty box
%\bgroup 
%\stepcounter{hintLevel}
%\BODY
%\egroup\ignorespacesafterend
%\addtocounter{hintLevel}{-1}
%\else
%\ifhints
%\begin{trivlist}\item[\hskip \labelsep\small\slshape\bfseries Hint:\hspace{2ex}]
%\small\slshape
%\stepcounter{hintLevel}
%\BODY
%\end{trivlist}
%\addtocounter{hintLevel}{-1}
%\fi
%\fi
%}

% Onafhankelijk van \ifhandout ...? TO BE VERIFIED
\RenewEnviron{hint}
{
\ifhints
\begin{trivlist}\item[\hskip \labelsep\small\bfseries Hint:\hspace{2ex}]
\small%\slshape
\stepcounter{hintLevel}
\BODY
\end{trivlist}
\addtocounter{hintLevel}{-1}
\else
\iftikzexport   % anders worden de tikz tekeningen in hints niet gegenereerd ?
\setbox0\vbox\bgroup
\stepcounter{hintLevel}
\BODY
\egroup\ignorespacesafterend
\addtocounter{hintLevel}{-1}
\fi % ifhandout
\fi %ifhints
}

%
% \tab sets typewriter-tabs (e.g. to format questions)
% (Has no effect in HTML :-( ))
%
\usepackage{tabto}
\ifdefined\HCode
  \renewcommand{\tab}{\quad}    % otherwise dummy .png's are generated ...?
\fi


% Also redefined in  preamble to get correct styling 
% for tikz images for (\tikzexport)
%

\theoremstyle{definition} % Bold titels
\makeatletter
\let\proposition\relax
\let\c@proposition\relax
\let\endproposition\relax
\makeatother
\newtheorem{proposition}{Eigenschap}


%\instructornotesfalse

% logic with \ifhandoutin ximera.cls unclear;so overwrite ...
\makeatletter
\@ifundefined{ifinstructornotes}{%
  \newif\ifinstructornotes
  \instructornotesfalse
  \newenvironment{instructorNotes}{}{}
}{}
\makeatother
\ifinstructornotes
\else
\renewenvironment{instructorNotes}%
{%
    \setbox0\vbox\bgroup
}
{%
    \egroup
}
\fi

% \RedeclareMathOperator
% from https://tex.stackexchange.com/questions/175251/how-to-redefine-a-command-using-declaremathoperator
\makeatletter
\newcommand\RedeclareMathOperator{%
    \@ifstar{\def\rmo@s{m}\rmo@redeclare}{\def\rmo@s{o}\rmo@redeclare}%
}
% this is taken from \renew@command
\newcommand\rmo@redeclare[2]{%
    \begingroup \escapechar\m@ne\xdef\@gtempa{{\string#1}}\endgroup
    \expandafter\@ifundefined\@gtempa
    {\@latex@error{\noexpand#1undefined}\@ehc}%
    \relax
    \expandafter\rmo@declmathop\rmo@s{#1}{#2}}
% This is just \@declmathop without \@ifdefinable
\newcommand\rmo@declmathop[3]{%
    \DeclareRobustCommand{#2}{\qopname\newmcodes@#1{#3}}%
}
\@onlypreamble\RedeclareMathOperator
\makeatother


%
% Engelse vertaling, vooral in mathmode
%
% 1. Algemene procedure
%
\ifdefined\isEn
 \newcommand{\nlen}[2]{#2}
 \newcommand{\nlentext}[2]{\text{#2}}
 \newcommand{\nlentextbf}[2]{\textbf{#2}}
\else
 \newcommand{\nlen}[2]{#1}
 \newcommand{\nlentext}[2]{\text{#1}}
 \newcommand{\nlentextbf}[2]{\textbf{#1}}
\fi

%
% 2. Lijst van erg veel gebruikte uitdrukkingen
%

% Ja/Nee/Fout/Juits etc
%\newcommand{\TJa}{\nlentext{ Ja }{ and }}
%\newcommand{\TNee}{\nlentext{ Nee }{ No }}
%\newcommand{\TJuist}{\nlentext{ Juist }{ Correct }
%\newcommand{\TFout}{\nlentext{ Fout }{ Wrong }
\newcommand{\TWaar}{\nlentext{ Waar }{ True }}
\newcommand{\TOnwaar}{\nlentext{ Vals }{ False }}
% Korte bindwoorden en, of, dus, ...
\newcommand{\Ten}{\nlentext{ en }{ and }}
\newcommand{\Tof}{\nlentext{ of }{ or }}
\newcommand{\Tdus}{\nlentext{ dus }{ so }}
\newcommand{\Tendus}{\nlentext{ en dus }{ and thus }}
\newcommand{\Tvooralle}{\nlentext{ voor alle }{ for all }}
\newcommand{\Took}{\nlentext{ ook }{ also }}
\newcommand{\Tals}{\nlentext{ als }{ when }} %of if?
\newcommand{\Twant}{\nlentext{ want }{ as }}
\newcommand{\Tmaal}{\nlentext{ maal }{ times }}
\newcommand{\Toptellen}{\nlentext{ optellen }{ add }}
\newcommand{\Tde}{\nlentext{ de }{ the }}
\newcommand{\Thet}{\nlentext{ het }{ the }}
\newcommand{\Tis}{\nlentext{ is }{ is }} %zodat is in text staat in mathmode (geen italics)
\newcommand{\Tmet}{\nlentext{ met }{ where }} % in situaties e.g met p < n --> where p < n
\newcommand{\Tnooit}{\nlentext{ nooit }{ never }}
\newcommand{\Tmaar}{\nlentext{ maar }{ but }}
\newcommand{\Tniet}{\nlentext{ niet }{ not }}
\newcommand{\Tuit}{\nlentext{ uit }{ from }}
\newcommand{\Ttov}{\nlentext{ t.o.v. }{ w.r.t. }}
\newcommand{\Tzodat}{\nlentext{ zodat }{ such that }}
\newcommand{\Tdeth}{\nlentext{de }{th }}
\newcommand{\Tomdat}{\nlentext{omdat }{because }} 


%
% Overschrijf addhoc commando's
%
\ifdefined\isEn
\renewcommand{\pernot}{\overset{\mathrm{notation}}{=}}
\RedeclareMathOperator{\bld}{im}     % beeld
\RedeclareMathOperator{\graf}{graph}   % grafiek
\RedeclareMathOperator{\rico}{slope}   % richtingcoëfficient
\RedeclareMathOperator{\co}{co}       % coordinaat
\RedeclareMathOperator{\gr}{deg}       % graad

% Operators
\RedeclareMathOperator{\bgsin}{arcsin}
\RedeclareMathOperator{\bgcos}{arccos}
\RedeclareMathOperator{\bgtan}{arctan}
\RedeclareMathOperator{\bgcot}{arccot}
\RedeclareMathOperator{\bgsinh}{arcsinh}
\RedeclareMathOperator{\bgcosh}{arccosh}
\RedeclareMathOperator{\bgtanh}{arctanh}
\RedeclareMathOperator{\bgcoth}{arccoth}

\fi


% HACK: use 'oplossing' for 'explanation' ...
\let\explanation\relax
\let\endexplanation\relax
% \newenvironment{explanation}{\begin{oplossing}}{\end{oplossing}}
\newcounter{explanation}

\ifhandout%
    \NewEnviron{explanation}[1][toon]%
    {%
    \RenewEnviron{verbatim}{ \red{VERBATIM CONTENT MISSING IN THIS PDF}} %% \expandafter\verb|\BODY|}

    \ifthenelse{\equal{\detokenize{#1}}{\detokenize{toon}}}
    {
    \def\PH@Command{#1}% Use PH@Command to hold the content and be a target for "\expandafter" to expand once.

    \begin{trivlist}% Begin the trivlist to use formating of the "Feedback" label.
    \item[\hskip \labelsep\small\slshape\bfseries Explanation:% Format the "Feedback" label. Don't forget the space.
    %(\texttt{\detokenize\expandafter{\PH@Command}}):% Format (and detokenize) the condition for feedback to trigger
    \hspace{2ex}]\small%\slshape% Insert some space before the actual feedback given.
    \BODY
    \end{trivlist}
    }
    {  % \begin{feedback}[solution]   \BODY     \end{feedback}  }
    }
    }    
\else
% ONLY for HTML; xmoplossing is styled with css, and is not, and need not be a LaTeX environment
% THUS: it does NOT use feedback anymore ...
%    \NewEnviron{oplossing}{\begin{expandable}{xmoplossing}{\nlen{Toon uitwerking}{Show solution}}{\BODY}\end{expandable}}
    \newenvironment{explanation}[1][toon]
   {%
       \begin{expandable}{xmoplossing}{}
   }
   {%
   	   \end{expandable}
   } 
\fi

\title{Introduction to Linear Transformations} \license{CC BY-NC-SA 4.0}

\begin{document}

\begin{abstract}
 \end{abstract}
\maketitle


\section*{Introduction to Linear Transformations}

We start by reviewing the definition of a function.

\begin{definition} \label{def:function}
  
Let $V$ and $W$ be sets.  A \dfn{function} $f$ from $V$ into $W$, denoted by 
$$f:V\rightarrow W$$
assigns to each element $x$ of $V$, an element $y=f(x)$ of $W$. 
\vskip 10pt
The set $V$ is called the \dfn{domain} of $f$, and the set $W$ is called the \dfn{codomain}.
\vskip 10pt
If $y=f(x)$, we say that $x$ \dfn{maps to} $y$, and $y$ is the \dfn{image} of $x$.
\vskip 10pt
The collection of images of all points of $V$ is called the \dfn{image} of $V$ under $f$, or the \dfn{image} of $f$.  (It is also known as the \dfn{ range} of $f$.)  
\end{definition}


In algebra and calculus you worked with functions $f:\RR\rightarrow\RR$ whose domain and codomain were each the set of all real numbers.  In linear algebra, we call our functions \dfn{transformations}.  The domain and codomain of a transformation are \dfn{vector spaces}.  %A typical transformation $T$ will map vectors of $\RR^n$ into $\RR^m$.

\begin{exploration}\label{init:lintransintro} In this exercise we will introduce a very special type of transformation by contrasting the effects of two transformations on vectors of $\RR^2$.  We will see that some transformations have ``nice" properties, while others do not.  Define $T_1$ and $T_2$ as follows:
$$T_1:\RR^2\rightarrow\RR^2$$
$$T_1\left(\begin{bmatrix}
x\\
y
\end{bmatrix}\right)=\begin{bmatrix}
x-y\\
x
\end{bmatrix}$$
$$T_2:\RR^2\rightarrow\RR^2$$
$$T_2\left(\begin{bmatrix}
x\\
y
\end{bmatrix}\right)=\begin{bmatrix}
-x+y+1\\
y-2
\end{bmatrix}$$

Each of these transformations takes a vector in $\RR^2$, and maps it to another vector in $\RR^2$.  To see if you understand how these transformations are defined, see if you can determine what these transformations do to the vector $\begin{bmatrix}
4\\
3
\end{bmatrix}$. 

$$
T_1\left(\begin{bmatrix}
4\\
3
\end{bmatrix}\right)=\begin{bmatrix}
\answer{1}\\
\answer{4}
\end{bmatrix} \quad \text{and} \quad
T_2\left(\begin{bmatrix}
4\\
3
\end{bmatrix}\right)=\begin{bmatrix}
\answer{0}\\
\answer{1}\end{bmatrix}.$$

%If you did not get correct answers, watch Video \ref{computation}



Now, let's take the vector $\begin{bmatrix}
4\\
3
\end{bmatrix}$ and multiply it by a scalar, say $7$.
$$7\begin{bmatrix}
4\\
3
\end{bmatrix} = \begin{bmatrix}
28\\
21
\end{bmatrix}$$.  

Now let's compare how $T_1$ and $T_2$ ``handle" this product.  Starting with $T_1$, we compute:

$$
T_1\left(7\begin{bmatrix}
4\\
3
\end{bmatrix}\right)=T_1\left(\begin{bmatrix}
28\\
21
\end{bmatrix}\right)=\begin{bmatrix}
7\\
28
\end{bmatrix} 
% \quad \text{and} \quad
% T_2\left(7\begin{bmatrix}
% 4\\
% 3
% \end{bmatrix}\right)=T_2\left(\begin{bmatrix}
% 28\\
% 21
% \end{bmatrix}\right)=\begin{bmatrix}
% -6\\
% 19\end{bmatrix}
$$

Observe that multiplying the original vector by $7$, then applying $T_1$, has the same effect as applying $T_1$ to the original vector, then multiplying the image by $7$.  In other words,

$$
T_1\left(7\begin{bmatrix}
4\\
3
\end{bmatrix}\right)=\begin{bmatrix}
7\\
28
\end{bmatrix}=7\begin{bmatrix}
1\\
4
\end{bmatrix}=7T_1\left(\begin{bmatrix}
4\\
3
\end{bmatrix}\right)$$

Diagrammatically, this can be represented as follows.

\begin{center}
 \begin{tikzpicture} 
      
    \fill[blue, opacity=0.3] (1,1) rectangle (4,6);
   \fill[red, opacity=0.2] (7,1) rectangle (12,6);
   
   \node[] at (2.5, 6.3)  (p2)    {Domain of $T_1$};
   \node[] at (9.5, 6.3)  (r3)    {Codomain of $T_1$};
   
      \node[] at (2.5, 5)  (a)    {$\begin{bmatrix}4\\3\end{bmatrix}$};
     \node[] at (2.5, 2)  (b)    {$7\begin{bmatrix}4\\3\end{bmatrix}$};
     
    
    \node[] at (10, 5)  (c)    {$T\left(\begin{bmatrix}4\\3\end{bmatrix}\right)$};
     \node[] at (9.5, 2)  (d)    {$T\left(7\begin{bmatrix}4\\3\end{bmatrix}\right)=7T\left(\begin{bmatrix}4\\3\end{bmatrix}\right)$};
     \node[] at (10, 2.2)  (e) {};
     
    
     \draw [->,line width=1pt,-stealth, red]  (a.east)to(c.west);
     \draw [->,line width=1pt,-stealth, blue]  (b.east)to(d.west);
     
     \draw [->,line width=1pt,-stealth, blue]  (a.south)to(b.north);
     \draw [->,line width=1pt,-stealth,red]  (c.south)to(e.north);
     
%Function labels
      \node[red] at (5.5, 5.3)    {$T_1$};
      \node[blue] at (5.5, 2.3)    {$T_1$};
      
      \node[blue] at (1.7, 3.7)    {Multiply};
      \node[blue] at (1.7, 3.2)    {by $7$ first};
      \node[red] at (5.5, 4.7)    {Apply $T_1$ first};   
      
  \end{tikzpicture}
\end{center}





You should try to verify that this property does not hold for transformation $T_2$.  In other words,

$$
T_2\left(7\begin{bmatrix}
4\\
3
\end{bmatrix}\right)\neq 7T_2\left(\begin{bmatrix}
4\\
3
\end{bmatrix}\right)$$

%If you are not sure how to do this, watch Video \ref{computation2}.



There is nothing special about the number $7$, and it is not hard to prove that for any scalar $k$ and vector $\vec{u}$ of $\RR^2$, $T_1$ satisfies
\begin{align}\label{lin1} kT_1(\vec{u})= T_1(k\vec{u}).\end{align}

It turns out that $T_1$ satisfies another important property. For all vectors $\vec{u}$ and $\vec{v}$ of $\RR^2$ we have:
\begin{align}\label{lin2} T_1(\vec{u}+\vec{v}) = T_1(\vec{u})+T_1(\vec{v})\end{align}
We leave it to the reader to illustrate this property with a specific example (see Practice Problem \ref{prob:sum}).  We will show that $T_1$ satisfies (\ref{lin2}) in general.

Let $\vec{u}=\begin{bmatrix}
u_1\\
u_2
\end{bmatrix}$ and $\vec{v}=\begin{bmatrix}
v_1\\
v_2
\end{bmatrix}$, then
\begin{align*}
T_1(\vec{u}+\vec{v})&=T_1\left(\begin{bmatrix}
u_1\\
u_2
\end{bmatrix}+\begin{bmatrix}
v_1\\
v_2
\end{bmatrix}\right)=T_1\left(\begin{bmatrix}
u_1+v_1\\
u_2+v_2
\end{bmatrix}\right)=\begin{bmatrix}
u_1+v_1-u_2-v_2\\
u_1+v_1
\end{bmatrix}\\
&=\begin{bmatrix}
u_1-u_2\\
u_1
\end{bmatrix}+\begin{bmatrix}
v_1-v_2\\
v_1
\end{bmatrix}=T_1\left(\begin{bmatrix}
u_1\\
u_2
\end{bmatrix}\right)+T_1\left(\begin{bmatrix}
v_1\\
v_2
\end{bmatrix}\right)\\
&=T_1(\vec{u})+T_1(\vec{v})
\end{align*}
It turns out that $T_2$ fails to satisfy this property.  Can you prove that this is the case?  Remember that to prove that a property DOES NOT hold, it suffices to find a counter-example.  See if you can find vectors $\vec{u}$ and $\vec{v}$ such that 
\begin{align}\label{t2}T_2(\vec{u}+\vec{v}) \neq T_2(\vec{u})+T_2(\vec{v}).\end{align}
(See Practice Problem \ref{prob:prob2}.) 
\end{exploration}

Transformations satisfying (\ref{lin1}) and (\ref{lin2}), like $T_1$, are called \dfn{linear transformations}. Transformations like $T_2$ are not linear.
 You have already encountered several linear transformations in the form of matrix transformations in sections \href{\xmbaseurl/LTR-0005/main}{Matrix Transformations} and \href{\xmbaseurl/LTR-0070/main}{Geometric Transformations of the Plane}.  

  \begin{definition} \label{def:lin}
 A transformation $T:\RR^n\rightarrow \RR^m$ is called a \dfn{ linear transformation} if the following are true for all vectors $\vec{u}$ and $\vec{v}$ in $\RR^n$, and scalars $k$.
\begin{equation}\label{eq:lintrans1}
T(k\vec{u})= kT(\vec{u})
\end{equation}
\begin{equation}\label{eq:lintrans2}
T(\vec{u}+\vec{v})= T(\vec{u})+T(\vec{v})
\end{equation}
\end{definition}

Equations (\ref{eq:lintrans1}) and (\ref{eq:lintrans2}) of the above definition can be illustrated diagrammatically as follows.

\begin{center}
 \begin{tikzpicture} 
      
    \fill[blue, opacity=0.3] (1,1) rectangle (4,5);
   \fill[red, opacity=0.2] (7,1) rectangle (10,5);
   
   \node[] at (2.5, 5.3)  (p2)    {Domain of $T$};
   \node[] at (8.5, 5.3)  (r3)    {Codomain of $T$};
   
      \node[] at (2.5, 4.5)  (a)    {$\vec{v}$};
     \node[] at (2.5, 1.5)  (b)    {$k\vec{v}$};
     
    
    \node[] at (9, 4.5)  (c)    {$T(\vec{v})$};
     \node[] at (8.5, 1.5)  (d)    {$T(k\vec{v})=kT(\vec{v})$};
     \node[] at (9, 1.6)  (e) {};
     
    
     \draw [->,line width=1pt,-stealth, red]  (a.east)to(c.west);
     \draw [->,line width=1pt,-stealth, blue]  (b.east)to(d.west);
     
     \draw [->,line width=1pt,-stealth, blue]  (a.south)to(b.north);
     \draw [->,line width=1pt,-stealth,red]  (c.south)to(e.north);
     
%Function labels
      \node[red] at (5.5, 4.8)    {$T$};
      \node[blue] at (5.5, 1.8)    {$T$};
      
      \node[blue] at (1.7, 3.2)    {Multiply};
      \node[blue] at (1.7, 2.8)    {first};
      \node[red] at (5.5, 4.2)    {Apply $T$ first};   
      
  \end{tikzpicture}
\end{center}

\begin{center}
 \begin{tikzpicture} 
      
    \fill[blue, opacity=0.3] (1,1) rectangle (4,6);
   \fill[red, opacity=0.2] (7,1) rectangle (12,6);
   
   \node[] at (2.5, 6.3)  (p2)    {Domain of $T$};
   \node[] at (9.5, 6.3)  (r3)    {Codomain of $T$};
   
      \node[] at (2.5, 5)  (a)    {$\vec{u}, \vec{v}$};
     \node[] at (2.5, 2)  (b)    {$\vec{u}+\vec{v}$};
     
    
    \node[] at (10.5, 5)  (c)    {$T(\vec{u}),T(\vec{v})$};
     \node[] at (9.5, 2)  (d)    {$T(\vec{u}+\vec{v})=T(\vec{u})+T(\vec{v})$};
     \node[] at (10.5, 2.2)  (e) {};
     
    
     \draw [->,line width=1pt,-stealth, red]  (a.east)to(c.west);
     \draw [->,line width=1pt,-stealth, blue]  (b.east)to(d.west);
     
     \draw [->,line width=1pt,-stealth, blue]  (a.south)to(b.north);
     \draw [->,line width=1pt,-stealth,red]  (c.south)to(e.north);
     
%Function labels
      \node[red] at (5.5, 5.3)    {$T$};
      \node[blue] at (5.5, 2.3)    {$T$};
      
      \node[blue] at (1.7, 3.7)    {Add};
      \node[blue] at (1.7, 3.2)    {first};
      \node[red] at (5.5, 4.7)    {Apply $T$ first};   
      
  \end{tikzpicture}
\end{center}

\begin{remark}\label{rem:combinedLinearDef}
Properties (\ref{eq:lintrans1}) and (\ref{eq:lintrans2}) are often combined into a single property.
\begin{equation}\label{eq:lintranscombinedprop}
T(k_1\vec{u}+k_2\vec{v})= k_1T(\vec{u})+k_2T(\vec{v})
\end{equation}
\end{remark}

\begin{example}\label{ex:lintransfirst}
Suppose $T:\RR^2\rightarrow \RR^3$ is a linear transformation such that 
$$T\left(\begin{bmatrix}1\\2\end{bmatrix}\right)=\begin{bmatrix}-1\\0\\3\end{bmatrix}\quad\text{and}\quad T\left(\begin{bmatrix}0\\-1\end{bmatrix}\right)=\begin{bmatrix}2\\-1\\0\end{bmatrix}$$
Find each of the following:
\begin{enumerate}
    \item \label{item:lintransfirsta} $T\left(\begin{bmatrix}2\\5\end{bmatrix}\right)=T\left(2\begin{bmatrix}1\\2\end{bmatrix}-\begin{bmatrix}0\\-1\end{bmatrix}\right)$
    \item \label{item:lintransfirstb}
    $T\left(\begin{bmatrix}1\\1\end{bmatrix}\right)$
\end{enumerate}
\begin{explanation}
\ref{item:lintransfirsta} Because $T$ is a linear transformation, it satisfies (\ref{eq:lintranscombinedprop}).  We compute:
\begin{align*}
    T\left(\begin{bmatrix}2\\5\end{bmatrix}\right)&=T\left(2\begin{bmatrix}1\\2\end{bmatrix}-\begin{bmatrix}0\\-1\end{bmatrix}\right)\\
    &=2T\left(\begin{bmatrix}1\\2\end{bmatrix}\right)-T\left(\begin{bmatrix}0\\-1\end{bmatrix}\right)\\
    &=2\begin{bmatrix}-1\\0\\3\end{bmatrix}-\begin{bmatrix}2\\-1\\0\end{bmatrix}\\
    &=\begin{bmatrix}-2\\0\\6\end{bmatrix}-\begin{bmatrix}2\\-1\\0\end{bmatrix}=\begin{bmatrix}-4\\1\\6\end{bmatrix}
\end{align*}

\ref{item:lintransfirstb} Observe that $\begin{bmatrix}1\\1\end{bmatrix}=\begin{bmatrix}1\\2\end{bmatrix}+\begin{bmatrix}0\\-1\end{bmatrix}$.  By (\ref{eq:lintranscombinedprop}) we have:
\begin{align*}
    T\left(\begin{bmatrix}1\\1\end{bmatrix}\right)&=T\left(\begin{bmatrix}1\\2\end{bmatrix}+\begin{bmatrix}0\\-1\end{bmatrix}\right)\\
    &=T\left(\begin{bmatrix}1\\2\end{bmatrix}\right)+T\left(\begin{bmatrix}0\\-1\end{bmatrix}\right)\\
    &=\begin{bmatrix}-1\\0\\3\end{bmatrix}+\begin{bmatrix}2\\-1\\0\end{bmatrix}=\begin{bmatrix}1\\-1\\3\end{bmatrix}
\end{align*}
\end{explanation}
\end{example}

\begin{observation}\label{obs:basisIsSufficient}
In Example \ref{ex:lintransfirst} we were given the images of two vectors, $\begin{bmatrix}1\\2\end{bmatrix}$ and $\begin{bmatrix}0\\-1\end{bmatrix}$, under a linear transformation $T$.  Based on this information, we were able to determine the images of two additional vectors: $\begin{bmatrix}2\\5\end{bmatrix}$ and $\begin{bmatrix}1\\1\end{bmatrix}$. 
The reason we were able to determine $T\left(\begin{bmatrix}2\\5\end{bmatrix}\right)$ and $T\left(\begin{bmatrix}1\\1\end{bmatrix}\right)$ is because $\begin{bmatrix}2\\5\end{bmatrix}$ and $\begin{bmatrix}1\\1\end{bmatrix}$ can be written as unique linear combinations of $\begin{bmatrix}1\\2\end{bmatrix}$ and $\begin{bmatrix}0\\-1\end{bmatrix}$.

Can every vector of $\RR^2$ be written as a linear combination of $\begin{bmatrix}1\\2\end{bmatrix}$ and $\begin{bmatrix}0\\-1\end{bmatrix}$?
\begin{multipleChoice}
  \choice[correct]{Yes}
  \choice{No}
    \end{multipleChoice}
    
    Is the information provided in Example \ref{ex:lintransfirst} sufficient to determine the image of every vector in $\RR^2$ under $T$?
    \begin{multipleChoice}
  \choice[correct]{Yes}
  \choice{No}
    \end{multipleChoice}
\end{observation}

\begin{example}\label{ex:lineartrans1}
Suppose $T:\RR^2\rightarrow\RR^2$ is a transformation such that 
$$T\left(\begin{bmatrix}
2\\
1
\end{bmatrix}\right)=\begin{bmatrix}
3\\
2
\end{bmatrix},\,\,\,T\left(\begin{bmatrix}
-1\\
0
\end{bmatrix}\right)=\begin{bmatrix}
1\\
1
\end{bmatrix},\,\,\,T\left(\begin{bmatrix}
-1\\
1
\end{bmatrix}\right)=\begin{bmatrix}
2\\
-4
\end{bmatrix}$$
Determine whether $T$ is a linear transformation.


\begin{explanation} Observe that
$$\begin{bmatrix}
-1\\
1
\end{bmatrix}=\begin{bmatrix}
2\\
1
\end{bmatrix}+3\begin{bmatrix}
-1\\
0
\end{bmatrix}$$
If $T$ were a linear transformation, then we would have: $$T\left(\begin{bmatrix}
-1\\
1
\end{bmatrix}\right)=T\left(\begin{bmatrix}
2\\
1
\end{bmatrix}+3\begin{bmatrix}
-1\\
0
\end{bmatrix}\right)=\begin{bmatrix}
3\\
2
\end{bmatrix}+3\begin{bmatrix}
1\\
1
\end{bmatrix}=\begin{bmatrix}
6\\
5
\end{bmatrix}$$
But according to the given,
$$T\left(\begin{bmatrix}-1\\1\end{bmatrix}\right)=\begin{bmatrix}2\\-4\end{bmatrix}$$
Since $\begin{bmatrix}
6\\
5
\end{bmatrix}\neq \begin{bmatrix}
2\\
-4
\end{bmatrix}$
we conclude that transformation $T$ is not linear.
\end{explanation}
\end{example}

In Exploration \ref{init:lintransintro} we introduced a transformation $T_2$ which turned out to be non-linear.  It took some work to show that $T_2$ is not linear.  The following theorem would have made our work easier.

\begin{theorem}\label{th:zerotozero} Let $T:\RR^n\rightarrow \RR^m$ be a linear transformation.  Then 
\begin{enumerate}
\item\label{item:zerotozero}
$T(\vec{0})=\vec{0}$.  In other words, linear transformations map the zero vector to the zero vector.
\item\label{item:linetoline}
$T$ maps any line in $\RR^n$ to a line (or the zero vector) in $\RR^m$.
\end{enumerate}
\end{theorem}
\begin{proof}
To prove part \ref{item:zerotozero}, let $\vec{v}$ be any vector in $\RR^n$.  By linearity of $T$, we have:
$$T(\vec{0})=T(\vec{v}-\vec{v})=T(\vec{v})-T(\vec{v})=\vec{0}$$

Part \ref{item:linetoline} will become evident after we prove Theorem \ref{th:matlin} in \href{\xmbaseurl/LTR-0020/main}{Standard Matrix of a Linear Transformation} and combine it with Practice Problem \ref{prb:linesToLines}.
\end{proof}

\begin{example}\label{ex:zerotozero}
Use Theorem \ref{th:zerotozero} to show that transformation $T_2$ of Exploration \ref{init:lintransintro} is not linear.
\begin{explanation}
Recall that $T_2:\RR^2\rightarrow\RR^2$ was defined by
$$T_2\left(\begin{bmatrix}
x\\
y
\end{bmatrix}\right)=\begin{bmatrix}
-x+y+1\\
y-2
\end{bmatrix}$$
We evaluate $T_2$ at $\vec{0}$:
$$T_2(\vec{0})=\begin{bmatrix}
-0+0+1\\
0-2
\end{bmatrix}=\begin{bmatrix}1\\-2\end{bmatrix}\neq\vec{0}$$
Since $T_2(\vec{0})\neq\vec{0}$, $T_2$ is not linear.
\end{explanation}
\end{example}




\subsection*{Linear Transformations Induced by Matrices}

%\begin{exploration}\label{init:matrixtrans}   Let $$A=\begin{bmatrix}
%3&-2&1\\
%1&0&2
%\end{bmatrix}$$
%Let $\vec{v}$ be a vector in $\RR^3$.  Then $A\vec{v}$ is a vector in $\RR^2$.  We can use this observation to define a transformation $T:\RR^3\rightarrow\RR^2$ by $T(\vec{v})=A\vec{v}$.  For example,
%$$T\left(\begin{bmatrix}
%2\\
%1\\
%-3
%\end{bmatrix}\right)=\begin{bmatrix}
%3&-2&1\\
%1&0&2
%\end{bmatrix}\begin{bmatrix}
%2\\
%1\\
%-3
%\end{bmatrix}=\begin{bmatrix}
%1\\
%-4
%\end{bmatrix}$$

%\end{exploration}

Recall that a transformation $T:\RR^n\rightarrow\RR^m$ defined by $T(\vec{v})=A\vec{v}$, where $A$ is some $m\times n$ matrix, is called a matrix transformation (or transformation induced by $A$).  As we had discovered in \href{\xmbaseurl/LTR-0005/main}{Matrix Transformations}, all matrix transformations are linear.  We now formalize this result as a theorem.

 \begin{theorem}\label{th:matrixtran} Let $A$ be an $m\times n$ matrix.  Define $T:\RR^n\rightarrow\RR^m$ by $T(\vec{v})=A\vec{v}$.  Then $T$ is a linear transformation.
\end{theorem}

\begin{proof}  Let $\vec{u}$ and $\vec{v}$ be vectors in $\RR^n$, and let $k$ be a scalar.  By properties of matrix multiplication we have:
$$T(\vec{u}+\vec{v})=A(\vec{u}+\vec{v})=A\vec{u}+A\vec{v}=T(\vec{u})+T(\vec{v})$$
$$T(k\vec{u})=A(k\vec{u})=kA\vec{u}=kT(\vec{u})$$
Therefore $T$ is a linear transformation.
\end{proof}

\begin{example}\label{ex:lineartrans2}
Let $T:\RR^n\rightarrow\RR^m$ be a linear transformation induced by $$A=\begin{bmatrix}
2&0\\
1&4\\
0&1
\end{bmatrix}$$
\begin{enumerate}
\item \label{item:exlineartrans2a}
Find $n$ and $m$.
\item \label{item:exlineartrans2d}
Find the image of $T$.
\end{enumerate}

\begin{explanation}
\ref{item:exlineartrans2a} $A$ is a $3\times 2$ matrix, so for the expression $T(\vec{x})=A\vec{x}$ to make sense, $\vec{x}$ has to be a $2\times 1$ vector.  Thus, the domain of $T$ is $\RR^2$ ($n=2$).  The product $A\vec{x}$ is a $3\times 1$ vector.  The codomain of $T$ is $\RR^3$ ($m=3$).

\ref{item:exlineartrans2d}  By Definition \ref{def:function}, the image of $T$ consists of images of all individual vectors in $\RR^2$ under $T$.  Every vector $\vec{v}$  in $\RR^2$ can be written as $\vec{v}=a\vec{i}+b\vec{j}$ for some real numbers  $a$ and $b$.  Consider the image of $\vec{v}$
$$T(\vec{v})=T(a\vec{i}+b\vec{j})=aT(\vec{i})+bT(\vec{j})=a\begin{bmatrix}2\\1\\0\end{bmatrix}+b\begin{bmatrix}0\\4\\1\end{bmatrix}$$
This shows that the range, or the image, of $T$ consists of all linear combinations of the columns of $A$.  In other words, the image of $T$ is the span of vectors $\begin{bmatrix}2\\1\\0\end{bmatrix}$ and $\begin{bmatrix}0\\4\\1\end{bmatrix}$. The two vectors are not scalar multiples of each other, therefore they span a plane in $\RR^3$.
\end{explanation}  
\end{example}
 


\begin{example}\label{ex:lineartrans3}
Let $T:\RR^n\rightarrow\RR^m$ be a linear transformation induced by
$$A=\begin{bmatrix}
-2&1&3\\
4&-2&-6
\end{bmatrix}$$
\begin{enumerate}
\item\label{item:lintrans3a}
Find $n$ and $m$. 
\item\label{item:lintrans3b}
Find and draw the image of $T$.
\end{enumerate}
\begin{explanation}
\ref{item:lintrans3a} 
$$n=\answer{3}, \quad m=\answer{2}$$

\ref{item:lintrans3b} To find the image of $T$, we will take a slightly different approach from what we did in Example \ref{ex:lineartrans2}\ref{item:exlineartrans2d}.

Let $\vec{v}=\begin{bmatrix}a\\b\\c\end{bmatrix}$ be an arbitrary vector of $\RR^3$.
The image of $\vec{v}$ is given by
\begin{align*}T(\vec{v})=\begin{bmatrix}
-2&1&3\\
4&-2&-6
\end{bmatrix}\begin{bmatrix}a\\b\\c\end{bmatrix}&=a\begin{bmatrix}-2\\4\end{bmatrix}+b\begin{bmatrix}1\\-2\end{bmatrix}+c\begin{bmatrix}3\\-6\end{bmatrix}\\
&=(a(-2)+b+c(3))\begin{bmatrix}1\\-2\end{bmatrix}
\end{align*}

This shows that the image of every vector in $\RR^3$ is a scalar multiple of $\begin{bmatrix}1\\-2\end{bmatrix}$.  This means that the image of $T$ is a line in $\RR^2$.

\begin{center}
\begin{tikzpicture}[scale=.8]

  \draw[->] (0,0)--(3,0);
  \draw[->] (0,0)--(0,3);
  \draw[->] (0,0)--(-1.5,-1);

  
  \draw[<->] (4,2)--(8,2);
  \draw[<->] (5,-1)--(5,3);
  \draw[line width=2pt,blue](4.5,3)--(6.5,-1) node[above, right]{Image of $T$};
  
  \draw [->,line width=1pt,-stealth]  (2,-1) to[out=-60, in=240] (4.5, -1);
   \node[] at (3.25, -1)   (b) {$T$};
 \end{tikzpicture}
\end{center}
\end{explanation}
\end{example}

\subsection*{Linear Transformations of Subspaces of $\RR^n$}

Definition \ref{def:lin} defines a linear transformation as a map from $\RR^n$ into $\RR^m$.  We will now make this definition more general by allowing the domain and the codomain of the transformation to be subspaces of $\RR^n$ and $\RR^m$.  Eventually, a linear transformation will be defined as a mapping between vector spaces.

\begin{definition}\label{def:lin2}
Let $V$ and $W$ be subspaces of $\RR^n$ and $\RR^m$.  A transformation $T:V\rightarrow W$ is called a \dfn{ linear transformation} if the following are true for all vectors $\vec{u}$ and $\vec{v}$ in $V$, and scalars $k$.

$$T(k\vec{u})= kT(\vec{u})$$
$$T(\vec{u}+\vec{v})= T(\vec{u})+T(\vec{v})$$
\end{definition}

\begin{example}\label{ex:subspacetransex1}
Let $V$ be a subspace of $\RR^3$ consisting of all vectors in the $xy$-plane.  Let $W$ be a subspace of $\RR^3$ consisting of all vectors along the $z$-axis.  (Do a quick verification that $V$ and $W$ are subspaces of $\RR^3$.)  Define a transformation $T:V\rightarrow W$ by 
$$T\left(\begin{bmatrix}a\\b\\0\end{bmatrix}\right)=\begin{bmatrix}0\\0\\a+b\end{bmatrix}$$
Show that $T$ is a linear transformation, and describe its action geometrically.
\begin{explanation}
Consider two arbitrary elements $\begin{bmatrix}a_1\\b_1\\0\end{bmatrix}$ and $\begin{bmatrix}a_2\\b_2\\0\end{bmatrix}$ of $V$.
\begin{align*}
T\left(\begin{bmatrix}a_1\\b_1\\0\end{bmatrix}+ \begin{bmatrix}a_2\\b_2\\0\end{bmatrix}\right)&=T\left(\begin{bmatrix}a_1+a_2\\b_1+b_2\\0\end{bmatrix}\right)\\
&=\begin{bmatrix}0\\0\\(a_1+a_2)+(b_1+b_2)\end{bmatrix}\\
&=\begin{bmatrix}0\\0\\(a_1+b_1)+(a_2+b_2)\end{bmatrix}\\
&=\begin{bmatrix}0\\0\\a_1+b_1\end{bmatrix}+\begin{bmatrix}0\\0\\a_2+b_2\end{bmatrix}\\
&=T\left(\begin{bmatrix}a_1\\b_1\\0\end{bmatrix}\right)+ T\left(\begin{bmatrix}a_2\\b_2\\0\end{bmatrix}\right)
\end{align*}
Verification of the fact that $T\left(k\begin{bmatrix}a_1\\b_1\\0\end{bmatrix}\right)=kT\left(\begin{bmatrix}a_1\\b_1\\0\end{bmatrix}\right)$ is similar, and we omit the details.

We have shown that $T$ is a linear transformation.  $T$ maps all vectors in the $xy$-plane to the $z$-axis.  The following diagram helps us visualize the action of $T$ on a specific vector.

\begin{center}
\begin{tikzpicture}[scale=.8]

  \draw[->] (0,0)--(3,0);
  \draw[->] (0,0)--(0,3);
  \draw[->] (0,0)--(-1.5,-1);

\filldraw[blue, opacity=0.3](1.5,-1)--(4.5,1)--(-1.5,1)--(-4.5,-1)--cycle;
  
\draw[line width=1pt,red,-stealth](0,0)--(2,-0.5) node[below right]{$\begin{bmatrix}1\\2\\0\end{bmatrix}$};

  
 \draw[->] (6,0)--(9,0);
  %\draw[->] (5,0)--(5,3);
  \draw[->] (6,0)--(4.5,-1);
  
  \draw[line width=2pt,blue](6,-1)--(6,3) node[above, right]{Image of $T$};
  \draw[line width=2pt,red,-stealth](6,0)--(6,1.5) node[right]{$T\left(\begin{bmatrix}1\\2\\0\end{bmatrix}\right)=\begin{bmatrix}0\\0\\3\end{bmatrix}$};
  \draw [->,line width=1pt,-stealth]  (2.5,2) to[out=30, in=150] (5, 2);
   \node[] at (3.75, 3)   (b) {$T$};
 \end{tikzpicture}
\end{center}

We can investigate further.  Recall that $T$ is defined by $$T\left(\begin{bmatrix}a\\b\\0\end{bmatrix}\right)=\begin{bmatrix}0\\0\\a+b\end{bmatrix}$$
Complete each of the following statements referring to the diagram below.

\begin{center}
\begin{tikzpicture}[scale=.8]

  \draw[->] (0,0)--(3,0);
  \draw[->] (0,0)--(0,3);
  \draw[->] (0,0)--(-1.5,-1);
\draw[-] (4.5,1)--(-4.5,-1);
\filldraw[blue, opacity=0.3](4.5,1)--(-1.5,1)--(-4.5,-1)--cycle;
\filldraw[orange, opacity=0.3](1.5,-1)--(4.5,1)--(-4.5,-1)--cycle;
   \draw[->] (6,0)--(9,0);
  %\draw[->] (5,0)--(5,3);
  \draw[->] (6,0)--(4.5,-1);
  
  \draw[line width=2pt,blue](6,-1)--(6,3) node[above, right]{Image of $T$};
  %\draw[line width=2pt,red,-stealth](6,0)--(6,1.5) node[right]{$T\left(\begin{bmatrix}1\\2\\0\end{bmatrix}\right)=\begin{bmatrix}0\\0\\3\end{bmatrix}$};
  \draw [->,line width=1pt,-stealth]  (2.5,2) to[out=30, in=150] (5, 2);
   \node[] at (3.75, 3)   (b) {$T$};
 \end{tikzpicture}
\end{center}

The image of the line $y=-x$ under $T$ is \wordChoice{\choice{a line} \choice{a plane} \choice[correct]{the zero vector}\choice{all of $\RR^3$}}

The image of the orange part of the domain (the front triangle) is \wordChoice{\choice[correct]{the positive $z$-axis} \choice{the negative $z$-axis} \choice{the zero vector}\choice{the entire $z$-axis}} 

The image of the purple part of the domain (the back triangle) is \wordChoice{\choice{the positive $z$-axis} \choice[correct]{the negative $z$-axis} \choice{the zero vector}\choice{the entire $z$-axis}} 

\end{explanation}
\end{example}

We conclude this section by introducing two simple but important transformations.

\begin{definition}\label{def:idtransonrn}
The \dfn{identity transformation} on $V$, denoted by $\id_V$, is a transformation that maps each element of $V$ to itself.

In other words,
$$\id_V:V\rightarrow V$$ is a transformation such that $$\id_V(\vec{v})=\vec{v}\quad\text{for all}\quad \vec{v} \in V$$
\end{definition}

\begin{definition}\label{def:zerotransonrn}
The \dfn{zero transformation}, $Z$, maps every element of the domain to the zero vector.

In other words,
$$Z:V\rightarrow W$$ is a transformation such that $$Z(\vec{v})=\vec{0}\quad\text{for all}\quad \vec{v} \in V$$
\end{definition}

\begin{theorem}\label{th:idlintrans}
The identity transformation is linear.
\end{theorem}
\begin{proof}
Left to the reader. (See Practice Problem \ref{prob:idtrans})
\end{proof}

\begin{theorem}\label{th:zerolintrans}
The zero transformation is linear.
\end{theorem}
\begin{proof}
Left to the reader.  (See Practice Problem \ref{prob:zerotrans})
\end{proof}


\section*{Practice Problems}
\begin{problem}\label{prob:sum}

Show that (\ref{lin2}) of Exploration \ref{init:lintransintro} holds for vectors $\begin{bmatrix}3\\4\end{bmatrix}$ and $\begin{bmatrix}-2\\1\end{bmatrix}$.
\end{problem}

\begin{problem}\label{prob:prob2}
Use a counter-example to prove (\ref{t2}) of Exploration \ref{init:lintransintro}.
\end{problem}

\begin{problem}\label{prob:imageoflincomb}
Suppose $T:\RR^{10}\rightarrow\RR^2$ is a linear transformation such that $T(\vec{u})=\begin{bmatrix}2\\-1\end{bmatrix}$ and $T(\vec{v})=\begin{bmatrix}-5\\3\end{bmatrix}$.  Find the image of $3\vec{u}-\vec{v}$.

$$T(3\vec{u}-\vec{v})=\begin{bmatrix}\answer{11}\\\answer{-6}\end{bmatrix}$$
\end{problem}


\begin{problem}\label{prob:notlinear} 
Let $\vec{u}$ be a fixed vector.  Define $T_{\vec{u}}:\RR^2\rightarrow\RR^2$, by $T_{\vec{u}}(\vec{x})=\vec{u}-\vec{x}$.
  \begin{enumerate}
  \item 
  Describe the effect of this transformation by sketching ${\bf x}$ and $T_{\vec{u}}({\bf x})$ for at least four vectors ${\bf x}$ and a fixed vector $\vec{u}$ of your choice.
  \item 
  Is $T_{\vec{u}}$ a linear transformation?
  \end{enumerate}
\end{problem}

\begin{problem}\label{prob:projectiontrans}
Define $P_{xy}:\RR^3\rightarrow\RR^2$, by $P_{xy}\left(\begin{bmatrix}
x\\
y\\
z
\end{bmatrix} \right)=\begin{bmatrix}
x\\
y\\
0
\end{bmatrix}$.  This transformation is called an \dfn{{orthogonal projection}} onto the $xy$-plane.  Show that $P_{xy}$ is a linear transformation.
\end{problem}

\begin{problem}\label{prob:imagesofijk}
Suppose a linear transformation $T:\RR^3\rightarrow\RR^3$ maps ${\bf i}$ to $\begin{bmatrix}2\\-1\\0\end{bmatrix}$, ${\bf j}$ to $\begin{bmatrix}-2\\4\\1\end{bmatrix}$, and ${\bf k}$ to $\begin{bmatrix}3\\0\\-5\end{bmatrix}$.  Find the image of $\begin{bmatrix}1\\1\\-2\end{bmatrix}$ under $T$.

$$T\left(\begin{bmatrix}1\\1\\-2\end{bmatrix}\right)=\begin{bmatrix}\answer{-6}\\\answer{3}\\\answer{11}\end{bmatrix}$$
\end{problem}

\begin{problem}\label{prob:idtrans} Prove Theorem \ref{th:idlintrans}\end{problem}

\begin{problem}\label{prob:zerotrans} Prove Theorem \ref{th:zerolintrans}\end{problem}

\begin{problem}
For each matrix $A$ below, find the domain and the codomain of the linear transformation $T:\RR^n\rightarrow\RR^m$ induced by $A$; find and draw the image of $T$. (Hint: See Example \ref{ex:lineartrans3}.)

  \begin{problem}\label{prob:domaincodomain1}
  $$A=\begin{bmatrix}0&0\\1&1\\2&0\end{bmatrix}$$
  Domain: $\RR^n$, where $n=\answer{2}$.

  Codomain: $\RR^m$, where $m=\answer{3}$.
  \end{problem}
  \begin{problem}\label{prob:domaincodomain2}
  $$A=\begin{bmatrix}3&-1\\-3&1\end{bmatrix}$$
   \end{problem}
\end{problem}

\section*{Exercise Source}
Practice Problem \ref{prob:notlinear} is adopted from Problem 5.1.3 of Ken Kuttler's \href{https://open.umn.edu/opentextbooks/textbooks/a-first-course-in-linear-algebra-2017}{\it A First Course in Linear Algebra}. (CC-BY)

Ken Kuttler, {\it  A First Course in Linear Algebra}, Lyryx 2017, Open Edition, p. 272.

\end{document} 