\documentclass{ximera}
%%% Begin Laad packages

\makeatletter
\@ifclassloaded{xourse}{%
    \typeout{Start loading preamble.tex (in a XOURSE)}%
    \def\isXourse{true}   % automatically defined; pre 112022 it had to be set 'manually' in a xourse
}{%
    \typeout{Start loading preamble.tex (NOT in a XOURSE)}%
}
\makeatother

\def\isEn\true 

\pgfplotsset{compat=1.16}

\usepackage{currfile}

% 201908/202301: PAS OP: babel en doclicense lijken problemen te veroorzaken in .jax bestand
% (wegens syntax error met toegevoegde \newcommands ...)
\pdfOnly{
    \usepackage[type={CC},modifier={by-nc-sa},version={4.0}]{doclicense}
    %\usepackage[hyperxmp=false,type={CC},modifier={by-nc-sa},version={4.0}]{doclicense}
    %%% \usepackage[dutch]{babel}
}



\usepackage[utf8]{inputenc}
\usepackage{morewrites}   % nav zomercursus (answer...?)
\usepackage{multirow}
\usepackage{multicol}
\usepackage{tikzsymbols}
\usepackage{ifthen}
%\usepackage{animate} BREAKS HTML STRUCTURE USED BY XIMERA
\usepackage{relsize}

\usepackage{eurosym}    % \euro  (€ werkt niet in xake ...?)
\usepackage{fontawesome} % smileys etc

% Nuttig als ook interactieve beamer slides worden voorzien:
\providecommand{\p}{} % default nothing ; potentially usefull for slides: redefine as \pause
%providecommand{\p}{\pause}

    % Layout-parameters voor het onderschrift bij figuren
\usepackage[margin=10pt,font=small,labelfont=bf, labelsep=endash,format=hang]{caption}
%\usepackage{caption} % captionof
%\usepackage{pdflscape}    % landscape environment

% Met "\newcommand\showtodonotes{}" kan je todonotes tonen (in pdf/online)
% 201908: online werkt het niet (goed)
\providecommand\showtodonotes{disable}
\providecommand\todo[1]{\typeout{TODO #1}}
%\usepackage[\showtodonotes]{todonotes}
%\usepackage{todonotes}

%
% Poging tot aanpassen layout
%
\usepackage{tcolorbox}
\tcbuselibrary{theorems}

%%% Einde laad packages

%%% Begin Ximera specifieke zaken

\graphicspath{
	{../../}
	{../}
	{./}
  	{../../pictures/}
   	{../pictures/}
   	{./pictures/}
	{./explog/}    % M05 in groeimodellen       
}

%%% Einde Ximera specifieke zaken

%
% define softer blue/red/green, use KU Leuven base colors for blue (and dark orange for red ?)
%
% todo: rather redefine blue/red/green ...?
%\definecolor{xmblue}{rgb}{0.01, 0.31, 0.59}
%\definecolor{xmred}{rgb}{0.89, 0.02, 0.17}
\definecolor{xmdarkblue}{rgb}{0.122, 0.671, 0.835}   % KU Leuven Blauw
\definecolor{xmblue}{rgb}{0.114, 0.553, 0.69}        % KU Leuven Blauw
\definecolor{xmgreen}{rgb}{0.13, 0.55, 0.13}         % No KULeuven variant for green found ...

\definecolor{xmaccent}{rgb}{0.867, 0.541, 0.18}      % KU Leuven Accent (orange ...)
\definecolor{kuaccent}{rgb}{0.867, 0.541, 0.18}      % KU Leuven Accent (orange ...)

\colorlet{xmred}{xmaccent!50!black}                  % Darker version of KU Leuven Accent

\providecommand{\blue}[1]{{\color{blue}#1}}    
\providecommand{\red}[1]{{\color{red}#1}}

\renewcommand\CancelColor{\color{xmaccent!50!black}}

% werkt in math en text mode om MATH met oranje (of grijze...)  achtergond te tonen (ook \important{\text{blabla}} lijkt te werken)
%\newcommand{\important}[1]{\ensuremath{\colorbox{xmaccent!50!white}{$#1$}}}   % werkt niet in Mathjax
%\newcommand{\important}[1]{\ensuremath{\colorbox{lightgray}{$#1$}}}
\newcommand{\important}[1]{\ensuremath{\colorbox{orange}{$#1$}}}   % TODO: kleur aanpassen voor mathjax; wordt overschreven infra!


% Uitzonderlijk kan met \pdfnl in de PDF een newline worden geforceerd, die online niet nodig/nuttig is omdat daar de regellengte hoe dan ook niet gekend is.
\ifdefined\HCode%
\providecommand{\pdfnl}{}%
\else%
\providecommand{\pdfnl}{%
  \\%
}%
\fi

% Uitzonderlijk kan met \handoutnl in de handout-PDF een newline worden geforceerd, die noch online noch in de PDF-met-antwoorden nuttig is.
\ifdefined\HCode
\providecommand{\handoutnl}{}
\else
\providecommand{\handoutnl}{%
\ifhandout%
  \nl%
\fi%
}
\fi



% \cellcolor IGNORED by tex4ht ?
% \begin{center} seems not to wordk
    % (missing margin-left: auto;   on tabular-inside-center ???)
%\newcommand{\importantcell}[1]{\ensuremath{\cellcolor{lightgray}#1}}  %  in tabular; usablility to be checked
\providecommand{\importantcell}[1]{\ensuremath{#1}}     % no mathjax2 support for colloring array cells

\pdfOnly{
  \renewcommand{\important}[1]{\ensuremath{\colorbox{kuaccent!50!white}{$#1$}}}
  \renewcommand{\importantcell}[1]{\ensuremath{\cellcolor{kuaccent!40!white}#1}}   
}

%%% Tikz styles


\pgfplotsset{compat=1.16}

\usetikzlibrary{trees,positioning,arrows,fit,shapes,math,calc,decorations.markings,through,intersections,patterns,matrix}

\usetikzlibrary{decorations.pathreplacing,backgrounds}    % 5/2023: from experimental


\usetikzlibrary{angles,quotes}

\usepgfplotslibrary{fillbetween} % bepaalde_integraal
\usepgfplotslibrary{polar}    % oa voor poolcoordinaten.tex

\pgfplotsset{ownstyle/.style={axis lines = center, axis equal image, xlabel = $x$, ylabel = $y$, enlargelimits}} 

\pgfplotsset{
	plot/.style={no marks,samples=50}
}

\newcommand{\xmPlotsColor}{
	\pgfplotsset{
		plot1/.style={darkgray,no marks,samples=100},
		plot2/.style={lightgray,no marks,samples=100},
		plotresult/.style={blue,no marks,samples=100},
		plotblue/.style={blue,no marks,samples=100},
		plotred/.style={red,no marks,samples=100},
		plotgreen/.style={green,no marks,samples=100},
		plotpurple/.style={purple,no marks,samples=100}
	}
}
\newcommand{\xmPlotsBlackWhite}{
	\pgfplotsset{
		plot1/.style={black,loosely dashed,no marks,samples=100},
		plot2/.style={black,loosely dotted,no marks,samples=100},
		plotresult/.style={black,no marks,samples=100},
		plotblue/.style={black,no marks,samples=100},
		plotred/.style={black,dotted,no marks,samples=100},
		plotgreen/.style={black,dashed,no marks,samples=100},
		plotpurple/.style={black,dashdotted,no marks,samples=100}
	}
}


\newcommand{\xmPlotsColorAndStyle}{
	\pgfplotsset{
		plot1/.style={darkgray,no marks,samples=100},
		plot2/.style={lightgray,no marks,samples=100},
		plotresult/.style={blue,no marks,samples=100},
		plotblue/.style={xmblue,no marks,samples=100},
		plotred/.style={xmred,dashed,thick,no marks,samples=100},
		plotgreen/.style={xmgreen,dotted,very thick,no marks,samples=100},
		plotpurple/.style={purple,no marks,samples=100}
	}
}


%\iftikzexport
\xmPlotsColorAndStyle
%\else
%\xmPlotsBlackWhite
%\fi
%%%


%
% Om venndiagrammen te arceren ...
%
\makeatletter
\pgfdeclarepatternformonly[\hatchdistance,\hatchthickness]{north east hatch}% name
{\pgfqpoint{-1pt}{-1pt}}% below left
{\pgfqpoint{\hatchdistance}{\hatchdistance}}% above right
{\pgfpoint{\hatchdistance-1pt}{\hatchdistance-1pt}}%
{
	\pgfsetcolor{\tikz@pattern@color}
	\pgfsetlinewidth{\hatchthickness}
	\pgfpathmoveto{\pgfqpoint{0pt}{0pt}}
	\pgfpathlineto{\pgfqpoint{\hatchdistance}{\hatchdistance}}
	\pgfusepath{stroke}
}
\pgfdeclarepatternformonly[\hatchdistance,\hatchthickness]{north west hatch}% name
{\pgfqpoint{-\hatchthickness}{-\hatchthickness}}% below left
{\pgfqpoint{\hatchdistance+\hatchthickness}{\hatchdistance+\hatchthickness}}% above right
{\pgfpoint{\hatchdistance}{\hatchdistance}}%
{
	\pgfsetcolor{\tikz@pattern@color}
	\pgfsetlinewidth{\hatchthickness}
	\pgfpathmoveto{\pgfqpoint{\hatchdistance+\hatchthickness}{-\hatchthickness}}
	\pgfpathlineto{\pgfqpoint{-\hatchthickness}{\hatchdistance+\hatchthickness}}
	\pgfusepath{stroke}
}
%\makeatother

\tikzset{
    hatch distance/.store in=\hatchdistance,
    hatch distance=10pt,
    hatch thickness/.store in=\hatchthickness,
   	hatch thickness=2pt
}

\colorlet{circle edge}{black}
\colorlet{circle area}{blue!20}


\tikzset{
    filled/.style={fill=green!30, draw=circle edge, thick},
    arceerl/.style={pattern=north east hatch, pattern color=blue!50, draw=circle edge},
    arceerr/.style={pattern=north west hatch, pattern color=yellow!50, draw=circle edge},
    outline/.style={draw=circle edge, thick}
}




%%% Updaten commando's
\def\hoofding #1#2#3{\maketitle}     % OBSOLETE ??

% we willen (bijna) altijd \geqslant ipv \geq ...!
\newcommand{\geqnoslant}{\geq}
\renewcommand{\geq}{\geqslant}
\newcommand{\leqnoslant}{\leq}
\renewcommand{\leq}{\leqslant}

% Todo: (201908) waarom komt er (soms) underlined voor emph ...?
\renewcommand{\emph}[1]{\textit{#1}}

% API commando's

\newcommand{\ds}{\displaystyle}
\newcommand{\ts}{\textstyle}  % tegenhanger van \ds   (Ximera zet PER  DEFAULT \ds!)

% uit Zomercursus-macro's: 
\newcommand{\bron}[1]{\begin{scriptsize} \emph{#1} \end{scriptsize}}     % deprecated ...?


%definities nieuwe commando's - afkortingen veel gebruikte symbolen
\newcommand{\R}{\ensuremath{\mathbb{R}}}
\newcommand{\Rnul}{\ensuremath{\mathbb{R}_0}}
\newcommand{\Reen}{\ensuremath{\mathbb{R}\setminus\{1\}}}
\newcommand{\Rnuleen}{\ensuremath{\mathbb{R}\setminus\{0,1\}}}
\newcommand{\Rplus}{\ensuremath{\mathbb{R}^+}}
\newcommand{\Rmin}{\ensuremath{\mathbb{R}^-}}
\newcommand{\Rnulplus}{\ensuremath{\mathbb{R}_0^+}}
\newcommand{\Rnulmin}{\ensuremath{\mathbb{R}_0^-}}
\newcommand{\Rnuleenplus}{\ensuremath{\mathbb{R}^+\setminus\{0,1\}}}
\newcommand{\N}{\ensuremath{\mathbb{N}}}
\newcommand{\Nnul}{\ensuremath{\mathbb{N}_0}}
\newcommand{\Z}{\ensuremath{\mathbb{Z}}}
\newcommand{\Znul}{\ensuremath{\mathbb{Z}_0}}
\newcommand{\Zplus}{\ensuremath{\mathbb{Z}^+}}
\newcommand{\Zmin}{\ensuremath{\mathbb{Z}^-}}
\newcommand{\Znulplus}{\ensuremath{\mathbb{Z}_0^+}}
\newcommand{\Znulmin}{\ensuremath{\mathbb{Z}_0^-}}
\newcommand{\C}{\ensuremath{\mathbb{C}}}
\newcommand{\Cnul}{\ensuremath{\mathbb{C}_0}}
\newcommand{\Cplus}{\ensuremath{\mathbb{C}^+}}
\newcommand{\Cmin}{\ensuremath{\mathbb{C}^-}}
\newcommand{\Cnulplus}{\ensuremath{\mathbb{C}_0^+}}
\newcommand{\Cnulmin}{\ensuremath{\mathbb{C}_0^-}}
\newcommand{\Q}{\ensuremath{\mathbb{Q}}}
\newcommand{\Qnul}{\ensuremath{\mathbb{Q}_0}}
\newcommand{\Qplus}{\ensuremath{\mathbb{Q}^+}}
\newcommand{\Qmin}{\ensuremath{\mathbb{Q}^-}}
\newcommand{\Qnulplus}{\ensuremath{\mathbb{Q}_0^+}}
\newcommand{\Qnulmin}{\ensuremath{\mathbb{Q}_0^-}}

\newcommand{\perdef}{\overset{\mathrm{def}}{=}}
\newcommand{\pernot}{\overset{\mathrm{notatie}}{=}}
\newcommand\perinderdaad{\overset{!}{=}}     % voorlopig gebruikt in limietenrekenregels
\newcommand\perhaps{\overset{?}{=}}          % voorlopig gebruikt in limietenrekenregels

\newcommand{\degree}{^\circ}


\DeclareMathOperator{\dom}{dom}     % domein
\DeclareMathOperator{\codom}{codom} % codomein
\DeclareMathOperator{\bld}{bld}     % beeld
\DeclareMathOperator{\graf}{graf}   % grafiek
\DeclareMathOperator{\rico}{rico}   % richtingcoëfficient
\DeclareMathOperator{\co}{co}       % coordinaat
\DeclareMathOperator{\gr}{gr}       % graad

\newcommand{\func}[5]{\ensuremath{#1: #2 \rightarrow #3: #4 \mapsto #5}} % Easy to write a function


% Operators
\DeclareMathOperator{\bgsin}{bgsin}
\DeclareMathOperator{\bgcos}{bgcos}
\DeclareMathOperator{\bgtan}{bgtan}
\DeclareMathOperator{\bgcot}{bgcot}
\DeclareMathOperator{\bgsinh}{bgsinh}
\DeclareMathOperator{\bgcosh}{bgcosh}
\DeclareMathOperator{\bgtanh}{bgtanh}
\DeclareMathOperator{\bgcoth}{bgcoth}

% Oude \Bgsin etc deprecated: gebruik \bgsin, en herdefinieer dat als je Bgsin wil!
%\DeclareMathOperator{\cosec}{cosec}    % not used? gebruik \csc en herdefinieer

% operatoren voor differentialen: to be verified; 1/2020: inconsequent gebruik bij afgeleiden/integralen
\renewcommand{\d}{\mathrm{d}}
\newcommand{\dx}{\d x}
\newcommand{\dd}[1]{\frac{\mathrm{d}}{\mathrm{d}#1}}
\newcommand{\ddx}{\dd{x}}

% om in voorbeelden/oefeningen de notatie voor afgeleiden te kunnen kiezen
% Usage: \afg{(2\sin(x))}  (en wordt d/dx, of accent, of D )
%\newcommand{\afg}[1]{{#1}'}
\newcommand{\afg}[1]{\left(#1\right)'}
%\renewcommand{\afg}[1]{\frac{\mathrm{d}#1}{\mathrm{d}x}}   % include in relevant exercises ...
%\renewcommand{\afg}[1]{D{#1}}

%
% \xmxxx commands: Extra KU Leuven functionaliteit van, boven of naast Ximera
%   ( Conventie 8/2019: xm+nederlandse omschrijving, maar is niet consequent gevolgd, en misschien ook niet erg handig !)
%
% (Met een minimale ximera.cls en preamble.tex zou een bruikbare .pdf moeten kunnen worden gemaakt van eender welke ximera)
%
% Usage: \xmtitle[Mijn korte abstract]{Mijn titel}{Mijn abstract}
% Eerste command na \begin{document}:
%  -> definieert de \title
%  -> definieert de abstract
%  -> doet \maketitle ( dus: print de hoofding als 'chapter' of 'sectie')
% Optionele parameter geeft eenn kort abstract (die met de globale setting \xmshortabstract{} al dan niet kan worden geprint.
% De optionele korte abstract kan worden gebruikt voor pseudo-grappige abtsarts, dus dus globaal al dan niet kunnen worden gebuikt...
% Globale settings:
%  de (optionele) 'korte abstract' wordt enkele getoond als \xmshortabstract is gezet
\providecommand\xmshortabstract{} % default: print (only!) short abstract if present
\newcommand{\xmtitle}[3][]{
	\title{#2}
	\begin{abstract}
		\ifdefined\xmshortabstract
		\ifstrempty{#1}{%
			#3
		}{%
			#1
		}%
		\else
		#3
		\fi
	\end{abstract}
	\maketitle
}

% 
% Kleine grapjes: moeten zonder verder gevolg kunnen worden verwijderd
%
%\newcommand{\xmopje}[1]{{\small#1{\reversemarginpar\marginpar{\Smiley}}}}   % probleem in floats!!
\newtoggle{showxmopje}
\toggletrue{showxmopje}

\newcommand{\xmopje}[1]{%
   \iftoggle{showxmopje}{#1}{}%
}


% -> geef een abstracte-formule-met-rechts-een-concreet-voorbeeld
% VB:  \formulevb{a^2+b^2=c^2}{3^2+4^2=5^2}
%
\ifdefined\HCode
\NewEnviron{xmdiv}[1]{\HCode{\Hnewline<div class="#1">\Hnewline}\BODY{\HCode{\Hnewline</div>\Hnewline}}}
\else
\NewEnviron{xmdiv}[1]{\BODY}
\fi

\providecommand{\formulevb}[2]{
	{\centering

    \begin{xmdiv}{xmformulevb}    % zie css voor online layout !!!
	\begin{tabular}{lcl}
		\important{#1}
		&  &
		Vb: $#2$
		\end{tabular}
	\end{xmdiv}

	}
}

\ifdefined\HCode
\providecommand{\vb}[1]{%
    \HCode{\Hnewline<span class="xmvb">}#1\HCode{</span>\Hnewline}%
}
\else
\providecommand{\vb}[1]{
    \colorbox{blue!10}{#1}
}
\fi

\ifdefined\HCode
\providecommand{\xmcolorbox}[2]{
	\HCode{\Hnewline<div class="xmcolorbox">\Hnewline}#2\HCode{\Hnewline</div>\Hnewline}
}
\else
\providecommand{\xmcolorbox}[2]{
  \cellcolor{#1}#2
}
\fi


\ifdefined\HCode
\providecommand{\xmopmerking}[1]{
 \HCode{\Hnewline<div class="xmopmerking">\Hnewline}#1\HCode{\Hnewline</div>\Hnewline}
}
\else
\providecommand{\xmopmerking}[1]{
	{\footnotesize #1}
}
\fi
% \providecommand{\voorbeeld}[1]{
% 	\colorbox{blue!10}{$#1$}
% }



% Hernoem Proof naar Bewijs, nodig voor HTML versie
\renewcommand*{\proofname}{Bewijs}

% Om opgave van oefening (wordt niet geprint bij oplossingenblad)
% (to be tested test)
\NewEnviron{statement}{\BODY}

% Environment 'oplossing' en 'uitkomst'
% voor resp. volledige 'uitwerking' dan wel 'enkel eindresultaat'
% geimplementeerd via feedback, omdat er in de ximera-server adhoc feedback-code is toegevoegd
%% Niet tonen indien handout
%% Te gebruiken om volledige oplossingen/uitwerkingen van oefeningen te tonen
%% \begin{oplossing}        De optelling is commutatief \end{oplossing}  : verschijnt online enkel 'op vraag'
%% \begin{oplossing}[toon]  De optelling is commutatief \end{oplossing}  : verschijnt steeds onmiddellijk online (bv te gebruiken bij voorbeelden) 

\ifhandout%
    \NewEnviron{oplossing}[1][onzichtbaar]%
    {%
    \ifthenelse{\equal{\detokenize{#1}}{\detokenize{toon}}}
    {
    \def\PH@Command{#1}% Use PH@Command to hold the content and be a target for "\expandafter" to expand once.

    \begin{trivlist}% Begin the trivlist to use formating of the "Feedback" label.
    \item[\hskip \labelsep\small\slshape\bfseries Oplossing% Format the "Feedback" label. Don't forget the space.
    %(\texttt{\detokenize\expandafter{\PH@Command}}):% Format (and detokenize) the condition for feedback to trigger
    \hspace{2ex}]\small%\slshape% Insert some space before the actual feedback given.
    \BODY
    \end{trivlist}
    }
    {  % \begin{feedback}[solution]   \BODY     \end{feedback}  }
    }
    }    
\else
% ONLY for HTML; xmoplossing is styled with css, and is not, and need not be a LaTeX environment
% THUS: it does NOT use feedback anymore ...
%    \NewEnviron{oplossing}{\begin{expandable}{xmoplossing}{\nlen{Toon uitwerking}{Show solution}}{\BODY}\end{expandable}}
    \newenvironment{oplossing}[1][onzichtbaar]
   {%
       \begin{expandable}{xmoplossing}{}
   }
   {%
   	   \end{expandable}
   } 
%     \newenvironment{oplossing}[1][onzichtbaar]
%    {%
%        \begin{feedback}[solution]   	
%    }
%    {%
%    	   \end{feedback}
%    } 
\fi

\ifhandout%
    \NewEnviron{uitkomst}[1][onzichtbaar]%
    {%
    \ifthenelse{\equal{\detokenize{#1}}{\detokenize{toon}}}
    {
    \def\PH@Command{#1}% Use PH@Command to hold the content and be a target for "\expandafter" to expand once.

    \begin{trivlist}% Begin the trivlist to use formating of the "Feedback" label.
    \item[\hskip \labelsep\small\slshape\bfseries Uitkomst:% Format the "Feedback" label. Don't forget the space.
    %(\texttt{\detokenize\expandafter{\PH@Command}}):% Format (and detokenize) the condition for feedback to trigger
    \hspace{2ex}]\small%\slshape% Insert some space before the actual feedback given.
    \BODY
    \end{trivlist}
    }
    {  % \begin{feedback}[solution]   \BODY     \end{feedback}  }
    }
    }    
\else
\ifdefined\HCode
   \newenvironment{uitkomst}[1][onzichtbaar]
    {%
        \begin{expandable}{xmuitkomst}{}%
    }
    {%
    	\end{expandable}%
    } 
\else
  % Do NOT print 'uitkomst' in non-handout
  %  (presumably, there is also an 'oplossing' ??)
  \newenvironment{uitkomst}[1][onzichtbaar]{}{}
\fi
\fi

%
% Uitweidingen zijn extra's die niet redelijkerwijze tot de leerstof behoren
% Uitbreidingen zijn extra's die wel redelijkerwijze tot de leerstof van bv meer geavanceerde versies kunnen behoren (B-programma/Wiskundestudenten/...?)
% Nog niet voorzien: design voor verschillende versies (A/B programma, BIO, voorkennis/ ...)
% Voor 'uitweidingen' is er een environment die online per default is ingeklapt, en in pdf al dan niet kan worden geincluded  (via \xmnouitweiding) 
%
% in een xourse, per default GEEN uitweidingen, tenzij \xmuitweiding expliciet ergens is gezet ...
\ifdefined\isXourse
   \ifdefined\xmuitweiding
   \else
       \def\xmnouitweiding{true}
   \fi
\fi

\ifdefined\xmnouitweiding
\newcounter{xmuitweiding}  % anders error undefined ...  
\excludecomment{xmuitweiding}
\else
\newtheoremstyle{dotless}{}{}{}{}{}{}{ }{}
\theoremstyle{dotless}
\newtheorem*{xmuitweidingnofrills}{}   % nofrills = no accordion; gebruikt dus de dotless theoremstyle!

\newcounter{xmuitweiding}
\newenvironment{xmuitweiding}[1][ ]%
{% 
	\refstepcounter{xmuitweiding}%
    \begin{expandable}{xmuitweiding}{\nlentext{Uitweiding \arabic{xmuitweiding}: #1}{Digression \arabic{xmuitweiding}: #1}}%
	\begin{xmuitweidingnofrills}%
}
{%
    \end{xmuitweidingnofrills}%
    \end{expandable}%
}   
% \newenvironment{xmuitweiding}[1][ ]%
% {% 
% 	\refstepcounter{xmuitweiding}
% 	\begin{accordion}\begin{accordion-item}[Uitweiding \arabic{xmuitweiding}: #1]%
% 			\begin{xmuitweidingnofrills}%
% 			}
% 			{\end{xmuitweidingnofrills}\end{accordion-item}\end{accordion}}   
\fi


\newenvironment{xmexpandable}[1][]{
	\begin{accordion}\begin{accordion-item}[#1]%
		}{\end{accordion-item}\end{accordion}}


% Command that gives a selection box online, but just prints the right answer in pdf
\newcommand{\xmonlineChoice}[1]{\pdfOnly{\wordchoicegiventrue}\wordChoice{#1}\pdfOnly{\wordchoicegivenfalse}}   % deprecated, gebruik onlineChoice ...
\newcommand{\onlineChoice}[1]{\pdfOnly{\wordchoicegiventrue}\wordChoice{#1}\pdfOnly{\wordchoicegivenfalse}}

\newcommand{\TJa}{\nlentext{ Ja }{ Yes }}
\newcommand{\TNee}{\nlentext{ Nee }{ No }}
\newcommand{\TJuist}{\nlentext{ Juist }{ True }}
\newcommand{\TFout}{\nlentext{ Fout }{ False }}

\newcommand{\choiceTrue }{{\renewcommand{\choiceminimumhorizontalsize}{4em}\wordChoice{\choice[correct]{\TJuist}\choice{\TFout}}}}
\newcommand{\choiceFalse}{{\renewcommand{\choiceminimumhorizontalsize}{4em}\wordChoice{\choice{\TJuist}\choice[correct]{\TFout}}}}

\newcommand{\choiceYes}{{\renewcommand{\choiceminimumhorizontalsize}{3em}\wordChoice{\choice[correct]{\TJa}\choice{\TNee}}}}
\newcommand{\choiceNo }{{\renewcommand{\choiceminimumhorizontalsize}{3em}\wordChoice{\choice{\TJa}\choice[correct]{\TNee}}}}

% Optional nicer formatting for wordChoice in PDF

\let\inlinechoiceorig\inlinechoice

%\makeatletter
%\providecommand{\choiceminimumverticalsize}{\vphantom{$\frac{\sqrt{2}}{2}$}}   % minimum height of boxes (cfr infra)
\providecommand{\choiceminimumverticalsize}{\vphantom{$\tfrac{2}{2}$}}   % minimum height of boxes (cfr infra)
\providecommand{\choiceminimumhorizontalsize}{1em}   % minimum width of boxes (cfr infra)

\newcommand{\inlinechoicesquares}[2][]{%
		\setkeys{choice}{#1}%
		\ifthenelse{\boolean{\choice@correct}}%
		{%
            \ifhandout%
               \fbox{\choiceminimumverticalsize #2}\allowbreak\ignorespaces%
            \else%
               \fcolorbox{blue}{blue!20}{\choiceminimumverticalsize #2}\allowbreak\ignorespaces\setkeys{choice}{correct=false}\ignorespaces%
            \fi%
		}%
		{% else
			\fbox{\choiceminimumverticalsize #2}\allowbreak\ignorespaces%  HACK: wat kleiner, zodat fits on line ... 	
		}%
}

\newcommand{\inlinechoicesquareX}[2][]{%
		\setkeys{choice}{#1}%
		\ifthenelse{\boolean{\choice@correct}}%
		{%
            \ifhandout%
               \framebox[\ifdim\choiceminimumhorizontalsize<\width\width\else\choiceminimumhorizontalsize\fi]{\choiceminimumverticalsize\ #2\ }\allowbreak\ignorespaces\setkeys{choice}{correct=false}\ignorespaces%
            \else%
               \fcolorbox{blue}{blue!20}{\makebox[\ifdim\choiceminimumhorizontalsize<\width\width\else\choiceminimumhorizontalsize\fi]{\choiceminimumverticalsize #2}}\allowbreak\ignorespaces\setkeys{choice}{correct=false}\ignorespaces%
            \fi%
		}%
		{% else
        \ifhandout%
			\framebox[\ifdim\choiceminimumhorizontalsize<\width\width\else\choiceminimumhorizontalsize\fi]{\choiceminimumverticalsize\ #2\ }\allowbreak\ignorespaces%  HACK: wat kleiner, zodat fits on line ... 	
        \fi
		}%
}


\newcommand{\inlinechoicepuntjes}[2][]{%
		\setkeys{choice}{#1}%
		\ifthenelse{\boolean{\choice@correct}}%
		{%
            \ifhandout%
               \dots\ldots\ignorespaces\setkeys{choice}{correct=false}\ignorespaces
            \else%
               \fcolorbox{blue}{blue!20}{\choiceminimumverticalsize #2}\allowbreak\ignorespaces\setkeys{choice}{correct=false}\ignorespaces%
            \fi%
		}%
		{% else
			%\fbox{\choiceminimumverticalsize #2}\allowbreak\ignorespaces%  HACK: wat kleiner, zodat fits on line ... 	
		}%
}

% print niets, maar definieer globale variable \myanswer
%  (gebruikt om oplossingsbladen te printen) 
\newcommand{\inlinechoicedefanswer}[2][]{%
		\setkeys{choice}{#1}%
		\ifthenelse{\boolean{\choice@correct}}%
		{%
               \gdef\myanswer{#2}\setkeys{choice}{correct=false}

		}%
		{% else
			%\fbox{\choiceminimumverticalsize #2}\allowbreak\ignorespaces%  HACK: wat kleiner, zodat fits on line ... 	
		}%
}



%\makeatother

\newcommand{\setchoicedefanswer}{
\ifdefined\HCode
\else
%    \renewenvironment{multipleChoice@}[1][]{}{} % remove trailing ')'
    \let\inlinechoice\inlinechoicedefanswer
\fi
}

\newcommand{\setchoicepuntjes}{
\ifdefined\HCode
\else
    \renewenvironment{multipleChoice@}[1][]{}{} % remove trailing ')'
    \let\inlinechoice\inlinechoicepuntjes
\fi
}
\newcommand{\setchoicesquares}{
\ifdefined\HCode
\else
    \renewenvironment{multipleChoice@}[1][]{}{} % remove trailing ')'
    \let\inlinechoice\inlinechoicesquares
\fi
}
%
\newcommand{\setchoicesquareX}{
\ifdefined\HCode
\else
    \renewenvironment{multipleChoice@}[1][]{}{} % remove trailing ')'
    \let\inlinechoice\inlinechoicesquareX
\fi
}
%
\newcommand{\setchoicelist}{
\ifdefined\HCode
\else
    \renewenvironment{multipleChoice@}[1][]{}{)}% re-add trailing ')'
    \let\inlinechoice\inlinechoiceorig
\fi
}

\setchoicesquareX  % by default list-of-squares with onlineChoice in PDF

% Omdat multicols niet werkt in html: enkel in pdf  (in html zijn langere pagina's misschien ook minder storend)
\newenvironment{xmmulticols}[1][2]{
 \pdfOnly{\begin{multicols}{#1}}%
}{ \pdfOnly{\end{multicols}}}

%
% Te gebruiken in plaats van \section\subsection
%  (in een printstyle kan dan het level worden aangepast
%    naargelang \chapter vs \section style )
% 3/2021: DO NOT USE \xmsubsection !
\newcommand\xmsection\subsection
\newcommand\xmsubsection\subsubsection

% Aanpassen printversie
%  (hier gedefinieerd, zodat ze in xourse kunnen worden gezet/overschreven)
\providebool{parttoc}
\providebool{printpartfrontpage}
\providebool{printactivitytitle}
\providebool{printactivityqrcode}
\providebool{printactivityurl}
\providebool{printcontinuouspagenumbers}
\providebool{numberactivitiesbysubpart}
\providebool{addtitlenumber}
\providebool{addsectiontitlenumber}
\addtitlenumbertrue
\addsectiontitlenumbertrue

% The following three commands are hardcoded in xake, you can't create other commands like these, without adding them to xake as well
%  ( gebruikt in xourses om juiste soort titelpagina te krijgen voor verschillende ximera's )
\newcommand{\activitychapter}[2][]{
    {    
    \ifstrequal{#1}{notnumbered}{
        \addtitlenumberfalse
    }{}
    \typeout{ACTIVITYCHAPTER #2}   % logging
	\chapterstyle
	\activity{#2}
    }
}
\newcommand{\activitysection}[2][]{
    {
    \ifstrequal{#1}{notnumbered}{
        \addsectiontitlenumberfalse
    }{}
	\typeout{ACTIVITYSECTION #2}   % logging
	\sectionstyle
	\activity{#2}
    }
}
% Practices worden als activity getoond om de grote blokken te krijgen online
\newcommand{\practicesection}[2][]{
    {
    \ifstrequal{#1}{notnumbered}{
        \addsectiontitlenumberfalse
    }{}
    \typeout{PRACTICESECTION #2}   % logging
	\sectionstyle
	\activity{#2}
    }
}
\newcommand{\activitychapterlink}[3][]{
    {
    \ifstrequal{#1}{notnumbered}{
        \addtitlenumberfalse
    }{}
    \typeout{ACTIVITYCHAPTERLINK #3}   % logging
	\chapterstyle
	\activitylink[#1]{#2}{#3}
    }
}

\newcommand{\activitysectionlink}[3][]{
    {
    \ifstrequal{#1}{notnumbered}{
        \addsectiontitlenumberfalse
    }{}
    \typeout{ACTIVITYSECTIONLINK #3}   % logging
	\sectionstyle
	\activitylink[#1]{#2}{#3}
    }
}


% Commando om de printstyle toe te voegen in ximera's. Zorgt ervoor dat er geen problemen zijn als je de xourses compileert
% hack om onhandige relative paden in TeX te omzeilen
% should work both in xourse and ximera (pre-112022 only in ximera; thus obsoletes adhoc setup in xourses)
% loads global.sty if present (cfr global.css for online settings!)
% use global.sty to overwrite settings in printstyle.sty ...
\newcommand{\addPrintStyle}[1]{
\iftikzexport\else   % only in PDF
  \makeatletter
  \ifx\@onlypreamble\@notprerr\else   % ONLY if in tex-preamble   (and e.g. not when included from xourse)
    \typeout{Loading printstyle}   % logging
    \usepackage{#1/printstyle} % mag enkel geinclude worden als je die apart compileert
    \IfFileExists{#1/global.sty}{
        \typeout{Loading printstyle-folder #1/global.sty}   % logging
        \usepackage{#1/global}
        }{
        \typeout{Info: No extra #1/global.sty}   % logging
    }   % load global.sty if present
    \IfFileExists{global.sty}{
        \typeout{Loading local-folder global.sty (or TEXINPUTPATH..)}   % logging
        \usepackage{global}
    }{
        \typeout{Info: No folder/global.sty}   % logging
    }   % load global.sty if present
    \IfFileExists{\currfilebase.sty}
    {
        \typeout{Loading \currfilebase.sty}
        \input{\currfilebase.sty}
    }{
        \typeout{Info: No local \currfilebase.sty}
    }
    \fi
  \makeatother
\fi
}

%
%  
% references: Ximera heeft adhoc logica	 om online labels te doen werken over verschillende files heen
% met \hyperref kan de getoonde tekst toch worden opgegeven, in plaats van af te hangen van de label-text
\ifdefined\HCode
% Link to standard \labels, but give your own description
% Usage:  Volg \hyperref[my_very_verbose_label]{deze link} voor wat tijdverlies
%   (01/2020: Ximera-server aangepast om bij class reference-keeptext de link-text NIET te vervangen door de label-text !!!) 
\renewcommand{\hyperref}[2][]{\HCode{<a class="reference reference-keeptext" href="\##1">}#2\HCode{</a>}}
%
%  Link to specific targets  (not tested ?)
\renewcommand{\hypertarget}[1]{\HCode{<a class="ximera-label" id="#1"></a>}}
\renewcommand{\hyperlink}[2]{\HCode{<a class="reference reference-keeptext" href="\##1">}#2\HCode{</a>}}
\fi

% Mmm, quid English ... (for keyword #1 !) ?
\newcommand{\wikilink}[2]{
    \hyperlink{https://nl.wikipedia.org/wiki/#1}{#2}
    \pdfOnly{\footnote{See \url{https://nl.wikipedia.org/wiki/#1}}
    }
}

\renewcommand{\figurename}{Figuur}
\renewcommand{\tablename}{Tabel}

%
% Gedoe om verschillende versies van xourse/ximera te maken afhankelijk van settings
%
% default: versie met antwoorden
% handout: versie voor de studenten, zonder antwoorden/oplossingen
% full: met alles erop en eraan, dus geschikt voor auteurs en/of lesgevers  (bevat in de pdf ook de 'online-only' stukken!)
%
%
% verder kunnen ook opties/variabele worden gezet voor hints/auteurs/uitweidingen/ etc
%
% 'Full' versie
\newtoggle{showonline}
\ifdefined\HCode   % zet default showOnline
    \toggletrue{showonline} 
\else
    \togglefalse{showonline}
\fi

% Full versie   % deprecated: see infra
\newcommand{\printFull}{
    \hintstrue
    \handoutfalse
    \toggletrue{showonline} 
}

\ifdefined\shouldPrintFull   % deprecated: see infra
    \printFull
\fi



% Overschrijf onlineOnly  (zoals gedefinieerd in ximera.cls)
\ifhandout   % in handout: gebruik de oorspronkelijke ximera.cls implementatie  (is dit wel nodig/nuttig?)
\else
    \iftoggle{showonline}{%
        \ifdefined\HCode
          \RenewEnviron{onlineOnly}{\bgroup\BODY\egroup}   % showOnline, en we zijn  online, dus toon de tekst
        \else
          \RenewEnviron{onlineOnly}{\bgroup\color{red!50!black}\BODY\egroup}   % showOnline, maar we zijn toch niet online: kleur de tekst rood 
        \fi
    }{%
      \RenewEnviron{onlineOnly}{}  % geen showOnline
    }
\fi

% hack om na hoofding van definition/proposition/... als dan niet op een nieuwe lijn te starten
% soms is dat goed en mooi, en soms niet; en in HTML is het nu (2/2020) anders dan in pdf
% vandaar suggestie om 
%     \begin{definition}[Nieuw concept] \nl
% te gebruiken als je zeker een newline wil na de hoofdig en titel
% (in het bijzonder itemize zonder \nl is 'lelijk' ...)
\ifdefined\HCode
\newcommand{\nl}{}
\else
\newcommand{\nl}{\ \par} % newline (achter heading van definition etc.)
\fi


% \nl enkel in handoutmode (ihb voor \wordChoice, die dan typisch veeeel langer wordt)
\ifdefined\HCode
\providecommand{\handoutnl}{}
\else
\providecommand{\handoutnl}{%
\ifhandout%
  \nl%
\fi%
}
\fi

% Could potentially replace \pdfOnline/\begin{onlineOnly} : 
% Usage= \ifonline{Hallo surfer}{Hallo PDFlezer}
\providecommand{\ifonline}[2]%
{
\begin{onlineOnly}#1\end{onlineOnly}%
\pdfOnly{#2}
}%


%
% Maak optionele 'basic' en 'extended' versies van een activity
%  met environment basicOnly, basicSkip en extendedOnly
%
%  (
%   Dit werkt ENKEL in de PDF; de online versies tonen (minstens voorklopig) steeds 
%   het default geval met printbasicversion en printextendversion beide FALSE
%  )
%
\providebool{printbasicversion}
\providebool{printextendedversion}   % not properly implemented
\providebool{printfullversion}       % presumably print everything (debug/auteur)
%
% only set these in xourses, and BEFORE loading this preamble
%
%\newif\ifshowbasic     \showbasictrue        % use this line in xourse to show 'basic' sections
%\newif\ifshowextended  \showextendedtrue     % use this line in xourse to show 'extended' sections
%
%
%\ifbool{showbasic}
%      { \NewEnviron{basicOnly}{\BODY} }    % if yes: just print contents
%      { \NewEnviron{basicOnly}{}      }    % if no:  completely ignore contents
%
%\ifbool{showbasic}
%      { \NewEnviron{basicSkip}{}      }
%      { \NewEnviron{basicSkip}{\BODY} }
%

\ifbool{printextendedversion}
      { \NewEnviron{extendedOnly}{\BODY} }
      { \NewEnviron{extendedOnly}{}      }
      


\ifdefined\HCode    % in html: always print
      {\newenvironment*{basicOnly}{}{}}    % if yes: just print contents
      {\newenvironment*{basicSkip}{}{}}    % if yes: just print contents
\else

\ifbool{printbasicversion}
      {\newenvironment*{basicOnly}{}{}}    % if yes: just print contents
      {\NewEnviron{basicOnly}{}      }    % if no:  completely ignore contents

\ifbool{printbasicversion}
      {\NewEnviron{basicSkip}{}      }
      {\newenvironment*{basicSkip}{}{}}

\fi

\usepackage{float}
\usepackage[rightbars,color]{changebar}

% Full versie
\ifbool{printfullversion}{
    \hintstrue
    \handoutfalse
    \toggletrue{showonline}
    \printbasicversionfalse
    \cbcolor{red}
    \renewenvironment*{basicOnly}{\cbstart}{\cbend}
    \renewenvironment*{basicSkip}{\cbstart}{\cbend}
    \def\xmtoonprintopties{FULL}   % will be printed in footer
}
{}
      
%
% Evalueer \ifhints IN de environment
%  
%
%\RenewEnviron{hint}
%{
%\ifhandout
%\ifhints\else\setbox0\vbox\fi%everything in een emty box
%\bgroup 
%\stepcounter{hintLevel}
%\BODY
%\egroup\ignorespacesafterend
%\addtocounter{hintLevel}{-1}
%\else
%\ifhints
%\begin{trivlist}\item[\hskip \labelsep\small\slshape\bfseries Hint:\hspace{2ex}]
%\small\slshape
%\stepcounter{hintLevel}
%\BODY
%\end{trivlist}
%\addtocounter{hintLevel}{-1}
%\fi
%\fi
%}

% Onafhankelijk van \ifhandout ...? TO BE VERIFIED
\RenewEnviron{hint}
{
\ifhints
\begin{trivlist}\item[\hskip \labelsep\small\bfseries Hint:\hspace{2ex}]
\small%\slshape
\stepcounter{hintLevel}
\BODY
\end{trivlist}
\addtocounter{hintLevel}{-1}
\else
\iftikzexport   % anders worden de tikz tekeningen in hints niet gegenereerd ?
\setbox0\vbox\bgroup
\stepcounter{hintLevel}
\BODY
\egroup\ignorespacesafterend
\addtocounter{hintLevel}{-1}
\fi % ifhandout
\fi %ifhints
}

%
% \tab sets typewriter-tabs (e.g. to format questions)
% (Has no effect in HTML :-( ))
%
\usepackage{tabto}
\ifdefined\HCode
  \renewcommand{\tab}{\quad}    % otherwise dummy .png's are generated ...?
\fi


% Also redefined in  preamble to get correct styling 
% for tikz images for (\tikzexport)
%

\theoremstyle{definition} % Bold titels
\makeatletter
\let\proposition\relax
\let\c@proposition\relax
\let\endproposition\relax
\makeatother
\newtheorem{proposition}{Eigenschap}


%\instructornotesfalse

% logic with \ifhandoutin ximera.cls unclear;so overwrite ...
\makeatletter
\@ifundefined{ifinstructornotes}{%
  \newif\ifinstructornotes
  \instructornotesfalse
  \newenvironment{instructorNotes}{}{}
}{}
\makeatother
\ifinstructornotes
\else
\renewenvironment{instructorNotes}%
{%
    \setbox0\vbox\bgroup
}
{%
    \egroup
}
\fi

% \RedeclareMathOperator
% from https://tex.stackexchange.com/questions/175251/how-to-redefine-a-command-using-declaremathoperator
\makeatletter
\newcommand\RedeclareMathOperator{%
    \@ifstar{\def\rmo@s{m}\rmo@redeclare}{\def\rmo@s{o}\rmo@redeclare}%
}
% this is taken from \renew@command
\newcommand\rmo@redeclare[2]{%
    \begingroup \escapechar\m@ne\xdef\@gtempa{{\string#1}}\endgroup
    \expandafter\@ifundefined\@gtempa
    {\@latex@error{\noexpand#1undefined}\@ehc}%
    \relax
    \expandafter\rmo@declmathop\rmo@s{#1}{#2}}
% This is just \@declmathop without \@ifdefinable
\newcommand\rmo@declmathop[3]{%
    \DeclareRobustCommand{#2}{\qopname\newmcodes@#1{#3}}%
}
\@onlypreamble\RedeclareMathOperator
\makeatother


%
% Engelse vertaling, vooral in mathmode
%
% 1. Algemene procedure
%
\ifdefined\isEn
 \newcommand{\nlen}[2]{#2}
 \newcommand{\nlentext}[2]{\text{#2}}
 \newcommand{\nlentextbf}[2]{\textbf{#2}}
\else
 \newcommand{\nlen}[2]{#1}
 \newcommand{\nlentext}[2]{\text{#1}}
 \newcommand{\nlentextbf}[2]{\textbf{#1}}
\fi

%
% 2. Lijst van erg veel gebruikte uitdrukkingen
%

% Ja/Nee/Fout/Juits etc
%\newcommand{\TJa}{\nlentext{ Ja }{ and }}
%\newcommand{\TNee}{\nlentext{ Nee }{ No }}
%\newcommand{\TJuist}{\nlentext{ Juist }{ Correct }
%\newcommand{\TFout}{\nlentext{ Fout }{ Wrong }
\newcommand{\TWaar}{\nlentext{ Waar }{ True }}
\newcommand{\TOnwaar}{\nlentext{ Vals }{ False }}
% Korte bindwoorden en, of, dus, ...
\newcommand{\Ten}{\nlentext{ en }{ and }}
\newcommand{\Tof}{\nlentext{ of }{ or }}
\newcommand{\Tdus}{\nlentext{ dus }{ so }}
\newcommand{\Tendus}{\nlentext{ en dus }{ and thus }}
\newcommand{\Tvooralle}{\nlentext{ voor alle }{ for all }}
\newcommand{\Took}{\nlentext{ ook }{ also }}
\newcommand{\Tals}{\nlentext{ als }{ when }} %of if?
\newcommand{\Twant}{\nlentext{ want }{ as }}
\newcommand{\Tmaal}{\nlentext{ maal }{ times }}
\newcommand{\Toptellen}{\nlentext{ optellen }{ add }}
\newcommand{\Tde}{\nlentext{ de }{ the }}
\newcommand{\Thet}{\nlentext{ het }{ the }}
\newcommand{\Tis}{\nlentext{ is }{ is }} %zodat is in text staat in mathmode (geen italics)
\newcommand{\Tmet}{\nlentext{ met }{ where }} % in situaties e.g met p < n --> where p < n
\newcommand{\Tnooit}{\nlentext{ nooit }{ never }}
\newcommand{\Tmaar}{\nlentext{ maar }{ but }}
\newcommand{\Tniet}{\nlentext{ niet }{ not }}
\newcommand{\Tuit}{\nlentext{ uit }{ from }}
\newcommand{\Ttov}{\nlentext{ t.o.v. }{ w.r.t. }}
\newcommand{\Tzodat}{\nlentext{ zodat }{ such that }}
\newcommand{\Tdeth}{\nlentext{de }{th }}
\newcommand{\Tomdat}{\nlentext{omdat }{because }} 


%
% Overschrijf addhoc commando's
%
\ifdefined\isEn
\renewcommand{\pernot}{\overset{\mathrm{notation}}{=}}
\RedeclareMathOperator{\bld}{im}     % beeld
\RedeclareMathOperator{\graf}{graph}   % grafiek
\RedeclareMathOperator{\rico}{slope}   % richtingcoëfficient
\RedeclareMathOperator{\co}{co}       % coordinaat
\RedeclareMathOperator{\gr}{deg}       % graad

% Operators
\RedeclareMathOperator{\bgsin}{arcsin}
\RedeclareMathOperator{\bgcos}{arccos}
\RedeclareMathOperator{\bgtan}{arctan}
\RedeclareMathOperator{\bgcot}{arccot}
\RedeclareMathOperator{\bgsinh}{arcsinh}
\RedeclareMathOperator{\bgcosh}{arccosh}
\RedeclareMathOperator{\bgtanh}{arctanh}
\RedeclareMathOperator{\bgcoth}{arccoth}

\fi


% HACK: use 'oplossing' for 'explanation' ...
\let\explanation\relax
\let\endexplanation\relax
% \newenvironment{explanation}{\begin{oplossing}}{\end{oplossing}}
\newcounter{explanation}

\ifhandout%
    \NewEnviron{explanation}[1][toon]%
    {%
    \RenewEnviron{verbatim}{ \red{VERBATIM CONTENT MISSING IN THIS PDF}} %% \expandafter\verb|\BODY|}

    \ifthenelse{\equal{\detokenize{#1}}{\detokenize{toon}}}
    {
    \def\PH@Command{#1}% Use PH@Command to hold the content and be a target for "\expandafter" to expand once.

    \begin{trivlist}% Begin the trivlist to use formating of the "Feedback" label.
    \item[\hskip \labelsep\small\slshape\bfseries Explanation:% Format the "Feedback" label. Don't forget the space.
    %(\texttt{\detokenize\expandafter{\PH@Command}}):% Format (and detokenize) the condition for feedback to trigger
    \hspace{2ex}]\small%\slshape% Insert some space before the actual feedback given.
    \BODY
    \end{trivlist}
    }
    {  % \begin{feedback}[solution]   \BODY     \end{feedback}  }
    }
    }    
\else
% ONLY for HTML; xmoplossing is styled with css, and is not, and need not be a LaTeX environment
% THUS: it does NOT use feedback anymore ...
%    \NewEnviron{oplossing}{\begin{expandable}{xmoplossing}{\nlen{Toon uitwerking}{Show solution}}{\BODY}\end{expandable}}
    \newenvironment{explanation}[1][toon]
   {%
       \begin{expandable}{xmoplossing}{}
   }
   {%
   	   \end{expandable}
   } 
\fi

\title{Isomorphic Vector Spaces} \license{CC BY-NC-SA 4.0}

\begin{document}
\begin{abstract}
\end{abstract}
\maketitle

\section*{Isomorphic Vector Spaces}

A vector space is defined as a collection of objects together with operations of addition and scalar multiplication that follow certain rules (Definition \ref{def:vectorspacegeneral} of \href{https://ximera.osu.edu/oerlinalg/LinearAlgebra/VSP-0050/main}{Abstract Vector Spaces}).  In our study of abstract vector spaces, we have encountered spaces that appeared very different from each other.  Just how different are they?  Does $\mathbb{L}$, a vector space whose elements have the form $mx+b$, have anything in common with $\RR^2$?  Is $\mathbb{P}^3$ fundamentally different from $\mathbb{M}_{2,2}$?

To answer these questions, we will have to look beyond the superficial appearance of the elements of a vector space and delve into its structure.  The ``structure" of a vector space is determined by how the elements of the vector space interact with each other through the operations of addition and scalar multiplication.  

Let us return to the question of what $\mathbb{L}$ has in common with $\RR^2$.  Consider two typical elements of $\mathbb{L}$:
\begin{equation}\label{eq:iso1}
    mx+b\quad\text{and}\quad qx+c
\end{equation}
We can add these elements together
\begin{equation}\label{eq:iso2}
(mx+b)+(qx+c)=(m+q)x+(b+c)\end{equation}
or multiply each one by a scalar
\begin{equation}\label{eq:iso3}
k(mx+b)=kmx+kb\quad\text{and}\quad t(qx+c)=tqx+tc\end{equation}

But suppose we get tired of having to write  $x$ down every time.  Could we leave off the $x$ and represent $mx+b$ by  $\begin{bmatrix}m\\b\end{bmatrix}$?  If we do this, expressions (\ref{eq:iso1}), (\ref{eq:iso2}) and (\ref{eq:iso3}) would be mimicked by the following expressions involving vectors of $\RR^2$:
$$\begin{bmatrix}m\\b\end{bmatrix}\quad\text{and}\quad\begin{bmatrix}q\\c\end{bmatrix}$$
$$\begin{bmatrix}m\\b\end{bmatrix}+\begin{bmatrix}q\\c\end{bmatrix}=\begin{bmatrix}m+q\\b+c\end{bmatrix}$$
$$k\begin{bmatrix}m\\b\end{bmatrix}=\begin{bmatrix}km\\kb\end{bmatrix}\quad\text{and}\quad t\begin{bmatrix}q\\c\end{bmatrix}=\begin{bmatrix}tq\\tc\end{bmatrix}$$
It appears that we should be able to switch back and forth between $\mathbb{L}$ and $\RR^2$, translating questions and answers from one space to the other and back again.  

\begin{center}
 \begin{tikzpicture} 
   \fill[blue, opacity=0.3] (0,0) rectangle (5,5);
   \fill[pink, opacity=0.5] (6,0) rectangle (11,5);
   
   \node[] at (0.5, 4.5)  (p2)    {$\mathbb{L}$};
   \node[] at (10.5, 4.5)  (r3)    {$\RR^2$};
   
   \node[] at (2, 3)  (p_1)    {$(mx+b)+(qx+c)$};
   \node[] at (2, 1)  (p_4)    {$(m+q)x+(b+c)$};
   \node[] at (2, 2)  (eq1)    {{\rotatebox{90}{$=$}}};
   \node[] at (8.5, 2)  (eq2)    {{\rotatebox{90}{$=$}}};
     
     \node[] at (8.5, 3)  (v_1)    {$\begin{bmatrix}m\\b\end{bmatrix}+\begin{bmatrix}q\\c\end{bmatrix}$};
  \node[] at (8.5, 1.2)  (v_4)    {$\begin{bmatrix}m+q\\b+c\end{bmatrix}$};
     
     \draw [->,line width=0.5pt,-stealth]  (3.6, 3)to[out=10, in=170](7.4,3);
     \draw [->,line width=0.5pt,-stealth]  (p_4.east)to[out=10, in=170](v_4.west);
     
     \draw [->,line width=0.5pt,-stealth]  (7.4,2.9)to[out=190, in=350](3.6, 2.9);
     \draw [->,line width=0.5pt,-stealth]  (7.7,1.1)to[out=190, in=350](3.6,0.9 );
     
    % \draw [->,line width=0.5pt,-stealth]  (8.5,-0.1)to[out=200, in=340](2.5,-0.1 );
     %\draw [->,line width=0.5pt,-stealth]  (2.5,5.1)to[out=20, in=160](8.5, 5.1);
     
  \end{tikzpicture}
\end{center}


We begin to suspect that $\mathbb{L}$ and $\RR^2$ have the same ``structure".  Spaces such as $\mathbb{L}$ and $\RR^2$ are said to be \dfn{isomorphic}.  This term is derived from the Greek ``iso," meaning ``same," and ``morphe," meaning ``form."  The term captures the idea that isomorphic vector spaces have the same structure.    Before we present a precise definition of the term, we need to better understand what we mean by  ``switching back and forth" between spaces.  The following Exploration will help us formulate this vague notion in terms of transformations.

\begin{exploration}\label{init:isomorph}
Recall that the set of all polynomials of degree $2$ or less, together with polynomial addition and scalar multiplication, is a vector space, denoted by $\mathbb{P}^2$.  Let $\mathcal{B}=\{1, x, x^2\}$. You should do a quick mental check that $\mathcal{B}$ is a basis of $\mathbb{P}^2$.

Define a transformation $T:\mathbb{P}^2\rightarrow \RR^3$ by
$T(a+bx+cx^2)=\begin{bmatrix}a\\b\\c\end{bmatrix}$.  You may have recognized $T$ as the transformation that maps each element of $\mathbb{P}^2$ to its coordinate vector with respect to $\mathcal{B}$.  

Transformation $T$ has several nice properties:
  \begin{enumerate}
      \item By Theorem \ref{th:coordvectmappinglinear} of \href{https://ximera.osu.edu/oerlinalg/LinearAlgebra/LTR-0022/main}{Linear Transformations of Abstract Vector Spaces}, $T$ is linear.
      \item It is easy to verify that $T$ is one-to-one and onto. (See Practice Problem \ref{prob:Tonetooneonto}.)
      \item By Theorem \ref{th:isomeansinvert} of \href{https://ximera.osu.edu/oerlinalg/LinearAlgebra/LTR-0035/main}{Existence of the Inverse of a Linear Transformation}, $T$ has an inverse.
  \end{enumerate}
  
  Our goal is to investigate and illustrate what these properties mean for transformation $T$, and for the relationship between $\mathbb{P}^2$ and $\RR^3$.  
  
  First, observe that $T$ being one-to-one and onto establishes ``pairings" between elements of $\mathbb{P}^2$ and $\RR^3$ in such a way that every element of one vector space is uniquely matched with exactly one element of the other vector space, as shown in the diagram below.
  
 \begin{center}
 \begin{tikzpicture} 
   \fill[blue, opacity=0.3] (0,0) rectangle (5,5);
   \fill[pink, opacity=0.5] (6,0) rectangle (11,5);
   
   \node[] at (0.5, 4.5)  (p2)    {$\mathbb{P}^2$};
   \node[] at (10.5, 4.5)  (r3)    {$\RR^3$};
   
   \node[] at (2.5, 4.2)  (p_1)    {$1+2x-3x^2$};
   \node[] at (1.2, 2.8)  (p_2)    {$-4$};
    \node[] at (3, 1.8)  (p_3)    {$x+5$};
     \node[] at (2, 0.6)  (p_4)    {$a+bx+cx^2$};
     
     \node[] at (8.5, 4.2)  (v_1)    {$\begin{bmatrix}1\\2\\-3\end{bmatrix}$};
   \node[] at (7.2, 3.2)  (v_2)    {$\begin{bmatrix}-4\\0\\0\end{bmatrix}$};
    \node[] at (9, 1.9)  (v_3)    {$\begin{bmatrix}5\\1\\0\end{bmatrix}$};
     \node[] at (7.5, 0.8)  (v_4)    {$\begin{bmatrix}a\\b\\c\end{bmatrix}$};
     
     \draw [->,line width=0.5pt,-stealth]  (3.6, 4.2)to[out=10, in=170](7.9,4.2);
     \draw [->,line width=0.5pt,-stealth]  (p_2.east)to[out=10, in=170](v_2.west);
     \draw [->,line width=0.5pt,-stealth]  (p_3.east)to[out=10, in=170](v_3.west);
     \draw [->,line width=0.5pt,-stealth]  (p_4.east)to[out=10, in=170](v_4.west);
     
     \draw [->,line width=0.5pt,-stealth]  (7.9,4.1)to[out=190, in=350](3.6, 4.1);
     \draw [->,line width=0.5pt,-stealth]  (6.5,3.1)to[out=190, in=350](1.6,2.7 );
     \draw [->,line width=0.5pt,-stealth]  (8.5,1.8)to[out=190, in=350](3.6,1.7 );
     \draw [->,line width=0.5pt,-stealth]  (7,0.7)to[out=190, in=350](3,0.5 );
     
     \draw [->,line width=0.5pt,-stealth]  (8.5,-0.1)to[out=200, in=340](2.5,-0.1 );
     \draw [->,line width=0.5pt,-stealth]  (2.5,5.1)to[out=20, in=160](8.5, 5.1);
     
     \node[] at (5.5, 6)    {$T$};
      \node[] at (5.5, -1)    {$T^{-1}$};
  \end{tikzpicture}
\end{center}

  Second, the fact that $T$ (and $T^{-1}$) are linear will allow us to translate questions related to linear combinations in one of the vector spaces to equivalent questions in the other vector space, then translate answers back to the original vector space.  To make this statement concrete, consider the following problem:
  
  Let $$p_1(x)=3-x+2x^2\quad\text{and}\quad p_2(x)=-1+3x+x^2$$ find $p_1(x)+p_2(x)$.  
  
  The answer is, of course
  $$p_1(x)+p_2(x)=2+2x+3x^2$$
  Easy.  But suppose for a moment that we did not know how to add polynomials, or that we found the process extremely difficult, or maybe instead of $\mathbb{P}^2$ we had another vector space that we did not want to deal with. 
  
  It turns out that we can use $T$ and $T^{-1}$ to answer the addition question.  We will start by applying $T$ to $p_1(x)$ and $p_2(x)$ separately:
  $$T(p_1(x))=\begin{bmatrix}3\\-1\\2\end{bmatrix},\quad T(p_2(x))=\begin{bmatrix}-1\\3\\1\end{bmatrix}$$
  Next, we add the images of $p_1(x)$ and $p_2(x)$ in $\RR^3$
  $$\begin{bmatrix}3\\-1\\2\end{bmatrix}+\begin{bmatrix}-1\\3\\1\end{bmatrix}=\begin{bmatrix}2\\2\\3\end{bmatrix}$$
  This maneuver allows us to avoid the addition question in $\mathbb{P}^2$ and answer the question in $\RR^3$ instead.  We use $T^{-1}$ to translate the answer back to $\mathbb{P}^2$:
  $$T^{-1}\left(\begin{bmatrix}2\\2\\3\end{bmatrix}\right)=2+2x+3x^2$$
  All of this relies on linearity.  Here is a formal justification for the process.  Try to spot where linearity is used.
  \begin{align}
      p_1(x)+p_2(x)&=(3-x+2x^2)+(-1+3x+x^2)\label{steplin1}\\
      &=T^{-1}\left(\begin{bmatrix}3\\-1\\2\end{bmatrix}\right)+T^{-1}\left(\begin{bmatrix}-1\\3\\1\end{bmatrix}\right)\label{steplin2}\\
      &=T^{-1}\left(\begin{bmatrix}3\\-1\\2\end{bmatrix}+\begin{bmatrix}-1\\3\\1\end{bmatrix}\right)\label{steplin3}\\
      &=T^{-1}\left(\begin{bmatrix}2\\2\\3\end{bmatrix}\right)\label{steplin4}\\
      &=2+2x+3x^2\label{steplin5}
  \end{align}
  Which transition requires linearity?
  \begin{multipleChoice}
  \choice{From expresion (\ref{steplin1}) to expresion (\ref{steplin2})}
  \choice[correct]{From expresion (\ref{steplin2}) to expresion (\ref{steplin3})}
  \choice{From expresion (\ref{steplin3}) to expresion (\ref{steplin4})}
  \choice{From expresion (\ref{steplin4}) to expresion (\ref{steplin5})}
    \end{multipleChoice}
\end{exploration}
Invertible linear transformations, such as transformation $T$ of Exploration \ref{init:isomorph}, are useful because they preserve the structure of interactions between elements as we move back and forth between two vector spaces, allowing us to answer questions about one vector space in a different vector space.  In particular, any question related to linear combinations can be addressed in this fashion. This includes questions concerning linear independence, span, basis and dimension. 

\begin{definition}\label{def:isomorphism} Let $V$ and $W$ be vector spaces.  If there exists an invertible linear transformation $T:V\rightarrow W$ we say that $V$ and $W$ are \dfn{isomorphic} and write $V\cong W$.  The invertible linear transformation $T$ is called an \dfn{isomorphism}.
\end{definition}

It is worth pointing out that if $T:V\rightarrow W$ is an isomorphism, then $T^{-1}:W\rightarrow V$, being linear and invertible, is also an isomorphism.

\begin{example}\label{ex:lisor2}
Our earlier discussion suggests that $\mathbb{L}\cong\RR^2$.  We postpone the proof until Theorem \ref{ex:coordmapiso}.
\end{example}

\begin{example}\label{ex:p2isor3}
Exploration \ref{init:isomorph} shows that $\mathbb{P}^2\cong\RR^3$.  
\end{example}



\begin{example}\label{ex:isomorphexample1}
Show that $\mathbb{M}_{2,2}$ and $\mathbb{P}^3$ are isomorphic.
\begin{explanation}
We will start by finding a plausible candidate for an isomorphism.  Define $T:\mathbb{M}_{2,2}\rightarrow \mathbb{P}^3$ by
$$T\left(\begin{bmatrix}a&b\\c&d\end{bmatrix}\right)=a+bx+cx^2+dx^3$$

We will first show that $T$ is a linear transformation,  then verify that $T$ is invertible.
\begin{align*}
    T\left(k\begin{bmatrix}a&b\\c&d\end{bmatrix}\right)&=T\left(\begin{bmatrix}ka&kb\\kc&kd\end{bmatrix}\right)\\
    &=ka+kbx+kcx^2+kdx^3=k(a+bx+cx^2+dx^3)\\
    &=kT\left(\begin{bmatrix}a&b\\c&d\end{bmatrix}\right)
\end{align*}

\begin{align*}
    T\left(\begin{bmatrix}a&b\\c&d\end{bmatrix}+\begin{bmatrix}m&n\\p&q\end{bmatrix}\right)&=T\left(\begin{bmatrix}a+m&b+n\\c+p&d+q\end{bmatrix}\right)\\
    &=(a+m)+(b+n)x+(c+p)x^2+(d+q)x^3\\
    &=(a+bx+cx^2+dx^3)+(m+nx+px^2+qx^3)\\
    &=T\left(\begin{bmatrix}a&b\\c&d\end{bmatrix}\right)+T\left(\begin{bmatrix}m&n\\p&q\end{bmatrix}\right)
\end{align*}
We can show that $T$ is one-to-one and onto, and therefore has an inverse.  We can also observe directly that $T^{-1}$ is given by 
$$T^{-1}(a+bx+cx^2+dx^3)=\begin{bmatrix}a&b\\c&d\end{bmatrix}$$
We conclude that $T$ is an isomorphism, and $\mathbb{M}_{2,2}\cong\mathbb{P}^3$.
\end{explanation}
\end{example}
Isomorphism $T$ in Example \ref{ex:isomorphexample1} establishes the fact that $\mathbb{M}_{2,2}\cong\mathbb{P}^3$. However, there is nothing special about $T$, as there are many other isomorphisms from $\mathbb{M}_{2,2}$ to $\mathbb{P}^3$.  Just for fun, try to verify that each of the following is an isomorphism.

\begin{equation}\label{eq:justforfuniso1}\tau_1:\mathbb{M}_{2,2}\rightarrow\mathbb{P}^3\end{equation}
$$\tau_1\left(\begin{bmatrix}a&b\\c&d\end{bmatrix}\right)=a-bx-cx^2+dx^3$$

\begin{equation}\label{eq:justforfuniso2}\tau_2:\mathbb{M}_{2,2}\rightarrow\mathbb{P}^3\end{equation}
$$\tau_2\left(\begin{bmatrix}a&b\\c&d\end{bmatrix}\right)=a+(a+c)x+(b-c)x^2+dx^3$$

\subsubsection*{The Coordinate Vector Isomorphism}
In Exploration \ref{init:isomorph} we made good use of a transformation that maps every element of $\mathbb{P}^2$ to its coordinate vector in $\RR^3$.  We observed that this transformation is linear and invertible, therefore it is an isomorphism.  The following example generalizes this result.

\begin{theorem}\label{ex:coordmapiso}
Let $V$ be an $n$-dimensional vector space, and let $\mathcal{B}$ be a basis for $V$.  Then  $T:V\rightarrow \RR^n$ given by $T(\vec{v})=[\vec{v}]_{\mathcal{B}}$ is an isomorphism.
\end{theorem}
\begin{proof}
We leave the proof of this result to the reader.  (See Practice Problem \ref{prob:verifyisomorphism}.)
\end{proof}

\subsection*{Properties of Isomorphic Vector Spaces and Isomorphisms}  In this section we will illustrate properties of isomorphisms with specific examples.  Formal proofs of properties will be presented in the next section.

\begin{exploration}\label{init:basestobasesiso}
In Exploration \ref{init:isomorph} we defined a transformation $T:\mathbb{P}^2\rightarrow \RR^3$ by
$T(a+bx+cx^2)=\begin{bmatrix}a\\b\\c\end{bmatrix}$. We later observed that  $T$ is an isomorphism.  We will now examine the effect of $T$ on two different basis of $\mathbb{P}^2$.

Let 
$\mathcal{B}_1=\{1, x, x^2\}$ and $\mathcal{B}_2=\{x, 1+x, x+x^2\}$.  (Recall that $\mathcal{B}_2$ is a basis of $\mathbb{P}^2$ by Example \ref{ex:coordvectorinpolyvectspace2} of \href{https://ximera.osu.edu/oerlinalg/LinearAlgebra/VSP-0060/main}{Bases and Dimension of Abstract Vector Spaces}.)

First,
$$T(\mathcal{B}_1)=\{T(1), T(x), T(x^2)\}=\left\{\begin{bmatrix}1\\0\\0\end{bmatrix}, \begin{bmatrix}0\\1\\0\end{bmatrix}, \begin{bmatrix}0\\0\\1\end{bmatrix}\right\}$$
Clearly, the images of the elements of $\mathcal{B}_1$ form a basis of $\RR^3$.

Now we consider $\mathcal{B}_2$.
$$T(\mathcal{B}_2)=\{T(x), T(1+x), T(x+x^2)\}=\left\{\begin{bmatrix}0\\1\\0\end{bmatrix}, \begin{bmatrix}1\\1\\0\end{bmatrix}, \begin{bmatrix}0\\1\\1\end{bmatrix}\right\}$$
It is easy to verify that $\begin{bmatrix}0\\1\\0\end{bmatrix}, \begin{bmatrix}1\\1\\0\end{bmatrix}, \begin{bmatrix}0\\1\\1\end{bmatrix}$ are linearly independent and span $\RR^3$, therefore the images of the elements of $\mathcal{B}_2$ from a basis of $\RR^3$.

\end{exploration}

We can try any number of bases of $\mathbb{P}^2$ and we will find that the image of each basis of $\mathbb{P}^2$ is a basis of $\RR^3$.  In general, we have the following result:
\begin{fact}
An isomorphism maps a basis of the domain to a basis of the codomain. (We will state this result more formally as Theorem \ref{th:bijectionsbasis} in the next section.)
\end{fact}

Isomorphisms preserve bases, but more generally, they preserve linear independence.  
\begin{fact}
If $T:V\rightarrow W$ is an isomorphism, then the subset $\{\vec{v_1}, \vec{v}_2,\ldots ,\vec{v}_n\}$ of $V$ is linearly independent if and only if $\{T(\vec{v_1}), T(\vec{v}_2),\ldots ,T(\vec{v}_n)\}$ is linearly independent in $W$.  (We will state and prove this result as Theorem \ref{th:linindtolinindiso}.)
\end{fact}

\begin{example}\label{ex:inverseimageoflinind}
Let $V$ be a vector space, and let $\mathcal{B}=\{\vec{v}_1, \vec{v}_2, \vec{v}_3, \vec{v}_4\}$ be a basis of $V$.  Let 
$$\vec{w}_1=2\vec{v}_1+3\vec{v}_2-5\vec{v}_3-2\vec{v}_4$$
$$\vec{w}_2=-\vec{v}_1+4\vec{v}_2-3\vec{v}_3+\vec{v}_4$$
$$\vec{w}_3=-3\vec{v}_1-\vec{v}_2+4\vec{v}_3+3\vec{v}_4$$

Are $\vec{w}_1, \vec{w}_2, \vec{w}_3$ linearly independent?
\begin{explanation}
We could approach this question head-on by considering the vector equation
$$a\vec{w}_1+b\vec{w}_2+c\vec{w}_3=\vec{0}$$
to see if the only solution is the trivial one. (See Practice Problem \ref{prob:noiso}.)

Instead, we will use isomorphisms.  Observe that we do not know anything about $V$ aside from the fact that it has four basis vectors.  Vectors $\vec{w}_1$, $\vec{w}_2$, $\vec{w}_3$ are given in terms of these basis vectors.  This should give us an idea for constructing an isomorphism between $V$ and $\RR^4$.  Consider $T:V\rightarrow\RR^4$ such that $T(\vec{w})=[\vec{w}]_{\mathcal{B}}$.
Then $$T(\vec{w}_1)=[\vec{w}_1]_{\mathcal{B}}=\begin{bmatrix}2\\3\\-5\\-2\end{bmatrix},\quad T(\vec{w}_2)=[\vec{w}_2]_{\mathcal{B}}=\begin{bmatrix}-1\\4\\-3\\1\end{bmatrix},\quad T(\vec{w}_3)=[\vec{w}_3]_{\mathcal{B}}=\begin{bmatrix}-3\\-1\\4\\3\end{bmatrix}$$
By Theorem \ref{ex:coordmapiso}, $T$   is an isomorphism.  This means that $\vec{w}_1$, $\vec{w}_2$, $\vec{w}_3$ are linearly independent if and only if their coordinate vectors are linearly independent.  There are multiple ways of determining whether 
$$\begin{bmatrix}2\\3\\-5\\-2\end{bmatrix}, \begin{bmatrix}-1\\4\\-3\\1\end{bmatrix}, \begin{bmatrix}-3\\-1\\4\\3\end{bmatrix}$$
are linearly independent.  One way is to find the reduced row echelon form of 
$$\begin{bmatrix}2&-1&-3\\3&4&-1\\-5&-3&4\\-2&1&3\end{bmatrix}$$
The matrix reduces as follows:
$$\begin{bmatrix}2&-1&-3\\3&4&-1\\-5&-3&4\\-2&1&3\end{bmatrix}\rightsquigarrow\begin{bmatrix}1&0&-13/11\\0&1&7/11\\0&0&0\\0&0&0\end{bmatrix}$$
We see that the rank of the matrix is $2$.  We conclude that the column vectors are not linearly independent.   Thus, the vectors $\vec{w}_1$, $\vec{w}_2$ and $\vec{w}_3$ are not linearly independent.
\end{explanation}
\end{example}



\subsection*{Proofs of Isomorphism Properties}

Recall that a transformation $T$ is \dfn{one-to-one} provided that $$T(\vec{v}_1)=T(\vec{v}_2)$$ implies that $$\vec{v}_1=\vec{v}_2$$

 We will show that images of linearly independent vectors under one-to-one linear transformations are linearly independent.  

\begin{theorem}\label{th:onetoonelinind} Let $T:V\rightarrow W$ be a one-to-one linear transformation.  Suppose $\{\vec{v}_1,\ldots,\vec{v}_n\}$ is linearly independent in $V$.  Then $\{T(\vec{v}_1),\ldots,T(\vec{v}_n)\}$ is linearly independent in $W$.
\end{theorem}

\begin{proof}
Suppose $a_1, \ldots, a_n$ satisfy
\begin{align}\label{onlysolution}a_1T(\vec{v}_1)+\ldots +a_nT(\vec{v}_n)=\vec{0}\end{align}
We will show that for each $i$, we must have $a_i=0$.

By linearity, we have:
\begin{align*}\vec{0}&=a_1T(\vec{v}_1)+\ldots +a_nT(\vec{v}_n)\\
&=T(a_1\vec{v}_1)+\ldots +T(a_n\vec{v}_n)\\
&=T(a_1\vec{v}_1+\ldots +a_n\vec{v}_n)
\end{align*}
By Theorem \ref{th:zerotozero}, $T(\vec{0})=\vec{0}$.  Therefore,
$$T(a_1\vec{v}_1+\ldots +a_n\vec{v}_n)=T(\vec{0})$$

Because $T$ is one-to-one, we conclude that 
\begin{align}\label{onlytrivial}a_1\vec{v}_1+\ldots +a_n\vec{v}_n=\vec{0}\end{align}


By assumption, $\{\vec{v}_1,\ldots,\vec{v}_n\}$ is linearly independent.  Therefore $a_i=0$ for $1\leq i\leq n$.  
\end{proof}

Recall that a transformation $T$ is \dfn{onto} provided that every vector of the codomain of $T$ is the image of some vector in the domain of $T$.  

We will show that an onto linear transformation maps sets that span the domain to sets that span the codomain. 

\begin{theorem}\label{th:ontospan}
Let $T:V\rightarrow W$ be an onto linear transformation.  Suppose $V=\mbox{span}(\vec{v}_1,\ldots ,\vec{v}_n)$.  Then $W=\mbox{span}(T(\vec{v}_1),\ldots ,T(\vec{v}_n))$.
\end{theorem}
\begin{proof}
Suppose $\vec{w}$ is an element of $W$. To show that $\{T(\vec{v}_1),\ldots ,T(\vec{v}_n)\}$ spans $W$, we will express $\vec{w}$ as a linear combination of $T(\vec{v}_1),\ldots ,T(\vec{v}_n)$.

Because $T$ is onto, $\vec{w}=T(\vec{v})$ for some $\vec{v}$ in $V$.  But $V=\mbox{span}(\vec{v}_1,\ldots ,\vec{v}_n)$.  Therefore, $\vec{v}=a_1\vec{v}_1+\ldots +a_n\vec{v}_n$ for some scalar coefficients $a_1,\ldots ,a_n$.
By linearity, we have:
\begin{align*}
\vec{w}=T(\vec{v})&=T(a_1\vec{v}_1+\ldots +a_n\vec{v}_n)\\
&=a_1T(\vec{v}_1)+\ldots +a_nT(\vec{v}_n)
\end{align*}
Thus, $\vec{w}$ is in the span of $T(\vec{v}_1),\ldots ,T(\vec{v}_n)$.
\end{proof}
 
We will now combine the results of Theorem \ref{th:onetoonelinind} and Theorem \ref{th:ontospan} to obtain a result about the effect of isomorphisms on a basis.

\begin{theorem}\label{th:bijectionsbasis}
Let $T:V\rightarrow W$ be an isomorphism.  Suppose $\mathcal{B}_V=\{\vec{v}_1,\ldots ,\vec{v}_n\}$ is a basis for $V$.  Then $\{T(\vec{v}_1),\ldots ,T(\vec{v}_n)\}$ is a basis for $W$.
\end{theorem}
\begin{proof}
Left to the reader.  (See Practice Problem \ref{prob:bijectionsbasisproof}) 
\end{proof}

\begin{theorem}\label{th:linindtolinindiso}
Suppose $T:V\rightarrow W$ is an isomorphism, then the subset $\{\vec{v_1}, \vec{v}_2,\ldots ,\vec{v}_n\}$ of $V$ is linearly independent if and only if $\{T(\vec{v_1}), T(\vec{v}_2),\ldots ,T(\vec{v}_n)\}$ is linearly independent in $W$.
\end{theorem} 
\begin{proof}
We have already proved one direction of this this ``if and only if" statement as Theorem \ref{th:onetoonelinind}.  To prove the other direction, suppose that $T(\vec{v_1}), T(\vec{v}_2),\ldots ,T(\vec{v}_n)$ are linearly independent vectors in $W$.  We need to show that this implies that $\vec{v_1}, \vec{v}_2,\ldots ,\vec{v}_n$ are linearly independent in $V$.  Observe that if $T$ is an isomorphism, then $T^{-1}:W\rightarrow V$ is also an isomorphism.  Thus, by Theorem \ref{th:onetoonelinind}, $T^{-1}(T(\vec{v_1})), T^{-1}(T(\vec{v}_2)),\ldots ,T^{-1}(T(\vec{v}_n))$ are linearly independent.  But this means that $\vec{v_1}, \vec{v}_2,\ldots ,\vec{v}_n$ are linearly independent.
\end{proof}

\begin{theorem}\label{th:isocompisiso}
Let $U$, $V$ and $W$ be vector spaces.  Suppose that $T_1:U\rightarrow V$ and $T_2:V\rightarrow W$ are isomorphisms.  Then $T_2\circ T_1:U\rightarrow W$ is an isomorphism.
\end{theorem}
\begin{proof}
The proof is left to the reader.  (See Practice Problem \ref{prob:isocompisisoproof}.)
\end{proof}
 
%%%% Moved to Exercises %%%%%%%%%%%%%% 
%\begin{theorem} A linear transformation $T:V\rightarrow W$ is one-to-one if and only if $\text{ker}(T)=\{\vec{0}\}$.
%\end{theorem}
%\begin{proof}
%First assume that $T$ is one-to-one.  Suppose $\vec{v}$ is in $\text{ker}(T)$.  Then $T(\vec{v})=\vec{0}=T(\vec{0})$.  But then $\vec{v}=\vec{0}$.

%Next, assume that $\text{ker}(T)=\{\vec{0}\}$.  To show that $T$ is one-to-one, suppose $T(\vec{v}_1)=T(\vec{v}_2)$, but then
%\begin{align*}
%T(\vec{v}_1)-T(\vec{v}_2)&=\vec{0}\\
%T(\vec{v}_1-\vec{v}_2)&=\vec{0}
%\end{align*}
%Since the kernel only contains the zero vector, we conclude that
%$$\vec{v}_1-\vec{v}_2=\vec{0}$$
%Therefore
%$$\vec{v}_1=\vec{v}_2$$
%\end{proof}
 
\subsection{Finite-dimensional Vector Spaces}

\begin{theorem}\label{th:ndimspacesisorn}
Let $V$ and $W$ be finite-dimensional vector spaces. Then
$$V\cong W\quad\text{if and only if}\quad \mbox{dim}(V)=\mbox{dim}(W)$$
\end{theorem}
\begin{proof}
First, assume that $V\cong W$.  Then there exists an isomorphism $T:V\rightarrow W$.  Suppose $\mbox{dim}(V)=n$ and let $\{\vec{v}_1,\vec{v}_2,\ldots ,\vec{v}_n\}$ be a basis for $V$. By Theorem \ref{th:bijectionsbasis} $\{T(\vec{v}_1),\ldots ,T(\vec{v}_n)\}$ is a basis for $W$. Therefore $\mbox{dim}(W)=n$.

Conversely, suppose $\mbox{dim}(V)=\mbox{dim}(W)=n$, and let $\mathcal{B}=\{\vec{v}_1,\vec{v}_2,\ldots ,\vec{v}_n\}$, $\mathcal{C}=\{\vec{w}_1,\vec{w}_2,\ldots ,\vec{w}_n\}$ be bases for $V$ and $W$, respectively.

Define a linear transformation $T:V\rightarrow W$ by $T(\vec{v}_i)=\vec{w}_i$ for $1\leq i\leq n$.  To show that $T$ is an isomorphism, we need to prove that $T$ is one-to-one and onto.

Suppose $T(\vec{u}_1)=T(\vec{u}_2)$ for some vectors $\vec{u}_1$, $\vec{u}_2$ in $V$.  We know that
$$\vec{u}_1=a_1\vec{v}_1+\ldots +a_n\vec{v}_n$$
$$\vec{u}_2=b_1\vec{v}_1+\ldots +b_n\vec{v}_n$$
for some scalars $a_i$'s and $b_i$'s.  Thus,
$$T(a_1\vec{v}_1+\ldots +a_n\vec{v}_n)=T(b_1\vec{v}_1+\ldots +b_n\vec{v}_n)$$
By linearity of $T$,
$$a_1\vec{w}_1+\ldots +a_n\vec{w}_n=b_1\vec{w}_1+\ldots +b_n\vec{w}_n$$
$$(a_1-b_1)\vec{w}_1+\ldots +(a_n-b_n)\vec{w}_n=\vec{0}$$
But $\vec{w}_1,\vec{w}_2,\ldots ,\vec{w}_n$ are linearly independent, so $a_i-b_i=0$ for all $1\leq i\leq n$.  Therefore $a_i=b_i$ for all $1\leq i\leq n$.  We conclude that $\vec{u}_1=\vec{u}_2$.

We now show that $T$ is onto. Suppose that $\vec{w}$ is an element of $W$.  Then $\vec{w}=c_1\vec{w}_1+\ldots +c_n\vec{w}_n$ for some scalars $c_i$'s.  But then
$$\vec{w}=c_1T(\vec{v}_1)+\ldots +c_nT(\vec{v}_n)=T(c_1\vec{v}_1+\ldots +c_n\vec{v}_n)$$
We conclude that $\vec{w}$ is an image of an element of $V$, so $T$ is onto.
\end{proof}

From this theorem follows an important corollary that shows why we spent so much time trying to understand $\RR^n$ in this course.

\begin{corollary}\label{cor:ndimisotorn}
Every $n$-dimensional vector space is isomorphic to $\RR^n$.
\end{corollary}

\begin{example}\label{ex:planeisoplane}
The span of any two linearly independent vectors in $\RR^3$ is isomorphic to $\RR^2$. \begin{center}
\tdplotsetmaincoords{70}{130}
\begin{tikzpicture}[scale=0.8]
	\draw[->](-2,0,0)--(3,0,0) ;
    \draw[->](0,-2,0)--(0,3,0) ;
    \draw[->](0,0,-2)--(0,0,5) ;
     \filldraw[blue, opacity=0.3] (2,3,0)--(0,2,2.5)--(-2,-3,0)--(0,-2,-2.5)--cycle;
     
    
    \draw[->, line width=2pt,red, -stealth](0,0,0)--(2,3,0);
     \draw[->, line width=2pt,blue, -stealth](0,0,0)--(0,2,2.5);
    
\end{tikzpicture}\quad\quad
\begin{tikzpicture}[scale=0.5]

  \draw[<->] (-4,0)--(4,0);
  \draw[<->] (0,-4)--(0,4);
   \filldraw[blue, opacity=0.3] (-3,-3)--(-3,3)--(3,3)--(3,-3)--cycle;
 \end{tikzpicture}
\end{center}

\end{example}

\begin{example}\label{ex:p2isor3b}
$\mathbb{P}^2\ncong \RR^2$ 
\begin{explanation}
Recall that $\mbox{dim}(\mathbb{P}^2)=3$.  Since $\mbox{dim}(\mathbb{P}^2)=3\neq 2=\mbox{dim}(\RR^2)$, we conclude that $\mathbb{P}^2$ is not isomorphic to $\RR^2$.
\end{explanation}
\end{example}

\section*{Practice Problems}

\begin{problem}\label{prob:Tonetooneonto}
Prove that transformation $T$ of Exploration \ref{init:isomorph} is one-to-one and onto.
\end{problem}

\begin{problem}\label{prob:tauone}
Verify that $$\tau_1:\mathbb{M}_{2,2}\rightarrow\mathbb{P}^3$$
given by
$$\tau_1\left(\begin{bmatrix}a&b\\c&d\end{bmatrix}\right)=a-bx-cx^2+dx^3$$
of Expression \ref{eq:justforfuniso1} is an isomorphism.
\end{problem}

\begin{problem}\label{prob:noiso}
Do Example \ref{ex:inverseimageoflinind} without using isomorphisms.  
\end{problem}

\begin{problem}\label{prob:useisoshowlinind}
Let 
$$\mathcal{S}=\left\{\begin{bmatrix}1&-3\\-2&2\end{bmatrix}, \begin{bmatrix}4&-2\\1&5\end{bmatrix}, \begin{bmatrix}5&5\\8&4\end{bmatrix}\right\}$$
Is $\mathcal{S}$ linearly independent in $\mathbb{M}_{2,2}$?  \wordChoice{\choice[correct]{Yes}\choice{No}}
\end{problem}

\begin{problem}\label{prob:basism22iso}
Let 
$$\mathcal{S}=\left\{\begin{bmatrix}1&2\\3&4\end{bmatrix}, \begin{bmatrix}5&6\\7&8\end{bmatrix}, \begin{bmatrix}9&10\\11&12\end{bmatrix}, \begin{bmatrix}13&14\\15&16\end{bmatrix}\right\}$$
Is $\mathcal{S}$ a basis for $\mathbb{M}_{2,2}$? \wordChoice{\choice{Yes}\choice[correct]{No}}
\end{problem}

\begin{problem}\label{prob:bijectionsbasisproof}
Prove Theorem \ref{th:bijectionsbasis}.
\end{problem}

\begin{problem}\label{prob:chooseisospace}
Let $V$ be a vector space, and suppose $\mathcal{B}=\{\vec{v}_1, \vec{v}_2, \vec{v}_3, \vec{v}_4, \vec{v}_5\}$ is a basis for $V$.  What can we conclude about $V$?  Check ALL that apply.
\begin{selectAll}
    \choice{We cannot conclude anything about $V$ because we don't know what $V$ is.}
    \choice[correct]{$V\cong \RR^5$}
    \choice{$V\cong \mathbb{P}^5$}
    \choice{$V\cong \RR^4$}
    \choice[correct]{$V\cong \mathbb{P}^4$}
    \choice{$V\cong \mathbb{M}_{5,5}$}
  \end{selectAll}
\end{problem}

\begin{problem}\label{prob:pickisospaces} Which of the followng statements are true? Check ALL that apply.
\begin{selectAll}
    \choice{$\mathbb{M}_{3,3}\cong \RR^6$}
    \choice[correct]{$\mathbb{M}_{3,3}\cong \RR^9$}
    \choice{$\mathbb{M}_{4,4}\cong \mathbb{P}^{16}$}
    \choice[correct]{$\mathbb{M}_{4,4}\cong \mathbb{P}^{15}$}
    \choice[correct]{$\mathbb{L}\cong \RR^2$}
    \end{selectAll}
\end{problem}

\begin{problem}\label{prob:verifyisomorphism}
Verify that $T:V\rightarrow \RR^n$ of Theorem \ref{ex:coordmapiso} is an isomorphism.
\begin{hint}
You may find the proof of Theorem \ref{th:ndimspacesisorn} helpful.
\end{hint}
\end{problem}

\begin{problem}\label{prob:isocompisisoproof}
Prove that the composition of two isomorphisms is an isomorphism.  (Theorem \ref{th:isocompisiso}.)
\end{problem}

\begin{problem} \label{prob:kerneliszero}
Prove that a linear transformation $T:V\rightarrow W$ is one-to-one if and only if $\text{ker}(T)=\{\vec{0}\}$.
\end{problem}


\end{document}
