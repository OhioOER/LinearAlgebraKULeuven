\documentclass{ximera}
%%% Begin Laad packages

\makeatletter
\@ifclassloaded{xourse}{%
    \typeout{Start loading preamble.tex (in a XOURSE)}%
    \def\isXourse{true}   % automatically defined; pre 112022 it had to be set 'manually' in a xourse
}{%
    \typeout{Start loading preamble.tex (NOT in a XOURSE)}%
}
\makeatother

\def\isEn\true 

\pgfplotsset{compat=1.16}

\usepackage{currfile}

% 201908/202301: PAS OP: babel en doclicense lijken problemen te veroorzaken in .jax bestand
% (wegens syntax error met toegevoegde \newcommands ...)
\pdfOnly{
    \usepackage[type={CC},modifier={by-nc-sa},version={4.0}]{doclicense}
    %\usepackage[hyperxmp=false,type={CC},modifier={by-nc-sa},version={4.0}]{doclicense}
    %%% \usepackage[dutch]{babel}
}



\usepackage[utf8]{inputenc}
\usepackage{morewrites}   % nav zomercursus (answer...?)
\usepackage{multirow}
\usepackage{multicol}
\usepackage{tikzsymbols}
\usepackage{ifthen}
%\usepackage{animate} BREAKS HTML STRUCTURE USED BY XIMERA
\usepackage{relsize}

\usepackage{eurosym}    % \euro  (€ werkt niet in xake ...?)
\usepackage{fontawesome} % smileys etc

% Nuttig als ook interactieve beamer slides worden voorzien:
\providecommand{\p}{} % default nothing ; potentially usefull for slides: redefine as \pause
%providecommand{\p}{\pause}

    % Layout-parameters voor het onderschrift bij figuren
\usepackage[margin=10pt,font=small,labelfont=bf, labelsep=endash,format=hang]{caption}
%\usepackage{caption} % captionof
%\usepackage{pdflscape}    % landscape environment

% Met "\newcommand\showtodonotes{}" kan je todonotes tonen (in pdf/online)
% 201908: online werkt het niet (goed)
\providecommand\showtodonotes{disable}
\providecommand\todo[1]{\typeout{TODO #1}}
%\usepackage[\showtodonotes]{todonotes}
%\usepackage{todonotes}

%
% Poging tot aanpassen layout
%
\usepackage{tcolorbox}
\tcbuselibrary{theorems}

%%% Einde laad packages

%%% Begin Ximera specifieke zaken

\graphicspath{
	{../../}
	{../}
	{./}
  	{../../pictures/}
   	{../pictures/}
   	{./pictures/}
	{./explog/}    % M05 in groeimodellen       
}

%%% Einde Ximera specifieke zaken

%
% define softer blue/red/green, use KU Leuven base colors for blue (and dark orange for red ?)
%
% todo: rather redefine blue/red/green ...?
%\definecolor{xmblue}{rgb}{0.01, 0.31, 0.59}
%\definecolor{xmred}{rgb}{0.89, 0.02, 0.17}
\definecolor{xmdarkblue}{rgb}{0.122, 0.671, 0.835}   % KU Leuven Blauw
\definecolor{xmblue}{rgb}{0.114, 0.553, 0.69}        % KU Leuven Blauw
\definecolor{xmgreen}{rgb}{0.13, 0.55, 0.13}         % No KULeuven variant for green found ...

\definecolor{xmaccent}{rgb}{0.867, 0.541, 0.18}      % KU Leuven Accent (orange ...)
\definecolor{kuaccent}{rgb}{0.867, 0.541, 0.18}      % KU Leuven Accent (orange ...)

\colorlet{xmred}{xmaccent!50!black}                  % Darker version of KU Leuven Accent

\providecommand{\blue}[1]{{\color{blue}#1}}    
\providecommand{\red}[1]{{\color{red}#1}}

\renewcommand\CancelColor{\color{xmaccent!50!black}}

% werkt in math en text mode om MATH met oranje (of grijze...)  achtergond te tonen (ook \important{\text{blabla}} lijkt te werken)
%\newcommand{\important}[1]{\ensuremath{\colorbox{xmaccent!50!white}{$#1$}}}   % werkt niet in Mathjax
%\newcommand{\important}[1]{\ensuremath{\colorbox{lightgray}{$#1$}}}
\newcommand{\important}[1]{\ensuremath{\colorbox{orange}{$#1$}}}   % TODO: kleur aanpassen voor mathjax; wordt overschreven infra!


% Uitzonderlijk kan met \pdfnl in de PDF een newline worden geforceerd, die online niet nodig/nuttig is omdat daar de regellengte hoe dan ook niet gekend is.
\ifdefined\HCode%
\providecommand{\pdfnl}{}%
\else%
\providecommand{\pdfnl}{%
  \\%
}%
\fi

% Uitzonderlijk kan met \handoutnl in de handout-PDF een newline worden geforceerd, die noch online noch in de PDF-met-antwoorden nuttig is.
\ifdefined\HCode
\providecommand{\handoutnl}{}
\else
\providecommand{\handoutnl}{%
\ifhandout%
  \nl%
\fi%
}
\fi



% \cellcolor IGNORED by tex4ht ?
% \begin{center} seems not to wordk
    % (missing margin-left: auto;   on tabular-inside-center ???)
%\newcommand{\importantcell}[1]{\ensuremath{\cellcolor{lightgray}#1}}  %  in tabular; usablility to be checked
\providecommand{\importantcell}[1]{\ensuremath{#1}}     % no mathjax2 support for colloring array cells

\pdfOnly{
  \renewcommand{\important}[1]{\ensuremath{\colorbox{kuaccent!50!white}{$#1$}}}
  \renewcommand{\importantcell}[1]{\ensuremath{\cellcolor{kuaccent!40!white}#1}}   
}

%%% Tikz styles


\pgfplotsset{compat=1.16}

\usetikzlibrary{trees,positioning,arrows,fit,shapes,math,calc,decorations.markings,through,intersections,patterns,matrix}

\usetikzlibrary{decorations.pathreplacing,backgrounds}    % 5/2023: from experimental


\usetikzlibrary{angles,quotes}

\usepgfplotslibrary{fillbetween} % bepaalde_integraal
\usepgfplotslibrary{polar}    % oa voor poolcoordinaten.tex

\pgfplotsset{ownstyle/.style={axis lines = center, axis equal image, xlabel = $x$, ylabel = $y$, enlargelimits}} 

\pgfplotsset{
	plot/.style={no marks,samples=50}
}

\newcommand{\xmPlotsColor}{
	\pgfplotsset{
		plot1/.style={darkgray,no marks,samples=100},
		plot2/.style={lightgray,no marks,samples=100},
		plotresult/.style={blue,no marks,samples=100},
		plotblue/.style={blue,no marks,samples=100},
		plotred/.style={red,no marks,samples=100},
		plotgreen/.style={green,no marks,samples=100},
		plotpurple/.style={purple,no marks,samples=100}
	}
}
\newcommand{\xmPlotsBlackWhite}{
	\pgfplotsset{
		plot1/.style={black,loosely dashed,no marks,samples=100},
		plot2/.style={black,loosely dotted,no marks,samples=100},
		plotresult/.style={black,no marks,samples=100},
		plotblue/.style={black,no marks,samples=100},
		plotred/.style={black,dotted,no marks,samples=100},
		plotgreen/.style={black,dashed,no marks,samples=100},
		plotpurple/.style={black,dashdotted,no marks,samples=100}
	}
}


\newcommand{\xmPlotsColorAndStyle}{
	\pgfplotsset{
		plot1/.style={darkgray,no marks,samples=100},
		plot2/.style={lightgray,no marks,samples=100},
		plotresult/.style={blue,no marks,samples=100},
		plotblue/.style={xmblue,no marks,samples=100},
		plotred/.style={xmred,dashed,thick,no marks,samples=100},
		plotgreen/.style={xmgreen,dotted,very thick,no marks,samples=100},
		plotpurple/.style={purple,no marks,samples=100}
	}
}


%\iftikzexport
\xmPlotsColorAndStyle
%\else
%\xmPlotsBlackWhite
%\fi
%%%


%
% Om venndiagrammen te arceren ...
%
\makeatletter
\pgfdeclarepatternformonly[\hatchdistance,\hatchthickness]{north east hatch}% name
{\pgfqpoint{-1pt}{-1pt}}% below left
{\pgfqpoint{\hatchdistance}{\hatchdistance}}% above right
{\pgfpoint{\hatchdistance-1pt}{\hatchdistance-1pt}}%
{
	\pgfsetcolor{\tikz@pattern@color}
	\pgfsetlinewidth{\hatchthickness}
	\pgfpathmoveto{\pgfqpoint{0pt}{0pt}}
	\pgfpathlineto{\pgfqpoint{\hatchdistance}{\hatchdistance}}
	\pgfusepath{stroke}
}
\pgfdeclarepatternformonly[\hatchdistance,\hatchthickness]{north west hatch}% name
{\pgfqpoint{-\hatchthickness}{-\hatchthickness}}% below left
{\pgfqpoint{\hatchdistance+\hatchthickness}{\hatchdistance+\hatchthickness}}% above right
{\pgfpoint{\hatchdistance}{\hatchdistance}}%
{
	\pgfsetcolor{\tikz@pattern@color}
	\pgfsetlinewidth{\hatchthickness}
	\pgfpathmoveto{\pgfqpoint{\hatchdistance+\hatchthickness}{-\hatchthickness}}
	\pgfpathlineto{\pgfqpoint{-\hatchthickness}{\hatchdistance+\hatchthickness}}
	\pgfusepath{stroke}
}
%\makeatother

\tikzset{
    hatch distance/.store in=\hatchdistance,
    hatch distance=10pt,
    hatch thickness/.store in=\hatchthickness,
   	hatch thickness=2pt
}

\colorlet{circle edge}{black}
\colorlet{circle area}{blue!20}


\tikzset{
    filled/.style={fill=green!30, draw=circle edge, thick},
    arceerl/.style={pattern=north east hatch, pattern color=blue!50, draw=circle edge},
    arceerr/.style={pattern=north west hatch, pattern color=yellow!50, draw=circle edge},
    outline/.style={draw=circle edge, thick}
}




%%% Updaten commando's
\def\hoofding #1#2#3{\maketitle}     % OBSOLETE ??

% we willen (bijna) altijd \geqslant ipv \geq ...!
\newcommand{\geqnoslant}{\geq}
\renewcommand{\geq}{\geqslant}
\newcommand{\leqnoslant}{\leq}
\renewcommand{\leq}{\leqslant}

% Todo: (201908) waarom komt er (soms) underlined voor emph ...?
\renewcommand{\emph}[1]{\textit{#1}}

% API commando's

\newcommand{\ds}{\displaystyle}
\newcommand{\ts}{\textstyle}  % tegenhanger van \ds   (Ximera zet PER  DEFAULT \ds!)

% uit Zomercursus-macro's: 
\newcommand{\bron}[1]{\begin{scriptsize} \emph{#1} \end{scriptsize}}     % deprecated ...?


%definities nieuwe commando's - afkortingen veel gebruikte symbolen
\newcommand{\R}{\ensuremath{\mathbb{R}}}
\newcommand{\Rnul}{\ensuremath{\mathbb{R}_0}}
\newcommand{\Reen}{\ensuremath{\mathbb{R}\setminus\{1\}}}
\newcommand{\Rnuleen}{\ensuremath{\mathbb{R}\setminus\{0,1\}}}
\newcommand{\Rplus}{\ensuremath{\mathbb{R}^+}}
\newcommand{\Rmin}{\ensuremath{\mathbb{R}^-}}
\newcommand{\Rnulplus}{\ensuremath{\mathbb{R}_0^+}}
\newcommand{\Rnulmin}{\ensuremath{\mathbb{R}_0^-}}
\newcommand{\Rnuleenplus}{\ensuremath{\mathbb{R}^+\setminus\{0,1\}}}
\newcommand{\N}{\ensuremath{\mathbb{N}}}
\newcommand{\Nnul}{\ensuremath{\mathbb{N}_0}}
\newcommand{\Z}{\ensuremath{\mathbb{Z}}}
\newcommand{\Znul}{\ensuremath{\mathbb{Z}_0}}
\newcommand{\Zplus}{\ensuremath{\mathbb{Z}^+}}
\newcommand{\Zmin}{\ensuremath{\mathbb{Z}^-}}
\newcommand{\Znulplus}{\ensuremath{\mathbb{Z}_0^+}}
\newcommand{\Znulmin}{\ensuremath{\mathbb{Z}_0^-}}
\newcommand{\C}{\ensuremath{\mathbb{C}}}
\newcommand{\Cnul}{\ensuremath{\mathbb{C}_0}}
\newcommand{\Cplus}{\ensuremath{\mathbb{C}^+}}
\newcommand{\Cmin}{\ensuremath{\mathbb{C}^-}}
\newcommand{\Cnulplus}{\ensuremath{\mathbb{C}_0^+}}
\newcommand{\Cnulmin}{\ensuremath{\mathbb{C}_0^-}}
\newcommand{\Q}{\ensuremath{\mathbb{Q}}}
\newcommand{\Qnul}{\ensuremath{\mathbb{Q}_0}}
\newcommand{\Qplus}{\ensuremath{\mathbb{Q}^+}}
\newcommand{\Qmin}{\ensuremath{\mathbb{Q}^-}}
\newcommand{\Qnulplus}{\ensuremath{\mathbb{Q}_0^+}}
\newcommand{\Qnulmin}{\ensuremath{\mathbb{Q}_0^-}}

\newcommand{\perdef}{\overset{\mathrm{def}}{=}}
\newcommand{\pernot}{\overset{\mathrm{notatie}}{=}}
\newcommand\perinderdaad{\overset{!}{=}}     % voorlopig gebruikt in limietenrekenregels
\newcommand\perhaps{\overset{?}{=}}          % voorlopig gebruikt in limietenrekenregels

\newcommand{\degree}{^\circ}


\DeclareMathOperator{\dom}{dom}     % domein
\DeclareMathOperator{\codom}{codom} % codomein
\DeclareMathOperator{\bld}{bld}     % beeld
\DeclareMathOperator{\graf}{graf}   % grafiek
\DeclareMathOperator{\rico}{rico}   % richtingcoëfficient
\DeclareMathOperator{\co}{co}       % coordinaat
\DeclareMathOperator{\gr}{gr}       % graad

\newcommand{\func}[5]{\ensuremath{#1: #2 \rightarrow #3: #4 \mapsto #5}} % Easy to write a function


% Operators
\DeclareMathOperator{\bgsin}{bgsin}
\DeclareMathOperator{\bgcos}{bgcos}
\DeclareMathOperator{\bgtan}{bgtan}
\DeclareMathOperator{\bgcot}{bgcot}
\DeclareMathOperator{\bgsinh}{bgsinh}
\DeclareMathOperator{\bgcosh}{bgcosh}
\DeclareMathOperator{\bgtanh}{bgtanh}
\DeclareMathOperator{\bgcoth}{bgcoth}

% Oude \Bgsin etc deprecated: gebruik \bgsin, en herdefinieer dat als je Bgsin wil!
%\DeclareMathOperator{\cosec}{cosec}    % not used? gebruik \csc en herdefinieer

% operatoren voor differentialen: to be verified; 1/2020: inconsequent gebruik bij afgeleiden/integralen
\renewcommand{\d}{\mathrm{d}}
\newcommand{\dx}{\d x}
\newcommand{\dd}[1]{\frac{\mathrm{d}}{\mathrm{d}#1}}
\newcommand{\ddx}{\dd{x}}

% om in voorbeelden/oefeningen de notatie voor afgeleiden te kunnen kiezen
% Usage: \afg{(2\sin(x))}  (en wordt d/dx, of accent, of D )
%\newcommand{\afg}[1]{{#1}'}
\newcommand{\afg}[1]{\left(#1\right)'}
%\renewcommand{\afg}[1]{\frac{\mathrm{d}#1}{\mathrm{d}x}}   % include in relevant exercises ...
%\renewcommand{\afg}[1]{D{#1}}

%
% \xmxxx commands: Extra KU Leuven functionaliteit van, boven of naast Ximera
%   ( Conventie 8/2019: xm+nederlandse omschrijving, maar is niet consequent gevolgd, en misschien ook niet erg handig !)
%
% (Met een minimale ximera.cls en preamble.tex zou een bruikbare .pdf moeten kunnen worden gemaakt van eender welke ximera)
%
% Usage: \xmtitle[Mijn korte abstract]{Mijn titel}{Mijn abstract}
% Eerste command na \begin{document}:
%  -> definieert de \title
%  -> definieert de abstract
%  -> doet \maketitle ( dus: print de hoofding als 'chapter' of 'sectie')
% Optionele parameter geeft eenn kort abstract (die met de globale setting \xmshortabstract{} al dan niet kan worden geprint.
% De optionele korte abstract kan worden gebruikt voor pseudo-grappige abtsarts, dus dus globaal al dan niet kunnen worden gebuikt...
% Globale settings:
%  de (optionele) 'korte abstract' wordt enkele getoond als \xmshortabstract is gezet
\providecommand\xmshortabstract{} % default: print (only!) short abstract if present
\newcommand{\xmtitle}[3][]{
	\title{#2}
	\begin{abstract}
		\ifdefined\xmshortabstract
		\ifstrempty{#1}{%
			#3
		}{%
			#1
		}%
		\else
		#3
		\fi
	\end{abstract}
	\maketitle
}

% 
% Kleine grapjes: moeten zonder verder gevolg kunnen worden verwijderd
%
%\newcommand{\xmopje}[1]{{\small#1{\reversemarginpar\marginpar{\Smiley}}}}   % probleem in floats!!
\newtoggle{showxmopje}
\toggletrue{showxmopje}

\newcommand{\xmopje}[1]{%
   \iftoggle{showxmopje}{#1}{}%
}


% -> geef een abstracte-formule-met-rechts-een-concreet-voorbeeld
% VB:  \formulevb{a^2+b^2=c^2}{3^2+4^2=5^2}
%
\ifdefined\HCode
\NewEnviron{xmdiv}[1]{\HCode{\Hnewline<div class="#1">\Hnewline}\BODY{\HCode{\Hnewline</div>\Hnewline}}}
\else
\NewEnviron{xmdiv}[1]{\BODY}
\fi

\providecommand{\formulevb}[2]{
	{\centering

    \begin{xmdiv}{xmformulevb}    % zie css voor online layout !!!
	\begin{tabular}{lcl}
		\important{#1}
		&  &
		Vb: $#2$
		\end{tabular}
	\end{xmdiv}

	}
}

\ifdefined\HCode
\providecommand{\vb}[1]{%
    \HCode{\Hnewline<span class="xmvb">}#1\HCode{</span>\Hnewline}%
}
\else
\providecommand{\vb}[1]{
    \colorbox{blue!10}{#1}
}
\fi

\ifdefined\HCode
\providecommand{\xmcolorbox}[2]{
	\HCode{\Hnewline<div class="xmcolorbox">\Hnewline}#2\HCode{\Hnewline</div>\Hnewline}
}
\else
\providecommand{\xmcolorbox}[2]{
  \cellcolor{#1}#2
}
\fi


\ifdefined\HCode
\providecommand{\xmopmerking}[1]{
 \HCode{\Hnewline<div class="xmopmerking">\Hnewline}#1\HCode{\Hnewline</div>\Hnewline}
}
\else
\providecommand{\xmopmerking}[1]{
	{\footnotesize #1}
}
\fi
% \providecommand{\voorbeeld}[1]{
% 	\colorbox{blue!10}{$#1$}
% }



% Hernoem Proof naar Bewijs, nodig voor HTML versie
\renewcommand*{\proofname}{Bewijs}

% Om opgave van oefening (wordt niet geprint bij oplossingenblad)
% (to be tested test)
\NewEnviron{statement}{\BODY}

% Environment 'oplossing' en 'uitkomst'
% voor resp. volledige 'uitwerking' dan wel 'enkel eindresultaat'
% geimplementeerd via feedback, omdat er in de ximera-server adhoc feedback-code is toegevoegd
%% Niet tonen indien handout
%% Te gebruiken om volledige oplossingen/uitwerkingen van oefeningen te tonen
%% \begin{oplossing}        De optelling is commutatief \end{oplossing}  : verschijnt online enkel 'op vraag'
%% \begin{oplossing}[toon]  De optelling is commutatief \end{oplossing}  : verschijnt steeds onmiddellijk online (bv te gebruiken bij voorbeelden) 

\ifhandout%
    \NewEnviron{oplossing}[1][onzichtbaar]%
    {%
    \ifthenelse{\equal{\detokenize{#1}}{\detokenize{toon}}}
    {
    \def\PH@Command{#1}% Use PH@Command to hold the content and be a target for "\expandafter" to expand once.

    \begin{trivlist}% Begin the trivlist to use formating of the "Feedback" label.
    \item[\hskip \labelsep\small\slshape\bfseries Oplossing% Format the "Feedback" label. Don't forget the space.
    %(\texttt{\detokenize\expandafter{\PH@Command}}):% Format (and detokenize) the condition for feedback to trigger
    \hspace{2ex}]\small%\slshape% Insert some space before the actual feedback given.
    \BODY
    \end{trivlist}
    }
    {  % \begin{feedback}[solution]   \BODY     \end{feedback}  }
    }
    }    
\else
% ONLY for HTML; xmoplossing is styled with css, and is not, and need not be a LaTeX environment
% THUS: it does NOT use feedback anymore ...
%    \NewEnviron{oplossing}{\begin{expandable}{xmoplossing}{\nlen{Toon uitwerking}{Show solution}}{\BODY}\end{expandable}}
    \newenvironment{oplossing}[1][onzichtbaar]
   {%
       \begin{expandable}{xmoplossing}{}
   }
   {%
   	   \end{expandable}
   } 
%     \newenvironment{oplossing}[1][onzichtbaar]
%    {%
%        \begin{feedback}[solution]   	
%    }
%    {%
%    	   \end{feedback}
%    } 
\fi

\ifhandout%
    \NewEnviron{uitkomst}[1][onzichtbaar]%
    {%
    \ifthenelse{\equal{\detokenize{#1}}{\detokenize{toon}}}
    {
    \def\PH@Command{#1}% Use PH@Command to hold the content and be a target for "\expandafter" to expand once.

    \begin{trivlist}% Begin the trivlist to use formating of the "Feedback" label.
    \item[\hskip \labelsep\small\slshape\bfseries Uitkomst:% Format the "Feedback" label. Don't forget the space.
    %(\texttt{\detokenize\expandafter{\PH@Command}}):% Format (and detokenize) the condition for feedback to trigger
    \hspace{2ex}]\small%\slshape% Insert some space before the actual feedback given.
    \BODY
    \end{trivlist}
    }
    {  % \begin{feedback}[solution]   \BODY     \end{feedback}  }
    }
    }    
\else
\ifdefined\HCode
   \newenvironment{uitkomst}[1][onzichtbaar]
    {%
        \begin{expandable}{xmuitkomst}{}%
    }
    {%
    	\end{expandable}%
    } 
\else
  % Do NOT print 'uitkomst' in non-handout
  %  (presumably, there is also an 'oplossing' ??)
  \newenvironment{uitkomst}[1][onzichtbaar]{}{}
\fi
\fi

%
% Uitweidingen zijn extra's die niet redelijkerwijze tot de leerstof behoren
% Uitbreidingen zijn extra's die wel redelijkerwijze tot de leerstof van bv meer geavanceerde versies kunnen behoren (B-programma/Wiskundestudenten/...?)
% Nog niet voorzien: design voor verschillende versies (A/B programma, BIO, voorkennis/ ...)
% Voor 'uitweidingen' is er een environment die online per default is ingeklapt, en in pdf al dan niet kan worden geincluded  (via \xmnouitweiding) 
%
% in een xourse, per default GEEN uitweidingen, tenzij \xmuitweiding expliciet ergens is gezet ...
\ifdefined\isXourse
   \ifdefined\xmuitweiding
   \else
       \def\xmnouitweiding{true}
   \fi
\fi

\ifdefined\xmnouitweiding
\newcounter{xmuitweiding}  % anders error undefined ...  
\excludecomment{xmuitweiding}
\else
\newtheoremstyle{dotless}{}{}{}{}{}{}{ }{}
\theoremstyle{dotless}
\newtheorem*{xmuitweidingnofrills}{}   % nofrills = no accordion; gebruikt dus de dotless theoremstyle!

\newcounter{xmuitweiding}
\newenvironment{xmuitweiding}[1][ ]%
{% 
	\refstepcounter{xmuitweiding}%
    \begin{expandable}{xmuitweiding}{\nlentext{Uitweiding \arabic{xmuitweiding}: #1}{Digression \arabic{xmuitweiding}: #1}}%
	\begin{xmuitweidingnofrills}%
}
{%
    \end{xmuitweidingnofrills}%
    \end{expandable}%
}   
% \newenvironment{xmuitweiding}[1][ ]%
% {% 
% 	\refstepcounter{xmuitweiding}
% 	\begin{accordion}\begin{accordion-item}[Uitweiding \arabic{xmuitweiding}: #1]%
% 			\begin{xmuitweidingnofrills}%
% 			}
% 			{\end{xmuitweidingnofrills}\end{accordion-item}\end{accordion}}   
\fi


\newenvironment{xmexpandable}[1][]{
	\begin{accordion}\begin{accordion-item}[#1]%
		}{\end{accordion-item}\end{accordion}}


% Command that gives a selection box online, but just prints the right answer in pdf
\newcommand{\xmonlineChoice}[1]{\pdfOnly{\wordchoicegiventrue}\wordChoice{#1}\pdfOnly{\wordchoicegivenfalse}}   % deprecated, gebruik onlineChoice ...
\newcommand{\onlineChoice}[1]{\pdfOnly{\wordchoicegiventrue}\wordChoice{#1}\pdfOnly{\wordchoicegivenfalse}}

\newcommand{\TJa}{\nlentext{ Ja }{ Yes }}
\newcommand{\TNee}{\nlentext{ Nee }{ No }}
\newcommand{\TJuist}{\nlentext{ Juist }{ True }}
\newcommand{\TFout}{\nlentext{ Fout }{ False }}

\newcommand{\choiceTrue }{{\renewcommand{\choiceminimumhorizontalsize}{4em}\wordChoice{\choice[correct]{\TJuist}\choice{\TFout}}}}
\newcommand{\choiceFalse}{{\renewcommand{\choiceminimumhorizontalsize}{4em}\wordChoice{\choice{\TJuist}\choice[correct]{\TFout}}}}

\newcommand{\choiceYes}{{\renewcommand{\choiceminimumhorizontalsize}{3em}\wordChoice{\choice[correct]{\TJa}\choice{\TNee}}}}
\newcommand{\choiceNo }{{\renewcommand{\choiceminimumhorizontalsize}{3em}\wordChoice{\choice{\TJa}\choice[correct]{\TNee}}}}

% Optional nicer formatting for wordChoice in PDF

\let\inlinechoiceorig\inlinechoice

%\makeatletter
%\providecommand{\choiceminimumverticalsize}{\vphantom{$\frac{\sqrt{2}}{2}$}}   % minimum height of boxes (cfr infra)
\providecommand{\choiceminimumverticalsize}{\vphantom{$\tfrac{2}{2}$}}   % minimum height of boxes (cfr infra)
\providecommand{\choiceminimumhorizontalsize}{1em}   % minimum width of boxes (cfr infra)

\newcommand{\inlinechoicesquares}[2][]{%
		\setkeys{choice}{#1}%
		\ifthenelse{\boolean{\choice@correct}}%
		{%
            \ifhandout%
               \fbox{\choiceminimumverticalsize #2}\allowbreak\ignorespaces%
            \else%
               \fcolorbox{blue}{blue!20}{\choiceminimumverticalsize #2}\allowbreak\ignorespaces\setkeys{choice}{correct=false}\ignorespaces%
            \fi%
		}%
		{% else
			\fbox{\choiceminimumverticalsize #2}\allowbreak\ignorespaces%  HACK: wat kleiner, zodat fits on line ... 	
		}%
}

\newcommand{\inlinechoicesquareX}[2][]{%
		\setkeys{choice}{#1}%
		\ifthenelse{\boolean{\choice@correct}}%
		{%
            \ifhandout%
               \framebox[\ifdim\choiceminimumhorizontalsize<\width\width\else\choiceminimumhorizontalsize\fi]{\choiceminimumverticalsize\ #2\ }\allowbreak\ignorespaces\setkeys{choice}{correct=false}\ignorespaces%
            \else%
               \fcolorbox{blue}{blue!20}{\makebox[\ifdim\choiceminimumhorizontalsize<\width\width\else\choiceminimumhorizontalsize\fi]{\choiceminimumverticalsize #2}}\allowbreak\ignorespaces\setkeys{choice}{correct=false}\ignorespaces%
            \fi%
		}%
		{% else
        \ifhandout%
			\framebox[\ifdim\choiceminimumhorizontalsize<\width\width\else\choiceminimumhorizontalsize\fi]{\choiceminimumverticalsize\ #2\ }\allowbreak\ignorespaces%  HACK: wat kleiner, zodat fits on line ... 	
        \fi
		}%
}


\newcommand{\inlinechoicepuntjes}[2][]{%
		\setkeys{choice}{#1}%
		\ifthenelse{\boolean{\choice@correct}}%
		{%
            \ifhandout%
               \dots\ldots\ignorespaces\setkeys{choice}{correct=false}\ignorespaces
            \else%
               \fcolorbox{blue}{blue!20}{\choiceminimumverticalsize #2}\allowbreak\ignorespaces\setkeys{choice}{correct=false}\ignorespaces%
            \fi%
		}%
		{% else
			%\fbox{\choiceminimumverticalsize #2}\allowbreak\ignorespaces%  HACK: wat kleiner, zodat fits on line ... 	
		}%
}

% print niets, maar definieer globale variable \myanswer
%  (gebruikt om oplossingsbladen te printen) 
\newcommand{\inlinechoicedefanswer}[2][]{%
		\setkeys{choice}{#1}%
		\ifthenelse{\boolean{\choice@correct}}%
		{%
               \gdef\myanswer{#2}\setkeys{choice}{correct=false}

		}%
		{% else
			%\fbox{\choiceminimumverticalsize #2}\allowbreak\ignorespaces%  HACK: wat kleiner, zodat fits on line ... 	
		}%
}



%\makeatother

\newcommand{\setchoicedefanswer}{
\ifdefined\HCode
\else
%    \renewenvironment{multipleChoice@}[1][]{}{} % remove trailing ')'
    \let\inlinechoice\inlinechoicedefanswer
\fi
}

\newcommand{\setchoicepuntjes}{
\ifdefined\HCode
\else
    \renewenvironment{multipleChoice@}[1][]{}{} % remove trailing ')'
    \let\inlinechoice\inlinechoicepuntjes
\fi
}
\newcommand{\setchoicesquares}{
\ifdefined\HCode
\else
    \renewenvironment{multipleChoice@}[1][]{}{} % remove trailing ')'
    \let\inlinechoice\inlinechoicesquares
\fi
}
%
\newcommand{\setchoicesquareX}{
\ifdefined\HCode
\else
    \renewenvironment{multipleChoice@}[1][]{}{} % remove trailing ')'
    \let\inlinechoice\inlinechoicesquareX
\fi
}
%
\newcommand{\setchoicelist}{
\ifdefined\HCode
\else
    \renewenvironment{multipleChoice@}[1][]{}{)}% re-add trailing ')'
    \let\inlinechoice\inlinechoiceorig
\fi
}

\setchoicesquareX  % by default list-of-squares with onlineChoice in PDF

% Omdat multicols niet werkt in html: enkel in pdf  (in html zijn langere pagina's misschien ook minder storend)
\newenvironment{xmmulticols}[1][2]{
 \pdfOnly{\begin{multicols}{#1}}%
}{ \pdfOnly{\end{multicols}}}

%
% Te gebruiken in plaats van \section\subsection
%  (in een printstyle kan dan het level worden aangepast
%    naargelang \chapter vs \section style )
% 3/2021: DO NOT USE \xmsubsection !
\newcommand\xmsection\subsection
\newcommand\xmsubsection\subsubsection

% Aanpassen printversie
%  (hier gedefinieerd, zodat ze in xourse kunnen worden gezet/overschreven)
\providebool{parttoc}
\providebool{printpartfrontpage}
\providebool{printactivitytitle}
\providebool{printactivityqrcode}
\providebool{printactivityurl}
\providebool{printcontinuouspagenumbers}
\providebool{numberactivitiesbysubpart}
\providebool{addtitlenumber}
\providebool{addsectiontitlenumber}
\addtitlenumbertrue
\addsectiontitlenumbertrue

% The following three commands are hardcoded in xake, you can't create other commands like these, without adding them to xake as well
%  ( gebruikt in xourses om juiste soort titelpagina te krijgen voor verschillende ximera's )
\newcommand{\activitychapter}[2][]{
    {    
    \ifstrequal{#1}{notnumbered}{
        \addtitlenumberfalse
    }{}
    \typeout{ACTIVITYCHAPTER #2}   % logging
	\chapterstyle
	\activity{#2}
    }
}
\newcommand{\activitysection}[2][]{
    {
    \ifstrequal{#1}{notnumbered}{
        \addsectiontitlenumberfalse
    }{}
	\typeout{ACTIVITYSECTION #2}   % logging
	\sectionstyle
	\activity{#2}
    }
}
% Practices worden als activity getoond om de grote blokken te krijgen online
\newcommand{\practicesection}[2][]{
    {
    \ifstrequal{#1}{notnumbered}{
        \addsectiontitlenumberfalse
    }{}
    \typeout{PRACTICESECTION #2}   % logging
	\sectionstyle
	\activity{#2}
    }
}
\newcommand{\activitychapterlink}[3][]{
    {
    \ifstrequal{#1}{notnumbered}{
        \addtitlenumberfalse
    }{}
    \typeout{ACTIVITYCHAPTERLINK #3}   % logging
	\chapterstyle
	\activitylink[#1]{#2}{#3}
    }
}

\newcommand{\activitysectionlink}[3][]{
    {
    \ifstrequal{#1}{notnumbered}{
        \addsectiontitlenumberfalse
    }{}
    \typeout{ACTIVITYSECTIONLINK #3}   % logging
	\sectionstyle
	\activitylink[#1]{#2}{#3}
    }
}


% Commando om de printstyle toe te voegen in ximera's. Zorgt ervoor dat er geen problemen zijn als je de xourses compileert
% hack om onhandige relative paden in TeX te omzeilen
% should work both in xourse and ximera (pre-112022 only in ximera; thus obsoletes adhoc setup in xourses)
% loads global.sty if present (cfr global.css for online settings!)
% use global.sty to overwrite settings in printstyle.sty ...
\newcommand{\addPrintStyle}[1]{
\iftikzexport\else   % only in PDF
  \makeatletter
  \ifx\@onlypreamble\@notprerr\else   % ONLY if in tex-preamble   (and e.g. not when included from xourse)
    \typeout{Loading printstyle}   % logging
    \usepackage{#1/printstyle} % mag enkel geinclude worden als je die apart compileert
    \IfFileExists{#1/global.sty}{
        \typeout{Loading printstyle-folder #1/global.sty}   % logging
        \usepackage{#1/global}
        }{
        \typeout{Info: No extra #1/global.sty}   % logging
    }   % load global.sty if present
    \IfFileExists{global.sty}{
        \typeout{Loading local-folder global.sty (or TEXINPUTPATH..)}   % logging
        \usepackage{global}
    }{
        \typeout{Info: No folder/global.sty}   % logging
    }   % load global.sty if present
    \IfFileExists{\currfilebase.sty}
    {
        \typeout{Loading \currfilebase.sty}
        \input{\currfilebase.sty}
    }{
        \typeout{Info: No local \currfilebase.sty}
    }
    \fi
  \makeatother
\fi
}

%
%  
% references: Ximera heeft adhoc logica	 om online labels te doen werken over verschillende files heen
% met \hyperref kan de getoonde tekst toch worden opgegeven, in plaats van af te hangen van de label-text
\ifdefined\HCode
% Link to standard \labels, but give your own description
% Usage:  Volg \hyperref[my_very_verbose_label]{deze link} voor wat tijdverlies
%   (01/2020: Ximera-server aangepast om bij class reference-keeptext de link-text NIET te vervangen door de label-text !!!) 
\renewcommand{\hyperref}[2][]{\HCode{<a class="reference reference-keeptext" href="\##1">}#2\HCode{</a>}}
%
%  Link to specific targets  (not tested ?)
\renewcommand{\hypertarget}[1]{\HCode{<a class="ximera-label" id="#1"></a>}}
\renewcommand{\hyperlink}[2]{\HCode{<a class="reference reference-keeptext" href="\##1">}#2\HCode{</a>}}
\fi

% Mmm, quid English ... (for keyword #1 !) ?
\newcommand{\wikilink}[2]{
    \hyperlink{https://nl.wikipedia.org/wiki/#1}{#2}
    \pdfOnly{\footnote{See \url{https://nl.wikipedia.org/wiki/#1}}
    }
}

\renewcommand{\figurename}{Figuur}
\renewcommand{\tablename}{Tabel}

%
% Gedoe om verschillende versies van xourse/ximera te maken afhankelijk van settings
%
% default: versie met antwoorden
% handout: versie voor de studenten, zonder antwoorden/oplossingen
% full: met alles erop en eraan, dus geschikt voor auteurs en/of lesgevers  (bevat in de pdf ook de 'online-only' stukken!)
%
%
% verder kunnen ook opties/variabele worden gezet voor hints/auteurs/uitweidingen/ etc
%
% 'Full' versie
\newtoggle{showonline}
\ifdefined\HCode   % zet default showOnline
    \toggletrue{showonline} 
\else
    \togglefalse{showonline}
\fi

% Full versie   % deprecated: see infra
\newcommand{\printFull}{
    \hintstrue
    \handoutfalse
    \toggletrue{showonline} 
}

\ifdefined\shouldPrintFull   % deprecated: see infra
    \printFull
\fi



% Overschrijf onlineOnly  (zoals gedefinieerd in ximera.cls)
\ifhandout   % in handout: gebruik de oorspronkelijke ximera.cls implementatie  (is dit wel nodig/nuttig?)
\else
    \iftoggle{showonline}{%
        \ifdefined\HCode
          \RenewEnviron{onlineOnly}{\bgroup\BODY\egroup}   % showOnline, en we zijn  online, dus toon de tekst
        \else
          \RenewEnviron{onlineOnly}{\bgroup\color{red!50!black}\BODY\egroup}   % showOnline, maar we zijn toch niet online: kleur de tekst rood 
        \fi
    }{%
      \RenewEnviron{onlineOnly}{}  % geen showOnline
    }
\fi

% hack om na hoofding van definition/proposition/... als dan niet op een nieuwe lijn te starten
% soms is dat goed en mooi, en soms niet; en in HTML is het nu (2/2020) anders dan in pdf
% vandaar suggestie om 
%     \begin{definition}[Nieuw concept] \nl
% te gebruiken als je zeker een newline wil na de hoofdig en titel
% (in het bijzonder itemize zonder \nl is 'lelijk' ...)
\ifdefined\HCode
\newcommand{\nl}{}
\else
\newcommand{\nl}{\ \par} % newline (achter heading van definition etc.)
\fi


% \nl enkel in handoutmode (ihb voor \wordChoice, die dan typisch veeeel langer wordt)
\ifdefined\HCode
\providecommand{\handoutnl}{}
\else
\providecommand{\handoutnl}{%
\ifhandout%
  \nl%
\fi%
}
\fi

% Could potentially replace \pdfOnline/\begin{onlineOnly} : 
% Usage= \ifonline{Hallo surfer}{Hallo PDFlezer}
\providecommand{\ifonline}[2]%
{
\begin{onlineOnly}#1\end{onlineOnly}%
\pdfOnly{#2}
}%


%
% Maak optionele 'basic' en 'extended' versies van een activity
%  met environment basicOnly, basicSkip en extendedOnly
%
%  (
%   Dit werkt ENKEL in de PDF; de online versies tonen (minstens voorklopig) steeds 
%   het default geval met printbasicversion en printextendversion beide FALSE
%  )
%
\providebool{printbasicversion}
\providebool{printextendedversion}   % not properly implemented
\providebool{printfullversion}       % presumably print everything (debug/auteur)
%
% only set these in xourses, and BEFORE loading this preamble
%
%\newif\ifshowbasic     \showbasictrue        % use this line in xourse to show 'basic' sections
%\newif\ifshowextended  \showextendedtrue     % use this line in xourse to show 'extended' sections
%
%
%\ifbool{showbasic}
%      { \NewEnviron{basicOnly}{\BODY} }    % if yes: just print contents
%      { \NewEnviron{basicOnly}{}      }    % if no:  completely ignore contents
%
%\ifbool{showbasic}
%      { \NewEnviron{basicSkip}{}      }
%      { \NewEnviron{basicSkip}{\BODY} }
%

\ifbool{printextendedversion}
      { \NewEnviron{extendedOnly}{\BODY} }
      { \NewEnviron{extendedOnly}{}      }
      


\ifdefined\HCode    % in html: always print
      {\newenvironment*{basicOnly}{}{}}    % if yes: just print contents
      {\newenvironment*{basicSkip}{}{}}    % if yes: just print contents
\else

\ifbool{printbasicversion}
      {\newenvironment*{basicOnly}{}{}}    % if yes: just print contents
      {\NewEnviron{basicOnly}{}      }    % if no:  completely ignore contents

\ifbool{printbasicversion}
      {\NewEnviron{basicSkip}{}      }
      {\newenvironment*{basicSkip}{}{}}

\fi

\usepackage{float}
\usepackage[rightbars,color]{changebar}

% Full versie
\ifbool{printfullversion}{
    \hintstrue
    \handoutfalse
    \toggletrue{showonline}
    \printbasicversionfalse
    \cbcolor{red}
    \renewenvironment*{basicOnly}{\cbstart}{\cbend}
    \renewenvironment*{basicSkip}{\cbstart}{\cbend}
    \def\xmtoonprintopties{FULL}   % will be printed in footer
}
{}
      
%
% Evalueer \ifhints IN de environment
%  
%
%\RenewEnviron{hint}
%{
%\ifhandout
%\ifhints\else\setbox0\vbox\fi%everything in een emty box
%\bgroup 
%\stepcounter{hintLevel}
%\BODY
%\egroup\ignorespacesafterend
%\addtocounter{hintLevel}{-1}
%\else
%\ifhints
%\begin{trivlist}\item[\hskip \labelsep\small\slshape\bfseries Hint:\hspace{2ex}]
%\small\slshape
%\stepcounter{hintLevel}
%\BODY
%\end{trivlist}
%\addtocounter{hintLevel}{-1}
%\fi
%\fi
%}

% Onafhankelijk van \ifhandout ...? TO BE VERIFIED
\RenewEnviron{hint}
{
\ifhints
\begin{trivlist}\item[\hskip \labelsep\small\bfseries Hint:\hspace{2ex}]
\small%\slshape
\stepcounter{hintLevel}
\BODY
\end{trivlist}
\addtocounter{hintLevel}{-1}
\else
\iftikzexport   % anders worden de tikz tekeningen in hints niet gegenereerd ?
\setbox0\vbox\bgroup
\stepcounter{hintLevel}
\BODY
\egroup\ignorespacesafterend
\addtocounter{hintLevel}{-1}
\fi % ifhandout
\fi %ifhints
}

%
% \tab sets typewriter-tabs (e.g. to format questions)
% (Has no effect in HTML :-( ))
%
\usepackage{tabto}
\ifdefined\HCode
  \renewcommand{\tab}{\quad}    % otherwise dummy .png's are generated ...?
\fi


% Also redefined in  preamble to get correct styling 
% for tikz images for (\tikzexport)
%

\theoremstyle{definition} % Bold titels
\makeatletter
\let\proposition\relax
\let\c@proposition\relax
\let\endproposition\relax
\makeatother
\newtheorem{proposition}{Eigenschap}


%\instructornotesfalse

% logic with \ifhandoutin ximera.cls unclear;so overwrite ...
\makeatletter
\@ifundefined{ifinstructornotes}{%
  \newif\ifinstructornotes
  \instructornotesfalse
  \newenvironment{instructorNotes}{}{}
}{}
\makeatother
\ifinstructornotes
\else
\renewenvironment{instructorNotes}%
{%
    \setbox0\vbox\bgroup
}
{%
    \egroup
}
\fi

% \RedeclareMathOperator
% from https://tex.stackexchange.com/questions/175251/how-to-redefine-a-command-using-declaremathoperator
\makeatletter
\newcommand\RedeclareMathOperator{%
    \@ifstar{\def\rmo@s{m}\rmo@redeclare}{\def\rmo@s{o}\rmo@redeclare}%
}
% this is taken from \renew@command
\newcommand\rmo@redeclare[2]{%
    \begingroup \escapechar\m@ne\xdef\@gtempa{{\string#1}}\endgroup
    \expandafter\@ifundefined\@gtempa
    {\@latex@error{\noexpand#1undefined}\@ehc}%
    \relax
    \expandafter\rmo@declmathop\rmo@s{#1}{#2}}
% This is just \@declmathop without \@ifdefinable
\newcommand\rmo@declmathop[3]{%
    \DeclareRobustCommand{#2}{\qopname\newmcodes@#1{#3}}%
}
\@onlypreamble\RedeclareMathOperator
\makeatother


%
% Engelse vertaling, vooral in mathmode
%
% 1. Algemene procedure
%
\ifdefined\isEn
 \newcommand{\nlen}[2]{#2}
 \newcommand{\nlentext}[2]{\text{#2}}
 \newcommand{\nlentextbf}[2]{\textbf{#2}}
\else
 \newcommand{\nlen}[2]{#1}
 \newcommand{\nlentext}[2]{\text{#1}}
 \newcommand{\nlentextbf}[2]{\textbf{#1}}
\fi

%
% 2. Lijst van erg veel gebruikte uitdrukkingen
%

% Ja/Nee/Fout/Juits etc
%\newcommand{\TJa}{\nlentext{ Ja }{ and }}
%\newcommand{\TNee}{\nlentext{ Nee }{ No }}
%\newcommand{\TJuist}{\nlentext{ Juist }{ Correct }
%\newcommand{\TFout}{\nlentext{ Fout }{ Wrong }
\newcommand{\TWaar}{\nlentext{ Waar }{ True }}
\newcommand{\TOnwaar}{\nlentext{ Vals }{ False }}
% Korte bindwoorden en, of, dus, ...
\newcommand{\Ten}{\nlentext{ en }{ and }}
\newcommand{\Tof}{\nlentext{ of }{ or }}
\newcommand{\Tdus}{\nlentext{ dus }{ so }}
\newcommand{\Tendus}{\nlentext{ en dus }{ and thus }}
\newcommand{\Tvooralle}{\nlentext{ voor alle }{ for all }}
\newcommand{\Took}{\nlentext{ ook }{ also }}
\newcommand{\Tals}{\nlentext{ als }{ when }} %of if?
\newcommand{\Twant}{\nlentext{ want }{ as }}
\newcommand{\Tmaal}{\nlentext{ maal }{ times }}
\newcommand{\Toptellen}{\nlentext{ optellen }{ add }}
\newcommand{\Tde}{\nlentext{ de }{ the }}
\newcommand{\Thet}{\nlentext{ het }{ the }}
\newcommand{\Tis}{\nlentext{ is }{ is }} %zodat is in text staat in mathmode (geen italics)
\newcommand{\Tmet}{\nlentext{ met }{ where }} % in situaties e.g met p < n --> where p < n
\newcommand{\Tnooit}{\nlentext{ nooit }{ never }}
\newcommand{\Tmaar}{\nlentext{ maar }{ but }}
\newcommand{\Tniet}{\nlentext{ niet }{ not }}
\newcommand{\Tuit}{\nlentext{ uit }{ from }}
\newcommand{\Ttov}{\nlentext{ t.o.v. }{ w.r.t. }}
\newcommand{\Tzodat}{\nlentext{ zodat }{ such that }}
\newcommand{\Tdeth}{\nlentext{de }{th }}
\newcommand{\Tomdat}{\nlentext{omdat }{because }} 


%
% Overschrijf addhoc commando's
%
\ifdefined\isEn
\renewcommand{\pernot}{\overset{\mathrm{notation}}{=}}
\RedeclareMathOperator{\bld}{im}     % beeld
\RedeclareMathOperator{\graf}{graph}   % grafiek
\RedeclareMathOperator{\rico}{slope}   % richtingcoëfficient
\RedeclareMathOperator{\co}{co}       % coordinaat
\RedeclareMathOperator{\gr}{deg}       % graad

% Operators
\RedeclareMathOperator{\bgsin}{arcsin}
\RedeclareMathOperator{\bgcos}{arccos}
\RedeclareMathOperator{\bgtan}{arctan}
\RedeclareMathOperator{\bgcot}{arccot}
\RedeclareMathOperator{\bgsinh}{arcsinh}
\RedeclareMathOperator{\bgcosh}{arccosh}
\RedeclareMathOperator{\bgtanh}{arctanh}
\RedeclareMathOperator{\bgcoth}{arccoth}

\fi


% HACK: use 'oplossing' for 'explanation' ...
\let\explanation\relax
\let\endexplanation\relax
% \newenvironment{explanation}{\begin{oplossing}}{\end{oplossing}}
\newcounter{explanation}

\ifhandout%
    \NewEnviron{explanation}[1][toon]%
    {%
    \RenewEnviron{verbatim}{ \red{VERBATIM CONTENT MISSING IN THIS PDF}} %% \expandafter\verb|\BODY|}

    \ifthenelse{\equal{\detokenize{#1}}{\detokenize{toon}}}
    {
    \def\PH@Command{#1}% Use PH@Command to hold the content and be a target for "\expandafter" to expand once.

    \begin{trivlist}% Begin the trivlist to use formating of the "Feedback" label.
    \item[\hskip \labelsep\small\slshape\bfseries Explanation:% Format the "Feedback" label. Don't forget the space.
    %(\texttt{\detokenize\expandafter{\PH@Command}}):% Format (and detokenize) the condition for feedback to trigger
    \hspace{2ex}]\small%\slshape% Insert some space before the actual feedback given.
    \BODY
    \end{trivlist}
    }
    {  % \begin{feedback}[solution]   \BODY     \end{feedback}  }
    }
    }    
\else
% ONLY for HTML; xmoplossing is styled with css, and is not, and need not be a LaTeX environment
% THUS: it does NOT use feedback anymore ...
%    \NewEnviron{oplossing}{\begin{expandable}{xmoplossing}{\nlen{Toon uitwerking}{Show solution}}{\BODY}\end{expandable}}
    \newenvironment{explanation}[1][toon]
   {%
       \begin{expandable}{xmoplossing}{}
   }
   {%
   	   \end{expandable}
   } 
\fi

\author{Paul Zachlin \and Anna Davis \and Paul Bender} \title{Abstract Vector Spaces} \license{CC-BY-NC-SA}
\begin{document}

\begin{abstract}
We state the definition of an abstract vector space, and learn how to determine if a given set with two operations is a vector space. We define a subspace of a vector space and state the subspace test.  We find linear combinations and span of elements of a vector space.
\end{abstract}
\maketitle


\section*{VSP-0050: Abstract Vector Spaces}
\subsection*{Properties of Vector Spaces} 

In VSP-0020 we discussed $\RR^n$ as a vector space and introduced the notion of a subspace of $\RR^n$.  
In this module we will consider sets other than $\RR^n$ that have two operations and satisfy the same properties.  Such sets, together with the operations of addition and scalar multiplication, will also be called vector spaces.

Recall that $\RR^n$ was said to be a vector space because
\begin{itemize}
    \item[] $\RR^n$ is closed under vector addition
    \item[] $\RR^n$ is closed under scalar multiplication
\end{itemize}
and satisfies the following properties:

  \begin{enumerate}
  \item 
  Commutative Property of Addition:\quad
  $\vec{u}+\vec{v}=\vec{v}+\vec{u}$
  \item 
  Associative Property of Addition:\quad
  $(\vec{u}+\vec{v})+\vec{w}=\vec{u}+(\vec{v}+\vec{w})$
  \item 
  Existence of Additive Identity:\quad
  $\vec{u}+\vec{0}=\vec{u}$
  \item 
  Existence of Additive Inverse:\quad
  $\vec{u}+(-\vec{u})=\vec{0}$
  \item
  Distributive Property over Vector Addition:\quad
  $k(\vec{u}+\vec{v})=k\vec{u}+k\vec{v}$
  \item
  Distributive Property over Scalar Addition:\quad
  $(k+p)\vec{u}=k\vec{u}+p\vec{u}$
  \item 
  Associative Property for Scalar Multiplication:\quad
  $k(p\vec{u})=(kp)\vec{u}$
  \item 
  Multiplication by $1$:\quad
  $1\vec{u}=\vec{u}$
  \end{enumerate}

In the next two examples we will explore two sets other than $\RR^n$ endowed with addition and scalar multiplication and satisfying the same properties.

\begin{example}\label{ex:setofmatricesvectorspace}
Let $\mathbb{M}_{m,n}$ be the set of all $m\times n$ matrices.  Matrix addition and scalar multiplication were defined in MAT-0010.

Observe that the sum of two $m\times n$ matrices is also an $m\times n$ matrix. Likewise, a scalar multiple of an $m\times n$ matrix is an $m\times n$ matrix.  Thus 
\begin{itemize}
    \item[] $\mathbb{M}_{m,n}$ is closed under matrix addition
    \item[] $\mathbb{M}_{m,n}$ is closed under scalar multiplication
\end{itemize}

In addition, Theorems \ref{th:propertiesofaddition} and \ref{th:propertiesscalarmult} of MAT-0010 give us the following properties of matrix addition and scalar multiplication.  Note that these properties are analogous to the eight vector properties.
\begin{enumerate}
  \item 
  Commutative Property of Addition:  $\quad A+B=B+A$
  \item 
  Associative Property of Addition: $\quad (A+B)+C=A+(B+C)$
  \item 
  Existence of Additive Identity:  $\quad A+\vec{0}=A$ where $\vec{0}$ is the $m \times n$ zero matrix
  \item 
  Existence of Additive Inverse:  $\quad A+(-A)=\vec{0}$
  \item
  Distributive Property over Matrix Addition:  $\quad k(A+B)=kA+kB$
  \item
  Distributive Property over Scalar Addition:  $\quad (k+p)A=kA+pA$
  \item 
  Associative Property for Scalar Multiplication: $\quad k(pA)=(kp)A$
  \item 
  Multiplication by $1$: $\quad 1A=A$
  \end{enumerate}
\end{example}

\begin{example}\label{ex:linfunctionsvectspace} Consider the set $\mathbb{L}$ of all linear functions.  This set includes all polynomials of degree $1$ and degree $0$.  We will use addition and scalar multiplication of polynomials as the two operations, and show that $\mathbb{L}$ is closed under those operations and satisfies eight properties analogous to those of vectors of $\RR^n$.
\begin{explanation}
Elements of $\mathbb{L}$ are functions $f$ given by
$$f(x)=mx+b$$
(Note that $m$ and $b$ can be equal to zero.)

Given $f_1$ and $f_2$ in $\mathbb{L}$, it is easy to verify that $f_1+f_2$ is also in $\mathbb{L}$.  This gives us closure under function addition.

For any scalar $k$, we have
$$kf(x)=k(mx+b)=(km)x+(kb)$$
Therefore $kf$ is in $\mathbb{L}$, and $\mathbb{L}$ is closed under scalar multiplication.

We now proceed to formulate eight properties analogous to those of vectors of $\RR^n$.

Let $f_1$, $f_2$ and $f_3$ be elements of $\mathbb{L}$ given by $f_1(x)=m_1 x + b_1$, $f_2(x)=m_2 x + b_2$, and $f_3(x)=m_3 x + b_3$. Let $k$ and $p$ be scalars.  
  \begin{enumerate}
  \item 
  Commutative Property of Addition:  
  \quad $f_1+f_2=f_2+f_1$
  
  This property holds because
  \begin{align*}f_1(x) + f_2(x) &= (m_1 x + b_1) + (m_2 x + b_2)\\ &= (m_2 x + b_2) + (m_1 x + b_1)\\ &= f_2(x) + f_1(x)
  \end{align*}
  \item 
  Associative Property of Addition:\quad
  $(f_1 + f_2) + f_3 = f_1 + (f_2 + f_3)$
  
  This property is easy to verify.
  \item 
  Existence of Additive Identity:\quad  
  $f_1 + f_0 = f_1$
  
  The additive identity $f_0$ is given by $f_0(x)=0$.  Note that $f_0$ is in $\mathbb{L}$.
  \item 
  Existence of Additive Inverse: \quad 
    $f_1 + (-f_1) = f_0$ 
    
The additive inverse of $f_1$ is a function $-f_1$ given by $-f_1(x)=-mx+(-b)$.    Note that $-f_1$ is in $\mathbb{L}$.
  \item
  Distributive Property over Vector Addition:\quad 
  $k(f_1+f_2)=kf_1+kf_2$
  
  This property holds because
 \begin{align*}
  k(f_1(x) + f_2(x)) &= k((m_1 x + b_1) + (m_2 x + b_2))\\  &= k(m_1 x + b_1) + k(m_2 x + b_2)\\  &= k f_1(x) + k f_2(x)
  \end{align*}
  
  \item
  Distributive Property over Scalar Addition:\quad $(k+p)f_1=kf_1+pf_1$  
  
  This property holds because
  \begin{align*}(k+p)f_1(x)&= (k+p)(m_1 x + b_1)\\ &=k(m_1 x + b_1) + p(m_1 x + b_1)\\ &= k f_1(x) + p f_1(x)\end{align*}
 
  \item 
  Associative Property for Scalar Multiplication:\quad $(k(pf_1))=(kp)f_1$ 
  
  This property holds because
  \begin{align*}k(p(f_1(x)))&=k(p(m_1 x + b_1))\\&=k(p m_1 x +p b_1)\\ &= (kp) m_1 x + (kp) b_1\\ &= (kp)(m_1 x + b_1)\\&=(kp)f_1(x)\end{align*}
  
  \item 
  Multiplication by $1$ 
  $$1 f_1=f_1$$
  \end{enumerate}
  \end{explanation}
\end{example}

\subsection*{Definition of a Vector Space}

Examples \ref{ex:setofmatricesvectorspace} and \ref{ex:linfunctionsvectspace} show us that there are many times in mathematics when we encounter a set with two operations (that we call addition and scalar multiplication) that satisfies the same properties as $\RR^n$.  We will refer to such sets as \dfn{vector spaces}.

  \begin{definition}\label{def:vectorspacegeneral} 
  Let $V$ be a nonempty set.  Suppose that elements of $V$ can be added together and multiplied by scalars.  The set $V$, together with operations of addition and scalar multiplication, is called a \dfn{vector space} provided that 
  \begin{itemize}
  \item[] $V$ is closed under addition
  \item[] $V$ is closed under scalar multiplication
  \end{itemize}
  and the following properties hold for $\vec{u}$, $\vec{v}$ and $\vec{w}$ in $V$ and scalars $k$ and $p$:
  \begin{enumerate}
   \item \label{item:commaddvectspdef}
  Commutative Property of Addition:\quad
  $\vec{u}+\vec{v}=\vec{v}+\vec{u}$
  \item \label{item:assaddvectspdef}
  Associative Property of Addition:\quad
  $(\vec{u}+\vec{v})+\vec{w}=\vec{u}+(\vec{v}+\vec{w})$
  \item \label{item:idaddvectspdef}
  Existence of Additive Identity:\quad
  $\vec{u}+\vec{0}=\vec{u}$
  \item \label{item:invaddvectspdef}
  Existence of Additive Inverse:\quad
  $\vec{u}+(-\vec{u})=\vec{0}$
  \item \label{item:distvectaddvectspdef}
  Distributive Property over Vector Addition:\quad
  $k(\vec{u}+\vec{v})=k\vec{u}+k\vec{v}$
  \item \label{item:distscalaraddvectspdef}
  Distributive Property over Scalar Addition:\quad
  $(k+p)\vec{u}=k\vec{u}+p\vec{u}$
  \item \label{item:assmultvectspdef}
  Associative Property for Scalar Multiplication:\quad
  $k(p\vec{u})=(kp)\vec{u}$
  \item \label{item:idmultvectspdef}
  Multiplication by $1$:\quad
  $1\vec{u}=\vec{u}$
  \end{enumerate}
We will refer to elements of $V$ as \dfn{vectors}.  
\end{definition}

\begin{example}\label{ex:MLexamplesofvectspaces}
$\mathbb{M}_{m,n}$ and $\mathbb{L}$ are vector spaces. (See Examples \ref{ex:setofmatricesvectorspace} and \ref{ex:linfunctionsvectspace})
\end{example}
%{\color{red}(Adopted from Nicholson)}
Sets of polynomials provide an important source of examples, so we review some basic facts. A \dfn{polynomial} in $x$ is an expression
\begin{equation*}
p(x) = a_0 + a_1x + a_2x^2 + \ldots + a_nx^n
\end{equation*}
where $a_{0}, a_{1}, a_{2}, \ldots, a_{n}$ are real numbers called the \dfn{coefficients} of the polynomial. If all the coefficients are zero, the polynomial is called the \dfn{zero polynomial} and is denoted simply as $0$. If $p(x) \neq 0$, the highest power of $x$ with a nonzero coefficient is called the \dfn{degree} of $p(x)$ denoted as $\mbox{deg} p(x)$. The coefficient itself is called the \dfn{leading coefficient} of $p(x)$. Hence $\mbox{deg}(3 + 5x) = 1$, $\mbox{deg}(1 + x + x^{2}) = 2$, and $\mbox{deg}(4) = 0$. (The degree of the zero polynomial is not defined.)

Let $\mathbb{P}$ denote the set of all polynomials and suppose that
\begin{align*}
p(x) &= a_0 + a_1x + a_2x^2 + \ldots \\
q(x) &= b_0 + b_1x + b_2x^2 + \ldots
\end{align*}
are two polynomials in $\mathbb{P}$ (possibly of different degrees). Then $p(x)$ and $q(x)$ are called \dfn{equal} [written $p(x) = q(x)$] if and only if all the corresponding coefficients are equal---that is, $a_{0} = b_{0}$, $a_{1} = b_{1}$, $a_{2} = b_{2}$, and so on. In particular, $a_{0} + a_{1}x + a_{2}x^{2} + \ldots = 0$ means that $a_{0} = 0$, $a_{1} = 0$, $a_{2} = 0$, $\ldots$. 

The set $\mathbb{P}$ has an addition and scalar multiplication defined on it as follows: if $p(x)$ and $q(x)$ are as before and $k$ is a real number,
\begin{align*}
p(x) + q(x) &= (a_0 + b_0) + (a_1 + b_1)x + (a_2 + b_2)x^2 + \ldots \\
kp(x) &= ka_0 + (ka_1)x + (ka_2)x^2 + \ldots
\end{align*} 

\begin{example}\label{ex:pisavectorspace}
$\mathbb{P}$ is a vector space.  
\begin{explanation}It is easy to see that the sum of two polynomials is again a polynomial, and that a scalar multiple of a polynomial is a polynomial.  Thus, $\mathbb{P}$ is closed under addition and scalar multiplication.  The other eight vector space properties are easily verified, and we conclude that $\mathbb{P}$ is a vector space.
\end{explanation}
\end{example}
\begin{example}\label{ex:deg2onlynotavecspace}
Let $Y$ be the set of all degree two polynomials in $x$.  In other words,
$$Y=\left\{ax^2+bx+c : a \ne 0 \right\}$$
We claim that $Y$ is not a vector space.
\begin{explanation}
Observe that $Y$ is not closed under addition.  To see this, let $y_1 = 2x^2+3x+4$ and let $y_2=-2x^2$.  Then $y_1$ and $y_2$ are both elements of $Y$.  However, $y_1+y_2 = 3x+4$ is not an element of $Y$, as it is only a degree one polynomial.  We require the coefficient $a$ of $x^2$ to be nonzero for a polynomial to be in $Y$, and this is not the case for $y_1+y_2$.

As an exercise, check the remaining vector space properties one-by-one to see which properties hold and which do not.  
\end{explanation} %Watch the next video if you need help.
\end{example}

%{\color{red}video link}
% \href{https://odu.wistia.com/medias/mtg9f07kyk}
Set $Y$ in Example \ref{ex:deg2onlynotavecspace} is not a vector space, but if we make a slight modification, we can make it into a vector space.  

\begin{example}\label{ex:deg_le_2vectorspace}
Let  $\mathbb{P}^2$ be the set of polynomials of degree two or less.  In other words,
$$\mathbb{P}^2=\left\{ax^2+bx+c : a,b,c \in \mathbb{R} \right\}$$

Note that $\mathbb{P}^2$ contains the zero polynomial (let $a=b=c=0$).  

Unlike set $Y$ in Example \ref{ex:deg2onlynotavecspace}, $\mathbb{P}^2$ is closed under polynomial addition and scalar multiplication.  It is easy to verify that all vector space properties hold, so $\mathbb{P}^2$ is a vector space.
\end{example}

\begin{example}\label{ex:pnisavectorspace}
Let $n$ be a natural number.  Define $\mathbb{P}^n$ to be the set of polynomials of degree $n$ or less than $n$, then by reasoning similar to Example \ref{ex:deg_le_2vectorspace}, $\mathbb{P}^n$ is a vector space.
\end{example}

\subsection*{Subspaces}
\begin{definition}\label{def:subspaceabstract}
A nonempty subset $U$ of a vector space $V$ is called a \dfn{subspace} of $V$, provided that $U$ is itself a vector space when given the same addition and scalar multiplication as $V$.
\end{definition}

\begin{example}\label{ex:subspaceabstract1}
In Example \ref{ex:deg_le_2vectorspace} we demonstrated that $\mathbb{P}^2$ is a vector space.  From Example \ref{ex:pisavectorspace} we know that $\mathbb{P}$ is a vector space. But $\mathbb{P}^2$ is a subset of $\mathbb{P}$, and uses the same operations of polynomial addition and scalar multiplication.  Therefore $\mathbb{P}^2$ is a subspace of $\mathbb{P}$.
\end{example}

Checking all ten properties to verify that a subset of a vector space is a subspace can be cumbersome.  Fortunately we have the following theorem.

\begin{theorem}[Subspace Test]\label{th:subspacetestabstract}
Let $U$ be a nonempty subset of a vector space $V$.  If $U$ is closed under the operations of addition and scalar multiplication of $V$, then $U$ is a subspace of $V$.
\end{theorem}
\begin{proof}
To prove that closure is a sufficient condition for $U$ to be a subspace, we will need to show that closure under addition and scalar multiplication of $V$ guarantees that the remaining eight properties are satisfied automatically.  

Observe that Properties \ref{item:commaddvectspdef}, \ref{item:assaddvectspdef}, \ref{item:distvectaddvectspdef}, \ref{item:distscalaraddvectspdef}, \ref{item:assmultvectspdef} and \ref{item:idmultvectspdef} hold for all elements of $V$.  Thus, these properties will hold for all elements of $U$.  We say that these properties are \dfn{inherited} from $V$.

To prove Property \ref{item:idaddvectspdef} we need to show that $\vec{0}$, which we know to be an element of $V$, is contained in $U$.  Let $\vec{u}$ be an element of $U$ (recall that $U$ is nonempty).  We will show that $0\vec{u}=\vec{0}$ in $V$.  Then, by closure under scalar multiplication, we will be able to conclude that $0\vec{u}=\vec{0}$ must be in $U$.
$$0\vec{u}=(0+0)\vec{u}=0\vec{u}+0\vec{u}$$
Adding the additive inverse of $0\vec{u}$ to both sides gives us
$$0\vec{u}+(-0\vec{u})=(0\vec{u}+0\vec{u})+(-0\vec{u})$$
By Properties \ref{item:assaddvectspdef} and \ref{item:invaddvectspdef}

$$\vec{0}=0\vec{u}+(0\vec{u}+(-0\vec{u}))$$
By Properties \ref{item:idaddvectspdef} and \ref{item:invaddvectspdef}
$$\vec{0}=0\vec{u}+\vec{0}=0\vec{u}$$

Because $U$ is closed under scalar multiplication $0\vec{u}=\vec{0}$ is in $U$.

We know that every element of $U$, being an element of $V$, has an additive inverse  in $V$.  We need to show that the additive inverse of every element of $U$ is contained in $U$. Let $\vec{u}$ be any element of $U$.  We will show that $(-1)\vec{u}$ is the additive inverse of $\vec{u}$.  Then by closure, $(-1)\vec{u}$ will have to be contained in $U$.  To show that $(-1)\vec{u}$ is the additive inverse of $\vec{u}$, we must show that $\vec{u}+(-1)\vec{u}=\vec{0}$.  We compute:
$$\vec{u}+(-1)\vec{u}=1\vec{u}+(-1)\vec{u}=(1+(-1))\vec{u}=0\vec{u}=\vec{0}$$
Thus $(-1)\vec{u}$ is the additive inverse of $\vec{u}$. By closure, $(-1)\vec{u}$ is in $U$.  
\end{proof}

\begin{example}\label{ex:centralizerofA}
Let $A$ be a fixed matrix in $\mathbb{M}_{n,n}$. Show that the set $C_A$ of all $n\times n$ matrices that commute with $A$ under matrix multiplication  is a subspace of $\mathbb{M}_{n,n}$.  

\begin{explanation}
The set $C_A$ consists of all $n\times n$ matrices $X$ such that $AX=XA$.  First, observe that $C_A$ is not empty because $I_n$ is an element.  Now we need to show that $C_A$ is closed under matrix addition and scalar multiplication.

Suppose that $X_1$ and $X_{2}$ lie in $C_A$.  Then $AX_1 = X_1A$ and $AX_{2} = X_{2}A$. Then
$$
A(X_1 + X_2) 	= AX_1 + AX_2 = X_1A + X_2A + (X_1 + X_2)A $$
Therefore $(X_1+X_2)$ commutes with $A$.  Thus $(X_1+X_2)$ is in $C_A$.  We conclude that $C_A$ is closed under matrix addition.

Now suppose $X$ is in $C_A$.  Let $k$ be a scalar, then
$$
A(kX)= k(AX) = k(XA) = (kX)A
$$
Therefore $(kX)$ commutes with $A$.  We conclude that $(kX)$ is in $C_A$, and $C_A$ is closed under scalar multiplication.
 Hence $C_A$ is a subspace of $\mathbb{M}_{n,n}$.
\end{explanation}
\end{example}

%{\color{red}Nicholson} 
Suppose $p(x)$ is a polynomial and $a$ is a number. Then the number $p(a)$ obtained by replacing $x$ by $a$ in the expression for $p(x)$ is called the \dfn{evaluation} of $p(x)$ at $a$. For example, if $p(x) = 5 - 6x + 2x^{2}$, then the evaluation of $p(x)$ at $a = 2$ is $p(2) = 5 - 12 + 8 = 1$. If $p(a) = 0$, the number $a$ is called a \dfn{root} of $p(x)$.

\begin{example}\label{ex:root3}
Consider the set $U$ of all polynomials in $\mathbb{P}$ that have $3$ as a root:
\begin{equation*}
U = \{p(x) \in \mathbb{P} : p(3) = 0 \}
\end{equation*}
Show that $U$ is a subspace of $\mathbb{P}$.

\begin{explanation}
  Observe that $U$ is not empty because $r(x)=x-3$ is an element of $U$.  Suppose $p(x)$ and $q(x)$ lie in $U$.  Then $p(3) = 0$ and $q(3) = 0$. We have $(p + q)(x) = p(x) + q(x)$ for all $x$, so $(p + q)(3) = p(3) + q(3) = 0 + 0 = 0$, and $U$ is closed under addition. The verification that $U$ is closed under scalar multiplication is similar.
\end{explanation}
\end{example}



\subsection*{Linear Combinations and Span}

\begin{definition}\label{def:lincombabstract}
Let $V$ be a vector space and let $\vec{v}_1, \vec{v}_2,\ldots ,\vec{v}_n$ be vectors in $V$.  A vector $\vec{v}$ is said to be a \dfn{linear combination} of vectors $\vec{v}_1, \vec{v}_2,\ldots, \vec{v}_n$ if 
$$\vec{v}=a_1\vec{v}_1+ a_2\vec{v}_2+\ldots + a_n\vec{v}_n$$
for some scalars $a_1, a_2, \ldots ,a_n$.
\end{definition}

\begin{definition}\label{def:spanabstract} Let $V$ be a vector space and let $\vec{v}_1, \vec{v}_2,\ldots ,\vec{v}_p$ be vectors in $V$.  The set $S$ of all linear combinations of $\vec{v}_1, \vec{v}_2,\ldots ,\vec{v}_p$ is called the \dfn{span} of $\vec{v}_1, \vec{v}_2,\ldots ,\vec{v}_p$.  We write 
$$S=\mbox{span}(\vec{v}_1, \vec{v}_2,\ldots ,\vec{v}_p)$$
and we say that vectors $\vec{v}_1, \vec{v}_2,\ldots ,\vec{v}_p$ \dfn{span} $S$.  Any vector in $S$ is said to be \dfn{in the span} of $\vec{v}_1, \vec{v}_2,\ldots ,\vec{v}_p$.  The set $\{\vec{v}_1, \vec{v}_2,\ldots ,\vec{v}_p\}$ is called a \dfn{spanning set} for $S$.
\end{definition}
%{\color{red}Nicholson}
\begin{example}\label{ex:inthespanpoly}
Consider $p_{1} = 1 + x + 4x^{2}$ and $p_{2} = 1 + 5x + x^{2}$ in $\mathbb{P}^{2}$. Determine whether $p_{1}$ and $p_{2}$ lie in $\mbox{span}\{1 + 2x - x^{2}, 3 + 5x + 2x^{2}\}$.

\begin{explanation}
For $p_{1}$, we want to determine if $a$ and $b$ exist such that
\begin{equation*}
p_1 = a(1 + 2x - x^2) + b(3 + 5x + 2x^2)
\end{equation*}
Expanding the right hand side gives us:
$$a+2ax-ax^2+3b+5bx+2bx^2$$
Combining like terms, we get:
$$(a+3b)+(2a+5b)x+(-a+2b)x^2$$
Setting this equal to $p_{1} = 1 + x + 4x^{2}$ and
equating coefficients of powers of $x$ gives us a system of equations
\begin{equation*}
1 = a + 3b,\quad 1 = 2a + 5b, \quad \mbox{ and } \quad 4 = -a + 2b
\end{equation*}
This system has the solution $a = -2$ and $b = 1$, so $p_{1}$ is indeed in $\mbox{span}\{1 + 2x - x^{2}, 3 + 5x + 2x^{2}\}$.

Turning to $p_{2} = 1 + 5x + x^{2}$, we are looking for $a$ and $b$ such that 
\begin{equation*}
p_{2} = a(1 + 2x - x^{2}) + b(3 + 5x + 2x^{2})
\end{equation*}
 Again equating coefficients of powers of $x$ gives equations $1 = a + 3b$, $5 = 2a + 5b$, and $1 = -a + 2b$. But in this case there is no solution, so $p_{2}$ is not in $\mbox{span}\{1 + 2x - x^{2}, 3 + 5x + 2x^{2}\}$.
\end{explanation}
\end{example}

\begin{theorem}\label{th:spanisasubspaceabstract}
Let $V$ be a vector space.  Let $S$ be any subset of $V$.  Then $U=\mbox{span}(S)$ is a subspace of $V$.
\end{theorem}
\begin{proof}
See Practice Problem \ref{prob:spanisasubspaceabstract}.
\end{proof}

\section*{Practice Problems}
\begin{problem}
(Adapted from Kuttler, Exercises 9.1.1-9.1.4) Is the set of all points in $\mathbb{R}^2$ a vector space under the given definitions of addition and scalar multiplication?    In each case be specific about which vector space properties hold and which properties fail.
  \begin{problem}\label{prob:abstractvectspace1}
  Addition: $(a, b)+(c, d)=(a+d, b+c)$\\ Scalar Multiplication: $k(a, b)=(ka, kb)$
  \end{problem}
  \begin{problem}\label{prob:abstractvectspace2}
  Addition: $(a, b)+(c, d)=(0, b+d)$\\ Scalar Multiplication: $k(a, b)=(ka, kb)$
  \end{problem}
  \begin{problem}\label{prob:abstractvectspace3}
  Addition: $(a, b)+(c, d)=(a+c, b+d)$\\ Scalar Multiplication: $k(a, b)=(a, kb)$
  \end{problem}
  \begin{problem}\label{prob:abstractvectspace4}
  Addition: $(a, b)+(c, d)=(a-c, b-d)$\\ Scalar Multiplication: $k(a, b)=(ka, kb)$
  \end{problem}
  \end{problem}
   
 \begin{problem}\label{prob:abstractvectspace5}
 Let $\mathcal{F}$ be the set of all real-valued functions whose domain is all real numbers.  Define addition and scalar multiplication as follows:
 $$(f+g)(x)=f(x)+g(x)\quad (cf)(x)=cf(x)$$
 Verify that $\mathcal{F}$ is a vector space.
 \end{problem}
 \begin{problem}\label{prob:abstractvectspacediffeq}
 A differential equation is an equation that contains derivatives.  Consider the differential equation:
 \begin{align}\label{diffeq} f''+f=0\end{align}
A solution to such an equation is a function.
  \begin{enumerate}
  \item Verify that $f(x)=\sin x$ is a solution to (\ref{diffeq}).
  \item Is $f(x)=2\sin x$ a solution?
  \item Is $f(x)=\cos x$ a solution?
  \item Is $f(x)=\sin x+\cos x$ a solution?
  \item Let $S$ be the set of all solutions to (\ref{diffeq}).  Prove that $S$ is a vector space.
  \end{enumerate}
  \end{problem}
\begin{problem}\label{prob:abstractvectspacecomplex}
In this problem we will check that the set $\mathbb{C}$ of all complex numbers is in fact a vector space.  Let $z_1 = a_1 + b_1 i$ be a complex number where $i=\sqrt{-1}$.  Similarly, let $z_2 = a_2 + b_2 i$ and $z_3 = a_3 + b_3 i$ be complex numbers and let $k$ and $p$ be real number scalars.  Check that complex numbers are closed under addition and multiplication, and that they satisfy each of the vector space properties.
\end{problem}

\begin{problem}\label{prob:abstractvectspace6}
Refer to Example \ref{ex:centralizerofA} and describe all elements of $C_I$, where $I$ is a $3\times 3$ identity matrix.
\end{problem}
  
\begin{problem}\label{prob:abstractvectspace7}
Is the subset of all invertible $n\times n$ matrices a subspace of $\mathbb{M}_{n,n}$?  Prove your claim. 
\end{problem}

\begin{problem}\label{prob:symmetricsubspace}
Is the subset of all symmetric (See Definition \ref{def:symmetricandskewsymmetric} of MAT-0025) $n\times n$ matrices a subspace of $\mathbb{M}_{n,n}$?  Prove your claim. 
\end{problem}

\begin{problem}\label{prob:abstractvectspace8}
Let $Z$ be a subset of $\mathbb{M}_{n,n}$ that consists of $n\times n$ matrices that commute with {\it every} matrix in $\mathbb{M}_{n,n}$ under matrix multiplication. In other words,
$$Z=\{B : BY=YB \mbox{ for all } Y \mbox{ in } \mathbb{M}_{n,n}\}$$

Is $Z$ a subspace of $\mathbb{M}_{n,n}$?

\begin{hint}
Don't forget to check that $Z$ is not empty!
\end{hint}
\end{problem}

\begin{problem}\label{prob:abstractvectspace9}
List several elements of $\mbox{span}\left(\begin{bmatrix}1&0\\0&1\end{bmatrix}, \begin{bmatrix}0&1\\1&0\end{bmatrix}\right)$.  Suggest a spanning set for $\mathbb{M}_{2,2}$.
\end{problem}

\begin{problem}\label{prob:abstractvectspace10}
Find $\mbox{span}(1, x, x^2, x^3)$.
\end{problem}

\begin{problem}\label{prob:spanisasubspaceabstract}
Prove Theorem \ref{th:spanisasubspaceabstract}.
\end{problem}

\section*{Text Source} The discussion on polynomials was adapted from Section 6.1 of Keith Nicholson's \href{https://open.umn.edu/opentextbooks/textbooks/linear-algebra-with-applications}{\it Linear Algebra with Applications}. (CC-BY-NC-SA)

W. Keith Nicholson, {\it Linear Algebra with Applications}, Lyryx 2018, Open Edition, p. 331-332 

\section*{Example Source}
Examples \ref{ex:root3} and \ref{ex:inthespanpoly} were adapted from Examples 6.2.4 and 6.2.7 of Keith Nicholson's \href{https://open.umn.edu/opentextbooks/textbooks/linear-algebra-with-applications}{\it Linear Algebra with Applications}. (CC-BY-NC-SA)

W. Keith Nicholson, {\it Linear Algebra with Applications}, Lyryx 2018, Open Edition, p. 340-341 

\section*{Exercise Source}
Practice Problems \ref{prob:abstractvectspace1}-\ref{prob:abstractvectspace4} is adopted from Problems 9.1.1-9.1.4 of Ken Kuttler's \href{https://open.umn.edu/opentextbooks/textbooks/a-first-course-in-linear-algebra-2017}{\it A First Course in Linear Algebra}. (CC-BY)

Ken Kuttler, {\it  A First Course in Linear Algebra}, Lyryx 2017, Open Edition, p. 469.

\end{document} 