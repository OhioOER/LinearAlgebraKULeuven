\documentclass{ximera}
%%% Begin Laad packages

\makeatletter
\@ifclassloaded{xourse}{%
    \typeout{Start loading preamble.tex (in a XOURSE)}%
    \def\isXourse{true}   % automatically defined; pre 112022 it had to be set 'manually' in a xourse
}{%
    \typeout{Start loading preamble.tex (NOT in a XOURSE)}%
}
\makeatother

\def\isEn\true 

\pgfplotsset{compat=1.16}

\usepackage{currfile}

% 201908/202301: PAS OP: babel en doclicense lijken problemen te veroorzaken in .jax bestand
% (wegens syntax error met toegevoegde \newcommands ...)
\pdfOnly{
    \usepackage[type={CC},modifier={by-nc-sa},version={4.0}]{doclicense}
    %\usepackage[hyperxmp=false,type={CC},modifier={by-nc-sa},version={4.0}]{doclicense}
    %%% \usepackage[dutch]{babel}
}



\usepackage[utf8]{inputenc}
\usepackage{morewrites}   % nav zomercursus (answer...?)
\usepackage{multirow}
\usepackage{multicol}
\usepackage{tikzsymbols}
\usepackage{ifthen}
%\usepackage{animate} BREAKS HTML STRUCTURE USED BY XIMERA
\usepackage{relsize}

\usepackage{eurosym}    % \euro  (€ werkt niet in xake ...?)
\usepackage{fontawesome} % smileys etc

% Nuttig als ook interactieve beamer slides worden voorzien:
\providecommand{\p}{} % default nothing ; potentially usefull for slides: redefine as \pause
%providecommand{\p}{\pause}

    % Layout-parameters voor het onderschrift bij figuren
\usepackage[margin=10pt,font=small,labelfont=bf, labelsep=endash,format=hang]{caption}
%\usepackage{caption} % captionof
%\usepackage{pdflscape}    % landscape environment

% Met "\newcommand\showtodonotes{}" kan je todonotes tonen (in pdf/online)
% 201908: online werkt het niet (goed)
\providecommand\showtodonotes{disable}
\providecommand\todo[1]{\typeout{TODO #1}}
%\usepackage[\showtodonotes]{todonotes}
%\usepackage{todonotes}

%
% Poging tot aanpassen layout
%
\usepackage{tcolorbox}
\tcbuselibrary{theorems}

%%% Einde laad packages

%%% Begin Ximera specifieke zaken

\graphicspath{
	{../../}
	{../}
	{./}
  	{../../pictures/}
   	{../pictures/}
   	{./pictures/}
	{./explog/}    % M05 in groeimodellen       
}

%%% Einde Ximera specifieke zaken

%
% define softer blue/red/green, use KU Leuven base colors for blue (and dark orange for red ?)
%
% todo: rather redefine blue/red/green ...?
%\definecolor{xmblue}{rgb}{0.01, 0.31, 0.59}
%\definecolor{xmred}{rgb}{0.89, 0.02, 0.17}
\definecolor{xmdarkblue}{rgb}{0.122, 0.671, 0.835}   % KU Leuven Blauw
\definecolor{xmblue}{rgb}{0.114, 0.553, 0.69}        % KU Leuven Blauw
\definecolor{xmgreen}{rgb}{0.13, 0.55, 0.13}         % No KULeuven variant for green found ...

\definecolor{xmaccent}{rgb}{0.867, 0.541, 0.18}      % KU Leuven Accent (orange ...)
\definecolor{kuaccent}{rgb}{0.867, 0.541, 0.18}      % KU Leuven Accent (orange ...)

\colorlet{xmred}{xmaccent!50!black}                  % Darker version of KU Leuven Accent

\providecommand{\blue}[1]{{\color{blue}#1}}    
\providecommand{\red}[1]{{\color{red}#1}}

\renewcommand\CancelColor{\color{xmaccent!50!black}}

% werkt in math en text mode om MATH met oranje (of grijze...)  achtergond te tonen (ook \important{\text{blabla}} lijkt te werken)
%\newcommand{\important}[1]{\ensuremath{\colorbox{xmaccent!50!white}{$#1$}}}   % werkt niet in Mathjax
%\newcommand{\important}[1]{\ensuremath{\colorbox{lightgray}{$#1$}}}
\newcommand{\important}[1]{\ensuremath{\colorbox{orange}{$#1$}}}   % TODO: kleur aanpassen voor mathjax; wordt overschreven infra!


% Uitzonderlijk kan met \pdfnl in de PDF een newline worden geforceerd, die online niet nodig/nuttig is omdat daar de regellengte hoe dan ook niet gekend is.
\ifdefined\HCode%
\providecommand{\pdfnl}{}%
\else%
\providecommand{\pdfnl}{%
  \\%
}%
\fi

% Uitzonderlijk kan met \handoutnl in de handout-PDF een newline worden geforceerd, die noch online noch in de PDF-met-antwoorden nuttig is.
\ifdefined\HCode
\providecommand{\handoutnl}{}
\else
\providecommand{\handoutnl}{%
\ifhandout%
  \nl%
\fi%
}
\fi



% \cellcolor IGNORED by tex4ht ?
% \begin{center} seems not to wordk
    % (missing margin-left: auto;   on tabular-inside-center ???)
%\newcommand{\importantcell}[1]{\ensuremath{\cellcolor{lightgray}#1}}  %  in tabular; usablility to be checked
\providecommand{\importantcell}[1]{\ensuremath{#1}}     % no mathjax2 support for colloring array cells

\pdfOnly{
  \renewcommand{\important}[1]{\ensuremath{\colorbox{kuaccent!50!white}{$#1$}}}
  \renewcommand{\importantcell}[1]{\ensuremath{\cellcolor{kuaccent!40!white}#1}}   
}

%%% Tikz styles


\pgfplotsset{compat=1.16}

\usetikzlibrary{trees,positioning,arrows,fit,shapes,math,calc,decorations.markings,through,intersections,patterns,matrix}

\usetikzlibrary{decorations.pathreplacing,backgrounds}    % 5/2023: from experimental


\usetikzlibrary{angles,quotes}

\usepgfplotslibrary{fillbetween} % bepaalde_integraal
\usepgfplotslibrary{polar}    % oa voor poolcoordinaten.tex

\pgfplotsset{ownstyle/.style={axis lines = center, axis equal image, xlabel = $x$, ylabel = $y$, enlargelimits}} 

\pgfplotsset{
	plot/.style={no marks,samples=50}
}

\newcommand{\xmPlotsColor}{
	\pgfplotsset{
		plot1/.style={darkgray,no marks,samples=100},
		plot2/.style={lightgray,no marks,samples=100},
		plotresult/.style={blue,no marks,samples=100},
		plotblue/.style={blue,no marks,samples=100},
		plotred/.style={red,no marks,samples=100},
		plotgreen/.style={green,no marks,samples=100},
		plotpurple/.style={purple,no marks,samples=100}
	}
}
\newcommand{\xmPlotsBlackWhite}{
	\pgfplotsset{
		plot1/.style={black,loosely dashed,no marks,samples=100},
		plot2/.style={black,loosely dotted,no marks,samples=100},
		plotresult/.style={black,no marks,samples=100},
		plotblue/.style={black,no marks,samples=100},
		plotred/.style={black,dotted,no marks,samples=100},
		plotgreen/.style={black,dashed,no marks,samples=100},
		plotpurple/.style={black,dashdotted,no marks,samples=100}
	}
}


\newcommand{\xmPlotsColorAndStyle}{
	\pgfplotsset{
		plot1/.style={darkgray,no marks,samples=100},
		plot2/.style={lightgray,no marks,samples=100},
		plotresult/.style={blue,no marks,samples=100},
		plotblue/.style={xmblue,no marks,samples=100},
		plotred/.style={xmred,dashed,thick,no marks,samples=100},
		plotgreen/.style={xmgreen,dotted,very thick,no marks,samples=100},
		plotpurple/.style={purple,no marks,samples=100}
	}
}


%\iftikzexport
\xmPlotsColorAndStyle
%\else
%\xmPlotsBlackWhite
%\fi
%%%


%
% Om venndiagrammen te arceren ...
%
\makeatletter
\pgfdeclarepatternformonly[\hatchdistance,\hatchthickness]{north east hatch}% name
{\pgfqpoint{-1pt}{-1pt}}% below left
{\pgfqpoint{\hatchdistance}{\hatchdistance}}% above right
{\pgfpoint{\hatchdistance-1pt}{\hatchdistance-1pt}}%
{
	\pgfsetcolor{\tikz@pattern@color}
	\pgfsetlinewidth{\hatchthickness}
	\pgfpathmoveto{\pgfqpoint{0pt}{0pt}}
	\pgfpathlineto{\pgfqpoint{\hatchdistance}{\hatchdistance}}
	\pgfusepath{stroke}
}
\pgfdeclarepatternformonly[\hatchdistance,\hatchthickness]{north west hatch}% name
{\pgfqpoint{-\hatchthickness}{-\hatchthickness}}% below left
{\pgfqpoint{\hatchdistance+\hatchthickness}{\hatchdistance+\hatchthickness}}% above right
{\pgfpoint{\hatchdistance}{\hatchdistance}}%
{
	\pgfsetcolor{\tikz@pattern@color}
	\pgfsetlinewidth{\hatchthickness}
	\pgfpathmoveto{\pgfqpoint{\hatchdistance+\hatchthickness}{-\hatchthickness}}
	\pgfpathlineto{\pgfqpoint{-\hatchthickness}{\hatchdistance+\hatchthickness}}
	\pgfusepath{stroke}
}
%\makeatother

\tikzset{
    hatch distance/.store in=\hatchdistance,
    hatch distance=10pt,
    hatch thickness/.store in=\hatchthickness,
   	hatch thickness=2pt
}

\colorlet{circle edge}{black}
\colorlet{circle area}{blue!20}


\tikzset{
    filled/.style={fill=green!30, draw=circle edge, thick},
    arceerl/.style={pattern=north east hatch, pattern color=blue!50, draw=circle edge},
    arceerr/.style={pattern=north west hatch, pattern color=yellow!50, draw=circle edge},
    outline/.style={draw=circle edge, thick}
}




%%% Updaten commando's
\def\hoofding #1#2#3{\maketitle}     % OBSOLETE ??

% we willen (bijna) altijd \geqslant ipv \geq ...!
\newcommand{\geqnoslant}{\geq}
\renewcommand{\geq}{\geqslant}
\newcommand{\leqnoslant}{\leq}
\renewcommand{\leq}{\leqslant}

% Todo: (201908) waarom komt er (soms) underlined voor emph ...?
\renewcommand{\emph}[1]{\textit{#1}}

% API commando's

\newcommand{\ds}{\displaystyle}
\newcommand{\ts}{\textstyle}  % tegenhanger van \ds   (Ximera zet PER  DEFAULT \ds!)

% uit Zomercursus-macro's: 
\newcommand{\bron}[1]{\begin{scriptsize} \emph{#1} \end{scriptsize}}     % deprecated ...?


%definities nieuwe commando's - afkortingen veel gebruikte symbolen
\newcommand{\R}{\ensuremath{\mathbb{R}}}
\newcommand{\Rnul}{\ensuremath{\mathbb{R}_0}}
\newcommand{\Reen}{\ensuremath{\mathbb{R}\setminus\{1\}}}
\newcommand{\Rnuleen}{\ensuremath{\mathbb{R}\setminus\{0,1\}}}
\newcommand{\Rplus}{\ensuremath{\mathbb{R}^+}}
\newcommand{\Rmin}{\ensuremath{\mathbb{R}^-}}
\newcommand{\Rnulplus}{\ensuremath{\mathbb{R}_0^+}}
\newcommand{\Rnulmin}{\ensuremath{\mathbb{R}_0^-}}
\newcommand{\Rnuleenplus}{\ensuremath{\mathbb{R}^+\setminus\{0,1\}}}
\newcommand{\N}{\ensuremath{\mathbb{N}}}
\newcommand{\Nnul}{\ensuremath{\mathbb{N}_0}}
\newcommand{\Z}{\ensuremath{\mathbb{Z}}}
\newcommand{\Znul}{\ensuremath{\mathbb{Z}_0}}
\newcommand{\Zplus}{\ensuremath{\mathbb{Z}^+}}
\newcommand{\Zmin}{\ensuremath{\mathbb{Z}^-}}
\newcommand{\Znulplus}{\ensuremath{\mathbb{Z}_0^+}}
\newcommand{\Znulmin}{\ensuremath{\mathbb{Z}_0^-}}
\newcommand{\C}{\ensuremath{\mathbb{C}}}
\newcommand{\Cnul}{\ensuremath{\mathbb{C}_0}}
\newcommand{\Cplus}{\ensuremath{\mathbb{C}^+}}
\newcommand{\Cmin}{\ensuremath{\mathbb{C}^-}}
\newcommand{\Cnulplus}{\ensuremath{\mathbb{C}_0^+}}
\newcommand{\Cnulmin}{\ensuremath{\mathbb{C}_0^-}}
\newcommand{\Q}{\ensuremath{\mathbb{Q}}}
\newcommand{\Qnul}{\ensuremath{\mathbb{Q}_0}}
\newcommand{\Qplus}{\ensuremath{\mathbb{Q}^+}}
\newcommand{\Qmin}{\ensuremath{\mathbb{Q}^-}}
\newcommand{\Qnulplus}{\ensuremath{\mathbb{Q}_0^+}}
\newcommand{\Qnulmin}{\ensuremath{\mathbb{Q}_0^-}}

\newcommand{\perdef}{\overset{\mathrm{def}}{=}}
\newcommand{\pernot}{\overset{\mathrm{notatie}}{=}}
\newcommand\perinderdaad{\overset{!}{=}}     % voorlopig gebruikt in limietenrekenregels
\newcommand\perhaps{\overset{?}{=}}          % voorlopig gebruikt in limietenrekenregels

\newcommand{\degree}{^\circ}


\DeclareMathOperator{\dom}{dom}     % domein
\DeclareMathOperator{\codom}{codom} % codomein
\DeclareMathOperator{\bld}{bld}     % beeld
\DeclareMathOperator{\graf}{graf}   % grafiek
\DeclareMathOperator{\rico}{rico}   % richtingcoëfficient
\DeclareMathOperator{\co}{co}       % coordinaat
\DeclareMathOperator{\gr}{gr}       % graad

\newcommand{\func}[5]{\ensuremath{#1: #2 \rightarrow #3: #4 \mapsto #5}} % Easy to write a function


% Operators
\DeclareMathOperator{\bgsin}{bgsin}
\DeclareMathOperator{\bgcos}{bgcos}
\DeclareMathOperator{\bgtan}{bgtan}
\DeclareMathOperator{\bgcot}{bgcot}
\DeclareMathOperator{\bgsinh}{bgsinh}
\DeclareMathOperator{\bgcosh}{bgcosh}
\DeclareMathOperator{\bgtanh}{bgtanh}
\DeclareMathOperator{\bgcoth}{bgcoth}

% Oude \Bgsin etc deprecated: gebruik \bgsin, en herdefinieer dat als je Bgsin wil!
%\DeclareMathOperator{\cosec}{cosec}    % not used? gebruik \csc en herdefinieer

% operatoren voor differentialen: to be verified; 1/2020: inconsequent gebruik bij afgeleiden/integralen
\renewcommand{\d}{\mathrm{d}}
\newcommand{\dx}{\d x}
\newcommand{\dd}[1]{\frac{\mathrm{d}}{\mathrm{d}#1}}
\newcommand{\ddx}{\dd{x}}

% om in voorbeelden/oefeningen de notatie voor afgeleiden te kunnen kiezen
% Usage: \afg{(2\sin(x))}  (en wordt d/dx, of accent, of D )
%\newcommand{\afg}[1]{{#1}'}
\newcommand{\afg}[1]{\left(#1\right)'}
%\renewcommand{\afg}[1]{\frac{\mathrm{d}#1}{\mathrm{d}x}}   % include in relevant exercises ...
%\renewcommand{\afg}[1]{D{#1}}

%
% \xmxxx commands: Extra KU Leuven functionaliteit van, boven of naast Ximera
%   ( Conventie 8/2019: xm+nederlandse omschrijving, maar is niet consequent gevolgd, en misschien ook niet erg handig !)
%
% (Met een minimale ximera.cls en preamble.tex zou een bruikbare .pdf moeten kunnen worden gemaakt van eender welke ximera)
%
% Usage: \xmtitle[Mijn korte abstract]{Mijn titel}{Mijn abstract}
% Eerste command na \begin{document}:
%  -> definieert de \title
%  -> definieert de abstract
%  -> doet \maketitle ( dus: print de hoofding als 'chapter' of 'sectie')
% Optionele parameter geeft eenn kort abstract (die met de globale setting \xmshortabstract{} al dan niet kan worden geprint.
% De optionele korte abstract kan worden gebruikt voor pseudo-grappige abtsarts, dus dus globaal al dan niet kunnen worden gebuikt...
% Globale settings:
%  de (optionele) 'korte abstract' wordt enkele getoond als \xmshortabstract is gezet
\providecommand\xmshortabstract{} % default: print (only!) short abstract if present
\newcommand{\xmtitle}[3][]{
	\title{#2}
	\begin{abstract}
		\ifdefined\xmshortabstract
		\ifstrempty{#1}{%
			#3
		}{%
			#1
		}%
		\else
		#3
		\fi
	\end{abstract}
	\maketitle
}

% 
% Kleine grapjes: moeten zonder verder gevolg kunnen worden verwijderd
%
%\newcommand{\xmopje}[1]{{\small#1{\reversemarginpar\marginpar{\Smiley}}}}   % probleem in floats!!
\newtoggle{showxmopje}
\toggletrue{showxmopje}

\newcommand{\xmopje}[1]{%
   \iftoggle{showxmopje}{#1}{}%
}


% -> geef een abstracte-formule-met-rechts-een-concreet-voorbeeld
% VB:  \formulevb{a^2+b^2=c^2}{3^2+4^2=5^2}
%
\ifdefined\HCode
\NewEnviron{xmdiv}[1]{\HCode{\Hnewline<div class="#1">\Hnewline}\BODY{\HCode{\Hnewline</div>\Hnewline}}}
\else
\NewEnviron{xmdiv}[1]{\BODY}
\fi

\providecommand{\formulevb}[2]{
	{\centering

    \begin{xmdiv}{xmformulevb}    % zie css voor online layout !!!
	\begin{tabular}{lcl}
		\important{#1}
		&  &
		Vb: $#2$
		\end{tabular}
	\end{xmdiv}

	}
}

\ifdefined\HCode
\providecommand{\vb}[1]{%
    \HCode{\Hnewline<span class="xmvb">}#1\HCode{</span>\Hnewline}%
}
\else
\providecommand{\vb}[1]{
    \colorbox{blue!10}{#1}
}
\fi

\ifdefined\HCode
\providecommand{\xmcolorbox}[2]{
	\HCode{\Hnewline<div class="xmcolorbox">\Hnewline}#2\HCode{\Hnewline</div>\Hnewline}
}
\else
\providecommand{\xmcolorbox}[2]{
  \cellcolor{#1}#2
}
\fi


\ifdefined\HCode
\providecommand{\xmopmerking}[1]{
 \HCode{\Hnewline<div class="xmopmerking">\Hnewline}#1\HCode{\Hnewline</div>\Hnewline}
}
\else
\providecommand{\xmopmerking}[1]{
	{\footnotesize #1}
}
\fi
% \providecommand{\voorbeeld}[1]{
% 	\colorbox{blue!10}{$#1$}
% }



% Hernoem Proof naar Bewijs, nodig voor HTML versie
\renewcommand*{\proofname}{Bewijs}

% Om opgave van oefening (wordt niet geprint bij oplossingenblad)
% (to be tested test)
\NewEnviron{statement}{\BODY}

% Environment 'oplossing' en 'uitkomst'
% voor resp. volledige 'uitwerking' dan wel 'enkel eindresultaat'
% geimplementeerd via feedback, omdat er in de ximera-server adhoc feedback-code is toegevoegd
%% Niet tonen indien handout
%% Te gebruiken om volledige oplossingen/uitwerkingen van oefeningen te tonen
%% \begin{oplossing}        De optelling is commutatief \end{oplossing}  : verschijnt online enkel 'op vraag'
%% \begin{oplossing}[toon]  De optelling is commutatief \end{oplossing}  : verschijnt steeds onmiddellijk online (bv te gebruiken bij voorbeelden) 

\ifhandout%
    \NewEnviron{oplossing}[1][onzichtbaar]%
    {%
    \ifthenelse{\equal{\detokenize{#1}}{\detokenize{toon}}}
    {
    \def\PH@Command{#1}% Use PH@Command to hold the content and be a target for "\expandafter" to expand once.

    \begin{trivlist}% Begin the trivlist to use formating of the "Feedback" label.
    \item[\hskip \labelsep\small\slshape\bfseries Oplossing% Format the "Feedback" label. Don't forget the space.
    %(\texttt{\detokenize\expandafter{\PH@Command}}):% Format (and detokenize) the condition for feedback to trigger
    \hspace{2ex}]\small%\slshape% Insert some space before the actual feedback given.
    \BODY
    \end{trivlist}
    }
    {  % \begin{feedback}[solution]   \BODY     \end{feedback}  }
    }
    }    
\else
% ONLY for HTML; xmoplossing is styled with css, and is not, and need not be a LaTeX environment
% THUS: it does NOT use feedback anymore ...
%    \NewEnviron{oplossing}{\begin{expandable}{xmoplossing}{\nlen{Toon uitwerking}{Show solution}}{\BODY}\end{expandable}}
    \newenvironment{oplossing}[1][onzichtbaar]
   {%
       \begin{expandable}{xmoplossing}{}
   }
   {%
   	   \end{expandable}
   } 
%     \newenvironment{oplossing}[1][onzichtbaar]
%    {%
%        \begin{feedback}[solution]   	
%    }
%    {%
%    	   \end{feedback}
%    } 
\fi

\ifhandout%
    \NewEnviron{uitkomst}[1][onzichtbaar]%
    {%
    \ifthenelse{\equal{\detokenize{#1}}{\detokenize{toon}}}
    {
    \def\PH@Command{#1}% Use PH@Command to hold the content and be a target for "\expandafter" to expand once.

    \begin{trivlist}% Begin the trivlist to use formating of the "Feedback" label.
    \item[\hskip \labelsep\small\slshape\bfseries Uitkomst:% Format the "Feedback" label. Don't forget the space.
    %(\texttt{\detokenize\expandafter{\PH@Command}}):% Format (and detokenize) the condition for feedback to trigger
    \hspace{2ex}]\small%\slshape% Insert some space before the actual feedback given.
    \BODY
    \end{trivlist}
    }
    {  % \begin{feedback}[solution]   \BODY     \end{feedback}  }
    }
    }    
\else
\ifdefined\HCode
   \newenvironment{uitkomst}[1][onzichtbaar]
    {%
        \begin{expandable}{xmuitkomst}{}%
    }
    {%
    	\end{expandable}%
    } 
\else
  % Do NOT print 'uitkomst' in non-handout
  %  (presumably, there is also an 'oplossing' ??)
  \newenvironment{uitkomst}[1][onzichtbaar]{}{}
\fi
\fi

%
% Uitweidingen zijn extra's die niet redelijkerwijze tot de leerstof behoren
% Uitbreidingen zijn extra's die wel redelijkerwijze tot de leerstof van bv meer geavanceerde versies kunnen behoren (B-programma/Wiskundestudenten/...?)
% Nog niet voorzien: design voor verschillende versies (A/B programma, BIO, voorkennis/ ...)
% Voor 'uitweidingen' is er een environment die online per default is ingeklapt, en in pdf al dan niet kan worden geincluded  (via \xmnouitweiding) 
%
% in een xourse, per default GEEN uitweidingen, tenzij \xmuitweiding expliciet ergens is gezet ...
\ifdefined\isXourse
   \ifdefined\xmuitweiding
   \else
       \def\xmnouitweiding{true}
   \fi
\fi

\ifdefined\xmnouitweiding
\newcounter{xmuitweiding}  % anders error undefined ...  
\excludecomment{xmuitweiding}
\else
\newtheoremstyle{dotless}{}{}{}{}{}{}{ }{}
\theoremstyle{dotless}
\newtheorem*{xmuitweidingnofrills}{}   % nofrills = no accordion; gebruikt dus de dotless theoremstyle!

\newcounter{xmuitweiding}
\newenvironment{xmuitweiding}[1][ ]%
{% 
	\refstepcounter{xmuitweiding}%
    \begin{expandable}{xmuitweiding}{\nlentext{Uitweiding \arabic{xmuitweiding}: #1}{Digression \arabic{xmuitweiding}: #1}}%
	\begin{xmuitweidingnofrills}%
}
{%
    \end{xmuitweidingnofrills}%
    \end{expandable}%
}   
% \newenvironment{xmuitweiding}[1][ ]%
% {% 
% 	\refstepcounter{xmuitweiding}
% 	\begin{accordion}\begin{accordion-item}[Uitweiding \arabic{xmuitweiding}: #1]%
% 			\begin{xmuitweidingnofrills}%
% 			}
% 			{\end{xmuitweidingnofrills}\end{accordion-item}\end{accordion}}   
\fi


\newenvironment{xmexpandable}[1][]{
	\begin{accordion}\begin{accordion-item}[#1]%
		}{\end{accordion-item}\end{accordion}}


% Command that gives a selection box online, but just prints the right answer in pdf
\newcommand{\xmonlineChoice}[1]{\pdfOnly{\wordchoicegiventrue}\wordChoice{#1}\pdfOnly{\wordchoicegivenfalse}}   % deprecated, gebruik onlineChoice ...
\newcommand{\onlineChoice}[1]{\pdfOnly{\wordchoicegiventrue}\wordChoice{#1}\pdfOnly{\wordchoicegivenfalse}}

\newcommand{\TJa}{\nlentext{ Ja }{ Yes }}
\newcommand{\TNee}{\nlentext{ Nee }{ No }}
\newcommand{\TJuist}{\nlentext{ Juist }{ True }}
\newcommand{\TFout}{\nlentext{ Fout }{ False }}

\newcommand{\choiceTrue }{{\renewcommand{\choiceminimumhorizontalsize}{4em}\wordChoice{\choice[correct]{\TJuist}\choice{\TFout}}}}
\newcommand{\choiceFalse}{{\renewcommand{\choiceminimumhorizontalsize}{4em}\wordChoice{\choice{\TJuist}\choice[correct]{\TFout}}}}

\newcommand{\choiceYes}{{\renewcommand{\choiceminimumhorizontalsize}{3em}\wordChoice{\choice[correct]{\TJa}\choice{\TNee}}}}
\newcommand{\choiceNo }{{\renewcommand{\choiceminimumhorizontalsize}{3em}\wordChoice{\choice{\TJa}\choice[correct]{\TNee}}}}

% Optional nicer formatting for wordChoice in PDF

\let\inlinechoiceorig\inlinechoice

%\makeatletter
%\providecommand{\choiceminimumverticalsize}{\vphantom{$\frac{\sqrt{2}}{2}$}}   % minimum height of boxes (cfr infra)
\providecommand{\choiceminimumverticalsize}{\vphantom{$\tfrac{2}{2}$}}   % minimum height of boxes (cfr infra)
\providecommand{\choiceminimumhorizontalsize}{1em}   % minimum width of boxes (cfr infra)

\newcommand{\inlinechoicesquares}[2][]{%
		\setkeys{choice}{#1}%
		\ifthenelse{\boolean{\choice@correct}}%
		{%
            \ifhandout%
               \fbox{\choiceminimumverticalsize #2}\allowbreak\ignorespaces%
            \else%
               \fcolorbox{blue}{blue!20}{\choiceminimumverticalsize #2}\allowbreak\ignorespaces\setkeys{choice}{correct=false}\ignorespaces%
            \fi%
		}%
		{% else
			\fbox{\choiceminimumverticalsize #2}\allowbreak\ignorespaces%  HACK: wat kleiner, zodat fits on line ... 	
		}%
}

\newcommand{\inlinechoicesquareX}[2][]{%
		\setkeys{choice}{#1}%
		\ifthenelse{\boolean{\choice@correct}}%
		{%
            \ifhandout%
               \framebox[\ifdim\choiceminimumhorizontalsize<\width\width\else\choiceminimumhorizontalsize\fi]{\choiceminimumverticalsize\ #2\ }\allowbreak\ignorespaces\setkeys{choice}{correct=false}\ignorespaces%
            \else%
               \fcolorbox{blue}{blue!20}{\makebox[\ifdim\choiceminimumhorizontalsize<\width\width\else\choiceminimumhorizontalsize\fi]{\choiceminimumverticalsize #2}}\allowbreak\ignorespaces\setkeys{choice}{correct=false}\ignorespaces%
            \fi%
		}%
		{% else
        \ifhandout%
			\framebox[\ifdim\choiceminimumhorizontalsize<\width\width\else\choiceminimumhorizontalsize\fi]{\choiceminimumverticalsize\ #2\ }\allowbreak\ignorespaces%  HACK: wat kleiner, zodat fits on line ... 	
        \fi
		}%
}


\newcommand{\inlinechoicepuntjes}[2][]{%
		\setkeys{choice}{#1}%
		\ifthenelse{\boolean{\choice@correct}}%
		{%
            \ifhandout%
               \dots\ldots\ignorespaces\setkeys{choice}{correct=false}\ignorespaces
            \else%
               \fcolorbox{blue}{blue!20}{\choiceminimumverticalsize #2}\allowbreak\ignorespaces\setkeys{choice}{correct=false}\ignorespaces%
            \fi%
		}%
		{% else
			%\fbox{\choiceminimumverticalsize #2}\allowbreak\ignorespaces%  HACK: wat kleiner, zodat fits on line ... 	
		}%
}

% print niets, maar definieer globale variable \myanswer
%  (gebruikt om oplossingsbladen te printen) 
\newcommand{\inlinechoicedefanswer}[2][]{%
		\setkeys{choice}{#1}%
		\ifthenelse{\boolean{\choice@correct}}%
		{%
               \gdef\myanswer{#2}\setkeys{choice}{correct=false}

		}%
		{% else
			%\fbox{\choiceminimumverticalsize #2}\allowbreak\ignorespaces%  HACK: wat kleiner, zodat fits on line ... 	
		}%
}



%\makeatother

\newcommand{\setchoicedefanswer}{
\ifdefined\HCode
\else
%    \renewenvironment{multipleChoice@}[1][]{}{} % remove trailing ')'
    \let\inlinechoice\inlinechoicedefanswer
\fi
}

\newcommand{\setchoicepuntjes}{
\ifdefined\HCode
\else
    \renewenvironment{multipleChoice@}[1][]{}{} % remove trailing ')'
    \let\inlinechoice\inlinechoicepuntjes
\fi
}
\newcommand{\setchoicesquares}{
\ifdefined\HCode
\else
    \renewenvironment{multipleChoice@}[1][]{}{} % remove trailing ')'
    \let\inlinechoice\inlinechoicesquares
\fi
}
%
\newcommand{\setchoicesquareX}{
\ifdefined\HCode
\else
    \renewenvironment{multipleChoice@}[1][]{}{} % remove trailing ')'
    \let\inlinechoice\inlinechoicesquareX
\fi
}
%
\newcommand{\setchoicelist}{
\ifdefined\HCode
\else
    \renewenvironment{multipleChoice@}[1][]{}{)}% re-add trailing ')'
    \let\inlinechoice\inlinechoiceorig
\fi
}

\setchoicesquareX  % by default list-of-squares with onlineChoice in PDF

% Omdat multicols niet werkt in html: enkel in pdf  (in html zijn langere pagina's misschien ook minder storend)
\newenvironment{xmmulticols}[1][2]{
 \pdfOnly{\begin{multicols}{#1}}%
}{ \pdfOnly{\end{multicols}}}

%
% Te gebruiken in plaats van \section\subsection
%  (in een printstyle kan dan het level worden aangepast
%    naargelang \chapter vs \section style )
% 3/2021: DO NOT USE \xmsubsection !
\newcommand\xmsection\subsection
\newcommand\xmsubsection\subsubsection

% Aanpassen printversie
%  (hier gedefinieerd, zodat ze in xourse kunnen worden gezet/overschreven)
\providebool{parttoc}
\providebool{printpartfrontpage}
\providebool{printactivitytitle}
\providebool{printactivityqrcode}
\providebool{printactivityurl}
\providebool{printcontinuouspagenumbers}
\providebool{numberactivitiesbysubpart}
\providebool{addtitlenumber}
\providebool{addsectiontitlenumber}
\addtitlenumbertrue
\addsectiontitlenumbertrue

% The following three commands are hardcoded in xake, you can't create other commands like these, without adding them to xake as well
%  ( gebruikt in xourses om juiste soort titelpagina te krijgen voor verschillende ximera's )
\newcommand{\activitychapter}[2][]{
    {    
    \ifstrequal{#1}{notnumbered}{
        \addtitlenumberfalse
    }{}
    \typeout{ACTIVITYCHAPTER #2}   % logging
	\chapterstyle
	\activity{#2}
    }
}
\newcommand{\activitysection}[2][]{
    {
    \ifstrequal{#1}{notnumbered}{
        \addsectiontitlenumberfalse
    }{}
	\typeout{ACTIVITYSECTION #2}   % logging
	\sectionstyle
	\activity{#2}
    }
}
% Practices worden als activity getoond om de grote blokken te krijgen online
\newcommand{\practicesection}[2][]{
    {
    \ifstrequal{#1}{notnumbered}{
        \addsectiontitlenumberfalse
    }{}
    \typeout{PRACTICESECTION #2}   % logging
	\sectionstyle
	\activity{#2}
    }
}
\newcommand{\activitychapterlink}[3][]{
    {
    \ifstrequal{#1}{notnumbered}{
        \addtitlenumberfalse
    }{}
    \typeout{ACTIVITYCHAPTERLINK #3}   % logging
	\chapterstyle
	\activitylink[#1]{#2}{#3}
    }
}

\newcommand{\activitysectionlink}[3][]{
    {
    \ifstrequal{#1}{notnumbered}{
        \addsectiontitlenumberfalse
    }{}
    \typeout{ACTIVITYSECTIONLINK #3}   % logging
	\sectionstyle
	\activitylink[#1]{#2}{#3}
    }
}


% Commando om de printstyle toe te voegen in ximera's. Zorgt ervoor dat er geen problemen zijn als je de xourses compileert
% hack om onhandige relative paden in TeX te omzeilen
% should work both in xourse and ximera (pre-112022 only in ximera; thus obsoletes adhoc setup in xourses)
% loads global.sty if present (cfr global.css for online settings!)
% use global.sty to overwrite settings in printstyle.sty ...
\newcommand{\addPrintStyle}[1]{
\iftikzexport\else   % only in PDF
  \makeatletter
  \ifx\@onlypreamble\@notprerr\else   % ONLY if in tex-preamble   (and e.g. not when included from xourse)
    \typeout{Loading printstyle}   % logging
    \usepackage{#1/printstyle} % mag enkel geinclude worden als je die apart compileert
    \IfFileExists{#1/global.sty}{
        \typeout{Loading printstyle-folder #1/global.sty}   % logging
        \usepackage{#1/global}
        }{
        \typeout{Info: No extra #1/global.sty}   % logging
    }   % load global.sty if present
    \IfFileExists{global.sty}{
        \typeout{Loading local-folder global.sty (or TEXINPUTPATH..)}   % logging
        \usepackage{global}
    }{
        \typeout{Info: No folder/global.sty}   % logging
    }   % load global.sty if present
    \IfFileExists{\currfilebase.sty}
    {
        \typeout{Loading \currfilebase.sty}
        \input{\currfilebase.sty}
    }{
        \typeout{Info: No local \currfilebase.sty}
    }
    \fi
  \makeatother
\fi
}

%
%  
% references: Ximera heeft adhoc logica	 om online labels te doen werken over verschillende files heen
% met \hyperref kan de getoonde tekst toch worden opgegeven, in plaats van af te hangen van de label-text
\ifdefined\HCode
% Link to standard \labels, but give your own description
% Usage:  Volg \hyperref[my_very_verbose_label]{deze link} voor wat tijdverlies
%   (01/2020: Ximera-server aangepast om bij class reference-keeptext de link-text NIET te vervangen door de label-text !!!) 
\renewcommand{\hyperref}[2][]{\HCode{<a class="reference reference-keeptext" href="\##1">}#2\HCode{</a>}}
%
%  Link to specific targets  (not tested ?)
\renewcommand{\hypertarget}[1]{\HCode{<a class="ximera-label" id="#1"></a>}}
\renewcommand{\hyperlink}[2]{\HCode{<a class="reference reference-keeptext" href="\##1">}#2\HCode{</a>}}
\fi

% Mmm, quid English ... (for keyword #1 !) ?
\newcommand{\wikilink}[2]{
    \hyperlink{https://nl.wikipedia.org/wiki/#1}{#2}
    \pdfOnly{\footnote{See \url{https://nl.wikipedia.org/wiki/#1}}
    }
}

\renewcommand{\figurename}{Figuur}
\renewcommand{\tablename}{Tabel}

%
% Gedoe om verschillende versies van xourse/ximera te maken afhankelijk van settings
%
% default: versie met antwoorden
% handout: versie voor de studenten, zonder antwoorden/oplossingen
% full: met alles erop en eraan, dus geschikt voor auteurs en/of lesgevers  (bevat in de pdf ook de 'online-only' stukken!)
%
%
% verder kunnen ook opties/variabele worden gezet voor hints/auteurs/uitweidingen/ etc
%
% 'Full' versie
\newtoggle{showonline}
\ifdefined\HCode   % zet default showOnline
    \toggletrue{showonline} 
\else
    \togglefalse{showonline}
\fi

% Full versie   % deprecated: see infra
\newcommand{\printFull}{
    \hintstrue
    \handoutfalse
    \toggletrue{showonline} 
}

\ifdefined\shouldPrintFull   % deprecated: see infra
    \printFull
\fi



% Overschrijf onlineOnly  (zoals gedefinieerd in ximera.cls)
\ifhandout   % in handout: gebruik de oorspronkelijke ximera.cls implementatie  (is dit wel nodig/nuttig?)
\else
    \iftoggle{showonline}{%
        \ifdefined\HCode
          \RenewEnviron{onlineOnly}{\bgroup\BODY\egroup}   % showOnline, en we zijn  online, dus toon de tekst
        \else
          \RenewEnviron{onlineOnly}{\bgroup\color{red!50!black}\BODY\egroup}   % showOnline, maar we zijn toch niet online: kleur de tekst rood 
        \fi
    }{%
      \RenewEnviron{onlineOnly}{}  % geen showOnline
    }
\fi

% hack om na hoofding van definition/proposition/... als dan niet op een nieuwe lijn te starten
% soms is dat goed en mooi, en soms niet; en in HTML is het nu (2/2020) anders dan in pdf
% vandaar suggestie om 
%     \begin{definition}[Nieuw concept] \nl
% te gebruiken als je zeker een newline wil na de hoofdig en titel
% (in het bijzonder itemize zonder \nl is 'lelijk' ...)
\ifdefined\HCode
\newcommand{\nl}{}
\else
\newcommand{\nl}{\ \par} % newline (achter heading van definition etc.)
\fi


% \nl enkel in handoutmode (ihb voor \wordChoice, die dan typisch veeeel langer wordt)
\ifdefined\HCode
\providecommand{\handoutnl}{}
\else
\providecommand{\handoutnl}{%
\ifhandout%
  \nl%
\fi%
}
\fi

% Could potentially replace \pdfOnline/\begin{onlineOnly} : 
% Usage= \ifonline{Hallo surfer}{Hallo PDFlezer}
\providecommand{\ifonline}[2]%
{
\begin{onlineOnly}#1\end{onlineOnly}%
\pdfOnly{#2}
}%


%
% Maak optionele 'basic' en 'extended' versies van een activity
%  met environment basicOnly, basicSkip en extendedOnly
%
%  (
%   Dit werkt ENKEL in de PDF; de online versies tonen (minstens voorklopig) steeds 
%   het default geval met printbasicversion en printextendversion beide FALSE
%  )
%
\providebool{printbasicversion}
\providebool{printextendedversion}   % not properly implemented
\providebool{printfullversion}       % presumably print everything (debug/auteur)
%
% only set these in xourses, and BEFORE loading this preamble
%
%\newif\ifshowbasic     \showbasictrue        % use this line in xourse to show 'basic' sections
%\newif\ifshowextended  \showextendedtrue     % use this line in xourse to show 'extended' sections
%
%
%\ifbool{showbasic}
%      { \NewEnviron{basicOnly}{\BODY} }    % if yes: just print contents
%      { \NewEnviron{basicOnly}{}      }    % if no:  completely ignore contents
%
%\ifbool{showbasic}
%      { \NewEnviron{basicSkip}{}      }
%      { \NewEnviron{basicSkip}{\BODY} }
%

\ifbool{printextendedversion}
      { \NewEnviron{extendedOnly}{\BODY} }
      { \NewEnviron{extendedOnly}{}      }
      


\ifdefined\HCode    % in html: always print
      {\newenvironment*{basicOnly}{}{}}    % if yes: just print contents
      {\newenvironment*{basicSkip}{}{}}    % if yes: just print contents
\else

\ifbool{printbasicversion}
      {\newenvironment*{basicOnly}{}{}}    % if yes: just print contents
      {\NewEnviron{basicOnly}{}      }    % if no:  completely ignore contents

\ifbool{printbasicversion}
      {\NewEnviron{basicSkip}{}      }
      {\newenvironment*{basicSkip}{}{}}

\fi

\usepackage{float}
\usepackage[rightbars,color]{changebar}

% Full versie
\ifbool{printfullversion}{
    \hintstrue
    \handoutfalse
    \toggletrue{showonline}
    \printbasicversionfalse
    \cbcolor{red}
    \renewenvironment*{basicOnly}{\cbstart}{\cbend}
    \renewenvironment*{basicSkip}{\cbstart}{\cbend}
    \def\xmtoonprintopties{FULL}   % will be printed in footer
}
{}
      
%
% Evalueer \ifhints IN de environment
%  
%
%\RenewEnviron{hint}
%{
%\ifhandout
%\ifhints\else\setbox0\vbox\fi%everything in een emty box
%\bgroup 
%\stepcounter{hintLevel}
%\BODY
%\egroup\ignorespacesafterend
%\addtocounter{hintLevel}{-1}
%\else
%\ifhints
%\begin{trivlist}\item[\hskip \labelsep\small\slshape\bfseries Hint:\hspace{2ex}]
%\small\slshape
%\stepcounter{hintLevel}
%\BODY
%\end{trivlist}
%\addtocounter{hintLevel}{-1}
%\fi
%\fi
%}

% Onafhankelijk van \ifhandout ...? TO BE VERIFIED
\RenewEnviron{hint}
{
\ifhints
\begin{trivlist}\item[\hskip \labelsep\small\bfseries Hint:\hspace{2ex}]
\small%\slshape
\stepcounter{hintLevel}
\BODY
\end{trivlist}
\addtocounter{hintLevel}{-1}
\else
\iftikzexport   % anders worden de tikz tekeningen in hints niet gegenereerd ?
\setbox0\vbox\bgroup
\stepcounter{hintLevel}
\BODY
\egroup\ignorespacesafterend
\addtocounter{hintLevel}{-1}
\fi % ifhandout
\fi %ifhints
}

%
% \tab sets typewriter-tabs (e.g. to format questions)
% (Has no effect in HTML :-( ))
%
\usepackage{tabto}
\ifdefined\HCode
  \renewcommand{\tab}{\quad}    % otherwise dummy .png's are generated ...?
\fi


% Also redefined in  preamble to get correct styling 
% for tikz images for (\tikzexport)
%

\theoremstyle{definition} % Bold titels
\makeatletter
\let\proposition\relax
\let\c@proposition\relax
\let\endproposition\relax
\makeatother
\newtheorem{proposition}{Eigenschap}


%\instructornotesfalse

% logic with \ifhandoutin ximera.cls unclear;so overwrite ...
\makeatletter
\@ifundefined{ifinstructornotes}{%
  \newif\ifinstructornotes
  \instructornotesfalse
  \newenvironment{instructorNotes}{}{}
}{}
\makeatother
\ifinstructornotes
\else
\renewenvironment{instructorNotes}%
{%
    \setbox0\vbox\bgroup
}
{%
    \egroup
}
\fi

% \RedeclareMathOperator
% from https://tex.stackexchange.com/questions/175251/how-to-redefine-a-command-using-declaremathoperator
\makeatletter
\newcommand\RedeclareMathOperator{%
    \@ifstar{\def\rmo@s{m}\rmo@redeclare}{\def\rmo@s{o}\rmo@redeclare}%
}
% this is taken from \renew@command
\newcommand\rmo@redeclare[2]{%
    \begingroup \escapechar\m@ne\xdef\@gtempa{{\string#1}}\endgroup
    \expandafter\@ifundefined\@gtempa
    {\@latex@error{\noexpand#1undefined}\@ehc}%
    \relax
    \expandafter\rmo@declmathop\rmo@s{#1}{#2}}
% This is just \@declmathop without \@ifdefinable
\newcommand\rmo@declmathop[3]{%
    \DeclareRobustCommand{#2}{\qopname\newmcodes@#1{#3}}%
}
\@onlypreamble\RedeclareMathOperator
\makeatother


%
% Engelse vertaling, vooral in mathmode
%
% 1. Algemene procedure
%
\ifdefined\isEn
 \newcommand{\nlen}[2]{#2}
 \newcommand{\nlentext}[2]{\text{#2}}
 \newcommand{\nlentextbf}[2]{\textbf{#2}}
\else
 \newcommand{\nlen}[2]{#1}
 \newcommand{\nlentext}[2]{\text{#1}}
 \newcommand{\nlentextbf}[2]{\textbf{#1}}
\fi

%
% 2. Lijst van erg veel gebruikte uitdrukkingen
%

% Ja/Nee/Fout/Juits etc
%\newcommand{\TJa}{\nlentext{ Ja }{ and }}
%\newcommand{\TNee}{\nlentext{ Nee }{ No }}
%\newcommand{\TJuist}{\nlentext{ Juist }{ Correct }
%\newcommand{\TFout}{\nlentext{ Fout }{ Wrong }
\newcommand{\TWaar}{\nlentext{ Waar }{ True }}
\newcommand{\TOnwaar}{\nlentext{ Vals }{ False }}
% Korte bindwoorden en, of, dus, ...
\newcommand{\Ten}{\nlentext{ en }{ and }}
\newcommand{\Tof}{\nlentext{ of }{ or }}
\newcommand{\Tdus}{\nlentext{ dus }{ so }}
\newcommand{\Tendus}{\nlentext{ en dus }{ and thus }}
\newcommand{\Tvooralle}{\nlentext{ voor alle }{ for all }}
\newcommand{\Took}{\nlentext{ ook }{ also }}
\newcommand{\Tals}{\nlentext{ als }{ when }} %of if?
\newcommand{\Twant}{\nlentext{ want }{ as }}
\newcommand{\Tmaal}{\nlentext{ maal }{ times }}
\newcommand{\Toptellen}{\nlentext{ optellen }{ add }}
\newcommand{\Tde}{\nlentext{ de }{ the }}
\newcommand{\Thet}{\nlentext{ het }{ the }}
\newcommand{\Tis}{\nlentext{ is }{ is }} %zodat is in text staat in mathmode (geen italics)
\newcommand{\Tmet}{\nlentext{ met }{ where }} % in situaties e.g met p < n --> where p < n
\newcommand{\Tnooit}{\nlentext{ nooit }{ never }}
\newcommand{\Tmaar}{\nlentext{ maar }{ but }}
\newcommand{\Tniet}{\nlentext{ niet }{ not }}
\newcommand{\Tuit}{\nlentext{ uit }{ from }}
\newcommand{\Ttov}{\nlentext{ t.o.v. }{ w.r.t. }}
\newcommand{\Tzodat}{\nlentext{ zodat }{ such that }}
\newcommand{\Tdeth}{\nlentext{de }{th }}
\newcommand{\Tomdat}{\nlentext{omdat }{because }} 


%
% Overschrijf addhoc commando's
%
\ifdefined\isEn
\renewcommand{\pernot}{\overset{\mathrm{notation}}{=}}
\RedeclareMathOperator{\bld}{im}     % beeld
\RedeclareMathOperator{\graf}{graph}   % grafiek
\RedeclareMathOperator{\rico}{slope}   % richtingcoëfficient
\RedeclareMathOperator{\co}{co}       % coordinaat
\RedeclareMathOperator{\gr}{deg}       % graad

% Operators
\RedeclareMathOperator{\bgsin}{arcsin}
\RedeclareMathOperator{\bgcos}{arccos}
\RedeclareMathOperator{\bgtan}{arctan}
\RedeclareMathOperator{\bgcot}{arccot}
\RedeclareMathOperator{\bgsinh}{arcsinh}
\RedeclareMathOperator{\bgcosh}{arccosh}
\RedeclareMathOperator{\bgtanh}{arctanh}
\RedeclareMathOperator{\bgcoth}{arccoth}

\fi


% HACK: use 'oplossing' for 'explanation' ...
\let\explanation\relax
\let\endexplanation\relax
% \newenvironment{explanation}{\begin{oplossing}}{\end{oplossing}}
\newcounter{explanation}

\ifhandout%
    \NewEnviron{explanation}[1][toon]%
    {%
    \RenewEnviron{verbatim}{ \red{VERBATIM CONTENT MISSING IN THIS PDF}} %% \expandafter\verb|\BODY|}

    \ifthenelse{\equal{\detokenize{#1}}{\detokenize{toon}}}
    {
    \def\PH@Command{#1}% Use PH@Command to hold the content and be a target for "\expandafter" to expand once.

    \begin{trivlist}% Begin the trivlist to use formating of the "Feedback" label.
    \item[\hskip \labelsep\small\slshape\bfseries Explanation:% Format the "Feedback" label. Don't forget the space.
    %(\texttt{\detokenize\expandafter{\PH@Command}}):% Format (and detokenize) the condition for feedback to trigger
    \hspace{2ex}]\small%\slshape% Insert some space before the actual feedback given.
    \BODY
    \end{trivlist}
    }
    {  % \begin{feedback}[solution]   \BODY     \end{feedback}  }
    }
    }    
\else
% ONLY for HTML; xmoplossing is styled with css, and is not, and need not be a LaTeX environment
% THUS: it does NOT use feedback anymore ...
%    \NewEnviron{oplossing}{\begin{expandable}{xmoplossing}{\nlen{Toon uitwerking}{Show solution}}{\BODY}\end{expandable}}
    \newenvironment{explanation}[1][toon]
   {%
       \begin{expandable}{xmoplossing}{}
   }
   {%
   	   \end{expandable}
   } 
\fi

\title{Tedious Proofs Concerning Determinants} \license{CC BY-NC-SA 4.0}

\begin{document}

\begin{abstract}

\end{abstract}
\maketitle


\section*{Tedious Proofs Concerning Determinants}
In \href{\xmbaseurl/DET-0010/main}{Finding the Determinant} we described the determinant as a function that assigns a scalar to every square matrix.  The value of the function in the original definition was given by cofactor expansion along the first row of the matrix.  We also observed, through examples, that cofactor expansion along any row or column produces the same value.  Examples, however, do not constitute a sufficient proof of equivalency of different cofactor expansions.  In this section we will prove that cofactor expansions along any row or column produce the same outcome.  This result is known as the Laplace Expansion Theorem.  We will also prove several results concerning elementary row operations.

\subsection*{Cofactor Expansion Along the Top Row}
Let 
$$A=\begin{bmatrix}a&b&c\\d&e&f\\g&h&i\end{bmatrix}$$

Definition \ref{def:threebythreedet} of cofactor expansion along the top row for a $3\times 3$ matrix requires three \dfn{minor} matrices associated with $A$.  
\begin{itemize}
\item $A_{11}$ is obtained from $A$ by deleting the first row and the first column of $A$.
\begin{center}
\begin{tikzpicture}
  \matrix (m)[
    matrix of math nodes,
    nodes in empty cells,
    left delimiter={[},right delimiter={]},minimum width=width("a"),minimum height=height("b")] {
    a    & b  & c  \\
    d & e   & f   \\
    g   & h    & i     \\
  } ;

  \draw (m-3-2.south west) rectangle (m-2-3.north east);
  %\draw[blue](m-1-2.west) -- (m-1-3.east);
  %\draw[blue](m-2-1.north) -- (m-3-1.south);
 \end{tikzpicture}
 \end{center} 
 $$A_{11}=\begin{bmatrix}e&f\\h&i\end{bmatrix}$$
 \item $A_{12}$ is obtained from $A$ by deleting the first row and the second column of $A$.
 \begin{center}
\begin{tikzpicture}
  \matrix (m)[
    matrix of math nodes,
    nodes in empty cells,
    left delimiter={[},right delimiter={]},minimum width=width("a")] {
    a    & b  & c  \\
    d & e   & f   \\
    g   & h    & i     \\
  } ;
\draw (m-3-1.south west) rectangle (m-2-1.north east);
  \draw (m-3-3.south west) rectangle (m-2-3.north east);
 \end{tikzpicture}
 \end{center} 
 $$A_{12}=\begin{bmatrix}d&f\\g&i\end{bmatrix}$$
 \item $A_{13}$ is obtained from $A$ by deleting the first row and the third column of $A$.
 \begin{center}
\begin{tikzpicture}
  \matrix (m)[
    matrix of math nodes,
    nodes in empty cells,
    left delimiter={[},right delimiter={]},minimum width=width("a")] {
    a    & b  & c  \\
    d & e   & f   \\
    g   & h    & i     \\
  } ;
\draw (m-3-1.south west) rectangle (m-2-2.north east);
  
 \end{tikzpicture}
 \end{center} 
 $$A_{13}=\begin{bmatrix}d&e\\g&h\end{bmatrix}$$
\end{itemize}

The determinant of $A$ is given by
\begin{align*}\det{A}=|A|&=a\begin{vmatrix}e&f\\h&i\end{vmatrix}-b\begin{vmatrix}d&f\\g&i\end{vmatrix}+c\begin{vmatrix}d&e\\g&h\end{vmatrix}\\
&=a\big(\det{A_{11}}\big)-b\big(\det{A_{12}}\big)+c\big(\det{A_{13}}\big)
\end{align*}

Now we are ready for an $n\times n$ matrix.  Let 
$$A=\begin{bmatrix}a_{11} & a_{12} & \dots & a_{1j-1} & a_{1j} & a_{1j+1} & \dots & a_{1n}  \\
    a_{21} & a_{22} & \dots & a_{2j-1} & a_{2j} & a_{2j+1} & \dots & a_{2n}  \\
   \vdots & \vdots &  & \vdots & \vdots & \vdots &  & \vdots  \\
   a_{n1} & a_{n2} & \dots & a_{nj-1} & a_{nj} & a_{nj+1} & \dots & a_{nn}\end{bmatrix}$$
   Define $A_{1j}$ to be an $(n-1)\times (n-1)$ matrix obtained from $A$ by deleting the first row and the $j^{th}$ column of $A$.  We say that $A_{1j}$ is the \dfn{$(1, j)$-minor} of $A$.
\begin{center}
\begin{tikzpicture}
  \matrix (m)[
    matrix of math nodes,
    nodes in empty cells,
    left delimiter={[},right delimiter={]},minimum width=width("a22")] {
    a_{11} & a_{12} & \dots & a_{1j-1} & a_{1j} & a_{1j+1} & \dots & a_{1n}  \\
    a_{21} & a_{22} & \dots & a_{2j-1} & a_{2j} & a_{2j+1} & \dots & a_{2n}  \\
   \vdots & \vdots &  & \vdots & \vdots & \vdots &  & \vdots  \\
   a_{n1} & a_{n2} & \dots & a_{nj-1} & a_{nj} & a_{nj+1} & \dots & a_{nn}  \\
  } ;
\draw (m-4-1.south west) rectangle (m-2-4.north east);
\draw (m-4-6.south west) rectangle (m-2-8.north east);
 
 %\draw[blue](m-4-5.south) -- (m-2-5.north); 
 %\draw[blue](m-1-1.west) -- (m-1-4.east);
 %\draw[blue](m-1-6.west) -- (m-1-8.east);
 \end{tikzpicture}
 \end{center} 
For a $3\times 3$ matrix $A$ we have
$$\det{A}=a_{11}\big(\det{A_{11}}\big)-a_{12}\big(\det{A_{12}}\big)+a_{13}\big(\det{A_{13}}\big)$$
We want to follow the same pattern  to define the determinant of a larger matrix.  A distinct feature of this expression is the alternating sign pattern.  We want to preserve this feature as we increase matrix size. 

\begin{definition}\label{def:toprowexpansion}  Let $A=\begin{bmatrix}a_{ij}\end{bmatrix}$ be an $n\times n$ matrix.  Define the \dfn{determinant} of $A$ by
\begin{align*}\det{A}=(-1)^{1+1}a_{11}\det{A_{11}}&+(-1)^{1+2}a_{12}\det{A_{12}}+\ldots \\
\ldots &+(-1)^{1+j}a_{1j}\det{A_{1j}}+\ldots \\
\ldots &+(-1)^{1+n}a_{1n}\det{A_{1n}}\\
=\sum_{j=1}^n(-1)^{1+j}a_{1j}\det{A_{1j}}
\end{align*}
\end{definition}
Naturally, we would like to condense this formula.  To accomplish this, let
$$C_{1j}=(-1)^{1+j}\det{A_{1j}}$$
We will refer to $C_{1j}$ as the \dfn{$(1,j)$-cofactor of $A$}.  When we use the cofactor notation, the expression in Definition \ref{def:toprowexpansion} turns into the following:
$$\det{A}=a_{11}C_{11}+a_{12}C_{12}+\ldots +a_{1n}C_{1n}=\sum_{j=1}^n a_{1j}C_{1j}$$

This process of computing the determinant is called the \dfn{cofactor expansion along the first row}.  


\subsection*{Cofactor Expansion Along the First Column}
As we have observed in several examples in \href{\xmbaseurl/DET-0010/main}{Finding the Determinant}, cofactor expansion along the first column produces the same result as cofactor expansion along the top row.  We will now formalize the process of cofactor expansion along the first column for an $n\times n$ matrix and prove that this process produces the same result as our original definition of the determinant.
Let $A$ be an $n\times n$ matrix. 
$$A=\begin{bmatrix}a_{11} & a_{12} & \dots  & a_{1n}  \\
    a_{21} & a_{22} &\dots  & a_{2n}  \\
   \vdots & \vdots &  & \vdots \\
   a_{i1} & a_{i2} & \dots  & a_{in}\\
   \vdots & \vdots &  & \vdots  \\
   a_{n1} & a_{n2} & \dots  & a_{nn}\end{bmatrix}$$
   Define $A_{i1}$ to be an $(n-1)\times (n-1)$ matrix obtained from $A$ by deleting the first column and the $i^{th}$ row of $A$.  We say that $A_{i1}$ is the \dfn{$(i, 1)$-minor} of $A$.
\begin{center}
\begin{tikzpicture}
  \matrix (m)[
    matrix of math nodes,
    nodes in empty cells,
    left delimiter={[},right delimiter={]},minimum width=width("a22")] {
    a_{11} & a_{12} & \dots  & a_{1n}  \\
    a_{21} & a_{22} &\dots  & a_{2n}  \\
   \vdots & \vdots &  & \vdots \\
   a_{i1} & a_{i2} & \dots  & a_{in}\\
   \vdots & \vdots &  & \vdots  \\
   a_{n1} & a_{n2} & \dots  & a_{nn}\\
  } ;
\draw (m-1-2.north west) rectangle (m-3-4.south east);
\draw (m-6-2.south west) rectangle (m-5-4.north east);
 \end{tikzpicture}
 \end{center} 
Define $C_{i1}=(-1)^{i+1}\det{A_{i1}}$ to be the
 \dfn{$(i,1)$-cofactor of $A$}.

\begin{definition}\label{def:firstcolexpansion1}  Let $A=\begin{bmatrix}a_{ij}\end{bmatrix}$ be an $n\times n$ matrix.  Define the \dfn{determinant} of $A$ by
\begin{align*}\det{A}=(-1)^{1+1}a_{11}\det{A_{11}}&+(-1)^{2+1}a_{21}\det{A_{21}}+\ldots \\
\ldots &+(-1)^{i+1}a_{i1}\det{A_{i1}}+\ldots \\
\ldots &+(-1)^{n+1}a_{n1}\det{A_{n1}}\\
=\sum_{i=1}^n(-1)^{i+1}a_{i1}\det{A_{i1}}
\end{align*}
or
$$\det{A}=a_{11}C_{11}+a_{21}C_{21}+\ldots +a_{n1}C_{n1}=\sum_{i=1}^n a_{i1}C_{i1}$$
\end{definition}

\subsubsection*{Proof of Definition Equivalence}
We will now show that cofactor expansion along the first row produces the same result as cofactor expansion along the first column.
\begin{theorem}\label{th:rowcolexpequivalence}
Let $A=\begin{bmatrix}a_{ij}\end{bmatrix}$ be an $n\times n$ matrix.  Then
$$\sum_{j=1}^n(-1)^{1+j}a_{1j}\det{A_{1j}}=\sum_{i=1}^n(-1)^{i+1}a_{i1}\det{A_{i1}}$$
\end{theorem}
\begin{proof}
We will proceed by induction on $n$.  Clearly, the result holds for $n=1$.  Just for practice you should also verify the equality for $n=2, 3$. (See Practice Problem \ref{prob:extrainductionsteps}.)  We will assume that the result holds for $(n-1)\times (n-1)$ matrices and show that it must hold for $n\times n$ matrices.

You will find the following matrix a useful reference as we proceed.

$$A=\begin{bmatrix}a_{11} & a_{12} & \dots &  a_{1j} &  \dots & a_{1n}  \\
    a_{21} & a_{22} & \dots &  a_{2j} &  \dots & a_{2n}  \\
   \vdots & \vdots &  & \vdots &   & \vdots  \\
  a_{i1} & a_{i2} & \ldots &  a_{ij} &  \ldots & a_{in}\\
  \vdots & \vdots &  & \vdots &   & \vdots  \\
   a_{n1} & a_{n2} & \dots &  a_{nj} &  \dots & a_{nn}\end{bmatrix}$$
   For convenience, we will refer to the Right-Hand Side (RHS) and the Left-Hand Side (LHS) of the equality we are trying to prove.
   $$\text{LHS}=\sum_{j=1}^n(-1)^{1+j}a_{1j}\det{A_{1j}}\overset{?}{=}\sum_{i=1}^n(-1)^{i+1}a_{i1}\det{A_{i1}}=\text{RHS}$$
   
   Note that the first term $a_{11}(-1)^{1+1}\det{A_{11}}$ is the same for LHS and RHS, so we will only need to consider $i,j\geq 2$.
   
We will start by analyzing RHS.  Consider an arbitrary entry $a_{i1}$ of the fist column.  This entry will only appear in the term $a_{i1}(-1)^{i+1}\det{A_{i1}}$. We will find $\det{A_{i1}}$ by cofactor expansion along the first row.  As we proceed, we have to pay special attention to the subscripts.  Because the first column of $A$ was removed, the $j^{th}$ column of $A$ contains the $(j-1)$ column of $A_{i1}$.  
\begin{align*}
&a_{i1}(-1)^{i+1}\det{A_{i1}}=\\
=&a_{i1}(-1)^{i+1}\Big(a_{12}(-1)^{1+1}\det{(A_{i1})_{11}}+\ldots +a_{1j}(-1)^{1+(j-1)}\det{(A_{i1})_{1(j-1)}}+\ldots \Big)
\end{align*}
Note that the entry $a_{1j}$ will only appear in the term $$a_{1j}(-1)^{1+(j-1)}\det{(A_{i1})_{1(j-1)}}$$
So, after we distribute $a_{i1}(-1)^{i+1}$, RHS will contain only one term of the form
$$a_{i1}a_{1j}(-1)^{p+q}\det{(A_{st})_{pq}}$$
We will perform a similar analysis on LHS.  Consider an arbitrary entry $a_{1j}$ of the fist row.  This entry will only appear in the term $a_{1j}(-1)^{1+j}\big(\det{A_{1j}}\big)$. Invoking the induction hypothesis, we will find $\det{A_{1j}}$ by cofactor expansion along the first column.
\begin{align*}
&a_{1j}(-1)^{1+j}\det{A_{1j}}=\\
=&a_{1j}(-1)^{1+j}\Big(a_{21}(-1)^{1+1}\det{(A_{1j})_{11}}+\ldots +a_{i1}(-1)^{(i-1)+1}\det{(A_{1j})_{(i-1)1}}+\ldots \Big)
\end{align*}
The entry $a_{i1}$ will only appear in the term $$a_{i1}(-1)^{(i-1)+1}\det{(A_{1j})_{(i-1)1}}$$
So, after we distribute $a_{1j}(-1)^{1+j}$, LHS will contain only one term of the form
$$a_{i1}a_{1j}(-1)^{p+q}\det{(A_{st})_{pq}}$$
But RHS also has only one term of this form.  We now need to show that these two terms are equal.  The two terms are
\begin{align*}a_{i1}(-1)^{i+1}a_{1j}(-1)^{1+(j-1)}\det{(A_{i1})_{1(j-1)}}=a_{i1}a_{1j}(-1)^{i+j+1}\det{(A_{i1})_{1(j-1)}}\end{align*}
and 
\begin{align*}a_{1j}(-1)^{1+j}a_{i1}(-1)^{(i-1)+1}\det{(A_{1j})_{(i-1)1}}=a_{i1}a_{1j}(-1)^{i+j+1}\det{(A_{1j})_{(i-1)1}}\end{align*}
Observe that $(A_{i1})_{1(j-1)}$ and $(A_{1j})_{(i-1)1}$ are the same matrix because both were obtained from matrix $A$ by deleting the first and the $i^{th}$ rows of $A$, and the first and the $j^{th}$ columns of $A$.  Therefore $\det{(A_{i1})_{1(j-1)}}=\det{(A_{1j})_{(i-1)1}}$.


We conclude that the terms of LHS and RHS match.  This establishes the desired equality.
\end{proof}

Now we know that cofactor expansion along the first row and cofactor expansion along the first column produce the same result, so either expansion can be used to find the determinant.  

\subsection*{Proof of Results Concerning Elementary Row Operations}
In \href{\xmbaseurl/DET-0030/main}{Elementary Row Operations} we observed, without proof, the following properties of the determinant. (See Theorem \ref{th:elemrowopsanddet}.)
\begin{summary}
    Let $A=\begin{bmatrix}a_{ij}\end{bmatrix}$ be an $n\times n$ matrix.  
\begin{enumerate}
\item\label{item:rowswapanddetSUMM}
If $B$ is obtained from $A$ by interchanging two different rows, then $$\det{B}=-\det{A}$$
\item \label{item:rowconstantmultanddetSUMM}
If $B$ is obtained from $A$ by multiplying one of the rows of $A$ by a non-zero constant $k$.  Then $$\det{B}=k\det{A}$$
\item \label{item:addmultotherrowdetSUMM}
If $B$ is obtained from $A$ by adding a multiple of one row of $A$ to another row, then
$$\det{B}=\det{A}$$
\end{enumerate}
\end{summary}

We now prove these properties.
\begin{proof}[Proof of Theorem \ref{th:elemrowopsanddet}\ref{item:rowswapanddetSUMM}] We will start by showing that the result holds if two consecutive rows are interchanged.  Suppose $B$ is obtained from $A$ by swapping rows $p$ and $p+1$ of $A$. 

We proceed by induction on $n$.  The result is not applicable for $n=1$.  In Practice Problem \ref{prob:proofofrowswapanddet}, you will be asked to verify that the result holds for $2\times 2$ matrices.  Suppose that the result holds for $(n-1)\times (n-1)$ matrices.  We need to show that it holds for $n\times n$ matrices.
You may find the following diagram useful throughout the proof.
\begin{center}
\begin{tikzpicture}
  \matrix (mA)[
    matrix of math nodes,
    nodes in empty cells,
    left delimiter={[},right delimiter={]},minimum width=width("a22")] {
    a_{11} & a_{12} & \dots  & a_{1n}  \\
   \vdots & \vdots &  & \vdots \\
   a_{p1} & a_{p2} &\dots  & a_{pn}  \\
   a_{(p+1)1} & a_{(p+1)2} & \dots  & a_{(p+1)n}\\
   \vdots & \vdots &  & \vdots  \\
   a_{n1} & a_{n2} & \dots  & a_{nn}\\
  } ;
  
  \matrix (mB)[
    matrix of math nodes,
    nodes in empty cells,
    left delimiter={[},right delimiter={]},minimum width=width("a22")] at ($(mA.east)+(4,0)$)
    {
    a_{11} & a_{12} & \dots  & a_{1n}  \\
   \vdots & \vdots &  & \vdots \\
      a_{(p+1)1} & a_{(p+1)2} & \dots  & a_{(p+1)n}\\
      a_{p1} & a_{p2} &\dots  & a_{pn}  \\
   \vdots & \vdots &  & \vdots  \\
   a_{n1} & a_{n2} & \dots  & a_{nn}\\
  } ;
\draw [->](mA-3-4.east) -- (mB-4-1.west);
\draw [->](mA-4-4.east) -- (mB-3-1.west);
 \end{tikzpicture}
 \end{center} 
 Observe that for $i\neq p, p+1$ we have: $$b_{ij}=a_{ij}$$ Because $B_{ij}$ is obtained from $A_{ij}$ by switching two rows of $A_{ij}$, our induction hypothesis give us:
 $$\det{B_{ij}}=-\det{A_{ij}}$$
 
 For $i=p$ and $i=p+1$ we have:
 $$b_{p1}=a_{(p+1)1}\quad\text{and}\quad b_{(p+1)1}=a_{p1}$$
 $$B_{p1}=A_{(p+1)1}\quad\text{and}\quad B_{(p+1)1}=A_{p1}$$
 
 We compute the determinant of $B$ by cofactor expansion along the first column.
 \begin{align*}
 \det{B}&=b_{11}(-1)^{1+1}\det{B_{11}}+\ldots +b_{i1}(-1)^{i+1}\det{B_{i1}}+\ldots\\
 &+b_{p1}(-1)^{p+1}\det{B_{p1}}+b_{(p+1)1}(-1)^{(p+1)+1}\det{B_{(p+1)1}}+\ldots\\
 &+b_{n1}(-1)^{n+1}\det{B_{n1}}\\
 \\
 &=a_{11}(-1)^{1+1}(-1)\det{A_{11}}+\ldots +a_{i1}(-1)^{i+1}(-1)\det{A}_{i1}+\ldots\\
 &+a_{(p+1)1}(-1)^{p+1}\det{A_{(p+1)1}}+a_{p1}(-1)^{(p+1)+1}\det{A_{p1}}+\ldots\\
 &+a_{n1}(-1)^{n+1}(-1)\det{A_{n1}}\\
 \\
 &=a_{11}(-1)^{1+1}(-1)\det{A_{11}}+\ldots +a_{i1}(-1)^{i+1}(-1)\det{A_{i1}}+\ldots\\
 &+a_{(p+1)1}(-1)^{(p+1)+1}(-1)\det{A_{(p+1)1}}+a_{p1}(-1)^{p+1}(-1)\det{A_{p1}}+\ldots\\
 &+a_{n1}(-1)^{n+1}(-1)\det{A_{n1}}\\
 \\
 &=(-1)\Big(a_{11}(-1)^{1+1}\det{A_{11}}+\ldots +a_{i1}(-1)^{i+1}\det{A_{i1}}+\ldots\\
 &+a_{(p+1)1}(-1)^{(p+1)+1}\det{A_{(p+1)1}}+a_{p1}(-1)^{p+1}\det{A_{p1}}+\ldots\\
 &+a_{n1}(-1)^{n+1}\det{A_{n1}}\Big)\\
 \\
 &=-\det{A}
 \end{align*}
 
 If two non-adjacent rows are switched, then the switch can be carried out by performing an odd number of adjacent row interchanges (See Practice Problem \ref{prob:numberofrowswitches}), so the result still holds.
\end{proof}

\begin{proof}[Proof of Theorem \ref{th:elemrowopsanddet}\ref{item:rowconstantmultanddetSUMM}]
We proceed by induction on $n$.  Clearly the statement is true for $n=1$.  Just for fun, you might want to verify directly that it holds for $2\times 2$ matrices.  Now suppose the statement is true for all $(n-1)\times (n-1)$ matrices. We will show that it holds for $n\times n$ matrices.

Suppose $B$ is obtained from $A$ by multiplying the $p$'s row of $A$ by $k$.  

$$A=\begin{bmatrix} a_{11} & a_{12} & \dots  & a_{1n}  \\
   \vdots & \vdots &  & \vdots \\
   a_{p1} & a_{p2} &\dots  & a_{pn}  \\
   \vdots & \vdots &  & \vdots  \\
   a_{n1} & a_{n2} & \dots  & a_{nn}\end{bmatrix}\quad
   B=\begin{bmatrix} a_{11} & a_{12} & \dots  & a_{1n}  \\
   \vdots & \vdots &  & \vdots \\
   ka_{p1} & ka_{p2} &\dots  & ka_{pn}  \\
   \vdots & \vdots &  & \vdots  \\
   a_{n1} & a_{n2} & \dots  & a_{nn}\end{bmatrix}$$
   
   We compute the determinant of $B$ by cofactor expansion along the first column.
   \begin{align*}
   \det B &=a_{11}(-1)^{1+1}\det B_{11}+\ldots +ka_{p1}(-1)^{p+1}\det B_{p1}+\ldots \\
   &+ a_{n1}(-1)^{n+1}\det B_{n1}\\
   \\
   &=ka_{11}(-1)^{1+1}\det A_{11}+\ldots +ka_{p1}(-1)^{p+1}\det A_{p1}+\ldots \\
   &+ ka_{n1}(-1)^{n+1}\det A_{n1}\\
   \\
   &=k\Big(a_{11}(-1)^{1+1}\det A_{11}+\ldots +a_{p1}(-1)^{p+1}\det A_{p1}+\ldots \\
   &+ a_{n1}(-1)^{n+1}\det A_{n1}\Big)\\
   \\
   &=k\det A
   \end{align*}
\end{proof}

Before we tackle the proof of Part \ref{item:addmultotherrowdetSUMM} of Theorem \ref{th:elemrowopsanddet} we will need to prove the following lemma.




\begin{lemma}\label{lemma:arowsumofbc}
Let $A$, $B$ and $C$ be $n\times n$ matrices. Suppose $A$, $B$ and $C$ are identical, except for the $p^{th}$ row.  If the $p^{th}$ row of $C$ is the sum of the $p^{th}$ rows of $A$ and $B$, then
$$\det{C}=\det{A}+\det{B}$$
\end{lemma}
\begin{proof}
We will proceed by induction on $n$.  We leave it to the reader to verify cases $n=1, 2$.  We will assume that the statement holds for all $(n-1)\times (n-1)$ matrices and show that it holds for $n\times n$ matrices.

You may find the following representations of $A$, $B$ and $C$ helpful.  Identical entries in $A$, $B$ and $C$ are labeled $d_{ij}$.

$$A=\begin{bmatrix} d_{11} & d_{12} & \dots  & d_{1n}  \\
   \vdots & \vdots &  & \vdots \\
   a_{p1} & a_{p2} &\dots  & a_{pn}  \\
   \vdots & \vdots &  & \vdots  \\
   d_{n1} & d_{n2} & \dots  & d_{nn}\end{bmatrix}\quad
   B=\begin{bmatrix} d_{11} & d_{12} & \dots  & d_{1n}  \\
   \vdots & \vdots &  & \vdots \\
   b_{p1} & b_{p2} &\dots  & b_{pn}  \\
   \vdots & \vdots &  & \vdots  \\
   d_{n1} & d_{n2} & \dots  & d_{nn}\end{bmatrix}$$
   $$C=\begin{bmatrix} d_{11} & d_{12} & \dots  & d_{1n}  \\
   \vdots & \vdots &  & \vdots \\
   a_{p1}+b_{p1} & a_{p2}+b_{p2} &\dots  & a_{pn}+b_{pn}  \\
   \vdots & \vdots &  & \vdots  \\
   d_{n1} & d_{n2} & \dots  & d_{nn}\end{bmatrix}$$
Observe that $$A_{p1}=B_{p1}=C_{p1}$$  For $i\neq p$, the induction hypothesis gives us
$$\det{C_{i1}}=\det{A_{i1}}+\det{B_{i1}}$$
   
We now compute the determinant of $C$ by cofactor expansion along the first column.
\begin{align*}
\det{C}&=d_{11}(-1)^{1+1}\det{C_{11}}+\ldots +(a_{p1}+b_{p1})(-1)^{p+1}\det{C_{p1}}+\ldots\\
&+d_{n1}(-1)^{n+1}\det{C_{n1}}\\
\\
&=d_{11}(-1)^{1+1}\Big(\det{A_{11}}+\det{B_{11}}\Big)+\ldots\\
&+a_{p1}(-1)^{p+1}\det{C_{p1}}+b_{p1}(-1)^{p+1}\det{C_{p1}}+\ldots\\
&+d_{n1}(-1)^{n+1}\Big(\det{A_{n1}}+\det{B_{n1}}\Big)\\
\\
&=d_{11}(-1)^{1+1}\det{A_{11}}+d_{11}(-1)^{1+1}\det{B_{11}}+\ldots\\
&+a_{p1}(-1)^{p+1}\det{A_{p1}}+b_{p1}(-1)^{p+1}\det{B_{p1}}+\ldots\\
&+d_{n1}(-1)^{n+1}\det{A_{n1}}+d_{n1}(-1)^{n+1}\det{B_{n1}}\\
\\
&=\det{A}+\det{B}
\end{align*}
\end{proof}

We are now ready to finish the proof of Theorem \ref{th:elemrowopsanddet}
\begin{proof}[Proof of Theorem \ref{th:elemrowopsanddet}\ref{item:addmultotherrowdetSUMM}]
Suppose $B$ is obtained from $A$ by adding $k$ times row $p$ to row $q$.  ($p\neq q$ and $k\neq 0$)  You may find the following representations of $A$ and $B$ useful.
$$A=\begin{bmatrix} a_{11} & a_{12} & \dots  & a_{1n}  \\
   \vdots & \vdots &  & \vdots \\
   a_{p1} & a_{p2} &\dots  & a_{pn}  \\
   \vdots & \vdots &  & \vdots \\
   a_{q1} & a_{q2} &\dots  & a_{qn}  \\
   \vdots & \vdots &  & \vdots  \\
   a_{n1} & a_{n2} & \dots  & a_{nn}\end{bmatrix}\quad
   B=\begin{bmatrix} a_{11} & a_{12} & \dots  & a_{1n}  \\
   \vdots & \vdots &  & \vdots \\
   a_{p1} & a_{p2} & \dots  & a_{pn}  \\
   \vdots & \vdots &  & \vdots \\
   a_{q1}+ka_{p1} & a_{q2}+ka_{p2} &\dots  & a_{qn}+ka_{pn}  \\
   \vdots & \vdots &  & \vdots  \\
   a_{n1} & a_{n2} & \dots  & a_{nn}\end{bmatrix}$$
We will form another matrix $A'$ by replacing the $q^{th}$ row of $A$ with $k$ times the $p^{th}$ row.
$$A'=\begin{bmatrix} a_{11} & a_{12} & \dots  & a_{1n}  \\
   \vdots & \vdots &  & \vdots \\
   a_{p1} & a_{p2} &\dots  & a_{pn}  \\
   \vdots & \vdots &  & \vdots \\
   ka_{p1} & ka_{p2} &\dots  & ka_{pn}  \\
   \vdots & \vdots &  & \vdots  \\
   a_{n1} & a_{n2} & \dots  & a_{nn}\end{bmatrix}$$
Observe that matrices $A$, $A'$ and $B$ are identical except for the $q^{th}$ row, and the $q^{th}$ row of matrix $B$ is the sum of the $q^{th}$ rows of $A$ and $A'$.  Thus, by Lemma \ref{lemma:arowsumofbc}, we have:
$$\det{B}=\det{A}+\det{A'}$$
Since one row of $A'$ is a scalar multiple of another row, we know $\det{A'}=0$ (see Practice Problem \ref{prob:kAdet}). Therefore 
$\det{B}=\det{A}$.
   \end{proof}

\subsection*{The Laplace Expansion Theorem}
As we have seen in examples, the value of the determinant can be computed by expanding along any row or column.  This result is known as the Laplace Expansion Theorem.  We begin by generalizing some earlier definitions.

Given an $n\times n$ matrix 
$$A=\begin{bmatrix}a_{11} & a_{12} & \dots & a_{1(j-1)} & a_{1j} & a_{1(j+1)} & \dots & a_{1n}  \\
    a_{21} & a_{22} & \dots & a_{2(j-1)} & a_{2j} & a_{2(j+1)} & \dots & a_{2n}  \\
   \vdots & \vdots &  & \vdots & \vdots & \vdots &  & \vdots  \\
   a_{(i-1)1} & a_{(i-1)2} & \ldots & a_{(i-1)(j-1)} & a_{(i-1)j} & a_{(i-1)(j+1)} & \ldots & a_{(i-1)n}\\
  a_{i1} & a_{i2} & \ldots & a_{i(j-1)} & a_{ij} & a_{i(j+1)} & \ldots & a_{in}\\
  a_{(i+1)1} & a_{(i+1)2} & \ldots & a_{(i+1)(j-1)} & a_{(i+1)j} & a_{(i+1)(j+1)} & \ldots & a_{(i+1)n}\\
  \vdots & \vdots &  & \vdots & \vdots & \vdots &  & \vdots  \\
   a_{n1} & a_{n2} & \dots & a_{n(j-1)} & a_{nj} & a_{n(j+1)} & \dots & a_{nn}\end{bmatrix}$$
   define the \dfn{minor} $A_{ij}$ to be an $(n-1)\times (n-1)$ matrix obtained from $A$ by deleting the $i^{th}$ row and the $j^{th}$ column of $A$.
\begin{center}
\begin{tikzpicture}
  \matrix (m)[
    matrix of math nodes,
    nodes in empty cells,
    left delimiter={[},right delimiter={]},minimum width=width("a(i-1)(j-1)")] {
    a_{11} & a_{12} & \dots & a_{1(j-1)} & a_{1j} & a_{1(j+1)} & \dots & a_{1n}  \\
    a_{21} & a_{22} & \dots & a_{2(j-1)} & a_{2j} & a_{2(j+1)} & \dots & a_{2n}  \\
   \vdots & \vdots &  & \vdots & \vdots & \vdots &  & \vdots  \\
   a_{(i-1)1} & a_{(i-1)2} & \ldots & a_{(i-1)(j-1)} & a_{(i-1)j} & a_{(i-1)(j+1)} & \ldots & a_{(i-1)n}\\
  a_{i1} & a_{i2} & \ldots & a_{i(j-1)} & a_{ij} & a_{i(j+1)} & \ldots & a_{in}\\
  a_{(i+1)1} & a_{(i+1)2} & \ldots & a_{(i+1)(j-1)} & a_{(i+1)j} & a_{(i+1)(j+1)} & \ldots & a_{(i+1)n}\\
  \vdots & \vdots &  & \vdots & \vdots & \vdots &  & \vdots  \\
   a_{n1} & a_{n2} & \dots & a_{n(j-1)} & a_{nj} & a_{n(j+1)} & \dots & a_{nn}  \\
  } ;
\draw (m-1-1.north west) rectangle (m-4-4.south east);
\draw (m-1-6.north west) rectangle (m-4-8.south east);
\draw (m-6-1.north west) rectangle (m-8-4.south east);
\draw (m-6-6.north west) rectangle (m-8-8.south east);
 
  \end{tikzpicture}
 \end{center} 
 
 Define the \dfn{$(i,j)$-cofactor} of $A$ by
 $$C_{ij}=(-1)^{i+j}\det(A_{ij})$$
 Note that the sign of $(-1)^{i+j}$ follows a checkerboard pattern.
 $$\begin{bmatrix}+&-&+&-&+&\ldots\\-&+&-&+&-&\ldots\\
 +&-&+&-&+&\ldots\\-&+&-&+&-&\ldots\\\vdots &\vdots  & \vdots & \vdots &\vdots &\ddots \end{bmatrix}$$

\begin{theorem}[Laplace Expansion Theorem]\label{th:laplace1}
Let $A=\begin{bmatrix}a_{ij}\end{bmatrix}$ be an $n\times n$ matrix.  Then each of the following computations produces $\det{A}$.
\begin{enumerate}
    \item \label{eq:laplace1a} \dfn{Cofactor Expansion along the $i^{th}$ row}
\begin{align*}
\det{A}&=a_{i1}C_{i1}+a_{i2}C_{i2}+\ldots +a_{in}C_{in}\\
&=a_{i1}(-1)^{i+1}\det{A_{i1}}+a_{i2}(-1)^{i+2}\det{A_{i2}}+\ldots +a_{in}(-1)^{i+n}\det{A_{in}}\\
&=\sum_{j=1}^na_{ij}(-1)^{i+j}\det{A_{ij}}
\end{align*}
\item \label{eq:laplace1b} \dfn{Cofactor Expansion along the $j^{th}$ column}
\begin{align*}
\det{A}&=a_{1j}C_{1j}+a_{2j}C_{2j}+\ldots +a_{nj}C_{nj}\\
&=a_{1j}(-1)^{1+j}\det{A_{1j}}+a_{2j}(-1)^{2+j}\det{A_{2j}}+\ldots +a_{nj}(-1)^{n+j}\det{A_{nj}}\\
&=\sum_{i=1}^na_{ij}(-1)^{i+j}\det{A_{ij}}
\end{align*}
\end{enumerate}
\end{theorem}
 
 %Computation (\ref{eq:laplace1a}) is called the \dfn{cofactor expansion along the $i^{th}$ row}.  Computation (\ref{eq:laplace1b}) is called \dfn{cofactor expansion along the $j^{th}$ column}.  
 
%  \begin{example}\label{ex:laplace1}
% Let  
% $$A=\begin{bmatrix}4&-1&2&1\\3&0&1&-2\\
% 2&1&5&1\\-2&1&3&-1\end{bmatrix}$$
% Find $\det{A}$ by cofactor expansion along the second row.
% \begin{explanation}
% Matrix $A$ is the same as the matrix in Example \ref{ex:expansiontoprow} of DET-0010.  According to the Laplace Expansion Theorem we should get the same value for the determinant as we did in Example \ref{ex:expansiontoprow} regardless of which row or column we expand along.  The second row has the advantage over other rows in that it contains a zero.  This makes computing one of the cofactors unnecessary.  Following the checker board sign pattern along the second row we get

% \begin{align*}
% \det{A}&=-(3)\det{A_{21}}+(0)\det{A_{22}}-(1)\det{A_{23}}+(-2)\det{A_{24}}\\
% &=-3\begin{vmatrix}-1&2&1\\1&5&1\\1&3&-1\end{vmatrix}-\begin{vmatrix}4&-1&1\\2&1&1\\-2&1&-1\end{vmatrix}+(-2)\begin{vmatrix}4&-1&2\\2&1&5\\-2&1&3\end{vmatrix}\\&=-3(\answer{10})-(\answer{-4})-2(\answer{16})\\
% &=\answer{-58}
% \end{align*}
% This answer is the same as the answer we got using cofactor expansion along the first row in Example \ref{ex:expansiontoprow}.
% \end{explanation}
%  \end{example}
 
%It is clear that having zeros as entries in the matrix significantly reduces the number of computations necessary to find the determinant.  The following example demonstrates how Laplace Expansion Theorem allows us to use zeros to our advantage.

% \begin{example}\label{ex:laplace2}
% Find $\det{A}$ if
% $$A=\begin{bmatrix}4&0&0&0&2\\0&-1&1&0&0\\2&0&0&-5&3\\0&1&4&0&-1\\1&1&5&0&0\end{bmatrix}$$

% \begin{explanation}
% The fourth column contains the most zeros, so we will expand along that column.  The  $(3, 4)$-entry is the only non-zero entry in the fourth column.  So the sign is given by $(-1)^{3+4}=-1$.  You can also consult the checker board pattern to confirm this.  
% $$\det{A}=-(-5)\begin{vmatrix}4&0&0&2\\0&-1&1&0\\0&1&4&-1\\1&1&5&0\end{vmatrix}
% $$
% Next we will expand along the top row.
% $$\det{A}=5\left(4\begin{vmatrix}-1&1&0\\1&4&-1\\1&5&0\end{vmatrix}-2\begin{vmatrix}0&-1&1\\0&1&4&\\1&1&5\end{vmatrix}\right)$$
% Try the next step on your own.  We suggest that you expand the first matrix along the last column and expand the second matrix along the first column.
% $$\det(A)=5\big(4(\answer{-6})-2(\answer{-5})\big)=\answer{-70}$$
% \end{explanation}
% \end{example}
\begin{proof}
We will start by showing that cofactor expansion along column $j$ produces the same result as cofactor expansion along the first column.  Observe that column $j$ can be shifted into the first column position by $j-1$ consecutive row switches.  Let $A'=\begin{bmatrix}a'_{ij}\end{bmatrix}$ be the matrix obtained from $A$ by performing the necessary column switches.  Then 
\begin{align*}
\det{A}&=(-1)^{j-1}\det{A'}\\
&=(-1)^{j-1}\sum_{i=1}^na'_{i1}(-1)^{i+1}\det{A'_{i1}}\\
&=(-1)^{j-1}\sum_{i=1}^na_{ij}(-1)^{i+1}\det{A_{ij}}\\
&=\sum_{i=1}^na_{ij}(-1)^{j-1}(-1)^{i+1}\det{A_{ij}}\\
&=\sum_{i=1}^na_{ij}(-1)^{i+j}\det{A_{ij}}
\end{align*}

To show that the determinant of $A$ can also be computed by cofactor expansion along any row follows from the fact that $\det{A}=\det{A^T}$. (Theorem \ref{th:detoftrans})
\end{proof}
\section*{Practice Problems}

\begin{problem}\label{prob:proofofrowswapanddet}
Complete the proof of Theorem \ref{th:elemrowopsanddet}\ref{item:rowswapanddet} by showing that the result holds for a $2\times 2$ matrix.
\end{problem}

\begin{problem}\label{prob:numberofrowswitches}
Let $p$ and $q$ be two rows of a matrix, with $p<q$.  Show that the switch of $p$ and $q$ requires $2(q-p)-1$ adjacent row interchanges.
\end{problem}



\end{document} 