\documentclass{ximera}
%%% Begin Laad packages

\makeatletter
\@ifclassloaded{xourse}{%
    \typeout{Start loading preamble.tex (in a XOURSE)}%
    \def\isXourse{true}   % automatically defined; pre 112022 it had to be set 'manually' in a xourse
}{%
    \typeout{Start loading preamble.tex (NOT in a XOURSE)}%
}
\makeatother

\def\isEn\true 

\pgfplotsset{compat=1.16}

\usepackage{currfile}

% 201908/202301: PAS OP: babel en doclicense lijken problemen te veroorzaken in .jax bestand
% (wegens syntax error met toegevoegde \newcommands ...)
\pdfOnly{
    \usepackage[type={CC},modifier={by-nc-sa},version={4.0}]{doclicense}
    %\usepackage[hyperxmp=false,type={CC},modifier={by-nc-sa},version={4.0}]{doclicense}
    %%% \usepackage[dutch]{babel}
}



\usepackage[utf8]{inputenc}
\usepackage{morewrites}   % nav zomercursus (answer...?)
\usepackage{multirow}
\usepackage{multicol}
\usepackage{tikzsymbols}
\usepackage{ifthen}
%\usepackage{animate} BREAKS HTML STRUCTURE USED BY XIMERA
\usepackage{relsize}

\usepackage{eurosym}    % \euro  (€ werkt niet in xake ...?)
\usepackage{fontawesome} % smileys etc

% Nuttig als ook interactieve beamer slides worden voorzien:
\providecommand{\p}{} % default nothing ; potentially usefull for slides: redefine as \pause
%providecommand{\p}{\pause}

    % Layout-parameters voor het onderschrift bij figuren
\usepackage[margin=10pt,font=small,labelfont=bf, labelsep=endash,format=hang]{caption}
%\usepackage{caption} % captionof
%\usepackage{pdflscape}    % landscape environment

% Met "\newcommand\showtodonotes{}" kan je todonotes tonen (in pdf/online)
% 201908: online werkt het niet (goed)
\providecommand\showtodonotes{disable}
\providecommand\todo[1]{\typeout{TODO #1}}
%\usepackage[\showtodonotes]{todonotes}
%\usepackage{todonotes}

%
% Poging tot aanpassen layout
%
\usepackage{tcolorbox}
\tcbuselibrary{theorems}

%%% Einde laad packages

%%% Begin Ximera specifieke zaken

\graphicspath{
	{../../}
	{../}
	{./}
  	{../../pictures/}
   	{../pictures/}
   	{./pictures/}
	{./explog/}    % M05 in groeimodellen       
}

%%% Einde Ximera specifieke zaken

%
% define softer blue/red/green, use KU Leuven base colors for blue (and dark orange for red ?)
%
% todo: rather redefine blue/red/green ...?
%\definecolor{xmblue}{rgb}{0.01, 0.31, 0.59}
%\definecolor{xmred}{rgb}{0.89, 0.02, 0.17}
\definecolor{xmdarkblue}{rgb}{0.122, 0.671, 0.835}   % KU Leuven Blauw
\definecolor{xmblue}{rgb}{0.114, 0.553, 0.69}        % KU Leuven Blauw
\definecolor{xmgreen}{rgb}{0.13, 0.55, 0.13}         % No KULeuven variant for green found ...

\definecolor{xmaccent}{rgb}{0.867, 0.541, 0.18}      % KU Leuven Accent (orange ...)
\definecolor{kuaccent}{rgb}{0.867, 0.541, 0.18}      % KU Leuven Accent (orange ...)

\colorlet{xmred}{xmaccent!50!black}                  % Darker version of KU Leuven Accent

\providecommand{\blue}[1]{{\color{blue}#1}}    
\providecommand{\red}[1]{{\color{red}#1}}

\renewcommand\CancelColor{\color{xmaccent!50!black}}

% werkt in math en text mode om MATH met oranje (of grijze...)  achtergond te tonen (ook \important{\text{blabla}} lijkt te werken)
%\newcommand{\important}[1]{\ensuremath{\colorbox{xmaccent!50!white}{$#1$}}}   % werkt niet in Mathjax
%\newcommand{\important}[1]{\ensuremath{\colorbox{lightgray}{$#1$}}}
\newcommand{\important}[1]{\ensuremath{\colorbox{orange}{$#1$}}}   % TODO: kleur aanpassen voor mathjax; wordt overschreven infra!


% Uitzonderlijk kan met \pdfnl in de PDF een newline worden geforceerd, die online niet nodig/nuttig is omdat daar de regellengte hoe dan ook niet gekend is.
\ifdefined\HCode%
\providecommand{\pdfnl}{}%
\else%
\providecommand{\pdfnl}{%
  \\%
}%
\fi

% Uitzonderlijk kan met \handoutnl in de handout-PDF een newline worden geforceerd, die noch online noch in de PDF-met-antwoorden nuttig is.
\ifdefined\HCode
\providecommand{\handoutnl}{}
\else
\providecommand{\handoutnl}{%
\ifhandout%
  \nl%
\fi%
}
\fi



% \cellcolor IGNORED by tex4ht ?
% \begin{center} seems not to wordk
    % (missing margin-left: auto;   on tabular-inside-center ???)
%\newcommand{\importantcell}[1]{\ensuremath{\cellcolor{lightgray}#1}}  %  in tabular; usablility to be checked
\providecommand{\importantcell}[1]{\ensuremath{#1}}     % no mathjax2 support for colloring array cells

\pdfOnly{
  \renewcommand{\important}[1]{\ensuremath{\colorbox{kuaccent!50!white}{$#1$}}}
  \renewcommand{\importantcell}[1]{\ensuremath{\cellcolor{kuaccent!40!white}#1}}   
}

%%% Tikz styles


\pgfplotsset{compat=1.16}

\usetikzlibrary{trees,positioning,arrows,fit,shapes,math,calc,decorations.markings,through,intersections,patterns,matrix}

\usetikzlibrary{decorations.pathreplacing,backgrounds}    % 5/2023: from experimental


\usetikzlibrary{angles,quotes}

\usepgfplotslibrary{fillbetween} % bepaalde_integraal
\usepgfplotslibrary{polar}    % oa voor poolcoordinaten.tex

\pgfplotsset{ownstyle/.style={axis lines = center, axis equal image, xlabel = $x$, ylabel = $y$, enlargelimits}} 

\pgfplotsset{
	plot/.style={no marks,samples=50}
}

\newcommand{\xmPlotsColor}{
	\pgfplotsset{
		plot1/.style={darkgray,no marks,samples=100},
		plot2/.style={lightgray,no marks,samples=100},
		plotresult/.style={blue,no marks,samples=100},
		plotblue/.style={blue,no marks,samples=100},
		plotred/.style={red,no marks,samples=100},
		plotgreen/.style={green,no marks,samples=100},
		plotpurple/.style={purple,no marks,samples=100}
	}
}
\newcommand{\xmPlotsBlackWhite}{
	\pgfplotsset{
		plot1/.style={black,loosely dashed,no marks,samples=100},
		plot2/.style={black,loosely dotted,no marks,samples=100},
		plotresult/.style={black,no marks,samples=100},
		plotblue/.style={black,no marks,samples=100},
		plotred/.style={black,dotted,no marks,samples=100},
		plotgreen/.style={black,dashed,no marks,samples=100},
		plotpurple/.style={black,dashdotted,no marks,samples=100}
	}
}


\newcommand{\xmPlotsColorAndStyle}{
	\pgfplotsset{
		plot1/.style={darkgray,no marks,samples=100},
		plot2/.style={lightgray,no marks,samples=100},
		plotresult/.style={blue,no marks,samples=100},
		plotblue/.style={xmblue,no marks,samples=100},
		plotred/.style={xmred,dashed,thick,no marks,samples=100},
		plotgreen/.style={xmgreen,dotted,very thick,no marks,samples=100},
		plotpurple/.style={purple,no marks,samples=100}
	}
}


%\iftikzexport
\xmPlotsColorAndStyle
%\else
%\xmPlotsBlackWhite
%\fi
%%%


%
% Om venndiagrammen te arceren ...
%
\makeatletter
\pgfdeclarepatternformonly[\hatchdistance,\hatchthickness]{north east hatch}% name
{\pgfqpoint{-1pt}{-1pt}}% below left
{\pgfqpoint{\hatchdistance}{\hatchdistance}}% above right
{\pgfpoint{\hatchdistance-1pt}{\hatchdistance-1pt}}%
{
	\pgfsetcolor{\tikz@pattern@color}
	\pgfsetlinewidth{\hatchthickness}
	\pgfpathmoveto{\pgfqpoint{0pt}{0pt}}
	\pgfpathlineto{\pgfqpoint{\hatchdistance}{\hatchdistance}}
	\pgfusepath{stroke}
}
\pgfdeclarepatternformonly[\hatchdistance,\hatchthickness]{north west hatch}% name
{\pgfqpoint{-\hatchthickness}{-\hatchthickness}}% below left
{\pgfqpoint{\hatchdistance+\hatchthickness}{\hatchdistance+\hatchthickness}}% above right
{\pgfpoint{\hatchdistance}{\hatchdistance}}%
{
	\pgfsetcolor{\tikz@pattern@color}
	\pgfsetlinewidth{\hatchthickness}
	\pgfpathmoveto{\pgfqpoint{\hatchdistance+\hatchthickness}{-\hatchthickness}}
	\pgfpathlineto{\pgfqpoint{-\hatchthickness}{\hatchdistance+\hatchthickness}}
	\pgfusepath{stroke}
}
%\makeatother

\tikzset{
    hatch distance/.store in=\hatchdistance,
    hatch distance=10pt,
    hatch thickness/.store in=\hatchthickness,
   	hatch thickness=2pt
}

\colorlet{circle edge}{black}
\colorlet{circle area}{blue!20}


\tikzset{
    filled/.style={fill=green!30, draw=circle edge, thick},
    arceerl/.style={pattern=north east hatch, pattern color=blue!50, draw=circle edge},
    arceerr/.style={pattern=north west hatch, pattern color=yellow!50, draw=circle edge},
    outline/.style={draw=circle edge, thick}
}




%%% Updaten commando's
\def\hoofding #1#2#3{\maketitle}     % OBSOLETE ??

% we willen (bijna) altijd \geqslant ipv \geq ...!
\newcommand{\geqnoslant}{\geq}
\renewcommand{\geq}{\geqslant}
\newcommand{\leqnoslant}{\leq}
\renewcommand{\leq}{\leqslant}

% Todo: (201908) waarom komt er (soms) underlined voor emph ...?
\renewcommand{\emph}[1]{\textit{#1}}

% API commando's

\newcommand{\ds}{\displaystyle}
\newcommand{\ts}{\textstyle}  % tegenhanger van \ds   (Ximera zet PER  DEFAULT \ds!)

% uit Zomercursus-macro's: 
\newcommand{\bron}[1]{\begin{scriptsize} \emph{#1} \end{scriptsize}}     % deprecated ...?


%definities nieuwe commando's - afkortingen veel gebruikte symbolen
\newcommand{\R}{\ensuremath{\mathbb{R}}}
\newcommand{\Rnul}{\ensuremath{\mathbb{R}_0}}
\newcommand{\Reen}{\ensuremath{\mathbb{R}\setminus\{1\}}}
\newcommand{\Rnuleen}{\ensuremath{\mathbb{R}\setminus\{0,1\}}}
\newcommand{\Rplus}{\ensuremath{\mathbb{R}^+}}
\newcommand{\Rmin}{\ensuremath{\mathbb{R}^-}}
\newcommand{\Rnulplus}{\ensuremath{\mathbb{R}_0^+}}
\newcommand{\Rnulmin}{\ensuremath{\mathbb{R}_0^-}}
\newcommand{\Rnuleenplus}{\ensuremath{\mathbb{R}^+\setminus\{0,1\}}}
\newcommand{\N}{\ensuremath{\mathbb{N}}}
\newcommand{\Nnul}{\ensuremath{\mathbb{N}_0}}
\newcommand{\Z}{\ensuremath{\mathbb{Z}}}
\newcommand{\Znul}{\ensuremath{\mathbb{Z}_0}}
\newcommand{\Zplus}{\ensuremath{\mathbb{Z}^+}}
\newcommand{\Zmin}{\ensuremath{\mathbb{Z}^-}}
\newcommand{\Znulplus}{\ensuremath{\mathbb{Z}_0^+}}
\newcommand{\Znulmin}{\ensuremath{\mathbb{Z}_0^-}}
\newcommand{\C}{\ensuremath{\mathbb{C}}}
\newcommand{\Cnul}{\ensuremath{\mathbb{C}_0}}
\newcommand{\Cplus}{\ensuremath{\mathbb{C}^+}}
\newcommand{\Cmin}{\ensuremath{\mathbb{C}^-}}
\newcommand{\Cnulplus}{\ensuremath{\mathbb{C}_0^+}}
\newcommand{\Cnulmin}{\ensuremath{\mathbb{C}_0^-}}
\newcommand{\Q}{\ensuremath{\mathbb{Q}}}
\newcommand{\Qnul}{\ensuremath{\mathbb{Q}_0}}
\newcommand{\Qplus}{\ensuremath{\mathbb{Q}^+}}
\newcommand{\Qmin}{\ensuremath{\mathbb{Q}^-}}
\newcommand{\Qnulplus}{\ensuremath{\mathbb{Q}_0^+}}
\newcommand{\Qnulmin}{\ensuremath{\mathbb{Q}_0^-}}

\newcommand{\perdef}{\overset{\mathrm{def}}{=}}
\newcommand{\pernot}{\overset{\mathrm{notatie}}{=}}
\newcommand\perinderdaad{\overset{!}{=}}     % voorlopig gebruikt in limietenrekenregels
\newcommand\perhaps{\overset{?}{=}}          % voorlopig gebruikt in limietenrekenregels

\newcommand{\degree}{^\circ}


\DeclareMathOperator{\dom}{dom}     % domein
\DeclareMathOperator{\codom}{codom} % codomein
\DeclareMathOperator{\bld}{bld}     % beeld
\DeclareMathOperator{\graf}{graf}   % grafiek
\DeclareMathOperator{\rico}{rico}   % richtingcoëfficient
\DeclareMathOperator{\co}{co}       % coordinaat
\DeclareMathOperator{\gr}{gr}       % graad

\newcommand{\func}[5]{\ensuremath{#1: #2 \rightarrow #3: #4 \mapsto #5}} % Easy to write a function


% Operators
\DeclareMathOperator{\bgsin}{bgsin}
\DeclareMathOperator{\bgcos}{bgcos}
\DeclareMathOperator{\bgtan}{bgtan}
\DeclareMathOperator{\bgcot}{bgcot}
\DeclareMathOperator{\bgsinh}{bgsinh}
\DeclareMathOperator{\bgcosh}{bgcosh}
\DeclareMathOperator{\bgtanh}{bgtanh}
\DeclareMathOperator{\bgcoth}{bgcoth}

% Oude \Bgsin etc deprecated: gebruik \bgsin, en herdefinieer dat als je Bgsin wil!
%\DeclareMathOperator{\cosec}{cosec}    % not used? gebruik \csc en herdefinieer

% operatoren voor differentialen: to be verified; 1/2020: inconsequent gebruik bij afgeleiden/integralen
\renewcommand{\d}{\mathrm{d}}
\newcommand{\dx}{\d x}
\newcommand{\dd}[1]{\frac{\mathrm{d}}{\mathrm{d}#1}}
\newcommand{\ddx}{\dd{x}}

% om in voorbeelden/oefeningen de notatie voor afgeleiden te kunnen kiezen
% Usage: \afg{(2\sin(x))}  (en wordt d/dx, of accent, of D )
%\newcommand{\afg}[1]{{#1}'}
\newcommand{\afg}[1]{\left(#1\right)'}
%\renewcommand{\afg}[1]{\frac{\mathrm{d}#1}{\mathrm{d}x}}   % include in relevant exercises ...
%\renewcommand{\afg}[1]{D{#1}}

%
% \xmxxx commands: Extra KU Leuven functionaliteit van, boven of naast Ximera
%   ( Conventie 8/2019: xm+nederlandse omschrijving, maar is niet consequent gevolgd, en misschien ook niet erg handig !)
%
% (Met een minimale ximera.cls en preamble.tex zou een bruikbare .pdf moeten kunnen worden gemaakt van eender welke ximera)
%
% Usage: \xmtitle[Mijn korte abstract]{Mijn titel}{Mijn abstract}
% Eerste command na \begin{document}:
%  -> definieert de \title
%  -> definieert de abstract
%  -> doet \maketitle ( dus: print de hoofding als 'chapter' of 'sectie')
% Optionele parameter geeft eenn kort abstract (die met de globale setting \xmshortabstract{} al dan niet kan worden geprint.
% De optionele korte abstract kan worden gebruikt voor pseudo-grappige abtsarts, dus dus globaal al dan niet kunnen worden gebuikt...
% Globale settings:
%  de (optionele) 'korte abstract' wordt enkele getoond als \xmshortabstract is gezet
\providecommand\xmshortabstract{} % default: print (only!) short abstract if present
\newcommand{\xmtitle}[3][]{
	\title{#2}
	\begin{abstract}
		\ifdefined\xmshortabstract
		\ifstrempty{#1}{%
			#3
		}{%
			#1
		}%
		\else
		#3
		\fi
	\end{abstract}
	\maketitle
}

% 
% Kleine grapjes: moeten zonder verder gevolg kunnen worden verwijderd
%
%\newcommand{\xmopje}[1]{{\small#1{\reversemarginpar\marginpar{\Smiley}}}}   % probleem in floats!!
\newtoggle{showxmopje}
\toggletrue{showxmopje}

\newcommand{\xmopje}[1]{%
   \iftoggle{showxmopje}{#1}{}%
}


% -> geef een abstracte-formule-met-rechts-een-concreet-voorbeeld
% VB:  \formulevb{a^2+b^2=c^2}{3^2+4^2=5^2}
%
\ifdefined\HCode
\NewEnviron{xmdiv}[1]{\HCode{\Hnewline<div class="#1">\Hnewline}\BODY{\HCode{\Hnewline</div>\Hnewline}}}
\else
\NewEnviron{xmdiv}[1]{\BODY}
\fi

\providecommand{\formulevb}[2]{
	{\centering

    \begin{xmdiv}{xmformulevb}    % zie css voor online layout !!!
	\begin{tabular}{lcl}
		\important{#1}
		&  &
		Vb: $#2$
		\end{tabular}
	\end{xmdiv}

	}
}

\ifdefined\HCode
\providecommand{\vb}[1]{%
    \HCode{\Hnewline<span class="xmvb">}#1\HCode{</span>\Hnewline}%
}
\else
\providecommand{\vb}[1]{
    \colorbox{blue!10}{#1}
}
\fi

\ifdefined\HCode
\providecommand{\xmcolorbox}[2]{
	\HCode{\Hnewline<div class="xmcolorbox">\Hnewline}#2\HCode{\Hnewline</div>\Hnewline}
}
\else
\providecommand{\xmcolorbox}[2]{
  \cellcolor{#1}#2
}
\fi


\ifdefined\HCode
\providecommand{\xmopmerking}[1]{
 \HCode{\Hnewline<div class="xmopmerking">\Hnewline}#1\HCode{\Hnewline</div>\Hnewline}
}
\else
\providecommand{\xmopmerking}[1]{
	{\footnotesize #1}
}
\fi
% \providecommand{\voorbeeld}[1]{
% 	\colorbox{blue!10}{$#1$}
% }



% Hernoem Proof naar Bewijs, nodig voor HTML versie
\renewcommand*{\proofname}{Bewijs}

% Om opgave van oefening (wordt niet geprint bij oplossingenblad)
% (to be tested test)
\NewEnviron{statement}{\BODY}

% Environment 'oplossing' en 'uitkomst'
% voor resp. volledige 'uitwerking' dan wel 'enkel eindresultaat'
% geimplementeerd via feedback, omdat er in de ximera-server adhoc feedback-code is toegevoegd
%% Niet tonen indien handout
%% Te gebruiken om volledige oplossingen/uitwerkingen van oefeningen te tonen
%% \begin{oplossing}        De optelling is commutatief \end{oplossing}  : verschijnt online enkel 'op vraag'
%% \begin{oplossing}[toon]  De optelling is commutatief \end{oplossing}  : verschijnt steeds onmiddellijk online (bv te gebruiken bij voorbeelden) 

\ifhandout%
    \NewEnviron{oplossing}[1][onzichtbaar]%
    {%
    \ifthenelse{\equal{\detokenize{#1}}{\detokenize{toon}}}
    {
    \def\PH@Command{#1}% Use PH@Command to hold the content and be a target for "\expandafter" to expand once.

    \begin{trivlist}% Begin the trivlist to use formating of the "Feedback" label.
    \item[\hskip \labelsep\small\slshape\bfseries Oplossing% Format the "Feedback" label. Don't forget the space.
    %(\texttt{\detokenize\expandafter{\PH@Command}}):% Format (and detokenize) the condition for feedback to trigger
    \hspace{2ex}]\small%\slshape% Insert some space before the actual feedback given.
    \BODY
    \end{trivlist}
    }
    {  % \begin{feedback}[solution]   \BODY     \end{feedback}  }
    }
    }    
\else
% ONLY for HTML; xmoplossing is styled with css, and is not, and need not be a LaTeX environment
% THUS: it does NOT use feedback anymore ...
%    \NewEnviron{oplossing}{\begin{expandable}{xmoplossing}{\nlen{Toon uitwerking}{Show solution}}{\BODY}\end{expandable}}
    \newenvironment{oplossing}[1][onzichtbaar]
   {%
       \begin{expandable}{xmoplossing}{}
   }
   {%
   	   \end{expandable}
   } 
%     \newenvironment{oplossing}[1][onzichtbaar]
%    {%
%        \begin{feedback}[solution]   	
%    }
%    {%
%    	   \end{feedback}
%    } 
\fi

\ifhandout%
    \NewEnviron{uitkomst}[1][onzichtbaar]%
    {%
    \ifthenelse{\equal{\detokenize{#1}}{\detokenize{toon}}}
    {
    \def\PH@Command{#1}% Use PH@Command to hold the content and be a target for "\expandafter" to expand once.

    \begin{trivlist}% Begin the trivlist to use formating of the "Feedback" label.
    \item[\hskip \labelsep\small\slshape\bfseries Uitkomst:% Format the "Feedback" label. Don't forget the space.
    %(\texttt{\detokenize\expandafter{\PH@Command}}):% Format (and detokenize) the condition for feedback to trigger
    \hspace{2ex}]\small%\slshape% Insert some space before the actual feedback given.
    \BODY
    \end{trivlist}
    }
    {  % \begin{feedback}[solution]   \BODY     \end{feedback}  }
    }
    }    
\else
\ifdefined\HCode
   \newenvironment{uitkomst}[1][onzichtbaar]
    {%
        \begin{expandable}{xmuitkomst}{}%
    }
    {%
    	\end{expandable}%
    } 
\else
  % Do NOT print 'uitkomst' in non-handout
  %  (presumably, there is also an 'oplossing' ??)
  \newenvironment{uitkomst}[1][onzichtbaar]{}{}
\fi
\fi

%
% Uitweidingen zijn extra's die niet redelijkerwijze tot de leerstof behoren
% Uitbreidingen zijn extra's die wel redelijkerwijze tot de leerstof van bv meer geavanceerde versies kunnen behoren (B-programma/Wiskundestudenten/...?)
% Nog niet voorzien: design voor verschillende versies (A/B programma, BIO, voorkennis/ ...)
% Voor 'uitweidingen' is er een environment die online per default is ingeklapt, en in pdf al dan niet kan worden geincluded  (via \xmnouitweiding) 
%
% in een xourse, per default GEEN uitweidingen, tenzij \xmuitweiding expliciet ergens is gezet ...
\ifdefined\isXourse
   \ifdefined\xmuitweiding
   \else
       \def\xmnouitweiding{true}
   \fi
\fi

\ifdefined\xmnouitweiding
\newcounter{xmuitweiding}  % anders error undefined ...  
\excludecomment{xmuitweiding}
\else
\newtheoremstyle{dotless}{}{}{}{}{}{}{ }{}
\theoremstyle{dotless}
\newtheorem*{xmuitweidingnofrills}{}   % nofrills = no accordion; gebruikt dus de dotless theoremstyle!

\newcounter{xmuitweiding}
\newenvironment{xmuitweiding}[1][ ]%
{% 
	\refstepcounter{xmuitweiding}%
    \begin{expandable}{xmuitweiding}{\nlentext{Uitweiding \arabic{xmuitweiding}: #1}{Digression \arabic{xmuitweiding}: #1}}%
	\begin{xmuitweidingnofrills}%
}
{%
    \end{xmuitweidingnofrills}%
    \end{expandable}%
}   
% \newenvironment{xmuitweiding}[1][ ]%
% {% 
% 	\refstepcounter{xmuitweiding}
% 	\begin{accordion}\begin{accordion-item}[Uitweiding \arabic{xmuitweiding}: #1]%
% 			\begin{xmuitweidingnofrills}%
% 			}
% 			{\end{xmuitweidingnofrills}\end{accordion-item}\end{accordion}}   
\fi


\newenvironment{xmexpandable}[1][]{
	\begin{accordion}\begin{accordion-item}[#1]%
		}{\end{accordion-item}\end{accordion}}


% Command that gives a selection box online, but just prints the right answer in pdf
\newcommand{\xmonlineChoice}[1]{\pdfOnly{\wordchoicegiventrue}\wordChoice{#1}\pdfOnly{\wordchoicegivenfalse}}   % deprecated, gebruik onlineChoice ...
\newcommand{\onlineChoice}[1]{\pdfOnly{\wordchoicegiventrue}\wordChoice{#1}\pdfOnly{\wordchoicegivenfalse}}

\newcommand{\TJa}{\nlentext{ Ja }{ Yes }}
\newcommand{\TNee}{\nlentext{ Nee }{ No }}
\newcommand{\TJuist}{\nlentext{ Juist }{ True }}
\newcommand{\TFout}{\nlentext{ Fout }{ False }}

\newcommand{\choiceTrue }{{\renewcommand{\choiceminimumhorizontalsize}{4em}\wordChoice{\choice[correct]{\TJuist}\choice{\TFout}}}}
\newcommand{\choiceFalse}{{\renewcommand{\choiceminimumhorizontalsize}{4em}\wordChoice{\choice{\TJuist}\choice[correct]{\TFout}}}}

\newcommand{\choiceYes}{{\renewcommand{\choiceminimumhorizontalsize}{3em}\wordChoice{\choice[correct]{\TJa}\choice{\TNee}}}}
\newcommand{\choiceNo }{{\renewcommand{\choiceminimumhorizontalsize}{3em}\wordChoice{\choice{\TJa}\choice[correct]{\TNee}}}}

% Optional nicer formatting for wordChoice in PDF

\let\inlinechoiceorig\inlinechoice

%\makeatletter
%\providecommand{\choiceminimumverticalsize}{\vphantom{$\frac{\sqrt{2}}{2}$}}   % minimum height of boxes (cfr infra)
\providecommand{\choiceminimumverticalsize}{\vphantom{$\tfrac{2}{2}$}}   % minimum height of boxes (cfr infra)
\providecommand{\choiceminimumhorizontalsize}{1em}   % minimum width of boxes (cfr infra)

\newcommand{\inlinechoicesquares}[2][]{%
		\setkeys{choice}{#1}%
		\ifthenelse{\boolean{\choice@correct}}%
		{%
            \ifhandout%
               \fbox{\choiceminimumverticalsize #2}\allowbreak\ignorespaces%
            \else%
               \fcolorbox{blue}{blue!20}{\choiceminimumverticalsize #2}\allowbreak\ignorespaces\setkeys{choice}{correct=false}\ignorespaces%
            \fi%
		}%
		{% else
			\fbox{\choiceminimumverticalsize #2}\allowbreak\ignorespaces%  HACK: wat kleiner, zodat fits on line ... 	
		}%
}

\newcommand{\inlinechoicesquareX}[2][]{%
		\setkeys{choice}{#1}%
		\ifthenelse{\boolean{\choice@correct}}%
		{%
            \ifhandout%
               \framebox[\ifdim\choiceminimumhorizontalsize<\width\width\else\choiceminimumhorizontalsize\fi]{\choiceminimumverticalsize\ #2\ }\allowbreak\ignorespaces\setkeys{choice}{correct=false}\ignorespaces%
            \else%
               \fcolorbox{blue}{blue!20}{\makebox[\ifdim\choiceminimumhorizontalsize<\width\width\else\choiceminimumhorizontalsize\fi]{\choiceminimumverticalsize #2}}\allowbreak\ignorespaces\setkeys{choice}{correct=false}\ignorespaces%
            \fi%
		}%
		{% else
        \ifhandout%
			\framebox[\ifdim\choiceminimumhorizontalsize<\width\width\else\choiceminimumhorizontalsize\fi]{\choiceminimumverticalsize\ #2\ }\allowbreak\ignorespaces%  HACK: wat kleiner, zodat fits on line ... 	
        \fi
		}%
}


\newcommand{\inlinechoicepuntjes}[2][]{%
		\setkeys{choice}{#1}%
		\ifthenelse{\boolean{\choice@correct}}%
		{%
            \ifhandout%
               \dots\ldots\ignorespaces\setkeys{choice}{correct=false}\ignorespaces
            \else%
               \fcolorbox{blue}{blue!20}{\choiceminimumverticalsize #2}\allowbreak\ignorespaces\setkeys{choice}{correct=false}\ignorespaces%
            \fi%
		}%
		{% else
			%\fbox{\choiceminimumverticalsize #2}\allowbreak\ignorespaces%  HACK: wat kleiner, zodat fits on line ... 	
		}%
}

% print niets, maar definieer globale variable \myanswer
%  (gebruikt om oplossingsbladen te printen) 
\newcommand{\inlinechoicedefanswer}[2][]{%
		\setkeys{choice}{#1}%
		\ifthenelse{\boolean{\choice@correct}}%
		{%
               \gdef\myanswer{#2}\setkeys{choice}{correct=false}

		}%
		{% else
			%\fbox{\choiceminimumverticalsize #2}\allowbreak\ignorespaces%  HACK: wat kleiner, zodat fits on line ... 	
		}%
}



%\makeatother

\newcommand{\setchoicedefanswer}{
\ifdefined\HCode
\else
%    \renewenvironment{multipleChoice@}[1][]{}{} % remove trailing ')'
    \let\inlinechoice\inlinechoicedefanswer
\fi
}

\newcommand{\setchoicepuntjes}{
\ifdefined\HCode
\else
    \renewenvironment{multipleChoice@}[1][]{}{} % remove trailing ')'
    \let\inlinechoice\inlinechoicepuntjes
\fi
}
\newcommand{\setchoicesquares}{
\ifdefined\HCode
\else
    \renewenvironment{multipleChoice@}[1][]{}{} % remove trailing ')'
    \let\inlinechoice\inlinechoicesquares
\fi
}
%
\newcommand{\setchoicesquareX}{
\ifdefined\HCode
\else
    \renewenvironment{multipleChoice@}[1][]{}{} % remove trailing ')'
    \let\inlinechoice\inlinechoicesquareX
\fi
}
%
\newcommand{\setchoicelist}{
\ifdefined\HCode
\else
    \renewenvironment{multipleChoice@}[1][]{}{)}% re-add trailing ')'
    \let\inlinechoice\inlinechoiceorig
\fi
}

\setchoicesquareX  % by default list-of-squares with onlineChoice in PDF

% Omdat multicols niet werkt in html: enkel in pdf  (in html zijn langere pagina's misschien ook minder storend)
\newenvironment{xmmulticols}[1][2]{
 \pdfOnly{\begin{multicols}{#1}}%
}{ \pdfOnly{\end{multicols}}}

%
% Te gebruiken in plaats van \section\subsection
%  (in een printstyle kan dan het level worden aangepast
%    naargelang \chapter vs \section style )
% 3/2021: DO NOT USE \xmsubsection !
\newcommand\xmsection\subsection
\newcommand\xmsubsection\subsubsection

% Aanpassen printversie
%  (hier gedefinieerd, zodat ze in xourse kunnen worden gezet/overschreven)
\providebool{parttoc}
\providebool{printpartfrontpage}
\providebool{printactivitytitle}
\providebool{printactivityqrcode}
\providebool{printactivityurl}
\providebool{printcontinuouspagenumbers}
\providebool{numberactivitiesbysubpart}
\providebool{addtitlenumber}
\providebool{addsectiontitlenumber}
\addtitlenumbertrue
\addsectiontitlenumbertrue

% The following three commands are hardcoded in xake, you can't create other commands like these, without adding them to xake as well
%  ( gebruikt in xourses om juiste soort titelpagina te krijgen voor verschillende ximera's )
\newcommand{\activitychapter}[2][]{
    {    
    \ifstrequal{#1}{notnumbered}{
        \addtitlenumberfalse
    }{}
    \typeout{ACTIVITYCHAPTER #2}   % logging
	\chapterstyle
	\activity{#2}
    }
}
\newcommand{\activitysection}[2][]{
    {
    \ifstrequal{#1}{notnumbered}{
        \addsectiontitlenumberfalse
    }{}
	\typeout{ACTIVITYSECTION #2}   % logging
	\sectionstyle
	\activity{#2}
    }
}
% Practices worden als activity getoond om de grote blokken te krijgen online
\newcommand{\practicesection}[2][]{
    {
    \ifstrequal{#1}{notnumbered}{
        \addsectiontitlenumberfalse
    }{}
    \typeout{PRACTICESECTION #2}   % logging
	\sectionstyle
	\activity{#2}
    }
}
\newcommand{\activitychapterlink}[3][]{
    {
    \ifstrequal{#1}{notnumbered}{
        \addtitlenumberfalse
    }{}
    \typeout{ACTIVITYCHAPTERLINK #3}   % logging
	\chapterstyle
	\activitylink[#1]{#2}{#3}
    }
}

\newcommand{\activitysectionlink}[3][]{
    {
    \ifstrequal{#1}{notnumbered}{
        \addsectiontitlenumberfalse
    }{}
    \typeout{ACTIVITYSECTIONLINK #3}   % logging
	\sectionstyle
	\activitylink[#1]{#2}{#3}
    }
}


% Commando om de printstyle toe te voegen in ximera's. Zorgt ervoor dat er geen problemen zijn als je de xourses compileert
% hack om onhandige relative paden in TeX te omzeilen
% should work both in xourse and ximera (pre-112022 only in ximera; thus obsoletes adhoc setup in xourses)
% loads global.sty if present (cfr global.css for online settings!)
% use global.sty to overwrite settings in printstyle.sty ...
\newcommand{\addPrintStyle}[1]{
\iftikzexport\else   % only in PDF
  \makeatletter
  \ifx\@onlypreamble\@notprerr\else   % ONLY if in tex-preamble   (and e.g. not when included from xourse)
    \typeout{Loading printstyle}   % logging
    \usepackage{#1/printstyle} % mag enkel geinclude worden als je die apart compileert
    \IfFileExists{#1/global.sty}{
        \typeout{Loading printstyle-folder #1/global.sty}   % logging
        \usepackage{#1/global}
        }{
        \typeout{Info: No extra #1/global.sty}   % logging
    }   % load global.sty if present
    \IfFileExists{global.sty}{
        \typeout{Loading local-folder global.sty (or TEXINPUTPATH..)}   % logging
        \usepackage{global}
    }{
        \typeout{Info: No folder/global.sty}   % logging
    }   % load global.sty if present
    \IfFileExists{\currfilebase.sty}
    {
        \typeout{Loading \currfilebase.sty}
        \input{\currfilebase.sty}
    }{
        \typeout{Info: No local \currfilebase.sty}
    }
    \fi
  \makeatother
\fi
}

%
%  
% references: Ximera heeft adhoc logica	 om online labels te doen werken over verschillende files heen
% met \hyperref kan de getoonde tekst toch worden opgegeven, in plaats van af te hangen van de label-text
\ifdefined\HCode
% Link to standard \labels, but give your own description
% Usage:  Volg \hyperref[my_very_verbose_label]{deze link} voor wat tijdverlies
%   (01/2020: Ximera-server aangepast om bij class reference-keeptext de link-text NIET te vervangen door de label-text !!!) 
\renewcommand{\hyperref}[2][]{\HCode{<a class="reference reference-keeptext" href="\##1">}#2\HCode{</a>}}
%
%  Link to specific targets  (not tested ?)
\renewcommand{\hypertarget}[1]{\HCode{<a class="ximera-label" id="#1"></a>}}
\renewcommand{\hyperlink}[2]{\HCode{<a class="reference reference-keeptext" href="\##1">}#2\HCode{</a>}}
\fi

% Mmm, quid English ... (for keyword #1 !) ?
\newcommand{\wikilink}[2]{
    \hyperlink{https://nl.wikipedia.org/wiki/#1}{#2}
    \pdfOnly{\footnote{See \url{https://nl.wikipedia.org/wiki/#1}}
    }
}

\renewcommand{\figurename}{Figuur}
\renewcommand{\tablename}{Tabel}

%
% Gedoe om verschillende versies van xourse/ximera te maken afhankelijk van settings
%
% default: versie met antwoorden
% handout: versie voor de studenten, zonder antwoorden/oplossingen
% full: met alles erop en eraan, dus geschikt voor auteurs en/of lesgevers  (bevat in de pdf ook de 'online-only' stukken!)
%
%
% verder kunnen ook opties/variabele worden gezet voor hints/auteurs/uitweidingen/ etc
%
% 'Full' versie
\newtoggle{showonline}
\ifdefined\HCode   % zet default showOnline
    \toggletrue{showonline} 
\else
    \togglefalse{showonline}
\fi

% Full versie   % deprecated: see infra
\newcommand{\printFull}{
    \hintstrue
    \handoutfalse
    \toggletrue{showonline} 
}

\ifdefined\shouldPrintFull   % deprecated: see infra
    \printFull
\fi



% Overschrijf onlineOnly  (zoals gedefinieerd in ximera.cls)
\ifhandout   % in handout: gebruik de oorspronkelijke ximera.cls implementatie  (is dit wel nodig/nuttig?)
\else
    \iftoggle{showonline}{%
        \ifdefined\HCode
          \RenewEnviron{onlineOnly}{\bgroup\BODY\egroup}   % showOnline, en we zijn  online, dus toon de tekst
        \else
          \RenewEnviron{onlineOnly}{\bgroup\color{red!50!black}\BODY\egroup}   % showOnline, maar we zijn toch niet online: kleur de tekst rood 
        \fi
    }{%
      \RenewEnviron{onlineOnly}{}  % geen showOnline
    }
\fi

% hack om na hoofding van definition/proposition/... als dan niet op een nieuwe lijn te starten
% soms is dat goed en mooi, en soms niet; en in HTML is het nu (2/2020) anders dan in pdf
% vandaar suggestie om 
%     \begin{definition}[Nieuw concept] \nl
% te gebruiken als je zeker een newline wil na de hoofdig en titel
% (in het bijzonder itemize zonder \nl is 'lelijk' ...)
\ifdefined\HCode
\newcommand{\nl}{}
\else
\newcommand{\nl}{\ \par} % newline (achter heading van definition etc.)
\fi


% \nl enkel in handoutmode (ihb voor \wordChoice, die dan typisch veeeel langer wordt)
\ifdefined\HCode
\providecommand{\handoutnl}{}
\else
\providecommand{\handoutnl}{%
\ifhandout%
  \nl%
\fi%
}
\fi

% Could potentially replace \pdfOnline/\begin{onlineOnly} : 
% Usage= \ifonline{Hallo surfer}{Hallo PDFlezer}
\providecommand{\ifonline}[2]%
{
\begin{onlineOnly}#1\end{onlineOnly}%
\pdfOnly{#2}
}%


%
% Maak optionele 'basic' en 'extended' versies van een activity
%  met environment basicOnly, basicSkip en extendedOnly
%
%  (
%   Dit werkt ENKEL in de PDF; de online versies tonen (minstens voorklopig) steeds 
%   het default geval met printbasicversion en printextendversion beide FALSE
%  )
%
\providebool{printbasicversion}
\providebool{printextendedversion}   % not properly implemented
\providebool{printfullversion}       % presumably print everything (debug/auteur)
%
% only set these in xourses, and BEFORE loading this preamble
%
%\newif\ifshowbasic     \showbasictrue        % use this line in xourse to show 'basic' sections
%\newif\ifshowextended  \showextendedtrue     % use this line in xourse to show 'extended' sections
%
%
%\ifbool{showbasic}
%      { \NewEnviron{basicOnly}{\BODY} }    % if yes: just print contents
%      { \NewEnviron{basicOnly}{}      }    % if no:  completely ignore contents
%
%\ifbool{showbasic}
%      { \NewEnviron{basicSkip}{}      }
%      { \NewEnviron{basicSkip}{\BODY} }
%

\ifbool{printextendedversion}
      { \NewEnviron{extendedOnly}{\BODY} }
      { \NewEnviron{extendedOnly}{}      }
      


\ifdefined\HCode    % in html: always print
      {\newenvironment*{basicOnly}{}{}}    % if yes: just print contents
      {\newenvironment*{basicSkip}{}{}}    % if yes: just print contents
\else

\ifbool{printbasicversion}
      {\newenvironment*{basicOnly}{}{}}    % if yes: just print contents
      {\NewEnviron{basicOnly}{}      }    % if no:  completely ignore contents

\ifbool{printbasicversion}
      {\NewEnviron{basicSkip}{}      }
      {\newenvironment*{basicSkip}{}{}}

\fi

\usepackage{float}
\usepackage[rightbars,color]{changebar}

% Full versie
\ifbool{printfullversion}{
    \hintstrue
    \handoutfalse
    \toggletrue{showonline}
    \printbasicversionfalse
    \cbcolor{red}
    \renewenvironment*{basicOnly}{\cbstart}{\cbend}
    \renewenvironment*{basicSkip}{\cbstart}{\cbend}
    \def\xmtoonprintopties{FULL}   % will be printed in footer
}
{}
      
%
% Evalueer \ifhints IN de environment
%  
%
%\RenewEnviron{hint}
%{
%\ifhandout
%\ifhints\else\setbox0\vbox\fi%everything in een emty box
%\bgroup 
%\stepcounter{hintLevel}
%\BODY
%\egroup\ignorespacesafterend
%\addtocounter{hintLevel}{-1}
%\else
%\ifhints
%\begin{trivlist}\item[\hskip \labelsep\small\slshape\bfseries Hint:\hspace{2ex}]
%\small\slshape
%\stepcounter{hintLevel}
%\BODY
%\end{trivlist}
%\addtocounter{hintLevel}{-1}
%\fi
%\fi
%}

% Onafhankelijk van \ifhandout ...? TO BE VERIFIED
\RenewEnviron{hint}
{
\ifhints
\begin{trivlist}\item[\hskip \labelsep\small\bfseries Hint:\hspace{2ex}]
\small%\slshape
\stepcounter{hintLevel}
\BODY
\end{trivlist}
\addtocounter{hintLevel}{-1}
\else
\iftikzexport   % anders worden de tikz tekeningen in hints niet gegenereerd ?
\setbox0\vbox\bgroup
\stepcounter{hintLevel}
\BODY
\egroup\ignorespacesafterend
\addtocounter{hintLevel}{-1}
\fi % ifhandout
\fi %ifhints
}

%
% \tab sets typewriter-tabs (e.g. to format questions)
% (Has no effect in HTML :-( ))
%
\usepackage{tabto}
\ifdefined\HCode
  \renewcommand{\tab}{\quad}    % otherwise dummy .png's are generated ...?
\fi


% Also redefined in  preamble to get correct styling 
% for tikz images for (\tikzexport)
%

\theoremstyle{definition} % Bold titels
\makeatletter
\let\proposition\relax
\let\c@proposition\relax
\let\endproposition\relax
\makeatother
\newtheorem{proposition}{Eigenschap}


%\instructornotesfalse

% logic with \ifhandoutin ximera.cls unclear;so overwrite ...
\makeatletter
\@ifundefined{ifinstructornotes}{%
  \newif\ifinstructornotes
  \instructornotesfalse
  \newenvironment{instructorNotes}{}{}
}{}
\makeatother
\ifinstructornotes
\else
\renewenvironment{instructorNotes}%
{%
    \setbox0\vbox\bgroup
}
{%
    \egroup
}
\fi

% \RedeclareMathOperator
% from https://tex.stackexchange.com/questions/175251/how-to-redefine-a-command-using-declaremathoperator
\makeatletter
\newcommand\RedeclareMathOperator{%
    \@ifstar{\def\rmo@s{m}\rmo@redeclare}{\def\rmo@s{o}\rmo@redeclare}%
}
% this is taken from \renew@command
\newcommand\rmo@redeclare[2]{%
    \begingroup \escapechar\m@ne\xdef\@gtempa{{\string#1}}\endgroup
    \expandafter\@ifundefined\@gtempa
    {\@latex@error{\noexpand#1undefined}\@ehc}%
    \relax
    \expandafter\rmo@declmathop\rmo@s{#1}{#2}}
% This is just \@declmathop without \@ifdefinable
\newcommand\rmo@declmathop[3]{%
    \DeclareRobustCommand{#2}{\qopname\newmcodes@#1{#3}}%
}
\@onlypreamble\RedeclareMathOperator
\makeatother


%
% Engelse vertaling, vooral in mathmode
%
% 1. Algemene procedure
%
\ifdefined\isEn
 \newcommand{\nlen}[2]{#2}
 \newcommand{\nlentext}[2]{\text{#2}}
 \newcommand{\nlentextbf}[2]{\textbf{#2}}
\else
 \newcommand{\nlen}[2]{#1}
 \newcommand{\nlentext}[2]{\text{#1}}
 \newcommand{\nlentextbf}[2]{\textbf{#1}}
\fi

%
% 2. Lijst van erg veel gebruikte uitdrukkingen
%

% Ja/Nee/Fout/Juits etc
%\newcommand{\TJa}{\nlentext{ Ja }{ and }}
%\newcommand{\TNee}{\nlentext{ Nee }{ No }}
%\newcommand{\TJuist}{\nlentext{ Juist }{ Correct }
%\newcommand{\TFout}{\nlentext{ Fout }{ Wrong }
\newcommand{\TWaar}{\nlentext{ Waar }{ True }}
\newcommand{\TOnwaar}{\nlentext{ Vals }{ False }}
% Korte bindwoorden en, of, dus, ...
\newcommand{\Ten}{\nlentext{ en }{ and }}
\newcommand{\Tof}{\nlentext{ of }{ or }}
\newcommand{\Tdus}{\nlentext{ dus }{ so }}
\newcommand{\Tendus}{\nlentext{ en dus }{ and thus }}
\newcommand{\Tvooralle}{\nlentext{ voor alle }{ for all }}
\newcommand{\Took}{\nlentext{ ook }{ also }}
\newcommand{\Tals}{\nlentext{ als }{ when }} %of if?
\newcommand{\Twant}{\nlentext{ want }{ as }}
\newcommand{\Tmaal}{\nlentext{ maal }{ times }}
\newcommand{\Toptellen}{\nlentext{ optellen }{ add }}
\newcommand{\Tde}{\nlentext{ de }{ the }}
\newcommand{\Thet}{\nlentext{ het }{ the }}
\newcommand{\Tis}{\nlentext{ is }{ is }} %zodat is in text staat in mathmode (geen italics)
\newcommand{\Tmet}{\nlentext{ met }{ where }} % in situaties e.g met p < n --> where p < n
\newcommand{\Tnooit}{\nlentext{ nooit }{ never }}
\newcommand{\Tmaar}{\nlentext{ maar }{ but }}
\newcommand{\Tniet}{\nlentext{ niet }{ not }}
\newcommand{\Tuit}{\nlentext{ uit }{ from }}
\newcommand{\Ttov}{\nlentext{ t.o.v. }{ w.r.t. }}
\newcommand{\Tzodat}{\nlentext{ zodat }{ such that }}
\newcommand{\Tdeth}{\nlentext{de }{th }}
\newcommand{\Tomdat}{\nlentext{omdat }{because }} 


%
% Overschrijf addhoc commando's
%
\ifdefined\isEn
\renewcommand{\pernot}{\overset{\mathrm{notation}}{=}}
\RedeclareMathOperator{\bld}{im}     % beeld
\RedeclareMathOperator{\graf}{graph}   % grafiek
\RedeclareMathOperator{\rico}{slope}   % richtingcoëfficient
\RedeclareMathOperator{\co}{co}       % coordinaat
\RedeclareMathOperator{\gr}{deg}       % graad

% Operators
\RedeclareMathOperator{\bgsin}{arcsin}
\RedeclareMathOperator{\bgcos}{arccos}
\RedeclareMathOperator{\bgtan}{arctan}
\RedeclareMathOperator{\bgcot}{arccot}
\RedeclareMathOperator{\bgsinh}{arcsinh}
\RedeclareMathOperator{\bgcosh}{arccosh}
\RedeclareMathOperator{\bgtanh}{arctanh}
\RedeclareMathOperator{\bgcoth}{arccoth}

\fi


% HACK: use 'oplossing' for 'explanation' ...
\let\explanation\relax
\let\endexplanation\relax
% \newenvironment{explanation}{\begin{oplossing}}{\end{oplossing}}
\newcounter{explanation}

\ifhandout%
    \NewEnviron{explanation}[1][toon]%
    {%
    \RenewEnviron{verbatim}{ \red{VERBATIM CONTENT MISSING IN THIS PDF}} %% \expandafter\verb|\BODY|}

    \ifthenelse{\equal{\detokenize{#1}}{\detokenize{toon}}}
    {
    \def\PH@Command{#1}% Use PH@Command to hold the content and be a target for "\expandafter" to expand once.

    \begin{trivlist}% Begin the trivlist to use formating of the "Feedback" label.
    \item[\hskip \labelsep\small\slshape\bfseries Explanation:% Format the "Feedback" label. Don't forget the space.
    %(\texttt{\detokenize\expandafter{\PH@Command}}):% Format (and detokenize) the condition for feedback to trigger
    \hspace{2ex}]\small%\slshape% Insert some space before the actual feedback given.
    \BODY
    \end{trivlist}
    }
    {  % \begin{feedback}[solution]   \BODY     \end{feedback}  }
    }
    }    
\else
% ONLY for HTML; xmoplossing is styled with css, and is not, and need not be a LaTeX environment
% THUS: it does NOT use feedback anymore ...
%    \NewEnviron{oplossing}{\begin{expandable}{xmoplossing}{\nlen{Toon uitwerking}{Show solution}}{\BODY}\end{expandable}}
    \newenvironment{explanation}[1][toon]
   {%
       \begin{expandable}{xmoplossing}{}
   }
   {%
   	   \end{expandable}
   } 
\fi

\title{Additional Exercises for Ch 7} \license{CC BY-NC-SA 4.0}

\begin{document}

\begin{abstract}
\end{abstract}
\maketitle

\section*{Exercises for Ch 7 Determinants}

\begin{problem}\label{prb:7.1} Find the determinants of the following matrices.

\begin{enumerate}
\item $\left[
\begin{array}{rr}
1 & 3 \\
0 & 2
\end{array}
\right]$

\item $\left[
\begin{array}{rr}
0 & 3 \\
0 & 2
\end{array}
\right]$

\item $\left[
\begin{array}{rr}
4 & 3 \\
6 & 2
\end{array}
\right]$
\end{enumerate}
%\begin{hint}
%\begin{enumerate}
%\item
%\end{enumerate}
%\end{hint}
\end{problem}

\begin{problem}\label{prb:7.2} Let $A = \left[ \begin{array}{rrr}
1 & 2 & 4 \\
0 & 1 & 3 \\
-2 & 5 & 1
\end{array} \right]$. Find the following.
\begin{enumerate}
\item $minor(A)_{11}$
\item $minor(A)_{21}$
\item $minor(A)_{32}$
\item $\mbox{cof}(A)_{11}$
\item $\mbox{cof}(A)_{21}$
\item $\mbox{cof}(A)_{32}$
\end{enumerate}
%\begin{hint}
%\begin{enumerate}
%\item
%\end{enumerate}
%\end{hint}
\end{problem}

\begin{problem}\label{prb:7.3} Find the determinants of the following matrices.
\begin{enumerate}
\item $\left[
\begin{array}{rrr}
1 & 2 & 3 \\
3 & 2 & 2 \\
0 & 9 & 8
\end{array}
\right] $
\item $\left[
\begin{array}{rrr}
4 & 3 & 2 \\
1 & 7 & 8 \\
3 & -9 & 3
\end{array}
\right] $
\item $\left[
\begin{array}{rrrr}
1 & 2 & 3 & 2 \\
1 & 3 & 2 & 3 \\
4 & 1 & 5 & 0 \\
1 & 2 & 1 & 2
\end{array}
\right] $
\end{enumerate}
\begin{hint}
\begin{enumerate}
\item The answer is $31$.
\item The answer is $375$.
\item The answer is $-2$.
\end{enumerate}
\end{hint}
\end{problem}

\begin{problem}\label{prb:7.4} Find the following determinant by expanding along the first row and
second column.
\begin{equation*}
\left|
\begin{array}{rrr}
1 & 2 & 1 \\
2 & 1 & 3 \\
2 & 1 & 1
\end{array}
\right|
\end{equation*}
\begin{hint}
\[
\left|
\begin{array}{ccc}
1 & 2 & 1 \\
2 & 1 & 3 \\
2 & 1 & 1
\end{array}
\right| =  6
\]
\end{hint}
\end{problem}

\begin{problem}\label{prb:7.5} Find the following determinant by expanding along the first column and
third row.
\begin{equation*}
\left|
\begin{array}{rrr}
1 & 2 & 1 \\
1 & 0 & 1 \\
2 & 1 & 1
\end{array}
\right|
\end{equation*}
\begin{hint}
\[
\left|
\begin{array}{ccc}
1 & 2 & 1 \\
1 & 0 & 1 \\
2 & 1 & 1
\end{array}
\right| =  2
\]
\end{hint}
\end{problem}

\begin{problem}\label{prb:7.6} Find the following determinant by expanding along the second row and
first column.
\begin{equation*}
\left|
\begin{array}{rrr}
1 & 2 & 1 \\
2 & 1 & 3 \\
2 & 1 & 1
\end{array}
\right|
\end{equation*}
\begin{hint}
\[
\left|
\begin{array}{ccc}
1 & 2 & 1 \\
2 & 1 & 3 \\
2 & 1 & 1
\end{array}
\right| = 6
\]
\end{hint}
\end{problem}

\begin{problem}\label{prb:7.7} Compute the determinant by cofactor expansion. Pick the easiest row or
column to use.
\begin{equation*}
\left|
\begin{array}{rrrr}
1 & 0 & 0 & 1 \\
2 & 1 & 1 & 0 \\
0 & 0 & 0 & 2 \\
2 & 1 & 3 & 1
\end{array}
\right|
\end{equation*}
\begin{hint}
\[
\left|
\begin{array}{cccc}
1 & 0 & 0 & 1 \\
2 & 1 & 1 & 0 \\
0 & 0 & 0 & 2 \\
2 & 1 & 3 & 1
\end{array}
\right| = -4
\]
\end{hint}
\end{problem}

\begin{problem}\label{prb:7.8} Find the determinant of the following matrices.

\begin{enumerate}
\item
$A = \left[ \begin{array}{rr}
1 & -34 \\
0 & 2
\end{array} \right] $

\item
$A = \left[ \begin{array}{rrr}
4 & 3 & 14 \\
 0 & -2 & 0 \\
0 & 0 & 5
\end{array} \right]$

\item
$A = \left[ \begin{array}{rrrr}
2 & 3 & 15 & 0 \\
0 & 4 & 1 & 7 \\
0 & 0 & -3 & 5 \\
0 & 0 & 0 & 1
\end{array} \right]$
\end{enumerate}

%\begin{hint}
%\begin{enumerate}
%\item
%\end{enumerate}
%\end{hint}

\end{problem}

\begin{problem}\label{prb:7.9} An operation is done to get from the first matrix to the second.
Identify what was done and tell how it will affect the value of the
determinant.
\begin{equation*}
\left[
\begin{array}{cc}
a & b \\
c & d
\end{array}
\right]  \rightarrow \cdots \rightarrow \left[
\begin{array}{cc}
a & c \\
b & d
\end{array}
\right]
\end{equation*}
\begin{hint}
It does not change the determinant. This was just taking the transpose.
\end{hint}
\end{problem}

\begin{problem}\label{prb:7.10} An operation is done to get from the first matrix to the second.
Identify what was done and tell how it will affect the value of the
determinant.
\begin{equation*}
\left[
\begin{array}{cc}
a & b \\
c & d
\end{array}
\right] \rightarrow \cdots \rightarrow \left[
\begin{array}{cc}
c & d \\
a & b
\end{array}
\right]
\end{equation*}
\begin{hint}
In this case two rows were switched and so the resulting determinant is $-1$
times the first.
\end{hint}
\end{problem}


\begin{problem}\label{prb:7.11} An operation is done to get from the first matrix to the second.
Identify what was done and tell how it will affect the value of the
determinant.
\begin{equation*}
\left[
\begin{array}{cc}
a & b \\
c & d
\end{array}
\right] \rightarrow \cdots \rightarrow \left[
\begin{array}{cc}
a & b \\
a+c & b+d
\end{array}
\right]
\end{equation*}
\begin{hint}
The determinant is unchanged. It was just the first row added to the second.
\end{hint}
\end{problem}


\begin{problem}\label{prb:7.12} An operation is done to get from the first matrix to the second.
Identify what was done and tell how it will affect the value of the
determinant.
\begin{equation*}
\left[
\begin{array}{cc}
a & b \\
c & d
\end{array}
\right] \rightarrow \cdots \rightarrow \left[
\begin{array}{cc}
a & b \\
2c & 2d
\end{array}
\right]
\end{equation*}
\begin{hint}
The second row was multiplied by 2 so the determinant of the result is 2
times the original determinant.
\end{hint}
\end{problem}

\begin{problem}\label{prb:7.13} An operation is done to get from the first matrix to the second.
Identify what was done and tell how it will affect the value of the
determinant.
\begin{equation*}
\left[
\begin{array}{cc}
a & b \\
c & d
\end{array}
\right] \rightarrow \cdots \rightarrow \left[
\begin{array}{cc}
b & a \\
d & c
\end{array}
\right]
\end{equation*}
\begin{hint}
In this case the two columns were switched so the determinant of the second
is $-1$ times the determinant of the first.
\end{hint}
\end{problem}


\begin{problem}\label{prb:7.14} Let $A$ be an $r\times r$ matrix and suppose there are $r-1$ rows
(columns) such that all rows (columns) are linear combinations of these $r-1$
rows (columns). Show $\det \left( A\right) =0.$
\begin{hint}
If the determinant is nonzero, then it will remain nonzero with row operations applied to the matrix.
However, by assumption, you can obtain a row of zeros by doing row
operations. Thus the determinant must have been zero after all.
\end{hint}
\end{problem}

\begin{problem}\label{prb:7.15} Show $\det \left( aA\right) =a^{n}\det \left( A\right) $ for an $n \times n $ matrix $A
$ and scalar $a$.
\begin{hint}
$\det \left( aA\right) =\det
\left( aIA\right) =\det \left( aI\right) \det \left( A\right) =a^{n}\det
\left( A\right) .$ The matrix which has $a$ down the main diagonal has
determinant equal to $a^{n}$.
\end{hint}
\end{problem}


\begin{problem}\label{prb:7.16} Construct $2\times 2$ matrices $A$ and $B$ to show that the
$\det A \det B = \det (AB)$.
\begin{hint}
\[
\det
\left( \left[
\begin{array}{cc}
1 & 2 \\
3 & 4
\end{array}
\right] \left[
\begin{array}{rr}
-1 & 2 \\
-5 & 6
\end{array}
\right] \right) = -8
\]
\[
\det \left[
\begin{array}{cc}
1 & 2 \\
3 & 4
\end{array}
\right] \det \left[
\begin{array}{rr}
-1 & 2 \\
-5 & 6
\end{array}
\right] = -2 \times 4 = -8
\]
\end{hint}
\end{problem}

\begin{problem}\label{prb:7.17} Is it true that $\det \left( A+B\right) =\det \left( A\right) +\det
\left( B\right) ?$ If this is so, explain why. If it is not so,
give a counter example.
\begin{hint}
This is not true at all. Consider $A=\left[
\begin{array}{cc}
1 & 0 \\
0 & 1
\end{array}
\right] ,B=\left[
\begin{array}{rr}
-1 & 0 \\
0 & -1
\end{array}
\right] .$
\end{hint}
\end{problem}

\begin{problem}\label{prb:7.18} An $n\times n$ matrix is called \textbf{nilpotent}
\index{nilpotent} if for some positive integer, $k$ it follows $A^{k}=0.$ If
$A$ is a nilpotent matrix and $k$ is the smallest possible integer such that
$A^{k}=0,$ what are the possible values of $\det \left( A\right) ?$
\begin{hint}
It must
be 0 because $0=\det \left( 0\right) =\det \left( A^{k}\right) =\left( \det
\left( A\right) \right) ^{k}.$
\end{hint}
\end{problem}

\begin{problem}\label{prb:7.19} \label{exerorthogonal}A matrix is said to be \textbf{orthogonal} \index{matrix!orthogonal} if
$A^{T}A=I.$ Thus the inverse of an orthogonal matrix is just its transpose.
What are the possible values of $\det \left( A\right) $ if $A$ is an
orthogonal matrix?
\begin{hint}
You would need $\det \left( AA^{T}\right) =\det
\left( A\right) \det \left( A^{T}\right) =\det \left( A\right) ^{2}=1$ and
so $\det \left( A\right) =1,$ or $-1$.
\end{hint}
\end{problem}

\begin{problem}\label{prb:7.20} Let $A$ and $B$ be two $n\times n$ matrices. $A\sim B$
($A$ is \textbf{similar} to $B$) means there exists an invertible matrix $P$
such that $A=P^{-1}BP.$ Show that if $A\sim B,$ then
$\det \left( A\right) =\det \left( B\right) .$
\begin{hint}
$\det \left( A\right) =\det
\left( S^{-1}BS\right) =\det \left( S^{-1}\right) \det \left( B\right) \det
\left( S\right) =\det \left( B\right) \det \left( S^{-1}S\right) =\det
\left( B\right) $.
\end{hint}
\end{problem}

\begin{problem}\label{prb:7.21} Tell whether each statement is true or false. If true, provide a proof. If false, provide a counter example.
\begin{enumerate}
\item If $A$ is a $3\times 3$ matrix with a zero determinant, then one
column must be a multiple of some other column.

\item If any two columns of a square matrix are equal, then the determinant
of the matrix equals zero.

\item For two $n\times n$ matrices $A$ and $B$, $\det \left( A+B\right)
=\det \left( A\right) +\det \left( B\right) .$

\item For an $n\times n$ matrix $A$, $\det \left( 3A\right) =3\det \left(
A\right) $

\item If $A^{-1}$ exists then $\det \left( A^{-1}\right) =\det \left(
A\right) ^{-1}.$

\item If $B$ is obtained by multiplying a single row of $A$ by $4$ then $%
\det \left( B\right) =4\det \left( A\right) .$

\item For $A$ an $n\times n$ matrix, $\det \left( -A\right) =\left(
-1\right) ^{n}\det \left( A\right) .$

\item If $A$ is a real $n\times n$ matrix, then $\det \left( A^{T}A\right)
\geq 0.$

\item If $A^{k}=0$ for some positive integer $k,$ then $\det \left(
A\right) =0.$

\item If $AX=0$ for some $X \neq 0,$ then $\det \left(
A\right) =0.$
\end{enumerate}
\begin{hint}
\begin{enumerate}
\item False. Consider $\left[
\begin{array}{rrr}
1 & 1 & 2 \\
-1 & 5 & 4 \\
0 & 3 & 3
\end{array}
\right] $
\item True.
\item False.
\item False.
\item True.
\item True.
\item True.
\item True.
\item True.
\item True.
\end{enumerate}
\end{hint}
\end{problem}

\begin{problem}\label{prb:7.22} Find the determinant using row operations to first simplify.
\begin{equation*}
\left|
\begin{array}{rrr}
1 & 2 & 1 \\
2 & 3 & 2 \\
-4 & 1 & 2
\end{array}
\right|
\end{equation*}
\begin{hint}
\[
\left|
\begin{array}{rrr}
1 & 2 & 1 \\
2 & 3 & 2 \\
-4 & 1 & 2
\end{array}
\right| = -6
\]
\end{hint}
\end{problem}

\begin{problem}\label{prb:7.23} Find the determinant using row operations to first simplify.
\begin{equation*}
\left|
\begin{array}{rrr}
2 & 1 & 3 \\
2 & 4 & 2 \\
1 & 4 & -5
\end{array}
\right|
\end{equation*}
\begin{hint}
\[
\left|
\begin{array}{rrr}
2 & 1 & 3 \\
2 & 4 & 2 \\
1 & 4 & -5
\end{array}
\right| = -32
\]
\end{hint}
\end{problem}

\begin{problem}\label{prb:7.24} Find the determinant using row operations to first simplify.
\begin{equation*}
\left|
\begin{array}{rrrr}
1 & 2 & 1 & 2 \\
3 & 1 & -2 & 3 \\
-1 & 0 & 3 & 1 \\
2 & 3 & 2 & -2
\end{array}
\right|
\end{equation*}
\begin{hint}
One can row reduce this using only row operation 3 to
\[
\left[
\begin{array}{rrrr}
1 & 2 & 1 & 2 \\
0 & -5 & -5 & -3 \\
0 & 0 & 2 & \frac{9}{5} \\
0 & 0 & 0 & -\frac{63}{10}
\end{array}
\right]
\]
and therefore, the determinant is $-63.$
\[
\left|
\begin{array}{rrrr}
1 & 2 & 1 & 2 \\
3 & 1 & -2 & 3 \\
-1 & 0 & 3 & 1 \\
2 & 3 & 2 & -2
\end{array}
\right| = 63
\]
\end{hint}
\end{problem}

\begin{problem}\label{prb:7.25} Find the determinant using row operations to first simplify.
\begin{equation*}
\left|
\begin{array}{rrrr}
1 & 4 & 1 & 2 \\
3 & 2 & -2 & 3 \\
-1 & 0 & 3 & 3 \\
2 & 1 & 2 & -2
\end{array}
\right|
\end{equation*}
\begin{hint}
One can row reduce this using only row operation 3 to$\allowbreak $%
\[
\left[
\begin{array}{rrrr}
1 & 4 & 1 & 2 \\
0 & -10 & -5 & -3 \\
0 & 0 & 2 & \frac{19}{5} \\
0 & 0 & 0 & -\frac{211}{20}
\end{array}
\right]
\]
Thus the determinant is given by
\[
\left|
\begin{array}{rrrr}
1 & 4 & 1 & 2 \\
3 & 2 & -2 & 3 \\
-1 & 0 & 3 & 3 \\
2 & 1 & 2 & -2
\end{array}
\right| = 211
\]
\end{hint}
\end{problem}

\begin{problem}\label{prb:7.26} Let
\begin{equation*}
A=
\left[
\begin{array}{rrr}
1 & 2 & 3 \\
0 & 2 & 1 \\
3 & 1 & 0
\end{array}
\right]
\end{equation*}
Determine whether the matrix $A$ has an inverse by finding whether the
determinant is non zero. If the determinant is nonzero, find the inverse
using the formula for the inverse which involves the cofactor matrix.
\begin{hint}
$\det
\left[
\begin{array}{ccc}
1 & 2 & 3 \\
0 & 2 & 1 \\
3 & 1 & 0
\end{array}
\right] = -13$ and so it has an inverse. This inverse is
\begin{eqnarray*}
\frac{1}{-13}\left[
\begin{array}{rrr}
\left\vert
\begin{array}{cc}
2 & 1 \\
1 & 0
\end{array}
\right\vert  & -\left\vert
\begin{array}{cc}
0 & 1 \\
3 & 0
\end{array}
\right\vert  & \left\vert
\begin{array}{cc}
0 & 2 \\
3 & 1
\end{array}
\right\vert  \\
-\left\vert
\begin{array}{cc}
2 & 3 \\
1 & 0
\end{array}
\right\vert  & \left\vert
\begin{array}{cc}
1 & 3 \\
3 & 0
\end{array}
\right\vert  & -\left\vert
\begin{array}{cc}
1 & 2 \\
3 & 1
\end{array}
\right\vert  \\
\left\vert
\begin{array}{cc}
2 & 3 \\
2 & 1
\end{array}
\right\vert  & -\left\vert
\begin{array}{cc}
1 & 3 \\
0 & 1
\end{array}
\right\vert  & \left\vert
\begin{array}{cc}
1 & 2 \\
0 & 2
\end{array}
\right\vert
\end{array}
\right] ^{T} &=&\frac{1}{-13}\left[
\begin{array}{rrr}
-1 & 3 & -6 \\
3 & -9 & 5 \\
-4 & -1 & 2
\end{array}
\right] ^{T} \\
&=& \left[
\begin{array}{rrr}
\vspace{0.05in}\frac{1}{13} & -\vspace{0.05in}\frac{3}{13} & \vspace{0.05in}\frac{4}{13} \\
-\vspace{0.05in}\frac{3}{13} & \vspace{0.05in}\frac{9}{13} & \vspace{0.05in}\frac{1}{13} \\
\vspace{0.05in}\frac{6}{13} & -\vspace{0.05in}\frac{5}{13} & -\vspace{0.05in}\frac{2}{13}
\end{array}
\right]
\end{eqnarray*}
\end{hint}
\end{problem}

\begin{problem}\label{prb:7.27} Let
\begin{equation*}
A=
\left[
\begin{array}{rrr}
1 & 2 & 0 \\
0 & 2 & 1 \\
3 & 1 & 1
\end{array}
\right]
\end{equation*}
Determine whether the matrix $A$ has an inverse by finding whether the
determinant is non zero. If the determinant is nonzero, find the inverse
using the formula for the inverse.
\begin{hint}
$\det
\left[
\begin{array}{ccc}
1 & 2 & 0 \\
0 & 2 & 1 \\
3 & 1 & 1
\end{array}
\right] = 7$ so it has an inverse. This inverse is $\frac{1}{7}
\left[
\begin{array}{rrr}
1 & 3 & -6 \\
-2 & 1 & 5 \\
2 & -1 & 2
\end{array}
\right]^{T} = \left[
\begin{array}{rrr}
\vspace{0.05in}\frac{1}{7} & -\vspace{0.05in}\frac{2}{7} & \vspace{0.05in}\frac{2}{7} \\
\vspace{0.05in}\frac{3}{7} & \vspace{0.05in}\frac{1}{7} & -\vspace{0.05in}\frac{1}{7} \\
-\vspace{0.05in}\frac{6}{7} & \vspace{0.05in}\frac{5}{7} & \vspace{0.05in}\frac{2}{7}
\end{array}
\right] $
\end{hint}
\end{problem}

\begin{problem}\label{prb:7.28} Let
\begin{equation*}
A=
\left[
\begin{array}{rrr}
1 & 3 & 3 \\
2 & 4 & 1 \\
0 & 1 & 1
\end{array}
\right]
\end{equation*}
Determine whether the matrix $A$ has an inverse by finding whether the
determinant is non zero. If the determinant is nonzero, find the inverse
using the formula for the inverse.
\begin{hint}
\[
\det \left[
\begin{array}{ccc}
1 & 3 & 3 \\
2 & 4 & 1 \\
0 & 1 & 1
\end{array}
\right] = 3
\]
so it has an inverse which is
\[
\left[
\begin{array}{rrr}
1 & 0 & -3 \\
-\vspace{0.05in}\frac{2}{3} & \vspace{0.05in}\frac{1}{3} & \vspace{0.05in}\frac{5}{3} \\
\vspace{0.05in}\frac{2}{3} & -\vspace{0.05in}\frac{1}{3} & -\vspace{0.05in}\frac{2}{3}
\end{array}
\right]
\]
\end{hint}
\end{problem}

\begin{problem}\label{prb:7.29} Let
\begin{equation*}
A =
\left[
\begin{array}{rrr}
1 & 2 & 3 \\
0 & 2 & 1 \\
2 & 6 & 7
\end{array}
\right]
\end{equation*}
Determine whether the matrix $A$ has an inverse by finding whether the
determinant is non zero. If the determinant is nonzero, find the inverse
using the formula for the inverse.
%\begin{hint}
%\end{hint}
\end{problem}


\begin{problem}\label{prb:7.30} Let
\begin{equation*}
A =
\left[
\begin{array}{rrr}
1 & 0 & 3 \\
1 & 0 & 1 \\
3 & 1 & 0
\end{array}
\right]
\end{equation*}
Determine whether the matrix $A$ has an inverse by finding whether the
determinant is non zero. If the determinant is nonzero, find the inverse
using the formula for the inverse.
\begin{hint}
\[
\det \left[
\begin{array}{rrr}
1 & 0 & 3 \\
1 & 0 & 1 \\
3 & 1 & 0
\end{array}
\right] = 2
\]
and so it has an inverse. The inverse turns out to equal
\[
\left[
\begin{array}{rrr}
-\vspace{0.05in}\frac{1}{2} & \vspace{0.05in}\frac{3}{2} & 0 \\
\vspace{0.05in}\frac{3}{2} & -\vspace{0.05in}\frac{9}{2} & 1 \\
\vspace{0.05in}\frac{1}{2} & -\vspace{0.05in}\frac{1}{2} & 0
\end{array}
\right]
\]
\end{hint}
\end{problem}

\begin{problem}\label{prb:7.31} For the following matrices, determine if they are invertible. If so, use the formula for the inverse in terms of the cofactor matrix to
find each inverse. If the inverse does not exist, explain why.
\begin{enumerate}
\item
$\left[
\begin{array}{rr}
1 & 1 \\
1 & 2
\end{array}
\right]$
\item
$\left[
\begin{array}{rrr}
1 & 2 & 3 \\
0 & 2 & 1 \\
4 & 1 & 1
\end{array}
\right]$
\item
$\left[
\begin{array}{rrr}
1 & 2 & 1 \\
2 & 3 & 0 \\
0 & 1 & 2
\end{array}
\right] $
\end{enumerate}
\begin{hint}
\begin{enumerate}
\item $\left\vert
\begin{array}{cc}
1 & 1 \\
1 & 2
\end{array}
\right\vert = 1$
\item $\left\vert
\begin{array}{ccc}
1 & 2 & 3 \\
0 & 2 & 1 \\
4 & 1 & 1%
\end{array}
\right\vert = -15$
\item $\left\vert
\begin{array}{ccc}
1 & 2 & 1 \\
2 & 3 & 0 \\
0 & 1 & 2
\end{array}
\right\vert = 0$
\end{enumerate}
\end{hint}
\end{problem}

\begin{problem}\label{prb:7.32} Consider the matrix
\begin{equation*}
A =
\left[
\begin{array}{ccc}
1 & 0 & 0 \\
0 & \cos t & -\sin t \\
0 & \sin t & \cos t
\end{array}
\right]
\end{equation*}
Does there exist a value of $t$ for which this matrix fails to have an
inverse? Explain.
\begin{hint}
 No. It has a nonzero determinant for all $t$
\end{hint}
\end{problem}


\begin{problem}\label{prb:7.33} Consider the matrix
\begin{equation*}
A =
\left[
\begin{array}{rrr}
1 & t & t^{2} \\
0 & 1 & 2t \\
t & 0 & 2
\end{array}
\right]
\end{equation*}
Does there exist a value of $t$ for which this matrix fails to have an
inverse? Explain.
\begin{hint}
\[
\det \left[
\begin{array}{ccc}
1 & t & t^{2} \\
0 & 1 & 2t \\
t & 0 & 2
\end{array}
\right] = t^{3}+2
\]
and so it has no inverse when $t=-\sqrt[3]{2}$
\end{hint}
\end{problem}

\begin{problem}\label{prb:7.34} Consider the matrix
\begin{equation*}
A =
\left[
\begin{array}{ccc}
e^{t} & \cosh t & \sinh t \\
e^{t} & \sinh t & \cosh t \\
e^{t} & \cosh t & \sinh t
\end{array}
\right]
\end{equation*}
Does there exist a value of $t$ for which this matrix fails to have an
inverse? Explain.
\begin{hint}
\[
\det \left[
\begin{array}{ccc}
e^{t} & \cosh t & \sinh t \\
e^{t} & \sinh t & \cosh t \\
e^{t} & \cosh t & \sinh t
\end{array}
\right] = 0
\]
and so this matrix fails to have a nonzero determinant at any value of $t$.
\end{hint}
\end{problem}

\begin{problem}\label{prb:7.35} Consider the matrix
\begin{equation*}
A =
\left[
\begin{array}{ccc}
e^{t} & e^{-t}\cos t & e^{-t}\sin t \\
e^{t} & -e^{-t}\cos t-e^{-t}\sin t & -e^{-t}\sin t+e^{-t}\cos t \\
e^{t} & 2e^{-t}\sin t & -2e^{-t}\cos t
\end{array}
\right]
\end{equation*}
Does there exist a value of $t$ for which this matrix fails to have an
inverse? Explain.
\begin{hint}
\[
\det \left[
\begin{array}{ccc}
e^{t} & e^{-t}\cos t & e^{-t}\sin t \\
e^{t} & -e^{-t}\cos t-e^{-t}\sin t & -e^{-t}\sin t+e^{-t}\cos t \\
e^{t} & 2e^{-t}\sin t & -2e^{-t}\cos t%
\end{array}
\right] = 5e^{-t} \neq 0
\]
and so this matrix is always invertible.
\end{hint}
\end{problem}

\begin{problem}\label{prb:7.36} \label{exerdeterminant3}Show that if $\det \left( A\right) \neq 0$ for $A$
an $n\times n$ matrix, it follows that if $AX=0,$ then $X=0$.
\begin{hint}
If $\det \left( A\right) \neq 0,$ then $A^{-1}$ exists and so you could
multiply on both sides on the left by $A^{-1}$ and obtain that $X=0$.
\end{hint}
\end{problem}

\begin{problem}\label{prb:7.37} Suppose $A,B$ are $n\times n$ matrices and that $AB=I.$ Show that then
$BA=I.$ \textbf{Hint:\ } First explain why
$\det \left( A\right) ,\det \left( B\right) $ are both nonzero. Then $\left(
AB\right) A=A$ and then show $BA\left( BA-I\right) =0.$ From this use what
is given to conclude $A\left( BA-I\right) =0.$ Then use Problem
\ref{prb:7.36}.
\begin{hint}
You have $1=\det \left( A\right) \det \left( B\right) $.
Hence both $A$ and $B$ have inverses. Letting $X$ be given,
\[
A\left( BA-I\right) X=\left( AB\right) AX-AX=AX-AX = 0
\]
and so it follows from the above problem that $\left( BA-I\right)X=0.$ Since $X$ is arbitrary, it follows that $BA=I.$
\end{hint}
\end{problem}

\begin{problem}\label{prb:7.38} Use the formula for the inverse in terms of the cofactor matrix to
find the inverse of the matrix
\begin{equation*}
A=\left[
\begin{array}{ccc}
e^{t} & 0 & 0 \\
0 & e^{t}\cos t & e^{t}\sin t \\
0 & e^{t}\cos t-e^{t}\sin t & e^{t}\cos t+e^{t}\sin t
\end{array}
\right]
\end{equation*}
\begin{hint}
\[
\det \left[
\begin{array}{ccc}
e^{t} & 0 & 0 \\
0 & e^{t}\cos t & e^{t}\sin t \\
0 & e^{t}\cos t-e^{t}\sin t & e^{t}\cos t+e^{t}\sin t
\end{array}
\right] = e^{3t}.
\]
Hence the inverse is
\begin{eqnarray*}
&&e^{-3t}\left[
\begin{array}{ccc}
e^{2t} & 0 & 0 \\
0 & e^{2t}\cos t+e^{2t}\sin t & -\left( e^{2t}\cos t-e^{2t}\sin \right) t \\
0 & -e^{2t}\sin t & e^{2t}\cos \left( t\right)
\end{array}
\right] ^{T} \\
&=& \left[
\begin{array}{ccc}
e^{-t} & 0 & 0 \\
0 & e^{-t}\left( \cos t+\sin t\right)  & -\left( \sin t\right) e^{-t} \\
0 & -e^{-t}\left( \cos t-\sin t\right)  & \left( \cos t\right) e^{-t}
\end{array}
\right]
\end{eqnarray*}
\end{hint}
\end{problem}

\begin{problem}\label{prb:7.39} Find the inverse, if it exists, of the matrix
\begin{equation*}
A =
\left[
\begin{array}{ccc}
e^{t} & \cos t & \sin t \\
e^{t} & -\sin t & \cos t \\
e^{t} & -\cos t & -\sin t
\end{array}
\right]
\end{equation*}
\begin{hint}
\begin{eqnarray*}
&&\left[
\begin{array}{ccc}
e^{t} & \cos t & \sin t \\
e^{t} & -\sin t & \cos t \\
e^{t} & -\cos t & -\sin t
\end{array}
\right] ^{-1} \\
&=&\left[
\begin{array}{ccc}
\frac{1}{2}e^{-t} & 0 & \frac{1}{2}e^{-t} \\
\frac{1}{2}\cos t+\frac{1}{2}\sin t & -\sin t & \frac{1}{2}\sin t-\frac{1}{2}
\cos t \\
\frac{1}{2}\sin t-\frac{1}{2}\cos t & \cos t & -\frac{1}{2}\cos t-\frac{1}{2}
\sin t
\end{array}
\right]
\end{eqnarray*}
\end{hint}
\end{problem}

\begin{problem}\label{prb:7.40} Suppose $A$ is an upper triangular matrix. Show that $A^{-1}$ exists
if and only if all elements of the main diagonal are non zero. Is it true
that $A^{-1}$ will also be upper triangular? Explain. Could the same be concluded for lower triangular matrices?
\begin{hint}
The given condition is what it takes for the
determinant to be non zero. Recall that the determinant of an upper
triangular matrix is just the product of the entries on the main diagonal.
\end{hint}
\end{problem}

\begin{problem}\label{prb:7.41} If $A,B,$ and $C$ are each $n\times n$ matrices and $ABC$ is
invertible, show why each of $A,B,$ and $C$ are invertible.
\begin{hint}
This follows
because $\det \left( ABC\right) =\det \left( A\right) \det \left( B\right)
\det \left( C\right) $ and if this product is nonzero, then each determinant
in the product is nonzero and so each of these matrices is invertible.
\end{hint}
\end{problem}

\section*{Practice Problem Source}
These problems come from Chapter 3 of Ken Kuttler's \href{https://open.umn.edu/opentextbooks/textbooks/a-first-course-in-linear-algebra-2017}{\it A First Course in Linear Algebra}. (CC-BY)

Ken Kuttler, {\it  A First Course in Linear Algebra}, Lyryx 2017, Open Edition, pp. 272--315.   

\end{document}