\documentclass{ximera}
%%% Begin Laad packages

\makeatletter
\@ifclassloaded{xourse}{%
    \typeout{Start loading preamble.tex (in a XOURSE)}%
    \def\isXourse{true}   % automatically defined; pre 112022 it had to be set 'manually' in a xourse
}{%
    \typeout{Start loading preamble.tex (NOT in a XOURSE)}%
}
\makeatother

\def\isEn\true 

\pgfplotsset{compat=1.16}

\usepackage{currfile}

% 201908/202301: PAS OP: babel en doclicense lijken problemen te veroorzaken in .jax bestand
% (wegens syntax error met toegevoegde \newcommands ...)
\pdfOnly{
    \usepackage[type={CC},modifier={by-nc-sa},version={4.0}]{doclicense}
    %\usepackage[hyperxmp=false,type={CC},modifier={by-nc-sa},version={4.0}]{doclicense}
    %%% \usepackage[dutch]{babel}
}



\usepackage[utf8]{inputenc}
\usepackage{morewrites}   % nav zomercursus (answer...?)
\usepackage{multirow}
\usepackage{multicol}
\usepackage{tikzsymbols}
\usepackage{ifthen}
%\usepackage{animate} BREAKS HTML STRUCTURE USED BY XIMERA
\usepackage{relsize}

\usepackage{eurosym}    % \euro  (€ werkt niet in xake ...?)
\usepackage{fontawesome} % smileys etc

% Nuttig als ook interactieve beamer slides worden voorzien:
\providecommand{\p}{} % default nothing ; potentially usefull for slides: redefine as \pause
%providecommand{\p}{\pause}

    % Layout-parameters voor het onderschrift bij figuren
\usepackage[margin=10pt,font=small,labelfont=bf, labelsep=endash,format=hang]{caption}
%\usepackage{caption} % captionof
%\usepackage{pdflscape}    % landscape environment

% Met "\newcommand\showtodonotes{}" kan je todonotes tonen (in pdf/online)
% 201908: online werkt het niet (goed)
\providecommand\showtodonotes{disable}
\providecommand\todo[1]{\typeout{TODO #1}}
%\usepackage[\showtodonotes]{todonotes}
%\usepackage{todonotes}

%
% Poging tot aanpassen layout
%
\usepackage{tcolorbox}
\tcbuselibrary{theorems}

%%% Einde laad packages

%%% Begin Ximera specifieke zaken

\graphicspath{
	{../../}
	{../}
	{./}
  	{../../pictures/}
   	{../pictures/}
   	{./pictures/}
	{./explog/}    % M05 in groeimodellen       
}

%%% Einde Ximera specifieke zaken

%
% define softer blue/red/green, use KU Leuven base colors for blue (and dark orange for red ?)
%
% todo: rather redefine blue/red/green ...?
%\definecolor{xmblue}{rgb}{0.01, 0.31, 0.59}
%\definecolor{xmred}{rgb}{0.89, 0.02, 0.17}
\definecolor{xmdarkblue}{rgb}{0.122, 0.671, 0.835}   % KU Leuven Blauw
\definecolor{xmblue}{rgb}{0.114, 0.553, 0.69}        % KU Leuven Blauw
\definecolor{xmgreen}{rgb}{0.13, 0.55, 0.13}         % No KULeuven variant for green found ...

\definecolor{xmaccent}{rgb}{0.867, 0.541, 0.18}      % KU Leuven Accent (orange ...)
\definecolor{kuaccent}{rgb}{0.867, 0.541, 0.18}      % KU Leuven Accent (orange ...)

\colorlet{xmred}{xmaccent!50!black}                  % Darker version of KU Leuven Accent

\providecommand{\blue}[1]{{\color{blue}#1}}    
\providecommand{\red}[1]{{\color{red}#1}}

\renewcommand\CancelColor{\color{xmaccent!50!black}}

% werkt in math en text mode om MATH met oranje (of grijze...)  achtergond te tonen (ook \important{\text{blabla}} lijkt te werken)
%\newcommand{\important}[1]{\ensuremath{\colorbox{xmaccent!50!white}{$#1$}}}   % werkt niet in Mathjax
%\newcommand{\important}[1]{\ensuremath{\colorbox{lightgray}{$#1$}}}
\newcommand{\important}[1]{\ensuremath{\colorbox{orange}{$#1$}}}   % TODO: kleur aanpassen voor mathjax; wordt overschreven infra!


% Uitzonderlijk kan met \pdfnl in de PDF een newline worden geforceerd, die online niet nodig/nuttig is omdat daar de regellengte hoe dan ook niet gekend is.
\ifdefined\HCode%
\providecommand{\pdfnl}{}%
\else%
\providecommand{\pdfnl}{%
  \\%
}%
\fi

% Uitzonderlijk kan met \handoutnl in de handout-PDF een newline worden geforceerd, die noch online noch in de PDF-met-antwoorden nuttig is.
\ifdefined\HCode
\providecommand{\handoutnl}{}
\else
\providecommand{\handoutnl}{%
\ifhandout%
  \nl%
\fi%
}
\fi



% \cellcolor IGNORED by tex4ht ?
% \begin{center} seems not to wordk
    % (missing margin-left: auto;   on tabular-inside-center ???)
%\newcommand{\importantcell}[1]{\ensuremath{\cellcolor{lightgray}#1}}  %  in tabular; usablility to be checked
\providecommand{\importantcell}[1]{\ensuremath{#1}}     % no mathjax2 support for colloring array cells

\pdfOnly{
  \renewcommand{\important}[1]{\ensuremath{\colorbox{kuaccent!50!white}{$#1$}}}
  \renewcommand{\importantcell}[1]{\ensuremath{\cellcolor{kuaccent!40!white}#1}}   
}

%%% Tikz styles


\pgfplotsset{compat=1.16}

\usetikzlibrary{trees,positioning,arrows,fit,shapes,math,calc,decorations.markings,through,intersections,patterns,matrix}

\usetikzlibrary{decorations.pathreplacing,backgrounds}    % 5/2023: from experimental


\usetikzlibrary{angles,quotes}

\usepgfplotslibrary{fillbetween} % bepaalde_integraal
\usepgfplotslibrary{polar}    % oa voor poolcoordinaten.tex

\pgfplotsset{ownstyle/.style={axis lines = center, axis equal image, xlabel = $x$, ylabel = $y$, enlargelimits}} 

\pgfplotsset{
	plot/.style={no marks,samples=50}
}

\newcommand{\xmPlotsColor}{
	\pgfplotsset{
		plot1/.style={darkgray,no marks,samples=100},
		plot2/.style={lightgray,no marks,samples=100},
		plotresult/.style={blue,no marks,samples=100},
		plotblue/.style={blue,no marks,samples=100},
		plotred/.style={red,no marks,samples=100},
		plotgreen/.style={green,no marks,samples=100},
		plotpurple/.style={purple,no marks,samples=100}
	}
}
\newcommand{\xmPlotsBlackWhite}{
	\pgfplotsset{
		plot1/.style={black,loosely dashed,no marks,samples=100},
		plot2/.style={black,loosely dotted,no marks,samples=100},
		plotresult/.style={black,no marks,samples=100},
		plotblue/.style={black,no marks,samples=100},
		plotred/.style={black,dotted,no marks,samples=100},
		plotgreen/.style={black,dashed,no marks,samples=100},
		plotpurple/.style={black,dashdotted,no marks,samples=100}
	}
}


\newcommand{\xmPlotsColorAndStyle}{
	\pgfplotsset{
		plot1/.style={darkgray,no marks,samples=100},
		plot2/.style={lightgray,no marks,samples=100},
		plotresult/.style={blue,no marks,samples=100},
		plotblue/.style={xmblue,no marks,samples=100},
		plotred/.style={xmred,dashed,thick,no marks,samples=100},
		plotgreen/.style={xmgreen,dotted,very thick,no marks,samples=100},
		plotpurple/.style={purple,no marks,samples=100}
	}
}


%\iftikzexport
\xmPlotsColorAndStyle
%\else
%\xmPlotsBlackWhite
%\fi
%%%


%
% Om venndiagrammen te arceren ...
%
\makeatletter
\pgfdeclarepatternformonly[\hatchdistance,\hatchthickness]{north east hatch}% name
{\pgfqpoint{-1pt}{-1pt}}% below left
{\pgfqpoint{\hatchdistance}{\hatchdistance}}% above right
{\pgfpoint{\hatchdistance-1pt}{\hatchdistance-1pt}}%
{
	\pgfsetcolor{\tikz@pattern@color}
	\pgfsetlinewidth{\hatchthickness}
	\pgfpathmoveto{\pgfqpoint{0pt}{0pt}}
	\pgfpathlineto{\pgfqpoint{\hatchdistance}{\hatchdistance}}
	\pgfusepath{stroke}
}
\pgfdeclarepatternformonly[\hatchdistance,\hatchthickness]{north west hatch}% name
{\pgfqpoint{-\hatchthickness}{-\hatchthickness}}% below left
{\pgfqpoint{\hatchdistance+\hatchthickness}{\hatchdistance+\hatchthickness}}% above right
{\pgfpoint{\hatchdistance}{\hatchdistance}}%
{
	\pgfsetcolor{\tikz@pattern@color}
	\pgfsetlinewidth{\hatchthickness}
	\pgfpathmoveto{\pgfqpoint{\hatchdistance+\hatchthickness}{-\hatchthickness}}
	\pgfpathlineto{\pgfqpoint{-\hatchthickness}{\hatchdistance+\hatchthickness}}
	\pgfusepath{stroke}
}
%\makeatother

\tikzset{
    hatch distance/.store in=\hatchdistance,
    hatch distance=10pt,
    hatch thickness/.store in=\hatchthickness,
   	hatch thickness=2pt
}

\colorlet{circle edge}{black}
\colorlet{circle area}{blue!20}


\tikzset{
    filled/.style={fill=green!30, draw=circle edge, thick},
    arceerl/.style={pattern=north east hatch, pattern color=blue!50, draw=circle edge},
    arceerr/.style={pattern=north west hatch, pattern color=yellow!50, draw=circle edge},
    outline/.style={draw=circle edge, thick}
}




%%% Updaten commando's
\def\hoofding #1#2#3{\maketitle}     % OBSOLETE ??

% we willen (bijna) altijd \geqslant ipv \geq ...!
\newcommand{\geqnoslant}{\geq}
\renewcommand{\geq}{\geqslant}
\newcommand{\leqnoslant}{\leq}
\renewcommand{\leq}{\leqslant}

% Todo: (201908) waarom komt er (soms) underlined voor emph ...?
\renewcommand{\emph}[1]{\textit{#1}}

% API commando's

\newcommand{\ds}{\displaystyle}
\newcommand{\ts}{\textstyle}  % tegenhanger van \ds   (Ximera zet PER  DEFAULT \ds!)

% uit Zomercursus-macro's: 
\newcommand{\bron}[1]{\begin{scriptsize} \emph{#1} \end{scriptsize}}     % deprecated ...?


%definities nieuwe commando's - afkortingen veel gebruikte symbolen
\newcommand{\R}{\ensuremath{\mathbb{R}}}
\newcommand{\Rnul}{\ensuremath{\mathbb{R}_0}}
\newcommand{\Reen}{\ensuremath{\mathbb{R}\setminus\{1\}}}
\newcommand{\Rnuleen}{\ensuremath{\mathbb{R}\setminus\{0,1\}}}
\newcommand{\Rplus}{\ensuremath{\mathbb{R}^+}}
\newcommand{\Rmin}{\ensuremath{\mathbb{R}^-}}
\newcommand{\Rnulplus}{\ensuremath{\mathbb{R}_0^+}}
\newcommand{\Rnulmin}{\ensuremath{\mathbb{R}_0^-}}
\newcommand{\Rnuleenplus}{\ensuremath{\mathbb{R}^+\setminus\{0,1\}}}
\newcommand{\N}{\ensuremath{\mathbb{N}}}
\newcommand{\Nnul}{\ensuremath{\mathbb{N}_0}}
\newcommand{\Z}{\ensuremath{\mathbb{Z}}}
\newcommand{\Znul}{\ensuremath{\mathbb{Z}_0}}
\newcommand{\Zplus}{\ensuremath{\mathbb{Z}^+}}
\newcommand{\Zmin}{\ensuremath{\mathbb{Z}^-}}
\newcommand{\Znulplus}{\ensuremath{\mathbb{Z}_0^+}}
\newcommand{\Znulmin}{\ensuremath{\mathbb{Z}_0^-}}
\newcommand{\C}{\ensuremath{\mathbb{C}}}
\newcommand{\Cnul}{\ensuremath{\mathbb{C}_0}}
\newcommand{\Cplus}{\ensuremath{\mathbb{C}^+}}
\newcommand{\Cmin}{\ensuremath{\mathbb{C}^-}}
\newcommand{\Cnulplus}{\ensuremath{\mathbb{C}_0^+}}
\newcommand{\Cnulmin}{\ensuremath{\mathbb{C}_0^-}}
\newcommand{\Q}{\ensuremath{\mathbb{Q}}}
\newcommand{\Qnul}{\ensuremath{\mathbb{Q}_0}}
\newcommand{\Qplus}{\ensuremath{\mathbb{Q}^+}}
\newcommand{\Qmin}{\ensuremath{\mathbb{Q}^-}}
\newcommand{\Qnulplus}{\ensuremath{\mathbb{Q}_0^+}}
\newcommand{\Qnulmin}{\ensuremath{\mathbb{Q}_0^-}}

\newcommand{\perdef}{\overset{\mathrm{def}}{=}}
\newcommand{\pernot}{\overset{\mathrm{notatie}}{=}}
\newcommand\perinderdaad{\overset{!}{=}}     % voorlopig gebruikt in limietenrekenregels
\newcommand\perhaps{\overset{?}{=}}          % voorlopig gebruikt in limietenrekenregels

\newcommand{\degree}{^\circ}


\DeclareMathOperator{\dom}{dom}     % domein
\DeclareMathOperator{\codom}{codom} % codomein
\DeclareMathOperator{\bld}{bld}     % beeld
\DeclareMathOperator{\graf}{graf}   % grafiek
\DeclareMathOperator{\rico}{rico}   % richtingcoëfficient
\DeclareMathOperator{\co}{co}       % coordinaat
\DeclareMathOperator{\gr}{gr}       % graad

\newcommand{\func}[5]{\ensuremath{#1: #2 \rightarrow #3: #4 \mapsto #5}} % Easy to write a function


% Operators
\DeclareMathOperator{\bgsin}{bgsin}
\DeclareMathOperator{\bgcos}{bgcos}
\DeclareMathOperator{\bgtan}{bgtan}
\DeclareMathOperator{\bgcot}{bgcot}
\DeclareMathOperator{\bgsinh}{bgsinh}
\DeclareMathOperator{\bgcosh}{bgcosh}
\DeclareMathOperator{\bgtanh}{bgtanh}
\DeclareMathOperator{\bgcoth}{bgcoth}

% Oude \Bgsin etc deprecated: gebruik \bgsin, en herdefinieer dat als je Bgsin wil!
%\DeclareMathOperator{\cosec}{cosec}    % not used? gebruik \csc en herdefinieer

% operatoren voor differentialen: to be verified; 1/2020: inconsequent gebruik bij afgeleiden/integralen
\renewcommand{\d}{\mathrm{d}}
\newcommand{\dx}{\d x}
\newcommand{\dd}[1]{\frac{\mathrm{d}}{\mathrm{d}#1}}
\newcommand{\ddx}{\dd{x}}

% om in voorbeelden/oefeningen de notatie voor afgeleiden te kunnen kiezen
% Usage: \afg{(2\sin(x))}  (en wordt d/dx, of accent, of D )
%\newcommand{\afg}[1]{{#1}'}
\newcommand{\afg}[1]{\left(#1\right)'}
%\renewcommand{\afg}[1]{\frac{\mathrm{d}#1}{\mathrm{d}x}}   % include in relevant exercises ...
%\renewcommand{\afg}[1]{D{#1}}

%
% \xmxxx commands: Extra KU Leuven functionaliteit van, boven of naast Ximera
%   ( Conventie 8/2019: xm+nederlandse omschrijving, maar is niet consequent gevolgd, en misschien ook niet erg handig !)
%
% (Met een minimale ximera.cls en preamble.tex zou een bruikbare .pdf moeten kunnen worden gemaakt van eender welke ximera)
%
% Usage: \xmtitle[Mijn korte abstract]{Mijn titel}{Mijn abstract}
% Eerste command na \begin{document}:
%  -> definieert de \title
%  -> definieert de abstract
%  -> doet \maketitle ( dus: print de hoofding als 'chapter' of 'sectie')
% Optionele parameter geeft eenn kort abstract (die met de globale setting \xmshortabstract{} al dan niet kan worden geprint.
% De optionele korte abstract kan worden gebruikt voor pseudo-grappige abtsarts, dus dus globaal al dan niet kunnen worden gebuikt...
% Globale settings:
%  de (optionele) 'korte abstract' wordt enkele getoond als \xmshortabstract is gezet
\providecommand\xmshortabstract{} % default: print (only!) short abstract if present
\newcommand{\xmtitle}[3][]{
	\title{#2}
	\begin{abstract}
		\ifdefined\xmshortabstract
		\ifstrempty{#1}{%
			#3
		}{%
			#1
		}%
		\else
		#3
		\fi
	\end{abstract}
	\maketitle
}

% 
% Kleine grapjes: moeten zonder verder gevolg kunnen worden verwijderd
%
%\newcommand{\xmopje}[1]{{\small#1{\reversemarginpar\marginpar{\Smiley}}}}   % probleem in floats!!
\newtoggle{showxmopje}
\toggletrue{showxmopje}

\newcommand{\xmopje}[1]{%
   \iftoggle{showxmopje}{#1}{}%
}


% -> geef een abstracte-formule-met-rechts-een-concreet-voorbeeld
% VB:  \formulevb{a^2+b^2=c^2}{3^2+4^2=5^2}
%
\ifdefined\HCode
\NewEnviron{xmdiv}[1]{\HCode{\Hnewline<div class="#1">\Hnewline}\BODY{\HCode{\Hnewline</div>\Hnewline}}}
\else
\NewEnviron{xmdiv}[1]{\BODY}
\fi

\providecommand{\formulevb}[2]{
	{\centering

    \begin{xmdiv}{xmformulevb}    % zie css voor online layout !!!
	\begin{tabular}{lcl}
		\important{#1}
		&  &
		Vb: $#2$
		\end{tabular}
	\end{xmdiv}

	}
}

\ifdefined\HCode
\providecommand{\vb}[1]{%
    \HCode{\Hnewline<span class="xmvb">}#1\HCode{</span>\Hnewline}%
}
\else
\providecommand{\vb}[1]{
    \colorbox{blue!10}{#1}
}
\fi

\ifdefined\HCode
\providecommand{\xmcolorbox}[2]{
	\HCode{\Hnewline<div class="xmcolorbox">\Hnewline}#2\HCode{\Hnewline</div>\Hnewline}
}
\else
\providecommand{\xmcolorbox}[2]{
  \cellcolor{#1}#2
}
\fi


\ifdefined\HCode
\providecommand{\xmopmerking}[1]{
 \HCode{\Hnewline<div class="xmopmerking">\Hnewline}#1\HCode{\Hnewline</div>\Hnewline}
}
\else
\providecommand{\xmopmerking}[1]{
	{\footnotesize #1}
}
\fi
% \providecommand{\voorbeeld}[1]{
% 	\colorbox{blue!10}{$#1$}
% }



% Hernoem Proof naar Bewijs, nodig voor HTML versie
\renewcommand*{\proofname}{Bewijs}

% Om opgave van oefening (wordt niet geprint bij oplossingenblad)
% (to be tested test)
\NewEnviron{statement}{\BODY}

% Environment 'oplossing' en 'uitkomst'
% voor resp. volledige 'uitwerking' dan wel 'enkel eindresultaat'
% geimplementeerd via feedback, omdat er in de ximera-server adhoc feedback-code is toegevoegd
%% Niet tonen indien handout
%% Te gebruiken om volledige oplossingen/uitwerkingen van oefeningen te tonen
%% \begin{oplossing}        De optelling is commutatief \end{oplossing}  : verschijnt online enkel 'op vraag'
%% \begin{oplossing}[toon]  De optelling is commutatief \end{oplossing}  : verschijnt steeds onmiddellijk online (bv te gebruiken bij voorbeelden) 

\ifhandout%
    \NewEnviron{oplossing}[1][onzichtbaar]%
    {%
    \ifthenelse{\equal{\detokenize{#1}}{\detokenize{toon}}}
    {
    \def\PH@Command{#1}% Use PH@Command to hold the content and be a target for "\expandafter" to expand once.

    \begin{trivlist}% Begin the trivlist to use formating of the "Feedback" label.
    \item[\hskip \labelsep\small\slshape\bfseries Oplossing% Format the "Feedback" label. Don't forget the space.
    %(\texttt{\detokenize\expandafter{\PH@Command}}):% Format (and detokenize) the condition for feedback to trigger
    \hspace{2ex}]\small%\slshape% Insert some space before the actual feedback given.
    \BODY
    \end{trivlist}
    }
    {  % \begin{feedback}[solution]   \BODY     \end{feedback}  }
    }
    }    
\else
% ONLY for HTML; xmoplossing is styled with css, and is not, and need not be a LaTeX environment
% THUS: it does NOT use feedback anymore ...
%    \NewEnviron{oplossing}{\begin{expandable}{xmoplossing}{\nlen{Toon uitwerking}{Show solution}}{\BODY}\end{expandable}}
    \newenvironment{oplossing}[1][onzichtbaar]
   {%
       \begin{expandable}{xmoplossing}{}
   }
   {%
   	   \end{expandable}
   } 
%     \newenvironment{oplossing}[1][onzichtbaar]
%    {%
%        \begin{feedback}[solution]   	
%    }
%    {%
%    	   \end{feedback}
%    } 
\fi

\ifhandout%
    \NewEnviron{uitkomst}[1][onzichtbaar]%
    {%
    \ifthenelse{\equal{\detokenize{#1}}{\detokenize{toon}}}
    {
    \def\PH@Command{#1}% Use PH@Command to hold the content and be a target for "\expandafter" to expand once.

    \begin{trivlist}% Begin the trivlist to use formating of the "Feedback" label.
    \item[\hskip \labelsep\small\slshape\bfseries Uitkomst:% Format the "Feedback" label. Don't forget the space.
    %(\texttt{\detokenize\expandafter{\PH@Command}}):% Format (and detokenize) the condition for feedback to trigger
    \hspace{2ex}]\small%\slshape% Insert some space before the actual feedback given.
    \BODY
    \end{trivlist}
    }
    {  % \begin{feedback}[solution]   \BODY     \end{feedback}  }
    }
    }    
\else
\ifdefined\HCode
   \newenvironment{uitkomst}[1][onzichtbaar]
    {%
        \begin{expandable}{xmuitkomst}{}%
    }
    {%
    	\end{expandable}%
    } 
\else
  % Do NOT print 'uitkomst' in non-handout
  %  (presumably, there is also an 'oplossing' ??)
  \newenvironment{uitkomst}[1][onzichtbaar]{}{}
\fi
\fi

%
% Uitweidingen zijn extra's die niet redelijkerwijze tot de leerstof behoren
% Uitbreidingen zijn extra's die wel redelijkerwijze tot de leerstof van bv meer geavanceerde versies kunnen behoren (B-programma/Wiskundestudenten/...?)
% Nog niet voorzien: design voor verschillende versies (A/B programma, BIO, voorkennis/ ...)
% Voor 'uitweidingen' is er een environment die online per default is ingeklapt, en in pdf al dan niet kan worden geincluded  (via \xmnouitweiding) 
%
% in een xourse, per default GEEN uitweidingen, tenzij \xmuitweiding expliciet ergens is gezet ...
\ifdefined\isXourse
   \ifdefined\xmuitweiding
   \else
       \def\xmnouitweiding{true}
   \fi
\fi

\ifdefined\xmnouitweiding
\newcounter{xmuitweiding}  % anders error undefined ...  
\excludecomment{xmuitweiding}
\else
\newtheoremstyle{dotless}{}{}{}{}{}{}{ }{}
\theoremstyle{dotless}
\newtheorem*{xmuitweidingnofrills}{}   % nofrills = no accordion; gebruikt dus de dotless theoremstyle!

\newcounter{xmuitweiding}
\newenvironment{xmuitweiding}[1][ ]%
{% 
	\refstepcounter{xmuitweiding}%
    \begin{expandable}{xmuitweiding}{\nlentext{Uitweiding \arabic{xmuitweiding}: #1}{Digression \arabic{xmuitweiding}: #1}}%
	\begin{xmuitweidingnofrills}%
}
{%
    \end{xmuitweidingnofrills}%
    \end{expandable}%
}   
% \newenvironment{xmuitweiding}[1][ ]%
% {% 
% 	\refstepcounter{xmuitweiding}
% 	\begin{accordion}\begin{accordion-item}[Uitweiding \arabic{xmuitweiding}: #1]%
% 			\begin{xmuitweidingnofrills}%
% 			}
% 			{\end{xmuitweidingnofrills}\end{accordion-item}\end{accordion}}   
\fi


\newenvironment{xmexpandable}[1][]{
	\begin{accordion}\begin{accordion-item}[#1]%
		}{\end{accordion-item}\end{accordion}}


% Command that gives a selection box online, but just prints the right answer in pdf
\newcommand{\xmonlineChoice}[1]{\pdfOnly{\wordchoicegiventrue}\wordChoice{#1}\pdfOnly{\wordchoicegivenfalse}}   % deprecated, gebruik onlineChoice ...
\newcommand{\onlineChoice}[1]{\pdfOnly{\wordchoicegiventrue}\wordChoice{#1}\pdfOnly{\wordchoicegivenfalse}}

\newcommand{\TJa}{\nlentext{ Ja }{ Yes }}
\newcommand{\TNee}{\nlentext{ Nee }{ No }}
\newcommand{\TJuist}{\nlentext{ Juist }{ True }}
\newcommand{\TFout}{\nlentext{ Fout }{ False }}

\newcommand{\choiceTrue }{{\renewcommand{\choiceminimumhorizontalsize}{4em}\wordChoice{\choice[correct]{\TJuist}\choice{\TFout}}}}
\newcommand{\choiceFalse}{{\renewcommand{\choiceminimumhorizontalsize}{4em}\wordChoice{\choice{\TJuist}\choice[correct]{\TFout}}}}

\newcommand{\choiceYes}{{\renewcommand{\choiceminimumhorizontalsize}{3em}\wordChoice{\choice[correct]{\TJa}\choice{\TNee}}}}
\newcommand{\choiceNo }{{\renewcommand{\choiceminimumhorizontalsize}{3em}\wordChoice{\choice{\TJa}\choice[correct]{\TNee}}}}

% Optional nicer formatting for wordChoice in PDF

\let\inlinechoiceorig\inlinechoice

%\makeatletter
%\providecommand{\choiceminimumverticalsize}{\vphantom{$\frac{\sqrt{2}}{2}$}}   % minimum height of boxes (cfr infra)
\providecommand{\choiceminimumverticalsize}{\vphantom{$\tfrac{2}{2}$}}   % minimum height of boxes (cfr infra)
\providecommand{\choiceminimumhorizontalsize}{1em}   % minimum width of boxes (cfr infra)

\newcommand{\inlinechoicesquares}[2][]{%
		\setkeys{choice}{#1}%
		\ifthenelse{\boolean{\choice@correct}}%
		{%
            \ifhandout%
               \fbox{\choiceminimumverticalsize #2}\allowbreak\ignorespaces%
            \else%
               \fcolorbox{blue}{blue!20}{\choiceminimumverticalsize #2}\allowbreak\ignorespaces\setkeys{choice}{correct=false}\ignorespaces%
            \fi%
		}%
		{% else
			\fbox{\choiceminimumverticalsize #2}\allowbreak\ignorespaces%  HACK: wat kleiner, zodat fits on line ... 	
		}%
}

\newcommand{\inlinechoicesquareX}[2][]{%
		\setkeys{choice}{#1}%
		\ifthenelse{\boolean{\choice@correct}}%
		{%
            \ifhandout%
               \framebox[\ifdim\choiceminimumhorizontalsize<\width\width\else\choiceminimumhorizontalsize\fi]{\choiceminimumverticalsize\ #2\ }\allowbreak\ignorespaces\setkeys{choice}{correct=false}\ignorespaces%
            \else%
               \fcolorbox{blue}{blue!20}{\makebox[\ifdim\choiceminimumhorizontalsize<\width\width\else\choiceminimumhorizontalsize\fi]{\choiceminimumverticalsize #2}}\allowbreak\ignorespaces\setkeys{choice}{correct=false}\ignorespaces%
            \fi%
		}%
		{% else
        \ifhandout%
			\framebox[\ifdim\choiceminimumhorizontalsize<\width\width\else\choiceminimumhorizontalsize\fi]{\choiceminimumverticalsize\ #2\ }\allowbreak\ignorespaces%  HACK: wat kleiner, zodat fits on line ... 	
        \fi
		}%
}


\newcommand{\inlinechoicepuntjes}[2][]{%
		\setkeys{choice}{#1}%
		\ifthenelse{\boolean{\choice@correct}}%
		{%
            \ifhandout%
               \dots\ldots\ignorespaces\setkeys{choice}{correct=false}\ignorespaces
            \else%
               \fcolorbox{blue}{blue!20}{\choiceminimumverticalsize #2}\allowbreak\ignorespaces\setkeys{choice}{correct=false}\ignorespaces%
            \fi%
		}%
		{% else
			%\fbox{\choiceminimumverticalsize #2}\allowbreak\ignorespaces%  HACK: wat kleiner, zodat fits on line ... 	
		}%
}

% print niets, maar definieer globale variable \myanswer
%  (gebruikt om oplossingsbladen te printen) 
\newcommand{\inlinechoicedefanswer}[2][]{%
		\setkeys{choice}{#1}%
		\ifthenelse{\boolean{\choice@correct}}%
		{%
               \gdef\myanswer{#2}\setkeys{choice}{correct=false}

		}%
		{% else
			%\fbox{\choiceminimumverticalsize #2}\allowbreak\ignorespaces%  HACK: wat kleiner, zodat fits on line ... 	
		}%
}



%\makeatother

\newcommand{\setchoicedefanswer}{
\ifdefined\HCode
\else
%    \renewenvironment{multipleChoice@}[1][]{}{} % remove trailing ')'
    \let\inlinechoice\inlinechoicedefanswer
\fi
}

\newcommand{\setchoicepuntjes}{
\ifdefined\HCode
\else
    \renewenvironment{multipleChoice@}[1][]{}{} % remove trailing ')'
    \let\inlinechoice\inlinechoicepuntjes
\fi
}
\newcommand{\setchoicesquares}{
\ifdefined\HCode
\else
    \renewenvironment{multipleChoice@}[1][]{}{} % remove trailing ')'
    \let\inlinechoice\inlinechoicesquares
\fi
}
%
\newcommand{\setchoicesquareX}{
\ifdefined\HCode
\else
    \renewenvironment{multipleChoice@}[1][]{}{} % remove trailing ')'
    \let\inlinechoice\inlinechoicesquareX
\fi
}
%
\newcommand{\setchoicelist}{
\ifdefined\HCode
\else
    \renewenvironment{multipleChoice@}[1][]{}{)}% re-add trailing ')'
    \let\inlinechoice\inlinechoiceorig
\fi
}

\setchoicesquareX  % by default list-of-squares with onlineChoice in PDF

% Omdat multicols niet werkt in html: enkel in pdf  (in html zijn langere pagina's misschien ook minder storend)
\newenvironment{xmmulticols}[1][2]{
 \pdfOnly{\begin{multicols}{#1}}%
}{ \pdfOnly{\end{multicols}}}

%
% Te gebruiken in plaats van \section\subsection
%  (in een printstyle kan dan het level worden aangepast
%    naargelang \chapter vs \section style )
% 3/2021: DO NOT USE \xmsubsection !
\newcommand\xmsection\subsection
\newcommand\xmsubsection\subsubsection

% Aanpassen printversie
%  (hier gedefinieerd, zodat ze in xourse kunnen worden gezet/overschreven)
\providebool{parttoc}
\providebool{printpartfrontpage}
\providebool{printactivitytitle}
\providebool{printactivityqrcode}
\providebool{printactivityurl}
\providebool{printcontinuouspagenumbers}
\providebool{numberactivitiesbysubpart}
\providebool{addtitlenumber}
\providebool{addsectiontitlenumber}
\addtitlenumbertrue
\addsectiontitlenumbertrue

% The following three commands are hardcoded in xake, you can't create other commands like these, without adding them to xake as well
%  ( gebruikt in xourses om juiste soort titelpagina te krijgen voor verschillende ximera's )
\newcommand{\activitychapter}[2][]{
    {    
    \ifstrequal{#1}{notnumbered}{
        \addtitlenumberfalse
    }{}
    \typeout{ACTIVITYCHAPTER #2}   % logging
	\chapterstyle
	\activity{#2}
    }
}
\newcommand{\activitysection}[2][]{
    {
    \ifstrequal{#1}{notnumbered}{
        \addsectiontitlenumberfalse
    }{}
	\typeout{ACTIVITYSECTION #2}   % logging
	\sectionstyle
	\activity{#2}
    }
}
% Practices worden als activity getoond om de grote blokken te krijgen online
\newcommand{\practicesection}[2][]{
    {
    \ifstrequal{#1}{notnumbered}{
        \addsectiontitlenumberfalse
    }{}
    \typeout{PRACTICESECTION #2}   % logging
	\sectionstyle
	\activity{#2}
    }
}
\newcommand{\activitychapterlink}[3][]{
    {
    \ifstrequal{#1}{notnumbered}{
        \addtitlenumberfalse
    }{}
    \typeout{ACTIVITYCHAPTERLINK #3}   % logging
	\chapterstyle
	\activitylink[#1]{#2}{#3}
    }
}

\newcommand{\activitysectionlink}[3][]{
    {
    \ifstrequal{#1}{notnumbered}{
        \addsectiontitlenumberfalse
    }{}
    \typeout{ACTIVITYSECTIONLINK #3}   % logging
	\sectionstyle
	\activitylink[#1]{#2}{#3}
    }
}


% Commando om de printstyle toe te voegen in ximera's. Zorgt ervoor dat er geen problemen zijn als je de xourses compileert
% hack om onhandige relative paden in TeX te omzeilen
% should work both in xourse and ximera (pre-112022 only in ximera; thus obsoletes adhoc setup in xourses)
% loads global.sty if present (cfr global.css for online settings!)
% use global.sty to overwrite settings in printstyle.sty ...
\newcommand{\addPrintStyle}[1]{
\iftikzexport\else   % only in PDF
  \makeatletter
  \ifx\@onlypreamble\@notprerr\else   % ONLY if in tex-preamble   (and e.g. not when included from xourse)
    \typeout{Loading printstyle}   % logging
    \usepackage{#1/printstyle} % mag enkel geinclude worden als je die apart compileert
    \IfFileExists{#1/global.sty}{
        \typeout{Loading printstyle-folder #1/global.sty}   % logging
        \usepackage{#1/global}
        }{
        \typeout{Info: No extra #1/global.sty}   % logging
    }   % load global.sty if present
    \IfFileExists{global.sty}{
        \typeout{Loading local-folder global.sty (or TEXINPUTPATH..)}   % logging
        \usepackage{global}
    }{
        \typeout{Info: No folder/global.sty}   % logging
    }   % load global.sty if present
    \IfFileExists{\currfilebase.sty}
    {
        \typeout{Loading \currfilebase.sty}
        \input{\currfilebase.sty}
    }{
        \typeout{Info: No local \currfilebase.sty}
    }
    \fi
  \makeatother
\fi
}

%
%  
% references: Ximera heeft adhoc logica	 om online labels te doen werken over verschillende files heen
% met \hyperref kan de getoonde tekst toch worden opgegeven, in plaats van af te hangen van de label-text
\ifdefined\HCode
% Link to standard \labels, but give your own description
% Usage:  Volg \hyperref[my_very_verbose_label]{deze link} voor wat tijdverlies
%   (01/2020: Ximera-server aangepast om bij class reference-keeptext de link-text NIET te vervangen door de label-text !!!) 
\renewcommand{\hyperref}[2][]{\HCode{<a class="reference reference-keeptext" href="\##1">}#2\HCode{</a>}}
%
%  Link to specific targets  (not tested ?)
\renewcommand{\hypertarget}[1]{\HCode{<a class="ximera-label" id="#1"></a>}}
\renewcommand{\hyperlink}[2]{\HCode{<a class="reference reference-keeptext" href="\##1">}#2\HCode{</a>}}
\fi

% Mmm, quid English ... (for keyword #1 !) ?
\newcommand{\wikilink}[2]{
    \hyperlink{https://nl.wikipedia.org/wiki/#1}{#2}
    \pdfOnly{\footnote{See \url{https://nl.wikipedia.org/wiki/#1}}
    }
}

\renewcommand{\figurename}{Figuur}
\renewcommand{\tablename}{Tabel}

%
% Gedoe om verschillende versies van xourse/ximera te maken afhankelijk van settings
%
% default: versie met antwoorden
% handout: versie voor de studenten, zonder antwoorden/oplossingen
% full: met alles erop en eraan, dus geschikt voor auteurs en/of lesgevers  (bevat in de pdf ook de 'online-only' stukken!)
%
%
% verder kunnen ook opties/variabele worden gezet voor hints/auteurs/uitweidingen/ etc
%
% 'Full' versie
\newtoggle{showonline}
\ifdefined\HCode   % zet default showOnline
    \toggletrue{showonline} 
\else
    \togglefalse{showonline}
\fi

% Full versie   % deprecated: see infra
\newcommand{\printFull}{
    \hintstrue
    \handoutfalse
    \toggletrue{showonline} 
}

\ifdefined\shouldPrintFull   % deprecated: see infra
    \printFull
\fi



% Overschrijf onlineOnly  (zoals gedefinieerd in ximera.cls)
\ifhandout   % in handout: gebruik de oorspronkelijke ximera.cls implementatie  (is dit wel nodig/nuttig?)
\else
    \iftoggle{showonline}{%
        \ifdefined\HCode
          \RenewEnviron{onlineOnly}{\bgroup\BODY\egroup}   % showOnline, en we zijn  online, dus toon de tekst
        \else
          \RenewEnviron{onlineOnly}{\bgroup\color{red!50!black}\BODY\egroup}   % showOnline, maar we zijn toch niet online: kleur de tekst rood 
        \fi
    }{%
      \RenewEnviron{onlineOnly}{}  % geen showOnline
    }
\fi

% hack om na hoofding van definition/proposition/... als dan niet op een nieuwe lijn te starten
% soms is dat goed en mooi, en soms niet; en in HTML is het nu (2/2020) anders dan in pdf
% vandaar suggestie om 
%     \begin{definition}[Nieuw concept] \nl
% te gebruiken als je zeker een newline wil na de hoofdig en titel
% (in het bijzonder itemize zonder \nl is 'lelijk' ...)
\ifdefined\HCode
\newcommand{\nl}{}
\else
\newcommand{\nl}{\ \par} % newline (achter heading van definition etc.)
\fi


% \nl enkel in handoutmode (ihb voor \wordChoice, die dan typisch veeeel langer wordt)
\ifdefined\HCode
\providecommand{\handoutnl}{}
\else
\providecommand{\handoutnl}{%
\ifhandout%
  \nl%
\fi%
}
\fi

% Could potentially replace \pdfOnline/\begin{onlineOnly} : 
% Usage= \ifonline{Hallo surfer}{Hallo PDFlezer}
\providecommand{\ifonline}[2]%
{
\begin{onlineOnly}#1\end{onlineOnly}%
\pdfOnly{#2}
}%


%
% Maak optionele 'basic' en 'extended' versies van een activity
%  met environment basicOnly, basicSkip en extendedOnly
%
%  (
%   Dit werkt ENKEL in de PDF; de online versies tonen (minstens voorklopig) steeds 
%   het default geval met printbasicversion en printextendversion beide FALSE
%  )
%
\providebool{printbasicversion}
\providebool{printextendedversion}   % not properly implemented
\providebool{printfullversion}       % presumably print everything (debug/auteur)
%
% only set these in xourses, and BEFORE loading this preamble
%
%\newif\ifshowbasic     \showbasictrue        % use this line in xourse to show 'basic' sections
%\newif\ifshowextended  \showextendedtrue     % use this line in xourse to show 'extended' sections
%
%
%\ifbool{showbasic}
%      { \NewEnviron{basicOnly}{\BODY} }    % if yes: just print contents
%      { \NewEnviron{basicOnly}{}      }    % if no:  completely ignore contents
%
%\ifbool{showbasic}
%      { \NewEnviron{basicSkip}{}      }
%      { \NewEnviron{basicSkip}{\BODY} }
%

\ifbool{printextendedversion}
      { \NewEnviron{extendedOnly}{\BODY} }
      { \NewEnviron{extendedOnly}{}      }
      


\ifdefined\HCode    % in html: always print
      {\newenvironment*{basicOnly}{}{}}    % if yes: just print contents
      {\newenvironment*{basicSkip}{}{}}    % if yes: just print contents
\else

\ifbool{printbasicversion}
      {\newenvironment*{basicOnly}{}{}}    % if yes: just print contents
      {\NewEnviron{basicOnly}{}      }    % if no:  completely ignore contents

\ifbool{printbasicversion}
      {\NewEnviron{basicSkip}{}      }
      {\newenvironment*{basicSkip}{}{}}

\fi

\usepackage{float}
\usepackage[rightbars,color]{changebar}

% Full versie
\ifbool{printfullversion}{
    \hintstrue
    \handoutfalse
    \toggletrue{showonline}
    \printbasicversionfalse
    \cbcolor{red}
    \renewenvironment*{basicOnly}{\cbstart}{\cbend}
    \renewenvironment*{basicSkip}{\cbstart}{\cbend}
    \def\xmtoonprintopties{FULL}   % will be printed in footer
}
{}
      
%
% Evalueer \ifhints IN de environment
%  
%
%\RenewEnviron{hint}
%{
%\ifhandout
%\ifhints\else\setbox0\vbox\fi%everything in een emty box
%\bgroup 
%\stepcounter{hintLevel}
%\BODY
%\egroup\ignorespacesafterend
%\addtocounter{hintLevel}{-1}
%\else
%\ifhints
%\begin{trivlist}\item[\hskip \labelsep\small\slshape\bfseries Hint:\hspace{2ex}]
%\small\slshape
%\stepcounter{hintLevel}
%\BODY
%\end{trivlist}
%\addtocounter{hintLevel}{-1}
%\fi
%\fi
%}

% Onafhankelijk van \ifhandout ...? TO BE VERIFIED
\RenewEnviron{hint}
{
\ifhints
\begin{trivlist}\item[\hskip \labelsep\small\bfseries Hint:\hspace{2ex}]
\small%\slshape
\stepcounter{hintLevel}
\BODY
\end{trivlist}
\addtocounter{hintLevel}{-1}
\else
\iftikzexport   % anders worden de tikz tekeningen in hints niet gegenereerd ?
\setbox0\vbox\bgroup
\stepcounter{hintLevel}
\BODY
\egroup\ignorespacesafterend
\addtocounter{hintLevel}{-1}
\fi % ifhandout
\fi %ifhints
}

%
% \tab sets typewriter-tabs (e.g. to format questions)
% (Has no effect in HTML :-( ))
%
\usepackage{tabto}
\ifdefined\HCode
  \renewcommand{\tab}{\quad}    % otherwise dummy .png's are generated ...?
\fi


% Also redefined in  preamble to get correct styling 
% for tikz images for (\tikzexport)
%

\theoremstyle{definition} % Bold titels
\makeatletter
\let\proposition\relax
\let\c@proposition\relax
\let\endproposition\relax
\makeatother
\newtheorem{proposition}{Eigenschap}


%\instructornotesfalse

% logic with \ifhandoutin ximera.cls unclear;so overwrite ...
\makeatletter
\@ifundefined{ifinstructornotes}{%
  \newif\ifinstructornotes
  \instructornotesfalse
  \newenvironment{instructorNotes}{}{}
}{}
\makeatother
\ifinstructornotes
\else
\renewenvironment{instructorNotes}%
{%
    \setbox0\vbox\bgroup
}
{%
    \egroup
}
\fi

% \RedeclareMathOperator
% from https://tex.stackexchange.com/questions/175251/how-to-redefine-a-command-using-declaremathoperator
\makeatletter
\newcommand\RedeclareMathOperator{%
    \@ifstar{\def\rmo@s{m}\rmo@redeclare}{\def\rmo@s{o}\rmo@redeclare}%
}
% this is taken from \renew@command
\newcommand\rmo@redeclare[2]{%
    \begingroup \escapechar\m@ne\xdef\@gtempa{{\string#1}}\endgroup
    \expandafter\@ifundefined\@gtempa
    {\@latex@error{\noexpand#1undefined}\@ehc}%
    \relax
    \expandafter\rmo@declmathop\rmo@s{#1}{#2}}
% This is just \@declmathop without \@ifdefinable
\newcommand\rmo@declmathop[3]{%
    \DeclareRobustCommand{#2}{\qopname\newmcodes@#1{#3}}%
}
\@onlypreamble\RedeclareMathOperator
\makeatother


%
% Engelse vertaling, vooral in mathmode
%
% 1. Algemene procedure
%
\ifdefined\isEn
 \newcommand{\nlen}[2]{#2}
 \newcommand{\nlentext}[2]{\text{#2}}
 \newcommand{\nlentextbf}[2]{\textbf{#2}}
\else
 \newcommand{\nlen}[2]{#1}
 \newcommand{\nlentext}[2]{\text{#1}}
 \newcommand{\nlentextbf}[2]{\textbf{#1}}
\fi

%
% 2. Lijst van erg veel gebruikte uitdrukkingen
%

% Ja/Nee/Fout/Juits etc
%\newcommand{\TJa}{\nlentext{ Ja }{ and }}
%\newcommand{\TNee}{\nlentext{ Nee }{ No }}
%\newcommand{\TJuist}{\nlentext{ Juist }{ Correct }
%\newcommand{\TFout}{\nlentext{ Fout }{ Wrong }
\newcommand{\TWaar}{\nlentext{ Waar }{ True }}
\newcommand{\TOnwaar}{\nlentext{ Vals }{ False }}
% Korte bindwoorden en, of, dus, ...
\newcommand{\Ten}{\nlentext{ en }{ and }}
\newcommand{\Tof}{\nlentext{ of }{ or }}
\newcommand{\Tdus}{\nlentext{ dus }{ so }}
\newcommand{\Tendus}{\nlentext{ en dus }{ and thus }}
\newcommand{\Tvooralle}{\nlentext{ voor alle }{ for all }}
\newcommand{\Took}{\nlentext{ ook }{ also }}
\newcommand{\Tals}{\nlentext{ als }{ when }} %of if?
\newcommand{\Twant}{\nlentext{ want }{ as }}
\newcommand{\Tmaal}{\nlentext{ maal }{ times }}
\newcommand{\Toptellen}{\nlentext{ optellen }{ add }}
\newcommand{\Tde}{\nlentext{ de }{ the }}
\newcommand{\Thet}{\nlentext{ het }{ the }}
\newcommand{\Tis}{\nlentext{ is }{ is }} %zodat is in text staat in mathmode (geen italics)
\newcommand{\Tmet}{\nlentext{ met }{ where }} % in situaties e.g met p < n --> where p < n
\newcommand{\Tnooit}{\nlentext{ nooit }{ never }}
\newcommand{\Tmaar}{\nlentext{ maar }{ but }}
\newcommand{\Tniet}{\nlentext{ niet }{ not }}
\newcommand{\Tuit}{\nlentext{ uit }{ from }}
\newcommand{\Ttov}{\nlentext{ t.o.v. }{ w.r.t. }}
\newcommand{\Tzodat}{\nlentext{ zodat }{ such that }}
\newcommand{\Tdeth}{\nlentext{de }{th }}
\newcommand{\Tomdat}{\nlentext{omdat }{because }} 


%
% Overschrijf addhoc commando's
%
\ifdefined\isEn
\renewcommand{\pernot}{\overset{\mathrm{notation}}{=}}
\RedeclareMathOperator{\bld}{im}     % beeld
\RedeclareMathOperator{\graf}{graph}   % grafiek
\RedeclareMathOperator{\rico}{slope}   % richtingcoëfficient
\RedeclareMathOperator{\co}{co}       % coordinaat
\RedeclareMathOperator{\gr}{deg}       % graad

% Operators
\RedeclareMathOperator{\bgsin}{arcsin}
\RedeclareMathOperator{\bgcos}{arccos}
\RedeclareMathOperator{\bgtan}{arctan}
\RedeclareMathOperator{\bgcot}{arccot}
\RedeclareMathOperator{\bgsinh}{arcsinh}
\RedeclareMathOperator{\bgcosh}{arccosh}
\RedeclareMathOperator{\bgtanh}{arctanh}
\RedeclareMathOperator{\bgcoth}{arccoth}

\fi


% HACK: use 'oplossing' for 'explanation' ...
\let\explanation\relax
\let\endexplanation\relax
% \newenvironment{explanation}{\begin{oplossing}}{\end{oplossing}}
\newcounter{explanation}

\ifhandout%
    \NewEnviron{explanation}[1][toon]%
    {%
    \RenewEnviron{verbatim}{ \red{VERBATIM CONTENT MISSING IN THIS PDF}} %% \expandafter\verb|\BODY|}

    \ifthenelse{\equal{\detokenize{#1}}{\detokenize{toon}}}
    {
    \def\PH@Command{#1}% Use PH@Command to hold the content and be a target for "\expandafter" to expand once.

    \begin{trivlist}% Begin the trivlist to use formating of the "Feedback" label.
    \item[\hskip \labelsep\small\slshape\bfseries Explanation:% Format the "Feedback" label. Don't forget the space.
    %(\texttt{\detokenize\expandafter{\PH@Command}}):% Format (and detokenize) the condition for feedback to trigger
    \hspace{2ex}]\small%\slshape% Insert some space before the actual feedback given.
    \BODY
    \end{trivlist}
    }
    {  % \begin{feedback}[solution]   \BODY     \end{feedback}  }
    }
    }    
\else
% ONLY for HTML; xmoplossing is styled with css, and is not, and need not be a LaTeX environment
% THUS: it does NOT use feedback anymore ...
%    \NewEnviron{oplossing}{\begin{expandable}{xmoplossing}{\nlen{Toon uitwerking}{Show solution}}{\BODY}\end{expandable}}
    \newenvironment{explanation}[1][toon]
   {%
       \begin{expandable}{xmoplossing}{}
   }
   {%
   	   \end{expandable}
   } 
\fi
 \title{Matrices of Linear Transformations with Respect to Arbitrary Bases} \license{CC BY-NC-SA 4.0}

\begin{document}
\begin{abstract}
 \end{abstract}
\maketitle

\section*{Matrices of Linear Transformations with Respect to Arbitrary Bases}

We know that every linear transformation from $\RR^n$ into $\RR^m$ is a matrix transformation  (Theorem \ref{th:matlin} of \href{\xmbaseurl/LTR-0020/main}{Standard Matrix of a Linear Transformation from $\RR^n$ to $\RR^m$}).  What about linear transformations between vector spaces other than $\RR^n$?  In this section we will learn to represent linear transformations between arbitrary finite-dimensional vector spaces using matrices.  To do so, we will use the fact that every $n$-dimensional vector space is isomorphic to $\RR^n$  (Corollary \ref{cor:ndimisotorn} of \href{\xmbaseurl/LTR-0060/main}{Isomorphic Vector Spaces}).  What we do here will serve as yet another example of how isomorphisms can be used to translate problems in one vector space to another, more convenient, vector space.

\begin{exploration}\label{init:taumatrix}
In \href{\xmbaseurl/LTR-0060/main}{Isomorphic Vector Spaces} we introduced the transformation
$$\tau_2:\mathbb{M}_{2,2}\rightarrow\mathbb{P}^3$$
defined by
$$\tau_2\left(\begin{bmatrix}a&b\\c&d\end{bmatrix}\right)=a+(a+c)x+(b-c)x^2+dx^3$$

You should verify that $\tau_2$ is linear.  (See Practice Problem \ref{prob:taulinear}.)

We will examine $\tau_2$ in an effort to find a way to represent it with a matrix.  (In the process, we will also end up proving that $\tau_2$ is an isomorphism, which is what you were challenged to do in \href{\xmbaseurl/LTR-0060/main}{Isomorphic Vector Spaces}.)

We will start by selecting a basis for each of $\mathbb{M}_{2,2}$ and $\mathbb{P}^3$.  We can choose any basis for either space, but we will choose bases that will make computations easier.

Let 
$$\mathcal{B}=\left\{\begin{bmatrix}1&0\\0&0\end{bmatrix}, \begin{bmatrix}0&1\\0&0\end{bmatrix}, \begin{bmatrix}0&0\\1&0\end{bmatrix}, \begin{bmatrix}0&0\\0&1\end{bmatrix}\right\}$$
$$\mathcal{C}=\{1, x, x^2, x^3\}$$
be our ordered bases or choice for $\mathbb{M}_{2,2}$ and $\mathbb{P}^3$, respectively.

Recall that a \dfn{coordinate vector isomorphism} maps a vector to its coordinate vector with respect to the given ordered basis (Theorem \ref{ex:coordmapiso} of \href{\xmbaseurl/LTR-0060/main}{Isomorphic Vector Spaces}).  In the diagram below, let $R$ and $S$ be coordinate vector isomorphisms with respect to $\mathcal{B}$ and $\mathcal{C}$.

\begin{center}
 \begin{tikzpicture} 
   \fill[blue, opacity=0.5] (1,0) rectangle (4,3);
   \fill[orange, opacity=0.6] (5,0) rectangle (10,3);
   
    \fill[blue, opacity=0.3] (1,4) rectangle (4,7);
   \fill[orange, opacity=0.4] (5,4) rectangle (10,7);
   
   \node[] at (1.5, 6.5)  (p2)    {$\mathbb{M}_{2,2}$};
   \node[] at (9.5, 6.5)  (r3)    {$\mathbb{P}^3$};
   
   \node[] at (1.5, 2.5)  (p2)    {$\RR^4$};
   \node[] at (9.5, 2.5)  (r3)    {$\RR^4$};
   
  
    \node[] at (2.5, 5.5)  (a)    {$\begin{bmatrix}a&b\\c&d\end{bmatrix}$};
     \node[] at (2.5, 1.5)  (b)    {$\begin{bmatrix}a\\b\\c\\d\end{bmatrix}$};
     
    
    \node[] at (7.5, 5.5)  (c)    {$a+(a+c)x+(b-c)x^2+dx^3$};
     \node[] at (7.5, 1.5)  (d)    {$\begin{bmatrix}a\\a+c\\b-c\\d\end{bmatrix}$};
     
    
     \draw [->,line width=0.5pt,-stealth]  (a.east)to(c.west);
     \draw [->,line width=0.5pt,-stealth, dashed]  (b.east)to(d.west);
     
     \draw [->,line width=0.5pt,-stealth]  (a.south)to(b.north);
     \draw [->,line width=0.5pt,-stealth]  (c.south)to(d.north);
     
%T and tau arrows     
     \draw [->,line width=0.5pt,-stealth, dashed]  (2.5,-0.1)to[out=340, in=200](8.5,-0.1 );
     \draw [->,line width=0.5pt,-stealth]  (2.5,7.1)to[out=20, in=160](8.5, 7.1);
     
%S and S-1 arrows  
     \draw [->,line width=0.5pt,-stealth]  (10.1,6)to[out=290, in=70](10.1,1 );
     \draw [->,line width=0.5pt,-stealth]  (10.1,2)to[out=70, in=290](10.1, 5);
     
%R and R-1 arrows  
     \draw [->,line width=0.5pt,-stealth]  (0.9,6)to[out=250, in=110](0.9,1 );
     \draw [->,line width=0.5pt,-stealth]  (0.9,2)to[out=110, in=250](0.9, 5);
     
%Function labels
      \node[] at (5.5, 8)    {$\tau_2$};
      \node[] at (5.5, -1)    {$T$};
      
      \node[] at (0.1, 3.5)    {$R$};
      \node[] at (1, 3.5)    {$R^{-1}$};
      
      \node[] at (10.9, 3.5)    {$S$};
      \node[] at (10, 3.5)    {$S^{-1}$};
      
      
  \end{tikzpicture}
\end{center}

Define $T:\RR^4\rightarrow \RR^4$ by $$T\left(\begin{bmatrix}a\\b\\c\\d\end{bmatrix}\right)=\begin{bmatrix}a\\a+c\\b-c\\d\end{bmatrix}$$
Observe that $T=S\circ\tau_2\circ R^{-1}$.  Because $T$ is a composition of linear transformations, $T$ itself is linear (Theorem \ref{th:complinear} of \href{\xmbaseurl/LTR-0030/main}{Composition and Inverses of Linear Transformations}).  Thus, we should be able to find the standard matrix for $T$.  To do this, find the images of the standard unit vectors and use them to  create the standard matrix $A$ for $T$.
$$A=\begin{bmatrix}\answer{1}&\answer{0}&\answer{0}&\answer{0}\\\answer{1}&\answer{0}&\answer{1}&\answer{0}\\\answer{0}&\answer{1}&\answer{-1}&\answer{0}\\\answer{0}&\answer{0}&\answer{0}&\answer{1}\end{bmatrix}$$

We say that $A$ is the matrix of $\tau_2$ with respect to ordered bases $\mathcal{B}$ and $\mathcal{C}$.

As a side-note, observe that $\tau_2=S^{-1}\circ T\circ R$.  Observe also that $T$ is invertible because $A$ is invertible.  So, $T$ is an isomorphism.  As a composition of isomorphisms, $\tau_2$ is an isomorphism (Theorem \ref{th:isocompisiso} of \href{\xmbaseurl/LTR-0060/main}{Isomorphic Vector Spaces}).  While we could have proved this result directly, as you were challenged to do in \href{\xmbaseurl/LTR-0060/main}{Isomorphic Vector Spaces}, this approach is much less tedious.

\end{exploration}

\begin{exploration}\label{init:matlintransgeneral}
Let
$$\vec{v}_1=\begin{bmatrix}1\\2\\0\end{bmatrix}\quad\text{and}\quad\vec{v}_2=\begin{bmatrix}0\\1\\1\end{bmatrix}$$
$$\vec{w}_1=\begin{bmatrix}1\\0\\1\end{bmatrix}\quad\text{and}\quad\vec{w}_2=\begin{bmatrix}1\\0\\0\end{bmatrix}$$

In Example \ref{ex:subtosub1} we defined $V$ and $W$ as follows:

$$V=\text{span}(\vec{v}_1, \vec{v}_2)\quad\text{and}\quad W=\text{span}(\vec{w}_1, \vec{w}_2)$$
Geometrically, $V$ and $W$ are planes in $\RR^3$.
We chose
$$\mathcal{B}=\{\vec{v}_1, \vec{v}_2\}\quad\text{and}\quad\mathcal{C}=\{\vec{w}_1, \vec{w}_2\}$$
as ordered bases of $V$ and $W$, respectively.

We also defined a linear transformation $T:V\rightarrow W$ by 
$$T(\vec{v}_1)=2\vec{w}_1-3\vec{w}_2\quad\text{and} \quad T(\vec{v}_2)=-\vec{w}_1+4\vec{w}_2$$

Our goal now is to find a matrix for $T$ with respect to $\mathcal{B}$ and $\mathcal{C}$.  

The information given in this problem is slightly different from the information in Exploration \ref{init:taumatrix}.  Instead of being given an expression for the image of a generic vector of $V$, we are only given the images of the two basis vectors of $V$.  But this information is sufficient to determine the linear transformation.

As before, we will map vectors of $V$ and $W$ to their coordinate vectors.  Where are the coordinate vectors located? 
\begin{enumerate}
    \item $[\vec{v}_1]_{\mathcal{B}}$ and $[\vec{v}_2]_{\mathcal{B}}$ are elements of \wordChoice{\choice{$\RR$}, \choice[correct]{$\RR^2$}, \choice{$\RR^3$}}
     \item $[\vec{w}_1]_{\mathcal{C}}$ and $[\vec{w}_2]_{\mathcal{C}}$ are elements of \wordChoice{\choice{$\RR$}, \choice[correct]{$\RR^2$}, \choice{$\RR^3$}}
\end{enumerate}
Now we find the coordinate vectors.
\begin{enumerate}
     \item 
     $$[\vec{v}_1]_{\mathcal{B}}=\begin{bmatrix}\answer{1}\\\answer{0}\end{bmatrix}\quad\text{and}\quad [\vec{v}_2]_{\mathcal{B}}=\begin{bmatrix}\answer{0}\\\answer{1}\end{bmatrix}$$
     \item 
     $$[T(\vec{v}_1)]_{\mathcal{C}}=\begin{bmatrix}\answer{2}\\\answer{-3}\end{bmatrix}\quad\text{and}\quad [T(\vec{v}_2)]_{\mathcal{C}}=\begin{bmatrix}\answer{-1}\\\answer{4}\end{bmatrix}$$
\end{enumerate}


\begin{expandable}
\begin{center}
 \begin{tikzpicture} 
   \fill[blue, opacity=0.5] (1,0) rectangle (4,3);
   \fill[orange, opacity=0.6] (5,0) rectangle (10,3);
   
    \fill[blue, opacity=0.3] (1,4) rectangle (4,7);
   \fill[orange, opacity=0.4] (5,4) rectangle (10,7);
   
   \node[] at (1.5, 7.5)  (p2)    {$V$};
   \node[] at (9.5, 7.5)  (r3)    {$W$};
   
   \node[] at (1.5, -0.5)  (p2)    {$\RR^2$};
   \node[] at (9.5, -0.5)  (r3)    {$\RR^2$};
   
  
    \node[] at (2, 6)  (a)    {$\begin{bmatrix}1\\2\\0\end{bmatrix}$};
     \node[] at (2, 1)  (b)    {$\begin{bmatrix}1\\0\end{bmatrix}$};
     
    
    \node[] at (8.5, 6)  (c)    {$2\vec{w}_1-3\vec{w}_2$};
     \node[] at (8.5, 1)  (d)    {$\begin{bmatrix}2\\-3\end{bmatrix}$};
     
     \node[] at (3, 5)  (e)    {$\begin{bmatrix}0\\1\\1\end{bmatrix}$};
     \node[] at (3, 2)  (f)    {$\begin{bmatrix}0\\1\end{bmatrix}$};
     
    
    \node[] at (7, 5)  (g)    {$-\vec{w}_1+4\vec{w}_2$};
     \node[] at (7, 2)  (h)    {$\begin{bmatrix}-1\\4\end{bmatrix}$};
     
    
     \draw [->,line width=0.5pt,-stealth]  (a.east)to(c.west);
     \draw [->,line width=0.5pt,-stealth, dashed]  (b.east)to(d.west);
     
     \draw [->,line width=0.5pt,-stealth]  (a.south)to(b.north);
     \draw [->,line width=0.5pt,-stealth]  (c.south)to(d.north);
     
     \draw [->,line width=0.5pt,-stealth]  (e.east)to(g.west);
     \draw [->,line width=0.5pt,-stealth, dashed]  (f.east)to(h.west);
     
     \draw [->,line width=0.5pt,-stealth]  (e.south)to(f.north);
     \draw [->,line width=0.5pt,-stealth]  (g.south)to(h.north);
     
%T and tau arrows     
     \draw [->,line width=0.5pt,-stealth, dashed]  (2.5,-0.1)to[out=340, in=200](8.5,-0.1 );
     \draw [->,line width=0.5pt,-stealth]  (2.5,7.1)to[out=20, in=160](8.5, 7.1);
     
%S and S-1 arrows  
     \draw [->,line width=0.5pt,-stealth]  (10.1,6)to[out=290, in=70](10.1,1 );
     \draw [->,line width=0.5pt,-stealth]  (10.1,2)to[out=70, in=290](10.1, 5);
     
%R and R-1 arrows  
     \draw [->,line width=0.5pt,-stealth]  (0.9,6)to[out=250, in=110](0.9,1 );
     \draw [->,line width=0.5pt,-stealth]  (0.9,2)to[out=110, in=250](0.9, 5);
     
%Function labels
      \node[] at (5.5, 8)    {$T$};
      \node[] at (5.5, -1)    {$F$};
      
      \node[] at (0.1, 3.5)    {$R$};
      \node[] at (1, 3.5)    {$R^{-1}$};
      
      \node[] at (10.9, 3.5)    {$S$};
      \node[] at (10, 3.5)    {$S^{-1}$};
      
      
  \end{tikzpicture}
\end{center}
\end{expandable}

Here is a diagram that summarizes this information.  (Press the arrow on the right to expand.)

Define $F:\RR^2\rightarrow \RR^2$ by $F=S\circ T\circ R^{-1}$.  $F$ is a linear transformation that maps $\begin{bmatrix}1\\0\end{bmatrix}$ and $\begin{bmatrix}0\\1\end{bmatrix}$ to $\begin{bmatrix}2\\-3\end{bmatrix}$ and $\begin{bmatrix}-1\\4\end{bmatrix}$, respectively.  Thus, the standard matrix for $F$ is:
$$A=\begin{bmatrix}2&-1\\-3&4\end{bmatrix}$$
We say that $A$ is a matrix for $T$ with respect to $\mathcal{B}$ and $\mathcal{C}$.  

Let's take a look at what this matrix can do for us.  Recall that in Example \ref{ex:subtosub1} we found that the image of $\vec{v}=2\vec{v}_1+\vec{v}_2$ is $$T(\vec{v})=T(2\vec{v}_1+\vec{v}_2)=2T(\vec{v}_1)+T(\vec{v}_2)=3\vec{w}_1-2\vec{w}_2$$
This result can be obtained by finding the product of $A$ with the coordinate vector of  $\vec{v}$.
$$\begin{bmatrix}2&-1\\-3&4\end{bmatrix}\begin{bmatrix}2\\1\end{bmatrix}=\begin{bmatrix}3\\-2\end{bmatrix}$$

Given any vector $\vec{u}=a\vec{v}_1+b\vec{v}_2$ of $V$, we can find $T(\vec{u})$ as follows:
$$\begin{bmatrix}2&-1\\-3&4\end{bmatrix}\begin{bmatrix}a\\b\end{bmatrix}=\begin{bmatrix}2a-b\\-3a+4b\end{bmatrix}$$

This gives us
$$T(\vec{u})=T(a\vec{v}_1+b\vec{v}_2)=(2a-b)\vec{w}_1+(-3a+4b)\vec{w}_2$$

\end{exploration}

%Matrix $A$ of Exploration Problem \ref{init:matlintransgeneral} is called the \dfn{matrix of linear transformation $T$ with respect to $\mathcal{B}$ and $\mathcal{C}$}.

\begin{example}\label{ex:transmatrix1}
Let $V$ and $W$ be vector spaces with ordered bases $\mathcal{B}=\{\vec{v}_1, \vec{v}_2, \vec{v}_3\}$ and $\mathcal{C}=\{\vec{w}_1, \vec{w}_2\}$, respectively.  Define a linear transformation $T:V\rightarrow W$ by $$T(\vec{v}_1)=2\vec{w}_1-\vec{w}_2,\quad T(\vec{v}_2)=-\vec{w}_1,\quad T(\vec{v}_3)=\vec{w}_1+3\vec{w}_2$$
Find the matrix of $T$ with respect to $\mathcal{B}$ and $\mathcal{C}$, and use it to find $T(2\vec{v}_1-\vec{v}_2+3\vec{v}_3)$.  Verify your answer by computing $T(2\vec{v}_1-\vec{v}_2+3\vec{v}_3)$ directly.

\begin{explanation}
We start with a diagram:
\begin{center}
 \begin{tikzpicture} 
   \fill[blue, opacity=0.5] (1,-1) rectangle (4,3);
   \fill[orange, opacity=0.6] (5,-1) rectangle (10,3);
   
    \fill[blue, opacity=0.3] (1,4) rectangle (4,7);
   \fill[orange, opacity=0.4] (5,4) rectangle (10,7);
   
   \node[] at (1.5, 7.5)  (p2)    {$V$};
   \node[] at (9.5, 7.5)  (r3)    {$W$};
   
   \node[] at (1.5, -1.5)  (p2)    {$\RR^3$};
   \node[] at (9.5, -1.5)  (r3)    {$\RR^2$};
   
  
    \node[] at (1.5, 6.5)  (a)    {$\vec{v}_1$};
     \node[] at (1.5, 0)  (b)    {$\begin{bmatrix}1\\0\\0\end{bmatrix}$};
     
    
    \node[] at (9, 6.5)  (c)    {$2\vec{w}_1-\vec{w}_2$};
     \node[] at (9, 0)  (d)    {$\begin{bmatrix}2\\-1\end{bmatrix}$};
     
     \node[] at (3.5, 4.5)  (e)    {$\vec{v}_3$};
     \node[] at (3.5, 2)  (f)    {$\begin{bmatrix}0\\0\\1\end{bmatrix}$};
     
     \node[] at (6, 4.5)  (g)    {$\vec{w}_1+3\vec{w}_2$};
     \node[] at (6, 2)  (h)    {$\begin{bmatrix}1\\3\end{bmatrix}$};
    
    \node[] at (7.5, 5.5)  (i)    {$-\vec{w}_1$};
     \node[] at (7.5,1)  (j)    {$\begin{bmatrix}-1\\0\end{bmatrix}$};
     
     \node[] at (2.5, 5.5)  (k)    {$\vec{v}_2$};
     \node[] at (2.5, 1)  (l)    {$\begin{bmatrix}0\\1\\0\end{bmatrix}$};
     
    
    
     \draw [->,line width=0.5pt,-stealth]  (a.east)to(c.west);
     \draw [->,line width=0.5pt,-stealth, dashed]  (b.east)to(d.west);
     
     \draw [->,line width=0.5pt,-stealth]  (a.south)to(b.north);
     \draw [->,line width=0.5pt,-stealth]  (c.south)to(d.north);
     
     \draw [->,line width=0.5pt,-stealth]  (e.east)to(g.west);
     \draw [->,line width=0.5pt,-stealth, dashed]  (f.east)to(h.west);
     
     \draw [->,line width=0.5pt,-stealth]  (e.south)to(f.north);
     \draw [->,line width=0.5pt,-stealth]  (g.south)to(h.north);
     
     \draw [->,line width=0.5pt,-stealth]  (k.east)to(i.west);
     \draw [->,line width=0.5pt,-stealth, dashed]  (l.east)to(j.west);
     
     \draw [->,line width=0.5pt,-stealth]  (k.south)to(l.north);
     \draw [->,line width=0.5pt,-stealth]  (i.south)to(j.north);
     
%T and tau arrows     
     \draw [->,line width=0.5pt,-stealth, dashed]  (2.5,-1.1)to[out=340, in=200](8.5,-1.1 );
     \draw [->,line width=0.5pt,-stealth]  (2.5,7.1)to[out=20, in=160](8.5, 7.1);
     
%S and S-1 arrows  
     \draw [->,line width=0.5pt,-stealth]  (10.1,6)to[out=290, in=70](10.1,1 );
     \draw [->,line width=0.5pt,-stealth]  (10.1,2)to[out=70, in=290](10.1, 5);
     
%R and R-1 arrows  
     \draw [->,line width=0.5pt,-stealth]  (0.9,6)to[out=250, in=110](0.9,1 );
     \draw [->,line width=0.5pt,-stealth]  (0.9,2)to[out=110, in=250](0.9, 5);
     
%Function labels
      \node[] at (5.5, 8)    {$T$};
      \node[] at (5.5, -2)    {$F$};
      
      \node[] at (0.1, 3.5)    {$R$};
      \node[] at (1, 3.5)    {$R^{-1}$};
      
      \node[] at (10.9, 3.5)    {$S$};
      \node[] at (10, 3.5)    {$S^{-1}$};
      
      
  \end{tikzpicture}
\end{center}

Looking at the images of the standard unit vectors in $\RR^3$, we can construct the standard matrix $A$ of $T$ with respect to $\mathcal{B}$ and $\mathcal{C}$. 
$$A=\begin{bmatrix}2&-1&1\\-1&0&3\end{bmatrix}$$
Applying this matrix to the coordinate vector of $2\vec{v}_1-\vec{v}_2+3\vec{v}_3$ we get
$$\begin{bmatrix}2&-1&1\\-1&0&3\end{bmatrix}\begin{bmatrix}2\\-1\\3\end{bmatrix}=\begin{bmatrix}8\\7\end{bmatrix}$$

This means that $T(2\vec{v}_1-\vec{v}_2+3\vec{v}_3)=8\vec{w}_1+7\vec{w}_2$.
We can verify this by direct computation as follows:
\begin{align*}
T(2\vec{v}_1-\vec{v}_2+3\vec{v}_3)&=2T(\vec{v}_1)-T(\vec{v}_2)+3T(\vec{v}_3)\\&=2(2\vec{w}_1-\vec{w}_2)-(-\vec{w}_1)+3(\vec{w}_1+3\vec{w}_2)\\&=8\vec{w}_1+7\vec{w}_2
\end{align*}
\end{explanation}

\end{example}

%The following example shows that not only do the basis elements matter, but also their order.
%\begin{example}\label{ex:transmatrix2}
%Let $\mathcal{B}=\left\{\begin{bmatrix}0\\1\end{bmatrix},\begin{bmatrix}1\\0\end{bmatrix}\right\}$ be a basis for $\RR^2$ and let $\mathcal{C}=\left\{\begin{bmatrix}0\\0\\1\end{bmatrix},\begin{bmatrix}0\\1\\0\end{bmatrix}, \begin{bmatrix}1\\0\\0\end{bmatrix}\right\}$ be a basis for $\RR^3$.  Define a linear transformation $T:\RR^2\rightarrow \RR^3$ by 
%$$T\left(\begin{bmatrix}0\\1\end{bmatrix}\right)=\begin{bmatrix}2\\0\\1\end{bmatrix}\quad\text{and}\quad T\left(\begin{bmatrix}1\\0\end{bmatrix}\right)=\begin{bmatrix}-1\\3\\4\end{bmatrix}$$
%Find the matrix of $T$ with respect to $\mathcal{B}$ and $\mathcal{C}$.

%\begin{explanation}
%For simplicity, we will label the elements of $\mathcal{B}$
%$$\vec{v}_1=\begin{bmatrix}0\\1\end{bmatrix},\quad\vec{v}_2=\begin{bmatrix}1\\0\end{bmatrix}$$
%and elements of $\mathcal{C}$
%$$\vec{w}_1=\begin{bmatrix}0\\0\\1\end{bmatrix},\quad\vec{w}_2=\begin{bmatrix}0\\1\\0\end{bmatrix},\quad\vec{w}_3=\begin{bmatrix}1\\0\\0\end{bmatrix}$$
%Note the order of these vectors.  The order is very important!

%Observe that $$T(\vec{v}_1)=\vec{w}_1+2\vec{w}_3,\quad T(\vec{v}_2)=4\vec{w}_1+3\vec{w}_2-\vec{w}_3$$

%This gives us
%$$1\vec{v}_1+0\vec{v}_2\mapsto 1\vec{w}_1+0\vec{w}_2+2\vec{w}_3$$
%$$0\vec{v}_1+1\vec{v}_2\mapsto 4\vec{w}_1+3\vec{w}_2+(-1)\vec{w}_3$$

%So, the ``mapping of coordinate vectors" looks like this

%$$\begin{bmatrix}1\\0\end{bmatrix}\mapsto\begin{bmatrix}1\\0\\2\end{bmatrix}$$
%$$\begin{bmatrix}0\\1\end{bmatrix}\mapsto\begin{bmatrix}4\\3\\-1\end{bmatrix}$$
%The matrix of $T$ with respect to $\mathcal{B}$ and $\mathcal{C}$ is 
%$$A=\begin{bmatrix}1&4\\0&3\\-1&2\end{bmatrix}$$

%Compare this to the matrix of $T$ with respect to the standard basis (in standard order!).
%\end{explanation}
%\end{example}

\subsection*{The Matrix of a Linear Transformation}
In this section we will formalize the process for finding the matrix of a linear transformation with respect to arbitrary bases that we established through earlier examples.

Let $V$ and $W$ be vector spaces with ordered bases $\mathcal{B}=\{\vec{v}_1, \vec{v}_2,\ldots ,\vec{v}_n\}$ and $\mathcal{C}=\{\vec{w}_1, \vec{w}_2,\ldots ,\vec{w}_m\}$, respectively.   Suppose $T:V\rightarrow W$ is a linear transformation.  Our goal is to find a matrix for $T$ with respect to $\mathcal{B}$ and $\mathcal{C}$.

Observe that $\text{dim}(V)=n=\text{dim}(\RR^n)$ and $\text{dim}(W)=m=\text{dim}(\RR^m)$. Let $R:V\rightarrow \RR^n$ and $S:W\rightarrow \RR^m$ be coordinate vector isomorphisms defined by
$$R(\vec{v})=[\vec{v}]_{\mathcal{B}}\quad\text{and}\quad S(\vec{w})=[\vec{w}]_{\mathcal{C}}$$
%Define a linear transformation $R:V\rightarrow \RR^n$ by $R(\vec{v}_i)=\vec{e}_i$ for $1\leq i\leq n$.  In Practice Problem \ref{prob:coordinatemappingisanisom} you will be asked to show that $R$ is one-to-one and onto.  This means that $R$ is an isomorphism.  (Compare this to Exploration Problem \ref{init:coordmapping} of LTR-M-0060)

%Suppose $\vec{v}$ is an element of $V$, then $\vec{v}=a_1\vec{v}_1+a_2\vec{v}_2+\ldots +a_n\vec{v}_n$.  The image of $\vec{v}$ is given by
%\begin{align*}
%R(\vec{v})&=R(a_1\vec{v}_1+a_2\vec{v}_2+\ldots +a_n\vec{v}_n)\\&=a_1R(\vec{v}_1)+a_2R(\vec{v}_2)+\ldots +a_nR(\vec{v}_n)=\begin{bmatrix}a_1\\a_2\\\vdots\\a_n\end{bmatrix}
%\end{align*}

%Thus, the image of $\vec{v}$ is the \dfn{coordinate vector of $\vec{v}$ with respect to $\mathcal{B}$}. We will denote it by $[\vec{v}]_{\mathcal{B}}$.

%Similarly, we can define an isomorphism $S:W\rightarrow \RR^m$ by $R(\vec{w}_i)=\vec{e}_i$ for $1\leq i\leq m$. Thus, the image of a vector $\vec{w}=b_1\vec{w}_1+b_2\vec{w}_2+\ldots +b_m\vec{w}_m$ in $W$ is given by

%\begin{align*}
%S(\vec{w})&=S(b_1\vec{w}_1+b_2\vec{w}_2+\ldots +b_m\vec{w}_m)\\&=b_1S(\vec{w}_1)+b_2S(\vec{w}_2)+\ldots +b_mS(\vec{w}_m)=\begin{bmatrix}b_1\\b_2\\\vdots\\b_n\end{bmatrix}=[\vec{w}]_{\mathcal{C}}
%\end{align*}

%This is the coordinate vector of $\vec{w}$ with respect to $\mathcal{C}$.

We know that $R$ is an isomorphism and $R^{-1}$ exists. Consider the transformation
$$S\circ T\circ R^{-1}:\RR^n\rightarrow \RR^m$$

%\begin{center}
%\begin{tikzpicture}
%\node{
%\begin{tikzcd}[column sep=6em, row sep=huge]
%V  \arrow[d, "R" black, shift left=0.5ex] \arrow[red]{r}[black]{T}
%& W \arrow[red]{d}[black]{S} \\
%\RR^n \arrow[u,red, "R^{-1}" black, shift left=0.5ex]\arrow{r}{S\circ T\circ R^{-1}}
%& \RR^m
%\end{tikzcd}
%};
%\end{tikzpicture}
%\end{center}

\begin{center}
 \begin{tikzpicture} 
   \fill[blue, opacity=0.5] (1,0) rectangle (3,2);
   \fill[orange, opacity=0.6] (4,0) rectangle (6,2);
   
    \fill[blue, opacity=0.3] (1,3) rectangle (3,5);
   \fill[orange, opacity=0.4] (4,3) rectangle (6,5);
   
   \node[] at (1.25, 4.75)  (p2)    {$V$};
   \node[] at (5.75, 4.75)  (r3)    {$W$};
   
   \node[] at (1.25, 0.25)  (p2)    {$\RR^n$};
   \node[] at (5.75, 0.25)  (r3)    {$\RR^m$};
   
%T and tau arrows     
     \draw [->,line width=0.5pt,-stealth, dashed]  (2,-0.1)to[out=340, in=200](5,-0.1 );
     \draw [->,line width=0.5pt,-stealth, blue]  (2,5.1)to[out=20, in=160](5, 5.1);
     
%S and S-1 arrows  
     \draw [->,line width=0.5pt,-stealth, blue]  (6.1,4)to[out=290, in=70](6.1,1 );
     
%R and R-1 arrows  
     \draw [->,line width=0.5pt,-stealth]  (0.9,4.8)to[out=240, in=120](0.9,0.2 );
     \draw [->,line width=0.5pt,-stealth, blue]  (0.9,1)to[out=110, in=250](0.9, 4);
     
%Function labels
      \node[] at (3.5, 5.75)    {$T$};
      \node[] at (3.5, -0.75)    {$S\circ T\circ R^{-1}$};
      
      \node[] at (-0.1, 2.5)    {$R$};
      \node[] at (1, 2.5)    {$R^{-1}$};
      
      \node[] at (6.75, 2.5)    {$S$};
      
  \end{tikzpicture}
\end{center}


As a composition of linear transformation, $S\circ T\circ R^{-1}$ is linear and thus has a standard matrix.
To find it, we need to determine the images of standard unit vectors $\vec{e}_i$ under $S\circ T\circ R^{-1}$. We have the following:

%\begin{center}
%\begin{tikzpicture}
%\node{
%\begin{tikzcd}[column sep=6em, row sep=huge]
%V  \arrow[d, "R" black, shift left=0.5ex] \arrow[red]{r}[black]{T}
%& W \arrow[red]{d}[black]{S} \\
%\RR^n \arrow[u,red, "R^{-1}" black, shift %left=0.5ex]\arrow{r}{S\circ T\circ R^{-1}}
%& [T(\vec{v}_i)]_{\mathcal{C}}
%\end{tikzcd}
%};
%\end{tikzpicture}
%\end{center}

\begin{center}
 \begin{tikzpicture} 
   \fill[blue, opacity=0.5] (1,0) rectangle (3,2);
   \fill[orange, opacity=0.6] (4,0) rectangle (6,2);
   
    \fill[blue, opacity=0.3] (1,3) rectangle (3,5);
   \fill[orange, opacity=0.4] (4,3) rectangle (6,5);
   
   \node[] at (1.25, 4.75)  (p2)    {$V$};
   \node[] at (5.75, 4.75)  (r3)    {$W$};
   
   \node[] at (1.25, 0.25)  (p2)    {$\RR^n$};
   \node[] at (5.75, 0.25)  (r3)    {$\RR^m$};
   
  
    \node[] at (2, 4)  (a)    {$\vec{v}_i$};
     \node[] at (2, 1)  (b)    {$\vec{e}_i=[\vec{v}_i]_{\mathcal{B}}$};
     
    
    \node[] at (5, 4)  (c)    {$T(\vec{v}_i)$};
     \node[] at (5, 1)  (d)    {$[T(\vec{v}_i)]_{\mathcal{C}}$};
     
    
     \draw [->,line width=0.5pt,-stealth, blue]  (a.east)to(c.west);
     \draw [->,line width=0.5pt,-stealth, dashed]  (b.east)to(d.west);
     
     \draw [->,line width=0.5pt,-stealth, blue]  (b.north)to(a.south);
     \draw [->,line width=0.5pt,-stealth, blue]  (c.south)to(d.north);
     
%T and tau arrows     
     \draw [->,line width=0.5pt,-stealth, dashed]  (2,-0.1)to[out=340, in=200](5,-0.1 );
     \draw [->,line width=0.5pt,-stealth, blue]  (2,5.1)to[out=20, in=160](5, 5.1);
     
%S and S-1 arrows  
     \draw [->,line width=0.5pt,-stealth, blue]  (6.1,4)to[out=290, in=70](6.1,1 );
     
%R and R-1 arrows  
     \draw [->,line width=0.5pt,-stealth]  (0.9,4.8)to[out=240, in=120](0.9,0.2 );
     \draw [->,line width=0.5pt,-stealth, blue]  (0.9,1)to[out=110, in=250](0.9, 4);
     
%Function labels
      \node[] at (3.5, 5.75)    {$T$};
      \node[] at (3.5, -0.75)    {$S\circ T\circ R^{-1}$};
      
      \node[] at (-0.1, 2.5)    {$R$};
      \node[] at (1, 2.5)    {$R^{-1}$};
      
      \node[] at (6.75, 2.5)    {$S$};
      
  \end{tikzpicture}
\end{center}

Vectors $[T(\vec{v}_i)]_{\mathcal{C}}$ will become the columns of the standard matrix.  We summarize this discussion as a theorem.

\begin{theorem}\label{th:matlintransgeneral}
Let $V$ and $W$ be finite-dimensional vector spaces with ordered bases $\mathcal{B}=\{\vec{v}_1,\vec{v}_2,\ldots,\vec{v}_n\}$ and $\mathcal{C}$, respectively.  Suppose $T:V\rightarrow W$ is a linear transformation.  
$$\text{Let}\quad A=\begin{bmatrix}
           | & |& &|\\
		[T(\vec{v}_1)]_{\mathcal{C}} & [T(\vec{v}_2)]_{\mathcal{C}}&\dots &[T(\vec{v}_n)]_{\mathcal{C}}\\
		|&| & &|
         \end{bmatrix}$$
Then $A[\vec{v}]_{\mathcal{B}}=[T(\vec{v})]_{\mathcal{C}}$ for all vectors $\vec{v}$ in $V$.       
\end{theorem}

\begin{definition}\label{def:matlintransgenera}
Matrix $A$ of Theorem \ref{th:matlintransgeneral} is called the matrix of $T$ with respect to ordered bases $\mathcal{B}$ and $\mathcal{C}$.
\end{definition}

In conclusion, observe how isomorphisms helped us solve the matrix of a linear transformation problem.  The coordinate mappings $R$ and $S$ are isomorphisms.  This means that $V$ and $\RR^n$ are isomorphic and have the same structural properties.  The same is true for $W$ and $\RR^m$.  In this abstract discussion, we do not know anything about the elements of $V$ and $W$, but isomorphisms allow us to take a problem that we do not know much about and transform it to a familiar problem involving familiar spaces.

\subsection*{The Inverse of a Linear Transformation and its Matrix}

Let $V$ and $W$ be vector spaces.  Suppose $T:V\rightarrow W$ is an invertible linear transformation.  This, of course, means that $T$ is an isomorphism, which means that 
$$\mbox{dim}(V)=\mbox{dim}(W)$$
Let
$\mathcal{B}=\{\vec{v}_1, \vec{v}_2,\ldots ,\vec{v}_n\}$ and $\mathcal{C}=\{\vec{w}_1, \vec{w}_2,\ldots ,\vec{w}_n\}$ be ordered bases of $V$ and $W$, respectively.   We can find the matrix of $T^{-1}$ with respect to $\mathcal{C}$ and $\mathcal{B}$ by finding the standard matrix of the linear transformation $R\circ T^{-1}\circ S^{-1}:\RR^n\rightarrow \RR^n$.

\begin{center}
 \begin{tikzpicture} 
   \fill[blue, opacity=0.5] (1,0) rectangle (3,2);
   \fill[orange, opacity=0.6] (4,0) rectangle (6,2);
   
    \fill[blue, opacity=0.3] (1,3) rectangle (3,5);
   \fill[orange, opacity=0.4] (4,3) rectangle (6,5);
   
   \node[] at (1.25, 4.75)  (p2)    {$V$};
   \node[] at (5.75, 4.75)  (r3)    {$W$};
   
   \node[] at (1.25, 0.25)  (p2)    {$\RR^n$};
   \node[] at (5.75, 0.25)  (r3)    {$\RR^n$};
   
%T and tau arrows     
     \draw [->,line width=0.5pt,-stealth, dashed, blue]  (2,-0.1)to[out=340, in=200](5,-0.1 );
     \draw [->,line width=0.5pt,-stealth, dashed, red]  (5.9,-0.1)to[out=240, in=300](1.1,-0.1 );
     \draw [->,line width=0.5pt,-stealth, blue]  (2,5.1)to[out=20, in=160](5, 5.1);
     \draw [->,line width=0.5pt,-stealth,red]  (5.9,5.1)to[out=150, in=30](1.1, 5.1);
     
%S and S-1 arrows  
     \draw [->,line width=0.5pt,-stealth, blue]  (6.1,4)to[out=290, in=70](6.1,1 );
     \draw [->,line width=0.5pt,-stealth, red]  (6.1,0.1)to[out=60, in=300](6.1,4.9 );
     
%R and R-1 arrows  
     \draw [->,line width=0.5pt,-stealth, red]  (0.9,4.8)to[out=240, in=120](0.9,0.2 );
     \draw [->,line width=0.5pt,-stealth, blue]  (0.9,1)to[out=110, in=250](0.9, 4);
     
%Function labels
      \node[] at (3.5, 5.15)    {$T$};
      \node[] at (3.5, 6.1)    {$T^{-1}$};
      \node[] at (3.5, -0.75)    {$S\circ T\circ R^{-1}$};
      \node[] at (3.5, -1.7)    {$R\circ T^{-1}\circ S^{-1}$};
      
      \node[] at (-0.1, 2.5)    {$R$};
      \node[] at (1, 2.5)    {$R^{-1}$};
      
      \node[] at (6, 2.5)    {$S$};
      \node[] at (7.25, 2.5)    {$S^{-1}$};
      
  \end{tikzpicture}
\end{center}

Observe that $R\circ T^{-1}\circ S^{-1}$ is the inverse of $S\circ T\circ R^{-1}$. So, if $A$ is the standard matrix of $S\circ T\circ R^{-1}$, then $A^{-1}$ is the standard matrix of $R\circ T^{-1}\circ S^{-1}$.  Thus,  $A^{-1}$ is the matrix of $T^{-1}$ with respect to $\mathcal{C}$ and $\mathcal{B}$.

\begin{example}\label{ex:inversematrixoftransform}
In Exploration Problem \ref{init:subtosub} of \href{\xmbaseurl/LTR-0030/main}{Composition and Inverses of Linear Transformations}, we introduced the following set up.

Let $$V=\text{span}\left(\begin{bmatrix}1\\0\\0\end{bmatrix}, \begin{bmatrix}1\\1\\1\end{bmatrix}\right)$$
Define a linear transformation $$T:V\rightarrow \RR^2$$
by $$T\left(\begin{bmatrix}1\\0\\0\end{bmatrix}\right)=\begin{bmatrix}1\\1\end{bmatrix}\quad \text{and} \quad T\left(\begin{bmatrix}1\\1\\1\end{bmatrix}\right)=\begin{bmatrix}0\\1\end{bmatrix}$$

In Example \ref{ex:subtosubinvert} of \href{\xmbaseurl/LTR-0035/main}{Existence of the Inverse of a Linear Transformation}, we proved that $T$ is invertible.  

Find the matrix of $T^{-1}$ with respect to ordered bases $\mathcal{C}=\left\{\begin{bmatrix}1\\0\end{bmatrix},\begin{bmatrix}0\\1\end{bmatrix}\right\}$ and $\mathcal{B}=\left\{\begin{bmatrix}1\\0\\0\end{bmatrix}, \begin{bmatrix}1\\1\\1\end{bmatrix}\right\}$
\begin{explanation}
Consider the diagram:
\begin{center}
 \begin{tikzpicture} 
   \fill[blue, opacity=0.5] (1,0) rectangle (4,3);
   \fill[orange, opacity=0.6] (5,0) rectangle (10,3);
   
    \fill[blue, opacity=0.3] (1,4) rectangle (4,7);
   \fill[orange, opacity=0.4] (5,4) rectangle (10,7);
   
   \node[] at (1.5, 7.5)  (p2)    {$V$};
   \node[] at (9.5, 7.5)  (r3)    {$\RR^2$};
   
   \node[] at (1.5, -0.5)  (p2)    {$\RR^2$};
   \node[] at (9.5, -0.5)  (r3)    {$\RR^2$};
   
  
    \node[] at (2, 6)  (a)    {$\begin{bmatrix}1\\0\\0\end{bmatrix}$};
     \node[] at (2, 1)  (b)    {$\begin{bmatrix}1\\0\end{bmatrix}$};
     
    
    \node[] at (8.5, 6)  (c)    {$\begin{bmatrix}1\\1\end{bmatrix}$};
     \node[] at (8.5, 1)  (d)    {$\begin{bmatrix}1\\1\end{bmatrix}$};
     
     \node[] at (3, 5)  (e)    {$\begin{bmatrix}1\\1\\1\end{bmatrix}$};
     \node[] at (3, 2)  (f)    {$\begin{bmatrix}0\\1\end{bmatrix}$};
     
    
    \node[] at (7, 5)  (g)    {$\begin{bmatrix}0\\1\end{bmatrix}$};
     \node[] at (7, 2)  (h)    {$\begin{bmatrix}0\\1\end{bmatrix}$};
     
    
     \draw [->,line width=0.5pt,-stealth]  (a.east)to(c.west);
     \draw [->,line width=0.5pt,-stealth, dashed]  (b.east)to(d.west);
     
     \draw [->,line width=0.5pt,-stealth]  (a.south)to(b.north);
     \draw [->,line width=0.5pt,-stealth]  (c.south)to(d.north);
     
     \draw [->,line width=0.5pt,-stealth]  (e.east)to(g.west);
     \draw [->,line width=0.5pt,-stealth, dashed]  (f.east)to(h.west);
     
     \draw [->,line width=0.5pt,-stealth]  (e.south)to(f.north);
     \draw [->,line width=0.5pt,-stealth]  (g.south)to(h.north);
     
%T and tau arrows     
    % \draw [->,line width=0.5pt,-stealth, dashed]  (2.5,-0.1)to[out=340, in=200](8.5,-0.1 );
     \draw [->,line width=0.5pt,-stealth]  (2.5,7.1)to[out=20, in=160](8.5, 7.1);
     
%S and S-1 arrows  
     \draw [->,line width=0.5pt,-stealth]  (10.1,6)to[out=290, in=70](10.1,1 );
     \draw [->,line width=0.5pt,-stealth]  (10.1,2)to[out=70, in=290](10.1, 5);
     
%R and R-1 arrows  
     \draw [->,line width=0.5pt,-stealth]  (0.9,6)to[out=250, in=110](0.9,1 );
     \draw [->,line width=0.5pt,-stealth]  (0.9,2)to[out=110, in=250](0.9, 5);
     
%Function labels
      \node[] at (5.5, 8)    {$T$};
      %\node[] at (5.5, -1)    {$F$};
      
      \node[] at (0.1, 3.5)    {$R$};
      \node[] at (1, 3.5)    {$R^{-1}$};
      
      \node[] at (10.9, 3.5)    {$S$};
      \node[] at (10, 3.5)    {$S^{-1}$};
      
      
  \end{tikzpicture}
\end{center}






This gives us the matrix of $T$ with respect to $\mathcal{B}$ and $\mathcal{C}$:
$$A=\begin{bmatrix}\answer{1}&\answer{0}\\\answer{1}&\answer{1}\end{bmatrix}$$
We now find $A^{-1}$. 
$$A^{-1}=\begin{bmatrix}\answer{1}&\answer{0}\\\answer{-1}&\answer{1}\end{bmatrix}$$
$A^{-1}$ is the matrix of $T^{-1}$ with respect to $\mathcal{C}$ and $\mathcal{B}$.
\end{explanation}

\end{example}

\section*{Practice Problems}

\begin{problem}\label{prob:taulinear}
Verify that $\tau_2$ of Exploration \ref{init:taumatrix} is linear.
\end{problem}

\begin{problem}
Define $T:\mathbb{M}_{2,2}\rightarrow \mathbb{L}$ by 
$$T\left(\begin{bmatrix}a&b\\c&d\end{bmatrix}\right)=(a+b)+(c+d)x$$

\begin{problem}\label{prob:m22tolpart1}
Show that $T$ is linear.
\end{problem}

\begin{problem}\label{prob:m22tolpart2}
Is $T$ an isomorphism? \wordChoice{\choice{Yes}, \choice[correct]{No}}
\end{problem}


\begin{problem}\label{prob:m22tolpart3}
Let
$$\mathcal{B}=\left\{\begin{bmatrix}1&0\\0&0\end{bmatrix}, \begin{bmatrix}0&1\\0&0\end{bmatrix}, \begin{bmatrix}0&0\\1&0\end{bmatrix}, \begin{bmatrix}0&0\\0&1\end{bmatrix}\right\}$$
$$\mathcal{C}=\{1, x\}$$
be ordered bases for $\mathbb{M}_{2,2}$ and $\mathbb{L}$, respectively.

Find the matrix for $T$ with respect to $\mathcal{B}$ and $\mathcal{C}$.

Answer:
$$\begin{bmatrix}\answer{1}&\answer{1}&\answer{0}&\answer{0}\\\answer{0}&\answer{0}&\answer{1}&\answer{1}\end{bmatrix}$$

\end{problem}
\end{problem}

\begin{problem}\label{prob:matlintransabstract1}
Let $V$ and $W$ be vector spaces with ordered bases $\mathcal{B}=\{\vec{v}_1, \vec{v}_2\}$ and $\mathcal{C}=\{\vec{w}_1, \vec{w}_2, \vec{w}_3\}$, respectively.  Define a linear transformation $T:V\rightarrow W$ by $$T(\vec{v}_1)=\vec{w}_1+2\vec{w}_2-\vec{w}_3,\quad T(\vec{v}_2)=3\vec{w}_2+\vec{w}_3$$
Find the matrix $A$ of $T$ with respect to $\mathcal{B}$ and $\mathcal{C}$, and use it to find $T(-\vec{v}_1-3\vec{v}_2)$.  Verify your answer by computing $T(-\vec{v}_1-3\vec{v}_2)$ directly.

$$A=\begin{bmatrix}\answer{1}&\answer{0}\\\answer{2}&\answer{3}\\\answer{-1}&\answer{1}\end{bmatrix}$$

$$T(-\vec{v}_1-3\vec{v}_2)=\answer{-1}\vec{w}_1+\answer{-11}\vec{w}_2+\answer{-2}\vec{w}_3$$
\end{problem}

\begin{problem}
Let $$\vec{v}_1=\begin{bmatrix}2\\0\\1\end{bmatrix}\quad\text{and}\quad\vec{v}_2=\begin{bmatrix}3\\1\\1\end{bmatrix}$$
$$\vec{w}_1=\begin{bmatrix}-1\\1\\1\end{bmatrix}\quad\text{and}\quad\vec{w}_2=\begin{bmatrix}1\\1\\0\end{bmatrix}$$
Let $V$ and $W$ be subspaces of $\RR^3$ with ordered bases $$\mathcal{B}=\{\vec{v}_1, \vec{v}_2\}\quad\text{and}\quad\mathcal{C}=\{\vec{w}_1, \vec{w}_2\}$$
respectively.

Let $T:V\rightarrow W$ be a linear transformation such that 
$$T(\vec{v}_1)=\begin{bmatrix}-1\\3\\2\end{bmatrix}\quad\text{and}\quad T(\vec{v}_2)=\begin{bmatrix}2\\4\\1\end{bmatrix}$$

	\begin{problem}\label{prob:matlintransabstract2}
    Show that $\begin{bmatrix}-1\\3\\2\end{bmatrix}, \begin{bmatrix}2\\4\\1\end{bmatrix}$ lie in $W$ by expressing them as linear combinations of $\vec{w}_1$ and $\vec{w}_2$.
     $$\begin{bmatrix}-1\\3\\2\end{bmatrix}=\answer{2}\vec{w}_1+\answer{1}\vec{w}_2$$
    $$\begin{bmatrix}2\\4\\1\end{bmatrix}=\answer{1}\vec{w}_1+\answer{3}\vec{w}_2$$
    \end{problem}

	\begin{problem}\label{prob:matlintransabstract3}
    Find the matrix of $T$ with respect to $\mathcal{B}$ and $\mathcal{C}$.
    
    $$\begin{bmatrix}\answer{2}&\answer{1}\\\answer{1}&\answer{3}\end{bmatrix}$$
    \end{problem}

	\begin{problem}\label{prob:matlintransabstract4}
    Find the matrix of $T^{-1}$ with respect to $\mathcal{C}$ and $\mathcal{B}$.
    
    $$\begin{bmatrix}\answer{3/5}&\answer{-1/5}\\\answer{-1/5}&\answer{2/5}\end{bmatrix}$$
    \end{problem}
    
    \begin{problem}\label{prob:matlintransabstract5}
    Find $T^{-1}\left(\begin{bmatrix}10\\10\\0\end{bmatrix}\right)$.
    
    $$T^{-1}\left(\begin{bmatrix}10\\10\\0\end{bmatrix}\right)=\begin{bmatrix}\answer{8}\\\answer{4}\\\answer{2}\end{bmatrix}$$
    \end{problem}

\end{problem}

\end{document}
