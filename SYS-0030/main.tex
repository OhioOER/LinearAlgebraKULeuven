\documentclass{ximera}
%%% Begin Laad packages

\makeatletter
\@ifclassloaded{xourse}{%
    \typeout{Start loading preamble.tex (in a XOURSE)}%
    \def\isXourse{true}   % automatically defined; pre 112022 it had to be set 'manually' in a xourse
}{%
    \typeout{Start loading preamble.tex (NOT in a XOURSE)}%
}
\makeatother

\def\isEn\true 

\pgfplotsset{compat=1.16}

\usepackage{currfile}

% 201908/202301: PAS OP: babel en doclicense lijken problemen te veroorzaken in .jax bestand
% (wegens syntax error met toegevoegde \newcommands ...)
\pdfOnly{
    \usepackage[type={CC},modifier={by-nc-sa},version={4.0}]{doclicense}
    %\usepackage[hyperxmp=false,type={CC},modifier={by-nc-sa},version={4.0}]{doclicense}
    %%% \usepackage[dutch]{babel}
}



\usepackage[utf8]{inputenc}
\usepackage{morewrites}   % nav zomercursus (answer...?)
\usepackage{multirow}
\usepackage{multicol}
\usepackage{tikzsymbols}
\usepackage{ifthen}
%\usepackage{animate} BREAKS HTML STRUCTURE USED BY XIMERA
\usepackage{relsize}

\usepackage{eurosym}    % \euro  (€ werkt niet in xake ...?)
\usepackage{fontawesome} % smileys etc

% Nuttig als ook interactieve beamer slides worden voorzien:
\providecommand{\p}{} % default nothing ; potentially usefull for slides: redefine as \pause
%providecommand{\p}{\pause}

    % Layout-parameters voor het onderschrift bij figuren
\usepackage[margin=10pt,font=small,labelfont=bf, labelsep=endash,format=hang]{caption}
%\usepackage{caption} % captionof
%\usepackage{pdflscape}    % landscape environment

% Met "\newcommand\showtodonotes{}" kan je todonotes tonen (in pdf/online)
% 201908: online werkt het niet (goed)
\providecommand\showtodonotes{disable}
\providecommand\todo[1]{\typeout{TODO #1}}
%\usepackage[\showtodonotes]{todonotes}
%\usepackage{todonotes}

%
% Poging tot aanpassen layout
%
\usepackage{tcolorbox}
\tcbuselibrary{theorems}

%%% Einde laad packages

%%% Begin Ximera specifieke zaken

\graphicspath{
	{../../}
	{../}
	{./}
  	{../../pictures/}
   	{../pictures/}
   	{./pictures/}
	{./explog/}    % M05 in groeimodellen       
}

%%% Einde Ximera specifieke zaken

%
% define softer blue/red/green, use KU Leuven base colors for blue (and dark orange for red ?)
%
% todo: rather redefine blue/red/green ...?
%\definecolor{xmblue}{rgb}{0.01, 0.31, 0.59}
%\definecolor{xmred}{rgb}{0.89, 0.02, 0.17}
\definecolor{xmdarkblue}{rgb}{0.122, 0.671, 0.835}   % KU Leuven Blauw
\definecolor{xmblue}{rgb}{0.114, 0.553, 0.69}        % KU Leuven Blauw
\definecolor{xmgreen}{rgb}{0.13, 0.55, 0.13}         % No KULeuven variant for green found ...

\definecolor{xmaccent}{rgb}{0.867, 0.541, 0.18}      % KU Leuven Accent (orange ...)
\definecolor{kuaccent}{rgb}{0.867, 0.541, 0.18}      % KU Leuven Accent (orange ...)

\colorlet{xmred}{xmaccent!50!black}                  % Darker version of KU Leuven Accent

\providecommand{\blue}[1]{{\color{blue}#1}}    
\providecommand{\red}[1]{{\color{red}#1}}

\renewcommand\CancelColor{\color{xmaccent!50!black}}

% werkt in math en text mode om MATH met oranje (of grijze...)  achtergond te tonen (ook \important{\text{blabla}} lijkt te werken)
%\newcommand{\important}[1]{\ensuremath{\colorbox{xmaccent!50!white}{$#1$}}}   % werkt niet in Mathjax
%\newcommand{\important}[1]{\ensuremath{\colorbox{lightgray}{$#1$}}}
\newcommand{\important}[1]{\ensuremath{\colorbox{orange}{$#1$}}}   % TODO: kleur aanpassen voor mathjax; wordt overschreven infra!


% Uitzonderlijk kan met \pdfnl in de PDF een newline worden geforceerd, die online niet nodig/nuttig is omdat daar de regellengte hoe dan ook niet gekend is.
\ifdefined\HCode%
\providecommand{\pdfnl}{}%
\else%
\providecommand{\pdfnl}{%
  \\%
}%
\fi

% Uitzonderlijk kan met \handoutnl in de handout-PDF een newline worden geforceerd, die noch online noch in de PDF-met-antwoorden nuttig is.
\ifdefined\HCode
\providecommand{\handoutnl}{}
\else
\providecommand{\handoutnl}{%
\ifhandout%
  \nl%
\fi%
}
\fi



% \cellcolor IGNORED by tex4ht ?
% \begin{center} seems not to wordk
    % (missing margin-left: auto;   on tabular-inside-center ???)
%\newcommand{\importantcell}[1]{\ensuremath{\cellcolor{lightgray}#1}}  %  in tabular; usablility to be checked
\providecommand{\importantcell}[1]{\ensuremath{#1}}     % no mathjax2 support for colloring array cells

\pdfOnly{
  \renewcommand{\important}[1]{\ensuremath{\colorbox{kuaccent!50!white}{$#1$}}}
  \renewcommand{\importantcell}[1]{\ensuremath{\cellcolor{kuaccent!40!white}#1}}   
}

%%% Tikz styles


\pgfplotsset{compat=1.16}

\usetikzlibrary{trees,positioning,arrows,fit,shapes,math,calc,decorations.markings,through,intersections,patterns,matrix}

\usetikzlibrary{decorations.pathreplacing,backgrounds}    % 5/2023: from experimental


\usetikzlibrary{angles,quotes}

\usepgfplotslibrary{fillbetween} % bepaalde_integraal
\usepgfplotslibrary{polar}    % oa voor poolcoordinaten.tex

\pgfplotsset{ownstyle/.style={axis lines = center, axis equal image, xlabel = $x$, ylabel = $y$, enlargelimits}} 

\pgfplotsset{
	plot/.style={no marks,samples=50}
}

\newcommand{\xmPlotsColor}{
	\pgfplotsset{
		plot1/.style={darkgray,no marks,samples=100},
		plot2/.style={lightgray,no marks,samples=100},
		plotresult/.style={blue,no marks,samples=100},
		plotblue/.style={blue,no marks,samples=100},
		plotred/.style={red,no marks,samples=100},
		plotgreen/.style={green,no marks,samples=100},
		plotpurple/.style={purple,no marks,samples=100}
	}
}
\newcommand{\xmPlotsBlackWhite}{
	\pgfplotsset{
		plot1/.style={black,loosely dashed,no marks,samples=100},
		plot2/.style={black,loosely dotted,no marks,samples=100},
		plotresult/.style={black,no marks,samples=100},
		plotblue/.style={black,no marks,samples=100},
		plotred/.style={black,dotted,no marks,samples=100},
		plotgreen/.style={black,dashed,no marks,samples=100},
		plotpurple/.style={black,dashdotted,no marks,samples=100}
	}
}


\newcommand{\xmPlotsColorAndStyle}{
	\pgfplotsset{
		plot1/.style={darkgray,no marks,samples=100},
		plot2/.style={lightgray,no marks,samples=100},
		plotresult/.style={blue,no marks,samples=100},
		plotblue/.style={xmblue,no marks,samples=100},
		plotred/.style={xmred,dashed,thick,no marks,samples=100},
		plotgreen/.style={xmgreen,dotted,very thick,no marks,samples=100},
		plotpurple/.style={purple,no marks,samples=100}
	}
}


%\iftikzexport
\xmPlotsColorAndStyle
%\else
%\xmPlotsBlackWhite
%\fi
%%%


%
% Om venndiagrammen te arceren ...
%
\makeatletter
\pgfdeclarepatternformonly[\hatchdistance,\hatchthickness]{north east hatch}% name
{\pgfqpoint{-1pt}{-1pt}}% below left
{\pgfqpoint{\hatchdistance}{\hatchdistance}}% above right
{\pgfpoint{\hatchdistance-1pt}{\hatchdistance-1pt}}%
{
	\pgfsetcolor{\tikz@pattern@color}
	\pgfsetlinewidth{\hatchthickness}
	\pgfpathmoveto{\pgfqpoint{0pt}{0pt}}
	\pgfpathlineto{\pgfqpoint{\hatchdistance}{\hatchdistance}}
	\pgfusepath{stroke}
}
\pgfdeclarepatternformonly[\hatchdistance,\hatchthickness]{north west hatch}% name
{\pgfqpoint{-\hatchthickness}{-\hatchthickness}}% below left
{\pgfqpoint{\hatchdistance+\hatchthickness}{\hatchdistance+\hatchthickness}}% above right
{\pgfpoint{\hatchdistance}{\hatchdistance}}%
{
	\pgfsetcolor{\tikz@pattern@color}
	\pgfsetlinewidth{\hatchthickness}
	\pgfpathmoveto{\pgfqpoint{\hatchdistance+\hatchthickness}{-\hatchthickness}}
	\pgfpathlineto{\pgfqpoint{-\hatchthickness}{\hatchdistance+\hatchthickness}}
	\pgfusepath{stroke}
}
%\makeatother

\tikzset{
    hatch distance/.store in=\hatchdistance,
    hatch distance=10pt,
    hatch thickness/.store in=\hatchthickness,
   	hatch thickness=2pt
}

\colorlet{circle edge}{black}
\colorlet{circle area}{blue!20}


\tikzset{
    filled/.style={fill=green!30, draw=circle edge, thick},
    arceerl/.style={pattern=north east hatch, pattern color=blue!50, draw=circle edge},
    arceerr/.style={pattern=north west hatch, pattern color=yellow!50, draw=circle edge},
    outline/.style={draw=circle edge, thick}
}




%%% Updaten commando's
\def\hoofding #1#2#3{\maketitle}     % OBSOLETE ??

% we willen (bijna) altijd \geqslant ipv \geq ...!
\newcommand{\geqnoslant}{\geq}
\renewcommand{\geq}{\geqslant}
\newcommand{\leqnoslant}{\leq}
\renewcommand{\leq}{\leqslant}

% Todo: (201908) waarom komt er (soms) underlined voor emph ...?
\renewcommand{\emph}[1]{\textit{#1}}

% API commando's

\newcommand{\ds}{\displaystyle}
\newcommand{\ts}{\textstyle}  % tegenhanger van \ds   (Ximera zet PER  DEFAULT \ds!)

% uit Zomercursus-macro's: 
\newcommand{\bron}[1]{\begin{scriptsize} \emph{#1} \end{scriptsize}}     % deprecated ...?


%definities nieuwe commando's - afkortingen veel gebruikte symbolen
\newcommand{\R}{\ensuremath{\mathbb{R}}}
\newcommand{\Rnul}{\ensuremath{\mathbb{R}_0}}
\newcommand{\Reen}{\ensuremath{\mathbb{R}\setminus\{1\}}}
\newcommand{\Rnuleen}{\ensuremath{\mathbb{R}\setminus\{0,1\}}}
\newcommand{\Rplus}{\ensuremath{\mathbb{R}^+}}
\newcommand{\Rmin}{\ensuremath{\mathbb{R}^-}}
\newcommand{\Rnulplus}{\ensuremath{\mathbb{R}_0^+}}
\newcommand{\Rnulmin}{\ensuremath{\mathbb{R}_0^-}}
\newcommand{\Rnuleenplus}{\ensuremath{\mathbb{R}^+\setminus\{0,1\}}}
\newcommand{\N}{\ensuremath{\mathbb{N}}}
\newcommand{\Nnul}{\ensuremath{\mathbb{N}_0}}
\newcommand{\Z}{\ensuremath{\mathbb{Z}}}
\newcommand{\Znul}{\ensuremath{\mathbb{Z}_0}}
\newcommand{\Zplus}{\ensuremath{\mathbb{Z}^+}}
\newcommand{\Zmin}{\ensuremath{\mathbb{Z}^-}}
\newcommand{\Znulplus}{\ensuremath{\mathbb{Z}_0^+}}
\newcommand{\Znulmin}{\ensuremath{\mathbb{Z}_0^-}}
\newcommand{\C}{\ensuremath{\mathbb{C}}}
\newcommand{\Cnul}{\ensuremath{\mathbb{C}_0}}
\newcommand{\Cplus}{\ensuremath{\mathbb{C}^+}}
\newcommand{\Cmin}{\ensuremath{\mathbb{C}^-}}
\newcommand{\Cnulplus}{\ensuremath{\mathbb{C}_0^+}}
\newcommand{\Cnulmin}{\ensuremath{\mathbb{C}_0^-}}
\newcommand{\Q}{\ensuremath{\mathbb{Q}}}
\newcommand{\Qnul}{\ensuremath{\mathbb{Q}_0}}
\newcommand{\Qplus}{\ensuremath{\mathbb{Q}^+}}
\newcommand{\Qmin}{\ensuremath{\mathbb{Q}^-}}
\newcommand{\Qnulplus}{\ensuremath{\mathbb{Q}_0^+}}
\newcommand{\Qnulmin}{\ensuremath{\mathbb{Q}_0^-}}

\newcommand{\perdef}{\overset{\mathrm{def}}{=}}
\newcommand{\pernot}{\overset{\mathrm{notatie}}{=}}
\newcommand\perinderdaad{\overset{!}{=}}     % voorlopig gebruikt in limietenrekenregels
\newcommand\perhaps{\overset{?}{=}}          % voorlopig gebruikt in limietenrekenregels

\newcommand{\degree}{^\circ}


\DeclareMathOperator{\dom}{dom}     % domein
\DeclareMathOperator{\codom}{codom} % codomein
\DeclareMathOperator{\bld}{bld}     % beeld
\DeclareMathOperator{\graf}{graf}   % grafiek
\DeclareMathOperator{\rico}{rico}   % richtingcoëfficient
\DeclareMathOperator{\co}{co}       % coordinaat
\DeclareMathOperator{\gr}{gr}       % graad

\newcommand{\func}[5]{\ensuremath{#1: #2 \rightarrow #3: #4 \mapsto #5}} % Easy to write a function


% Operators
\DeclareMathOperator{\bgsin}{bgsin}
\DeclareMathOperator{\bgcos}{bgcos}
\DeclareMathOperator{\bgtan}{bgtan}
\DeclareMathOperator{\bgcot}{bgcot}
\DeclareMathOperator{\bgsinh}{bgsinh}
\DeclareMathOperator{\bgcosh}{bgcosh}
\DeclareMathOperator{\bgtanh}{bgtanh}
\DeclareMathOperator{\bgcoth}{bgcoth}

% Oude \Bgsin etc deprecated: gebruik \bgsin, en herdefinieer dat als je Bgsin wil!
%\DeclareMathOperator{\cosec}{cosec}    % not used? gebruik \csc en herdefinieer

% operatoren voor differentialen: to be verified; 1/2020: inconsequent gebruik bij afgeleiden/integralen
\renewcommand{\d}{\mathrm{d}}
\newcommand{\dx}{\d x}
\newcommand{\dd}[1]{\frac{\mathrm{d}}{\mathrm{d}#1}}
\newcommand{\ddx}{\dd{x}}

% om in voorbeelden/oefeningen de notatie voor afgeleiden te kunnen kiezen
% Usage: \afg{(2\sin(x))}  (en wordt d/dx, of accent, of D )
%\newcommand{\afg}[1]{{#1}'}
\newcommand{\afg}[1]{\left(#1\right)'}
%\renewcommand{\afg}[1]{\frac{\mathrm{d}#1}{\mathrm{d}x}}   % include in relevant exercises ...
%\renewcommand{\afg}[1]{D{#1}}

%
% \xmxxx commands: Extra KU Leuven functionaliteit van, boven of naast Ximera
%   ( Conventie 8/2019: xm+nederlandse omschrijving, maar is niet consequent gevolgd, en misschien ook niet erg handig !)
%
% (Met een minimale ximera.cls en preamble.tex zou een bruikbare .pdf moeten kunnen worden gemaakt van eender welke ximera)
%
% Usage: \xmtitle[Mijn korte abstract]{Mijn titel}{Mijn abstract}
% Eerste command na \begin{document}:
%  -> definieert de \title
%  -> definieert de abstract
%  -> doet \maketitle ( dus: print de hoofding als 'chapter' of 'sectie')
% Optionele parameter geeft eenn kort abstract (die met de globale setting \xmshortabstract{} al dan niet kan worden geprint.
% De optionele korte abstract kan worden gebruikt voor pseudo-grappige abtsarts, dus dus globaal al dan niet kunnen worden gebuikt...
% Globale settings:
%  de (optionele) 'korte abstract' wordt enkele getoond als \xmshortabstract is gezet
\providecommand\xmshortabstract{} % default: print (only!) short abstract if present
\newcommand{\xmtitle}[3][]{
	\title{#2}
	\begin{abstract}
		\ifdefined\xmshortabstract
		\ifstrempty{#1}{%
			#3
		}{%
			#1
		}%
		\else
		#3
		\fi
	\end{abstract}
	\maketitle
}

% 
% Kleine grapjes: moeten zonder verder gevolg kunnen worden verwijderd
%
%\newcommand{\xmopje}[1]{{\small#1{\reversemarginpar\marginpar{\Smiley}}}}   % probleem in floats!!
\newtoggle{showxmopje}
\toggletrue{showxmopje}

\newcommand{\xmopje}[1]{%
   \iftoggle{showxmopje}{#1}{}%
}


% -> geef een abstracte-formule-met-rechts-een-concreet-voorbeeld
% VB:  \formulevb{a^2+b^2=c^2}{3^2+4^2=5^2}
%
\ifdefined\HCode
\NewEnviron{xmdiv}[1]{\HCode{\Hnewline<div class="#1">\Hnewline}\BODY{\HCode{\Hnewline</div>\Hnewline}}}
\else
\NewEnviron{xmdiv}[1]{\BODY}
\fi

\providecommand{\formulevb}[2]{
	{\centering

    \begin{xmdiv}{xmformulevb}    % zie css voor online layout !!!
	\begin{tabular}{lcl}
		\important{#1}
		&  &
		Vb: $#2$
		\end{tabular}
	\end{xmdiv}

	}
}

\ifdefined\HCode
\providecommand{\vb}[1]{%
    \HCode{\Hnewline<span class="xmvb">}#1\HCode{</span>\Hnewline}%
}
\else
\providecommand{\vb}[1]{
    \colorbox{blue!10}{#1}
}
\fi

\ifdefined\HCode
\providecommand{\xmcolorbox}[2]{
	\HCode{\Hnewline<div class="xmcolorbox">\Hnewline}#2\HCode{\Hnewline</div>\Hnewline}
}
\else
\providecommand{\xmcolorbox}[2]{
  \cellcolor{#1}#2
}
\fi


\ifdefined\HCode
\providecommand{\xmopmerking}[1]{
 \HCode{\Hnewline<div class="xmopmerking">\Hnewline}#1\HCode{\Hnewline</div>\Hnewline}
}
\else
\providecommand{\xmopmerking}[1]{
	{\footnotesize #1}
}
\fi
% \providecommand{\voorbeeld}[1]{
% 	\colorbox{blue!10}{$#1$}
% }



% Hernoem Proof naar Bewijs, nodig voor HTML versie
\renewcommand*{\proofname}{Bewijs}

% Om opgave van oefening (wordt niet geprint bij oplossingenblad)
% (to be tested test)
\NewEnviron{statement}{\BODY}

% Environment 'oplossing' en 'uitkomst'
% voor resp. volledige 'uitwerking' dan wel 'enkel eindresultaat'
% geimplementeerd via feedback, omdat er in de ximera-server adhoc feedback-code is toegevoegd
%% Niet tonen indien handout
%% Te gebruiken om volledige oplossingen/uitwerkingen van oefeningen te tonen
%% \begin{oplossing}        De optelling is commutatief \end{oplossing}  : verschijnt online enkel 'op vraag'
%% \begin{oplossing}[toon]  De optelling is commutatief \end{oplossing}  : verschijnt steeds onmiddellijk online (bv te gebruiken bij voorbeelden) 

\ifhandout%
    \NewEnviron{oplossing}[1][onzichtbaar]%
    {%
    \ifthenelse{\equal{\detokenize{#1}}{\detokenize{toon}}}
    {
    \def\PH@Command{#1}% Use PH@Command to hold the content and be a target for "\expandafter" to expand once.

    \begin{trivlist}% Begin the trivlist to use formating of the "Feedback" label.
    \item[\hskip \labelsep\small\slshape\bfseries Oplossing% Format the "Feedback" label. Don't forget the space.
    %(\texttt{\detokenize\expandafter{\PH@Command}}):% Format (and detokenize) the condition for feedback to trigger
    \hspace{2ex}]\small%\slshape% Insert some space before the actual feedback given.
    \BODY
    \end{trivlist}
    }
    {  % \begin{feedback}[solution]   \BODY     \end{feedback}  }
    }
    }    
\else
% ONLY for HTML; xmoplossing is styled with css, and is not, and need not be a LaTeX environment
% THUS: it does NOT use feedback anymore ...
%    \NewEnviron{oplossing}{\begin{expandable}{xmoplossing}{\nlen{Toon uitwerking}{Show solution}}{\BODY}\end{expandable}}
    \newenvironment{oplossing}[1][onzichtbaar]
   {%
       \begin{expandable}{xmoplossing}{}
   }
   {%
   	   \end{expandable}
   } 
%     \newenvironment{oplossing}[1][onzichtbaar]
%    {%
%        \begin{feedback}[solution]   	
%    }
%    {%
%    	   \end{feedback}
%    } 
\fi

\ifhandout%
    \NewEnviron{uitkomst}[1][onzichtbaar]%
    {%
    \ifthenelse{\equal{\detokenize{#1}}{\detokenize{toon}}}
    {
    \def\PH@Command{#1}% Use PH@Command to hold the content and be a target for "\expandafter" to expand once.

    \begin{trivlist}% Begin the trivlist to use formating of the "Feedback" label.
    \item[\hskip \labelsep\small\slshape\bfseries Uitkomst:% Format the "Feedback" label. Don't forget the space.
    %(\texttt{\detokenize\expandafter{\PH@Command}}):% Format (and detokenize) the condition for feedback to trigger
    \hspace{2ex}]\small%\slshape% Insert some space before the actual feedback given.
    \BODY
    \end{trivlist}
    }
    {  % \begin{feedback}[solution]   \BODY     \end{feedback}  }
    }
    }    
\else
\ifdefined\HCode
   \newenvironment{uitkomst}[1][onzichtbaar]
    {%
        \begin{expandable}{xmuitkomst}{}%
    }
    {%
    	\end{expandable}%
    } 
\else
  % Do NOT print 'uitkomst' in non-handout
  %  (presumably, there is also an 'oplossing' ??)
  \newenvironment{uitkomst}[1][onzichtbaar]{}{}
\fi
\fi

%
% Uitweidingen zijn extra's die niet redelijkerwijze tot de leerstof behoren
% Uitbreidingen zijn extra's die wel redelijkerwijze tot de leerstof van bv meer geavanceerde versies kunnen behoren (B-programma/Wiskundestudenten/...?)
% Nog niet voorzien: design voor verschillende versies (A/B programma, BIO, voorkennis/ ...)
% Voor 'uitweidingen' is er een environment die online per default is ingeklapt, en in pdf al dan niet kan worden geincluded  (via \xmnouitweiding) 
%
% in een xourse, per default GEEN uitweidingen, tenzij \xmuitweiding expliciet ergens is gezet ...
\ifdefined\isXourse
   \ifdefined\xmuitweiding
   \else
       \def\xmnouitweiding{true}
   \fi
\fi

\ifdefined\xmnouitweiding
\newcounter{xmuitweiding}  % anders error undefined ...  
\excludecomment{xmuitweiding}
\else
\newtheoremstyle{dotless}{}{}{}{}{}{}{ }{}
\theoremstyle{dotless}
\newtheorem*{xmuitweidingnofrills}{}   % nofrills = no accordion; gebruikt dus de dotless theoremstyle!

\newcounter{xmuitweiding}
\newenvironment{xmuitweiding}[1][ ]%
{% 
	\refstepcounter{xmuitweiding}%
    \begin{expandable}{xmuitweiding}{\nlentext{Uitweiding \arabic{xmuitweiding}: #1}{Digression \arabic{xmuitweiding}: #1}}%
	\begin{xmuitweidingnofrills}%
}
{%
    \end{xmuitweidingnofrills}%
    \end{expandable}%
}   
% \newenvironment{xmuitweiding}[1][ ]%
% {% 
% 	\refstepcounter{xmuitweiding}
% 	\begin{accordion}\begin{accordion-item}[Uitweiding \arabic{xmuitweiding}: #1]%
% 			\begin{xmuitweidingnofrills}%
% 			}
% 			{\end{xmuitweidingnofrills}\end{accordion-item}\end{accordion}}   
\fi


\newenvironment{xmexpandable}[1][]{
	\begin{accordion}\begin{accordion-item}[#1]%
		}{\end{accordion-item}\end{accordion}}


% Command that gives a selection box online, but just prints the right answer in pdf
\newcommand{\xmonlineChoice}[1]{\pdfOnly{\wordchoicegiventrue}\wordChoice{#1}\pdfOnly{\wordchoicegivenfalse}}   % deprecated, gebruik onlineChoice ...
\newcommand{\onlineChoice}[1]{\pdfOnly{\wordchoicegiventrue}\wordChoice{#1}\pdfOnly{\wordchoicegivenfalse}}

\newcommand{\TJa}{\nlentext{ Ja }{ Yes }}
\newcommand{\TNee}{\nlentext{ Nee }{ No }}
\newcommand{\TJuist}{\nlentext{ Juist }{ True }}
\newcommand{\TFout}{\nlentext{ Fout }{ False }}

\newcommand{\choiceTrue }{{\renewcommand{\choiceminimumhorizontalsize}{4em}\wordChoice{\choice[correct]{\TJuist}\choice{\TFout}}}}
\newcommand{\choiceFalse}{{\renewcommand{\choiceminimumhorizontalsize}{4em}\wordChoice{\choice{\TJuist}\choice[correct]{\TFout}}}}

\newcommand{\choiceYes}{{\renewcommand{\choiceminimumhorizontalsize}{3em}\wordChoice{\choice[correct]{\TJa}\choice{\TNee}}}}
\newcommand{\choiceNo }{{\renewcommand{\choiceminimumhorizontalsize}{3em}\wordChoice{\choice{\TJa}\choice[correct]{\TNee}}}}

% Optional nicer formatting for wordChoice in PDF

\let\inlinechoiceorig\inlinechoice

%\makeatletter
%\providecommand{\choiceminimumverticalsize}{\vphantom{$\frac{\sqrt{2}}{2}$}}   % minimum height of boxes (cfr infra)
\providecommand{\choiceminimumverticalsize}{\vphantom{$\tfrac{2}{2}$}}   % minimum height of boxes (cfr infra)
\providecommand{\choiceminimumhorizontalsize}{1em}   % minimum width of boxes (cfr infra)

\newcommand{\inlinechoicesquares}[2][]{%
		\setkeys{choice}{#1}%
		\ifthenelse{\boolean{\choice@correct}}%
		{%
            \ifhandout%
               \fbox{\choiceminimumverticalsize #2}\allowbreak\ignorespaces%
            \else%
               \fcolorbox{blue}{blue!20}{\choiceminimumverticalsize #2}\allowbreak\ignorespaces\setkeys{choice}{correct=false}\ignorespaces%
            \fi%
		}%
		{% else
			\fbox{\choiceminimumverticalsize #2}\allowbreak\ignorespaces%  HACK: wat kleiner, zodat fits on line ... 	
		}%
}

\newcommand{\inlinechoicesquareX}[2][]{%
		\setkeys{choice}{#1}%
		\ifthenelse{\boolean{\choice@correct}}%
		{%
            \ifhandout%
               \framebox[\ifdim\choiceminimumhorizontalsize<\width\width\else\choiceminimumhorizontalsize\fi]{\choiceminimumverticalsize\ #2\ }\allowbreak\ignorespaces\setkeys{choice}{correct=false}\ignorespaces%
            \else%
               \fcolorbox{blue}{blue!20}{\makebox[\ifdim\choiceminimumhorizontalsize<\width\width\else\choiceminimumhorizontalsize\fi]{\choiceminimumverticalsize #2}}\allowbreak\ignorespaces\setkeys{choice}{correct=false}\ignorespaces%
            \fi%
		}%
		{% else
        \ifhandout%
			\framebox[\ifdim\choiceminimumhorizontalsize<\width\width\else\choiceminimumhorizontalsize\fi]{\choiceminimumverticalsize\ #2\ }\allowbreak\ignorespaces%  HACK: wat kleiner, zodat fits on line ... 	
        \fi
		}%
}


\newcommand{\inlinechoicepuntjes}[2][]{%
		\setkeys{choice}{#1}%
		\ifthenelse{\boolean{\choice@correct}}%
		{%
            \ifhandout%
               \dots\ldots\ignorespaces\setkeys{choice}{correct=false}\ignorespaces
            \else%
               \fcolorbox{blue}{blue!20}{\choiceminimumverticalsize #2}\allowbreak\ignorespaces\setkeys{choice}{correct=false}\ignorespaces%
            \fi%
		}%
		{% else
			%\fbox{\choiceminimumverticalsize #2}\allowbreak\ignorespaces%  HACK: wat kleiner, zodat fits on line ... 	
		}%
}

% print niets, maar definieer globale variable \myanswer
%  (gebruikt om oplossingsbladen te printen) 
\newcommand{\inlinechoicedefanswer}[2][]{%
		\setkeys{choice}{#1}%
		\ifthenelse{\boolean{\choice@correct}}%
		{%
               \gdef\myanswer{#2}\setkeys{choice}{correct=false}

		}%
		{% else
			%\fbox{\choiceminimumverticalsize #2}\allowbreak\ignorespaces%  HACK: wat kleiner, zodat fits on line ... 	
		}%
}



%\makeatother

\newcommand{\setchoicedefanswer}{
\ifdefined\HCode
\else
%    \renewenvironment{multipleChoice@}[1][]{}{} % remove trailing ')'
    \let\inlinechoice\inlinechoicedefanswer
\fi
}

\newcommand{\setchoicepuntjes}{
\ifdefined\HCode
\else
    \renewenvironment{multipleChoice@}[1][]{}{} % remove trailing ')'
    \let\inlinechoice\inlinechoicepuntjes
\fi
}
\newcommand{\setchoicesquares}{
\ifdefined\HCode
\else
    \renewenvironment{multipleChoice@}[1][]{}{} % remove trailing ')'
    \let\inlinechoice\inlinechoicesquares
\fi
}
%
\newcommand{\setchoicesquareX}{
\ifdefined\HCode
\else
    \renewenvironment{multipleChoice@}[1][]{}{} % remove trailing ')'
    \let\inlinechoice\inlinechoicesquareX
\fi
}
%
\newcommand{\setchoicelist}{
\ifdefined\HCode
\else
    \renewenvironment{multipleChoice@}[1][]{}{)}% re-add trailing ')'
    \let\inlinechoice\inlinechoiceorig
\fi
}

\setchoicesquareX  % by default list-of-squares with onlineChoice in PDF

% Omdat multicols niet werkt in html: enkel in pdf  (in html zijn langere pagina's misschien ook minder storend)
\newenvironment{xmmulticols}[1][2]{
 \pdfOnly{\begin{multicols}{#1}}%
}{ \pdfOnly{\end{multicols}}}

%
% Te gebruiken in plaats van \section\subsection
%  (in een printstyle kan dan het level worden aangepast
%    naargelang \chapter vs \section style )
% 3/2021: DO NOT USE \xmsubsection !
\newcommand\xmsection\subsection
\newcommand\xmsubsection\subsubsection

% Aanpassen printversie
%  (hier gedefinieerd, zodat ze in xourse kunnen worden gezet/overschreven)
\providebool{parttoc}
\providebool{printpartfrontpage}
\providebool{printactivitytitle}
\providebool{printactivityqrcode}
\providebool{printactivityurl}
\providebool{printcontinuouspagenumbers}
\providebool{numberactivitiesbysubpart}
\providebool{addtitlenumber}
\providebool{addsectiontitlenumber}
\addtitlenumbertrue
\addsectiontitlenumbertrue

% The following three commands are hardcoded in xake, you can't create other commands like these, without adding them to xake as well
%  ( gebruikt in xourses om juiste soort titelpagina te krijgen voor verschillende ximera's )
\newcommand{\activitychapter}[2][]{
    {    
    \ifstrequal{#1}{notnumbered}{
        \addtitlenumberfalse
    }{}
    \typeout{ACTIVITYCHAPTER #2}   % logging
	\chapterstyle
	\activity{#2}
    }
}
\newcommand{\activitysection}[2][]{
    {
    \ifstrequal{#1}{notnumbered}{
        \addsectiontitlenumberfalse
    }{}
	\typeout{ACTIVITYSECTION #2}   % logging
	\sectionstyle
	\activity{#2}
    }
}
% Practices worden als activity getoond om de grote blokken te krijgen online
\newcommand{\practicesection}[2][]{
    {
    \ifstrequal{#1}{notnumbered}{
        \addsectiontitlenumberfalse
    }{}
    \typeout{PRACTICESECTION #2}   % logging
	\sectionstyle
	\activity{#2}
    }
}
\newcommand{\activitychapterlink}[3][]{
    {
    \ifstrequal{#1}{notnumbered}{
        \addtitlenumberfalse
    }{}
    \typeout{ACTIVITYCHAPTERLINK #3}   % logging
	\chapterstyle
	\activitylink[#1]{#2}{#3}
    }
}

\newcommand{\activitysectionlink}[3][]{
    {
    \ifstrequal{#1}{notnumbered}{
        \addsectiontitlenumberfalse
    }{}
    \typeout{ACTIVITYSECTIONLINK #3}   % logging
	\sectionstyle
	\activitylink[#1]{#2}{#3}
    }
}


% Commando om de printstyle toe te voegen in ximera's. Zorgt ervoor dat er geen problemen zijn als je de xourses compileert
% hack om onhandige relative paden in TeX te omzeilen
% should work both in xourse and ximera (pre-112022 only in ximera; thus obsoletes adhoc setup in xourses)
% loads global.sty if present (cfr global.css for online settings!)
% use global.sty to overwrite settings in printstyle.sty ...
\newcommand{\addPrintStyle}[1]{
\iftikzexport\else   % only in PDF
  \makeatletter
  \ifx\@onlypreamble\@notprerr\else   % ONLY if in tex-preamble   (and e.g. not when included from xourse)
    \typeout{Loading printstyle}   % logging
    \usepackage{#1/printstyle} % mag enkel geinclude worden als je die apart compileert
    \IfFileExists{#1/global.sty}{
        \typeout{Loading printstyle-folder #1/global.sty}   % logging
        \usepackage{#1/global}
        }{
        \typeout{Info: No extra #1/global.sty}   % logging
    }   % load global.sty if present
    \IfFileExists{global.sty}{
        \typeout{Loading local-folder global.sty (or TEXINPUTPATH..)}   % logging
        \usepackage{global}
    }{
        \typeout{Info: No folder/global.sty}   % logging
    }   % load global.sty if present
    \IfFileExists{\currfilebase.sty}
    {
        \typeout{Loading \currfilebase.sty}
        \input{\currfilebase.sty}
    }{
        \typeout{Info: No local \currfilebase.sty}
    }
    \fi
  \makeatother
\fi
}

%
%  
% references: Ximera heeft adhoc logica	 om online labels te doen werken over verschillende files heen
% met \hyperref kan de getoonde tekst toch worden opgegeven, in plaats van af te hangen van de label-text
\ifdefined\HCode
% Link to standard \labels, but give your own description
% Usage:  Volg \hyperref[my_very_verbose_label]{deze link} voor wat tijdverlies
%   (01/2020: Ximera-server aangepast om bij class reference-keeptext de link-text NIET te vervangen door de label-text !!!) 
\renewcommand{\hyperref}[2][]{\HCode{<a class="reference reference-keeptext" href="\##1">}#2\HCode{</a>}}
%
%  Link to specific targets  (not tested ?)
\renewcommand{\hypertarget}[1]{\HCode{<a class="ximera-label" id="#1"></a>}}
\renewcommand{\hyperlink}[2]{\HCode{<a class="reference reference-keeptext" href="\##1">}#2\HCode{</a>}}
\fi

% Mmm, quid English ... (for keyword #1 !) ?
\newcommand{\wikilink}[2]{
    \hyperlink{https://nl.wikipedia.org/wiki/#1}{#2}
    \pdfOnly{\footnote{See \url{https://nl.wikipedia.org/wiki/#1}}
    }
}

\renewcommand{\figurename}{Figuur}
\renewcommand{\tablename}{Tabel}

%
% Gedoe om verschillende versies van xourse/ximera te maken afhankelijk van settings
%
% default: versie met antwoorden
% handout: versie voor de studenten, zonder antwoorden/oplossingen
% full: met alles erop en eraan, dus geschikt voor auteurs en/of lesgevers  (bevat in de pdf ook de 'online-only' stukken!)
%
%
% verder kunnen ook opties/variabele worden gezet voor hints/auteurs/uitweidingen/ etc
%
% 'Full' versie
\newtoggle{showonline}
\ifdefined\HCode   % zet default showOnline
    \toggletrue{showonline} 
\else
    \togglefalse{showonline}
\fi

% Full versie   % deprecated: see infra
\newcommand{\printFull}{
    \hintstrue
    \handoutfalse
    \toggletrue{showonline} 
}

\ifdefined\shouldPrintFull   % deprecated: see infra
    \printFull
\fi



% Overschrijf onlineOnly  (zoals gedefinieerd in ximera.cls)
\ifhandout   % in handout: gebruik de oorspronkelijke ximera.cls implementatie  (is dit wel nodig/nuttig?)
\else
    \iftoggle{showonline}{%
        \ifdefined\HCode
          \RenewEnviron{onlineOnly}{\bgroup\BODY\egroup}   % showOnline, en we zijn  online, dus toon de tekst
        \else
          \RenewEnviron{onlineOnly}{\bgroup\color{red!50!black}\BODY\egroup}   % showOnline, maar we zijn toch niet online: kleur de tekst rood 
        \fi
    }{%
      \RenewEnviron{onlineOnly}{}  % geen showOnline
    }
\fi

% hack om na hoofding van definition/proposition/... als dan niet op een nieuwe lijn te starten
% soms is dat goed en mooi, en soms niet; en in HTML is het nu (2/2020) anders dan in pdf
% vandaar suggestie om 
%     \begin{definition}[Nieuw concept] \nl
% te gebruiken als je zeker een newline wil na de hoofdig en titel
% (in het bijzonder itemize zonder \nl is 'lelijk' ...)
\ifdefined\HCode
\newcommand{\nl}{}
\else
\newcommand{\nl}{\ \par} % newline (achter heading van definition etc.)
\fi


% \nl enkel in handoutmode (ihb voor \wordChoice, die dan typisch veeeel langer wordt)
\ifdefined\HCode
\providecommand{\handoutnl}{}
\else
\providecommand{\handoutnl}{%
\ifhandout%
  \nl%
\fi%
}
\fi

% Could potentially replace \pdfOnline/\begin{onlineOnly} : 
% Usage= \ifonline{Hallo surfer}{Hallo PDFlezer}
\providecommand{\ifonline}[2]%
{
\begin{onlineOnly}#1\end{onlineOnly}%
\pdfOnly{#2}
}%


%
% Maak optionele 'basic' en 'extended' versies van een activity
%  met environment basicOnly, basicSkip en extendedOnly
%
%  (
%   Dit werkt ENKEL in de PDF; de online versies tonen (minstens voorklopig) steeds 
%   het default geval met printbasicversion en printextendversion beide FALSE
%  )
%
\providebool{printbasicversion}
\providebool{printextendedversion}   % not properly implemented
\providebool{printfullversion}       % presumably print everything (debug/auteur)
%
% only set these in xourses, and BEFORE loading this preamble
%
%\newif\ifshowbasic     \showbasictrue        % use this line in xourse to show 'basic' sections
%\newif\ifshowextended  \showextendedtrue     % use this line in xourse to show 'extended' sections
%
%
%\ifbool{showbasic}
%      { \NewEnviron{basicOnly}{\BODY} }    % if yes: just print contents
%      { \NewEnviron{basicOnly}{}      }    % if no:  completely ignore contents
%
%\ifbool{showbasic}
%      { \NewEnviron{basicSkip}{}      }
%      { \NewEnviron{basicSkip}{\BODY} }
%

\ifbool{printextendedversion}
      { \NewEnviron{extendedOnly}{\BODY} }
      { \NewEnviron{extendedOnly}{}      }
      


\ifdefined\HCode    % in html: always print
      {\newenvironment*{basicOnly}{}{}}    % if yes: just print contents
      {\newenvironment*{basicSkip}{}{}}    % if yes: just print contents
\else

\ifbool{printbasicversion}
      {\newenvironment*{basicOnly}{}{}}    % if yes: just print contents
      {\NewEnviron{basicOnly}{}      }    % if no:  completely ignore contents

\ifbool{printbasicversion}
      {\NewEnviron{basicSkip}{}      }
      {\newenvironment*{basicSkip}{}{}}

\fi

\usepackage{float}
\usepackage[rightbars,color]{changebar}

% Full versie
\ifbool{printfullversion}{
    \hintstrue
    \handoutfalse
    \toggletrue{showonline}
    \printbasicversionfalse
    \cbcolor{red}
    \renewenvironment*{basicOnly}{\cbstart}{\cbend}
    \renewenvironment*{basicSkip}{\cbstart}{\cbend}
    \def\xmtoonprintopties{FULL}   % will be printed in footer
}
{}
      
%
% Evalueer \ifhints IN de environment
%  
%
%\RenewEnviron{hint}
%{
%\ifhandout
%\ifhints\else\setbox0\vbox\fi%everything in een emty box
%\bgroup 
%\stepcounter{hintLevel}
%\BODY
%\egroup\ignorespacesafterend
%\addtocounter{hintLevel}{-1}
%\else
%\ifhints
%\begin{trivlist}\item[\hskip \labelsep\small\slshape\bfseries Hint:\hspace{2ex}]
%\small\slshape
%\stepcounter{hintLevel}
%\BODY
%\end{trivlist}
%\addtocounter{hintLevel}{-1}
%\fi
%\fi
%}

% Onafhankelijk van \ifhandout ...? TO BE VERIFIED
\RenewEnviron{hint}
{
\ifhints
\begin{trivlist}\item[\hskip \labelsep\small\bfseries Hint:\hspace{2ex}]
\small%\slshape
\stepcounter{hintLevel}
\BODY
\end{trivlist}
\addtocounter{hintLevel}{-1}
\else
\iftikzexport   % anders worden de tikz tekeningen in hints niet gegenereerd ?
\setbox0\vbox\bgroup
\stepcounter{hintLevel}
\BODY
\egroup\ignorespacesafterend
\addtocounter{hintLevel}{-1}
\fi % ifhandout
\fi %ifhints
}

%
% \tab sets typewriter-tabs (e.g. to format questions)
% (Has no effect in HTML :-( ))
%
\usepackage{tabto}
\ifdefined\HCode
  \renewcommand{\tab}{\quad}    % otherwise dummy .png's are generated ...?
\fi


% Also redefined in  preamble to get correct styling 
% for tikz images for (\tikzexport)
%

\theoremstyle{definition} % Bold titels
\makeatletter
\let\proposition\relax
\let\c@proposition\relax
\let\endproposition\relax
\makeatother
\newtheorem{proposition}{Eigenschap}


%\instructornotesfalse

% logic with \ifhandoutin ximera.cls unclear;so overwrite ...
\makeatletter
\@ifundefined{ifinstructornotes}{%
  \newif\ifinstructornotes
  \instructornotesfalse
  \newenvironment{instructorNotes}{}{}
}{}
\makeatother
\ifinstructornotes
\else
\renewenvironment{instructorNotes}%
{%
    \setbox0\vbox\bgroup
}
{%
    \egroup
}
\fi

% \RedeclareMathOperator
% from https://tex.stackexchange.com/questions/175251/how-to-redefine-a-command-using-declaremathoperator
\makeatletter
\newcommand\RedeclareMathOperator{%
    \@ifstar{\def\rmo@s{m}\rmo@redeclare}{\def\rmo@s{o}\rmo@redeclare}%
}
% this is taken from \renew@command
\newcommand\rmo@redeclare[2]{%
    \begingroup \escapechar\m@ne\xdef\@gtempa{{\string#1}}\endgroup
    \expandafter\@ifundefined\@gtempa
    {\@latex@error{\noexpand#1undefined}\@ehc}%
    \relax
    \expandafter\rmo@declmathop\rmo@s{#1}{#2}}
% This is just \@declmathop without \@ifdefinable
\newcommand\rmo@declmathop[3]{%
    \DeclareRobustCommand{#2}{\qopname\newmcodes@#1{#3}}%
}
\@onlypreamble\RedeclareMathOperator
\makeatother


%
% Engelse vertaling, vooral in mathmode
%
% 1. Algemene procedure
%
\ifdefined\isEn
 \newcommand{\nlen}[2]{#2}
 \newcommand{\nlentext}[2]{\text{#2}}
 \newcommand{\nlentextbf}[2]{\textbf{#2}}
\else
 \newcommand{\nlen}[2]{#1}
 \newcommand{\nlentext}[2]{\text{#1}}
 \newcommand{\nlentextbf}[2]{\textbf{#1}}
\fi

%
% 2. Lijst van erg veel gebruikte uitdrukkingen
%

% Ja/Nee/Fout/Juits etc
%\newcommand{\TJa}{\nlentext{ Ja }{ and }}
%\newcommand{\TNee}{\nlentext{ Nee }{ No }}
%\newcommand{\TJuist}{\nlentext{ Juist }{ Correct }
%\newcommand{\TFout}{\nlentext{ Fout }{ Wrong }
\newcommand{\TWaar}{\nlentext{ Waar }{ True }}
\newcommand{\TOnwaar}{\nlentext{ Vals }{ False }}
% Korte bindwoorden en, of, dus, ...
\newcommand{\Ten}{\nlentext{ en }{ and }}
\newcommand{\Tof}{\nlentext{ of }{ or }}
\newcommand{\Tdus}{\nlentext{ dus }{ so }}
\newcommand{\Tendus}{\nlentext{ en dus }{ and thus }}
\newcommand{\Tvooralle}{\nlentext{ voor alle }{ for all }}
\newcommand{\Took}{\nlentext{ ook }{ also }}
\newcommand{\Tals}{\nlentext{ als }{ when }} %of if?
\newcommand{\Twant}{\nlentext{ want }{ as }}
\newcommand{\Tmaal}{\nlentext{ maal }{ times }}
\newcommand{\Toptellen}{\nlentext{ optellen }{ add }}
\newcommand{\Tde}{\nlentext{ de }{ the }}
\newcommand{\Thet}{\nlentext{ het }{ the }}
\newcommand{\Tis}{\nlentext{ is }{ is }} %zodat is in text staat in mathmode (geen italics)
\newcommand{\Tmet}{\nlentext{ met }{ where }} % in situaties e.g met p < n --> where p < n
\newcommand{\Tnooit}{\nlentext{ nooit }{ never }}
\newcommand{\Tmaar}{\nlentext{ maar }{ but }}
\newcommand{\Tniet}{\nlentext{ niet }{ not }}
\newcommand{\Tuit}{\nlentext{ uit }{ from }}
\newcommand{\Ttov}{\nlentext{ t.o.v. }{ w.r.t. }}
\newcommand{\Tzodat}{\nlentext{ zodat }{ such that }}
\newcommand{\Tdeth}{\nlentext{de }{th }}
\newcommand{\Tomdat}{\nlentext{omdat }{because }} 


%
% Overschrijf addhoc commando's
%
\ifdefined\isEn
\renewcommand{\pernot}{\overset{\mathrm{notation}}{=}}
\RedeclareMathOperator{\bld}{im}     % beeld
\RedeclareMathOperator{\graf}{graph}   % grafiek
\RedeclareMathOperator{\rico}{slope}   % richtingcoëfficient
\RedeclareMathOperator{\co}{co}       % coordinaat
\RedeclareMathOperator{\gr}{deg}       % graad

% Operators
\RedeclareMathOperator{\bgsin}{arcsin}
\RedeclareMathOperator{\bgcos}{arccos}
\RedeclareMathOperator{\bgtan}{arctan}
\RedeclareMathOperator{\bgcot}{arccot}
\RedeclareMathOperator{\bgsinh}{arcsinh}
\RedeclareMathOperator{\bgcosh}{arccosh}
\RedeclareMathOperator{\bgtanh}{arctanh}
\RedeclareMathOperator{\bgcoth}{arccoth}

\fi


% HACK: use 'oplossing' for 'explanation' ...
\let\explanation\relax
\let\endexplanation\relax
% \newenvironment{explanation}{\begin{oplossing}}{\end{oplossing}}
\newcounter{explanation}

\ifhandout%
    \NewEnviron{explanation}[1][toon]%
    {%
    \RenewEnviron{verbatim}{ \red{VERBATIM CONTENT MISSING IN THIS PDF}} %% \expandafter\verb|\BODY|}

    \ifthenelse{\equal{\detokenize{#1}}{\detokenize{toon}}}
    {
    \def\PH@Command{#1}% Use PH@Command to hold the content and be a target for "\expandafter" to expand once.

    \begin{trivlist}% Begin the trivlist to use formating of the "Feedback" label.
    \item[\hskip \labelsep\small\slshape\bfseries Explanation:% Format the "Feedback" label. Don't forget the space.
    %(\texttt{\detokenize\expandafter{\PH@Command}}):% Format (and detokenize) the condition for feedback to trigger
    \hspace{2ex}]\small%\slshape% Insert some space before the actual feedback given.
    \BODY
    \end{trivlist}
    }
    {  % \begin{feedback}[solution]   \BODY     \end{feedback}  }
    }
    }    
\else
% ONLY for HTML; xmoplossing is styled with css, and is not, and need not be a LaTeX environment
% THUS: it does NOT use feedback anymore ...
%    \NewEnviron{oplossing}{\begin{expandable}{xmoplossing}{\nlen{Toon uitwerking}{Show solution}}{\BODY}\end{expandable}}
    \newenvironment{explanation}[1][toon]
   {%
       \begin{expandable}{xmoplossing}{}
   }
   {%
   	   \end{expandable}
   } 
\fi


\title{Gaussian Elimination and Rank} \license{CC BY-NC-SA 4.0}
\begin{document}

\begin{abstract}
 
\end{abstract}
\maketitle

\section*{Gaussian Elimination and Rank}
\subsection*{Row Echelon and Reduced Row Echelon Forms}

In \href{https://ximera.osu.edu/oerlinalg/LinearAlgebra/SYS-0020/main}{Augmented Matrix Notation and Elementary Row Operations}, we learned to write linear systems in \dfn{augmented matrix} form and use elementary row operations to transform an augmented matrix to \dfn{row-echelon form} and the \dfn{reduced row-echelon form} in order to solve linear systems.  

Recall that a matrix (or augmented matrix) is in \dfn{row-echelon form} if:
\begin{itemize}
\item All entries {\it below} each leading entry are $0$.
\item Each leading entry is in a column to the right of the leading entries in the rows above it.
\item All rows of zeros, if there are any, are located below non-zero rows.
\end{itemize}

A matrix in row-echelon form is said to be in \dfn{reduced row-echelon form} if it has the following additional properties
\begin{itemize}
\item Each leading entry is a $1$
\item All entries {\it above} each leading $1$ are $0$
\end{itemize}


Matrices in row-echelon form and reduced row-echelon form are ``convenient" because their corresponding systems of equations are easy to solve.  

In this module we turn to the question of existence and uniqueness of the two echelon forms.  Can every matrix (augmented matrix) be carried to a row-echelon form?  If so, is the row-echelon form unique?  Can every matrix be reduced to a reduced row-echelon form?  If so, is the reduced row-echelon form unique?  The answers we find will have long-lasting implications.


\begin{exploration}\label{init:gaussianelim1}
Solve the following system of equations.
$$\begin{array}{ccccccccc}
      x &+ &2y&-&3z&= &1 \\
	 -5x& +&2y&-&3z&=&1\\
      x&- &2y&+&z&=&1
    \end{array}$$

We create an augmented matrix corresponding to the system and apply row operations until the matrix is in row-echelon form.  
$$\left[\begin{array}{ccc|c}  
 1&2&-3&1\\-5&2&-3&1\\1&-2&1&1
 \end{array}\right]
 \begin{array}{c}
 \\
 \xrightarrow{R_2+5R_1}\\
 \xrightarrow{R_3-R_1}\\
 \end{array}
\left[\begin{array}{ccc|c}  
 1&2&-3&1\\0&12&-18&6\\0&-4&4&0
 \end{array}\right]
 \begin{array}{c}
 \\
 \\
 \xrightarrow{R_3+\frac{1}{3}R_2}\\
 \end{array}$$
 \begin{equation}\label{eq:ref1}
 \left[\begin{array}{ccc|c}  
 1&2&-3&1\\0&12&-18&6\\0&0&-2&2
 \end{array}\right]
\end{equation}

Note that the elementary row operations that lead to (\ref{eq:ref1}) were not prescribed.  We may employ row-operations in a different manner and obtain a different matrix in row-echelon form.  For example, suppose for some reason we had begun by switching the first and third rows.

$$\left[\begin{array}{ccc|c}  
 1&2&-3&1\\-5&2&-3&1\\1&-2&1&1
 \end{array}\right]
 \begin{array}{c}
 \\
 \xrightarrow{R_1\leftrightarrow R_3}\\
\\
 \end{array}
\left[\begin{array}{ccc|c}  
 1&-2&1&1\\-5&2&-3&1\\1&2&-3&1
 \end{array}\right]
 \begin{array}{c}
 \\
\\
\\
 \end{array}$$

Next we would reduce this matrix to row-echelon form, perhaps in this way:

$$\left[\begin{array}{ccc|c}  
 1&-2&1&1\\-5&2&-3&1\\1&2&-3&1
 \end{array}\right]
 \begin{array}{c}
 \\
 \xrightarrow{R_2+5R_1}\\
 \xrightarrow{R_3-R_1}\\
 \end{array}
\left[\begin{array}{ccc|c}  
 1&-2&1&1\\0&-8&2&6\\0&4&-4&0
 \end{array}\right]
 \begin{array}{c}
 \\
 \\
 \xrightarrow{R_3+\frac{1}{2}R_2}\\
 \end{array}
$$
\begin{equation}\label{eq:ref2}
 \left[\begin{array}{ccc|c}  
 1&-2&1&1\\0&-8&2&6\\0&0&-3&3
 \end{array}\right]
\end{equation}

The augmented matrices in (\ref{eq:ref1}) and (\ref{eq:ref2}) are clearly not the same, but both are in row-echelon form.

If we write the systems of equations corresponding to (\ref{eq:ref1}) and (\ref{eq:ref2}), we can employ back substitution to solve them.  The matrix in (\ref{eq:ref1}) corresponds to
$$\begin{array}{ccccccccc}
      x &+ &2y&-&3z&= &1 \\
	 & &12y&-&18z&=&6\\
      &&&&-2z&=&2
    \end{array}$$
    
The matrix in (\ref{eq:ref2}) corresponds to    
$$\begin{array}{ccccccccc}
      x &- &2y&+&z&= &1 \\
	 & &-8y&+&2z&=&6\\
      &&&&-3z&=&3
    \end{array}$$
    
Because both systems are equivalent to the original system, it is not surprising that back substitution yields the same solution for both systems. 

$$x=\answer{0},\quad y=\answer{-1},\quad z=\answer{-1}$$

%From the last equation we get $z=-1$.  
%Next, from the second equation we have $12y-18(-1)=6$, which yields $y=-1$.
%Finally, the first equation gives us $x+2(-1)-3(-1)=1$.  Solving for $x$ we get $x=0$.  
%Therefore, the solution to the system is $$x=0, y=-1, z=-1$$
%We arrive at a different matrix in row-echelon form, but back-substitution yields the same solution to the system (you should check this):  $$x=0, y=-1, z=-1$$
\end{exploration}

It is clear from Exploration \ref{init:gaussianelim1} that a row-echelon form corresponding to a matrix is not unique.  But what about the reduced row-echelon form?

\begin{exploration}\label{init:gaussianelim2} In this problem we revisit the system
$$\begin{array}{ccccccccc}
      x &+ &2y&-&3z&= &1 \\
	 -5x& +&2y&-&3z&=&1\\
      x&- &2y&+&z&=&1
    \end{array}$$
    
Following the steps we took to get (\ref{eq:ref1}), but taking the process a little further, we get the reduced row-echelon form.
$$\left[\begin{array}{ccc|c}  
 1&2&-3&1\\-5&2&-3&1\\1&-2&1&1
 \end{array}\right]
 \begin{array}{c}
 \\
 \xrightarrow{R_2+5R_1}\\
 \xrightarrow{R_3-R_1}\\
 \end{array}
\left[\begin{array}{ccc|c}  
 1&2&-3&1\\0&12&-18&6\\0&-4&4&0
 \end{array}\right]
 \begin{array}{c}
 \\
 \\
 \xrightarrow{R_3+\frac{1}{3}R_2}\\
 \end{array}$$
 $$\left[\begin{array}{ccc|c}  
 1&2&-3&1\\0&12&-18&6\\0&0&-2&2
 \end{array}\right]
  \color{blue}
 \begin{array}{c}
 \\
  \xrightarrow{\frac{1}{12}R_2}\\
 \xrightarrow{-\frac{1}{2}R_3}\\
 \end{array}
 \left[\begin{array}{ccc|c}  
 1&2&-3&1\\0&1&-\frac{3}{2}&\frac{1}{2}\\0&0&1&-1
 \end{array}\right]
  \begin{array}{c}
  \xrightarrow{R_1+3R_3}\\
 \xrightarrow{R_2+\frac{3}{2}R_3}\\
\\
 \end{array}$$
\begin{equation}\label{eq:rref3}  \color{blue}
 \left[\begin{array}{ccc|c}  
 1&2&0&-2\\0&1&0&-1\\0&0&1&-1
 \end{array}\right]
  \begin{array}{c}
  \xrightarrow{R_1-2R_2}\\
 \\
\\
 \end{array}
 \color{black}
  \left[\begin{array}{ccc|c}  
 1&0&0&0\\0&1&0&-1\\0&0&1&-1
 \end{array}\right]\end{equation}
Do you think it is possible to start with (\ref{eq:ref2}) and obtain the same reduced row-echelon form?  Try to justify your response.  If possible, find the elementary row operations that take (\ref{eq:ref2}) to the reduced row-echelon form in (\ref{eq:rref3}).

$$\left[\begin{array}{ccc|c}  
 1&2&-3&1\\-5&2&-3&1\\1&-2&1&1
 \end{array}\right]
 \begin{array}{c}
 \\
 \xrightarrow{R_1\leftrightarrow R_3}\\
\\
 \end{array}
\left[\begin{array}{ccc|c}  
 1&-2&1&1\\-5&2&-3&1\\1&2&-3&1
 \end{array}\right]$$
$$\begin{array}{c}
 \\
 \xrightarrow{R_2+5R_1}\\
 \xrightarrow{R_3-R_1}\\
 \end{array}
\left[\begin{array}{ccc|c}  
 1&-2&1&1\\0&-8&2&6\\0&4&-4&0
 \end{array}\right]
 \begin{array}{c}
 \\
 \\
 \xrightarrow{R_3+\frac{1}{2}R_2}\\
 \end{array}
$$
\begin{equation*}
  \left[\begin{array}{ccc|c}  
 1&-2&1&1\\0&-8&2&6\\0&0&-3&3
 \end{array}\right]
 \color{red}
 \begin{array}{c}
\rightsquigarrow\text{Elementary}\rightsquigarrow\\
\text{Row Ops.}
\end{array}
 \color{black}
  \left[\begin{array}{ccc|c}  
 1&0&0&0\\0&1&0&-1\\0&0&1&-1
 \end{array}\right]
\end{equation*}
You will be asked to fill in the elementary row operations in Practice Problem \ref{prob:same_rref}.
\end{exploration}

Our observations in Exploration \ref{init:gaussianelim2} are summarized in the following diagram.

\begin{center}
\begin{tikzpicture}
  \node[] at (0, 0)  (o)    {$\left[\begin{array}{ccc|c}  
 1&2&-3&1\\-5&2&-3&1\\1&-2&1&1
 \end{array}\right]$};
 \node[] at (-3, -2.5)  (left1)    {$\left[\begin{array}{ccc|c}  
 1&2&-3&1\\0&12&-18&6\\0&0&-2&2
 \end{array}\right]$};
 \node[] at (3, -2.5)  (right1)    {$\left[\begin{array}{ccc|c}  
 1&-2&1&1\\0&-8&2&6\\0&0&-3&3
 \end{array}\right]$};
 \node[gray] at (0, -2.5)  (c1)    {\shortstack{Row-\\Echelon\\Forms}};
 \node[] at (0, -5)  (c2)    {$\left[\begin{array}{ccc|c}  1&0&0&0\\0&1&0&-1\\0&0&1&-1
 \end{array}\right]$};
 \node[gray] at (0, -6)  (c3)    {Reduced Row-Echelon Form};
 
 \draw [->,line width=0.5pt,-stealth]  (o.west)to[out=180, in=90](left1.north);
  \draw [->,line width=0.5pt,-stealth]  (o.south)to[out=270, in=90](c1.north west);
  \draw [->,line width=0.5pt,-stealth]  (o.south)to[out=270, in=90](c1.north east);
  \draw [->,line width=0.5pt,-stealth]  (o.south)to[out=270, in=90](c1.north);
 \draw [->,line width=0.5pt,-stealth]  (o.east)to[out=0, in=90](right1.north);
 \draw [->,line width=0.5pt,-stealth,blue]  (left1.south)to[out=270, in=180](c2.west);
 \draw [->,line width=0.5pt,-stealth,red]  (right1.south)to[out=270, in=0](c2.east);
 \end{tikzpicture}
 \end{center} 
We observed that a row-echelon form associated with a matrix is not unique.  In contrast, we also saw how different sequences of elementary row operations lead to the same solution set and the same reduced row-echelon form.  It turns out that the reduced row-echelon form of a matrix is unique.  

\begin{theorem}\label{th:uniquenessofrref} The reduced row-echelon form of a matrix is unique.
\end{theorem}
A proof of this result can be found in [Yuster]. 

The reduced row-echelon form of a matrix is an instance of a row-echelon form of the matrix.  While a given matrix may have multiple row-echelon forms, all row-echelon forms will share one characteristic: the number of nonzero rows in a row-echelon form of the given matrix will be the same.
We will prove this result in Theorem \ref{th:samenumberofnonzerorows}.

\section*{Gaussian and Gauss-Jordan Elimination}

\begin{definition}[Gaussian Elimination]\label{def:GaussianElimination}
The process of using the elementary row operations on a matrix to transform it into row-echelon form is called \dfn{Gaussian Elimination}.
\end{definition}

As we saw in the previous section, it is possible to follow different sequences of row operations to arrive at various row-echelon forms.  However, it was not clear whether it is {\it always possible} to find a row-echelon form.  The following algorithm takes any matrix (or augmented matrix) and transforms it into row-echelon form:
\begin{algorithm}[Gaussian Algorithm] \label{alg:gaussian} 
Let $A$ be an $m\times n$ matrix.

Set $i=1$ initially.
\begin{itemize}
\item[] Step 1. If $A$ consists entirely of zeros, stop;  $A$ is already in row-echelon form.

\item[] Step 2. Otherwise, find the first column from the left containing a nonzero entry in row $i$ or below row $i$.  This column will be called a \dfn{pivot column}.  Scan the pivot column from top to bottom, starting with row $i$.  Pick the topmost nonzero entry and call it $a$.  Switch rows, if necessary, to move the row containing $a$ to row $i$.  Now $a$ is the \dfn{leading entry} in its row.  We will also refer to $a$ as a \dfn{pivot}.  

\item[] Step 3. By subtracting multiples of the row containing $a$ from rows below it, make each entry below $a$ zero.

\item[] Step 4.  Set $i=i+1$.  If $i>m$ then stop; $A$ is in row-echelon form.

\end{itemize}

Repeat steps 1--4 on the matrix consisting of the remaining rows.
When the process stops, $A$ will be in row echelon form.
\end{algorithm}
Gaussian Algorithm guarantees that every matrix will have a row-echelon form.  

\begin{example}\label{ex:non-augmented}

Use the Gaussian Algorithm to find a row-echelon form of $A$ if $$A=\begin{bmatrix}2&4\\1&2\\-1&1\\3&5\end{bmatrix}$$ 
\begin{explanation}
Following Step 2, we choose the first entry, $2$, as our pivot.  We then perform step 3, using the top row to get zeros in all entries below the $2$.
$$\begin{bmatrix} \fbox{$2$}&4\\1&2\\-1&1\\3&5\end{bmatrix}
  \begin{array}{c}
  \\
  \xrightarrow{R_2-(1/2)R_1}\\
  \xrightarrow{R_3+(1/2)R_1}\\
 \xrightarrow{R_4-(3/2)R_1}\\
 \end{array}
\begin{bmatrix}2&4\\0&0\\0&3\\0&-1\end{bmatrix}
$$
The first row is now complete, and we repeat the process on the rows below it. We identify $3$ as a pivot entry in the second column and move the row containing $3$ to be directly below the first completed row.  We then use the $3$ to make each entry below the $3$ a zero.  

$$\begin{bmatrix}2&4\\0&0\\0&\fbox{$3$}\\0&-1\end{bmatrix}
\xrightarrow{R_2\leftrightarrow R_3}\\
\begin{bmatrix}2&4\\0&\fbox{$3$}\\0&0\\0&-1\end{bmatrix}
  \begin{array}{c}
 \\
\\
  \\
 \xrightarrow{R_4+\frac{1}{3}R_2}\\
 \end{array}
 \begin{bmatrix}2&4\\0&3\\0&0\\0&0\end{bmatrix}
$$
This time the algorithm terminates since row 3 and row 4 are zero rows. 
\end{explanation}
\end{example}
 


\begin{definition}[Gauss-Jordan Elimination]\label{def:GaussJordanElimination}
The process of using the elementary row operations on a matrix to transform it into reduced row-echelon form is called \dfn{Gauss-Jordan elimination}.
\end{definition}

Given a matrix in row-echelon form, it is easy to bring it the reduced row-echelon form.  For example, continuing with Example \ref{ex:non-augmented}, we can start where we left off and compute $\mbox{rref}(A)$.  From our earlier computations we have:

$$\begin{bmatrix}2&4\\-1&1\\3&5\\1&2\end{bmatrix}\rightsquigarrow\begin{bmatrix}2&4\\0&3\\0&0\\0&0\end{bmatrix}$$

Now we create leading $1's$ and use them to to wipe out all non-zero entries above them.
$$\begin{bmatrix}2&4\\0&3\\0&0\\0&0\end{bmatrix}
  \begin{array}{c}
    \xrightarrow{(1/2)R_1}\\
  \xrightarrow{(1/3)R_2}\\
  \\ 
  \\
 \end{array}
\begin{bmatrix}1&2\\0&1\\0&0\\0&0\end{bmatrix}
  \begin{array}{c}
  \xrightarrow{R_1-2R_2}\\
\\
\\
 \\
 \end{array}
\begin{bmatrix}1&0\\0&1\\0&0\\0&0\end{bmatrix}=\mbox{rref}(A)$$

The following modification to the Gaussian Algorithm produces the reduced row-echelon form of a matrix.  This algorithm guarantees the existence of the reduced row-echelon form.

\begin{algorithm}[Gauss-Jordan Algorithm] \label{alg:gauss-jordan} 
Let $A$ be an $m\times n$ matrix.
Follow the steps of the Gaussian Algorithm but modify Step 2 to create leading $1's$ by multiplying the row containing $a$ by $\frac{1}{a}$.
%Set $i=1$ initially.
%\begin{itemize}
%\item[] Step 1. If $A$ consists entirely of zeros, stop.  $A$ is already in row-echelon form.

%\item[] Step 2*. Otherwise, find the first column from the left containing a nonzero entry in row $i$ or below row $i$.  This column will be called a \dfn{pivot column}.  Scan the pivot column from top to bottom, starting with row $i$.  Pick the topmost nonzero entry and call it $a$.  Switch rows, if necessary, to move the row containing $a$ to row $i$.  Now $a$ is the \dfn{leading entry} in its row.  We will also refer to $a$ as a \dfn{pivot}.  Multiply the row containing $a$ by $\frac{1}{a}$ to create a leading $1$.  

%\item[] Step 3*. By subtracting multiples of the row containing the leading $1$ from rows {\it above} and below it, make each entry above and below the leading $1$ zero.

%\item[] Step 4.  Set $i=i+1$.  If $i>m$ then stop, and $A$ will be in row-echelon form.

%\end{itemize}

%Repeat steps 1--4 on the matrix consisting of the remaining rows.
%When the process stops, $A$ will be in reduced row echelon form.
When the Gaussian Algorithm terminates, subtract multiples of the rows containing leading $1's$ from the rows above to make all entries above the pivots zero.
\end{algorithm}





\begin{example}\label{ex:gaussjordanalg}
Use the Gauss-Jordan Algorithm to solve the system
$$\begin{array}{ccccccccc}
      3x &+ &y&+&7z&= &7 \\
	 5x& +&3y&+&9z&=&13\\
      2x&+ &y&+&4z&=&5
    \end{array}$$
\begin{explanation}
\begin{align*}&\left[\begin{array}{ccc|c}  
 \fbox{$3$}&1&7&7\\5&3&9&13\\2&1&4&5
 \end{array}\right]\\
 \begin{array}{c}
  \xrightarrow{\frac{1}{3}R_1}\\
\\
\\
 \end{array}
 &\left[\begin{array}{ccc|c}  
 \fbox{$1$}&1/3&7/3&7/3\\5&3&9&13\\2&1&4&5
 \end{array}\right]\\
 \begin{array}{c}
 \\
 \xrightarrow{R_2-5R_1}\\
\\
\end{array}
&\left[\begin{array}{ccc|c}  
 \fbox{$1$}&1/3&7/3&7/3\\0&4/3&-8/3&4/3\\2&1&4&5
 \end{array}\right]\\
 \begin{array}{c}
  \\
\\
 \xrightarrow{R_3-2R_1}\\
\end{array}&\left[\begin{array}{ccc|c}  
 1&1/3&7/3&7/3\\0&\fbox{$4/3$}&-8/3&4/3\\0&1/3&-2/3&1/3
 \end{array}\right]\\
 \begin{array}{c}
\\
 \xrightarrow{\frac{3}{4}R_2}\\
\\
\end{array}
&\left[\begin{array}{ccc|c}  
 1&1/3&7/3&7/3\\0&\fbox{$1$}&-2&1\\0&1/3&-2/3&1/3
 \end{array}\right]\\
 \begin{array}{c}
\\
\\
 \xrightarrow{R_3-\frac{1}{3}R_2}\\
\end{array}
&\left[\begin{array}{ccc|c}  
 1&1/3&7/3&7/3\\0&\fbox{$1$}&-2&1\\0&0&0&0
 \end{array}\right]\\
 \begin{array}{c}
 \xrightarrow{R_1-\frac{1}{3}R_2}\\
 \\
\\
\end{array}
&\left[\begin{array}{ccc|c}  
 1&0&3&2\\0&1&-2&1\\0&0&0&0
 \end{array}\right]
 \end{align*}
 
 We convert the reduced row-echelon form to a system of equations and find the solution.  The last equation contributes nothing to the system so we omit writing it down.
 
 $$\begin{array}{ccccccccc}
      x & &&+&3z&= &2 \\
	 & &y&-&2z&=&1
    \end{array}$$
    The solution is
    $$x=2-3t,\quad y=1+2t,\quad z=t$$
\end{explanation}
\end{example}
The Gauss-Jordan Algorithm guarantees the existence of the reduced row-echelon form for all matrices.  
When doing computations by hand, however, the algorithm may not always be the optimal method of finding a row-echelon form or the reduced row-echelon form because the procedure often leads to fractions early in the process. 
The following video shows how to arrive at the same reduced row-echelon form for the matrix in Example \ref{ex:gaussjordanalg} without doing any fraction arithmetic.  You will see that we still employ the row operations, but in a different order.

\youtube{76Y41ncuLeQ}

The video, again, highlights the fact that regardless of what sequence of elementary row operations we take to arrive at the reduced row-echelon form, the end result is the same.  



\subsection*{Rank}

As stated in Theorem \ref{th:uniquenessofrref}, the reduced row-echelon form of a matrix $A$ is uniquely determined by $A$. That is, no matter which series of row operations is used to transform $A$ to its reduced row-echelon matrix, the result will always be the same matrix. By contrast, this is not true for row-echelon matrices: Different sequences of row operations can transform the same matrix $A$ to different row-echelon matrices. However, it is true that the number $r$ of nonzero rows must be the same in each of these row-echelon matrices, as we will see in Theorem \ref{th:dimofrowA}. Hence, the number $r$ depends only on $A$ and not on the way in which $A$ is carried to row-echelon form.  

\begin{example}\label{ex:rowechofA}
Matrices (\ref{eq:ref1}) and (\ref{eq:ref2}) of Exploration \ref{init:gaussianelim1} are both row-echelon forms of $A$.  Both matrices have three nonzero rows.  The same is true for $\mbox{rref}(A)$.  
\end{example}

\begin{definition}\label{def:rankofamatrix}
The \dfn{rank} of matrix $A$, denoted by $\mbox{rank}(A)$, is the number of nonzero rows that remain after we reduce $A$ to row-echelon form by elementary row operations.
\end{definition}

\begin{example}\label{ex:rankofA1}
Compute the rank of 
$$A =  
\begin{bmatrix}
	1 & 1 & -1 & 4 \\
	2 & 1 &  3 & 0 \\
	0 & 1 & -5 & 8
\end{bmatrix}$$

\begin{explanation}
A reduction of $A$ to row-echelon form is
$$
A =  
\begin{bmatrix}
1 & 1 & -1 & 4 \\
2 & 1 &  3 & 0 \\
0 & 1 & -5 & 8
\end{bmatrix} \rightsquigarrow\begin{bmatrix}
1 & 1 & -1 & 4 \\
0 & -1 &  5 & -8 \\
0 &  0 & 0 & 0
\end{bmatrix} 
$$
Because the row-echelon form has two nonzero rows, $\mbox{rank}(A) = 2$.
\end{explanation}
\end{example}

\begin{theorem}\label{th:rankandsolutions}
Suppose a system of $m$ equations in $n$ variables is consistent, and that the rank of the {\it coefficient} matrix is $r$.

\begin{enumerate}
\item The set of solutions involves exactly $n - r$ parameters, corresponding to $n-r$ free variables.

\item If $r < n$, the system has infinitely many solutions.

\item If $r = n$, the system has a unique solution.

\end{enumerate}
\end{theorem}

\begin{proof}
The fact that the rank of the coefficient matrix is $r$ means that there are exactly $r$ leading variables in the coefficient matrix, and hence exactly $n - r$ nonleading variables. The nonleading variables are called free variables.  All free variables are assigned parameters, so the set of solutions involves exactly $n - r$ parameters. Hence if $r < n$, there is at least one parameter, and so infinitely many solutions. If $r = n$, there are no parameters and the resulting solution is unique.
\end{proof}

Theorem \ref{th:rankandsolutions} shows that, for any system of linear equations, exactly three possibilities exist:

\begin{enumerate}

\item Unique solution. This occurs when every variable is a leading variable.

\item Infinitely many solutions. This occurs when the system is consistent and there is at least one nonleading variable, so at least one parameter is involved.

\item No solution.  This occurs when a row $\left[\begin{array}{cccc|c}  0&0&\ldots &0&1
 \end{array}\right]$ appears in the row-echelon form. Such a row corresponds to an equation with no solutions. (See Example \ref{ex:nosolutionssys}.)

\end{enumerate}


 
 


\section*{Practice Problems}

\begin{problem}\label{prob:same_rref}
Show that applying Gauss-Jordan elimination to the matrix in Exploration \ref{init:gaussianelim2} yields the same reduced row-echelon form as the matrix we obtained in Exploration \ref{init:gaussianelim1}.
\end{problem}


\begin{problem}\label{prob:twowaystorref1}
Follow the indicated steps of the Gauss-Jordan algorithm to transform the matrix to its reduced row-echelon form.  Steps will unfold automatically as you enter correct answers.

$$\left[\begin{array}{ccc|c}  2&1&1&3\\-1&0&1&2\\1&1&-2&0
 \end{array}\right]$$

\begin{prompt}  
  $$\begin{array}{c}
  \xrightarrow{\frac{1}{2}R_1}\\
\\
\\
 \end{array}
 \left[\begin{array}{ccc|c}  
 \answer{1}&\answer{1/2}&\answer{1/2}&\answer{3/2}\\-1&0&1&2\\1&1&-2&0
 \end{array}\right]$$
\end{prompt} 

\begin{problem} 
\begin{prompt}
$$\begin{array}{c}
\\
  \xrightarrow{R_2+R_1}\\
\\
 \end{array}
\left[\begin{array}{ccc|c}  1&1/2&1/2&3/2\\\answer{0}&\answer{1/2}&\answer{3/2}&\answer{7/2}\\1&1&-2&0
 \end{array}\right]$$
\end{prompt}
% \begin{problem}
% \begin{prompt}
% \[\begin{array}{c}
%   \\
% \\
%  \xrightarrow{R_3-R_1}
% \end{array}
% \left[\begin{array}{ccc|c}  
%  1&1/2&1/2&3/2\\0&1/2&3/2&7/2\\\answer{0}&\answer{1/2}&\answer{-5/2}&\answer{-3/2}
%  \end{array}\right]\]
%  \end{prompt}
%  \begin{problem}
%  \begin{prompt}
% \[\begin{array}{c}
% \\
%  \xrightarrow{2R_2}\\
% \\
% \end{array}
% \left[\begin{array}{ccc|c}  
% 1&1/2&1/2&3/2\\\answer{0}&\answer{1}&\answer{3}&\answer{7}\\0&1/2&-5/2&-3/2
%  \end{array}\right]\]
% \end{prompt}
% \begin{problem}
% \begin{prompt}
% \[\begin{array}{c}
% \\
% \\
%  \xrightarrow{R_3-\frac{1}{2}R_2}
% \end{array}
% \left[\begin{array}{ccc|c}  
% 1&1/2&1/2&3/2\\0&1&3&7\\\answer{0}&\answer{0}&\answer{-4}&\answer{-5}
%  \end{array}\right]\]
% \end{prompt}
% \begin{problem}
% \begin{prompt}
% \[\begin{array}{c}
%  \\
%  \\
%  \xrightarrow{-\frac{1}{4}R_3}
% \end{array}
% \left[\begin{array}{ccc|c}  
%  1&1/2&1/2&3/2\\0&1&3&7\\\answer{0}&\answer{0}&\answer{1}&\answer{5/4}
%  \end{array}\right]\]
%  \end{prompt}
% \begin{problem}
% \begin{prompt}
% \[\begin{array}{c}
%  \\
%  \xrightarrow{R_2-3R_3}\\
% \\
% \end{array}
% \left[\begin{array}{ccc|c}  
%  1&1/2&1/2&3/2\\\answer{0}&\answer{1}&\answer{0}&\answer{13/4}\\0&0&1&5/4
%  \end{array}\right]\]
% \end{prompt}
% \begin{problem}
% \begin{prompt}
 
%  \[\begin{array}{c}
%  \xrightarrow{R_1-\frac{1}{2}R_3}\\
% \\
% \\
% \end{array}
% \left[\begin{array}{ccc|c}  
% \answer{1}&\answer{1/2}&\answer{0}&\answer{7/8}\\0&1&0&13/4\\0&0&1&5/4
%  \end{array}\right]\]
%  \end{prompt}
%  \begin{problem}
% \begin{prompt}
% \[\begin{array}{c}
%  \xrightarrow{R_1-\frac{1}{2}R_2}\\
% \\
% \\
% \end{array}
% \left[\begin{array}{ccc|c}  
% \answer{1}&\answer{0}&\answer{0}&\answer{-3/4}\\0&1&0&13/4\\0&0&1&5/4
%  \end{array}\right]\]
%  \end{prompt}
% \end{problem}
%  \end{problem}
%  \end{problem}
%  \end{problem}
%  \end{problem}
%  \end{problem}
%  \end{problem}
 \end{problem}
 \end{problem}
 
 \begin{problem}\label{prob:twowaystorref2}
The following reduction steps do not follow the algorithm but are easier for a human to carry out because they use fewer fractions. Follow the indicated steps to transform the matrix to its reduced row-echelon form.

Your entries must be fractions or integers - no decimals.

\begin{center}
$\left[\begin{array}{ccc|c}  2&1&1&3\\-1&0&1&2\\1&1&-2&0
 \end{array}\right]$
 
$\begin{array}{c}
  \xrightarrow{R_1+R_2}\\
\\
\\
 \end{array}$
$\left[\begin{array}{ccc|c}  
 \answer{1}&\answer{1}&\answer{2}&\answer{5}\\-1&0&1&2\\1&1&-2&0
 \end{array}\right]$
 
 
$ \begin{array}{c}
 \\
 \\
 \xrightarrow{R_3+R_2}
\end{array}$
$\left[\begin{array}{ccc|c}  
 \answer{1}&\answer{1}&\answer{2}&\answer{5}\\-1&0&1&2\\\answer{0}&\answer{1}&\answer{-1}&\answer{2}
 \end{array}\right]$
 
 
$ \begin{array}{c}
 \xrightarrow{R_1+R_2}\\
  \\
\\
\end{array}$
$\left[\begin{array}{ccc|c}  
 \answer{0}&\answer{1}&\answer{3}&\answer{7}\\-1&0&1&2\\\answer{0}&\answer{1}&\answer{-1}&\answer{2}
 \end{array}\right]$
 
 
$ \begin{array}{c}
 \xrightarrow{R_1-R_3}\\
\\
\\
\end{array}$
$\left[\begin{array}{ccc|c}  
 \answer{0}&\answer{0}&\answer{4}&\answer{5}\\-1&0&1&2\\\answer{0}&\answer{1}&\answer{-1}&\answer{2}
 \end{array}\right]$
 
 
$ \begin{array}{c}
 \xrightarrow{\frac{1}{4}R_1}\\
 \xrightarrow{-R_2}\\
\\
\end{array}$
$\left[\begin{array}{ccc|c}  
\answer{0}&\answer{0}&\answer{1}&\answer{5/4}\\\answer{1}&\answer{0}&\answer{-1}&\answer{-2}\\\answer{0}&\answer{1}&\answer{-1}&\answer{2}
 \end{array}\right]$
 
 
$ \begin{array}{c}
 \\
 \xrightarrow{R_2+R_1}\\
 \\
\end{array}$
$\left[\begin{array}{ccc|c}  
 \answer{0}&\answer{0}&\answer{1}&\answer{5/4}\\\answer{1}&\answer{0}&\answer{0}&\answer{-3/4}\\\answer{0}&\answer{1}&\answer{-1}&\answer{2}
 \end{array}\right]$
 
 
$ \begin{array}{c}
 \\
 \\
 \xrightarrow{R_3+R_1}
\end{array}$
$\left[\begin{array}{ccc|c}  
 \answer{0}&\answer{0}&\answer{1}&\answer{5/4}\\\answer{1}&\answer{0}&\answer{0}&\answer{-3/4}\\\answer{0}&\answer{1}&\answer{0}&\answer{13/4}
 \end{array}\right]$
 
 
$ \begin{array}{c}
 \\
 \xrightarrow{\mbox{row exchanges}}\\
\\
\end{array}$
$\left[\begin{array}{ccc|c}  
 1&0&0&\answer{-3/4}\\0&1&0&\answer{13/4}\\0&0&1&\answer{5/4}
 \end{array}\right]$
\end{center}
 \end{problem}


\begin{problem}
Find the rank of each matrix.
\begin{problem}\label{prob:rankofmat1}
$$A=\begin{bmatrix}4&3&-1\\-8&-6&2\end{bmatrix}$$
Answer:

$$\mbox{rank}(A)=\answer{1}$$
\end{problem}

\begin{problem}\label{prob:rankofmat2}
$$B=\begin{bmatrix}1&1\\2&-2\\3&-1\end{bmatrix}$$
Answer:

$$\mbox{rank}(B)=\answer{2}$$
\end{problem}

\begin{problem}\label{prob:rankofmat3}
$$C=\begin{bmatrix}1&0&1\\2&1&3\\0&1&-2\end{bmatrix}$$
Answer:

$$\mbox{rank}(C)=\answer{3}$$
\end{problem}

\begin{problem}\label{prob:rankofmat4}
$$D=\begin{bmatrix}1&1&2\\-1&-2&1\\1&0&5\end{bmatrix}$$
Answer:

$$\mbox{rank}(D)=\answer{2}$$
\end{problem}
\end{problem}

\begin{problem}\label{prob:rankofmat5}
Suppose $A$ is a $5\times 7$ matrix.  Which of the following can be true?
\begin{multipleChoice}
 \choice{$\mbox{rank}(A)=7$}
 \choice{$\mbox{rank}(A)=6$}
 \choice[correct]{$\mbox{rank}(A)=5$}
 \choice{All of the above}
 \end{multipleChoice}
\end{problem}

\begin{problem}\label{prob:4eq5un}
Suppose a linear system has $4$ equations and $5$ unknowns.  Which of the following is NOT a possibility?
\begin{multipleChoice}
  \choice[correct]{The system has a unique solution}
 \choice{The system has no solutions}
 \choice{The system has infinitely many solutions}
 \end{multipleChoice}
\end{problem}

\begin{problem}\label{prob:leadones}
Suppose $A$ is a matrix such that $\mbox{rref}(A)$ has $5$ leading $1's$.  What do we know to be true about $A$
\begin{selectAll}
 \choice[correct]{$\mbox{rank}(A)=5$}
 \choice[correct]{$A$ has at least $25$ entries}
 \choice[correct]{Any row-echelon form of $A$ will have exactly $5$ nonzero rows}
 \choice{Some row-echelon forms of $A$ may have more than $5$ nonzero rows}
 \choice{Some row-echelon forms of $A$ may have less than $5$ nonzero rows}
 \end{selectAll}
\end{problem}

\begin{problem}\label{prob:rankaugvscoeff}
In this problem we will discuss how the rank of the {\it coefficient matrix} associated with a linear system compares to the rank of the {\it augmented matrix} associated with the system.  
\begin{enumerate}
\item Explain why the rank of the augmented matrix has to be greater than or equal to the rank of the coefficient matrix.
    \item Prove that for a {\it consistent} system the rank of the coefficient matrix will be the same as the rank of the {\it augmented} matrix.
    \item Give an example of an inconsistent system for which the rank of the augmented matrix is greater than the rank of the coefficient matrix.
    \item Can the rank of an augmented matrix be greater than the number of variables?
    \item Is the following statement true?
    
    ``If the rank of the augmented matrix associated with a linear system is greater than the rank of the coefficient matrix, then the system is inconsistent."
\end{enumerate}
\end{problem}

\section*{Text Source}
The section on Rank is an adaptation of Section 1.2 of Keith Nicholson's \href{https://open.umn.edu/opentextbooks/textbooks/linear-algebra-with-applications}{\it Linear Algebra with Applications}.

W. Keith Nicholson, {\it Linear Algebra with Applications}, Lyryx 2018, Open Edition, p 15-17.

\section*{Bibliography}

[Yuster] Thomas Yuster, The Reduced Row Echelon Form of a Matrix is Unique: a
Simple Proof, Mathematics Magazine, vol. 57, no. 2 (Mar. 1984), pp. 93-94.
\end{document} 