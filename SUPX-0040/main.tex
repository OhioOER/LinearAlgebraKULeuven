\documentclass{ximera}
%%% Begin Laad packages

\makeatletter
\@ifclassloaded{xourse}{%
    \typeout{Start loading preamble.tex (in a XOURSE)}%
    \def\isXourse{true}   % automatically defined; pre 112022 it had to be set 'manually' in a xourse
}{%
    \typeout{Start loading preamble.tex (NOT in a XOURSE)}%
}
\makeatother

\def\isEn\true 

\pgfplotsset{compat=1.16}

\usepackage{currfile}

% 201908/202301: PAS OP: babel en doclicense lijken problemen te veroorzaken in .jax bestand
% (wegens syntax error met toegevoegde \newcommands ...)
\pdfOnly{
    \usepackage[type={CC},modifier={by-nc-sa},version={4.0}]{doclicense}
    %\usepackage[hyperxmp=false,type={CC},modifier={by-nc-sa},version={4.0}]{doclicense}
    %%% \usepackage[dutch]{babel}
}



\usepackage[utf8]{inputenc}
\usepackage{morewrites}   % nav zomercursus (answer...?)
\usepackage{multirow}
\usepackage{multicol}
\usepackage{tikzsymbols}
\usepackage{ifthen}
%\usepackage{animate} BREAKS HTML STRUCTURE USED BY XIMERA
\usepackage{relsize}

\usepackage{eurosym}    % \euro  (€ werkt niet in xake ...?)
\usepackage{fontawesome} % smileys etc

% Nuttig als ook interactieve beamer slides worden voorzien:
\providecommand{\p}{} % default nothing ; potentially usefull for slides: redefine as \pause
%providecommand{\p}{\pause}

    % Layout-parameters voor het onderschrift bij figuren
\usepackage[margin=10pt,font=small,labelfont=bf, labelsep=endash,format=hang]{caption}
%\usepackage{caption} % captionof
%\usepackage{pdflscape}    % landscape environment

% Met "\newcommand\showtodonotes{}" kan je todonotes tonen (in pdf/online)
% 201908: online werkt het niet (goed)
\providecommand\showtodonotes{disable}
\providecommand\todo[1]{\typeout{TODO #1}}
%\usepackage[\showtodonotes]{todonotes}
%\usepackage{todonotes}

%
% Poging tot aanpassen layout
%
\usepackage{tcolorbox}
\tcbuselibrary{theorems}

%%% Einde laad packages

%%% Begin Ximera specifieke zaken

\graphicspath{
	{../../}
	{../}
	{./}
  	{../../pictures/}
   	{../pictures/}
   	{./pictures/}
	{./explog/}    % M05 in groeimodellen       
}

%%% Einde Ximera specifieke zaken

%
% define softer blue/red/green, use KU Leuven base colors for blue (and dark orange for red ?)
%
% todo: rather redefine blue/red/green ...?
%\definecolor{xmblue}{rgb}{0.01, 0.31, 0.59}
%\definecolor{xmred}{rgb}{0.89, 0.02, 0.17}
\definecolor{xmdarkblue}{rgb}{0.122, 0.671, 0.835}   % KU Leuven Blauw
\definecolor{xmblue}{rgb}{0.114, 0.553, 0.69}        % KU Leuven Blauw
\definecolor{xmgreen}{rgb}{0.13, 0.55, 0.13}         % No KULeuven variant for green found ...

\definecolor{xmaccent}{rgb}{0.867, 0.541, 0.18}      % KU Leuven Accent (orange ...)
\definecolor{kuaccent}{rgb}{0.867, 0.541, 0.18}      % KU Leuven Accent (orange ...)

\colorlet{xmred}{xmaccent!50!black}                  % Darker version of KU Leuven Accent

\providecommand{\blue}[1]{{\color{blue}#1}}    
\providecommand{\red}[1]{{\color{red}#1}}

\renewcommand\CancelColor{\color{xmaccent!50!black}}

% werkt in math en text mode om MATH met oranje (of grijze...)  achtergond te tonen (ook \important{\text{blabla}} lijkt te werken)
%\newcommand{\important}[1]{\ensuremath{\colorbox{xmaccent!50!white}{$#1$}}}   % werkt niet in Mathjax
%\newcommand{\important}[1]{\ensuremath{\colorbox{lightgray}{$#1$}}}
\newcommand{\important}[1]{\ensuremath{\colorbox{orange}{$#1$}}}   % TODO: kleur aanpassen voor mathjax; wordt overschreven infra!


% Uitzonderlijk kan met \pdfnl in de PDF een newline worden geforceerd, die online niet nodig/nuttig is omdat daar de regellengte hoe dan ook niet gekend is.
\ifdefined\HCode%
\providecommand{\pdfnl}{}%
\else%
\providecommand{\pdfnl}{%
  \\%
}%
\fi

% Uitzonderlijk kan met \handoutnl in de handout-PDF een newline worden geforceerd, die noch online noch in de PDF-met-antwoorden nuttig is.
\ifdefined\HCode
\providecommand{\handoutnl}{}
\else
\providecommand{\handoutnl}{%
\ifhandout%
  \nl%
\fi%
}
\fi



% \cellcolor IGNORED by tex4ht ?
% \begin{center} seems not to wordk
    % (missing margin-left: auto;   on tabular-inside-center ???)
%\newcommand{\importantcell}[1]{\ensuremath{\cellcolor{lightgray}#1}}  %  in tabular; usablility to be checked
\providecommand{\importantcell}[1]{\ensuremath{#1}}     % no mathjax2 support for colloring array cells

\pdfOnly{
  \renewcommand{\important}[1]{\ensuremath{\colorbox{kuaccent!50!white}{$#1$}}}
  \renewcommand{\importantcell}[1]{\ensuremath{\cellcolor{kuaccent!40!white}#1}}   
}

%%% Tikz styles


\pgfplotsset{compat=1.16}

\usetikzlibrary{trees,positioning,arrows,fit,shapes,math,calc,decorations.markings,through,intersections,patterns,matrix}

\usetikzlibrary{decorations.pathreplacing,backgrounds}    % 5/2023: from experimental


\usetikzlibrary{angles,quotes}

\usepgfplotslibrary{fillbetween} % bepaalde_integraal
\usepgfplotslibrary{polar}    % oa voor poolcoordinaten.tex

\pgfplotsset{ownstyle/.style={axis lines = center, axis equal image, xlabel = $x$, ylabel = $y$, enlargelimits}} 

\pgfplotsset{
	plot/.style={no marks,samples=50}
}

\newcommand{\xmPlotsColor}{
	\pgfplotsset{
		plot1/.style={darkgray,no marks,samples=100},
		plot2/.style={lightgray,no marks,samples=100},
		plotresult/.style={blue,no marks,samples=100},
		plotblue/.style={blue,no marks,samples=100},
		plotred/.style={red,no marks,samples=100},
		plotgreen/.style={green,no marks,samples=100},
		plotpurple/.style={purple,no marks,samples=100}
	}
}
\newcommand{\xmPlotsBlackWhite}{
	\pgfplotsset{
		plot1/.style={black,loosely dashed,no marks,samples=100},
		plot2/.style={black,loosely dotted,no marks,samples=100},
		plotresult/.style={black,no marks,samples=100},
		plotblue/.style={black,no marks,samples=100},
		plotred/.style={black,dotted,no marks,samples=100},
		plotgreen/.style={black,dashed,no marks,samples=100},
		plotpurple/.style={black,dashdotted,no marks,samples=100}
	}
}


\newcommand{\xmPlotsColorAndStyle}{
	\pgfplotsset{
		plot1/.style={darkgray,no marks,samples=100},
		plot2/.style={lightgray,no marks,samples=100},
		plotresult/.style={blue,no marks,samples=100},
		plotblue/.style={xmblue,no marks,samples=100},
		plotred/.style={xmred,dashed,thick,no marks,samples=100},
		plotgreen/.style={xmgreen,dotted,very thick,no marks,samples=100},
		plotpurple/.style={purple,no marks,samples=100}
	}
}


%\iftikzexport
\xmPlotsColorAndStyle
%\else
%\xmPlotsBlackWhite
%\fi
%%%


%
% Om venndiagrammen te arceren ...
%
\makeatletter
\pgfdeclarepatternformonly[\hatchdistance,\hatchthickness]{north east hatch}% name
{\pgfqpoint{-1pt}{-1pt}}% below left
{\pgfqpoint{\hatchdistance}{\hatchdistance}}% above right
{\pgfpoint{\hatchdistance-1pt}{\hatchdistance-1pt}}%
{
	\pgfsetcolor{\tikz@pattern@color}
	\pgfsetlinewidth{\hatchthickness}
	\pgfpathmoveto{\pgfqpoint{0pt}{0pt}}
	\pgfpathlineto{\pgfqpoint{\hatchdistance}{\hatchdistance}}
	\pgfusepath{stroke}
}
\pgfdeclarepatternformonly[\hatchdistance,\hatchthickness]{north west hatch}% name
{\pgfqpoint{-\hatchthickness}{-\hatchthickness}}% below left
{\pgfqpoint{\hatchdistance+\hatchthickness}{\hatchdistance+\hatchthickness}}% above right
{\pgfpoint{\hatchdistance}{\hatchdistance}}%
{
	\pgfsetcolor{\tikz@pattern@color}
	\pgfsetlinewidth{\hatchthickness}
	\pgfpathmoveto{\pgfqpoint{\hatchdistance+\hatchthickness}{-\hatchthickness}}
	\pgfpathlineto{\pgfqpoint{-\hatchthickness}{\hatchdistance+\hatchthickness}}
	\pgfusepath{stroke}
}
%\makeatother

\tikzset{
    hatch distance/.store in=\hatchdistance,
    hatch distance=10pt,
    hatch thickness/.store in=\hatchthickness,
   	hatch thickness=2pt
}

\colorlet{circle edge}{black}
\colorlet{circle area}{blue!20}


\tikzset{
    filled/.style={fill=green!30, draw=circle edge, thick},
    arceerl/.style={pattern=north east hatch, pattern color=blue!50, draw=circle edge},
    arceerr/.style={pattern=north west hatch, pattern color=yellow!50, draw=circle edge},
    outline/.style={draw=circle edge, thick}
}




%%% Updaten commando's
\def\hoofding #1#2#3{\maketitle}     % OBSOLETE ??

% we willen (bijna) altijd \geqslant ipv \geq ...!
\newcommand{\geqnoslant}{\geq}
\renewcommand{\geq}{\geqslant}
\newcommand{\leqnoslant}{\leq}
\renewcommand{\leq}{\leqslant}

% Todo: (201908) waarom komt er (soms) underlined voor emph ...?
\renewcommand{\emph}[1]{\textit{#1}}

% API commando's

\newcommand{\ds}{\displaystyle}
\newcommand{\ts}{\textstyle}  % tegenhanger van \ds   (Ximera zet PER  DEFAULT \ds!)

% uit Zomercursus-macro's: 
\newcommand{\bron}[1]{\begin{scriptsize} \emph{#1} \end{scriptsize}}     % deprecated ...?


%definities nieuwe commando's - afkortingen veel gebruikte symbolen
\newcommand{\R}{\ensuremath{\mathbb{R}}}
\newcommand{\Rnul}{\ensuremath{\mathbb{R}_0}}
\newcommand{\Reen}{\ensuremath{\mathbb{R}\setminus\{1\}}}
\newcommand{\Rnuleen}{\ensuremath{\mathbb{R}\setminus\{0,1\}}}
\newcommand{\Rplus}{\ensuremath{\mathbb{R}^+}}
\newcommand{\Rmin}{\ensuremath{\mathbb{R}^-}}
\newcommand{\Rnulplus}{\ensuremath{\mathbb{R}_0^+}}
\newcommand{\Rnulmin}{\ensuremath{\mathbb{R}_0^-}}
\newcommand{\Rnuleenplus}{\ensuremath{\mathbb{R}^+\setminus\{0,1\}}}
\newcommand{\N}{\ensuremath{\mathbb{N}}}
\newcommand{\Nnul}{\ensuremath{\mathbb{N}_0}}
\newcommand{\Z}{\ensuremath{\mathbb{Z}}}
\newcommand{\Znul}{\ensuremath{\mathbb{Z}_0}}
\newcommand{\Zplus}{\ensuremath{\mathbb{Z}^+}}
\newcommand{\Zmin}{\ensuremath{\mathbb{Z}^-}}
\newcommand{\Znulplus}{\ensuremath{\mathbb{Z}_0^+}}
\newcommand{\Znulmin}{\ensuremath{\mathbb{Z}_0^-}}
\newcommand{\C}{\ensuremath{\mathbb{C}}}
\newcommand{\Cnul}{\ensuremath{\mathbb{C}_0}}
\newcommand{\Cplus}{\ensuremath{\mathbb{C}^+}}
\newcommand{\Cmin}{\ensuremath{\mathbb{C}^-}}
\newcommand{\Cnulplus}{\ensuremath{\mathbb{C}_0^+}}
\newcommand{\Cnulmin}{\ensuremath{\mathbb{C}_0^-}}
\newcommand{\Q}{\ensuremath{\mathbb{Q}}}
\newcommand{\Qnul}{\ensuremath{\mathbb{Q}_0}}
\newcommand{\Qplus}{\ensuremath{\mathbb{Q}^+}}
\newcommand{\Qmin}{\ensuremath{\mathbb{Q}^-}}
\newcommand{\Qnulplus}{\ensuremath{\mathbb{Q}_0^+}}
\newcommand{\Qnulmin}{\ensuremath{\mathbb{Q}_0^-}}

\newcommand{\perdef}{\overset{\mathrm{def}}{=}}
\newcommand{\pernot}{\overset{\mathrm{notatie}}{=}}
\newcommand\perinderdaad{\overset{!}{=}}     % voorlopig gebruikt in limietenrekenregels
\newcommand\perhaps{\overset{?}{=}}          % voorlopig gebruikt in limietenrekenregels

\newcommand{\degree}{^\circ}


\DeclareMathOperator{\dom}{dom}     % domein
\DeclareMathOperator{\codom}{codom} % codomein
\DeclareMathOperator{\bld}{bld}     % beeld
\DeclareMathOperator{\graf}{graf}   % grafiek
\DeclareMathOperator{\rico}{rico}   % richtingcoëfficient
\DeclareMathOperator{\co}{co}       % coordinaat
\DeclareMathOperator{\gr}{gr}       % graad

\newcommand{\func}[5]{\ensuremath{#1: #2 \rightarrow #3: #4 \mapsto #5}} % Easy to write a function


% Operators
\DeclareMathOperator{\bgsin}{bgsin}
\DeclareMathOperator{\bgcos}{bgcos}
\DeclareMathOperator{\bgtan}{bgtan}
\DeclareMathOperator{\bgcot}{bgcot}
\DeclareMathOperator{\bgsinh}{bgsinh}
\DeclareMathOperator{\bgcosh}{bgcosh}
\DeclareMathOperator{\bgtanh}{bgtanh}
\DeclareMathOperator{\bgcoth}{bgcoth}

% Oude \Bgsin etc deprecated: gebruik \bgsin, en herdefinieer dat als je Bgsin wil!
%\DeclareMathOperator{\cosec}{cosec}    % not used? gebruik \csc en herdefinieer

% operatoren voor differentialen: to be verified; 1/2020: inconsequent gebruik bij afgeleiden/integralen
\renewcommand{\d}{\mathrm{d}}
\newcommand{\dx}{\d x}
\newcommand{\dd}[1]{\frac{\mathrm{d}}{\mathrm{d}#1}}
\newcommand{\ddx}{\dd{x}}

% om in voorbeelden/oefeningen de notatie voor afgeleiden te kunnen kiezen
% Usage: \afg{(2\sin(x))}  (en wordt d/dx, of accent, of D )
%\newcommand{\afg}[1]{{#1}'}
\newcommand{\afg}[1]{\left(#1\right)'}
%\renewcommand{\afg}[1]{\frac{\mathrm{d}#1}{\mathrm{d}x}}   % include in relevant exercises ...
%\renewcommand{\afg}[1]{D{#1}}

%
% \xmxxx commands: Extra KU Leuven functionaliteit van, boven of naast Ximera
%   ( Conventie 8/2019: xm+nederlandse omschrijving, maar is niet consequent gevolgd, en misschien ook niet erg handig !)
%
% (Met een minimale ximera.cls en preamble.tex zou een bruikbare .pdf moeten kunnen worden gemaakt van eender welke ximera)
%
% Usage: \xmtitle[Mijn korte abstract]{Mijn titel}{Mijn abstract}
% Eerste command na \begin{document}:
%  -> definieert de \title
%  -> definieert de abstract
%  -> doet \maketitle ( dus: print de hoofding als 'chapter' of 'sectie')
% Optionele parameter geeft eenn kort abstract (die met de globale setting \xmshortabstract{} al dan niet kan worden geprint.
% De optionele korte abstract kan worden gebruikt voor pseudo-grappige abtsarts, dus dus globaal al dan niet kunnen worden gebuikt...
% Globale settings:
%  de (optionele) 'korte abstract' wordt enkele getoond als \xmshortabstract is gezet
\providecommand\xmshortabstract{} % default: print (only!) short abstract if present
\newcommand{\xmtitle}[3][]{
	\title{#2}
	\begin{abstract}
		\ifdefined\xmshortabstract
		\ifstrempty{#1}{%
			#3
		}{%
			#1
		}%
		\else
		#3
		\fi
	\end{abstract}
	\maketitle
}

% 
% Kleine grapjes: moeten zonder verder gevolg kunnen worden verwijderd
%
%\newcommand{\xmopje}[1]{{\small#1{\reversemarginpar\marginpar{\Smiley}}}}   % probleem in floats!!
\newtoggle{showxmopje}
\toggletrue{showxmopje}

\newcommand{\xmopje}[1]{%
   \iftoggle{showxmopje}{#1}{}%
}


% -> geef een abstracte-formule-met-rechts-een-concreet-voorbeeld
% VB:  \formulevb{a^2+b^2=c^2}{3^2+4^2=5^2}
%
\ifdefined\HCode
\NewEnviron{xmdiv}[1]{\HCode{\Hnewline<div class="#1">\Hnewline}\BODY{\HCode{\Hnewline</div>\Hnewline}}}
\else
\NewEnviron{xmdiv}[1]{\BODY}
\fi

\providecommand{\formulevb}[2]{
	{\centering

    \begin{xmdiv}{xmformulevb}    % zie css voor online layout !!!
	\begin{tabular}{lcl}
		\important{#1}
		&  &
		Vb: $#2$
		\end{tabular}
	\end{xmdiv}

	}
}

\ifdefined\HCode
\providecommand{\vb}[1]{%
    \HCode{\Hnewline<span class="xmvb">}#1\HCode{</span>\Hnewline}%
}
\else
\providecommand{\vb}[1]{
    \colorbox{blue!10}{#1}
}
\fi

\ifdefined\HCode
\providecommand{\xmcolorbox}[2]{
	\HCode{\Hnewline<div class="xmcolorbox">\Hnewline}#2\HCode{\Hnewline</div>\Hnewline}
}
\else
\providecommand{\xmcolorbox}[2]{
  \cellcolor{#1}#2
}
\fi


\ifdefined\HCode
\providecommand{\xmopmerking}[1]{
 \HCode{\Hnewline<div class="xmopmerking">\Hnewline}#1\HCode{\Hnewline</div>\Hnewline}
}
\else
\providecommand{\xmopmerking}[1]{
	{\footnotesize #1}
}
\fi
% \providecommand{\voorbeeld}[1]{
% 	\colorbox{blue!10}{$#1$}
% }



% Hernoem Proof naar Bewijs, nodig voor HTML versie
\renewcommand*{\proofname}{Bewijs}

% Om opgave van oefening (wordt niet geprint bij oplossingenblad)
% (to be tested test)
\NewEnviron{statement}{\BODY}

% Environment 'oplossing' en 'uitkomst'
% voor resp. volledige 'uitwerking' dan wel 'enkel eindresultaat'
% geimplementeerd via feedback, omdat er in de ximera-server adhoc feedback-code is toegevoegd
%% Niet tonen indien handout
%% Te gebruiken om volledige oplossingen/uitwerkingen van oefeningen te tonen
%% \begin{oplossing}        De optelling is commutatief \end{oplossing}  : verschijnt online enkel 'op vraag'
%% \begin{oplossing}[toon]  De optelling is commutatief \end{oplossing}  : verschijnt steeds onmiddellijk online (bv te gebruiken bij voorbeelden) 

\ifhandout%
    \NewEnviron{oplossing}[1][onzichtbaar]%
    {%
    \ifthenelse{\equal{\detokenize{#1}}{\detokenize{toon}}}
    {
    \def\PH@Command{#1}% Use PH@Command to hold the content and be a target for "\expandafter" to expand once.

    \begin{trivlist}% Begin the trivlist to use formating of the "Feedback" label.
    \item[\hskip \labelsep\small\slshape\bfseries Oplossing% Format the "Feedback" label. Don't forget the space.
    %(\texttt{\detokenize\expandafter{\PH@Command}}):% Format (and detokenize) the condition for feedback to trigger
    \hspace{2ex}]\small%\slshape% Insert some space before the actual feedback given.
    \BODY
    \end{trivlist}
    }
    {  % \begin{feedback}[solution]   \BODY     \end{feedback}  }
    }
    }    
\else
% ONLY for HTML; xmoplossing is styled with css, and is not, and need not be a LaTeX environment
% THUS: it does NOT use feedback anymore ...
%    \NewEnviron{oplossing}{\begin{expandable}{xmoplossing}{\nlen{Toon uitwerking}{Show solution}}{\BODY}\end{expandable}}
    \newenvironment{oplossing}[1][onzichtbaar]
   {%
       \begin{expandable}{xmoplossing}{}
   }
   {%
   	   \end{expandable}
   } 
%     \newenvironment{oplossing}[1][onzichtbaar]
%    {%
%        \begin{feedback}[solution]   	
%    }
%    {%
%    	   \end{feedback}
%    } 
\fi

\ifhandout%
    \NewEnviron{uitkomst}[1][onzichtbaar]%
    {%
    \ifthenelse{\equal{\detokenize{#1}}{\detokenize{toon}}}
    {
    \def\PH@Command{#1}% Use PH@Command to hold the content and be a target for "\expandafter" to expand once.

    \begin{trivlist}% Begin the trivlist to use formating of the "Feedback" label.
    \item[\hskip \labelsep\small\slshape\bfseries Uitkomst:% Format the "Feedback" label. Don't forget the space.
    %(\texttt{\detokenize\expandafter{\PH@Command}}):% Format (and detokenize) the condition for feedback to trigger
    \hspace{2ex}]\small%\slshape% Insert some space before the actual feedback given.
    \BODY
    \end{trivlist}
    }
    {  % \begin{feedback}[solution]   \BODY     \end{feedback}  }
    }
    }    
\else
\ifdefined\HCode
   \newenvironment{uitkomst}[1][onzichtbaar]
    {%
        \begin{expandable}{xmuitkomst}{}%
    }
    {%
    	\end{expandable}%
    } 
\else
  % Do NOT print 'uitkomst' in non-handout
  %  (presumably, there is also an 'oplossing' ??)
  \newenvironment{uitkomst}[1][onzichtbaar]{}{}
\fi
\fi

%
% Uitweidingen zijn extra's die niet redelijkerwijze tot de leerstof behoren
% Uitbreidingen zijn extra's die wel redelijkerwijze tot de leerstof van bv meer geavanceerde versies kunnen behoren (B-programma/Wiskundestudenten/...?)
% Nog niet voorzien: design voor verschillende versies (A/B programma, BIO, voorkennis/ ...)
% Voor 'uitweidingen' is er een environment die online per default is ingeklapt, en in pdf al dan niet kan worden geincluded  (via \xmnouitweiding) 
%
% in een xourse, per default GEEN uitweidingen, tenzij \xmuitweiding expliciet ergens is gezet ...
\ifdefined\isXourse
   \ifdefined\xmuitweiding
   \else
       \def\xmnouitweiding{true}
   \fi
\fi

\ifdefined\xmnouitweiding
\newcounter{xmuitweiding}  % anders error undefined ...  
\excludecomment{xmuitweiding}
\else
\newtheoremstyle{dotless}{}{}{}{}{}{}{ }{}
\theoremstyle{dotless}
\newtheorem*{xmuitweidingnofrills}{}   % nofrills = no accordion; gebruikt dus de dotless theoremstyle!

\newcounter{xmuitweiding}
\newenvironment{xmuitweiding}[1][ ]%
{% 
	\refstepcounter{xmuitweiding}%
    \begin{expandable}{xmuitweiding}{\nlentext{Uitweiding \arabic{xmuitweiding}: #1}{Digression \arabic{xmuitweiding}: #1}}%
	\begin{xmuitweidingnofrills}%
}
{%
    \end{xmuitweidingnofrills}%
    \end{expandable}%
}   
% \newenvironment{xmuitweiding}[1][ ]%
% {% 
% 	\refstepcounter{xmuitweiding}
% 	\begin{accordion}\begin{accordion-item}[Uitweiding \arabic{xmuitweiding}: #1]%
% 			\begin{xmuitweidingnofrills}%
% 			}
% 			{\end{xmuitweidingnofrills}\end{accordion-item}\end{accordion}}   
\fi


\newenvironment{xmexpandable}[1][]{
	\begin{accordion}\begin{accordion-item}[#1]%
		}{\end{accordion-item}\end{accordion}}


% Command that gives a selection box online, but just prints the right answer in pdf
\newcommand{\xmonlineChoice}[1]{\pdfOnly{\wordchoicegiventrue}\wordChoice{#1}\pdfOnly{\wordchoicegivenfalse}}   % deprecated, gebruik onlineChoice ...
\newcommand{\onlineChoice}[1]{\pdfOnly{\wordchoicegiventrue}\wordChoice{#1}\pdfOnly{\wordchoicegivenfalse}}

\newcommand{\TJa}{\nlentext{ Ja }{ Yes }}
\newcommand{\TNee}{\nlentext{ Nee }{ No }}
\newcommand{\TJuist}{\nlentext{ Juist }{ True }}
\newcommand{\TFout}{\nlentext{ Fout }{ False }}

\newcommand{\choiceTrue }{{\renewcommand{\choiceminimumhorizontalsize}{4em}\wordChoice{\choice[correct]{\TJuist}\choice{\TFout}}}}
\newcommand{\choiceFalse}{{\renewcommand{\choiceminimumhorizontalsize}{4em}\wordChoice{\choice{\TJuist}\choice[correct]{\TFout}}}}

\newcommand{\choiceYes}{{\renewcommand{\choiceminimumhorizontalsize}{3em}\wordChoice{\choice[correct]{\TJa}\choice{\TNee}}}}
\newcommand{\choiceNo }{{\renewcommand{\choiceminimumhorizontalsize}{3em}\wordChoice{\choice{\TJa}\choice[correct]{\TNee}}}}

% Optional nicer formatting for wordChoice in PDF

\let\inlinechoiceorig\inlinechoice

%\makeatletter
%\providecommand{\choiceminimumverticalsize}{\vphantom{$\frac{\sqrt{2}}{2}$}}   % minimum height of boxes (cfr infra)
\providecommand{\choiceminimumverticalsize}{\vphantom{$\tfrac{2}{2}$}}   % minimum height of boxes (cfr infra)
\providecommand{\choiceminimumhorizontalsize}{1em}   % minimum width of boxes (cfr infra)

\newcommand{\inlinechoicesquares}[2][]{%
		\setkeys{choice}{#1}%
		\ifthenelse{\boolean{\choice@correct}}%
		{%
            \ifhandout%
               \fbox{\choiceminimumverticalsize #2}\allowbreak\ignorespaces%
            \else%
               \fcolorbox{blue}{blue!20}{\choiceminimumverticalsize #2}\allowbreak\ignorespaces\setkeys{choice}{correct=false}\ignorespaces%
            \fi%
		}%
		{% else
			\fbox{\choiceminimumverticalsize #2}\allowbreak\ignorespaces%  HACK: wat kleiner, zodat fits on line ... 	
		}%
}

\newcommand{\inlinechoicesquareX}[2][]{%
		\setkeys{choice}{#1}%
		\ifthenelse{\boolean{\choice@correct}}%
		{%
            \ifhandout%
               \framebox[\ifdim\choiceminimumhorizontalsize<\width\width\else\choiceminimumhorizontalsize\fi]{\choiceminimumverticalsize\ #2\ }\allowbreak\ignorespaces\setkeys{choice}{correct=false}\ignorespaces%
            \else%
               \fcolorbox{blue}{blue!20}{\makebox[\ifdim\choiceminimumhorizontalsize<\width\width\else\choiceminimumhorizontalsize\fi]{\choiceminimumverticalsize #2}}\allowbreak\ignorespaces\setkeys{choice}{correct=false}\ignorespaces%
            \fi%
		}%
		{% else
        \ifhandout%
			\framebox[\ifdim\choiceminimumhorizontalsize<\width\width\else\choiceminimumhorizontalsize\fi]{\choiceminimumverticalsize\ #2\ }\allowbreak\ignorespaces%  HACK: wat kleiner, zodat fits on line ... 	
        \fi
		}%
}


\newcommand{\inlinechoicepuntjes}[2][]{%
		\setkeys{choice}{#1}%
		\ifthenelse{\boolean{\choice@correct}}%
		{%
            \ifhandout%
               \dots\ldots\ignorespaces\setkeys{choice}{correct=false}\ignorespaces
            \else%
               \fcolorbox{blue}{blue!20}{\choiceminimumverticalsize #2}\allowbreak\ignorespaces\setkeys{choice}{correct=false}\ignorespaces%
            \fi%
		}%
		{% else
			%\fbox{\choiceminimumverticalsize #2}\allowbreak\ignorespaces%  HACK: wat kleiner, zodat fits on line ... 	
		}%
}

% print niets, maar definieer globale variable \myanswer
%  (gebruikt om oplossingsbladen te printen) 
\newcommand{\inlinechoicedefanswer}[2][]{%
		\setkeys{choice}{#1}%
		\ifthenelse{\boolean{\choice@correct}}%
		{%
               \gdef\myanswer{#2}\setkeys{choice}{correct=false}

		}%
		{% else
			%\fbox{\choiceminimumverticalsize #2}\allowbreak\ignorespaces%  HACK: wat kleiner, zodat fits on line ... 	
		}%
}



%\makeatother

\newcommand{\setchoicedefanswer}{
\ifdefined\HCode
\else
%    \renewenvironment{multipleChoice@}[1][]{}{} % remove trailing ')'
    \let\inlinechoice\inlinechoicedefanswer
\fi
}

\newcommand{\setchoicepuntjes}{
\ifdefined\HCode
\else
    \renewenvironment{multipleChoice@}[1][]{}{} % remove trailing ')'
    \let\inlinechoice\inlinechoicepuntjes
\fi
}
\newcommand{\setchoicesquares}{
\ifdefined\HCode
\else
    \renewenvironment{multipleChoice@}[1][]{}{} % remove trailing ')'
    \let\inlinechoice\inlinechoicesquares
\fi
}
%
\newcommand{\setchoicesquareX}{
\ifdefined\HCode
\else
    \renewenvironment{multipleChoice@}[1][]{}{} % remove trailing ')'
    \let\inlinechoice\inlinechoicesquareX
\fi
}
%
\newcommand{\setchoicelist}{
\ifdefined\HCode
\else
    \renewenvironment{multipleChoice@}[1][]{}{)}% re-add trailing ')'
    \let\inlinechoice\inlinechoiceorig
\fi
}

\setchoicesquareX  % by default list-of-squares with onlineChoice in PDF

% Omdat multicols niet werkt in html: enkel in pdf  (in html zijn langere pagina's misschien ook minder storend)
\newenvironment{xmmulticols}[1][2]{
 \pdfOnly{\begin{multicols}{#1}}%
}{ \pdfOnly{\end{multicols}}}

%
% Te gebruiken in plaats van \section\subsection
%  (in een printstyle kan dan het level worden aangepast
%    naargelang \chapter vs \section style )
% 3/2021: DO NOT USE \xmsubsection !
\newcommand\xmsection\subsection
\newcommand\xmsubsection\subsubsection

% Aanpassen printversie
%  (hier gedefinieerd, zodat ze in xourse kunnen worden gezet/overschreven)
\providebool{parttoc}
\providebool{printpartfrontpage}
\providebool{printactivitytitle}
\providebool{printactivityqrcode}
\providebool{printactivityurl}
\providebool{printcontinuouspagenumbers}
\providebool{numberactivitiesbysubpart}
\providebool{addtitlenumber}
\providebool{addsectiontitlenumber}
\addtitlenumbertrue
\addsectiontitlenumbertrue

% The following three commands are hardcoded in xake, you can't create other commands like these, without adding them to xake as well
%  ( gebruikt in xourses om juiste soort titelpagina te krijgen voor verschillende ximera's )
\newcommand{\activitychapter}[2][]{
    {    
    \ifstrequal{#1}{notnumbered}{
        \addtitlenumberfalse
    }{}
    \typeout{ACTIVITYCHAPTER #2}   % logging
	\chapterstyle
	\activity{#2}
    }
}
\newcommand{\activitysection}[2][]{
    {
    \ifstrequal{#1}{notnumbered}{
        \addsectiontitlenumberfalse
    }{}
	\typeout{ACTIVITYSECTION #2}   % logging
	\sectionstyle
	\activity{#2}
    }
}
% Practices worden als activity getoond om de grote blokken te krijgen online
\newcommand{\practicesection}[2][]{
    {
    \ifstrequal{#1}{notnumbered}{
        \addsectiontitlenumberfalse
    }{}
    \typeout{PRACTICESECTION #2}   % logging
	\sectionstyle
	\activity{#2}
    }
}
\newcommand{\activitychapterlink}[3][]{
    {
    \ifstrequal{#1}{notnumbered}{
        \addtitlenumberfalse
    }{}
    \typeout{ACTIVITYCHAPTERLINK #3}   % logging
	\chapterstyle
	\activitylink[#1]{#2}{#3}
    }
}

\newcommand{\activitysectionlink}[3][]{
    {
    \ifstrequal{#1}{notnumbered}{
        \addsectiontitlenumberfalse
    }{}
    \typeout{ACTIVITYSECTIONLINK #3}   % logging
	\sectionstyle
	\activitylink[#1]{#2}{#3}
    }
}


% Commando om de printstyle toe te voegen in ximera's. Zorgt ervoor dat er geen problemen zijn als je de xourses compileert
% hack om onhandige relative paden in TeX te omzeilen
% should work both in xourse and ximera (pre-112022 only in ximera; thus obsoletes adhoc setup in xourses)
% loads global.sty if present (cfr global.css for online settings!)
% use global.sty to overwrite settings in printstyle.sty ...
\newcommand{\addPrintStyle}[1]{
\iftikzexport\else   % only in PDF
  \makeatletter
  \ifx\@onlypreamble\@notprerr\else   % ONLY if in tex-preamble   (and e.g. not when included from xourse)
    \typeout{Loading printstyle}   % logging
    \usepackage{#1/printstyle} % mag enkel geinclude worden als je die apart compileert
    \IfFileExists{#1/global.sty}{
        \typeout{Loading printstyle-folder #1/global.sty}   % logging
        \usepackage{#1/global}
        }{
        \typeout{Info: No extra #1/global.sty}   % logging
    }   % load global.sty if present
    \IfFileExists{global.sty}{
        \typeout{Loading local-folder global.sty (or TEXINPUTPATH..)}   % logging
        \usepackage{global}
    }{
        \typeout{Info: No folder/global.sty}   % logging
    }   % load global.sty if present
    \IfFileExists{\currfilebase.sty}
    {
        \typeout{Loading \currfilebase.sty}
        \input{\currfilebase.sty}
    }{
        \typeout{Info: No local \currfilebase.sty}
    }
    \fi
  \makeatother
\fi
}

%
%  
% references: Ximera heeft adhoc logica	 om online labels te doen werken over verschillende files heen
% met \hyperref kan de getoonde tekst toch worden opgegeven, in plaats van af te hangen van de label-text
\ifdefined\HCode
% Link to standard \labels, but give your own description
% Usage:  Volg \hyperref[my_very_verbose_label]{deze link} voor wat tijdverlies
%   (01/2020: Ximera-server aangepast om bij class reference-keeptext de link-text NIET te vervangen door de label-text !!!) 
\renewcommand{\hyperref}[2][]{\HCode{<a class="reference reference-keeptext" href="\##1">}#2\HCode{</a>}}
%
%  Link to specific targets  (not tested ?)
\renewcommand{\hypertarget}[1]{\HCode{<a class="ximera-label" id="#1"></a>}}
\renewcommand{\hyperlink}[2]{\HCode{<a class="reference reference-keeptext" href="\##1">}#2\HCode{</a>}}
\fi

% Mmm, quid English ... (for keyword #1 !) ?
\newcommand{\wikilink}[2]{
    \hyperlink{https://nl.wikipedia.org/wiki/#1}{#2}
    \pdfOnly{\footnote{See \url{https://nl.wikipedia.org/wiki/#1}}
    }
}

\renewcommand{\figurename}{Figuur}
\renewcommand{\tablename}{Tabel}

%
% Gedoe om verschillende versies van xourse/ximera te maken afhankelijk van settings
%
% default: versie met antwoorden
% handout: versie voor de studenten, zonder antwoorden/oplossingen
% full: met alles erop en eraan, dus geschikt voor auteurs en/of lesgevers  (bevat in de pdf ook de 'online-only' stukken!)
%
%
% verder kunnen ook opties/variabele worden gezet voor hints/auteurs/uitweidingen/ etc
%
% 'Full' versie
\newtoggle{showonline}
\ifdefined\HCode   % zet default showOnline
    \toggletrue{showonline} 
\else
    \togglefalse{showonline}
\fi

% Full versie   % deprecated: see infra
\newcommand{\printFull}{
    \hintstrue
    \handoutfalse
    \toggletrue{showonline} 
}

\ifdefined\shouldPrintFull   % deprecated: see infra
    \printFull
\fi



% Overschrijf onlineOnly  (zoals gedefinieerd in ximera.cls)
\ifhandout   % in handout: gebruik de oorspronkelijke ximera.cls implementatie  (is dit wel nodig/nuttig?)
\else
    \iftoggle{showonline}{%
        \ifdefined\HCode
          \RenewEnviron{onlineOnly}{\bgroup\BODY\egroup}   % showOnline, en we zijn  online, dus toon de tekst
        \else
          \RenewEnviron{onlineOnly}{\bgroup\color{red!50!black}\BODY\egroup}   % showOnline, maar we zijn toch niet online: kleur de tekst rood 
        \fi
    }{%
      \RenewEnviron{onlineOnly}{}  % geen showOnline
    }
\fi

% hack om na hoofding van definition/proposition/... als dan niet op een nieuwe lijn te starten
% soms is dat goed en mooi, en soms niet; en in HTML is het nu (2/2020) anders dan in pdf
% vandaar suggestie om 
%     \begin{definition}[Nieuw concept] \nl
% te gebruiken als je zeker een newline wil na de hoofdig en titel
% (in het bijzonder itemize zonder \nl is 'lelijk' ...)
\ifdefined\HCode
\newcommand{\nl}{}
\else
\newcommand{\nl}{\ \par} % newline (achter heading van definition etc.)
\fi


% \nl enkel in handoutmode (ihb voor \wordChoice, die dan typisch veeeel langer wordt)
\ifdefined\HCode
\providecommand{\handoutnl}{}
\else
\providecommand{\handoutnl}{%
\ifhandout%
  \nl%
\fi%
}
\fi

% Could potentially replace \pdfOnline/\begin{onlineOnly} : 
% Usage= \ifonline{Hallo surfer}{Hallo PDFlezer}
\providecommand{\ifonline}[2]%
{
\begin{onlineOnly}#1\end{onlineOnly}%
\pdfOnly{#2}
}%


%
% Maak optionele 'basic' en 'extended' versies van een activity
%  met environment basicOnly, basicSkip en extendedOnly
%
%  (
%   Dit werkt ENKEL in de PDF; de online versies tonen (minstens voorklopig) steeds 
%   het default geval met printbasicversion en printextendversion beide FALSE
%  )
%
\providebool{printbasicversion}
\providebool{printextendedversion}   % not properly implemented
\providebool{printfullversion}       % presumably print everything (debug/auteur)
%
% only set these in xourses, and BEFORE loading this preamble
%
%\newif\ifshowbasic     \showbasictrue        % use this line in xourse to show 'basic' sections
%\newif\ifshowextended  \showextendedtrue     % use this line in xourse to show 'extended' sections
%
%
%\ifbool{showbasic}
%      { \NewEnviron{basicOnly}{\BODY} }    % if yes: just print contents
%      { \NewEnviron{basicOnly}{}      }    % if no:  completely ignore contents
%
%\ifbool{showbasic}
%      { \NewEnviron{basicSkip}{}      }
%      { \NewEnviron{basicSkip}{\BODY} }
%

\ifbool{printextendedversion}
      { \NewEnviron{extendedOnly}{\BODY} }
      { \NewEnviron{extendedOnly}{}      }
      


\ifdefined\HCode    % in html: always print
      {\newenvironment*{basicOnly}{}{}}    % if yes: just print contents
      {\newenvironment*{basicSkip}{}{}}    % if yes: just print contents
\else

\ifbool{printbasicversion}
      {\newenvironment*{basicOnly}{}{}}    % if yes: just print contents
      {\NewEnviron{basicOnly}{}      }    % if no:  completely ignore contents

\ifbool{printbasicversion}
      {\NewEnviron{basicSkip}{}      }
      {\newenvironment*{basicSkip}{}{}}

\fi

\usepackage{float}
\usepackage[rightbars,color]{changebar}

% Full versie
\ifbool{printfullversion}{
    \hintstrue
    \handoutfalse
    \toggletrue{showonline}
    \printbasicversionfalse
    \cbcolor{red}
    \renewenvironment*{basicOnly}{\cbstart}{\cbend}
    \renewenvironment*{basicSkip}{\cbstart}{\cbend}
    \def\xmtoonprintopties{FULL}   % will be printed in footer
}
{}
      
%
% Evalueer \ifhints IN de environment
%  
%
%\RenewEnviron{hint}
%{
%\ifhandout
%\ifhints\else\setbox0\vbox\fi%everything in een emty box
%\bgroup 
%\stepcounter{hintLevel}
%\BODY
%\egroup\ignorespacesafterend
%\addtocounter{hintLevel}{-1}
%\else
%\ifhints
%\begin{trivlist}\item[\hskip \labelsep\small\slshape\bfseries Hint:\hspace{2ex}]
%\small\slshape
%\stepcounter{hintLevel}
%\BODY
%\end{trivlist}
%\addtocounter{hintLevel}{-1}
%\fi
%\fi
%}

% Onafhankelijk van \ifhandout ...? TO BE VERIFIED
\RenewEnviron{hint}
{
\ifhints
\begin{trivlist}\item[\hskip \labelsep\small\bfseries Hint:\hspace{2ex}]
\small%\slshape
\stepcounter{hintLevel}
\BODY
\end{trivlist}
\addtocounter{hintLevel}{-1}
\else
\iftikzexport   % anders worden de tikz tekeningen in hints niet gegenereerd ?
\setbox0\vbox\bgroup
\stepcounter{hintLevel}
\BODY
\egroup\ignorespacesafterend
\addtocounter{hintLevel}{-1}
\fi % ifhandout
\fi %ifhints
}

%
% \tab sets typewriter-tabs (e.g. to format questions)
% (Has no effect in HTML :-( ))
%
\usepackage{tabto}
\ifdefined\HCode
  \renewcommand{\tab}{\quad}    % otherwise dummy .png's are generated ...?
\fi


% Also redefined in  preamble to get correct styling 
% for tikz images for (\tikzexport)
%

\theoremstyle{definition} % Bold titels
\makeatletter
\let\proposition\relax
\let\c@proposition\relax
\let\endproposition\relax
\makeatother
\newtheorem{proposition}{Eigenschap}


%\instructornotesfalse

% logic with \ifhandoutin ximera.cls unclear;so overwrite ...
\makeatletter
\@ifundefined{ifinstructornotes}{%
  \newif\ifinstructornotes
  \instructornotesfalse
  \newenvironment{instructorNotes}{}{}
}{}
\makeatother
\ifinstructornotes
\else
\renewenvironment{instructorNotes}%
{%
    \setbox0\vbox\bgroup
}
{%
    \egroup
}
\fi

% \RedeclareMathOperator
% from https://tex.stackexchange.com/questions/175251/how-to-redefine-a-command-using-declaremathoperator
\makeatletter
\newcommand\RedeclareMathOperator{%
    \@ifstar{\def\rmo@s{m}\rmo@redeclare}{\def\rmo@s{o}\rmo@redeclare}%
}
% this is taken from \renew@command
\newcommand\rmo@redeclare[2]{%
    \begingroup \escapechar\m@ne\xdef\@gtempa{{\string#1}}\endgroup
    \expandafter\@ifundefined\@gtempa
    {\@latex@error{\noexpand#1undefined}\@ehc}%
    \relax
    \expandafter\rmo@declmathop\rmo@s{#1}{#2}}
% This is just \@declmathop without \@ifdefinable
\newcommand\rmo@declmathop[3]{%
    \DeclareRobustCommand{#2}{\qopname\newmcodes@#1{#3}}%
}
\@onlypreamble\RedeclareMathOperator
\makeatother


%
% Engelse vertaling, vooral in mathmode
%
% 1. Algemene procedure
%
\ifdefined\isEn
 \newcommand{\nlen}[2]{#2}
 \newcommand{\nlentext}[2]{\text{#2}}
 \newcommand{\nlentextbf}[2]{\textbf{#2}}
\else
 \newcommand{\nlen}[2]{#1}
 \newcommand{\nlentext}[2]{\text{#1}}
 \newcommand{\nlentextbf}[2]{\textbf{#1}}
\fi

%
% 2. Lijst van erg veel gebruikte uitdrukkingen
%

% Ja/Nee/Fout/Juits etc
%\newcommand{\TJa}{\nlentext{ Ja }{ and }}
%\newcommand{\TNee}{\nlentext{ Nee }{ No }}
%\newcommand{\TJuist}{\nlentext{ Juist }{ Correct }
%\newcommand{\TFout}{\nlentext{ Fout }{ Wrong }
\newcommand{\TWaar}{\nlentext{ Waar }{ True }}
\newcommand{\TOnwaar}{\nlentext{ Vals }{ False }}
% Korte bindwoorden en, of, dus, ...
\newcommand{\Ten}{\nlentext{ en }{ and }}
\newcommand{\Tof}{\nlentext{ of }{ or }}
\newcommand{\Tdus}{\nlentext{ dus }{ so }}
\newcommand{\Tendus}{\nlentext{ en dus }{ and thus }}
\newcommand{\Tvooralle}{\nlentext{ voor alle }{ for all }}
\newcommand{\Took}{\nlentext{ ook }{ also }}
\newcommand{\Tals}{\nlentext{ als }{ when }} %of if?
\newcommand{\Twant}{\nlentext{ want }{ as }}
\newcommand{\Tmaal}{\nlentext{ maal }{ times }}
\newcommand{\Toptellen}{\nlentext{ optellen }{ add }}
\newcommand{\Tde}{\nlentext{ de }{ the }}
\newcommand{\Thet}{\nlentext{ het }{ the }}
\newcommand{\Tis}{\nlentext{ is }{ is }} %zodat is in text staat in mathmode (geen italics)
\newcommand{\Tmet}{\nlentext{ met }{ where }} % in situaties e.g met p < n --> where p < n
\newcommand{\Tnooit}{\nlentext{ nooit }{ never }}
\newcommand{\Tmaar}{\nlentext{ maar }{ but }}
\newcommand{\Tniet}{\nlentext{ niet }{ not }}
\newcommand{\Tuit}{\nlentext{ uit }{ from }}
\newcommand{\Ttov}{\nlentext{ t.o.v. }{ w.r.t. }}
\newcommand{\Tzodat}{\nlentext{ zodat }{ such that }}
\newcommand{\Tdeth}{\nlentext{de }{th }}
\newcommand{\Tomdat}{\nlentext{omdat }{because }} 


%
% Overschrijf addhoc commando's
%
\ifdefined\isEn
\renewcommand{\pernot}{\overset{\mathrm{notation}}{=}}
\RedeclareMathOperator{\bld}{im}     % beeld
\RedeclareMathOperator{\graf}{graph}   % grafiek
\RedeclareMathOperator{\rico}{slope}   % richtingcoëfficient
\RedeclareMathOperator{\co}{co}       % coordinaat
\RedeclareMathOperator{\gr}{deg}       % graad

% Operators
\RedeclareMathOperator{\bgsin}{arcsin}
\RedeclareMathOperator{\bgcos}{arccos}
\RedeclareMathOperator{\bgtan}{arctan}
\RedeclareMathOperator{\bgcot}{arccot}
\RedeclareMathOperator{\bgsinh}{arcsinh}
\RedeclareMathOperator{\bgcosh}{arccosh}
\RedeclareMathOperator{\bgtanh}{arctanh}
\RedeclareMathOperator{\bgcoth}{arccoth}

\fi


% HACK: use 'oplossing' for 'explanation' ...
\let\explanation\relax
\let\endexplanation\relax
% \newenvironment{explanation}{\begin{oplossing}}{\end{oplossing}}
\newcounter{explanation}

\ifhandout%
    \NewEnviron{explanation}[1][toon]%
    {%
    \RenewEnviron{verbatim}{ \red{VERBATIM CONTENT MISSING IN THIS PDF}} %% \expandafter\verb|\BODY|}

    \ifthenelse{\equal{\detokenize{#1}}{\detokenize{toon}}}
    {
    \def\PH@Command{#1}% Use PH@Command to hold the content and be a target for "\expandafter" to expand once.

    \begin{trivlist}% Begin the trivlist to use formating of the "Feedback" label.
    \item[\hskip \labelsep\small\slshape\bfseries Explanation:% Format the "Feedback" label. Don't forget the space.
    %(\texttt{\detokenize\expandafter{\PH@Command}}):% Format (and detokenize) the condition for feedback to trigger
    \hspace{2ex}]\small%\slshape% Insert some space before the actual feedback given.
    \BODY
    \end{trivlist}
    }
    {  % \begin{feedback}[solution]   \BODY     \end{feedback}  }
    }
    }    
\else
% ONLY for HTML; xmoplossing is styled with css, and is not, and need not be a LaTeX environment
% THUS: it does NOT use feedback anymore ...
%    \NewEnviron{oplossing}{\begin{expandable}{xmoplossing}{\nlen{Toon uitwerking}{Show solution}}{\BODY}\end{expandable}}
    \newenvironment{explanation}[1][toon]
   {%
       \begin{expandable}{xmoplossing}{}
   }
   {%
   	   \end{expandable}
   } 
\fi

\author{Paul Zachlin \and Anna Davis} \title{Supplementary Exercises for Ch 4} \license{CC-BY 4.0}

\begin{document}

\begin{abstract}
\end{abstract}
\maketitle

\section*{Exercises for Ch 4 Matrices}

\begin{problem}\label{prb:4.1} For the following pairs of matrices, determine if the sum $A + B$ is defined. If so, find the sum.
\begin{enumerate}
\item
$A = \left[ \begin{array}{rr}
1 & 0 \\
0 & 1
\end{array} \right],
B = \left[ \begin{array}{rr}
0 & 1 \\
1 & 0
\end{array} \right]$

\item
$A = \left[ \begin{array}{rrr}
2 & 1 & 2 \\
1 & 1 & 0
\end{array} \right],  B = \left[ \begin{array}{rrr}
-1 & 0 & 3\\
0 & 1 & 4
\end{array} \right]$

\item
$A = \left[ \begin{array}{rr}
1 & 0 \\
-2 & 3 \\
4 & 2
\end{array} \right], B = \left[ \begin{array}{rrr}
2 & 7 & -1 \\
0 & 3 & 4
\end{array} \right]$
\end{enumerate}
%\begin{hint}
%\end{hint}
\end{problem}

\begin{problem}\label{prb:4.2} For each matrix $A$, find the matrix $-A$ such that $A + (-A) = 0$.
\begin{enumerate}
\item
$A = \left[ \begin{array}{rr}
1 & 2 \\
2 & 1
\end{array} \right]$

\item
$A = \left[ \begin{array}{rr}
-2 & 3 \\
0 & 2
\end{array} \right]$

\item
$A = \left[ \begin{array}{rrr}
0 & 1 & 2 \\
1 & -1 & 3 \\
4 & 2 & 0
\end{array} \right]$
\end{enumerate}
%\begin{hint}
%\end{hint}
\end{problem}

\begin{problem}\label{prb:4.3} In the context of Theorem~\ref{th:propertiesscalarmult}, describe $-A$ and the zero matrix.

\begin{hint}
To get $-A,$ just
replace every entry of $A$ with its additive inverse. The 0 matrix has all zeros in it.
\end{hint}
\end{problem}

\begin{problem}\label{prb:4.4} For each matrix $A$, find the product $(-2)A, 0A,$ and $3A$.
\begin{enumerate}
\item
$A = \left[ \begin{array}{rr}
1 & 2 \\
2 & 1
\end{array} \right]$

\item
$A = \left[ \begin{array}{rr}
-2 & 3 \\
0 & 2
\end{array} \right]$

\item
$A = \left[ \begin{array}{rrr}
0 & 1 & 2 \\
1 & -1 & 3 \\
4 & 2 & 0
\end{array} \right]$
\end{enumerate}
%\begin{hint}
%\end{hint}
\end{problem}

\begin{problem}\label{prb:4.5} \label{addinvrstunique} Using only the properties given in Theorem~\ref{th:propertiesofaddition}
 and Theorem~\ref{th:propertiesscalarmult},
show $-A$ is unique.

\begin{hint}
 Suppose $B$ also works. Then
\[
-A=-A+\left( A+B\right) =\left( -A+A\right) +B=0+B=B
\]
\end{hint}
\end{problem}

\begin{problem}\label{prb:4.6} Using only the properties given in Theorem~\ref{th:propertiesofaddition}
 and Theorem~\ref{th:propertiesscalarmult},
show $0$ is unique.

\begin{hint}
Suppose $0^{\prime }$ also works. Then $0^{\prime }=0^{\prime }+0=0.$
\end{hint}
\end{problem}

\begin{problem}\label{prb:4.7} Using only the properties given in Theorem~\ref{th:propertiesofaddition}
 and Theorem~\ref{th:propertiesscalarmult}, show $0A=0.$ Here
the $0$ on the left is the scalar $0$ and the $0$ on the right is the zero matrix of appropriate size.

\begin{hint}
$0A=\left( 0+0\right) A=0A+0A.$ Now add $-\left(
0A\right) $ to both sides. Then $0=0A$.
\end{hint}
\end{problem}

\begin{problem}\label{prb:4.8} Using only the properties given in Theorem~\ref{th:propertiesofaddition}
 and Theorem~\ref{th:propertiesscalarmult}, as well as previous
problems, show $\left( -1\right) A=-A.$

\begin{hint}
$A+\left( -1\right) A=\left( 1+\left(
-1\right) \right) A=0A=0.$ Therefore, from the uniqueness of the additive
inverse proved in the above Problem \ref{addinvrstunique}, it follows that $
-A=\left( -1\right) A$.
\end{hint}
\end{problem}

\begin{problem}\label{prb:4.9} Consider the matrices $
A =\left[
\begin{array}{rrr}
1 & 2 & 3 \\
2 & 1 & 7
\end{array}
\right],  B=\left[
\begin{array}{rrr}
3 & -1 & 2 \\
-3 & 2 & 1
\end{array}
\right],
C =\left[
\begin{array}{rr}
1 & 2 \\
3 & 1
\end{array}
\right], \\ D=\left[
\begin{array}{rr}
-1 & 2 \\
2 & -3
\end{array}
\right],  E=\left[
\begin{array}{r}
2 \\
3
\end{array}
\right]$.

Find the following if possible. If it is not possible explain why.
\begin{enumerate}
\item $-3A$

\begin{hint}
$\left[
\begin{array}{rrr}
-3 & -6 & -9 \\
-6 & -3 & -21
\end{array}
\right]$
\end{hint}

\item $3B-A$

\begin{hint}
$\left[
\begin{array}{rrr}
8 & -5 & 3 \\
-11 & 5 & -4
\end{array}
\right]$
\end{hint}

\item $AC$

\begin{hint}
Not possible
\end{hint}

\item $CB$

\begin{hint}
$\left[
\begin{array}{rrr}
-3 & 3 & 4 \\
6 & -1 & 7
\end{array}
\right]$
\end{hint}

\item $AE$

\begin{hint}
Not possible
\end{hint}

\item $EA$

\begin{hint}
Not possible
\end{hint}

\end{enumerate}
\end{problem}

\begin{problem}\label{prb:4.10} Consider the matrices $
A =\left[
\begin{array}{rr}
1 & 2 \\
3 & 2 \\
1 & -1
\end{array}
\right], B=\left[
\begin{array}{rrr}
2 & -5 & 2 \\
-3 & 2 & 1
\end{array}
\right] ,
C =\left[
\begin{array}{rr}
1 & 2 \\
5 & 0
\end{array}
\right], \\ D=\left[
\begin{array}{rr}
-1 & 1 \\
4 & -3
\end{array}
\right], E=\left[
\begin{array}{r}
1 \\
3
\end{array}
\right]$

Find the following if possible. If it is not possible explain
why.
\begin{enumerate}
\item $-3A$

\begin{hint}
$\left[
\begin{array}{rr}
-3 & -6 \\
-9 & -6 \\
-3 & 3
\end{array}
\right]$
\end{hint}

\item $3B-A$

\begin{hint}
Not possible
\end{hint}

\item $AC$

\begin{hint}
$\left[
\begin{array}{rr}
11 & 2 \\
13 & 6 \\
-4 & 2
\end{array}
\right]$
\end{hint}

\item $CA$

\begin{hint}
Not possible
\end{hint}

\item $AE$

\begin{hint}
$\left[
\begin{array}{r}
7 \\
9 \\
-2
\end{array}
\right]$
\end{hint}

\item $EA$

\begin{hint}
Not possible
\end{hint}

\item $BE$

\begin{hint}
Not possible
\end{hint}

\item $DE$

\begin{hint}
$\left[
\begin{array}{r}
2 \\
-5
\end{array}
\right]$
\end{hint}
\end{enumerate}
\end{problem}

\begin{problem}\label{prb:4.11} Let $A=\left[
\begin{array}{rr}
1 & 1 \\
-2 & -1 \\
1 & 2
\end{array}
\right] $, $B=\left[
\begin{array}{rrr}
1 & -1 & -2 \\
2 & 1 & -2
\end{array}
\right] ,$ and $C=\left[
\begin{array}{rrr}
1 & 1 & -3 \\
-1 & 2 & 0 \\
-3 & -1 & 0
\end{array}
\right] .$ Find the following if possible.

\begin{enumerate}
\item $AB$

\begin{hint}
$\left[
\begin{array}{rrr}
3 & 0 & -4 \\
-4 & 1 & 6 \\
5 & 1 & -6
\end{array}
\right] $
\end{hint}

\item $BA$

\begin{hint}
$\left[
\begin{array}{rr}
1 & -2 \\
-2 & -3
\end{array}
\right] $
\end{hint}

\item $AC$

\begin{hint}
Not possible
\end{hint}

\item $CA$

\begin{hint}
$\left[
\begin{array}{rr}
-4 & -6 \\
-5 & -3 \\
-1 & -2
\end{array}
\right] $
\end{hint}

\item $CB$

\begin{hint}
Not possible
\end{hint}

\item $BC$

\begin{hint}
$\left[
\begin{array}{rrr}
8 & 1 & -3 \\
7 & 6 & -6
\end{array}
\right] $
\end{hint}

\end{enumerate}
\end{problem}

\begin{problem}\label{prb:4.12} Let $A=\left[
\begin{array}{rr}
-1 & -1 \\
3 & 3
\end{array}
\right] $. Find all $2\times 2$ matrices, $B$
such that $AB=0.$
\begin{hint}
\begin{eqnarray*}
\left[
\begin{array}{rr}
-1 & -1 \\
3 & 3
\end{array}
\right] \left[
\begin{array}{cc}
x & y \\
z & w
\end{array}
\right]  &=&\left[
\begin{array}{cc}
-x-z & -w-y \\
3x+3z & 3w+3y
\end{array}
\right]  \\
&=&\left[
\begin{array}{cc}
0 & 0 \\
0 & 0
\end{array}
\right]
\end{eqnarray*}
Solution is: $ w=-y,x=-z $ so the
matrices are of the form $\left[
\begin{array}{rr}
x & y \\
-x & -y
\end{array}
\right].$
\end{hint}
\end{problem}


\begin{problem}\label{prb:4.13} Let $X=\left[
\begin{array}{rrr}
-1 & -1 & 1
\end{array}
\right] $ and $Y=\left[
\begin{array}{rrr}
0 & 1 & 2
\end{array}
\right] .$ Find $X^{T}Y$ and $XY^{T}$ if
possible.
\begin{hint}
$X^{T}Y = \left[ \begin{array}{rrr}
0 & -1 & -2 \\
0 & -1 & -2 \\
0 & 1 & 2
\end{array}
\right] , XY^{T} = 1$
\end{hint}
\end{problem}


\begin{problem}\label{prb:4.14} Let $A=\left[
\begin{array}{rr}
1 & 2 \\
3 & 4
\end{array}
\right] ,B=\left[
\begin{array}{rr}
1 & 2 \\
3 & k
\end{array}
\right] .$ Is it possible to choose $k$ such that $AB=BA?$ If so, what
should $k$ equal?

\begin{hint}
\begin{eqnarray*}
\left[
\begin{array}{cc}
1 & 2 \\
3 & 4
\end{array}
\right] \left[
\begin{array}{cc}
1 & 2 \\
3 & k
\end{array}
\right] &=& \left[
\begin{array}{cc}
7 & 2k+2 \\
15 & 4k+6
\end{array}
\right] \\
 \left[
\begin{array}{cc}
1 & 2 \\
3 & k
\end{array}
\right] \left[
\begin{array}{cc}
1 & 2 \\
3 & 4
\end{array}
\right] &=& \left[
\begin{array}{cc}
7 & 10 \\
3k+3 & 4k+6
\end{array}
\right]
\end{eqnarray*}
 Thus you must have $
\begin{array}{c}
3k+3=15 \\
2k+2=10
\end{array}
$.
\end{hint}

$$k=\answer{4}$$
\end{problem}

\begin{problem}\label{prb:4.15} Let $A=\left[
\begin{array}{rr}
1 & 2 \\
3 & 4
\end{array}
\right] ,B=\left[
\begin{array}{rr}
1 & 2 \\
1 & k
\end{array}
\right] .$ Is it possible to choose $k$ such that $AB=BA?$ If so, what
should $k$ equal?
\begin{hint}
\begin{eqnarray*}
\left[
\begin{array}{cc}
1 & 2 \\
3 & 4
\end{array}
\right] \left[
\begin{array}{cc}
1 & 2 \\
1 & k
\end{array}
\right] &=& \left[
\begin{array}{cc}
3 & 2k+2 \\
7 & 4k+6
\end{array}
\right] \\
\left[
\begin{array}{cc}
1 & 2 \\
1 & k
\end{array}
\right] \left[
\begin{array}{cc}
1 & 2 \\
3 & 4
\end{array}
\right] &=& \left[
\begin{array}{cc}
7 & 10 \\
3k+1 & 4k+2
\end{array}
\right]
\end{eqnarray*}
 However, $7\neq 3$ and so there is no possible choice of $k$ which
will make these matrices commute.
\end{hint}
\end{problem}

\begin{problem}\label{prb:4.16} Find $2\times 2$ matrices, $A$, $B,$ and $C$ such that $A\neq 0,C\neq B,$
but $AC=AB.$
\begin{hint}
Let $A = \left[
\begin{array}{rr}
1 & -1 \\
-1 & 1
\end{array}
\right], B = \left[
\begin{array}{cc}
1 & 1 \\
1 & 1
\end{array}
\right], C = \left[
\begin{array}{cc}
2 & 2 \\
2 & 2
\end{array}
\right]$.

\begin{eqnarray*}
\left[
\begin{array}{rr}
1 & -1 \\
-1 & 1
\end{array}
\right] \left[
\begin{array}{cc}
1 & 1 \\
1 & 1
\end{array}
\right]  &=& \left[
\begin{array}{cc}
0 & 0 \\
0 & 0
\end{array}
\right] \\
 \left[
\begin{array}{rr}
1 & -1 \\
-1 & 1
\end{array}
\right] \left[
\begin{array}{cc}
2 & 2 \\
2 & 2
\end{array}
\right] &=& \left[
\begin{array}{cc}
0 & 0 \\
0 & 0
\end{array}
\right]
\end{eqnarray*}
\end{hint}
\end{problem}

\begin{problem}\label{prb:4.17} Give an example of matrices (of any size), $A,B,C$ such that $B\neq C$, $A\neq 0,$
and yet $AB=AC.$
%\begin{hint}
%\end{hint}
\end{problem}

\begin{problem}\label{prb:4.18} Find $2 \times 2$ matrices $A$ and $B$ such that $A \neq 0$ and $B \neq 0$ but $AB = 0$.
\begin{hint}
Let $A = \left[
\begin{array}{rr}
1 & -1 \\
-1 & 1
\end{array}
\right], B = \left[
\begin{array}{cc}
1 & 1 \\
1 & 1
\end{array}
\right].$
\[
\left[
\begin{array}{rr}
1 & -1 \\
-1 & 1
\end{array}
\right] \left[
\begin{array}{cc}
1 & 1 \\
1 & 1
\end{array}
\right] = \left[
\begin{array}{cc}
0 & 0 \\
0 & 0
\end{array}
\right]
\]
\end{hint}
\end{problem}

\begin{problem}\label{prb:4.19} Give an example of matrices (of any size), $A,B$ such that $A \neq 0$ and $B \neq 0$ but  $AB=0.$
%\begin{hint}
%\end{hint}
\end{problem}

\begin{problem}\label{prb:4.20} Find $2 \times 2$ matrices $A$ and $B$ such that $A \neq 0$ and $B \neq 0$ with $AB \neq BA$.
\begin{hint}
Let $A = \left[
\begin{array}{cc}
0 & 1 \\
1 & 0
\end{array}
\right] , B = \left[
\begin{array}{cc}
1 & 2 \\
3 & 4
\end{array}
\right] $.
\begin{eqnarray*}
\left[
\begin{array}{cc}
0 & 1 \\
1 & 0
\end{array}
\right]
 \left[
\begin{array}{cc}
1 & 2 \\
3 & 4
\end{array}
\right]  &=&
 \left[
\begin{array}{cc}
3 & 4 \\
1 & 2
\end{array}
\right] \\
\left[
\begin{array}{cc}
1 & 2 \\
3 & 4
\end{array}
\right] \left[
\begin{array}{cc}
0 & 1 \\
1 & 0
\end{array}
\right]
&=& \left[
\begin{array}{cc}
2 & 1 \\
4 & 3
\end{array}
\right]
\end{eqnarray*}
\end{hint}
\end{problem}


\begin{problem}\label{prb:4.21} Write the system
\begin{equation*}
\begin{array}{c}
x_{1}-x_{2}+2x_{3} \\
2x_{3}+x_{1} \\
3x_{3} \\
3x_{4}+3x_{2}+x_{1}
\end{array}
\end{equation*}
 in the form $A\left[
\begin{array}{c}
x_{1} \\
x_{2} \\
x_{3} \\
x_{4}
\end{array}
\right] $ where $A$ is an appropriate matrix.

$$A=\left[
\begin{array}{rrrr}
\answer{1} & \answer{-1} & \answer{2} & \answer{0} \\
\answer{1} & \answer{0} & \answer{2} & \answer{0} \\
\answer{0} & \answer{0} & \answer{3} & \answer{0} \\
\answer{1} & \answer{3} & \answer{0} & \answer{3}
\end{array}
\right] $$

\end{problem}

\begin{problem}\label{prb:4.22} Write the system
\begin{equation*}
\begin{array}{c}
x_{1}+3x_{2}+2x_{3} \\
2x_{3}+x_{1} \\
6x_{3} \\
x_{4}+3x_{2}+x_{1}
\end{array}
\end{equation*}
 in the form $A\left[
\begin{array}{c}
x_{1} \\
x_{2} \\
x_{3} \\
x_{4}
\end{array}
\right] $ where $A$ is an appropriate matrix.
$$A=\left[
\begin{array}{rrrr}
\answer{1} & \answer{3} & \answer{2} & \answer{0} \\
\answer{1} & \answer{0} & \answer{2} & \answer{0} \\
\answer{0} & \answer{0} & \answer{6} & \answer{0} \\
\answer{1} & \answer{3} & \answer{0} & \answer{1}
\end{array}
\right] $$
\end{problem}

\begin{problem}\label{prb:4.23} Write the system
\begin{equation*}
\begin{array}{c}
x_{1}+x_{2}+x_{3} \\
2x_{3}+x_{1}+x_{2} \\
x_{3}-x_{1} \\
3x_{4}+x_{1}
\end{array}
\end{equation*}
 in the form $A\left[
\begin{array}{c}
x_{1} \\
x_{2} \\
x_{3} \\
x_{4}
\end{array}
\right] $ where $A$ is an appropriate matrix.
$$A=\left[
\begin{array}{rrrr}
\answer{1} & \answer{1} & \answer{1} & \answer{0} \\
\answer{1} & \answer{1} & \answer{2} & \answer{0} \\
\answer{-1} & \answer{0} & \answer{1} & \answer{0} \\
\answer{1} & \answer{0} & \answer{0} & \answer{3}
\end{array}
\right] $$
\end{problem}


\begin{problem}\label{prb:4.24} A matrix $A$ is called {\em idempotent \em}if $A^{2}=A.$
Let
\begin{equation*}
A=
\left[
\begin{array}{rrr}
2 & 0 & 2 \\
1 & 1 & 2 \\
-1 & 0 & -1
\end{array}
\right]
\end{equation*}
and show that $A$ is idempotent \index{idempotent}.
%\begin{hint}
%\end{hint}
\end{problem}

\begin{problem}\label{prb:4.25} For each pair of matrices, find the $(1,2)$-entry and $(2,3)$-entry of the product $AB$.
\begin{enumerate}
\item
$A = \left[ \begin{array}{rrr}
1 & 2 & -1 \\
3 & 4 & 0 \\
2 & 5 & 1
\end{array} \right], B = \left[ \begin{array}{rrr}
4 & 6 & -2 \\
7 & 2 & 1 \\
-1 & 0 & 0
\end{array} \right]$
\item
$A = \left[ \begin{array}{rrr}
1 & 3 & 1 \\
0 & 2 & 4 \\
1 & 0 & 5
\end{array} \right], B = \left[ \begin{array}{rrr}
2 & 3 & 0 \\
-4 & 16 & 1 \\
0 & 2 & 2
\end{array} \right]$
\end{enumerate}
%\begin{hint}
%\end{hint}
\end{problem}

\begin{problem}\label{prb:4.26}
 Suppose $A$ and $B$ are square matrices of the same size. Which of the
following are necessarily true?

\begin{enumerate}
\item $\left( A-B\right) ^{2}=A^{2}-2AB+B^{2}$ \
\wordChoice{\choice{Necessarily true}\choice[correct]{Not necessarily true}}

\item $\left( AB\right) ^{2}=A^{2}B^{2}$ \
\wordChoice{\choice{Necessarily true}\choice[correct]{Not necessarily true}}

\item $\left( A+B\right) ^{2}=A^{2}+2AB+B^{2}$ \
\wordChoice{\choice{Necessarily true}\choice[correct]{Not necessarily true}}

\item $\left( A+B\right) ^{2}=A^{2}+AB+BA+B^{2}$ \
\wordChoice{\choice[correct]{Necessarily true}\choice{Not necessarily true}}

\item $A^{2}B^{2}=A\left( AB\right) B$ \
\wordChoice{\choice[correct]{Necessarily true}\choice{Not necessarily true}}

\item $\left( A+B\right) ^{3}=A^{3}+3A^{2}B+3AB^{2}+B^{3}$ \
\wordChoice{\choice{Necessarily true}\choice[correct]{Not necessarily true}}

\item $\left( A+B\right) \left( A-B\right) =A^{2}-B^{2}$ \
\wordChoice{\choice{Necessarily true}\choice[correct]{Not necessarily true}}
\end{enumerate}
\end{problem}

\begin{problem}\label{prb:4.27} Consider the matrices 
$$
A =\left[
\begin{array}{rr}
1 & 2 \\
3 & 2 \\
1 & -1
\end{array}
\right], B=\left[
\begin{array}{rrr}
2 & -5 & 2 \\
-3 & 2 & 1
\end{array}
\right],
C =\left[
\begin{array}{rr}
1 & 2 \\
5 & 0
\end{array}
\right], \\ D=\left[
\begin{array}{rr}
-1 & 1 \\
4 & -3
\end{array}
\right], E=\left[
\begin{array}{r}
1 \\
3
\end{array}
\right]$$

Find the following if possible. If it is not possible explain why.
\begin{enumerate}
\item $-3A{^T}$
\item $3B - A^{T}$
\item $E^{T}B$
\item $EE^{T}$
\item $B^{T}B$
\item $CA^{T}$
\item $D^{T}BE$
\end{enumerate}

\begin{hint}
\begin{enumerate}
\item $\left[
\begin{array}{rrr}
-3 & -9 & -3 \\
-6 & -6 & 3
\end{array}
\right]$
\item $\left[
\begin{array}{rrr}
5 & -18 & 5 \\
-11 & 4 & 4
\end{array}
\right]$
\item $\left[
\begin{array}{rrr}
-7 & 1 & 5
\end{array}
\right]$
\item $\left[
\begin{array}{rr}
1 & 3 \\
3 & 9
\end{array}
\right]$
\item $\left[ \begin{array}{rrr}
13 & -16 & 1\\
-16 & 29 & -8 \\
1 & -8 & 5
\end{array}
\right]$
\item $\left[ \begin{array}{rrr}
5 & 7 & -1 \\
5 & 15 & 5
\end{array}
\right]$
\item Not possible.
\end{enumerate}
\end{hint}
\end{problem}

\begin{problem}\label{prb:4.28} Let $A$ be an $n\times n$ matrix. Show $A$ equals the sum of a
symmetric and a skew symmetric matrix.

\begin{hint}
Show that $\frac{1}{2}\left( A^{T}+A\right) $ is symmetric and then consider using this
as one of the matrices.

Click the arrow to see the answer.
\begin{expandable}
$A=\frac{A+A^{T}}{2}+\frac{A-A^{T}}{2}.$
\end{expandable}
\end{hint}
\end{problem}

\begin{problem}\label{prb:4.29} Show that the main diagonal of every skew symmetric matrix consists of only zeros. Recall that the main diagonal consists of every entry of the matrix which is of the form
$a_{ii}$.
\begin{hint}
If $A$ is skew-symmetric then $A=-A^{T}.$ It follows that $a_{ii}=-a_{ii}$ and so each $a_{ii}=0$.
\end{hint}
\end{problem}

\begin{problem}\label{prb:4.30} Show that for an $m \times n$ matrix $A$, an $n \times p$ matrix $B$, and scalars $r, s$, the following holds:
\[
\left( rA + sB \right) ^T = rA^{T} + sB^{T}
\]
%\begin{hint}
%\end{hint}
\end{problem}

\begin{problem}\label{prb:4.31} Prove that $I_{m}A=A$ where $A$ is an $m\times n$ matrix.
%\begin{hint}
% $\left(
%I_{m}A\right) _{ij}\equiv \sum_{j}\delta _{ik}A_{kj}=A_{ij}$
%\end{hint}
\end{problem}

\begin{problem}\label{prb:4.32} Suppose $AB=AC$ and $A$ is an invertible $n\times n$ matrix. Does it
follow that $B=C?$ Explain why or why not.
\begin{hint}
Yes $B=C$. Multiply $AB = AC$ on the left by $A^{-1}$.
\end{hint}
\end{problem}

\begin{problem}\label{prb:4.33} Suppose $AB=AC$ and $A$ is a non-invertible $n\times n$ matrix. Does it follow that $B=C$? Explain why or why not.
%\begin{hint}
%\end{hint}
\end{problem}

\begin{problem}\label{prb:4.34} Give an example of a matrix $A$ such that $A^{2}=I$ and yet $A\neq I$
and $A\neq -I.$
\begin{hint}
$A = \left[
\begin{array}{rrr}
1 & 0 & 0 \\
0 & -1 & 0 \\
0 & 0 & 1
\end{array}
\right] $
\end{hint}
\end{problem}

\begin{problem}\label{prb:4.35} Let
\begin{equation*}
A=\left[
\begin{array}{rr}
2 & 1 \\
-1 & 3
\end{array}
\right]
\end{equation*}
Find $A^{-1}$ if possible. If $A^{-1}$ does not exist, explain why.
\begin{hint}
$\left[
\begin{array}{rr}
2 & 1 \\
-1 & 3
\end{array}
\right]^{-1}= \left[
\begin{array}{rr}
\vspace{0.05in}\frac{3}{7} & -\vspace{0.05in}\frac{1}{7} \\
\vspace{0.05in}\frac{1}{7} & \vspace{0.05in}\frac{2}{7}
\end{array}
\right]$
\end{hint}
\end{problem}

\begin{problem}\label{prb:4.36}Let
\begin{equation*}
A=\left[
\begin{array}{rr}
0 & 1 \\
5 & 3
\end{array}
\right]
\end{equation*}
Find $A^{-1}$ if possible. If $A^{-1}$ does not exist, explain why.
\begin{hint}
$\left[
\begin{array}{cc}
0 & 1 \\
5 & 3
\end{array}
\right]^{-1}= \left[
\begin{array}{cc}
-\vspace{0.05in}\frac{3}{5} & \vspace{0.05in}\frac{1}{5} \\
1 & 0
\end{array}
\right]$
\end{hint}
\end{problem}

\begin{problem}\label{prb:4.37}Let
\begin{equation*}
A=\left[
\begin{array}{rr}
2 & 1 \\
3 & 0
\end{array}
\right]
\end{equation*}
Find $A^{-1}$ if possible. If $A^{-1}$ does not exist, explain why.
\begin{hint}
$\left[
\begin{array}{cc}
2 & 1 \\
3 & 0
\end{array}
\right]^{-1}= \left[
\begin{array}{cc}
0 & \vspace{0.05in}\frac{1}{3} \\
1 & -\vspace{0.05in}\frac{2}{3}
\end{array}
\right]$
\end{hint}
\end{problem}

\begin{problem}\label{prb:4.38}Let
\begin{equation*}
A=\left[
\begin{array}{rr}
2 & 1 \\
4 & 2
\end{array}
\right]
\end{equation*}
Find $A^{-1}$ if possible. If $A^{-1}$ does not exist, explain why.
\begin{hint}
$\left[
\begin{array}{cc}
2 & 1 \\
4 & 2
\end{array}
\right]^{-1}$ does not exist. The reduced row echelon form of this matrix
is $\left[
\begin{array}{cc}
1 & \vspace{0.05in}\frac{1}{2} \\
0 & 0
\end{array}
\right]$
\end{hint}
\end{problem}

\begin{problem}\label{prb:4.39}Let $A$ be a $2\times 2$ invertible matrix, with $A=\left[
\begin{array}{cc}
a & b \\
c & d
\end{array}
\right] .$ Find a formula for $A^{-1}$ in terms of $a,b,c,d$.
\begin{hint}
$\left[
\begin{array}{cc}
a & b \\
c & d
\end{array}
\right]^{-1}= \left[
\begin{array}{cc}
\frac{d}{ad-bc} & -\frac{b}{ad-bc} \\
-\frac{c}{ad-bc} & \frac{a}{ad-bc}
\end{array}
\right]$
\end{hint}
\end{problem}

\begin{problem}\label{prb:4.40}Let
\begin{equation*}
A=\left[
\begin{array}{rrr}
1 & 2 & 3 \\
2 & 1 & 4 \\
1 & 0 & 2
\end{array}
\right]
\end{equation*}
Find $A^{-1}$ if possible. If $A^{-1}$ does not exist, explain why.
\begin{hint}
$\left[
\begin{array}{ccc}
1 & 2 & 3 \\
2 & 1 & 4 \\
1 & 0 & 2
\end{array}
\right]^{-1}= \left[
\begin{array}{rrr}
-2 & 4 & -5 \\
0 & 1 & -2 \\
1 & -2 & 3
\end{array}
\right]$
\end{hint}
\end{problem}

\begin{problem}\label{prb:4.41}Let
\begin{equation*}
A=\left[
\begin{array}{rrr}
1 & 0 & 3 \\
2 & 3 & 4 \\
1 & 0 & 2
\end{array}
\right]
\end{equation*}
Find $A^{-1}$ if possible. If $A^{-1}$ does not exist, explain why.
\begin{hint}
$\left[
\begin{array}{ccc}
1 & 0 & 3 \\
2 & 3 & 4 \\
1 & 0 & 2
\end{array}
\right]^{-1}= \left[
\begin{array}{rrr}
-2 & 0 & 3 \\
0 & \vspace{0.05in}\frac{1}{3} & -\vspace{0.05in}\frac{2}{3} \\
1 & 0 & -1
\end{array}
\right]$
\end{hint}
\end{problem}

\begin{problem}\label{prb:4.42}Let
\begin{equation*}
A=\left[
\begin{array}{rrr}
1 & 2 & 3 \\
2 & 1 & 4 \\
4 & 5 & 10
\end{array}
\right]
\end{equation*}
Find $A^{-1}$ if possible. If $A^{-1}$ does not exist, explain why.
\begin{hint}
The reduced row echelon form is
$\left[
\begin{array}{ccc}
1 & 0 & \vspace{0.05in}\frac{5}{3} \\
0 & 1 & \vspace{0.05in}\frac{2}{3} \\
0 & 0 & 0
\end{array}
\right]$. There is no inverse.
\end{hint}
\end{problem}

\begin{problem}\label{prb:4.43}Let
\begin{equation*}
A=\left[
\begin{array}{rrrr}
1 & 2 & 0 & 2 \\
1 & 1 & 2 & 0 \\
2 & 1 & -3 & 2 \\
1 & 2 & 1 & 2
\end{array}
\right]
\end{equation*}
Find $A^{-1}$ if possible. If $A^{-1}$ does not exist, explain why.
\begin{hint}
$\left[
\begin{array}{rrrr}
1 & 2 & 0 & 2 \\
1 & 1 & 2 & 0 \\
2 & 1 & -3 & 2 \\
1 & 2 & 1 & 2
\end{array}
\right]^{-1}= \left[
\begin{array}{rrrr}
-1 & \vspace{0.05in}\frac{1}{2} &  \vspace{0.05in}\frac{1}{2} &  \vspace{0.05in}\frac{1}{2} \\
3 &  \vspace{0.05in}\frac{1}{2} & - \vspace{0.05in}\frac{1}{2} & - \vspace{0.05in}\frac{5}{2} \\
-1 & 0 & 0 & 1 \\
-2 & - \vspace{0.05in}\frac{3}{4} &  \vspace{0.05in}\frac{1}{4} &  \vspace{0.05in}\frac{9}{4}
\end{array}
\right]$
\end{hint}
\end{problem}

\begin{problem}\label{prb:4.44}Using the inverse of the matrix, find the solution to the systems:
\begin{enumerate}
\item
\begin{equation*}
\left[
\begin{array}{rr}
2 & 4  \\
1 & 1
\end{array}
\right]
\left[
\begin{array}{c}
x \\
y
\end{array}
\right] =\left[
\begin{array}{r}
1 \\
2
\end{array}
\right]
\end{equation*}

\item
\begin{equation*}
\left[
\begin{array}{rr}
2 & 4 \\
1 & 1
\end{array}
\right] \left[
\begin{array}{c}
x \\
y
\end{array}
\right] =\left[
\begin{array}{r}
2 \\
0
\end{array}
\right]
\end{equation*}
\end{enumerate}

\item Now give the solution in terms of $a$ and $b$ to
\[
\left[
\begin{array}{rr}
2 & 4 \\
1 & 1
\end{array} \right]
\left[
\begin{array}{c}
x \\
y
\end{array}\right]
=
\left[
\begin{array}{c}
a \\
b
\end{array} \right]
\]
%\begin{hint}
%\end{hint}
\end{problem}

\begin{problem}\label{prb:4.45}Using the inverse of the matrix, find the solution to the systems:

\begin{enumerate}
\item
\begin{equation*}
\left[
\begin{array}{rrr}
1 & 0 & 3 \\
2 & 3 & 4 \\
1 & 0 & 2
\end{array}
\right] \left[
\begin{array}{c}
x \\
y \\
z
\end{array}
\right] =\left[
\begin{array}{r}
1 \\
0 \\
1
\end{array}
\right]
\end{equation*}
\begin{hint}
$\left[
\begin{array}{c}
x \\
y \\
z
\end{array}
\right] =\left[
\begin{array}{c}
1 \\
-\vspace{0.05in}\frac{2}{3} \\
0
\end{array}
\right]$
\end{hint}

\item
\begin{equation*}
\left[
\begin{array}{rrr}
1 & 0 & 3 \\
2 & 3 & 4 \\
1 & 0 & 2
\end{array}
\right] \left[
\begin{array}{c}
x \\
y \\
z
\end{array}
\right] =\left[
\begin{array}{r}
3 \\
-1 \\
-2
\end{array}
\right]
\end{equation*}

\begin{hint}
$\left[
\begin{array}{c}
x \\
y \\
z
\end{array}
\right] = \left[
\begin{array}{r}
-12 \\
1 \\
5
\end{array}
\right]$
\end{hint}
\end{enumerate}

\item Now give the solution in terms of $a,b,$ and $c$ to the following:
\begin{equation*}
\left[
\begin{array}{rrr}
1 & 0 & 3 \\
2 & 3 & 4 \\
1 & 0 & 2
\end{array}
\right] \left[
\begin{array}{c}
x \\
y \\
z
\end{array}
\right] =\left[
\begin{array}{c}
a \\
b \\
c
\end{array}
\right]
\end{equation*}
\begin{hint}
$\left[
\begin{array}{c}
x \\
y \\
z
\end{array}
\right] =
\left[
\begin{array}{c}
3c-2a \\
\frac{1}{3}b-\frac{2}{3}c \\
a-c
\end{array}
\right]$
\end{hint}
\end{problem}

\begin{problem}\label{prb:4.46}Show that if $A$ is an $n\times n$ invertible matrix and $X$
is a $n\times 1$ matrix such that $AX=B$ for $B$ an
$n\times 1$ matrix, then $X=A^{-1}B$.
\begin{hint}
Multiply both sides of $AX=B$ on the left by $A^{-1}$.
\end{hint}
\end{problem}

\begin{problem}\label{prb:4.47}Prove that if $A^{-1}$ exists and $AX=0$ then $X=0$.
\begin{hint}
Multiply on both sides on the left by $A^{-1}.$ Thus
\[
0=A^{-1}0=A^{-1}\left( AX\right) =\left(
A^{-1}A\right) X=IX = X
\]
\end{hint}
\end{problem}

\begin{problem}\label{prb:4.48}\label{exerinverseprod}Show that if $A^{-1}$ exists for an $n\times n$
matrix, then it is unique. That is, if $BA=I$ and $AB=I,$ then $B=A^{-1}.$
\begin{hint}
 $A^{-1}=A^{-1}I=A^{-1}\left( AB\right) =\left( A^{-1}A\right) B=IB=B.$
\end{hint}
\end{problem}

\begin{problem}\label{prb:4.49}Show that if $A$ is an invertible $n\times n$ matrix, then so is
$A^{T} $ and $\left( A^{T}\right) ^{-1}=\left( A^{-1}\right) ^{T}.$
\begin{hint}
 You need to show that $\left( A^{-1}\right) ^{T}$ acts like the inverse of $A^{T}
$ because from uniqueness in the above problem, this will imply it is the
inverse. From properties of the transpose,
\begin{eqnarray*}
A^{T}\left( A^{-1}\right) ^{T} &=&\left( A^{-1}A\right) ^{T}=I^{T}=I \\
\left( A^{-1}\right) ^{T}A^{T} &=&\left( AA^{-1}\right) ^{T}=I^{T}=I
\end{eqnarray*}
Hence $\left( A^{-1}\right) ^{T}=\left( A^{T}\right) ^{-1}$ and this last
matrix exists.
\end{hint}
\end{problem}

\begin{problem}\label{prb:4.50}Show $\left( AB\right) ^{-1}=B^{-1}A^{-1}$ by verifying that
\begin{equation*}
AB\left(
B^{-1}A^{-1}\right) =I
\end{equation*} and
\begin{equation*}
B^{-1}A^{-1}\left( AB\right) =I
\end{equation*}

\begin{hint}
$\left( AB\right)
B^{-1}A^{-1}=A\left( BB^{-1}\right) A^{-1}=AA^{-1}=I$ $B^{-1}A^{-1}\left(
AB\right) =B^{-1}\left( A^{-1}A\right) B=B^{-1}IB=B^{-1}B=I$
\end{hint}
\end{problem}

\begin{problem}\label{prb:4.51}Show that $\left( ABC\right) ^{-1}=C^{-1}B^{-1}A^{-1}$ by verifying
that
\[
\left( ABC\right) \left( C^{-1}B^{-1}A^{-1}\right) =I
\]
and
\[\left( C^{-1}B^{-1}A^{-1}\right)\left( ABC\right) =I
\]

\begin{hint}
The proof of this exercise follows from the previous one.
\end{hint}
\end{problem}

\begin{problem}\label{prb:4.52}If $A$ is invertible, show $\left( A^{2}\right) ^{-1}=\left(
A^{-1}\right) ^{2}.$ 

\begin{hint}
$A^{2}\left( A^{-1}\right) ^{2}=AAA^{-1}A^{-1}=AIA^{-1}=AA^{-1}=I$ $\left(
A^{-1}\right) ^{2}A^{2}=A^{-1}A^{-1}AA=A^{-1}IA=A^{-1}A=I$
\end{hint}
\end{problem}

\begin{problem}\label{prb:4.53}If $A$ is invertible, show $\left( A^{-1}\right) ^{-1}=A.$

\begin{hint}
 $A^{-1}A=AA^{-1}=I$ and so by
uniqueness, $\left( A^{-1}\right) ^{-1}=A$.
\end{hint}
\end{problem}

\begin{problem}\label{prb:4.54}
Let $A = \left[ \begin{array}{rr}
2 & 3 \\
1 & 2
\end{array}\right]$. Suppose a row operation is applied to $A$ and the result is $B = \left[ \begin{array}{rr}
1 & 2 \\
2 & 3
\end{array}\right]$. Find the elementary matrix $E$ that represents this row operation.
%\begin{hint}
%\end{hint}
\end{problem}

\begin{problem}\label{prb:4.55}
Let $A = \left[ \begin{array}{rr}
4 & 0 \\
2 & 1
\end{array}\right]$. Suppose a row operation is applied to $A$ and the result is $B = \left[ \begin{array}{rr}
8 & 0 \\
2 & 1
\end{array}\right]$. Find the elementary matrix $E$ that represents this row operation.
%\begin{hint}
%\end{hint}
\end{problem}

\begin{problem}\label{prb:4.56}
Let $A = \left[ \begin{array}{rr}
1 & -3 \\
0 & 5
\end{array}\right]$. Suppose a row operation is applied to $A$ and the result is $B = \left[ \begin{array}{rr}
1 & -3 \\
2 & -1
\end{array}\right]$. Find the elementary matrix $E$ that represents this row operation.
%\begin{hint}
%\end{hint}
\end{problem}

\begin{problem}\label{prb:4.57}
Let $A = \left[ \begin{array}{rrr}
1 & 2 & 1  \\
0 & 5 & 1 \\
2 & -1 & 4
\end{array}\right]$. Suppose a row operation is applied to $A$ and the result is $B = \left[ \begin{array}{rrr}
1 & 2 & 1\\
2 & -1 & 4 \\
0 & 5 & 1
\end{array}\right]$.
\begin{enumerate}
\item Find the elementary matrix $E$ such that $EA = B$.

\item Find the inverse of $E$, $E^{-1}$, such that $E^{-1}B = A$.
\end{enumerate}
%\begin{hint}
%\end{hint}
\end{problem}

\begin{problem}\label{prb:4.58}
Let $A = \left[ \begin{array}{rrr}
1 & 2 & 1  \\
0 & 5 & 1 \\
2 & -1 & 4
\end{array}\right]$. Suppose a row operation is applied to $A$ and the result is $B = \left[ \begin{array}{rrr}
1 & 2 & 1\\
0 & 10 & 2 \\
2 & -1 & 4
\end{array}\right]$.
\begin{enumerate}
\item Find the elementary matrix $E$ such that $EA = B$.

\item Find the inverse of $E$, $E^{-1}$, such that $E^{-1}B = A$.
\end{enumerate}
%\begin{hint}
%\end{hint}
\end{problem}


\begin{problem}\label{prb:4.59}
Let $A = \left[ \begin{array}{rrr}
1 & 2 & 1  \\
0 & 5 & 1 \\
2 & -1 & 4
\end{array}\right]$. Suppose a row operation is applied to $A$ and the result is $B = \left[ \begin{array}{rrr}
1 & 2 & 1\\
0 & 5 & 1 \\
1 & -\vspace{0.05in}\frac{1}{2} & 2
\end{array}\right]$.
\begin{enumerate}
\item Find the elementary matrix $E$ such that $EA = B$.

\item Find the inverse of $E$, $E^{-1}$, such that $E^{-1}B = A$.
\end{enumerate}
%\begin{hint}
%\end{hint}
\end{problem}


\begin{problem}\label{prb:4.60}
Let $A = \left[ \begin{array}{rrr}
1 & 2 & 1  \\
0 & 5 & 1 \\
2 & -1 & 4
\end{array}\right]$. Suppose a row operation is applied to $A$ and the result is $B = \left[ \begin{array}{rrr}
1 & 2 & 1\\
2 & 4 & 5 \\
2 & -1 & 4
\end{array}\right]$.
\begin{enumerate}
\item Find the elementary matrix $E$ such that $EA = B$.

\item Find the inverse of $E$, $E^{-1}$, such that $E^{-1}B = A$.
\end{enumerate}
%\begin{hint}
%\end{hint}
\end{problem}

\begin{problem}\label{prb:4.61} Find an $LU$ factorization of $\left[
\begin{array}{rrr}
1 & 2 & 0 \\
2 & 1 & 3 \\
1 & 2 & 3
\end{array}
\right] .$
\begin{hint}
\[
\left[
\begin{array}{ccc}
1 & 2 & 0 \\
2 & 1 & 3 \\
1 & 2 & 3
\end{array}
\right] = \left[
\begin{array}{ccc}
1 & 0 & 0 \\
2 & 1 & 0 \\
1 & 0 & 1
\end{array}
\right] \left[
\begin{array}{rrr}
1 & 2 & 0 \\
0 & -3 & 3 \\
0 & 0 & 3
\end{array}
\right]
\]

\end{hint}
\end{problem}

\begin{problem}\label{prb:4.62} Find an $LU$ factorization of $\left[
\begin{array}{rrrr}
1 & 2 & 3 & 2 \\
1 & 3 & 2 & 1 \\
5 & 0 & 1 & 3
\end{array}
\right] .$
\begin{hint}
\[
\left[
\begin{array}{cccc}
1 & 2 & 3 & 2 \\
1 & 3 & 2 & 1 \\
5 & 0 & 1 & 3
\end{array}
\right] = \left[
\begin{array}{rrr}
1 & 0 & 0 \\
1 & 1 & 0 \\
5 & -10 & 1
\end{array}
\right] \left[
\begin{array}{rrrr}
1 & 2 & 3 & 2 \\
0 & 1 & -1 & -1 \\
0 & 0 & -24 & -17
\end{array}
\right]
\]
\end{hint}
\end{problem}


\begin{problem}\label{prb:4.63} Find an $LU$ factorization of the matrix $\left[
\begin{array}{rrrr}
1 & -2 & -5 & 0 \\
-2 & 5 & 11 & 3 \\
3 & -6 & -15 & 1
\end{array}
\right] .$
\begin{hint}
\[
\left[
\begin{array}{rrrr}
1 & -2 & -5 & 0 \\
-2 & 5 & 11 & 3 \\
3 & -6 & -15 & 1
\end{array}
\right] = \left[
\begin{array}{rrr}
1 & 0 & 0 \\
-2 & 1 & 0 \\
3 & 0 & 1
\end{array}
\right] \left[
\begin{array}{rrrr}
1 & -2 & -5 & 0 \\
0 & 1 & 1 & 3 \\
0 & 0 & 0 & 1
\end{array}
\right]
\]
\end{hint}
\end{problem}

\begin{problem}\label{prb:4.64} Find an $LU$ factorization of the matrix $\left[
\begin{array}{rrrr}
1 & -1 & -3 & -1 \\
-1 & 2 & 4 & 3 \\
2 & -3 & -7 & -3
\end{array} \right] .$
\begin{hint}
\[
\left[
\begin{array}{rrrr}
1 & -1 & -3 & -1 \\
-1 & 2 & 4 & 3 \\
2 & -3 & -7 & -3
\end{array} \right] = \left[
\begin{array}{rrr}
1 & 0 & 0 \\
-1 & 1 & 0 \\
2 & -1 & 1
\end{array}
\right] \left[
\begin{array}{rrrr}
1 & -1 & -3 & -1 \\
0 & 1 & 1 & 2 \\
0 & 0 & 0 & 1
\end{array}
\right]
\]
\end{hint}
\end{problem}

\begin{problem}\label{prb:4.65} Find an $LU$ factorization of the matrix $ 
\left[
\begin{array}{rrrr}
1 & -3 & -4 & -3 \\
-3 & 10 & 10 & 10 \\
1 & -6 & 2 & -5
\end{array}
 \right] .$
\begin{hint}
\[
 \left[
\begin{array}{rrrr}
1 & -3 & -4 & -3 \\
-3 & 10 & 10 & 10 \\
1 & -6 & 2 & -5
\end{array} \right] = \left[
\begin{array}{rrr}
1 & 0 & 0 \\
-3 & 1 & 0 \\
1 & -3 & 1
\end{array}
\right] \left[
\begin{array}{rrrr}
1 & -3 & -4 & -3 \\
0 & 1 & -2 & 1 \\
0 & 0 & 0 & 1
\end{array}
\right]
\]
\end{hint}
\end{problem}


\begin{problem}\label{prb:4.66} Find an $LU$ factorization of the matrix
$\left[
\begin{array}{rrrr}
1 & 3 & 1 & -1 \\
3 & 10 & 8 & -1 \\
2 & 5 & -3 & -3
\end{array}
 \right] .$
\begin{hint}
\[
\left[
\begin{array}{rrrr}
1 & 3 & 1 & -1 \\
3 & 10 & 8 & -1 \\
2 & 5 & -3 & -3
\end{array}
 \right] = \left[
\begin{array}{rrr}
1 & 0 & 0 \\
3 & 1 & 0 \\
2 & -1 & 1
\end{array}
\right] \left[
\begin{array}{rrrr}
1 & 3 & 1 & -1 \\
0 & 1 & 5 & 2 \\
0 & 0 & 0 & 1
\end{array}
\right]
\]
\end{hint}
\end{problem}

\begin{problem}\label{prb:4.67} Find an $LU$ factorization of the matrix $\left[
\begin{array}{rrr}
3 & -2 & 1 \\
9 & -8 & 6 \\
-6 & 2 & 2 \\
3 & 2 & -7
\end{array}
\right] .$
\begin{hint}
\[
\left[
\begin{array}{rrr}
3 & -2 & 1 \\
9 & -8 & 6 \\
-6 & 2 & 2 \\
3 & 2 & -7
\end{array}
\right] = \left[
\begin{array}{rrrr}
1 & 0 & 0 & 0 \\
3 & 1 & 0 & 0 \\
-2 & 1 & 1 & 0 \\
1 & -2 & -2 & 1
\end{array}
\right]  \left[
\begin{array}{rrr}
3 & -2 & 1 \\
0 & -2 & 3 \\
0 & 0 & 1 \\
0 & 0 & 0
\end{array}
\right]
\]
\end{hint}
\end{problem}


\begin{problem}\label{prb:4.68} Find an $LU$ factorization of the matrix $\left[
\begin{array}{rrr}
-3 & -1 & 3 \\
9 & 9 & -12 \\
3 & 19 & -16 \\
12 & 40 & -26
\end{array}
\right] .$
%\begin{hint}
%\end{hint}
\end{problem}

\begin{problem}\label{prb:4.69} Find an $LU$ factorization of the matrix $\left[
\begin{array}{rrr}
-1 & -3 & -1 \\
1 & 3 & 0 \\
3 & 9 & 0 \\
4 & 12 & 16
\end{array}
\right] .$
\begin{hint}
\[
\left[
\begin{array}{rrr}
-1 & -3 & -1 \\
1 & 3 & 0 \\
3 & 9 & 0 \\
4 & 12 & 16
\end{array}
\right] = \left[
\begin{array}{rrrr}
1 & 0 & 0 & 0 \\
-1 & 1 & 0 & 0 \\
-3 & 0 & 1 & 0 \\
-4 & 0 & -4 & 1
\end{array}
\right] \left[
\begin{array}{rrr}
-1 & -3 & -1 \\
0 & 0 & -1 \\
0 & 0 & -3 \\
0 & 0 & 0
\end{array}
\right]
\]
\end{hint}
\end{problem}

\begin{problem}\label{prb:4.70} Find the $LU$ factorization of the coefficient matrix and use it to solve the system of equations.
\begin{equation*}
\begin{array}{c}
x+2y=5 \\
2x+3y=6
\end{array}
\end{equation*}
\begin{hint}
An $LU$ factorization of the coefficient matrix is
\[
\left[
\begin{array}{cc}
1 & 2 \\
2 & 3
\end{array}
\right] =  \left[
\begin{array}{cc}
1 & 0 \\
2 & 1
\end{array}
\right] \left[
\begin{array}{cc}
1 & 2 \\
0 & -1
\end{array}
\right]
\]
First solve
\[
\left[
\begin{array}{cc}
1 & 0 \\
2 & 1
\end{array}
\right] \left[
\begin{array}{c}
u \\
v
\end{array}
\right] =\left[
\begin{array}{c}
5 \\
6
\end{array}
\right]
\]
which gives $\left[
\begin{array}{c}
u \\
v
\end{array}
\right] =$ $\left[
\begin{array}{r}
5 \\
-4
\end{array}
\right] .$ Then solve
\[
\left[
\begin{array}{rr}
1 & 2 \\
0 & -1
\end{array}
\right] \left[
\begin{array}{c}
x \\
y
\end{array}
\right] =\left[
\begin{array}{r}
5 \\
-4
\end{array}
\right]
\]
which says that $y=4$ and $x=-3.$
\end{hint}
\end{problem}

\begin{problem}\label{prb:4.71} Find the $LU$ factorization of the coefficient matrix and use it to solve the system of equations.
\begin{equation*}
\begin{array}{c}
x+2y+z=1 \\
y+3z=2 \\
2x+3y=6
\end{array}
\end{equation*}
\begin{hint}
An $LU$ factorization of the coefficient matrix is
\[
\left[
\begin{array}{rrr}
1 & 2 & 1 \\
0 & 1 & 3 \\
2 & 3 & 0
\end{array}
\right] = \left[
\begin{array}{rrr}
1 & 0 & 0 \\
0 & 1 & 0 \\
2 & -1 & 1
\end{array}
\right] \left[
\begin{array}{rrr}
1 & 2 & 1 \\
0 & 1 & 3 \\
0 & 0 & 1
\end{array}
\right]
\]
First solve
\[
 \left[
\begin{array}{rrr}
1 & 0 & 0 \\
0 & 1 & 0 \\
2 & -1 & 1
\end{array}
\right] \left[
\begin{array}{c}
u \\
v \\
w
\end{array}
\right] =\left[
\begin{array}{c}
1 \\
2 \\
6
\end{array}
\right]
\]
which yields $u=1,v=2,w=6$. Next solve
\[
\left[
\begin{array}{rrr}
1 & 2 & 1 \\
0 & 1 & 3 \\
0 & 0 & 1
\end{array}
\right] \left[
\begin{array}{c}
x \\
y \\
z
\end{array}
\right] =\left[
\begin{array}{c}
1 \\
2 \\
6
\end{array}
\right]
\]
This yields $z=6,y=-16,x=27.$
\end{hint}
\end{problem}

\begin{problem}\label{prb:4.72} Find the $LU$ factorization of the coefficient matrix and use it to solve the system of equations.
\begin{equation*}
\begin{array}{c}
x+2y+3z=5 \\
2x+3y+z=6 \\
x-y+z=2
\end{array}
\end{equation*}
%\begin{hint}
%\end{hint}
\end{problem}

\begin{problem}\label{prb:4.73} Find the $LU$ factorization of the coefficient matrix and use it to solve the system of equations.
\begin{equation*}
\begin{array}{c}
x+2y+3z=5 \\
2x+3y+z=6 \\
3x+5y+4z=11
\end{array}
\end{equation*}
\begin{hint}
An $LU$ factorization of the coefficient matrix is
\[
\left[
\begin{array}{rrr}
1 & 2 & 3 \\
2 & 3 & 1 \\
3 & 5 & 4
\end{array}
\right] = \left[
\begin{array}{rrr}
1 & 0 & 0 \\
2 & 1 & 0 \\
3 & 1 & 1
\end{array}
\right] \left[
\begin{array}{rrr}
1 & 2 & 3 \\
0 & -1 & -5 \\
0 & 0 & 0
\end{array}
\right]
\]
First solve
\[
 \left[
\begin{array}{rrr}
1 & 0 & 0 \\
2 & 1 & 0 \\
3 & 1 & 1
\end{array}
\right] \left[
\begin{array}{c}
u \\
v \\
w
\end{array}
\right] =\left[
\begin{array}{c}
5 \\
6 \\
11
\end{array}
\right]
\]
Solution is: $\left[
\begin{array}{c}
u \\
v \\
w
\end{array}
\right] =$ $\left[
\begin{array}{c}
5 \\
-4 \\
0
\end{array}
\right] .$ Next solve
\[
\left[
\begin{array}{rrr}
1 & 2 & 3 \\
0 & -1 & -5 \\
0 & 0 & 0
\end{array}
\right] \left[
\begin{array}{c}
x \\
y \\
z
\end{array}
\right] =\left[
\begin{array}{c}
5 \\
-4 \\
0
\end{array}
\right]
\]
Solution is: $\left[
\begin{array}{c}
x \\
y \\
z
\end{array}
\right] =\left[
\begin{array}{c}
7t-3 \\
4-5t \\
t
\end{array}
\right] ,t\in \mathbb{R}$.
\end{hint}
\end{problem}

\begin{problem}\label{prb:4.74} Is there only one $LU$ factorization for a given matrix? 
\begin{hint}
Consider the equation
\begin{equation*}
\left[
\begin{array}{rr}
0 & 1 \\
0 & 1
\end{array}
\right] =\left[
\begin{array}{rr}
1 & 0 \\
1 & 1
\end{array}
\right] \left[
\begin{array}{rr}
0 & 1 \\
0 & 0
\end{array}
\right] .
\end{equation*}
Look for all possible $LU$ factorizations.

Click the arrow for the answer.
\begin{expandable}
Sometimes there is more than one $LU$ factorization as is the case in this
example. The given equation clearly gives an $LU$ factorization. However, it
appears that the following equation gives another $LU$ factorization.
\[
\left[
\begin{array}{cc}
0 & 1 \\
0 & 1
\end{array}
\right] =\left[
\begin{array}{cc}
1 & 0 \\
0 & 1
\end{array}
\right] \left[
\begin{array}{cc}
0 & 1 \\
0 & 1
\end{array}
\right]
\]
\end{expandable}
\end{hint}
\end{problem}




\section*{Practice Problem Source}
These problems come from Chapter 2 of Ken Kuttler's \href{https://open.umn.edu/opentextbooks/textbooks/a-first-course-in-linear-algebra-2017}{\it A First Course in Linear Algebra}. (CC-BY)

Ken Kuttler, {\it  A First Course in Linear Algebra}, Lyryx 2017, Open Edition, pp. 90--98, 104--106.  

\end{document}