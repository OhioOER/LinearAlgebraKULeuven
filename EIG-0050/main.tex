\documentclass{ximera}
%%% Begin Laad packages

\makeatletter
\@ifclassloaded{xourse}{%
    \typeout{Start loading preamble.tex (in a XOURSE)}%
    \def\isXourse{true}   % automatically defined; pre 112022 it had to be set 'manually' in a xourse
}{%
    \typeout{Start loading preamble.tex (NOT in a XOURSE)}%
}
\makeatother

\def\isEn\true 

\pgfplotsset{compat=1.16}

\usepackage{currfile}

% 201908/202301: PAS OP: babel en doclicense lijken problemen te veroorzaken in .jax bestand
% (wegens syntax error met toegevoegde \newcommands ...)
\pdfOnly{
    \usepackage[type={CC},modifier={by-nc-sa},version={4.0}]{doclicense}
    %\usepackage[hyperxmp=false,type={CC},modifier={by-nc-sa},version={4.0}]{doclicense}
    %%% \usepackage[dutch]{babel}
}



\usepackage[utf8]{inputenc}
\usepackage{morewrites}   % nav zomercursus (answer...?)
\usepackage{multirow}
\usepackage{multicol}
\usepackage{tikzsymbols}
\usepackage{ifthen}
%\usepackage{animate} BREAKS HTML STRUCTURE USED BY XIMERA
\usepackage{relsize}

\usepackage{eurosym}    % \euro  (€ werkt niet in xake ...?)
\usepackage{fontawesome} % smileys etc

% Nuttig als ook interactieve beamer slides worden voorzien:
\providecommand{\p}{} % default nothing ; potentially usefull for slides: redefine as \pause
%providecommand{\p}{\pause}

    % Layout-parameters voor het onderschrift bij figuren
\usepackage[margin=10pt,font=small,labelfont=bf, labelsep=endash,format=hang]{caption}
%\usepackage{caption} % captionof
%\usepackage{pdflscape}    % landscape environment

% Met "\newcommand\showtodonotes{}" kan je todonotes tonen (in pdf/online)
% 201908: online werkt het niet (goed)
\providecommand\showtodonotes{disable}
\providecommand\todo[1]{\typeout{TODO #1}}
%\usepackage[\showtodonotes]{todonotes}
%\usepackage{todonotes}

%
% Poging tot aanpassen layout
%
\usepackage{tcolorbox}
\tcbuselibrary{theorems}

%%% Einde laad packages

%%% Begin Ximera specifieke zaken

\graphicspath{
	{../../}
	{../}
	{./}
  	{../../pictures/}
   	{../pictures/}
   	{./pictures/}
	{./explog/}    % M05 in groeimodellen       
}

%%% Einde Ximera specifieke zaken

%
% define softer blue/red/green, use KU Leuven base colors for blue (and dark orange for red ?)
%
% todo: rather redefine blue/red/green ...?
%\definecolor{xmblue}{rgb}{0.01, 0.31, 0.59}
%\definecolor{xmred}{rgb}{0.89, 0.02, 0.17}
\definecolor{xmdarkblue}{rgb}{0.122, 0.671, 0.835}   % KU Leuven Blauw
\definecolor{xmblue}{rgb}{0.114, 0.553, 0.69}        % KU Leuven Blauw
\definecolor{xmgreen}{rgb}{0.13, 0.55, 0.13}         % No KULeuven variant for green found ...

\definecolor{xmaccent}{rgb}{0.867, 0.541, 0.18}      % KU Leuven Accent (orange ...)
\definecolor{kuaccent}{rgb}{0.867, 0.541, 0.18}      % KU Leuven Accent (orange ...)

\colorlet{xmred}{xmaccent!50!black}                  % Darker version of KU Leuven Accent

\providecommand{\blue}[1]{{\color{blue}#1}}    
\providecommand{\red}[1]{{\color{red}#1}}

\renewcommand\CancelColor{\color{xmaccent!50!black}}

% werkt in math en text mode om MATH met oranje (of grijze...)  achtergond te tonen (ook \important{\text{blabla}} lijkt te werken)
%\newcommand{\important}[1]{\ensuremath{\colorbox{xmaccent!50!white}{$#1$}}}   % werkt niet in Mathjax
%\newcommand{\important}[1]{\ensuremath{\colorbox{lightgray}{$#1$}}}
\newcommand{\important}[1]{\ensuremath{\colorbox{orange}{$#1$}}}   % TODO: kleur aanpassen voor mathjax; wordt overschreven infra!


% Uitzonderlijk kan met \pdfnl in de PDF een newline worden geforceerd, die online niet nodig/nuttig is omdat daar de regellengte hoe dan ook niet gekend is.
\ifdefined\HCode%
\providecommand{\pdfnl}{}%
\else%
\providecommand{\pdfnl}{%
  \\%
}%
\fi

% Uitzonderlijk kan met \handoutnl in de handout-PDF een newline worden geforceerd, die noch online noch in de PDF-met-antwoorden nuttig is.
\ifdefined\HCode
\providecommand{\handoutnl}{}
\else
\providecommand{\handoutnl}{%
\ifhandout%
  \nl%
\fi%
}
\fi



% \cellcolor IGNORED by tex4ht ?
% \begin{center} seems not to wordk
    % (missing margin-left: auto;   on tabular-inside-center ???)
%\newcommand{\importantcell}[1]{\ensuremath{\cellcolor{lightgray}#1}}  %  in tabular; usablility to be checked
\providecommand{\importantcell}[1]{\ensuremath{#1}}     % no mathjax2 support for colloring array cells

\pdfOnly{
  \renewcommand{\important}[1]{\ensuremath{\colorbox{kuaccent!50!white}{$#1$}}}
  \renewcommand{\importantcell}[1]{\ensuremath{\cellcolor{kuaccent!40!white}#1}}   
}

%%% Tikz styles


\pgfplotsset{compat=1.16}

\usetikzlibrary{trees,positioning,arrows,fit,shapes,math,calc,decorations.markings,through,intersections,patterns,matrix}

\usetikzlibrary{decorations.pathreplacing,backgrounds}    % 5/2023: from experimental


\usetikzlibrary{angles,quotes}

\usepgfplotslibrary{fillbetween} % bepaalde_integraal
\usepgfplotslibrary{polar}    % oa voor poolcoordinaten.tex

\pgfplotsset{ownstyle/.style={axis lines = center, axis equal image, xlabel = $x$, ylabel = $y$, enlargelimits}} 

\pgfplotsset{
	plot/.style={no marks,samples=50}
}

\newcommand{\xmPlotsColor}{
	\pgfplotsset{
		plot1/.style={darkgray,no marks,samples=100},
		plot2/.style={lightgray,no marks,samples=100},
		plotresult/.style={blue,no marks,samples=100},
		plotblue/.style={blue,no marks,samples=100},
		plotred/.style={red,no marks,samples=100},
		plotgreen/.style={green,no marks,samples=100},
		plotpurple/.style={purple,no marks,samples=100}
	}
}
\newcommand{\xmPlotsBlackWhite}{
	\pgfplotsset{
		plot1/.style={black,loosely dashed,no marks,samples=100},
		plot2/.style={black,loosely dotted,no marks,samples=100},
		plotresult/.style={black,no marks,samples=100},
		plotblue/.style={black,no marks,samples=100},
		plotred/.style={black,dotted,no marks,samples=100},
		plotgreen/.style={black,dashed,no marks,samples=100},
		plotpurple/.style={black,dashdotted,no marks,samples=100}
	}
}


\newcommand{\xmPlotsColorAndStyle}{
	\pgfplotsset{
		plot1/.style={darkgray,no marks,samples=100},
		plot2/.style={lightgray,no marks,samples=100},
		plotresult/.style={blue,no marks,samples=100},
		plotblue/.style={xmblue,no marks,samples=100},
		plotred/.style={xmred,dashed,thick,no marks,samples=100},
		plotgreen/.style={xmgreen,dotted,very thick,no marks,samples=100},
		plotpurple/.style={purple,no marks,samples=100}
	}
}


%\iftikzexport
\xmPlotsColorAndStyle
%\else
%\xmPlotsBlackWhite
%\fi
%%%


%
% Om venndiagrammen te arceren ...
%
\makeatletter
\pgfdeclarepatternformonly[\hatchdistance,\hatchthickness]{north east hatch}% name
{\pgfqpoint{-1pt}{-1pt}}% below left
{\pgfqpoint{\hatchdistance}{\hatchdistance}}% above right
{\pgfpoint{\hatchdistance-1pt}{\hatchdistance-1pt}}%
{
	\pgfsetcolor{\tikz@pattern@color}
	\pgfsetlinewidth{\hatchthickness}
	\pgfpathmoveto{\pgfqpoint{0pt}{0pt}}
	\pgfpathlineto{\pgfqpoint{\hatchdistance}{\hatchdistance}}
	\pgfusepath{stroke}
}
\pgfdeclarepatternformonly[\hatchdistance,\hatchthickness]{north west hatch}% name
{\pgfqpoint{-\hatchthickness}{-\hatchthickness}}% below left
{\pgfqpoint{\hatchdistance+\hatchthickness}{\hatchdistance+\hatchthickness}}% above right
{\pgfpoint{\hatchdistance}{\hatchdistance}}%
{
	\pgfsetcolor{\tikz@pattern@color}
	\pgfsetlinewidth{\hatchthickness}
	\pgfpathmoveto{\pgfqpoint{\hatchdistance+\hatchthickness}{-\hatchthickness}}
	\pgfpathlineto{\pgfqpoint{-\hatchthickness}{\hatchdistance+\hatchthickness}}
	\pgfusepath{stroke}
}
%\makeatother

\tikzset{
    hatch distance/.store in=\hatchdistance,
    hatch distance=10pt,
    hatch thickness/.store in=\hatchthickness,
   	hatch thickness=2pt
}

\colorlet{circle edge}{black}
\colorlet{circle area}{blue!20}


\tikzset{
    filled/.style={fill=green!30, draw=circle edge, thick},
    arceerl/.style={pattern=north east hatch, pattern color=blue!50, draw=circle edge},
    arceerr/.style={pattern=north west hatch, pattern color=yellow!50, draw=circle edge},
    outline/.style={draw=circle edge, thick}
}




%%% Updaten commando's
\def\hoofding #1#2#3{\maketitle}     % OBSOLETE ??

% we willen (bijna) altijd \geqslant ipv \geq ...!
\newcommand{\geqnoslant}{\geq}
\renewcommand{\geq}{\geqslant}
\newcommand{\leqnoslant}{\leq}
\renewcommand{\leq}{\leqslant}

% Todo: (201908) waarom komt er (soms) underlined voor emph ...?
\renewcommand{\emph}[1]{\textit{#1}}

% API commando's

\newcommand{\ds}{\displaystyle}
\newcommand{\ts}{\textstyle}  % tegenhanger van \ds   (Ximera zet PER  DEFAULT \ds!)

% uit Zomercursus-macro's: 
\newcommand{\bron}[1]{\begin{scriptsize} \emph{#1} \end{scriptsize}}     % deprecated ...?


%definities nieuwe commando's - afkortingen veel gebruikte symbolen
\newcommand{\R}{\ensuremath{\mathbb{R}}}
\newcommand{\Rnul}{\ensuremath{\mathbb{R}_0}}
\newcommand{\Reen}{\ensuremath{\mathbb{R}\setminus\{1\}}}
\newcommand{\Rnuleen}{\ensuremath{\mathbb{R}\setminus\{0,1\}}}
\newcommand{\Rplus}{\ensuremath{\mathbb{R}^+}}
\newcommand{\Rmin}{\ensuremath{\mathbb{R}^-}}
\newcommand{\Rnulplus}{\ensuremath{\mathbb{R}_0^+}}
\newcommand{\Rnulmin}{\ensuremath{\mathbb{R}_0^-}}
\newcommand{\Rnuleenplus}{\ensuremath{\mathbb{R}^+\setminus\{0,1\}}}
\newcommand{\N}{\ensuremath{\mathbb{N}}}
\newcommand{\Nnul}{\ensuremath{\mathbb{N}_0}}
\newcommand{\Z}{\ensuremath{\mathbb{Z}}}
\newcommand{\Znul}{\ensuremath{\mathbb{Z}_0}}
\newcommand{\Zplus}{\ensuremath{\mathbb{Z}^+}}
\newcommand{\Zmin}{\ensuremath{\mathbb{Z}^-}}
\newcommand{\Znulplus}{\ensuremath{\mathbb{Z}_0^+}}
\newcommand{\Znulmin}{\ensuremath{\mathbb{Z}_0^-}}
\newcommand{\C}{\ensuremath{\mathbb{C}}}
\newcommand{\Cnul}{\ensuremath{\mathbb{C}_0}}
\newcommand{\Cplus}{\ensuremath{\mathbb{C}^+}}
\newcommand{\Cmin}{\ensuremath{\mathbb{C}^-}}
\newcommand{\Cnulplus}{\ensuremath{\mathbb{C}_0^+}}
\newcommand{\Cnulmin}{\ensuremath{\mathbb{C}_0^-}}
\newcommand{\Q}{\ensuremath{\mathbb{Q}}}
\newcommand{\Qnul}{\ensuremath{\mathbb{Q}_0}}
\newcommand{\Qplus}{\ensuremath{\mathbb{Q}^+}}
\newcommand{\Qmin}{\ensuremath{\mathbb{Q}^-}}
\newcommand{\Qnulplus}{\ensuremath{\mathbb{Q}_0^+}}
\newcommand{\Qnulmin}{\ensuremath{\mathbb{Q}_0^-}}

\newcommand{\perdef}{\overset{\mathrm{def}}{=}}
\newcommand{\pernot}{\overset{\mathrm{notatie}}{=}}
\newcommand\perinderdaad{\overset{!}{=}}     % voorlopig gebruikt in limietenrekenregels
\newcommand\perhaps{\overset{?}{=}}          % voorlopig gebruikt in limietenrekenregels

\newcommand{\degree}{^\circ}


\DeclareMathOperator{\dom}{dom}     % domein
\DeclareMathOperator{\codom}{codom} % codomein
\DeclareMathOperator{\bld}{bld}     % beeld
\DeclareMathOperator{\graf}{graf}   % grafiek
\DeclareMathOperator{\rico}{rico}   % richtingcoëfficient
\DeclareMathOperator{\co}{co}       % coordinaat
\DeclareMathOperator{\gr}{gr}       % graad

\newcommand{\func}[5]{\ensuremath{#1: #2 \rightarrow #3: #4 \mapsto #5}} % Easy to write a function


% Operators
\DeclareMathOperator{\bgsin}{bgsin}
\DeclareMathOperator{\bgcos}{bgcos}
\DeclareMathOperator{\bgtan}{bgtan}
\DeclareMathOperator{\bgcot}{bgcot}
\DeclareMathOperator{\bgsinh}{bgsinh}
\DeclareMathOperator{\bgcosh}{bgcosh}
\DeclareMathOperator{\bgtanh}{bgtanh}
\DeclareMathOperator{\bgcoth}{bgcoth}

% Oude \Bgsin etc deprecated: gebruik \bgsin, en herdefinieer dat als je Bgsin wil!
%\DeclareMathOperator{\cosec}{cosec}    % not used? gebruik \csc en herdefinieer

% operatoren voor differentialen: to be verified; 1/2020: inconsequent gebruik bij afgeleiden/integralen
\renewcommand{\d}{\mathrm{d}}
\newcommand{\dx}{\d x}
\newcommand{\dd}[1]{\frac{\mathrm{d}}{\mathrm{d}#1}}
\newcommand{\ddx}{\dd{x}}

% om in voorbeelden/oefeningen de notatie voor afgeleiden te kunnen kiezen
% Usage: \afg{(2\sin(x))}  (en wordt d/dx, of accent, of D )
%\newcommand{\afg}[1]{{#1}'}
\newcommand{\afg}[1]{\left(#1\right)'}
%\renewcommand{\afg}[1]{\frac{\mathrm{d}#1}{\mathrm{d}x}}   % include in relevant exercises ...
%\renewcommand{\afg}[1]{D{#1}}

%
% \xmxxx commands: Extra KU Leuven functionaliteit van, boven of naast Ximera
%   ( Conventie 8/2019: xm+nederlandse omschrijving, maar is niet consequent gevolgd, en misschien ook niet erg handig !)
%
% (Met een minimale ximera.cls en preamble.tex zou een bruikbare .pdf moeten kunnen worden gemaakt van eender welke ximera)
%
% Usage: \xmtitle[Mijn korte abstract]{Mijn titel}{Mijn abstract}
% Eerste command na \begin{document}:
%  -> definieert de \title
%  -> definieert de abstract
%  -> doet \maketitle ( dus: print de hoofding als 'chapter' of 'sectie')
% Optionele parameter geeft eenn kort abstract (die met de globale setting \xmshortabstract{} al dan niet kan worden geprint.
% De optionele korte abstract kan worden gebruikt voor pseudo-grappige abtsarts, dus dus globaal al dan niet kunnen worden gebuikt...
% Globale settings:
%  de (optionele) 'korte abstract' wordt enkele getoond als \xmshortabstract is gezet
\providecommand\xmshortabstract{} % default: print (only!) short abstract if present
\newcommand{\xmtitle}[3][]{
	\title{#2}
	\begin{abstract}
		\ifdefined\xmshortabstract
		\ifstrempty{#1}{%
			#3
		}{%
			#1
		}%
		\else
		#3
		\fi
	\end{abstract}
	\maketitle
}

% 
% Kleine grapjes: moeten zonder verder gevolg kunnen worden verwijderd
%
%\newcommand{\xmopje}[1]{{\small#1{\reversemarginpar\marginpar{\Smiley}}}}   % probleem in floats!!
\newtoggle{showxmopje}
\toggletrue{showxmopje}

\newcommand{\xmopje}[1]{%
   \iftoggle{showxmopje}{#1}{}%
}


% -> geef een abstracte-formule-met-rechts-een-concreet-voorbeeld
% VB:  \formulevb{a^2+b^2=c^2}{3^2+4^2=5^2}
%
\ifdefined\HCode
\NewEnviron{xmdiv}[1]{\HCode{\Hnewline<div class="#1">\Hnewline}\BODY{\HCode{\Hnewline</div>\Hnewline}}}
\else
\NewEnviron{xmdiv}[1]{\BODY}
\fi

\providecommand{\formulevb}[2]{
	{\centering

    \begin{xmdiv}{xmformulevb}    % zie css voor online layout !!!
	\begin{tabular}{lcl}
		\important{#1}
		&  &
		Vb: $#2$
		\end{tabular}
	\end{xmdiv}

	}
}

\ifdefined\HCode
\providecommand{\vb}[1]{%
    \HCode{\Hnewline<span class="xmvb">}#1\HCode{</span>\Hnewline}%
}
\else
\providecommand{\vb}[1]{
    \colorbox{blue!10}{#1}
}
\fi

\ifdefined\HCode
\providecommand{\xmcolorbox}[2]{
	\HCode{\Hnewline<div class="xmcolorbox">\Hnewline}#2\HCode{\Hnewline</div>\Hnewline}
}
\else
\providecommand{\xmcolorbox}[2]{
  \cellcolor{#1}#2
}
\fi


\ifdefined\HCode
\providecommand{\xmopmerking}[1]{
 \HCode{\Hnewline<div class="xmopmerking">\Hnewline}#1\HCode{\Hnewline</div>\Hnewline}
}
\else
\providecommand{\xmopmerking}[1]{
	{\footnotesize #1}
}
\fi
% \providecommand{\voorbeeld}[1]{
% 	\colorbox{blue!10}{$#1$}
% }



% Hernoem Proof naar Bewijs, nodig voor HTML versie
\renewcommand*{\proofname}{Bewijs}

% Om opgave van oefening (wordt niet geprint bij oplossingenblad)
% (to be tested test)
\NewEnviron{statement}{\BODY}

% Environment 'oplossing' en 'uitkomst'
% voor resp. volledige 'uitwerking' dan wel 'enkel eindresultaat'
% geimplementeerd via feedback, omdat er in de ximera-server adhoc feedback-code is toegevoegd
%% Niet tonen indien handout
%% Te gebruiken om volledige oplossingen/uitwerkingen van oefeningen te tonen
%% \begin{oplossing}        De optelling is commutatief \end{oplossing}  : verschijnt online enkel 'op vraag'
%% \begin{oplossing}[toon]  De optelling is commutatief \end{oplossing}  : verschijnt steeds onmiddellijk online (bv te gebruiken bij voorbeelden) 

\ifhandout%
    \NewEnviron{oplossing}[1][onzichtbaar]%
    {%
    \ifthenelse{\equal{\detokenize{#1}}{\detokenize{toon}}}
    {
    \def\PH@Command{#1}% Use PH@Command to hold the content and be a target for "\expandafter" to expand once.

    \begin{trivlist}% Begin the trivlist to use formating of the "Feedback" label.
    \item[\hskip \labelsep\small\slshape\bfseries Oplossing% Format the "Feedback" label. Don't forget the space.
    %(\texttt{\detokenize\expandafter{\PH@Command}}):% Format (and detokenize) the condition for feedback to trigger
    \hspace{2ex}]\small%\slshape% Insert some space before the actual feedback given.
    \BODY
    \end{trivlist}
    }
    {  % \begin{feedback}[solution]   \BODY     \end{feedback}  }
    }
    }    
\else
% ONLY for HTML; xmoplossing is styled with css, and is not, and need not be a LaTeX environment
% THUS: it does NOT use feedback anymore ...
%    \NewEnviron{oplossing}{\begin{expandable}{xmoplossing}{\nlen{Toon uitwerking}{Show solution}}{\BODY}\end{expandable}}
    \newenvironment{oplossing}[1][onzichtbaar]
   {%
       \begin{expandable}{xmoplossing}{}
   }
   {%
   	   \end{expandable}
   } 
%     \newenvironment{oplossing}[1][onzichtbaar]
%    {%
%        \begin{feedback}[solution]   	
%    }
%    {%
%    	   \end{feedback}
%    } 
\fi

\ifhandout%
    \NewEnviron{uitkomst}[1][onzichtbaar]%
    {%
    \ifthenelse{\equal{\detokenize{#1}}{\detokenize{toon}}}
    {
    \def\PH@Command{#1}% Use PH@Command to hold the content and be a target for "\expandafter" to expand once.

    \begin{trivlist}% Begin the trivlist to use formating of the "Feedback" label.
    \item[\hskip \labelsep\small\slshape\bfseries Uitkomst:% Format the "Feedback" label. Don't forget the space.
    %(\texttt{\detokenize\expandafter{\PH@Command}}):% Format (and detokenize) the condition for feedback to trigger
    \hspace{2ex}]\small%\slshape% Insert some space before the actual feedback given.
    \BODY
    \end{trivlist}
    }
    {  % \begin{feedback}[solution]   \BODY     \end{feedback}  }
    }
    }    
\else
\ifdefined\HCode
   \newenvironment{uitkomst}[1][onzichtbaar]
    {%
        \begin{expandable}{xmuitkomst}{}%
    }
    {%
    	\end{expandable}%
    } 
\else
  % Do NOT print 'uitkomst' in non-handout
  %  (presumably, there is also an 'oplossing' ??)
  \newenvironment{uitkomst}[1][onzichtbaar]{}{}
\fi
\fi

%
% Uitweidingen zijn extra's die niet redelijkerwijze tot de leerstof behoren
% Uitbreidingen zijn extra's die wel redelijkerwijze tot de leerstof van bv meer geavanceerde versies kunnen behoren (B-programma/Wiskundestudenten/...?)
% Nog niet voorzien: design voor verschillende versies (A/B programma, BIO, voorkennis/ ...)
% Voor 'uitweidingen' is er een environment die online per default is ingeklapt, en in pdf al dan niet kan worden geincluded  (via \xmnouitweiding) 
%
% in een xourse, per default GEEN uitweidingen, tenzij \xmuitweiding expliciet ergens is gezet ...
\ifdefined\isXourse
   \ifdefined\xmuitweiding
   \else
       \def\xmnouitweiding{true}
   \fi
\fi

\ifdefined\xmnouitweiding
\newcounter{xmuitweiding}  % anders error undefined ...  
\excludecomment{xmuitweiding}
\else
\newtheoremstyle{dotless}{}{}{}{}{}{}{ }{}
\theoremstyle{dotless}
\newtheorem*{xmuitweidingnofrills}{}   % nofrills = no accordion; gebruikt dus de dotless theoremstyle!

\newcounter{xmuitweiding}
\newenvironment{xmuitweiding}[1][ ]%
{% 
	\refstepcounter{xmuitweiding}%
    \begin{expandable}{xmuitweiding}{\nlentext{Uitweiding \arabic{xmuitweiding}: #1}{Digression \arabic{xmuitweiding}: #1}}%
	\begin{xmuitweidingnofrills}%
}
{%
    \end{xmuitweidingnofrills}%
    \end{expandable}%
}   
% \newenvironment{xmuitweiding}[1][ ]%
% {% 
% 	\refstepcounter{xmuitweiding}
% 	\begin{accordion}\begin{accordion-item}[Uitweiding \arabic{xmuitweiding}: #1]%
% 			\begin{xmuitweidingnofrills}%
% 			}
% 			{\end{xmuitweidingnofrills}\end{accordion-item}\end{accordion}}   
\fi


\newenvironment{xmexpandable}[1][]{
	\begin{accordion}\begin{accordion-item}[#1]%
		}{\end{accordion-item}\end{accordion}}


% Command that gives a selection box online, but just prints the right answer in pdf
\newcommand{\xmonlineChoice}[1]{\pdfOnly{\wordchoicegiventrue}\wordChoice{#1}\pdfOnly{\wordchoicegivenfalse}}   % deprecated, gebruik onlineChoice ...
\newcommand{\onlineChoice}[1]{\pdfOnly{\wordchoicegiventrue}\wordChoice{#1}\pdfOnly{\wordchoicegivenfalse}}

\newcommand{\TJa}{\nlentext{ Ja }{ Yes }}
\newcommand{\TNee}{\nlentext{ Nee }{ No }}
\newcommand{\TJuist}{\nlentext{ Juist }{ True }}
\newcommand{\TFout}{\nlentext{ Fout }{ False }}

\newcommand{\choiceTrue }{{\renewcommand{\choiceminimumhorizontalsize}{4em}\wordChoice{\choice[correct]{\TJuist}\choice{\TFout}}}}
\newcommand{\choiceFalse}{{\renewcommand{\choiceminimumhorizontalsize}{4em}\wordChoice{\choice{\TJuist}\choice[correct]{\TFout}}}}

\newcommand{\choiceYes}{{\renewcommand{\choiceminimumhorizontalsize}{3em}\wordChoice{\choice[correct]{\TJa}\choice{\TNee}}}}
\newcommand{\choiceNo }{{\renewcommand{\choiceminimumhorizontalsize}{3em}\wordChoice{\choice{\TJa}\choice[correct]{\TNee}}}}

% Optional nicer formatting for wordChoice in PDF

\let\inlinechoiceorig\inlinechoice

%\makeatletter
%\providecommand{\choiceminimumverticalsize}{\vphantom{$\frac{\sqrt{2}}{2}$}}   % minimum height of boxes (cfr infra)
\providecommand{\choiceminimumverticalsize}{\vphantom{$\tfrac{2}{2}$}}   % minimum height of boxes (cfr infra)
\providecommand{\choiceminimumhorizontalsize}{1em}   % minimum width of boxes (cfr infra)

\newcommand{\inlinechoicesquares}[2][]{%
		\setkeys{choice}{#1}%
		\ifthenelse{\boolean{\choice@correct}}%
		{%
            \ifhandout%
               \fbox{\choiceminimumverticalsize #2}\allowbreak\ignorespaces%
            \else%
               \fcolorbox{blue}{blue!20}{\choiceminimumverticalsize #2}\allowbreak\ignorespaces\setkeys{choice}{correct=false}\ignorespaces%
            \fi%
		}%
		{% else
			\fbox{\choiceminimumverticalsize #2}\allowbreak\ignorespaces%  HACK: wat kleiner, zodat fits on line ... 	
		}%
}

\newcommand{\inlinechoicesquareX}[2][]{%
		\setkeys{choice}{#1}%
		\ifthenelse{\boolean{\choice@correct}}%
		{%
            \ifhandout%
               \framebox[\ifdim\choiceminimumhorizontalsize<\width\width\else\choiceminimumhorizontalsize\fi]{\choiceminimumverticalsize\ #2\ }\allowbreak\ignorespaces\setkeys{choice}{correct=false}\ignorespaces%
            \else%
               \fcolorbox{blue}{blue!20}{\makebox[\ifdim\choiceminimumhorizontalsize<\width\width\else\choiceminimumhorizontalsize\fi]{\choiceminimumverticalsize #2}}\allowbreak\ignorespaces\setkeys{choice}{correct=false}\ignorespaces%
            \fi%
		}%
		{% else
        \ifhandout%
			\framebox[\ifdim\choiceminimumhorizontalsize<\width\width\else\choiceminimumhorizontalsize\fi]{\choiceminimumverticalsize\ #2\ }\allowbreak\ignorespaces%  HACK: wat kleiner, zodat fits on line ... 	
        \fi
		}%
}


\newcommand{\inlinechoicepuntjes}[2][]{%
		\setkeys{choice}{#1}%
		\ifthenelse{\boolean{\choice@correct}}%
		{%
            \ifhandout%
               \dots\ldots\ignorespaces\setkeys{choice}{correct=false}\ignorespaces
            \else%
               \fcolorbox{blue}{blue!20}{\choiceminimumverticalsize #2}\allowbreak\ignorespaces\setkeys{choice}{correct=false}\ignorespaces%
            \fi%
		}%
		{% else
			%\fbox{\choiceminimumverticalsize #2}\allowbreak\ignorespaces%  HACK: wat kleiner, zodat fits on line ... 	
		}%
}

% print niets, maar definieer globale variable \myanswer
%  (gebruikt om oplossingsbladen te printen) 
\newcommand{\inlinechoicedefanswer}[2][]{%
		\setkeys{choice}{#1}%
		\ifthenelse{\boolean{\choice@correct}}%
		{%
               \gdef\myanswer{#2}\setkeys{choice}{correct=false}

		}%
		{% else
			%\fbox{\choiceminimumverticalsize #2}\allowbreak\ignorespaces%  HACK: wat kleiner, zodat fits on line ... 	
		}%
}



%\makeatother

\newcommand{\setchoicedefanswer}{
\ifdefined\HCode
\else
%    \renewenvironment{multipleChoice@}[1][]{}{} % remove trailing ')'
    \let\inlinechoice\inlinechoicedefanswer
\fi
}

\newcommand{\setchoicepuntjes}{
\ifdefined\HCode
\else
    \renewenvironment{multipleChoice@}[1][]{}{} % remove trailing ')'
    \let\inlinechoice\inlinechoicepuntjes
\fi
}
\newcommand{\setchoicesquares}{
\ifdefined\HCode
\else
    \renewenvironment{multipleChoice@}[1][]{}{} % remove trailing ')'
    \let\inlinechoice\inlinechoicesquares
\fi
}
%
\newcommand{\setchoicesquareX}{
\ifdefined\HCode
\else
    \renewenvironment{multipleChoice@}[1][]{}{} % remove trailing ')'
    \let\inlinechoice\inlinechoicesquareX
\fi
}
%
\newcommand{\setchoicelist}{
\ifdefined\HCode
\else
    \renewenvironment{multipleChoice@}[1][]{}{)}% re-add trailing ')'
    \let\inlinechoice\inlinechoiceorig
\fi
}

\setchoicesquareX  % by default list-of-squares with onlineChoice in PDF

% Omdat multicols niet werkt in html: enkel in pdf  (in html zijn langere pagina's misschien ook minder storend)
\newenvironment{xmmulticols}[1][2]{
 \pdfOnly{\begin{multicols}{#1}}%
}{ \pdfOnly{\end{multicols}}}

%
% Te gebruiken in plaats van \section\subsection
%  (in een printstyle kan dan het level worden aangepast
%    naargelang \chapter vs \section style )
% 3/2021: DO NOT USE \xmsubsection !
\newcommand\xmsection\subsection
\newcommand\xmsubsection\subsubsection

% Aanpassen printversie
%  (hier gedefinieerd, zodat ze in xourse kunnen worden gezet/overschreven)
\providebool{parttoc}
\providebool{printpartfrontpage}
\providebool{printactivitytitle}
\providebool{printactivityqrcode}
\providebool{printactivityurl}
\providebool{printcontinuouspagenumbers}
\providebool{numberactivitiesbysubpart}
\providebool{addtitlenumber}
\providebool{addsectiontitlenumber}
\addtitlenumbertrue
\addsectiontitlenumbertrue

% The following three commands are hardcoded in xake, you can't create other commands like these, without adding them to xake as well
%  ( gebruikt in xourses om juiste soort titelpagina te krijgen voor verschillende ximera's )
\newcommand{\activitychapter}[2][]{
    {    
    \ifstrequal{#1}{notnumbered}{
        \addtitlenumberfalse
    }{}
    \typeout{ACTIVITYCHAPTER #2}   % logging
	\chapterstyle
	\activity{#2}
    }
}
\newcommand{\activitysection}[2][]{
    {
    \ifstrequal{#1}{notnumbered}{
        \addsectiontitlenumberfalse
    }{}
	\typeout{ACTIVITYSECTION #2}   % logging
	\sectionstyle
	\activity{#2}
    }
}
% Practices worden als activity getoond om de grote blokken te krijgen online
\newcommand{\practicesection}[2][]{
    {
    \ifstrequal{#1}{notnumbered}{
        \addsectiontitlenumberfalse
    }{}
    \typeout{PRACTICESECTION #2}   % logging
	\sectionstyle
	\activity{#2}
    }
}
\newcommand{\activitychapterlink}[3][]{
    {
    \ifstrequal{#1}{notnumbered}{
        \addtitlenumberfalse
    }{}
    \typeout{ACTIVITYCHAPTERLINK #3}   % logging
	\chapterstyle
	\activitylink[#1]{#2}{#3}
    }
}

\newcommand{\activitysectionlink}[3][]{
    {
    \ifstrequal{#1}{notnumbered}{
        \addsectiontitlenumberfalse
    }{}
    \typeout{ACTIVITYSECTIONLINK #3}   % logging
	\sectionstyle
	\activitylink[#1]{#2}{#3}
    }
}


% Commando om de printstyle toe te voegen in ximera's. Zorgt ervoor dat er geen problemen zijn als je de xourses compileert
% hack om onhandige relative paden in TeX te omzeilen
% should work both in xourse and ximera (pre-112022 only in ximera; thus obsoletes adhoc setup in xourses)
% loads global.sty if present (cfr global.css for online settings!)
% use global.sty to overwrite settings in printstyle.sty ...
\newcommand{\addPrintStyle}[1]{
\iftikzexport\else   % only in PDF
  \makeatletter
  \ifx\@onlypreamble\@notprerr\else   % ONLY if in tex-preamble   (and e.g. not when included from xourse)
    \typeout{Loading printstyle}   % logging
    \usepackage{#1/printstyle} % mag enkel geinclude worden als je die apart compileert
    \IfFileExists{#1/global.sty}{
        \typeout{Loading printstyle-folder #1/global.sty}   % logging
        \usepackage{#1/global}
        }{
        \typeout{Info: No extra #1/global.sty}   % logging
    }   % load global.sty if present
    \IfFileExists{global.sty}{
        \typeout{Loading local-folder global.sty (or TEXINPUTPATH..)}   % logging
        \usepackage{global}
    }{
        \typeout{Info: No folder/global.sty}   % logging
    }   % load global.sty if present
    \IfFileExists{\currfilebase.sty}
    {
        \typeout{Loading \currfilebase.sty}
        \input{\currfilebase.sty}
    }{
        \typeout{Info: No local \currfilebase.sty}
    }
    \fi
  \makeatother
\fi
}

%
%  
% references: Ximera heeft adhoc logica	 om online labels te doen werken over verschillende files heen
% met \hyperref kan de getoonde tekst toch worden opgegeven, in plaats van af te hangen van de label-text
\ifdefined\HCode
% Link to standard \labels, but give your own description
% Usage:  Volg \hyperref[my_very_verbose_label]{deze link} voor wat tijdverlies
%   (01/2020: Ximera-server aangepast om bij class reference-keeptext de link-text NIET te vervangen door de label-text !!!) 
\renewcommand{\hyperref}[2][]{\HCode{<a class="reference reference-keeptext" href="\##1">}#2\HCode{</a>}}
%
%  Link to specific targets  (not tested ?)
\renewcommand{\hypertarget}[1]{\HCode{<a class="ximera-label" id="#1"></a>}}
\renewcommand{\hyperlink}[2]{\HCode{<a class="reference reference-keeptext" href="\##1">}#2\HCode{</a>}}
\fi

% Mmm, quid English ... (for keyword #1 !) ?
\newcommand{\wikilink}[2]{
    \hyperlink{https://nl.wikipedia.org/wiki/#1}{#2}
    \pdfOnly{\footnote{See \url{https://nl.wikipedia.org/wiki/#1}}
    }
}

\renewcommand{\figurename}{Figuur}
\renewcommand{\tablename}{Tabel}

%
% Gedoe om verschillende versies van xourse/ximera te maken afhankelijk van settings
%
% default: versie met antwoorden
% handout: versie voor de studenten, zonder antwoorden/oplossingen
% full: met alles erop en eraan, dus geschikt voor auteurs en/of lesgevers  (bevat in de pdf ook de 'online-only' stukken!)
%
%
% verder kunnen ook opties/variabele worden gezet voor hints/auteurs/uitweidingen/ etc
%
% 'Full' versie
\newtoggle{showonline}
\ifdefined\HCode   % zet default showOnline
    \toggletrue{showonline} 
\else
    \togglefalse{showonline}
\fi

% Full versie   % deprecated: see infra
\newcommand{\printFull}{
    \hintstrue
    \handoutfalse
    \toggletrue{showonline} 
}

\ifdefined\shouldPrintFull   % deprecated: see infra
    \printFull
\fi



% Overschrijf onlineOnly  (zoals gedefinieerd in ximera.cls)
\ifhandout   % in handout: gebruik de oorspronkelijke ximera.cls implementatie  (is dit wel nodig/nuttig?)
\else
    \iftoggle{showonline}{%
        \ifdefined\HCode
          \RenewEnviron{onlineOnly}{\bgroup\BODY\egroup}   % showOnline, en we zijn  online, dus toon de tekst
        \else
          \RenewEnviron{onlineOnly}{\bgroup\color{red!50!black}\BODY\egroup}   % showOnline, maar we zijn toch niet online: kleur de tekst rood 
        \fi
    }{%
      \RenewEnviron{onlineOnly}{}  % geen showOnline
    }
\fi

% hack om na hoofding van definition/proposition/... als dan niet op een nieuwe lijn te starten
% soms is dat goed en mooi, en soms niet; en in HTML is het nu (2/2020) anders dan in pdf
% vandaar suggestie om 
%     \begin{definition}[Nieuw concept] \nl
% te gebruiken als je zeker een newline wil na de hoofdig en titel
% (in het bijzonder itemize zonder \nl is 'lelijk' ...)
\ifdefined\HCode
\newcommand{\nl}{}
\else
\newcommand{\nl}{\ \par} % newline (achter heading van definition etc.)
\fi


% \nl enkel in handoutmode (ihb voor \wordChoice, die dan typisch veeeel langer wordt)
\ifdefined\HCode
\providecommand{\handoutnl}{}
\else
\providecommand{\handoutnl}{%
\ifhandout%
  \nl%
\fi%
}
\fi

% Could potentially replace \pdfOnline/\begin{onlineOnly} : 
% Usage= \ifonline{Hallo surfer}{Hallo PDFlezer}
\providecommand{\ifonline}[2]%
{
\begin{onlineOnly}#1\end{onlineOnly}%
\pdfOnly{#2}
}%


%
% Maak optionele 'basic' en 'extended' versies van een activity
%  met environment basicOnly, basicSkip en extendedOnly
%
%  (
%   Dit werkt ENKEL in de PDF; de online versies tonen (minstens voorklopig) steeds 
%   het default geval met printbasicversion en printextendversion beide FALSE
%  )
%
\providebool{printbasicversion}
\providebool{printextendedversion}   % not properly implemented
\providebool{printfullversion}       % presumably print everything (debug/auteur)
%
% only set these in xourses, and BEFORE loading this preamble
%
%\newif\ifshowbasic     \showbasictrue        % use this line in xourse to show 'basic' sections
%\newif\ifshowextended  \showextendedtrue     % use this line in xourse to show 'extended' sections
%
%
%\ifbool{showbasic}
%      { \NewEnviron{basicOnly}{\BODY} }    % if yes: just print contents
%      { \NewEnviron{basicOnly}{}      }    % if no:  completely ignore contents
%
%\ifbool{showbasic}
%      { \NewEnviron{basicSkip}{}      }
%      { \NewEnviron{basicSkip}{\BODY} }
%

\ifbool{printextendedversion}
      { \NewEnviron{extendedOnly}{\BODY} }
      { \NewEnviron{extendedOnly}{}      }
      


\ifdefined\HCode    % in html: always print
      {\newenvironment*{basicOnly}{}{}}    % if yes: just print contents
      {\newenvironment*{basicSkip}{}{}}    % if yes: just print contents
\else

\ifbool{printbasicversion}
      {\newenvironment*{basicOnly}{}{}}    % if yes: just print contents
      {\NewEnviron{basicOnly}{}      }    % if no:  completely ignore contents

\ifbool{printbasicversion}
      {\NewEnviron{basicSkip}{}      }
      {\newenvironment*{basicSkip}{}{}}

\fi

\usepackage{float}
\usepackage[rightbars,color]{changebar}

% Full versie
\ifbool{printfullversion}{
    \hintstrue
    \handoutfalse
    \toggletrue{showonline}
    \printbasicversionfalse
    \cbcolor{red}
    \renewenvironment*{basicOnly}{\cbstart}{\cbend}
    \renewenvironment*{basicSkip}{\cbstart}{\cbend}
    \def\xmtoonprintopties{FULL}   % will be printed in footer
}
{}
      
%
% Evalueer \ifhints IN de environment
%  
%
%\RenewEnviron{hint}
%{
%\ifhandout
%\ifhints\else\setbox0\vbox\fi%everything in een emty box
%\bgroup 
%\stepcounter{hintLevel}
%\BODY
%\egroup\ignorespacesafterend
%\addtocounter{hintLevel}{-1}
%\else
%\ifhints
%\begin{trivlist}\item[\hskip \labelsep\small\slshape\bfseries Hint:\hspace{2ex}]
%\small\slshape
%\stepcounter{hintLevel}
%\BODY
%\end{trivlist}
%\addtocounter{hintLevel}{-1}
%\fi
%\fi
%}

% Onafhankelijk van \ifhandout ...? TO BE VERIFIED
\RenewEnviron{hint}
{
\ifhints
\begin{trivlist}\item[\hskip \labelsep\small\bfseries Hint:\hspace{2ex}]
\small%\slshape
\stepcounter{hintLevel}
\BODY
\end{trivlist}
\addtocounter{hintLevel}{-1}
\else
\iftikzexport   % anders worden de tikz tekeningen in hints niet gegenereerd ?
\setbox0\vbox\bgroup
\stepcounter{hintLevel}
\BODY
\egroup\ignorespacesafterend
\addtocounter{hintLevel}{-1}
\fi % ifhandout
\fi %ifhints
}

%
% \tab sets typewriter-tabs (e.g. to format questions)
% (Has no effect in HTML :-( ))
%
\usepackage{tabto}
\ifdefined\HCode
  \renewcommand{\tab}{\quad}    % otherwise dummy .png's are generated ...?
\fi


% Also redefined in  preamble to get correct styling 
% for tikz images for (\tikzexport)
%

\theoremstyle{definition} % Bold titels
\makeatletter
\let\proposition\relax
\let\c@proposition\relax
\let\endproposition\relax
\makeatother
\newtheorem{proposition}{Eigenschap}


%\instructornotesfalse

% logic with \ifhandoutin ximera.cls unclear;so overwrite ...
\makeatletter
\@ifundefined{ifinstructornotes}{%
  \newif\ifinstructornotes
  \instructornotesfalse
  \newenvironment{instructorNotes}{}{}
}{}
\makeatother
\ifinstructornotes
\else
\renewenvironment{instructorNotes}%
{%
    \setbox0\vbox\bgroup
}
{%
    \egroup
}
\fi

% \RedeclareMathOperator
% from https://tex.stackexchange.com/questions/175251/how-to-redefine-a-command-using-declaremathoperator
\makeatletter
\newcommand\RedeclareMathOperator{%
    \@ifstar{\def\rmo@s{m}\rmo@redeclare}{\def\rmo@s{o}\rmo@redeclare}%
}
% this is taken from \renew@command
\newcommand\rmo@redeclare[2]{%
    \begingroup \escapechar\m@ne\xdef\@gtempa{{\string#1}}\endgroup
    \expandafter\@ifundefined\@gtempa
    {\@latex@error{\noexpand#1undefined}\@ehc}%
    \relax
    \expandafter\rmo@declmathop\rmo@s{#1}{#2}}
% This is just \@declmathop without \@ifdefinable
\newcommand\rmo@declmathop[3]{%
    \DeclareRobustCommand{#2}{\qopname\newmcodes@#1{#3}}%
}
\@onlypreamble\RedeclareMathOperator
\makeatother


%
% Engelse vertaling, vooral in mathmode
%
% 1. Algemene procedure
%
\ifdefined\isEn
 \newcommand{\nlen}[2]{#2}
 \newcommand{\nlentext}[2]{\text{#2}}
 \newcommand{\nlentextbf}[2]{\textbf{#2}}
\else
 \newcommand{\nlen}[2]{#1}
 \newcommand{\nlentext}[2]{\text{#1}}
 \newcommand{\nlentextbf}[2]{\textbf{#1}}
\fi

%
% 2. Lijst van erg veel gebruikte uitdrukkingen
%

% Ja/Nee/Fout/Juits etc
%\newcommand{\TJa}{\nlentext{ Ja }{ and }}
%\newcommand{\TNee}{\nlentext{ Nee }{ No }}
%\newcommand{\TJuist}{\nlentext{ Juist }{ Correct }
%\newcommand{\TFout}{\nlentext{ Fout }{ Wrong }
\newcommand{\TWaar}{\nlentext{ Waar }{ True }}
\newcommand{\TOnwaar}{\nlentext{ Vals }{ False }}
% Korte bindwoorden en, of, dus, ...
\newcommand{\Ten}{\nlentext{ en }{ and }}
\newcommand{\Tof}{\nlentext{ of }{ or }}
\newcommand{\Tdus}{\nlentext{ dus }{ so }}
\newcommand{\Tendus}{\nlentext{ en dus }{ and thus }}
\newcommand{\Tvooralle}{\nlentext{ voor alle }{ for all }}
\newcommand{\Took}{\nlentext{ ook }{ also }}
\newcommand{\Tals}{\nlentext{ als }{ when }} %of if?
\newcommand{\Twant}{\nlentext{ want }{ as }}
\newcommand{\Tmaal}{\nlentext{ maal }{ times }}
\newcommand{\Toptellen}{\nlentext{ optellen }{ add }}
\newcommand{\Tde}{\nlentext{ de }{ the }}
\newcommand{\Thet}{\nlentext{ het }{ the }}
\newcommand{\Tis}{\nlentext{ is }{ is }} %zodat is in text staat in mathmode (geen italics)
\newcommand{\Tmet}{\nlentext{ met }{ where }} % in situaties e.g met p < n --> where p < n
\newcommand{\Tnooit}{\nlentext{ nooit }{ never }}
\newcommand{\Tmaar}{\nlentext{ maar }{ but }}
\newcommand{\Tniet}{\nlentext{ niet }{ not }}
\newcommand{\Tuit}{\nlentext{ uit }{ from }}
\newcommand{\Ttov}{\nlentext{ t.o.v. }{ w.r.t. }}
\newcommand{\Tzodat}{\nlentext{ zodat }{ such that }}
\newcommand{\Tdeth}{\nlentext{de }{th }}
\newcommand{\Tomdat}{\nlentext{omdat }{because }} 


%
% Overschrijf addhoc commando's
%
\ifdefined\isEn
\renewcommand{\pernot}{\overset{\mathrm{notation}}{=}}
\RedeclareMathOperator{\bld}{im}     % beeld
\RedeclareMathOperator{\graf}{graph}   % grafiek
\RedeclareMathOperator{\rico}{slope}   % richtingcoëfficient
\RedeclareMathOperator{\co}{co}       % coordinaat
\RedeclareMathOperator{\gr}{deg}       % graad

% Operators
\RedeclareMathOperator{\bgsin}{arcsin}
\RedeclareMathOperator{\bgcos}{arccos}
\RedeclareMathOperator{\bgtan}{arctan}
\RedeclareMathOperator{\bgcot}{arccot}
\RedeclareMathOperator{\bgsinh}{arcsinh}
\RedeclareMathOperator{\bgcosh}{arccosh}
\RedeclareMathOperator{\bgtanh}{arctanh}
\RedeclareMathOperator{\bgcoth}{arccoth}

\fi


% HACK: use 'oplossing' for 'explanation' ...
\let\explanation\relax
\let\endexplanation\relax
% \newenvironment{explanation}{\begin{oplossing}}{\end{oplossing}}
\newcounter{explanation}

\ifhandout%
    \NewEnviron{explanation}[1][toon]%
    {%
    \RenewEnviron{verbatim}{ \red{VERBATIM CONTENT MISSING IN THIS PDF}} %% \expandafter\verb|\BODY|}

    \ifthenelse{\equal{\detokenize{#1}}{\detokenize{toon}}}
    {
    \def\PH@Command{#1}% Use PH@Command to hold the content and be a target for "\expandafter" to expand once.

    \begin{trivlist}% Begin the trivlist to use formating of the "Feedback" label.
    \item[\hskip \labelsep\small\slshape\bfseries Explanation:% Format the "Feedback" label. Don't forget the space.
    %(\texttt{\detokenize\expandafter{\PH@Command}}):% Format (and detokenize) the condition for feedback to trigger
    \hspace{2ex}]\small%\slshape% Insert some space before the actual feedback given.
    \BODY
    \end{trivlist}
    }
    {  % \begin{feedback}[solution]   \BODY     \end{feedback}  }
    }
    }    
\else
% ONLY for HTML; xmoplossing is styled with css, and is not, and need not be a LaTeX environment
% THUS: it does NOT use feedback anymore ...
%    \NewEnviron{oplossing}{\begin{expandable}{xmoplossing}{\nlen{Toon uitwerking}{Show solution}}{\BODY}\end{expandable}}
    \newenvironment{explanation}[1][toon]
   {%
       \begin{expandable}{xmoplossing}{}
   }
   {%
   	   \end{expandable}
   } 
\fi

\title{Diagonalizable Matrices and Multiplicity} \license{CC BY-NC-SA 4.0}

\begin{document}
\begin{abstract}
\end{abstract}
\maketitle

\section*{Diagonalizable Matrices and Multiplicity}
%When a matrix is similar to a diagonal matrix, the matrix is said to be \dfn{diagonalizable}. 
Recall that a \dfn{diagonal matrix} $D$ is a matrix containing a zero in every entry except those on the main diagonal.  More precisely, if $d_{ij}$ is the $ij^{th}$ entry of a diagonal matrix $D$, then
$d_{ij}=0$ unless $i=j$. Such
matrices look like the following.
\begin{equation*}
D = 
\begin{bmatrix}
* &  & 0 \\
& \ddots &  \\
0 &  & *
\end{bmatrix}
\end{equation*}
where $*$ is a number which might not be zero.


Diagonal matrices have some nice properties, as we demonstrate below.

\begin{exploration}\label{init:multiplydiag}
Let
$M =\begin{bmatrix}1 & 2 & 3\\ 4&5&6\\7&8&9\end{bmatrix}$ and let $D =\begin{bmatrix}2 & 0 & 0\\ 0&-5&0\\0&0&10\end{bmatrix}$.  Compute $MD$ and $DM$ using Octave.

%\begin{octave}
M=[1 2 3; 4 5 6; 7 8 9]
D=diag([2 -5 10]) 
%The function diag(vector) creates a square matrix with diagonal vector and the remaining entries zero. The function diag() can also be used to return the diagonal of a matrix.  For example, diag(M) would return [1;5;9].
M*D
D*M
%\end{octave}

%$$MD = \begin{bmatrix} \answer{2} & \answer{-10} & \answer{30}\\ \answer{8}&\answer{-25}&\answer{60}\\\answer{14}&\answer{-40}&\answer{90}\end{bmatrix} \quad DM= \begin{bmatrix} \answer{2} & \answer{4} & \answer{6}\\ \answer{-20}&\answer{-25}&\answer{-30}\\\answer{70}&\answer{80}&\answer{90}\end{bmatrix}$$

Notice the patterns present in the product matrices.  Each row of $DM$ is the same as its corresponding row of $M$ multiplied by the scalar which is the corresponding diagonal element of $D$.  In the product $MD$, it is the columns of $M$ that have been multiplied by the diagonal elements. These patterns hold in general for any diagonal matrix, and they are fundamental to understanding diagonalization, the process we discuss below.
\end{exploration}

\begin{definition}\label{def:diagonalizable}
Let $A$ be an $n\times n$ matrix. Then $A$ is said to be \dfn{diagonalizable} if there exists an invertible matrix $P$ such that
\begin{equation*}
P^{-1}AP=D
\end{equation*}
where $D$ is a diagonal matrix.  In other words, a matrix $A$ is diagonalizable if it is similar to a diagonal matrix, $A \sim D$.
\end{definition}

% Notice that the above equation can be rearranged as $A=PDP^{-1}$. Suppose we wanted to compute $A^{100}$. By diagonalizing $A$ first it suffices to then compute $\left(PDP^{-1}\right)^{100}$, which reduces to $PD^{100}P^{-1}$. This last computation is much simpler than $A^{100}$. While this process is described in detail later, it provides motivation for diagonalization. 


If we are given a matrix $A$ that is diagonalizable, then we can write $P^{-1}AP=D$ for some matrix $P$, or, equivalently,
\begin{equation}\label{eq:understand_diag}
AP=PD   
\end{equation}
If we pause to examine Equation (\ref{eq:understand_diag}), the work that we did in Exploration  \ref{init:multiplydiag} can help us to understand how to find $P$ that will diagonalize $A$. The product $PD$ is formed by multiplying each column of $P$ by a scalar which is the corresponding element on the diagonal of $D$.  To restate this, if $\vec{x}_i$ is column $i$ in our matrix $P$, then Equation (\ref{eq:understand_diag}) tells us that 
\begin{equation}\label{eq:ev_ew_diag}
A \vec{x}_i = \lambda_i \vec{x}_i,  
\end{equation}
where $\lambda_i$ is the $i$th diagonal element of $D$.  

Of course, Equation (\ref{eq:ev_ew_diag}) is very familiar!  We see that if we are able to diagonalize a matrix $A$, the columns of matrix $P$ will be the eigenvectors of $A$, and the corresponding diagonal entries of $D$ will be the corresponding eigenvalues of $A$.  This is summed up in the following theorem.

\begin{theorem}\label{th:eigenvectorsanddiagonalizable}
An $n\times n$ matrix $A$ is diagonalizable if and only if there is an
invertible matrix $P$ given by
$$P=\begin{bmatrix}
| & | &   & | \\
\vec{x}_1 & \vec{x}_2  & \cdots & \vec{x}_n \\
| & | &   & |
\end{bmatrix},$$
where the columns $\vec{x}_i$ are eigenvectors of $A$.

Moreover, if $A$ is diagonalizable, the corresponding eigenvalues of $A$ are the
diagonal entries of the diagonal matrix $D$.
\end{theorem}

\begin{proof}
Suppose $P$ is given as above as an invertible matrix whose columns are eigenvectors of $A$. To show that $A$ is diagonalizable, we will show 
$$AP=PD,$$
which is equivalent to $P^{-1}AP=D$.  We have
$$AP=\begin{bmatrix}
| & | &   & | \\
A\vec{x}_1 & A\vec{x}_2  & \cdots & A\vec{x}_n \\
| & | &   & |
\end{bmatrix},$$
while
\begin{equation*}
PD=\begin{bmatrix}
| & | &   & | \\
\vec{x}_1 & \vec{x}_2  & \cdots & \vec{x}_n \\
| & | &   & |
\end{bmatrix} 
\begin{bmatrix}
\lambda _{1} &  & 0 \\
& \ddots &  \\
0 &  & \lambda _{n}
\end{bmatrix}=\begin{bmatrix}
| & | &   & | \\
\lambda _{1}\vec{x}_1 & \lambda _{2}\vec{x}_2 & \cdots & \lambda_{n}\vec{x}_n \\
| & | &   & | 
\end{bmatrix}
\end{equation*}
We can complete this half of the proof by comparing columns, and noting that 
\begin{equation}
A \vec{x}_i = \lambda_i \vec{x}_i 
\end{equation}
for $i=1,\ldots,n$ since the $ \vec{x}_i$ are eigenvectors of $A$ and the $\lambda_i$ are corresponding eigenvalues of $A$.

Conversely, suppose $A$ is diagonalizable so that $P^{-1}AP=D.$ Let
\begin{equation*}
P=\begin{bmatrix}
| & | &   & | \\
\vec{x}_1 & \vec{x}_2  & \cdots & \vec{x}_n \\
| & | &   & |
\end{bmatrix}
\end{equation*}
 where the columns are the vectors $\vec{x}_i$ and
\begin{equation*}
D=\begin{bmatrix}
\lambda _{1} &  & 0 \\
& \ddots &  \\
0 &  & \lambda _{n}
\end{bmatrix}
\end{equation*}
Then
\begin{equation*}
AP=PD=\begin{bmatrix}
| & | &   & | \\
\vec{x}_1 & \vec{x}_2  & \cdots & \vec{x}_n \\
| & | &   & |
\end{bmatrix} \begin{bmatrix}
\lambda _{1} &  & 0 \\
& \ddots &  \\
0 &  & \lambda _{n}
\end{bmatrix}
\end{equation*}
and so
\begin{equation*}
\begin{bmatrix}
| & | &   & | \\
A\vec{x}_1 & A\vec{x}_2  & \cdots & A\vec{x}_n \\
| & | &   & |
\end{bmatrix} =\begin{bmatrix}
| & | &   & | \\
\lambda _{1}\vec{x}_1 & \lambda _{2}\vec{x}_2 & \cdots & \lambda_{n}\vec{x}_n \\
| & | &   & | 
\end{bmatrix}
\end{equation*}
showing the $\vec{x}_i$ are eigenvectors of $A$ and the $\lambda _{i}$
are eigenvalues.
%% Now the $x_{k}$ form a basis for $\mathbb{R}^{n}$
%%because the matrix $S$ having these vectors as columns is given to be
%%invertible.
\end{proof}

Notice that because the matrix $P$ defined above is invertible it follows that the set of eigenvectors of $A$, $\left\{ \vec{x}_1 , \vec{x}_2 , \ldots, , \vec{x}_n  \right\}$, is a basis of $\mathbb{R}^n$.

We demonstrate the concept given in the above theorem in the next example. Note that not only
are the columns of the matrix $P$ formed by eigenvectors, but $P$ must
be invertible, and therefore must consist of a linearly independent set of eigenvectors. 
%We achieve this by using basic eigenvectors for the columns of $P$.

\begin{example}\label{ex:diagonalizematrix}
Let
\begin{equation*}
A=\begin{bmatrix}
2 & 0 & 0 \\
1 & 4 & -1 \\
-2 & -4 & 4
\end{bmatrix}
\end{equation*}
 Find an invertible matrix $P$ and a diagonal matrix $D$ such that $P^{-1}AP=D$.

\begin{explanation}
We will use eigenvectors of $A$ as the columns of $P$, and
the corresponding eigenvalues of $A$ as the diagonal entries of $D$. The eigenvalues of $A$ are $\lambda_1 =2,\lambda_2 = 2$, and $\lambda_3 = 6$.   We leave these computations as exercises, as well as the computations to find a basis for each eigenspace.  

One possible basis for $\mathcal{S}_2$, the eigenspace corresponding to $2$, is 
$\left\{
\begin{bmatrix}
-2 \\
1 \\
0
\end{bmatrix},
\begin{bmatrix}
1 \\
0 \\
1
\end{bmatrix}
\right\}$, 
while a basis for $\mathcal{S}_6$ is given by 
$\left\{\begin{bmatrix}
0 \\
1 \\
-2
\end{bmatrix}\right\}$.

We construct the matrix $P$ by using these basis elements as columns.
\begin{equation*}
P=\begin{bmatrix}
-2 & 1 & 0 \\
1 & 0 & 1 \\
0 & 1 & -2
\end{bmatrix}
\end{equation*}
You can verify (See Practice Problem ) that
\begin{equation*}
P^{-1}=\begin{bmatrix}
-1/4 & 1/2 & 1/4 \\
1/2 & 1 & 1/2 \\
1/4 & 1/2 & -1/4
\end{bmatrix}
\end{equation*}
Thus,
\begin{eqnarray*}
P^{-1}AP &=&\begin{bmatrix}
-1/4 & 1/2 & 1/4 \\
1/2 & 1 & 1/2 \\
1/4 & 1/2 & -1/4
\end{bmatrix} \begin{bmatrix}
2 & 0 & 0 \\
1 & 4 & -1 \\
-2 & -4 & 4
\end{bmatrix} \begin{bmatrix}
-2 & 1 & 0 \\
1 & 0 & 1 \\
0 & 1 & -2
\end{bmatrix} \\
&=&\begin{bmatrix}
2 & 0 & 0 \\
0 & 2 & 0 \\
0 & 0 & 6
\end{bmatrix}
\end{eqnarray*}
You can see that the result here is a diagonal matrix where the entries on the main diagonal are the eigenvalues of $A$. Notice that eigenvalues on the main diagonal {\it must} be in the same order as the corresponding eigenvectors in $P$.
\end{explanation}
\end{example}

It is often easier to work with matrices that are diagonalizable, as the next Exploration demonstrates.  

\begin{exploration}\label{exp:motivate_diagonalization}
Let $A=\begin{bmatrix}
2 & 0 & 0 \\
1 & 4 & -1 \\
-2 & -4 & 4
\end{bmatrix}$ and let $D=\begin{bmatrix}
2 & 0 & 0 \\
0 & 2 & 0 \\
0 & 0 & 6
\end{bmatrix}$.  
Would it be easier to compute $A^5$ or $D^5$ if you had to do so by hand, without a computer?  Certainly $D^5$ is easier, due to the number of zero entries!  Let's use Octave to compute both.

%\begin{octave}
A=[2 0 0; 1 4 -1; -2 -4 4]
D=diag([2 2 6]) 
%HARD = A^5
%EZ = D^5
%\end{octave}

We see that raising a diagonal matrix to a power amounts to simply raising each entry to that same power, whereas computing $A^5$ requires many more calculations.  However, we learned in Example~\ref{ex:diagonalizematrix} that $A$ is similar to $D$, and we can use this to make our computation easier.  This is because
\begin{align*}
    A^5&=\left(PDP^{-1}\right)^5 \\
       &=(PDP^{-1})(PDP^{-1})(PDP^{-1})(PDP^{-1})(PDP^{-1}) \\
       &=PD(P^{-1}P)D(P^{-1}P)D(P^{-1}P)D(P^{-1}P)DP^{-1} \\
       &=PD(I)D(I)D(I)D(I)DP^{-1} \\
       &=PD^5P^{-1}
\end{align*}
With this in mind, it is not as daunting to calculate $A^5$ by hand.  We can compute the product $PD^5$ quite easily since $D^5$ is diagonal, as we learned in Exploration~\ref{init:multiplydiag}.  That leaves just one product of $3 \times 3$ matrices to compute by hand to compute $A^5$.  And the savings in work would certainly be more pronounced for larger matrices or for powers larger that 5.

%\begin{octave}
A=[2 0 0; 1 4 -1; -2 -4 4]
D=diag([2 2 6])
P=[-2 1 0; 1 0 1; 0 1 -2]
%EZ = D^5
%P*EZ*P^{-1}
%HARD = A^5
%\end{octave}
\end{exploration}  

In Exploration~\ref{exp:motivate_diagonalization}, because matrix $A$ was diagonalizable, we were able to cut down on computations.  When we chose to work with $D$ and $P$ instead of $A$ we worked with the eigenvalues and eigenvectors of $A$.  Each column of $P$ is an eigenvector of $A$, and so we repeatedly made use of the following theorem (with $m=5$).

\begin{theorem}\label{th:eigpowers}
Let $A$ be an $n \times n$ matrix and suppose $A\vec{x}=\lambda \vec{x}$.  Then $A^m \vec{x} = \lambda^m \vec{x}$
\end{theorem}

\begin{proof}
We prove this theorem by induction on $m$.  Clearly $A^m \vec{x} = \lambda^m \vec{x}$ holds when $m=1$, as that was given.  For the inductive step, suppose that we know $A^{m-1} \vec{x} = \lambda^{m-1} \vec{x}$.  Then
\begin{align*}
    A^m \vec{x} &= (A A^{m-1}) \vec{x} \\
          &= A (A^{m-1} \vec{x}) \\
          &= A (\lambda^{m-1} \vec{x}) \\
          &= \lambda^{m-1} A\vec{x} \\
          &= \lambda^{m-1} \lambda \vec{x} \\
          &= \lambda^m \vec{x}
\end{align*}
as desired.
\end{proof}

Matrix $A$ from the explorations had a repeated eigenvalue of 2.  The next theorem and corollary show that matrices which have \emph{distinct} eigenvalues (where none are repeated) have desirable properties.

\begin{theorem}\label{th:linindepeigenvectors}
Let $A$ be an $n\times n$ matrix, and suppose that $A$
has distinct eigenvalues $\lambda_1, \lambda_2, \ldots, \lambda_m$.
For each $i$, let $\vec{x}_i$ be a $\lambda_i$-eigenvector of $A$.
Then $\{ \vec{x}_1, \vec{x}_2, \ldots, \vec{x}_m\}$ is
linearly independent.
\end{theorem}

\begin{proof}
We prove this by induction on $m$, the number of vectors in the set. If $m = 1$, then $\{\vec{x}_{1}\}$ is a linearly independent set because $\vec{x}_{1} \neq \vec{0}$. In general, suppose we have established that the theorem is true for some $m \geq 1$. Given eigenvectors $\{\vec{x}_{1}, \vec{x}_{2}, \dots, \vec{x}_{m+1}\}$, suppose
\begin{equation}
\label{eq:thm_proof_5_5_4_1}
c_1\vec{x}_1 + c_2\vec{x}_2 + \dots + c_{m+1}\vec{x}_{m+1} = \vec{0}.
\end{equation}
We must show that each $c_{i} = 0$. Multiply both sides of (\ref{eq:thm_proof_5_5_4_1}) on the left by $A$ and use the fact that $A\vec{x}_{i} = \lambda_{i}\vec{x}_{i}$ to get
\begin{equation}
\label{eq:thm_proof_5_5_4_2}
c_1\lambda_1\vec{x}_1 + c_2\lambda_2\vec{x}_2 + \dots + c_{m+1}\lambda_{m+1}\vec{x}_{m+1} = \vec{0}.
\end{equation}
If we multiply (\ref{eq:thm_proof_5_5_4_1}) by $\lambda_{1}$ and subtract the result from (\ref{eq:thm_proof_5_5_4_2}), the first terms cancel and we obtain
\begin{equation*}
c_2(\lambda_2 - \lambda_1)\vec{x}_2 + c_3(\lambda_3 - \lambda_1)\vec{x}_3 + \dots + c_{m+1}(\lambda_{m+1} - \lambda_1)\vec{x}_{m+1} = \vec{0}.
\end{equation*}
Since $\vec{x}_{2}, \vec{x}_{3}, \dots, \vec{x}_{m+1}$
correspond to distinct eigenvalues $\lambda_{2}, \lambda_{3}, \dots, \lambda_{m+1}$, the set $\{\vec{x}_{2}, \vec{x}_{3}, \dots, \vec{x}_{m+1}\}$ is linearly independent by the induction hypothesis. Hence,
\begin{equation*}
c_2(\lambda_2 - \lambda_1) = 0, \quad c_3(\lambda_3 - \lambda_1) = 0, \quad \dots, \quad c_{m+1}(\lambda_{m+1} - \lambda_1) = 0
\end{equation*}
and so $c_{2} = c_{3} = \dots = c_{m+1} = 0$ because the $\lambda_{i}$ are distinct. It follows that (\ref{eq:thm_proof_5_5_4_1}) becomes $c_{1}\vec{x}_{1} = \vec{0}$, which implies that $c_{1} = 0$ because $\vec{x}_{1} \neq \vec{0}$, and the proof is complete. 
\end{proof}

The corollary that follows from this theorem gives a useful tool in determining if $A$ is diagonalizable.

\begin{corollary}\label{th:distincteigenvalues}
Let $A$ be an $n \times n$ matrix and suppose it has $n$ distinct eigenvalues. Then it follows that $A$ is diagonalizable.
\end{corollary}

The corollary tells us that many matrices can be diagonalized in this way.
\begin{definition}\label{def:eigdecomposition}
If we are able to diagonalize $A$, say $A=PDP^{-1}$, we say that $PDP^{-1}$ is an \dfn{eigenvalue decomposition} of $A$.
\end{definition}
Not every matrix has an eigenvalue decomposition. Sometimes we cannot find an invertible matrix $P$ such that $P^{-1}AP=D$.  Consider the following example.

\begin{example}\label{ex:impossiblediagonalize}
Let
\begin{equation*}
A =
\begin{bmatrix}
1 & 1 \\
0 & 1
\end{bmatrix}
\end{equation*}
If possible, find an invertible matrix $P$ and a diagonal matrix $D$ so that $P^{-1}AP=D$.


\begin{explanation}
We see immediately (how?) that the eigenvalues of $A$ are $\lambda_1 =1$ and  $\lambda_2=1$.
To find $P$, the next step would be to find a basis for the corresponding eigenspace $\mathcal{S}_1$.  We solve the equation $\left( A - \lambda I \right) \vec{x} = \vec{0}$.
Writing this equation as an augmented matrix, we already have a matrix in row echelon form:
\begin{equation*}
\left[\begin{array}{cc|c}  
0 & -1 & 0 \\
0 & 0 & 0
 \end{array}\right]
\end{equation*}
We see that the eigenvectors in $\mathcal{S}_1$ are of the form
$$
\begin{bmatrix}
t \\
0
\end{bmatrix}
=t\begin{bmatrix}
1 \\
0
\end{bmatrix},
$$
so a basis for the eigenspace $\mathcal{S}_1$ is given by 
$\begin{bmatrix}
1 \\
0
\end{bmatrix}$.
It is easy to see that we cannot form an invertible matrix $P$, because any two eigenvectors will be of the form 
$\begin{bmatrix}
t \\
0
\end{bmatrix}$,
and so the second row of $P$ would have a row of zeros, and $P$ could not be invertible.  Hence $A$ cannot be diagonalized.
\end{explanation}
\end{example}
We saw earlier in Corollary \ref{th:distincteigenvalues} that an $n \times n$ matrix with $n$ distinct eigenvalues is diagonalizable.  It turns out that there are other useful diagonalizability tests.

Recall that the \dfn{algebraic multiplicity} of an eigenvalue $\lambda$ is the number of times that it occurs as a root of the characteristic polynomial.

\begin{definition}\label{def:geommulteig}
The \dfn{geometric multiplicity} of an eigenvalue $\lambda$ is the dimension of the corresponding eigenspace $\mathcal{S}_\lambda$.
\end{definition}

Consider now the following lemma.

\begin{lemma}\label{lemma:dimeigenspace}
Let $A$ be an $n\times n$ matrix, and let $\mathcal{S}_{\lambda_1}$ be the eigenspace corresponding to the eigenvalue $\lambda_1$ which has algebraic multiplicity $m$.  Then
$$\mbox{dim}(\mathcal{S}_{\lambda_1})\leq m$$
\end{lemma}

In other words, the geometric multiplicity of an eigenvalue is less than or equal to the algebraic multiplicity of that same eigenvalue.

\begin{proof}
Let $k$ be the geometric multiplicity of $\lambda_1$, i.e., $k=\mbox{dim}(\mathcal{S}_{\lambda_1})$.  Suppose $\left\{\vec{x}_1, \vec{x}_2, \ldots ,\vec{x}_k\right\}$ is a basis for the eigenspace $\mathcal{S}_{\lambda_1}$.  Let $P$ be any invertible matrix having $\vec{x}_1,  \vec{x}_2, \ldots ,\vec{x}_k$ as its first $k$ columns, say 
$$P=\begin{bmatrix}
| & | &  & | & | & & | \\
\vec{x}_1 & \vec{x}_2 & \cdots & \vec{x}_k & \vec{x}_{k+1} & \cdots & \vec{x}_n \\
| & | &  & | & | & & |
\end{bmatrix}.$$  
In block form we may write 
$$P=\begin{bmatrix}
B&C
\end{bmatrix} \quad \text{and} \quad P^{-1}=\begin{bmatrix}
D \\
E
\end{bmatrix}$$
where $B$ is $n \times k$, $C$ is  $n \times (n-k)$, $D$ is $k \times n$, and $E$ is $(n-k) \times n$.  We observe
$I_n = P^{-1}P = \left[\begin{array}{c|c}  
DB & DC \\ \hline
EB & EC
 \end{array}\right] $. 
This implies
$$DB = I_k,\quad DC=O_{k\,\,n-k},\quad EB = O_{n-k\,\,k} \quad\text{ and }\quad EC = I_{n-k}$$
Therefore,
\begin{equation*}
P^{-1}AP=\begin{bmatrix}
D \\
E
\end{bmatrix} 
A
\begin{bmatrix}
B&C
\end{bmatrix} =
\left[\begin{array}{c|c}  
DAB & DAC \\ \hline
EAB & EAC
 \end{array}\right]
\end{equation*}
\begin{equation*}
=  \left[\begin{array}{c|c}  
\lambda_1 DB & DAC \\ \hline
\lambda_1 EB & EAC
 \end{array}\right]  
=  \left[\begin{array}{c|c}  
\lambda_1 I_k & DAC \\ \hline
O & EAC
 \end{array}\right]  
\end{equation*}
We finish the proof by comparing the characteristic polynomials on both sides of this equation, and making use of the fact that similar matrices have the same characteristic polynomials.  
$$\det(A-\lambda I) = \det(P^{-1}AP-\lambda I)=(\lambda_1 - \lambda)^k \det(EAC)$$
We see that the characteristic polynomial of $A$ has $(\lambda_1 - \lambda)^k$ as a factor.  This tells us that algebraic multiplicity of $\lambda_1$ is at least $k$, proving the desired inequality. 
\end{proof}

This result tells us that if $\lambda$ is an eigenvalue of $A$, then
the number of linearly independent $\lambda$-eigenvectors
is never more than the multiplicity of $\lambda$. We now use this fact to provide a useful diagonalizability condition.

\begin{theorem}\label{th:diagonalizability}
Let $A$ be an $n \times n$ matrix $A$. Then $A$ is diagonalizable if and only if for each eigenvalue $\lambda$ of $A$, the algebraic multiplicity of $\lambda$ is equal to the geometric multiplicity of $\lambda$.
\end{theorem}

\begin{proof}
Suppose $A$ is diagonalizable and let $\lambda_1, \ldots, \lambda_t$ be the distinct eigenvalues of $A$, with algebraic multiplicities $m_1, \ldots, m_t$, respectively and geometric multiplicities $k_1, \ldots, k_t$, respectively.  Since $A$ is diagonalizable, Theorem \ref{th:eigenvectorsanddiagonalizable} implies that $ k_1+\cdots+k_t=n$.  By applying Lemma \ref{lemma:dimeigenspace} $t$ times, we have
$$n = k_1+\cdots+k_t \le m_1+\cdots+m_t = n,$$
which is only possible if $k_i=m_i$ for $i=1,\ldots,t$.

Conversely, if the geometric multiplicity equals the algebraic multiplicity of each eigenvalue, then obtaining a basis for each eigenspace yields $n$ eigenvectors.  Applying Theorem \ref{th:linindepeigenvectors}, we know that these $n$ eigenvectors are linearly independent, so Theorem \ref{th:eigenvectorsanddiagonalizable} implies that $A$ is diagonalizable.
\end{proof}

\section*{Practice Problems}
\begin{problem}
In this exercise you will "fill in the details" of Example \ref{ex:diagonalizematrix}.
\begin{problem}\label{prob:ex:diagonalizematrix1}
Find the eigenvalues of matrix 
\begin{equation*}
A=\begin{bmatrix}
2 & 0 & 0 \\
1 & 4 & -1 \\
-2 & -4 & 4
\end{bmatrix}
\end{equation*}
\end{problem} 
\begin{problem}\label{prob:ex:diagonalizematrix2}
Find a basis for each eigenspace of the matrix $A$.
\end{problem}
\begin{problem}\label{prob:ex:diagonalizematrix3}
Compute the inverse of \begin{equation*}
P=
\begin{bmatrix}
-2 & 1 & 0 \\
1 & 0 & 1 \\
0 & 1 & -2
\end{bmatrix}
\end{equation*}
\end{problem}
\begin{problem}\label{prob:ex:diagonalizematrix5}
Compute $D=P^{-1}AP$
\end{problem}
\begin{problem}\label{prob:ex:diagonalizematrix4}
Show that computing the inverse of $P$ is not really necessary by comparing the products  $PD$ and $AP$.
\end{problem}
  \end{problem}

\begin{problem}
In each case, decide whether the matrix $A$ is diagonalizable. If so, find $P$ such that $P^{-1}AP$ is diagonal.

\begin{problem}\label{prb:diagonalizable}
\item $\begin{bmatrix}
1 & 0 & 0 \\
1 & 2 & 1 \\
0 & 0 & 1
\end{bmatrix}$
\wordChoice{[correct]\choice{Diagonalizable}\choice{Not Diagonalizable}}
\item $\begin{bmatrix}
3 &  0 & 6 \\
0 & -3 & 0 \\
5 &  0 & 2
\end{bmatrix}$
\wordChoice{[correct]\choice{Diagonalizable}\choice{Not Diagonalizable}}
\item $\begin{bmatrix}
 3 &  1 &  6 \\
 2 &  1 &  0 \\
-1 &  0 & -3
\end{bmatrix}$
\wordChoice{[correct]\choice{Diagonalizable}\choice{Not Diagonalizable}}
\item $\begin{bmatrix}
4 & 0 & 0 \\
0 & 2 & 2 \\
2 & 3 & 1
\end{bmatrix}$
\wordChoice{\choice{Diagonalizable}\choice[correct]{Not Diagonalizable}}
\end{problem}

%\setcounter{enumi}{1}
%problem  Yes, $ P =
%\leftB \begin{array}{rrr}
%-1 & 0 & 6 \\
% 0 & 1 & 0 \\
% 1 & 0 & 5
%\end{array} \rightB$, $P^{-1}AP =
%\leftB \begin{array}{rrr}
%-3 &  0 & 0 \\
% 0 & -3 & 0 \\
% 0 &  0 & 8
%\end{array} \rightB$

%\setcounter{enumi}{3}
%problem  No, $c_{A}(x) = (x + 1)(x - 4)^{2}$ so $\lambda = 4$ has multiplicity 2. But $\func{dim}(E_{4}) = 1$ so Theorem~\ref{thm:016250} applies.
\end{problem}

\begin{problem}\label{prb:upper_triangular_case}
Let $A$ denote an $n \times n$ upper triangular matrix.
\begin{enumerate}

\item If all the main diagonal entries of $A$ are distinct, show that $A$ is diagonalizable.

\item If all the main diagonal entries of $A$ are equal, show that $A$ is diagonalizable only if it is \textit{already} diagonal.
Click the arrow to see the answer.
\begin{expandable}
The eigenvalues of $A$ are all equal (they are the diagonal elements), so if $P^{-1}AP = D$ is diagonal, then $D = \lambda I$. Hence $A = P^{-1}(\lambda I)P = \lambda I$.
\end{expandable}

\item Show that $\begin{bmatrix}
1 & 0 & 1 \\
0 & 1 & 0 \\
0 & 0 & 2
\end{bmatrix}$ is diagonalizable but that $\begin{bmatrix}
 1 & 1 & 0 \\
 0 & 1 & 0 \\
 0 & 0 & 2
 \end{bmatrix}$ is not diagonalizable.
 
\end{enumerate}

\end{problem}

\begin{problem}\label{prb:det_and_tr_diagonalizable}
Let $A$ be a diagonalizable $n \times n$ matrix with eigenvalues $\lambda_{1}, \lambda_{2}, \dots, \lambda_{n}$ (including multiplicities). Show that:

\begin{enumerate}
\item $\det A = \lambda_{1}\lambda_{2}\cdots \lambda_{n}$
\item $\mbox{tr} A = \lambda_{1} + \lambda_{2} + \cdots + \lambda_{n}$
\end{enumerate}
\end{problem}

\begin{problem}\label{prb:A_sim_A^T_diagonalizable}
\begin{enumerate}

\item Show that two diagonalizable matrices are similar if and only if they have the same eigenvalues with the same multiplicities.

\item If $A$ is diagonalizable, show that $A \sim A^{T}$.

\item Show that $A \sim A^{T}$ if
 $A = \begin{bmatrix}
 1 & 1 \\
 0 & 1
 \end{bmatrix}$

\end{enumerate}
\end{problem}

\section*{Text Source}
The text in this section is a compilation of material from Section 7.2.1 of Ken Kuttler's \href{https://open.umn.edu/opentextbooks/textbooks/a-first-course-in-linear-algebra-2017}{\it A First Course in Linear Algebra} (CC-BY) and Section 5.5 of Keith Nicholson's \href{https://open.umn.edu/opentextbooks/textbooks/linear-algebra-with-applications}{\it Linear Algebra with Applications} (CC-BY-NC-SA).

Ken Kuttler, {\it  A First Course in Linear Algebra}, Lyryx 2017, Open Edition, p. 362-364.

Many of the Practice Problems are Exercises from 
W. Keith Nicholson, {\it Linear Algebra with Applications}, Lyryx 2018, Open Edition, pp. 298-310.

\end{document}
