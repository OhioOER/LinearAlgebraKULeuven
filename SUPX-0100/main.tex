\documentclass{ximera}
%%% Begin Laad packages

\makeatletter
\@ifclassloaded{xourse}{%
    \typeout{Start loading preamble.tex (in a XOURSE)}%
    \def\isXourse{true}   % automatically defined; pre 112022 it had to be set 'manually' in a xourse
}{%
    \typeout{Start loading preamble.tex (NOT in a XOURSE)}%
}
\makeatother

\def\isEn\true 

\pgfplotsset{compat=1.16}

\usepackage{currfile}

% 201908/202301: PAS OP: babel en doclicense lijken problemen te veroorzaken in .jax bestand
% (wegens syntax error met toegevoegde \newcommands ...)
\pdfOnly{
    \usepackage[type={CC},modifier={by-nc-sa},version={4.0}]{doclicense}
    %\usepackage[hyperxmp=false,type={CC},modifier={by-nc-sa},version={4.0}]{doclicense}
    %%% \usepackage[dutch]{babel}
}



\usepackage[utf8]{inputenc}
\usepackage{morewrites}   % nav zomercursus (answer...?)
\usepackage{multirow}
\usepackage{multicol}
\usepackage{tikzsymbols}
\usepackage{ifthen}
%\usepackage{animate} BREAKS HTML STRUCTURE USED BY XIMERA
\usepackage{relsize}

\usepackage{eurosym}    % \euro  (€ werkt niet in xake ...?)
\usepackage{fontawesome} % smileys etc

% Nuttig als ook interactieve beamer slides worden voorzien:
\providecommand{\p}{} % default nothing ; potentially usefull for slides: redefine as \pause
%providecommand{\p}{\pause}

    % Layout-parameters voor het onderschrift bij figuren
\usepackage[margin=10pt,font=small,labelfont=bf, labelsep=endash,format=hang]{caption}
%\usepackage{caption} % captionof
%\usepackage{pdflscape}    % landscape environment

% Met "\newcommand\showtodonotes{}" kan je todonotes tonen (in pdf/online)
% 201908: online werkt het niet (goed)
\providecommand\showtodonotes{disable}
\providecommand\todo[1]{\typeout{TODO #1}}
%\usepackage[\showtodonotes]{todonotes}
%\usepackage{todonotes}

%
% Poging tot aanpassen layout
%
\usepackage{tcolorbox}
\tcbuselibrary{theorems}

%%% Einde laad packages

%%% Begin Ximera specifieke zaken

\graphicspath{
	{../../}
	{../}
	{./}
  	{../../pictures/}
   	{../pictures/}
   	{./pictures/}
	{./explog/}    % M05 in groeimodellen       
}

%%% Einde Ximera specifieke zaken

%
% define softer blue/red/green, use KU Leuven base colors for blue (and dark orange for red ?)
%
% todo: rather redefine blue/red/green ...?
%\definecolor{xmblue}{rgb}{0.01, 0.31, 0.59}
%\definecolor{xmred}{rgb}{0.89, 0.02, 0.17}
\definecolor{xmdarkblue}{rgb}{0.122, 0.671, 0.835}   % KU Leuven Blauw
\definecolor{xmblue}{rgb}{0.114, 0.553, 0.69}        % KU Leuven Blauw
\definecolor{xmgreen}{rgb}{0.13, 0.55, 0.13}         % No KULeuven variant for green found ...

\definecolor{xmaccent}{rgb}{0.867, 0.541, 0.18}      % KU Leuven Accent (orange ...)
\definecolor{kuaccent}{rgb}{0.867, 0.541, 0.18}      % KU Leuven Accent (orange ...)

\colorlet{xmred}{xmaccent!50!black}                  % Darker version of KU Leuven Accent

\providecommand{\blue}[1]{{\color{blue}#1}}    
\providecommand{\red}[1]{{\color{red}#1}}

\renewcommand\CancelColor{\color{xmaccent!50!black}}

% werkt in math en text mode om MATH met oranje (of grijze...)  achtergond te tonen (ook \important{\text{blabla}} lijkt te werken)
%\newcommand{\important}[1]{\ensuremath{\colorbox{xmaccent!50!white}{$#1$}}}   % werkt niet in Mathjax
%\newcommand{\important}[1]{\ensuremath{\colorbox{lightgray}{$#1$}}}
\newcommand{\important}[1]{\ensuremath{\colorbox{orange}{$#1$}}}   % TODO: kleur aanpassen voor mathjax; wordt overschreven infra!


% Uitzonderlijk kan met \pdfnl in de PDF een newline worden geforceerd, die online niet nodig/nuttig is omdat daar de regellengte hoe dan ook niet gekend is.
\ifdefined\HCode%
\providecommand{\pdfnl}{}%
\else%
\providecommand{\pdfnl}{%
  \\%
}%
\fi

% Uitzonderlijk kan met \handoutnl in de handout-PDF een newline worden geforceerd, die noch online noch in de PDF-met-antwoorden nuttig is.
\ifdefined\HCode
\providecommand{\handoutnl}{}
\else
\providecommand{\handoutnl}{%
\ifhandout%
  \nl%
\fi%
}
\fi



% \cellcolor IGNORED by tex4ht ?
% \begin{center} seems not to wordk
    % (missing margin-left: auto;   on tabular-inside-center ???)
%\newcommand{\importantcell}[1]{\ensuremath{\cellcolor{lightgray}#1}}  %  in tabular; usablility to be checked
\providecommand{\importantcell}[1]{\ensuremath{#1}}     % no mathjax2 support for colloring array cells

\pdfOnly{
  \renewcommand{\important}[1]{\ensuremath{\colorbox{kuaccent!50!white}{$#1$}}}
  \renewcommand{\importantcell}[1]{\ensuremath{\cellcolor{kuaccent!40!white}#1}}   
}

%%% Tikz styles


\pgfplotsset{compat=1.16}

\usetikzlibrary{trees,positioning,arrows,fit,shapes,math,calc,decorations.markings,through,intersections,patterns,matrix}

\usetikzlibrary{decorations.pathreplacing,backgrounds}    % 5/2023: from experimental


\usetikzlibrary{angles,quotes}

\usepgfplotslibrary{fillbetween} % bepaalde_integraal
\usepgfplotslibrary{polar}    % oa voor poolcoordinaten.tex

\pgfplotsset{ownstyle/.style={axis lines = center, axis equal image, xlabel = $x$, ylabel = $y$, enlargelimits}} 

\pgfplotsset{
	plot/.style={no marks,samples=50}
}

\newcommand{\xmPlotsColor}{
	\pgfplotsset{
		plot1/.style={darkgray,no marks,samples=100},
		plot2/.style={lightgray,no marks,samples=100},
		plotresult/.style={blue,no marks,samples=100},
		plotblue/.style={blue,no marks,samples=100},
		plotred/.style={red,no marks,samples=100},
		plotgreen/.style={green,no marks,samples=100},
		plotpurple/.style={purple,no marks,samples=100}
	}
}
\newcommand{\xmPlotsBlackWhite}{
	\pgfplotsset{
		plot1/.style={black,loosely dashed,no marks,samples=100},
		plot2/.style={black,loosely dotted,no marks,samples=100},
		plotresult/.style={black,no marks,samples=100},
		plotblue/.style={black,no marks,samples=100},
		plotred/.style={black,dotted,no marks,samples=100},
		plotgreen/.style={black,dashed,no marks,samples=100},
		plotpurple/.style={black,dashdotted,no marks,samples=100}
	}
}


\newcommand{\xmPlotsColorAndStyle}{
	\pgfplotsset{
		plot1/.style={darkgray,no marks,samples=100},
		plot2/.style={lightgray,no marks,samples=100},
		plotresult/.style={blue,no marks,samples=100},
		plotblue/.style={xmblue,no marks,samples=100},
		plotred/.style={xmred,dashed,thick,no marks,samples=100},
		plotgreen/.style={xmgreen,dotted,very thick,no marks,samples=100},
		plotpurple/.style={purple,no marks,samples=100}
	}
}


%\iftikzexport
\xmPlotsColorAndStyle
%\else
%\xmPlotsBlackWhite
%\fi
%%%


%
% Om venndiagrammen te arceren ...
%
\makeatletter
\pgfdeclarepatternformonly[\hatchdistance,\hatchthickness]{north east hatch}% name
{\pgfqpoint{-1pt}{-1pt}}% below left
{\pgfqpoint{\hatchdistance}{\hatchdistance}}% above right
{\pgfpoint{\hatchdistance-1pt}{\hatchdistance-1pt}}%
{
	\pgfsetcolor{\tikz@pattern@color}
	\pgfsetlinewidth{\hatchthickness}
	\pgfpathmoveto{\pgfqpoint{0pt}{0pt}}
	\pgfpathlineto{\pgfqpoint{\hatchdistance}{\hatchdistance}}
	\pgfusepath{stroke}
}
\pgfdeclarepatternformonly[\hatchdistance,\hatchthickness]{north west hatch}% name
{\pgfqpoint{-\hatchthickness}{-\hatchthickness}}% below left
{\pgfqpoint{\hatchdistance+\hatchthickness}{\hatchdistance+\hatchthickness}}% above right
{\pgfpoint{\hatchdistance}{\hatchdistance}}%
{
	\pgfsetcolor{\tikz@pattern@color}
	\pgfsetlinewidth{\hatchthickness}
	\pgfpathmoveto{\pgfqpoint{\hatchdistance+\hatchthickness}{-\hatchthickness}}
	\pgfpathlineto{\pgfqpoint{-\hatchthickness}{\hatchdistance+\hatchthickness}}
	\pgfusepath{stroke}
}
%\makeatother

\tikzset{
    hatch distance/.store in=\hatchdistance,
    hatch distance=10pt,
    hatch thickness/.store in=\hatchthickness,
   	hatch thickness=2pt
}

\colorlet{circle edge}{black}
\colorlet{circle area}{blue!20}


\tikzset{
    filled/.style={fill=green!30, draw=circle edge, thick},
    arceerl/.style={pattern=north east hatch, pattern color=blue!50, draw=circle edge},
    arceerr/.style={pattern=north west hatch, pattern color=yellow!50, draw=circle edge},
    outline/.style={draw=circle edge, thick}
}




%%% Updaten commando's
\def\hoofding #1#2#3{\maketitle}     % OBSOLETE ??

% we willen (bijna) altijd \geqslant ipv \geq ...!
\newcommand{\geqnoslant}{\geq}
\renewcommand{\geq}{\geqslant}
\newcommand{\leqnoslant}{\leq}
\renewcommand{\leq}{\leqslant}

% Todo: (201908) waarom komt er (soms) underlined voor emph ...?
\renewcommand{\emph}[1]{\textit{#1}}

% API commando's

\newcommand{\ds}{\displaystyle}
\newcommand{\ts}{\textstyle}  % tegenhanger van \ds   (Ximera zet PER  DEFAULT \ds!)

% uit Zomercursus-macro's: 
\newcommand{\bron}[1]{\begin{scriptsize} \emph{#1} \end{scriptsize}}     % deprecated ...?


%definities nieuwe commando's - afkortingen veel gebruikte symbolen
\newcommand{\R}{\ensuremath{\mathbb{R}}}
\newcommand{\Rnul}{\ensuremath{\mathbb{R}_0}}
\newcommand{\Reen}{\ensuremath{\mathbb{R}\setminus\{1\}}}
\newcommand{\Rnuleen}{\ensuremath{\mathbb{R}\setminus\{0,1\}}}
\newcommand{\Rplus}{\ensuremath{\mathbb{R}^+}}
\newcommand{\Rmin}{\ensuremath{\mathbb{R}^-}}
\newcommand{\Rnulplus}{\ensuremath{\mathbb{R}_0^+}}
\newcommand{\Rnulmin}{\ensuremath{\mathbb{R}_0^-}}
\newcommand{\Rnuleenplus}{\ensuremath{\mathbb{R}^+\setminus\{0,1\}}}
\newcommand{\N}{\ensuremath{\mathbb{N}}}
\newcommand{\Nnul}{\ensuremath{\mathbb{N}_0}}
\newcommand{\Z}{\ensuremath{\mathbb{Z}}}
\newcommand{\Znul}{\ensuremath{\mathbb{Z}_0}}
\newcommand{\Zplus}{\ensuremath{\mathbb{Z}^+}}
\newcommand{\Zmin}{\ensuremath{\mathbb{Z}^-}}
\newcommand{\Znulplus}{\ensuremath{\mathbb{Z}_0^+}}
\newcommand{\Znulmin}{\ensuremath{\mathbb{Z}_0^-}}
\newcommand{\C}{\ensuremath{\mathbb{C}}}
\newcommand{\Cnul}{\ensuremath{\mathbb{C}_0}}
\newcommand{\Cplus}{\ensuremath{\mathbb{C}^+}}
\newcommand{\Cmin}{\ensuremath{\mathbb{C}^-}}
\newcommand{\Cnulplus}{\ensuremath{\mathbb{C}_0^+}}
\newcommand{\Cnulmin}{\ensuremath{\mathbb{C}_0^-}}
\newcommand{\Q}{\ensuremath{\mathbb{Q}}}
\newcommand{\Qnul}{\ensuremath{\mathbb{Q}_0}}
\newcommand{\Qplus}{\ensuremath{\mathbb{Q}^+}}
\newcommand{\Qmin}{\ensuremath{\mathbb{Q}^-}}
\newcommand{\Qnulplus}{\ensuremath{\mathbb{Q}_0^+}}
\newcommand{\Qnulmin}{\ensuremath{\mathbb{Q}_0^-}}

\newcommand{\perdef}{\overset{\mathrm{def}}{=}}
\newcommand{\pernot}{\overset{\mathrm{notatie}}{=}}
\newcommand\perinderdaad{\overset{!}{=}}     % voorlopig gebruikt in limietenrekenregels
\newcommand\perhaps{\overset{?}{=}}          % voorlopig gebruikt in limietenrekenregels

\newcommand{\degree}{^\circ}


\DeclareMathOperator{\dom}{dom}     % domein
\DeclareMathOperator{\codom}{codom} % codomein
\DeclareMathOperator{\bld}{bld}     % beeld
\DeclareMathOperator{\graf}{graf}   % grafiek
\DeclareMathOperator{\rico}{rico}   % richtingcoëfficient
\DeclareMathOperator{\co}{co}       % coordinaat
\DeclareMathOperator{\gr}{gr}       % graad

\newcommand{\func}[5]{\ensuremath{#1: #2 \rightarrow #3: #4 \mapsto #5}} % Easy to write a function


% Operators
\DeclareMathOperator{\bgsin}{bgsin}
\DeclareMathOperator{\bgcos}{bgcos}
\DeclareMathOperator{\bgtan}{bgtan}
\DeclareMathOperator{\bgcot}{bgcot}
\DeclareMathOperator{\bgsinh}{bgsinh}
\DeclareMathOperator{\bgcosh}{bgcosh}
\DeclareMathOperator{\bgtanh}{bgtanh}
\DeclareMathOperator{\bgcoth}{bgcoth}

% Oude \Bgsin etc deprecated: gebruik \bgsin, en herdefinieer dat als je Bgsin wil!
%\DeclareMathOperator{\cosec}{cosec}    % not used? gebruik \csc en herdefinieer

% operatoren voor differentialen: to be verified; 1/2020: inconsequent gebruik bij afgeleiden/integralen
\renewcommand{\d}{\mathrm{d}}
\newcommand{\dx}{\d x}
\newcommand{\dd}[1]{\frac{\mathrm{d}}{\mathrm{d}#1}}
\newcommand{\ddx}{\dd{x}}

% om in voorbeelden/oefeningen de notatie voor afgeleiden te kunnen kiezen
% Usage: \afg{(2\sin(x))}  (en wordt d/dx, of accent, of D )
%\newcommand{\afg}[1]{{#1}'}
\newcommand{\afg}[1]{\left(#1\right)'}
%\renewcommand{\afg}[1]{\frac{\mathrm{d}#1}{\mathrm{d}x}}   % include in relevant exercises ...
%\renewcommand{\afg}[1]{D{#1}}

%
% \xmxxx commands: Extra KU Leuven functionaliteit van, boven of naast Ximera
%   ( Conventie 8/2019: xm+nederlandse omschrijving, maar is niet consequent gevolgd, en misschien ook niet erg handig !)
%
% (Met een minimale ximera.cls en preamble.tex zou een bruikbare .pdf moeten kunnen worden gemaakt van eender welke ximera)
%
% Usage: \xmtitle[Mijn korte abstract]{Mijn titel}{Mijn abstract}
% Eerste command na \begin{document}:
%  -> definieert de \title
%  -> definieert de abstract
%  -> doet \maketitle ( dus: print de hoofding als 'chapter' of 'sectie')
% Optionele parameter geeft eenn kort abstract (die met de globale setting \xmshortabstract{} al dan niet kan worden geprint.
% De optionele korte abstract kan worden gebruikt voor pseudo-grappige abtsarts, dus dus globaal al dan niet kunnen worden gebuikt...
% Globale settings:
%  de (optionele) 'korte abstract' wordt enkele getoond als \xmshortabstract is gezet
\providecommand\xmshortabstract{} % default: print (only!) short abstract if present
\newcommand{\xmtitle}[3][]{
	\title{#2}
	\begin{abstract}
		\ifdefined\xmshortabstract
		\ifstrempty{#1}{%
			#3
		}{%
			#1
		}%
		\else
		#3
		\fi
	\end{abstract}
	\maketitle
}

% 
% Kleine grapjes: moeten zonder verder gevolg kunnen worden verwijderd
%
%\newcommand{\xmopje}[1]{{\small#1{\reversemarginpar\marginpar{\Smiley}}}}   % probleem in floats!!
\newtoggle{showxmopje}
\toggletrue{showxmopje}

\newcommand{\xmopje}[1]{%
   \iftoggle{showxmopje}{#1}{}%
}


% -> geef een abstracte-formule-met-rechts-een-concreet-voorbeeld
% VB:  \formulevb{a^2+b^2=c^2}{3^2+4^2=5^2}
%
\ifdefined\HCode
\NewEnviron{xmdiv}[1]{\HCode{\Hnewline<div class="#1">\Hnewline}\BODY{\HCode{\Hnewline</div>\Hnewline}}}
\else
\NewEnviron{xmdiv}[1]{\BODY}
\fi

\providecommand{\formulevb}[2]{
	{\centering

    \begin{xmdiv}{xmformulevb}    % zie css voor online layout !!!
	\begin{tabular}{lcl}
		\important{#1}
		&  &
		Vb: $#2$
		\end{tabular}
	\end{xmdiv}

	}
}

\ifdefined\HCode
\providecommand{\vb}[1]{%
    \HCode{\Hnewline<span class="xmvb">}#1\HCode{</span>\Hnewline}%
}
\else
\providecommand{\vb}[1]{
    \colorbox{blue!10}{#1}
}
\fi

\ifdefined\HCode
\providecommand{\xmcolorbox}[2]{
	\HCode{\Hnewline<div class="xmcolorbox">\Hnewline}#2\HCode{\Hnewline</div>\Hnewline}
}
\else
\providecommand{\xmcolorbox}[2]{
  \cellcolor{#1}#2
}
\fi


\ifdefined\HCode
\providecommand{\xmopmerking}[1]{
 \HCode{\Hnewline<div class="xmopmerking">\Hnewline}#1\HCode{\Hnewline</div>\Hnewline}
}
\else
\providecommand{\xmopmerking}[1]{
	{\footnotesize #1}
}
\fi
% \providecommand{\voorbeeld}[1]{
% 	\colorbox{blue!10}{$#1$}
% }



% Hernoem Proof naar Bewijs, nodig voor HTML versie
\renewcommand*{\proofname}{Bewijs}

% Om opgave van oefening (wordt niet geprint bij oplossingenblad)
% (to be tested test)
\NewEnviron{statement}{\BODY}

% Environment 'oplossing' en 'uitkomst'
% voor resp. volledige 'uitwerking' dan wel 'enkel eindresultaat'
% geimplementeerd via feedback, omdat er in de ximera-server adhoc feedback-code is toegevoegd
%% Niet tonen indien handout
%% Te gebruiken om volledige oplossingen/uitwerkingen van oefeningen te tonen
%% \begin{oplossing}        De optelling is commutatief \end{oplossing}  : verschijnt online enkel 'op vraag'
%% \begin{oplossing}[toon]  De optelling is commutatief \end{oplossing}  : verschijnt steeds onmiddellijk online (bv te gebruiken bij voorbeelden) 

\ifhandout%
    \NewEnviron{oplossing}[1][onzichtbaar]%
    {%
    \ifthenelse{\equal{\detokenize{#1}}{\detokenize{toon}}}
    {
    \def\PH@Command{#1}% Use PH@Command to hold the content and be a target for "\expandafter" to expand once.

    \begin{trivlist}% Begin the trivlist to use formating of the "Feedback" label.
    \item[\hskip \labelsep\small\slshape\bfseries Oplossing% Format the "Feedback" label. Don't forget the space.
    %(\texttt{\detokenize\expandafter{\PH@Command}}):% Format (and detokenize) the condition for feedback to trigger
    \hspace{2ex}]\small%\slshape% Insert some space before the actual feedback given.
    \BODY
    \end{trivlist}
    }
    {  % \begin{feedback}[solution]   \BODY     \end{feedback}  }
    }
    }    
\else
% ONLY for HTML; xmoplossing is styled with css, and is not, and need not be a LaTeX environment
% THUS: it does NOT use feedback anymore ...
%    \NewEnviron{oplossing}{\begin{expandable}{xmoplossing}{\nlen{Toon uitwerking}{Show solution}}{\BODY}\end{expandable}}
    \newenvironment{oplossing}[1][onzichtbaar]
   {%
       \begin{expandable}{xmoplossing}{}
   }
   {%
   	   \end{expandable}
   } 
%     \newenvironment{oplossing}[1][onzichtbaar]
%    {%
%        \begin{feedback}[solution]   	
%    }
%    {%
%    	   \end{feedback}
%    } 
\fi

\ifhandout%
    \NewEnviron{uitkomst}[1][onzichtbaar]%
    {%
    \ifthenelse{\equal{\detokenize{#1}}{\detokenize{toon}}}
    {
    \def\PH@Command{#1}% Use PH@Command to hold the content and be a target for "\expandafter" to expand once.

    \begin{trivlist}% Begin the trivlist to use formating of the "Feedback" label.
    \item[\hskip \labelsep\small\slshape\bfseries Uitkomst:% Format the "Feedback" label. Don't forget the space.
    %(\texttt{\detokenize\expandafter{\PH@Command}}):% Format (and detokenize) the condition for feedback to trigger
    \hspace{2ex}]\small%\slshape% Insert some space before the actual feedback given.
    \BODY
    \end{trivlist}
    }
    {  % \begin{feedback}[solution]   \BODY     \end{feedback}  }
    }
    }    
\else
\ifdefined\HCode
   \newenvironment{uitkomst}[1][onzichtbaar]
    {%
        \begin{expandable}{xmuitkomst}{}%
    }
    {%
    	\end{expandable}%
    } 
\else
  % Do NOT print 'uitkomst' in non-handout
  %  (presumably, there is also an 'oplossing' ??)
  \newenvironment{uitkomst}[1][onzichtbaar]{}{}
\fi
\fi

%
% Uitweidingen zijn extra's die niet redelijkerwijze tot de leerstof behoren
% Uitbreidingen zijn extra's die wel redelijkerwijze tot de leerstof van bv meer geavanceerde versies kunnen behoren (B-programma/Wiskundestudenten/...?)
% Nog niet voorzien: design voor verschillende versies (A/B programma, BIO, voorkennis/ ...)
% Voor 'uitweidingen' is er een environment die online per default is ingeklapt, en in pdf al dan niet kan worden geincluded  (via \xmnouitweiding) 
%
% in een xourse, per default GEEN uitweidingen, tenzij \xmuitweiding expliciet ergens is gezet ...
\ifdefined\isXourse
   \ifdefined\xmuitweiding
   \else
       \def\xmnouitweiding{true}
   \fi
\fi

\ifdefined\xmnouitweiding
\newcounter{xmuitweiding}  % anders error undefined ...  
\excludecomment{xmuitweiding}
\else
\newtheoremstyle{dotless}{}{}{}{}{}{}{ }{}
\theoremstyle{dotless}
\newtheorem*{xmuitweidingnofrills}{}   % nofrills = no accordion; gebruikt dus de dotless theoremstyle!

\newcounter{xmuitweiding}
\newenvironment{xmuitweiding}[1][ ]%
{% 
	\refstepcounter{xmuitweiding}%
    \begin{expandable}{xmuitweiding}{\nlentext{Uitweiding \arabic{xmuitweiding}: #1}{Digression \arabic{xmuitweiding}: #1}}%
	\begin{xmuitweidingnofrills}%
}
{%
    \end{xmuitweidingnofrills}%
    \end{expandable}%
}   
% \newenvironment{xmuitweiding}[1][ ]%
% {% 
% 	\refstepcounter{xmuitweiding}
% 	\begin{accordion}\begin{accordion-item}[Uitweiding \arabic{xmuitweiding}: #1]%
% 			\begin{xmuitweidingnofrills}%
% 			}
% 			{\end{xmuitweidingnofrills}\end{accordion-item}\end{accordion}}   
\fi


\newenvironment{xmexpandable}[1][]{
	\begin{accordion}\begin{accordion-item}[#1]%
		}{\end{accordion-item}\end{accordion}}


% Command that gives a selection box online, but just prints the right answer in pdf
\newcommand{\xmonlineChoice}[1]{\pdfOnly{\wordchoicegiventrue}\wordChoice{#1}\pdfOnly{\wordchoicegivenfalse}}   % deprecated, gebruik onlineChoice ...
\newcommand{\onlineChoice}[1]{\pdfOnly{\wordchoicegiventrue}\wordChoice{#1}\pdfOnly{\wordchoicegivenfalse}}

\newcommand{\TJa}{\nlentext{ Ja }{ Yes }}
\newcommand{\TNee}{\nlentext{ Nee }{ No }}
\newcommand{\TJuist}{\nlentext{ Juist }{ True }}
\newcommand{\TFout}{\nlentext{ Fout }{ False }}

\newcommand{\choiceTrue }{{\renewcommand{\choiceminimumhorizontalsize}{4em}\wordChoice{\choice[correct]{\TJuist}\choice{\TFout}}}}
\newcommand{\choiceFalse}{{\renewcommand{\choiceminimumhorizontalsize}{4em}\wordChoice{\choice{\TJuist}\choice[correct]{\TFout}}}}

\newcommand{\choiceYes}{{\renewcommand{\choiceminimumhorizontalsize}{3em}\wordChoice{\choice[correct]{\TJa}\choice{\TNee}}}}
\newcommand{\choiceNo }{{\renewcommand{\choiceminimumhorizontalsize}{3em}\wordChoice{\choice{\TJa}\choice[correct]{\TNee}}}}

% Optional nicer formatting for wordChoice in PDF

\let\inlinechoiceorig\inlinechoice

%\makeatletter
%\providecommand{\choiceminimumverticalsize}{\vphantom{$\frac{\sqrt{2}}{2}$}}   % minimum height of boxes (cfr infra)
\providecommand{\choiceminimumverticalsize}{\vphantom{$\tfrac{2}{2}$}}   % minimum height of boxes (cfr infra)
\providecommand{\choiceminimumhorizontalsize}{1em}   % minimum width of boxes (cfr infra)

\newcommand{\inlinechoicesquares}[2][]{%
		\setkeys{choice}{#1}%
		\ifthenelse{\boolean{\choice@correct}}%
		{%
            \ifhandout%
               \fbox{\choiceminimumverticalsize #2}\allowbreak\ignorespaces%
            \else%
               \fcolorbox{blue}{blue!20}{\choiceminimumverticalsize #2}\allowbreak\ignorespaces\setkeys{choice}{correct=false}\ignorespaces%
            \fi%
		}%
		{% else
			\fbox{\choiceminimumverticalsize #2}\allowbreak\ignorespaces%  HACK: wat kleiner, zodat fits on line ... 	
		}%
}

\newcommand{\inlinechoicesquareX}[2][]{%
		\setkeys{choice}{#1}%
		\ifthenelse{\boolean{\choice@correct}}%
		{%
            \ifhandout%
               \framebox[\ifdim\choiceminimumhorizontalsize<\width\width\else\choiceminimumhorizontalsize\fi]{\choiceminimumverticalsize\ #2\ }\allowbreak\ignorespaces\setkeys{choice}{correct=false}\ignorespaces%
            \else%
               \fcolorbox{blue}{blue!20}{\makebox[\ifdim\choiceminimumhorizontalsize<\width\width\else\choiceminimumhorizontalsize\fi]{\choiceminimumverticalsize #2}}\allowbreak\ignorespaces\setkeys{choice}{correct=false}\ignorespaces%
            \fi%
		}%
		{% else
        \ifhandout%
			\framebox[\ifdim\choiceminimumhorizontalsize<\width\width\else\choiceminimumhorizontalsize\fi]{\choiceminimumverticalsize\ #2\ }\allowbreak\ignorespaces%  HACK: wat kleiner, zodat fits on line ... 	
        \fi
		}%
}


\newcommand{\inlinechoicepuntjes}[2][]{%
		\setkeys{choice}{#1}%
		\ifthenelse{\boolean{\choice@correct}}%
		{%
            \ifhandout%
               \dots\ldots\ignorespaces\setkeys{choice}{correct=false}\ignorespaces
            \else%
               \fcolorbox{blue}{blue!20}{\choiceminimumverticalsize #2}\allowbreak\ignorespaces\setkeys{choice}{correct=false}\ignorespaces%
            \fi%
		}%
		{% else
			%\fbox{\choiceminimumverticalsize #2}\allowbreak\ignorespaces%  HACK: wat kleiner, zodat fits on line ... 	
		}%
}

% print niets, maar definieer globale variable \myanswer
%  (gebruikt om oplossingsbladen te printen) 
\newcommand{\inlinechoicedefanswer}[2][]{%
		\setkeys{choice}{#1}%
		\ifthenelse{\boolean{\choice@correct}}%
		{%
               \gdef\myanswer{#2}\setkeys{choice}{correct=false}

		}%
		{% else
			%\fbox{\choiceminimumverticalsize #2}\allowbreak\ignorespaces%  HACK: wat kleiner, zodat fits on line ... 	
		}%
}



%\makeatother

\newcommand{\setchoicedefanswer}{
\ifdefined\HCode
\else
%    \renewenvironment{multipleChoice@}[1][]{}{} % remove trailing ')'
    \let\inlinechoice\inlinechoicedefanswer
\fi
}

\newcommand{\setchoicepuntjes}{
\ifdefined\HCode
\else
    \renewenvironment{multipleChoice@}[1][]{}{} % remove trailing ')'
    \let\inlinechoice\inlinechoicepuntjes
\fi
}
\newcommand{\setchoicesquares}{
\ifdefined\HCode
\else
    \renewenvironment{multipleChoice@}[1][]{}{} % remove trailing ')'
    \let\inlinechoice\inlinechoicesquares
\fi
}
%
\newcommand{\setchoicesquareX}{
\ifdefined\HCode
\else
    \renewenvironment{multipleChoice@}[1][]{}{} % remove trailing ')'
    \let\inlinechoice\inlinechoicesquareX
\fi
}
%
\newcommand{\setchoicelist}{
\ifdefined\HCode
\else
    \renewenvironment{multipleChoice@}[1][]{}{)}% re-add trailing ')'
    \let\inlinechoice\inlinechoiceorig
\fi
}

\setchoicesquareX  % by default list-of-squares with onlineChoice in PDF

% Omdat multicols niet werkt in html: enkel in pdf  (in html zijn langere pagina's misschien ook minder storend)
\newenvironment{xmmulticols}[1][2]{
 \pdfOnly{\begin{multicols}{#1}}%
}{ \pdfOnly{\end{multicols}}}

%
% Te gebruiken in plaats van \section\subsection
%  (in een printstyle kan dan het level worden aangepast
%    naargelang \chapter vs \section style )
% 3/2021: DO NOT USE \xmsubsection !
\newcommand\xmsection\subsection
\newcommand\xmsubsection\subsubsection

% Aanpassen printversie
%  (hier gedefinieerd, zodat ze in xourse kunnen worden gezet/overschreven)
\providebool{parttoc}
\providebool{printpartfrontpage}
\providebool{printactivitytitle}
\providebool{printactivityqrcode}
\providebool{printactivityurl}
\providebool{printcontinuouspagenumbers}
\providebool{numberactivitiesbysubpart}
\providebool{addtitlenumber}
\providebool{addsectiontitlenumber}
\addtitlenumbertrue
\addsectiontitlenumbertrue

% The following three commands are hardcoded in xake, you can't create other commands like these, without adding them to xake as well
%  ( gebruikt in xourses om juiste soort titelpagina te krijgen voor verschillende ximera's )
\newcommand{\activitychapter}[2][]{
    {    
    \ifstrequal{#1}{notnumbered}{
        \addtitlenumberfalse
    }{}
    \typeout{ACTIVITYCHAPTER #2}   % logging
	\chapterstyle
	\activity{#2}
    }
}
\newcommand{\activitysection}[2][]{
    {
    \ifstrequal{#1}{notnumbered}{
        \addsectiontitlenumberfalse
    }{}
	\typeout{ACTIVITYSECTION #2}   % logging
	\sectionstyle
	\activity{#2}
    }
}
% Practices worden als activity getoond om de grote blokken te krijgen online
\newcommand{\practicesection}[2][]{
    {
    \ifstrequal{#1}{notnumbered}{
        \addsectiontitlenumberfalse
    }{}
    \typeout{PRACTICESECTION #2}   % logging
	\sectionstyle
	\activity{#2}
    }
}
\newcommand{\activitychapterlink}[3][]{
    {
    \ifstrequal{#1}{notnumbered}{
        \addtitlenumberfalse
    }{}
    \typeout{ACTIVITYCHAPTERLINK #3}   % logging
	\chapterstyle
	\activitylink[#1]{#2}{#3}
    }
}

\newcommand{\activitysectionlink}[3][]{
    {
    \ifstrequal{#1}{notnumbered}{
        \addsectiontitlenumberfalse
    }{}
    \typeout{ACTIVITYSECTIONLINK #3}   % logging
	\sectionstyle
	\activitylink[#1]{#2}{#3}
    }
}


% Commando om de printstyle toe te voegen in ximera's. Zorgt ervoor dat er geen problemen zijn als je de xourses compileert
% hack om onhandige relative paden in TeX te omzeilen
% should work both in xourse and ximera (pre-112022 only in ximera; thus obsoletes adhoc setup in xourses)
% loads global.sty if present (cfr global.css for online settings!)
% use global.sty to overwrite settings in printstyle.sty ...
\newcommand{\addPrintStyle}[1]{
\iftikzexport\else   % only in PDF
  \makeatletter
  \ifx\@onlypreamble\@notprerr\else   % ONLY if in tex-preamble   (and e.g. not when included from xourse)
    \typeout{Loading printstyle}   % logging
    \usepackage{#1/printstyle} % mag enkel geinclude worden als je die apart compileert
    \IfFileExists{#1/global.sty}{
        \typeout{Loading printstyle-folder #1/global.sty}   % logging
        \usepackage{#1/global}
        }{
        \typeout{Info: No extra #1/global.sty}   % logging
    }   % load global.sty if present
    \IfFileExists{global.sty}{
        \typeout{Loading local-folder global.sty (or TEXINPUTPATH..)}   % logging
        \usepackage{global}
    }{
        \typeout{Info: No folder/global.sty}   % logging
    }   % load global.sty if present
    \IfFileExists{\currfilebase.sty}
    {
        \typeout{Loading \currfilebase.sty}
        \input{\currfilebase.sty}
    }{
        \typeout{Info: No local \currfilebase.sty}
    }
    \fi
  \makeatother
\fi
}

%
%  
% references: Ximera heeft adhoc logica	 om online labels te doen werken over verschillende files heen
% met \hyperref kan de getoonde tekst toch worden opgegeven, in plaats van af te hangen van de label-text
\ifdefined\HCode
% Link to standard \labels, but give your own description
% Usage:  Volg \hyperref[my_very_verbose_label]{deze link} voor wat tijdverlies
%   (01/2020: Ximera-server aangepast om bij class reference-keeptext de link-text NIET te vervangen door de label-text !!!) 
\renewcommand{\hyperref}[2][]{\HCode{<a class="reference reference-keeptext" href="\##1">}#2\HCode{</a>}}
%
%  Link to specific targets  (not tested ?)
\renewcommand{\hypertarget}[1]{\HCode{<a class="ximera-label" id="#1"></a>}}
\renewcommand{\hyperlink}[2]{\HCode{<a class="reference reference-keeptext" href="\##1">}#2\HCode{</a>}}
\fi

% Mmm, quid English ... (for keyword #1 !) ?
\newcommand{\wikilink}[2]{
    \hyperlink{https://nl.wikipedia.org/wiki/#1}{#2}
    \pdfOnly{\footnote{See \url{https://nl.wikipedia.org/wiki/#1}}
    }
}

\renewcommand{\figurename}{Figuur}
\renewcommand{\tablename}{Tabel}

%
% Gedoe om verschillende versies van xourse/ximera te maken afhankelijk van settings
%
% default: versie met antwoorden
% handout: versie voor de studenten, zonder antwoorden/oplossingen
% full: met alles erop en eraan, dus geschikt voor auteurs en/of lesgevers  (bevat in de pdf ook de 'online-only' stukken!)
%
%
% verder kunnen ook opties/variabele worden gezet voor hints/auteurs/uitweidingen/ etc
%
% 'Full' versie
\newtoggle{showonline}
\ifdefined\HCode   % zet default showOnline
    \toggletrue{showonline} 
\else
    \togglefalse{showonline}
\fi

% Full versie   % deprecated: see infra
\newcommand{\printFull}{
    \hintstrue
    \handoutfalse
    \toggletrue{showonline} 
}

\ifdefined\shouldPrintFull   % deprecated: see infra
    \printFull
\fi



% Overschrijf onlineOnly  (zoals gedefinieerd in ximera.cls)
\ifhandout   % in handout: gebruik de oorspronkelijke ximera.cls implementatie  (is dit wel nodig/nuttig?)
\else
    \iftoggle{showonline}{%
        \ifdefined\HCode
          \RenewEnviron{onlineOnly}{\bgroup\BODY\egroup}   % showOnline, en we zijn  online, dus toon de tekst
        \else
          \RenewEnviron{onlineOnly}{\bgroup\color{red!50!black}\BODY\egroup}   % showOnline, maar we zijn toch niet online: kleur de tekst rood 
        \fi
    }{%
      \RenewEnviron{onlineOnly}{}  % geen showOnline
    }
\fi

% hack om na hoofding van definition/proposition/... als dan niet op een nieuwe lijn te starten
% soms is dat goed en mooi, en soms niet; en in HTML is het nu (2/2020) anders dan in pdf
% vandaar suggestie om 
%     \begin{definition}[Nieuw concept] \nl
% te gebruiken als je zeker een newline wil na de hoofdig en titel
% (in het bijzonder itemize zonder \nl is 'lelijk' ...)
\ifdefined\HCode
\newcommand{\nl}{}
\else
\newcommand{\nl}{\ \par} % newline (achter heading van definition etc.)
\fi


% \nl enkel in handoutmode (ihb voor \wordChoice, die dan typisch veeeel langer wordt)
\ifdefined\HCode
\providecommand{\handoutnl}{}
\else
\providecommand{\handoutnl}{%
\ifhandout%
  \nl%
\fi%
}
\fi

% Could potentially replace \pdfOnline/\begin{onlineOnly} : 
% Usage= \ifonline{Hallo surfer}{Hallo PDFlezer}
\providecommand{\ifonline}[2]%
{
\begin{onlineOnly}#1\end{onlineOnly}%
\pdfOnly{#2}
}%


%
% Maak optionele 'basic' en 'extended' versies van een activity
%  met environment basicOnly, basicSkip en extendedOnly
%
%  (
%   Dit werkt ENKEL in de PDF; de online versies tonen (minstens voorklopig) steeds 
%   het default geval met printbasicversion en printextendversion beide FALSE
%  )
%
\providebool{printbasicversion}
\providebool{printextendedversion}   % not properly implemented
\providebool{printfullversion}       % presumably print everything (debug/auteur)
%
% only set these in xourses, and BEFORE loading this preamble
%
%\newif\ifshowbasic     \showbasictrue        % use this line in xourse to show 'basic' sections
%\newif\ifshowextended  \showextendedtrue     % use this line in xourse to show 'extended' sections
%
%
%\ifbool{showbasic}
%      { \NewEnviron{basicOnly}{\BODY} }    % if yes: just print contents
%      { \NewEnviron{basicOnly}{}      }    % if no:  completely ignore contents
%
%\ifbool{showbasic}
%      { \NewEnviron{basicSkip}{}      }
%      { \NewEnviron{basicSkip}{\BODY} }
%

\ifbool{printextendedversion}
      { \NewEnviron{extendedOnly}{\BODY} }
      { \NewEnviron{extendedOnly}{}      }
      


\ifdefined\HCode    % in html: always print
      {\newenvironment*{basicOnly}{}{}}    % if yes: just print contents
      {\newenvironment*{basicSkip}{}{}}    % if yes: just print contents
\else

\ifbool{printbasicversion}
      {\newenvironment*{basicOnly}{}{}}    % if yes: just print contents
      {\NewEnviron{basicOnly}{}      }    % if no:  completely ignore contents

\ifbool{printbasicversion}
      {\NewEnviron{basicSkip}{}      }
      {\newenvironment*{basicSkip}{}{}}

\fi

\usepackage{float}
\usepackage[rightbars,color]{changebar}

% Full versie
\ifbool{printfullversion}{
    \hintstrue
    \handoutfalse
    \toggletrue{showonline}
    \printbasicversionfalse
    \cbcolor{red}
    \renewenvironment*{basicOnly}{\cbstart}{\cbend}
    \renewenvironment*{basicSkip}{\cbstart}{\cbend}
    \def\xmtoonprintopties{FULL}   % will be printed in footer
}
{}
      
%
% Evalueer \ifhints IN de environment
%  
%
%\RenewEnviron{hint}
%{
%\ifhandout
%\ifhints\else\setbox0\vbox\fi%everything in een emty box
%\bgroup 
%\stepcounter{hintLevel}
%\BODY
%\egroup\ignorespacesafterend
%\addtocounter{hintLevel}{-1}
%\else
%\ifhints
%\begin{trivlist}\item[\hskip \labelsep\small\slshape\bfseries Hint:\hspace{2ex}]
%\small\slshape
%\stepcounter{hintLevel}
%\BODY
%\end{trivlist}
%\addtocounter{hintLevel}{-1}
%\fi
%\fi
%}

% Onafhankelijk van \ifhandout ...? TO BE VERIFIED
\RenewEnviron{hint}
{
\ifhints
\begin{trivlist}\item[\hskip \labelsep\small\bfseries Hint:\hspace{2ex}]
\small%\slshape
\stepcounter{hintLevel}
\BODY
\end{trivlist}
\addtocounter{hintLevel}{-1}
\else
\iftikzexport   % anders worden de tikz tekeningen in hints niet gegenereerd ?
\setbox0\vbox\bgroup
\stepcounter{hintLevel}
\BODY
\egroup\ignorespacesafterend
\addtocounter{hintLevel}{-1}
\fi % ifhandout
\fi %ifhints
}

%
% \tab sets typewriter-tabs (e.g. to format questions)
% (Has no effect in HTML :-( ))
%
\usepackage{tabto}
\ifdefined\HCode
  \renewcommand{\tab}{\quad}    % otherwise dummy .png's are generated ...?
\fi


% Also redefined in  preamble to get correct styling 
% for tikz images for (\tikzexport)
%

\theoremstyle{definition} % Bold titels
\makeatletter
\let\proposition\relax
\let\c@proposition\relax
\let\endproposition\relax
\makeatother
\newtheorem{proposition}{Eigenschap}


%\instructornotesfalse

% logic with \ifhandoutin ximera.cls unclear;so overwrite ...
\makeatletter
\@ifundefined{ifinstructornotes}{%
  \newif\ifinstructornotes
  \instructornotesfalse
  \newenvironment{instructorNotes}{}{}
}{}
\makeatother
\ifinstructornotes
\else
\renewenvironment{instructorNotes}%
{%
    \setbox0\vbox\bgroup
}
{%
    \egroup
}
\fi

% \RedeclareMathOperator
% from https://tex.stackexchange.com/questions/175251/how-to-redefine-a-command-using-declaremathoperator
\makeatletter
\newcommand\RedeclareMathOperator{%
    \@ifstar{\def\rmo@s{m}\rmo@redeclare}{\def\rmo@s{o}\rmo@redeclare}%
}
% this is taken from \renew@command
\newcommand\rmo@redeclare[2]{%
    \begingroup \escapechar\m@ne\xdef\@gtempa{{\string#1}}\endgroup
    \expandafter\@ifundefined\@gtempa
    {\@latex@error{\noexpand#1undefined}\@ehc}%
    \relax
    \expandafter\rmo@declmathop\rmo@s{#1}{#2}}
% This is just \@declmathop without \@ifdefinable
\newcommand\rmo@declmathop[3]{%
    \DeclareRobustCommand{#2}{\qopname\newmcodes@#1{#3}}%
}
\@onlypreamble\RedeclareMathOperator
\makeatother


%
% Engelse vertaling, vooral in mathmode
%
% 1. Algemene procedure
%
\ifdefined\isEn
 \newcommand{\nlen}[2]{#2}
 \newcommand{\nlentext}[2]{\text{#2}}
 \newcommand{\nlentextbf}[2]{\textbf{#2}}
\else
 \newcommand{\nlen}[2]{#1}
 \newcommand{\nlentext}[2]{\text{#1}}
 \newcommand{\nlentextbf}[2]{\textbf{#1}}
\fi

%
% 2. Lijst van erg veel gebruikte uitdrukkingen
%

% Ja/Nee/Fout/Juits etc
%\newcommand{\TJa}{\nlentext{ Ja }{ and }}
%\newcommand{\TNee}{\nlentext{ Nee }{ No }}
%\newcommand{\TJuist}{\nlentext{ Juist }{ Correct }
%\newcommand{\TFout}{\nlentext{ Fout }{ Wrong }
\newcommand{\TWaar}{\nlentext{ Waar }{ True }}
\newcommand{\TOnwaar}{\nlentext{ Vals }{ False }}
% Korte bindwoorden en, of, dus, ...
\newcommand{\Ten}{\nlentext{ en }{ and }}
\newcommand{\Tof}{\nlentext{ of }{ or }}
\newcommand{\Tdus}{\nlentext{ dus }{ so }}
\newcommand{\Tendus}{\nlentext{ en dus }{ and thus }}
\newcommand{\Tvooralle}{\nlentext{ voor alle }{ for all }}
\newcommand{\Took}{\nlentext{ ook }{ also }}
\newcommand{\Tals}{\nlentext{ als }{ when }} %of if?
\newcommand{\Twant}{\nlentext{ want }{ as }}
\newcommand{\Tmaal}{\nlentext{ maal }{ times }}
\newcommand{\Toptellen}{\nlentext{ optellen }{ add }}
\newcommand{\Tde}{\nlentext{ de }{ the }}
\newcommand{\Thet}{\nlentext{ het }{ the }}
\newcommand{\Tis}{\nlentext{ is }{ is }} %zodat is in text staat in mathmode (geen italics)
\newcommand{\Tmet}{\nlentext{ met }{ where }} % in situaties e.g met p < n --> where p < n
\newcommand{\Tnooit}{\nlentext{ nooit }{ never }}
\newcommand{\Tmaar}{\nlentext{ maar }{ but }}
\newcommand{\Tniet}{\nlentext{ niet }{ not }}
\newcommand{\Tuit}{\nlentext{ uit }{ from }}
\newcommand{\Ttov}{\nlentext{ t.o.v. }{ w.r.t. }}
\newcommand{\Tzodat}{\nlentext{ zodat }{ such that }}
\newcommand{\Tdeth}{\nlentext{de }{th }}
\newcommand{\Tomdat}{\nlentext{omdat }{because }} 


%
% Overschrijf addhoc commando's
%
\ifdefined\isEn
\renewcommand{\pernot}{\overset{\mathrm{notation}}{=}}
\RedeclareMathOperator{\bld}{im}     % beeld
\RedeclareMathOperator{\graf}{graph}   % grafiek
\RedeclareMathOperator{\rico}{slope}   % richtingcoëfficient
\RedeclareMathOperator{\co}{co}       % coordinaat
\RedeclareMathOperator{\gr}{deg}       % graad

% Operators
\RedeclareMathOperator{\bgsin}{arcsin}
\RedeclareMathOperator{\bgcos}{arccos}
\RedeclareMathOperator{\bgtan}{arctan}
\RedeclareMathOperator{\bgcot}{arccot}
\RedeclareMathOperator{\bgsinh}{arcsinh}
\RedeclareMathOperator{\bgcosh}{arccosh}
\RedeclareMathOperator{\bgtanh}{arctanh}
\RedeclareMathOperator{\bgcoth}{arccoth}

\fi


% HACK: use 'oplossing' for 'explanation' ...
\let\explanation\relax
\let\endexplanation\relax
% \newenvironment{explanation}{\begin{oplossing}}{\end{oplossing}}
\newcounter{explanation}

\ifhandout%
    \NewEnviron{explanation}[1][toon]%
    {%
    \RenewEnviron{verbatim}{ \red{VERBATIM CONTENT MISSING IN THIS PDF}} %% \expandafter\verb|\BODY|}

    \ifthenelse{\equal{\detokenize{#1}}{\detokenize{toon}}}
    {
    \def\PH@Command{#1}% Use PH@Command to hold the content and be a target for "\expandafter" to expand once.

    \begin{trivlist}% Begin the trivlist to use formating of the "Feedback" label.
    \item[\hskip \labelsep\small\slshape\bfseries Explanation:% Format the "Feedback" label. Don't forget the space.
    %(\texttt{\detokenize\expandafter{\PH@Command}}):% Format (and detokenize) the condition for feedback to trigger
    \hspace{2ex}]\small%\slshape% Insert some space before the actual feedback given.
    \BODY
    \end{trivlist}
    }
    {  % \begin{feedback}[solution]   \BODY     \end{feedback}  }
    }
    }    
\else
% ONLY for HTML; xmoplossing is styled with css, and is not, and need not be a LaTeX environment
% THUS: it does NOT use feedback anymore ...
%    \NewEnviron{oplossing}{\begin{expandable}{xmoplossing}{\nlen{Toon uitwerking}{Show solution}}{\BODY}\end{expandable}}
    \newenvironment{explanation}[1][toon]
   {%
       \begin{expandable}{xmoplossing}{}
   }
   {%
   	   \end{expandable}
   } 
\fi

\title{Additional Exercises for Ch 10} \license{CC BY-NC-SA 4.0}

\begin{document}

\begin{abstract}
\end{abstract}
\maketitle

\section*{Exercises for Ch 10 Abstract Vector Spaces}


\begin{problem}\label{prb:10.1} Suppose you have $\mathbb{R}^{2}$ and the $+$ operation is as
follows:\
\begin{equation*}
\left( a,b\right) +\left( c,d\right) =\left( a+d,b+c\right) .
\end{equation*}
Scalar multiplication is defined in the usual way. Is this a vector space?
Explain why or why not.
%\begin{hint}
%\end{hint}
\end{problem}

\begin{problem}\label{prb:10.2} Suppose you have $\mathbb{R}^{2}$ and the $+$ operation is defined as
follows.
\begin{equation*}
\left( a,b\right) +\left( c,d\right) =\left( 0,b+d\right)
\end{equation*}
Scalar multiplication is defined in the usual way. Is this a vector space?
Explain why or why not.
%\begin{hint}
%\end{hint}
\end{problem}

\begin{problem}\label{prb:10.3} Suppose you have $\mathbb{R}^{2}$ and scalar multiplication is defined
as $c\left( a,b\right) =\left( a,cb\right) $ while vector addition is
defined as usual. Is this a vector space? Explain why or why not.
%\begin{hint}
%\end{hint}
\end{problem}

\begin{problem}\label{prb:10.4} Suppose you have $\mathbb{R}^{2}$ and the $+$ operation is defined as
follows.
\begin{equation*}
\left( a,b\right) +\left( c,d\right) =\left( a-c,b-d\right)
\end{equation*}
Scalar multiplication is same as usual. Is this a vector space? Explain why
or why not.
%\begin{hint}
%\end{hint}
\end{problem}

\begin{problem}\label{prb:10.5} \label{functions}Consider all the functions defined on a non empty set
which have values in $\mathbb{R}$. Is this a vector space? Explain.
The operations are defined as follows. Here $f,g$ signify functions and $a$
is a scalar.
\begin{eqnarray*}
\left( f+g\right) \left( x\right) &=&f\left( x\right) +g\left( x\right) \\
\left( af\right) \left( x\right) &=&a\left( f\left( x\right) \right)
\end{eqnarray*}
%\begin{hint}
%\end{hint}
\end{problem}


\begin{problem}\label{prb:10.6} Denote by $\mathbb{R}^{\mathbb{N}}$ the set of real valued sequences.
For $\vec{a}\equiv \left\{ a_{n}\right\} _{n=1}^{\infty },\vec{b}\equiv
\left\{ b_{n}\right\} _{n=1}^{\infty }$ two of these, define their sum to be
given by
\begin{equation*}
\vec{a}+\vec{b} =  \left\{ a_{n}+b_{n}\right\} _{n=1}^{\infty }
\end{equation*}
and define scalar multiplication by
\begin{equation*}
c\vec{a}=\left\{ ca_{n}\right\} _{n=1}^{\infty }\text{ where }\vec{a}
=\left\{ a_{n}\right\} _{n=1}^{\infty }
\end{equation*}
Is this a special case of Problem \ref{prb:10.5}? Is this a vector space?
%\begin{hint}
%\end{hint}
\end{problem}

\begin{problem}\label{prb:10.7} Let $\mathbb{C}^{2}$ be the set of ordered pairs of complex numbers.
Define addition and scalar multiplication in the usual way.
\begin{equation*}
\left( z,w\right) +\left( \hat{z},\hat{w}\right) = \left( z+\hat{z},w+
\hat{w}\right) ,\ u\left( z,w\right) \equiv \left( uz,uw\right)
\end{equation*}
Here the scalars are from $\mathbb{C}$. Show this is a vector space.
%\begin{hint}
%\end{hint}
\end{problem}

\begin{problem}\label{prb:10.8} Let $V$ be the set of functions defined on a nonempty set which have
values in a vector space $W.$ Is this a vector space? Explain.
%\begin{hint}
%\end{hint}
\end{problem}

\begin{problem}\label{prb:10.9} Consider the space of $m\times n$ matrices with operation of addition
and scalar multiplication defined the usual way. That is, if $A,B$ are two $
m\times n$ matrices and $c$ a scalar,
\begin{equation*}
\left( A+B\right) _{ij}=A_{ij}+B_{ij},\ \left( cA\right) _{ij}\equiv c\left(
A_{ij}\right)
\end{equation*}
%\begin{hint}
%\end{hint}
\end{problem}

\begin{problem}\label{prb:10.10} Consider the set of $n\times n$ symmetric matrices. That is, $A=A^{T}.$
In other words, $A_{ij}=A_{ji}$. Show that this set of symmetric matrices is
a vector space and a subspace of the vector space of $n\times n$ matrices.
%\begin{hint}
%\end{hint}
\end{problem}

\begin{problem}\label{prb:10.11} Consider the set of all vectors in $\mathbb{R}^{2},\left( x,y\right) $
such that $x+y\geq 0.$ Let the vector space operations be the usual ones. Is
this a vector space? Is it a subspace of $\mathbb{R}^{2}$?
%\begin{hint}
%\end{hint}
\end{problem}

\begin{problem}\label{prb:10.12} Consider the vectors in $\mathbb{R}^{2},\left( x,y\right) $ such that $xy=0$. Is this a subspace of $\mathbb{R}^{2}?$ Is it a vector space? The
addition and scalar multiplication are the usual operations.
%\begin{hint}
%\end{hint}
\end{problem}

\begin{problem}\label{prb:10.13} Define the operation of vector addition on $\mathbb{R}^{2}$ by $\left(
x,y\right) +\left( u,v\right) =\left( x+u,y+v+1\right) .$ Let scalar
multiplication be the usual operation. Is this a vector space with these
operations? Explain.
%\begin{hint}
%\end{hint}
\end{problem}

\begin{problem}\label{prb:10.14} Let the vectors be real numbers. Define vector space operations in the
usual way. That is $x+y$ means to add the two numbers and $xy$ means to
multiply them. Is $\mathbb{R}$ with these operations a vector space? Explain.
%\begin{hint}
%\end{hint}
\end{problem}

\begin{problem}\label{prb:10.15} Let the scalars be the rational numbers and let the vectors be
real numbers which are the form $a+b\sqrt{2}$ for $a,b$ rational numbers.
Show that with the usual operations, this is a vector space.
%\begin{hint}
%\end{hint}
\end{problem}

\begin{problem}\label{prb:10.16} Let $\mathbb{P}_{2}$ be the set of all polynomials of degree 2 or
less. That is, these are of the form $a+bx+cx^{2}$. Addition is defined as
\begin{equation*}
\left( a+bx+cx^{2}\right) +\left( \hat{a}+\hat{b}x+\hat{c}x^{2}\right)
=\left( a+\hat{a}\right) +\left( b+\hat{b}\right) x+\left( c+\hat{c}\right)
x^{2}
\end{equation*}
and scalar multiplication is defined as
\begin{equation*}
d\left( a+bx+cx^{2}\right) =da+dbx+cdx^{2}
\end{equation*}
Show that, with this definition of the vector space operations that $\mathbb{
P}_{2}$ is a vector space. Now let $V$ denote those polynomials $a+bx+cx^{2}$
such that $a+b+c=0$. Is $V$ a subspace of $\mathbb{P}_{2}?$ Explain.
%\begin{hint}
%\end{hint}
\end{problem}

\begin{problem}\label{prb:10.17} Let $M,N$ be subspaces of a vector space $V$ and consider $M+N$
defined as the set of all $m+n$ where $m\in M$ and $n\in N$. Show that $M+N$
is a subspace of $V$.
%\begin{hint}
%\end{hint}
\end{problem}

\begin{problem}\label{prb:10.18} Let $M,N$ be subspaces of a vector space $V$. Then $M\cap N$ consists
of all vectors which are in both $M$ and $N$. Show that $M\cap N$ is a
subspace of $V$.
%\begin{hint}
%\end{hint}
\end{problem}

\begin{problem}\label{prb:10.19} Let $M,N$ be subspaces of a vector space $\mathbb{R}^{2}.$ Then $N\cup
M$ consists of all vectors which are in either $M$ or $N$. Show that $N\cup
M $ is not necessarily a subspace of $\mathbb{R}^{2}$ by giving an example
where $N\cup M$ fails to be a subspace.
%\begin{hint}
%\end{hint}
\end{problem}


\begin{problem}\label{prb:10.20} \label{4julyprob1}Let $X$ consist of the real valued functions which
are defined on an interval $\left[ a,b\right] .$ For $f,g\in X,\;f+g$ is the
name of the function which satisfies $\left( f+g\right) \left( x\right)
=f\left( x\right) +g\left( x\right)$. For $s$ a real number, $
\left( s f\right) \left( x\right) = s \left( f\left( x\right)
\right) $. Show this is a vector space.
\begin{hint}
The axioms of a vector space all hold because they
hold for a vector space. The only thing left to verify is the
assertions about the things which are supposed to exist. $0$ would
be the zero function which sends everything to $0$. This is an additive
identity. Now if $f$ is a function, $-f\left( x\right) \equiv \left(
-f\left( x\right) \right) $. Then
\[
\left( f+\left( -f\right) \right) \left( x\right) \equiv f\left( x\right)
+\left( -f\right) \left( x\right) \equiv f\left( x\right) +\left( -f\left(
x\right) \right) =0
\]
Hence $f+-f=\mathbf{0}$. For each $x\in \left[ a,b\right] ,$ let $%
f_{x}\left( x\right) =1$ and $f_{x}\left( y\right) =0$ if $y\neq x.$ Then
these vectors are obviously linearly independent.
\end{hint}
\end{problem}

\begin{problem}\label{prb:10.21} Consider functions defined on $\left\{ 1,2,\cdots ,n\right\} $ having
values in $\mathbb{R}$. Explain how, if $V$ is the set of all such
functions, $V$ can be considered as $\mathbb{R}^{n}$.
\begin{hint}
Let $f\left( i\right) $ be the $i^{th}$ component of a vector $
\vec{x}\in \mathbb{R}^{n}$. Thus a typical element in $\mathbb{R}^{n}$ is $
\left( f\left( 1\right) ,\cdots ,f\left( n\right) \right) $.
\end{hint}
\end{problem}

\begin{problem}\label{prb:10.22} Let the vectors be polynomials of degree no more than 3. Show that
with the usual definitions of scalar multiplication and addition wherein,
for $p\left( x\right) $ a polynomial, $\left( a p\right) \left(
x\right) = a p\left( x\right) $ and for $p,q$ polynomials $\left(
p+q\right) \left( x\right) =  p\left( x\right) +q\left( x\right) ,$ this
is a vector space.
\begin{hint}
This is just a subspace of the vector space of functions
because it is closed with respect to vector addition and scalar
multiplication. Hence this is a vector space.
\end{hint}
\end{problem}

\begin{problem}\label{prb:10.23} Let $V$ be a vector space and suppose $\left\{ \vec{x}_{1},\cdots ,
\vec{x}_{k}\right\}$ is a set of vectors in $V$. Show that $\vec{0}$
is in $\mbox{span}\left\{ \vec{x}_{1},\cdots ,\vec{x}_{k}\right\} .$
\begin{hint}
$\sum_{i=1}^{k}0\vec{x}_{k}=\vec{0}$
\end{hint}
\end{problem}

\begin{problem}\label{prb:10.24} Determine if $p(x) = 4x^2-x$ is in the span given by
\[
\mbox{span} \left\{ x^2+x, x^2-1, -x + 2 \right\}
\]
%\begin{hint}
%\end{hint}
\end{problem}

\begin{problem}\label{prb:10.25} Determine if $p(x) = - x^2 + x + 2 $ is in the span given by
\[
\mbox{span} \left\{ x^2 + x + 1, 2x^2 + x \right\}
\]
%\begin{hint}
%\end{hint}
\end{problem}

\begin{problem}\label{prb:10.26} \label{spanmatrices} Determine if $A = \left[ \begin{array}{rr}
1 & 3 \\
0 & 0
\end{array} \right]$ is in the span given by
\[
\mbox{span} \left\{
\left[ \begin{array}{rr}
1 & 0 \\
0 & 1
\end{array} \right], \left[ \begin{array}{rr}
0 & 1 \\
1 & 0
\end{array} \right], \left[ \begin{array}{rr}
1 & 0 \\
1 & 1
\end{array} \right], \left[ \begin{array}{rr}
0 & 1 \\
1 & 1
\end{array} \right]
\right\}
\]
%\begin{hint}
%\end{hint}
\end{problem}


\begin{problem}\label{prb:10.27} Show that the spanning set in Question \ref{prb:10.26} is a spanning set for $M_{22}$, the vector space of all $2 \times 2$ matrices.
%\begin{hint}
%\end{hint}
\end{problem}

\begin{problem}\label{prb:10.28} Consider the vector space of polynomials of degree at most $2,$ $%
\mathbb{P}_{2}$. Determine whether the following is a basis for $\mathbb{P}%
_{2}$.
\begin{equation*}
\left\{ x^{2}+x+1,2x^{2}+2x+1,x+1\right\}
\end{equation*}
\textbf{Hint:\ }There is a isomorphism from $\mathbb{R}^{3}$ to $\mathbb{P}
_{2}$. It is defined as follows:\
\begin{equation*}
T\vec{e}_{1}=1,T\vec{e}_{2}=x,T\vec{e}_{3}=x^{2}
\end{equation*}
Then extend $T$ linearly. Thus
\begin{equation*}
T\left[
\begin{array}{c}
1 \\
1 \\
1
\end{array}
\right] =x^{2}+x+1,T\left[
\begin{array}{c}
1 \\
2 \\
2
\end{array}
\right] =2x^{2}+2x+1,T\left[
\begin{array}{c}
1 \\
1 \\
0
\end{array}
\right] =1+x
\end{equation*}
It follows that if
\begin{equation*}
\left\{ \left[
\begin{array}{c}
1 \\
1 \\
1
\end{array}
\right] ,\left[
\begin{array}{c}
1 \\
2 \\
2
\end{array}
\right] ,\left[
\begin{array}{c}
1 \\
1 \\
0
\end{array}
\right] \right\}
\end{equation*}
is a basis for $\mathbb{R}^{3},$ then the polynomials will be a basis for $
\mathbb{P}_{2}$ because they will be independent. Recall that an isomorphism
takes a linearly independent set to a linearly independent set. Also, since $
T$ is an isomorphism, it preserves all linear relations.
%\begin{hint}
%\end{hint}
\end{problem}

\begin{problem}\label{prb:10.29} Find a basis in $\mathbb{P}_{2}$ for the subspace
\begin{equation*}
\mbox{span}\left\{ 1+x+x^{2},1+2x,1+5x-3x^{2}\right\}
\end{equation*}
If the above three vectors do not yield a basis, exhibit one of them as a
linear combination of the others. \textbf{Hint:\ }This is the situation in
which you have a spanning set and you want to cut it down to form a linearly
independent set which is also a spanning set. Use the same isomorphism
above. Since $T$ is an isomorphism, it preserves all linear relations so if
such can be found in $\mathbb{R}^{3},$ the same linear relations will be
present in $\mathbb{P}_{2}$.
%\begin{hint}
%\end{hint}
\end{problem}


\begin{problem}\label{prb:10.30} Find a basis in $\mathbb{P}_{3}$ for the subspace
\begin{equation*}
\mbox{span}\left\{
1+x-x^{2}+x^{3},1+2x+3x^{3},-1+3x+5x^{2}+7x^{3},1+6x+4x^{2}+11x^{3}\right\}
\end{equation*}
If the above three vectors do not yield a basis, exhibit one of them as a
linear combination of the others.
%\begin{hint}
%\end{hint}
\end{problem}


\begin{problem}\label{prb:10.31} Find a basis in $\mathbb{P}_{3}$ for the subspace
\begin{equation*}
\mbox{span}\left\{
1+x-x^{2}+x^{3},1+2x+3x^{3},-1+3x+5x^{2}+7x^{3},1+6x+4x^{2}+11x^{3}\right\}
\end{equation*}
If the above three vectors do not yield a basis, exhibit one of them as a
linear combination of the others.
%\begin{hint}
%\end{hint}
\end{problem}


\begin{problem}\label{prb:10.32} Find a basis in $\mathbb{P}_{3}$ for the subspace
\begin{equation*}
\mbox{span}\left\{
x^{3}-2x^{2}+x+2,3x^{3}-x^{2}+2x+2,7x^{3}+x^{2}+4x+2,5x^{3}+3x+2\right\}
\end{equation*}
If the above three vectors do not yield a basis, exhibit one of them as a
linear combination of the others.
%\begin{hint}
%\end{hint}
\end{problem}


\begin{problem}\label{prb:10.33} Find a basis in $\mathbb{P}_{3}$ for the subspace
\begin{equation*}
\mbox{span}\left\{
x^{3}+2x^{2}+x-2,3x^{3}+3x^{2}+2x-2,3x^{3}+x+2,3x^{3}+x+2\right\}
\end{equation*}
If the above three vectors do not yield a basis, exhibit one of them as a
linear combination of the others.
%\begin{hint}
%\end{hint}
\end{problem}


\begin{problem}\label{prb:10.34} Find a basis in $\mathbb{P}_{3}$ for the subspace
\begin{equation*}
\mbox{span}\left\{
x^{3}-5x^{2}+x+5,3x^{3}-4x^{2}+2x+5,5x^{3}+8x^{2}+2x-5,11x^{3}+6x+5\right\}
\end{equation*}
If the above three vectors do not yield a basis, exhibit one of them as a
linear combination of the others.
%\begin{hint}
%\end{hint}
\end{problem}


\begin{problem}\label{prb:10.35} Find a basis in $\mathbb{P}_{3}$ for the subspace
\begin{equation*}
\mbox{span}\left\{
x^{3}-3x^{2}+x+3,3x^{3}-2x^{2}+2x+3,7x^{3}+7x^{2}+3x-3,7x^{3}+4x+3\right\}
\end{equation*}
If the above three vectors do not yield a basis, exhibit one
of them as a linear combination of the others.
%\begin{hint}
%\end{hint}
\end{problem}


\begin{problem}\label{prb:10.36} Find a basis in $\mathbb{P}_{3}$ for the subspace
\begin{equation*}
\mbox{span}\left\{
x^{3}-x^{2}+x+1,3x^{3}+2x+1,4x^{3}+x^{2}+2x+1,3x^{3}+2x-1\right\}
\end{equation*}
If the above three vectors do not yield a basis, exhibit one
of them as a linear combination of the others.
%\begin{hint}
%\end{hint}
\end{problem}


\begin{problem}\label{prb:10.37} Find a basis in $\mathbb{P}_{3}$ for the subspace
\begin{equation*}
\mbox{span}\left\{
x^{3}-x^{2}+x+1,3x^{3}+2x+1,13x^{3}+x^{2}+8x+4,3x^{3}+2x-1\right\}
\end{equation*}
If the above three vectors do not yield a basis, exhibit one
of them as a linear combination of the others.
%\begin{hint}
%\end{hint}
\end{problem}


\begin{problem}\label{prb:10.38} Find a basis in $\mathbb{P}_{3}$ for the subspace
\begin{equation*}
\mbox{span}\left\{
x^{3}-3x^{2}+x+3,3x^{3}-2x^{2}+2x+3,-5x^{3}+5x^{2}-4x-6,7x^{3}+4x-3\right\}
\end{equation*}
If the above three vectors do not yield a basis, exhibit one
of them as a linear combination of the others.
%\begin{hint}
%\end{hint}
\end{problem}


\begin{problem}\label{prb:10.39} Find a basis in $\mathbb{P}_{3}$ for the subspace
\begin{equation*}
\mbox{span}\left\{
x^{3}-2x^{2}+x+2,3x^{3}-x^{2}+2x+2,7x^{3}-x^{2}+4x+4,5x^{3}+3x-2\right\}
\end{equation*}
If the above three vectors do not yield a basis, exhibit one
of them as a linear combination of the others.
%\begin{hint}
%\end{hint}
\end{problem}


\begin{problem}\label{prb:10.40} Find a basis in $\mathbb{P}_{3}$ for the subspace
\begin{equation*}
\mbox{span}\left\{
x^{3}-2x^{2}+x+2,3x^{3}-x^{2}+2x+2,3x^{3}+4x^{2}+x-2,7x^{3}-x^{2}+4x+4\right
\}
\end{equation*}
If the above three vectors do not yield a basis, exhibit one
of them as a linear combination of the others.
%\begin{hint}
%\end{hint}
\end{problem}


\begin{problem}\label{prb:10.41} Find a basis in $\mathbb{P}_{3}$ for the subspace
\begin{equation*}
\mbox{span}\left\{
x^{3}-4x^{2}+x+4,3x^{3}-3x^{2}+2x+4,-3x^{3}+3x^{2}-2x-4,-2x^{3}+4x^{2}-2x-4
\right\}
\end{equation*}
If the above three vectors do not yield a basis, exhibit one
of them as a linear combination of the others.
%\begin{hint}
%\end{hint}
\end{problem}


\begin{problem}\label{prb:10.42} Find a basis in $\mathbb{P}_{3}$ for the subspace
\begin{equation*}
\mbox{span}\left\{
x^{3}+2x^{2}+x-2,3x^{3}+3x^{2}+2x-2,5x^{3}+x^{2}+2x+2,10x^{3}+10x^{2}+6x-6
\right\}
\end{equation*}
If the above three vectors do not yield a basis, exhibit one
of them as a linear combination of the others.
%\begin{hint}
%\end{hint}
\end{problem}


\begin{problem}\label{prb:10.43} Find a basis in $\mathbb{P}_{3}$ for the subspace
\begin{equation*}
\mbox{span}\left\{
x^{3}+x^{2}+x-1,3x^{3}+2x^{2}+2x-1,x^{3}+1,4x^{3}+3x^{2}+2x-1\right\}
\end{equation*}
If the above three vectors do not yield a basis, exhibit one
of them as a linear combination of the others.
%\begin{hint}
%\end{hint}
\end{problem}


\begin{problem}\label{prb:10.44} Find a basis in $\mathbb{P}_{3}$ for the subspace
\begin{equation*}
\mbox{span}\left\{
x^{3}-x^{2}+x+1,3x^{3}+2x+1,x^{3}+2x^{2}-1,4x^{3}+x^{2}+2x+1\right\}
\end{equation*}
If the above three vectors do not yield a basis, exhibit one
of them as a linear combination of the others.
%\begin{hint}
%\end{hint}
\end{problem}


\begin{problem}\label{prb:10.45} Here are some vectors.
\begin{equation*}
\left\{ x^{3}+x^{2}-x-1,3x^{3}+2x^{2}+2x-1\right\}
\end{equation*}
If these are linearly independent, extend to a basis for all of $\mathbb{P}
_{3}$.
%\begin{hint}
%\end{hint}
\end{problem}


\begin{problem}\label{prb:10.46} Here are some vectors.
\begin{equation*}
\left\{ x^{3}-2x^{2}-x+2,3x^{3}-x^{2}+2x+2\right\}
\end{equation*}
If these are linearly independent, extend to a basis for all of $\mathbb{P}
_{3}$.
%\begin{hint}
%\end{hint}
\end{problem}


\begin{problem}\label{prb:10.47} Here are some vectors.
\begin{equation*}
\left\{ x^{3}-3x^{2}-x+3,3x^{3}-2x^{2}+2x+3\right\}
\end{equation*}
If these are linearly independent, extend to a basis for all of $\mathbb{P}
_{3}$.
%\begin{hint}
%\end{hint}
\end{problem}


\begin{problem}\label{prb:10.48} Here are some vectors.
\begin{equation*}
\left\{ x^{3}-2x^{2}-3x+2,3x^{3}-x^{2}-6x+2,-8x^{3}+18x+10\right\}
\end{equation*}
If these are linearly independent, extend to a basis for all of $\mathbb{P}
_{3}$.
%\begin{hint}
%\end{hint}
\end{problem}


\begin{problem}\label{prb:10.49} Here are some vectors.
\begin{equation*}
\left\{ x^{3}-3x^{2}-3x+3,3x^{3}-2x^{2}-6x+3,-8x^{3}+18x+40\right\}
\end{equation*}
If these are linearly independent, extend to a basis for all of $\mathbb{P}
_{3}$.
%\begin{hint}
%\end{hint}
\end{problem}


\begin{problem}\label{prb:10.50} Here are some vectors.
\begin{equation*}
\left\{ x^{3}-x^{2}+x+1,3x^{3}+2x+1,4x^{3}+2x+2\right\}
\end{equation*}
If these are linearly independent, extend to a basis for all of $\mathbb{P}
_{3}$.
%\begin{hint}
%\end{hint}
\end{problem}


\begin{problem}\label{prb:10.51} Here are some vectors.
\begin{equation*}
\left\{ x^{3}+x^{2}+2x-1,3x^{3}+2x^{2}+4x-1,7x^{3}+8x+23\right\}
\end{equation*}
If these are linearly independent, extend to a basis for all of $\mathbb{P}
_{3}$.
%\begin{hint}
%\end{hint}
\end{problem}


\begin{problem}\label{prb:10.52} Determine if the following set is linearly independent. If it is linearly dependent, write one vector as a linear combination of the other vectors in the set.
\[
\left\{ x+1, x^2 + 2, x^2 - x -3 \right\}
\]
%\begin{hint}
%\end{hint}
\end{problem}

\begin{problem}\label{prb:10.53} Determine if the following set is linearly independent. If it is linearly dependent, write one vector as a linear combination of the other vectors in the set.
\[
\left\{ x^2 + x, -2x^2 -4x -6 , 2x - 2 \right\}
\]
%\begin{hint}
%\end{hint}
\end{problem}

\begin{problem}\label{prb:10.54} Determine if the following set is linearly independent. If it is linearly dependent, write one vector as a linear combination of the other vectors in the set.
\[
\left\{ \left[ \begin{array}{rr}
1 & 2 \\
0 & 1
\end{array} \right], \left[ \begin{array}{rr}
-7 & 2 \\
-2 & -3
\end{array} \right], \left[ \begin{array}{rr}
4 & 0 \\
1 & 2
\end{array} \right]
 \right\}
\]
%\begin{hint}
%\end{hint}
\end{problem}

\begin{problem}\label{prb:10.55} Determine if the following set is linearly independent. If it is linearly dependent, write one vector as a linear combination of the other vectors in the set.
\[
\left\{ \left[ \begin{array}{rr}
1 & 0 \\
0 & 1
\end{array} \right], \left[ \begin{array}{rr}
0 & 1 \\
0 & 1
\end{array} \right], \left[ \begin{array}{rr}
1 & 0 \\
1 & 0
\end{array} \right], \left[ \begin{array}{rr}
0 & 0 \\
1 & 1
\end{array} \right]
 \right\}
\]
%\begin{hint}
%\end{hint}
\end{problem}

\begin{problem}\label{prb:10.56} If you have $5$ vectors in $\mathbb{R}^{5}$ and the vectors are
linearly independent, can it always be concluded they span $\mathbb{R}^{5}?$
\begin{hint}
Yes. If not, there would exist a vector not in the span. But then
you could add in this vector and obtain a linearly independent set of
vectors with more vectors than a basis.
\end{hint}
\end{problem}

\begin{problem}\label{prb:10.57} If you have $6$ vectors in $\mathbb{R}^{5},$ is it possible they are
linearly independent? Explain.
\begin{hint}
No. They can't be.
\end{hint}
\end{problem}

\begin{problem}\label{prb:10.58} Let $\mathbb{P}_3$ be the polynomials of degree no more than 3. Determine which
of the following are bases for this vector space.

\begin{enumerate}
\item $\left\{ x+1,x^{3}+x^{2}+2x,x^{2}+x,x^{3}+x^{2}+x\right\} $

\item $\left\{ x^{3}+1,x^{2}+x,2x^{3}+x^{2},2x^{3}-x^{2}-3x+1\right\} $
\end{enumerate}

\begin{hint}
\begin{enumerate}
\item
\item
Suppose
\[
c_{1}\left( x^{3}+1\right) +c_{2}\left( x^{2}+x\right) +c_{3}\left(
2x^{3}+x^{2}\right) +c_{4}\left( 2x^{3}-x^{2}-3x+1\right) =0
\]
Then combine the terms according to power of $x.$
\[
\left( c_{1}+2c_{3}+2c_{4}\right) x^{3}+\left( c_{2}+c_{3}-c_{4}\right)
x^{2}+\left( c_{2}-3c_{4}\right) x+\left( c_{1}+c_{4}\right) =0
\]
Is there a non zero solution to the system $
\begin{array}{c}
c_{1}+2c_{3}+2c_{4}=0 \\
c_{2}+c_{3}-c_{4}=0 \\
c_{2}-3c_{4}=0 \\
c_{1}+c_{4}=0
\end{array}
$, Solution is:
\[
\left[ c_{1}=0,c_{2}=0,c_{3}=0,c_{4}=0\right]
\]
Therefore, these are linearly independent.
\end{enumerate}
\end{hint}
\end{problem}

\begin{problem}\label{prb:10.59} In the context of the above problem, consider polynomials
\begin{equation*}
\left\{ a_{i}x^{3}+b_{i}x^{2}+c_{i}x+d_{i},\ i=1,2,3,4\right\}
\end{equation*}
Show that this collection of polynomials is linearly independent on an
interval $\left[ s,t\right] $ if and only if
\begin{equation*}
\left[
\begin{array}{cccc}
a_{1} & b_{1} & c_{1} & d_{1} \\
a_{2} & b_{2} & c_{2} & d_{2} \\
a_{3} & b_{3} & c_{3} & d_{3} \\
a_{4} & b_{4} & c_{4} & d_{4}
\end{array}
\right]
\end{equation*}
is an invertible matrix.
\begin{hint}
Let $p_{i}\left( x\right) $ denote the $i^{th}$ of
these polynomials. Suppose $\sum_{i}C_{i}p_{i}\left( x\right) =0.$ Then
collecting terms according to the exponent of $x,$ you need to have
\begin{eqnarray*}
C_{1}a_{1}+C_{2}a_{2}+C_{3}a_{3}+C_{4}a_{4} &=&0 \\
C_{1}b_{1}+C_{2}b_{2}+C_{3}b_{3}+C_{4}b_{4} &=&0 \\
C_{1}c_{1}+C_{2}c_{2}+C_{3}c_{3}+C_{4}c_{4} &=&0 \\
C_{1}d_{1}+C_{2}d_{2}+C_{3}d_{3}+C_{4}d_{4} &=&0
\end{eqnarray*}
The matrix of coefficients is just the transpose of the above matrix. There
exists a non trivial solution if and only if the determinant of this matrix
equals 0.
\end{hint}
\end{problem}

\begin{problem}\label{prb:10.60} Let the field of scalars be $\mathbb{Q}$, the rational numbers and let
the vectors be of the form $a+b\sqrt{2}$ where $a,b$ are rational numbers.
Show that this collection of vectors is a vector space with field of scalars
$\mathbb{Q}$ and give a basis for this vector space.
\begin{hint}
When you add two of these you get one and when you multiply one of these by
a scalar, you get another one. A basis is $\left\{ 1,\sqrt{2}\right\} $. By
definition, the span of these gives the collection of vectors. Are they
independent? Say $a+b\sqrt{2}=0$ where $a,b$ are rational numbers. If $a\neq
0,$ then $b\sqrt{2}=-a$ which can't happen since $a$ is rational. If $b\neq
0,$ then $-a=b\sqrt{2}$ which again can't happen because on the left is a
rational number and on the right is an irrational. Hence both $a,b=0$ and so
this is a basis.
\end{hint}
\end{problem}

\begin{problem}\label{prb:10.61} Suppose $V$ is a finite dimensional vector space. Based on the
exchange theorem above, it was shown that any two bases have the same number
of vectors in them. Give a different proof of this fact using the earlier
material in the book. \textbf{Hint: }Suppose $\left\{ \vec{x}_{1}\vec{
,\cdots ,x}_{n}\right\} $ and $\left\{ \vec{y}_{1}\vec{,\cdots ,y}
_{m}\right\} $ are two bases with $m<n.$ Then define
\begin{equation*}
\phi :\mathbb{R}^{n}\mapsto V,\psi :\mathbb{R}^{m}\mapsto V
\end{equation*}
by
\begin{equation*}
\phi \left( \vec{a}\right) = \sum_{k=1}^{n}a_{k}\vec{x}
_{k},\;\psi \left( \vec{b}\right) = \sum_{j=1}^{m}b_{j}\vec{y}_{j}
\end{equation*}
Consider the linear transformation, $\psi ^{-1}\circ \phi .$ Argue it is a
one to one and onto mapping from $\mathbb{R}^{n}$ to $\mathbb{R}^{m}.$ Now
consider a matrix of this linear transformation and its reduced row echelon form.
\begin{hint}
This is obvious because
when you add two of these you get one and when you multiply one of these by
a scalar, you get another one. A basis is $\left\{ 1,\sqrt{2}\right\} $. By
definition, the span of these gives the collection of vectors. Are they
independent? Say $a+b\sqrt{2}=0$ where $a,b$ are rational numbers. If $a\neq
0,$ then $b\sqrt{2}=-a$ which can't happen since $a$ is rational. If $b\neq
0,$ then $-a=b\sqrt{2}$ which again can't happen because on the left is a
rational number and on the right is an irrational. Hence both $a,b=0$ and so
this is a basis.
\end{hint}
\end{problem}

\begin{problem}\label{prb:10.62} Let $M=\left\{ \vec{u}=\left( u_{1},u_{2},u_{3},u_{4}\right) \in
\mathbb{R}^{4}:\left| u_{1}\right| \leq 4\right\} .$ Is $M$ a subspace of $\mathbb{R}^4$?
\begin{hint}
This is not a subspace. $\left[ \begin{array}{r}
1 \\
1 \\
1 \\
1
\end{array}
\right] $ is in
it, but $20\left[
\begin{array}{r}
1 \\
1 \\
1 \\
1
\end{array}
\right] $ is not.
\end{hint}
\end{problem}

\begin{problem}\label{prb:10.63} Let $M=\left\{ \vec{u}=\left( u_{1},u_{2},u_{3},u_{4}\right) \in
\mathbb{R}^{4}:\sin \left( u_{1}\right) =1\right\} .$ Is $M$ a subspace of $\mathbb{R}^4$?
\begin{hint}
This is not a subspace.
\end{hint}
\end{problem}

\begin{problem}\label{prb:10.64} Let $W$ be a subset of $M_{22}$ given by
\[
W = \left\{ A | A \in M_{22}, A^T = A \right\}
\]
In words, $W$ is the set of all symmetric $2 \times 2$ matrices. Is $W$ a subspace of $M_{22}$?
%\begin{hint}
%\end{hint}
\end{problem}

\begin{problem}\label{prb:10.65} Let $W$ be a subset of $M_{22}$ given by
\[
W = \left\{ \left[ \begin{array}{rr}
a  & b \\
c & d
\end{array} \right] | a,b,c,d \in \mathbb{R}, a + b = c + d \right\}
\]
Is $W$ a subspace of $M_{22}$?
%\begin{hint}
%\end{hint}
\end{problem}

\begin{problem}\label{prb:10.66} Let $W$ be a subset of $P_3$ given by
\[
W = \left\{
ax^3 + bx^2 + cx + d | a,b,c,d \in \mathbb{R}, d = 0 \right\}
\]
Is $W$ a subspace of $P_3$?
%\begin{hint}
%\end{hint}
\end{problem}

\begin{problem}\label{prb:10.67} Let $W$ be a subset of $P_3$ given by
\[
W = \left\{
p(x) = ax^3 + bx^2 + cx + d | a,b,c,d \in \mathbb{R}, p(2) = 1 \right\}
\]
Is $W$ a subspace of $P_3$?
%\begin{hint}
%\end{hint}
\end{problem}

\begin{problem}\label{prb:10.68}
Let $T:\mathbb{P}_2 \to \mathbb{R}$ be a linear transformation such that
\[ T(x^2)=1; T(x^2+x)=5; T(x^2+x+1)=-1.\]
Find $T(ax^2+bx+c)$.
\begin{hint}
By linearity we have
$T(x^2)=1$, $T(x) = T(x^2+x - x^2)= T(x^2+x) - T(x^2)= 5-1=5$, and
$T(1) = T(x^2+x+1 -(x^2+x))=T(x^2+x+1) -T(x^2+x))= -1-5=-6$.

Thus$T(ax^2+bx+c) = aT(x^2) + bT(x) + cT(1) = a+5b-6c$.
\end{hint}
\end{problem}

\begin{problem}\label{prb:10.69} Consider the following functions $T:\mathbb{R}^{3}\rightarrow \mathbb{R}^{2}.$
Explain why each of these functions $T$ is not linear.

\begin{enumerate}
\item $T\left[
\begin{array}{c}
x \\
y \\
z
\end{array}
\right] =\left[
\begin{array}{c}
x+2y+3z+1 \\
2y-3x+z
\end{array}
\right] $

\item $T\left[
\begin{array}{c}
x \\
y \\
z
\end{array}
\right] =\left[
\begin{array}{c}
x+2y^{2}+3z \\
2y+3x+z
\end{array}
\right] $

\item $T\left[
\begin{array}{c}
x \\
y \\
z
\end{array}
\right] =\left[
\begin{array}{c}
\sin x+2y+3z \\
2y+3x+z
\end{array}
\right] $

\item $T\left[
\begin{array}{c}
x \\
y \\
z
\end{array}
\right] =\left[
\begin{array}{c}
x+2y+3z \\
2y+3x-\ln z
\end{array}
\right] $
\end{enumerate}
%\begin{hint}
%\end{hint}
\end{problem}


\begin{problem}\label{prb:10.70} Suppose $T$ is a linear transformation such that
\begin{eqnarray*}
T\left[
\begin{array}{r}
1 \\
1 \\
-7
\end{array}
\right] &=&\left[
\begin{array}{r}
3 \\
3 \\
3
\end{array}
\right] \\
T\left[
\begin{array}{r}
-1 \\
0 \\
6
\end{array}
\right] &=&\left[
\begin{array}{r}
1 \\
2 \\
3
\end{array}
\right] \\
T\left[
\begin{array}{r}
0 \\
-1 \\
2
\end{array}
\right] &=&\left[
\begin{array}{r}
1 \\
3 \\
-1
\end{array}
\right]
\end{eqnarray*}
Find the matrix of $T$. That is find $A$ such that $T(\vec{x})=A\vec{x}$. \vspace{1mm}
\begin{hint}
\[
\left[
\begin{array}{rrr}
3 & 1 & 1 \\
3 & 2 & 3 \\
3 & 3 & -1
\end{array}
\right] \left[
\begin{array}{ccc}
6 & 2 & 1 \\
5 & 2 & 1 \\
6 & 1 & 1
\end{array}
\right] =\allowbreak \left[
\begin{array}{ccc}
29 & 9 & 5 \\
46 & 13 & 8 \\
27 & 11 & 5
\end{array}
\right]
\]
\end{hint}
\end{problem}

\begin{problem}\label{prb:10.71} Suppose $T$ is a linear transformation such that
\begin{eqnarray*}
T\left[
\begin{array}{r}
1 \\
2 \\
-18
\end{array}
\right] &=&\left[
\begin{array}{r}
5 \\
2 \\
5
\end{array}
\right] \\
T\left[
\begin{array}{r}
-1 \\
-1 \\
15
\end{array}
\right] &=&\left[
\begin{array}{r}
3 \\
3 \\
5
\end{array}
\right] \\
T\left[
\begin{array}{r}
0 \\
-1 \\
4
\end{array}
\right] &=&\left[
\begin{array}{r}
2 \\
5 \\
-2
\end{array}
\right]
\end{eqnarray*}
Find the matrix of $T$. That is find $A$ such that $T(\vec{x})=A\vec{x}$. \vspace{1mm}
\begin{hint}
\[
\left[
\begin{array}{rrr}
5 & 3 & 2 \\
2 & 3 & 5 \\
5 & 5 & -2
\end{array}
\right] \left[
\begin{array}{ccc}
11 & 4 & 1 \\
10 & 4 & 1 \\
12 & 3 & 1
\end{array}
\right] =\left[
\begin{array}{ccc}
109 & 38 & 10 \\
112 & 35 & 10 \\
81 & 34 & 8
\end{array}
\right]
\]
\end{hint}
\end{problem}


\begin{problem}\label{prb:10.72} Consider the following functions $T:\mathbb{R}^{3}\rightarrow \mathbb{R}^{2}$.
Show that each is a linear transformation and determine for each the matrix $A$ such that
$T(\vec{x})=A\vec{x}$.

\begin{enumerate}
\item $T\left[
\begin{array}{c}
x \\
y \\
z
\end{array}
\right] =\left[
\begin{array}{c}
x+2y+3z \\
2y-3x+z
\end{array}
\right] $

\item $T\left[
\begin{array}{c}
x \\
y \\
z
\end{array}
\right] =\left[
\begin{array}{c}
7x+2y+z \\
3x-11y+2z
\end{array}
\right] $

\item $T\left[
\begin{array}{c}
x \\
y \\
z
\end{array}
\right] =\left[
\begin{array}{c}
3x+2y+z \\
x+2y+6z
\end{array}
\right] $

\item $T\left[
\begin{array}{c}
x \\
y \\
z
\end{array}
\right] =\left[
\begin{array}{c}
2y-5x+z \\
x+y+z
\end{array}
\right] $
\end{enumerate}
%\begin{hint}
%\end{hint}
\end{problem}


\begin{problem}\label{prb:10.73} Suppose
\begin{equation*}
\left[
\begin{array}{ccc}
A_{1} & \cdots & A_{n}
\end{array}
\right] ^{-1}
\end{equation*}
 exists where each $A_{j}\in \mathbb{R}^{n}$ and let
vectors  $\left\{ B_{1},\cdots ,B_{n}\right\} $ in $\mathbb{R}^{m}$ be given.
Show that there \textbf{always }exists a linear
transformation $T$ such that $T(A_{i})=B_{i}$.
%\begin{hint}
%\end{hint}
\end{problem}

\begin{problem}\label{prb:10.74} Let $V$ and $W$ be subspaces of $\mathbb{R}^{n}$ and $\mathbb{R}^{m}$
respectively and let $T:V\rightarrow W$ be a linear transformation. Suppose
that $\left\{ T\vec{v}_{1},\cdots ,T\vec{v}_{r}\right\} $ is linearly
independent. Show that it must be the case that $\left\{ \vec{v}_{1},\cdots ,
\vec{v}_{r}\right\} $ is also linearly independent.
\begin{hint}
If $\sum_i^r a_i \vec{v}_r =0$, then using linearity properties of $T$ we get
\[ 0 = T(0) =  T(\sum_i^r a_i \vec{v}_r) =
\sum_i^r a_i T(\vec{v}_r).\]
Since we assume that  $\left\{ T\vec{v}_{1},\cdots ,T\vec{v}_{r}\right\} $ is linearly
independent, we must have all $a_i=0$, and therefore we conclude that
 $\left\{ \vec{v}_{1},\cdots ,
\vec{v}_{r}\right\} $ is also linearly independent.
\end{hint}
\end{problem}


\begin{problem}\label{prb:10.75} Let
\begin{equation*}
V=\mbox{span}\left\{ \left[
\begin{array}{c}
1 \\
1 \\
2 \\
0
\end{array}
\right] ,\left[
\begin{array}{c}
0 \\
1 \\
1 \\
1
\end{array}
\right] ,\left[
\begin{array}{c}
1 \\
1 \\
0 \\
1
\end{array}
\right] \right\}
\end{equation*}
Let $T\vec{x}=A\vec{x}$ where $A$ is the matrix
\begin{equation*}
\left[
\begin{array}{cccc}
1 & 1 & 1 & 1 \\
0 & 1 & 1 & 0 \\
0 & 1 & 2 & 1 \\
1 & 1 & 1 & 2
\end{array}
\right]
\end{equation*}
Give a basis for $\mbox{im}\left( T\right) $.
%\begin{hint}
%\end{hint}
\end{problem}


\begin{problem}\label{prb:10.76} Let
\begin{equation*}
V=\mbox{span}\left\{ \left[
\begin{array}{c}
1 \\
0 \\
0 \\
1
\end{array}
\right] ,\left[
\begin{array}{c}
1 \\
1 \\
1 \\
1
\end{array}
\right] ,\left[
\begin{array}{c}
1 \\
4 \\
4 \\
1
\end{array}
\right] \right\}
\end{equation*}
Let $T\vec{x}=A\vec{x}$ where $A$ is the matrix
\begin{equation*}
\left[
\begin{array}{cccc}
1 & 1 & 1 & 1 \\
0 & 1 & 1 & 0 \\
0 & 1 & 2 & 1 \\
1 & 1 & 1 & 2
\end{array}
\right]
\end{equation*}
Find a basis for $\mbox{im}\left( T\right) $. In this case, the original
vectors do not form an independent set.

\begin{hint}
Since the third vector is a linear combinations of the first two, then
the image of the third vector will also be a linear combinations of
the image of the first two.  However the image of the first two
vectors are linearly independent (check!), and hence form a basis of
the image.

Thus a basis for $\mbox{im}\left( T\right) $ is:

\begin{equation*}
V=\mbox{span}\left\{ \left[
\begin{array}{c}
2 \\
0 \\
1 \\
3
\end{array}
\right] ,\left[
\begin{array}{c}
4 \\
2 \\
4 \\
5
\end{array}
\right]  \right\}
\end{equation*}

\end{hint}
\end{problem}


\begin{problem}\label{prb:10.77} If $\left\{ \vec{v}_{1},\cdots ,\vec{v}_{r}\right\} $ is linearly
independent and $T$ is a one to one linear transformation, show that $
\left\{ T\vec{v}_{1},\cdots ,T\vec{v}_{r}\right\} $ is also linearly
independent. Give an example which shows that if $T$ is only linear, it can
happen that, although $\left\{ \vec{v}_{1},\cdots ,\vec{v}_{r}\right\} $ is
linearly independent, $\left\{ T\vec{v}_{1},\cdots ,T\vec{v}_{r}\right\} $
is not. In fact, show that it can happen that each of the $T\vec{v}_{j}$
equals 0.
%\begin{hint}
%\end{hint}
\end{problem}


\begin{problem}\label{prb:10.78} Let $V$ and $W$ be subspaces of $\mathbb{R}^{n}$ and $\mathbb{R}^{m}$
respectively and let $T:V\rightarrow W$ be a linear transformation. Show
that if $T$ is onto $W$ and if $\left\{ \vec{v}_{1},\cdots ,\vec{v}
_{r}\right\} $ is a basis for $V,$ then $\mbox{span}\left\{ T\vec{v}
_{1},\cdots ,T\vec{v}_{r}\right\} =W$.
%\begin{hint}
%\end{hint}
\end{problem}


\begin{problem}\label{prb:10.79} Define $T:\mathbb{R}^{4}\rightarrow \mathbb{R}^{3}$ as follows.
\begin{equation*}
T\vec{x}=\left[
\begin{array}{rrrr}
3 & 2 & 1 & 8 \\
2 & 2 & -2 & 6 \\
1 & 1 & -1 & 3
\end{array}
\right] \vec{x}
\end{equation*}
Find a basis for $\mbox{im}\left( T\right) $. Also find a basis for $\ker
\left( T\right) .$
%\begin{hint}
%\end{hint}
\end{problem}


\begin{problem}\label{prb:10.80} Define $T:\mathbb{R}^{3}\rightarrow \mathbb{R}^{3}$ as follows.
\begin{equation*}
T\vec{x}=\left[
\begin{array}{ccc}
1 & 2 & 0 \\
1 & 1 & 1 \\
0 & 1 & 1
\end{array}
\right] \vec{x}
\end{equation*}
where on the right, it is just matrix multiplication of the vector $\vec{x}$
which is meant. Explain why $T$ is an isomorphism of $\mathbb{R}^{3}$ to $
\mathbb{R}^{3}$.
%\begin{hint}
%\end{hint}
\end{problem}


\begin{problem}\label{prb:10.81} Suppose $T:\mathbb{R}^{3}\rightarrow \mathbb{R}^{3}$ is a linear
transformation given by
\begin{equation*}
T\vec{x}=A\vec{x}
\end{equation*}
where $A$ is a $3\times 3$ matrix. Show that $T$ is an isomorphism if and
only if $A$ is invertible.
%\begin{hint}
%\end{hint}
\end{problem}


\begin{problem}\label{prb:10.82} Suppose $T:\mathbb{R}^{n}\rightarrow \mathbb{R}^{m}$ is a linear
transformation given by
\begin{equation*}
T\vec{x}=A\vec{x}
\end{equation*}
where $A$ is an $m\times n$ matrix. Show that $T$ is never an isomorphism if
$m\neq n$. In particular, show that if $m>n,$ $T$ cannot be onto and if $
m<n, $ then $T$ cannot be one to one.
%\begin{hint}
%\end{hint}
\end{problem}


\begin{problem}\label{prb:10.83} Define $T:\mathbb{R}^{2}\rightarrow \mathbb{R}^{3}$ as follows.
\begin{equation*}
T\vec{x}=\left[
\begin{array}{cc}
1 & 0 \\
1 & 1 \\
0 & 1
\end{array}
\right] \vec{x}
\end{equation*}
where on the right, it is just matrix multiplication of the vector $\vec{x}$
which is meant. Show that $T$ is one to one. Next let $W=\mbox{im}\left(
T\right) .$ Show that $T$ is an isomorphism of $\mathbb{R}^{2}$ and $\mbox{im
}\left( T\right) $.
%\begin{hint}
%\end{hint}
\end{problem}


\begin{problem}\label{prb:10.84} In the above problem, find a $2\times 3$ matrix $A$ such that the
restriction of $A$ to $\mbox{im}\left( T\right) $ gives the same result as $
T^{-1}$ on $\mbox{im}\left( T\right) $. \textbf{Hint:\ }You might let $A$ be
such that
\begin{equation*}
A\left[
\begin{array}{c}
1 \\
1 \\
0
\end{array}
\right] =\left[
\begin{array}{c}
1 \\
0
\end{array}
\right] ,\ A\left[
\begin{array}{c}
0 \\
1 \\
1
\end{array}
\right] =\left[
\begin{array}{c}
0 \\
1
\end{array}
\right]
\end{equation*}
now find another vector $\vec{v}\in \mathbb{R}^{3}$ such that
\begin{equation*}
\left\{ \left[
\begin{array}{c}
1 \\
1 \\
0
\end{array}
\right] ,\left[
\begin{array}{c}
0 \\
1 \\
1
\end{array}
\right] ,\vec{v}\right\}
\end{equation*}
is a basis. You could pick
\begin{equation*}
\vec{v}=\left[
\begin{array}{c}
0 \\
0 \\
1
\end{array}
\right]
\end{equation*}
for example. Explain why this one works or one of your choice works. Then
you could define $A\vec{v}$ to equal some vector in $\mathbb{R}^{2}.$
Explain why there will be more than one such matrix $A$ which will deliver
the inverse isomorphism $T^{-1}$ on $\mbox{im}\left( T\right) $.
%\begin{hint}
%\end{hint}
\end{problem}


\begin{problem}\label{prb:10.85} Now let $V$ equal $\mbox{span}\left\{ \left[
\begin{array}{c}
1 \\
0 \\
1
\end{array}
\right] ,\left[
\begin{array}{c}
0 \\
1 \\
1
\end{array}
\right] \right\} $ and let $T:V\rightarrow W$ be a linear transformation
where
\begin{equation*}
W=\mbox{span}\left\{ \left[
\begin{array}{c}
1 \\
0 \\
1 \\
0
\end{array}
\right] ,\left[
\begin{array}{c}
0 \\
1 \\
1 \\
1
\end{array}
\right] \right\}
\end{equation*}
$\ $\ and
\begin{equation*}
T\left[
\begin{array}{c}
1 \\
0 \\
1
\end{array}
\right] =\left[
\begin{array}{c}
1 \\
0 \\
1 \\
0
\end{array}
\right] ,T\left[
\begin{array}{c}
0 \\
1 \\
1
\end{array}
\right] =\left[
\begin{array}{c}
0 \\
1 \\
1 \\
1
\end{array}
\right]
\end{equation*}
Explain why $T$ is an isomorphism. Determine a matrix $A$ which, when
multiplied on the left gives the same result as $T$ on $V$ and a matrix $B$
which delivers $T^{-1}$ on $W$. \textbf{Hint:\ }You need to have
\begin{equation*}
A\left[
\begin{array}{cc}
1 & 0 \\
0 & 1 \\
1 & 1
\end{array}
\right] =\left[
\begin{array}{cc}
1 & 0 \\
0 & 1 \\
1 & 1 \\
0 & 1
\end{array}
\right]
\end{equation*}
Now enlarge $\left[
\begin{array}{c}
1 \\
0 \\
1
\end{array}
\right] ,\left[
\begin{array}{c}
0 \\
1 \\
1
\end{array}
\right] $ to obtain a basis for $\mathbb{R}^{3}$. You could add in $\left[
\begin{array}{c}
0 \\
0 \\
1
\end{array}
\right] $ for example, and then pick another vector in $\mathbb{R}^{4}$ and
let $A\left[
\begin{array}{c}
0 \\
0 \\
1
\end{array}
\right] $ equal this other vector. Then you would have
\begin{equation*}
A\left[
\begin{array}{ccc}
1 & 0 & 0 \\
0 & 1 & 0 \\
1 & 1 & 1
\end{array}
\right] =\left[
\begin{array}{ccc}
1 & 0 & 0 \\
0 & 1 & 0 \\
1 & 1 & 0 \\
0 & 1 & 1
\end{array}
\right]
\end{equation*}
This would involve picking for the new vector in $\mathbb{R}^{4}$ the vector
$\left[
\begin{array}{cccc}
0 & 0 & 0 & 1
\end{array}
\right] ^{T}.$ Then you could find $A$. You can do something similar to find
a matrix for $T^{-1}$ denoted as $B$.
%\begin{hint}
%\end{hint}
\end{problem}

\begin{problem}\label{prb:10.86}
Let $V=\mathbb{R}^{3}$ and let
\begin{equation*}
W=\mbox{span} \left( S \right),  \mbox{ where } S=\left\{ \left[
\begin{array}{r}
1 \\
-1 \\
1
\end{array}
\right] ,\left[
\begin{array}{r}
-2 \\
2 \\
-2
\end{array}
\right],\left[
\begin{array}{r}
-1 \\
1 \\
1
\end{array}
\right],\left[
\begin{array}{r}
1 \\
-1 \\
3
\end{array}
\right] \right\}
\end{equation*}
Find a basis of $W$ consisting of vectors in $S$.

\begin{hint}
In this case $\dim (W)=1$ and a basis for $W$ consisting of vectors in $S$ can be obtained by taking any (nonzero) vector from $S$.
\end{hint}
\end{problem}


\begin{problem}\label{prb:10.87}
 Let $T$ be a linear transformation given by
\[
T \left[ \begin{array}{r}
x\\
y
\end{array}\right] = \left[ \begin{array}{rrr}
1 &1  \\
1 & 1
\end{array}\right]
\left[ \begin{array}{r}
x\\
y
\end{array}\right]
\]
Find a basis for $\ker \left( T\right)$ and $\mbox{im} \left( T\right) $.

\begin{hint}
A basis for $\ker \left( T\right)$ is
$\left\{ \left[
\begin{array}{r}
1 \\
-1
\end{array}
\right] \right\}$
and a basis for $\mbox{im} \left( T\right)$ is
$\left\{ \left[
\begin{array}{r}
1 \\
1
\end{array}
\right] \right\}$. \\
There are many other possibilities for the specific bases, but in this case
$\dim \left( \ker \left( T\right) \right)=1 $ and $\dim \left( \mbox{im} \left( T\right) \right)=1$.
\end{hint}

\end{problem}


\begin{problem}\label{prb:10.88}
 Let $T$ be a linear transformation given by
\[
T \left[ \begin{array}{r}
x\\
y
\end{array}\right] = \left[ \begin{array}{rrr}
1 & 0  \\
1 & 1
\end{array}\right]
\left[ \begin{array}{r}
x\\
y
\end{array}\right]
\]
Find a basis for $\ker \left( T\right)$ and $\mbox{im}
\left( T\right) $.

\begin{hint}
In this case $\ker \left( T\right) =\{0\}$
and $\mbox{im} \left( T\right) = \mathbb{R}^2$ (pick any basis of $\mathbb{R}^2$).
\end{hint}

\end{problem}



\begin{problem}\label{prb:10.89}
Let $V=\mathbb{R}^{3}$ and let
\begin{equation*}
W=\mbox{span}\left\{ \left[
\begin{array}{r}
1 \\
1 \\
1
\end{array}
\right] ,\left[
\begin{array}{r}
-1 \\
2 \\
-1
\end{array}
\right] \right\}
\end{equation*}
Extend this basis of $W$ to a basis of $V$.

\begin{hint}
There are many possible such extensions, one is (how do we know?):
\begin{equation*}
\left\{ \left[
\begin{array}{r}
1 \\
1 \\
1
\end{array}
\right] ,\left[
\begin{array}{r}
-1 \\
2 \\
-1
\end{array}
\right] ,\left[
\begin{array}{r}
0  \\
0\\
1
\end{array}
\right]
\right\}
\end{equation*}
\end{hint}
\end{problem}

\begin{problem}\label{prb:10.90}
 Let $T$ be a linear transformation given by
\[
T \left[ \begin{array}{r}
x\\
y \\
z
\end{array}\right] = \left[ \begin{array}{rrr}
1 & 1 & 1 \\
1 & 1 & 1
\end{array}\right]
\left[ \begin{array}{r}
x\\
y \\
z
\end{array}\right]
\]
What is $\dim  ( \ker \left( T \right) )$?

\begin{hint}
We can easily see that $\dim  ( \mbox{im} \left( T \right) ) =1$, and thus
$\dim  ( \ker \left( T \right) ) = 3 - \dim  ( \mbox{im} \left( T \right) ) = 3- 1 = 2$.
\end{hint}
\end{problem}

\begin{problem}\label{prb:10.91} \label{exerlineartransf}
Consider the following functions which map $\mathbb{R}^{n}$ to $\mathbb{R}^{n}$.

\begin{enumerate}
\item $T$ multiplies the $j^{th}$ component of $\vec{x}$ by a nonzero
number $b.$

\item $T$ replaces the $i^{th}$ component of $\vec{x}$ with $b$ times the
$j^{th}$ component added to the $i^{th}$ component.

\item $T$ switches the $i^{th}$ and $j^{th}$ components.
\end{enumerate}

Show these functions are linear transformations and describe their matrices $A$ such that $T\left(\vec{x}\right) = A\vec{x}$.
\begin{hint}
\begin{enumerate}
\item The matrix of $T$ is the elementary matrix which multiplies
the $j^{th}$ diagonal entry of the identity matrix by $b$.
\item The matrix of $T$ is the
elementary matrix which takes $b$ times the $j^{th}$ row and adds to the $%
i^{th}$ row.
\item The matrix of $T$ is the elementary matrix which switches the $%
i^{th}$ and the $j^{th}$ rows where the two components are in the $i^{th}$
and $j^{th}$ positions.
\end{enumerate}
\end{hint}
\end{problem}

\begin{problem}\label{prb:10.92} You are given a linear transformation $T:\mathbb{R}^{n}\rightarrow
\mathbb{R}^{m}$ and you know that
\begin{equation*}
T\left(A_{i}\right)=B_{i}
\end{equation*}
where $\left[
\begin{array}{ccc}
A_{1} & \cdots & A_{n}
\end{array}
\right] ^{-1}$ exists. Show that the matrix of $T$ is of the form
\begin{equation*}
\left[
\begin{array}{ccc}
B_{1} & \cdots & B_{n}
\end{array}
\right] \left[
\begin{array}{ccc}
A_{1} & \cdots & A_{n}
\end{array}
\right] ^{-1}
\end{equation*}
\begin{hint}
Suppose
\[
\left[
\begin{array}{c}
\vec{c}_{1}^{T} \\
\vdots \\
\vec{c}_{n}^{T}
\end{array}
\right] =\left[
\begin{array}{ccc}
\vec{a}_{1} & \cdots & \vec{a}_{n}
\end{array}
\right]^{-1}
\]
Thus $\vec{c}_{i}^{T}\vec{a}_{j}=\delta _{ij}$. Therefore,
\begin{eqnarray*}
\left[
\begin{array}{ccc}
\vec{b}_{1} & \cdots & \vec{b}_{n}
\end{array}
\right] \left[
\begin{array}{ccc}
\vec{a}_{1} & \cdots & \vec{a}_{n}
\end{array}
\right] ^{-1}\vec{a}_{i} &=&
\left[
\begin{array}{ccc}
\vec{b}_{1} & \cdots & \vec{b}_{n}
\end{array}
\right] \left[
\begin{array}{c}
\vec{c}_{1}^{T} \\
\vdots \\
\vec{c}_{n}^{T}
\end{array}
\right] \vec{a}_{i} \\
&=&\left[
\begin{array}{ccc}
\vec{b}_{1} & \cdots & \vec{b}_{n}
\end{array}
\right] \vec{e}_{i} \\
&=&\vec{b}_{i}
\end{eqnarray*}
Thus $T\vec{a}_{i}=\left[
\begin{array}{ccc}
\vec{b}_{1} & \cdots & \vec{b}_{n}
\end{array}
\right] \left[
\begin{array}{ccc}
\vec{a}_{1} & \cdots & \vec{a}_{n}
\end{array}
\right] ^{-1}\vec{a}_{i} =  A\vec{a}_{i}.$ If $\vec{x}$ is
arbitrary, then since the matrix $\left[
\begin{array}{ccc}
\vec{a}_{1} & \cdots & \vec{a}_{n}
\end{array}
\right] $ is invertible, there exists a unique $\vec{y}$ such that $
\left[
\begin{array}{ccc}
\vec{a}_{1} & \cdots & \vec{a}_{n}
\end{array}
\right] \vec{y}=\vec{x}$ Hence
\[
T\vec{x}=T\left( \sum_{i=1}^{n}y_{i}\vec{a}_{i}\right)
=\sum_{i=1}^{n}y_{i}T\vec{a}_{i}=\sum_{i=1}^{n}y_{i}A\vec{a}
_{i}=A\left( \sum_{i=1}^{n}y_{i}\vec{a}_{i}\right) =A\vec{x}
\]

\end{hint}
\end{problem}

\begin{problem}\label{prb:10.93} Suppose $T$ is a linear transformation such that
\begin{eqnarray*}
T\left[
\begin{array}{r}
1 \\
2 \\
-6
\end{array}
\right] &=&\left[
\begin{array}{r}
5 \\
1 \\
3
\end{array}
\right] \\
T\left[
\begin{array}{r}
-1 \\
-1 \\
5
\end{array}
\right] &=&\left[
\begin{array}{r}
1 \\
1 \\
5
\end{array}
\right] \\
T\left[
\begin{array}{r}
0 \\
-1 \\
2
\end{array}
\right] &=&\left[
\begin{array}{r}
5 \\
3 \\
-2
\end{array}
\right]
\end{eqnarray*}
Find the matrix of $T$. That is find $A$ such that $T(\vec{x})=A\vec{x}$. \vspace{1mm}
\begin{hint}
\[
\left[
\begin{array}{rrr}
5 & 1 & 5 \\
1 & 1 & 3 \\
3 & 5 & -2
\end{array}
\right] \left[
\begin{array}{ccc}
3 & 2 & 1 \\
2 & 2 & 1 \\
4 & 1 & 1
\end{array}
\right] =\left[
\begin{array}{ccc}
37 & 17 & 11 \\
17 & 7 & 5 \\
11 & 14 & 6
\end{array}
\right]
\]
\end{hint}
\end{problem}

\begin{problem}\label{prb:10.94} Suppose $T$ is a linear transformation such that
\begin{eqnarray*}
T\left[
\begin{array}{r}
1 \\
1 \\
-8
\end{array}
\right] &=&\left[
\begin{array}{r}
1 \\
3 \\
1
\end{array}
\right] \\
T\left[
\begin{array}{r}
-1 \\
0 \\
6
\end{array}
\right] &=&\left[
\begin{array}{r}
2 \\
4 \\
1
\end{array}
\right] \\
T\left[
\begin{array}{r}
0 \\
-1 \\
3
\end{array}
\right] &=&\left[
\begin{array}{r}
6 \\
1 \\
-1
\end{array}
\right]
\end{eqnarray*}
Find the matrix of $T$. That is find $A$ such that $T(\vec{x})=A\vec{x}$. \vspace{1mm}
\begin{hint}
\[
\left[
\begin{array}{rrr}
1 & 2 & 6 \\
3 & 4 & 1 \\
1 & 1 & -1
\end{array}
\right] \left[
\begin{array}{ccc}
6 & 3 & 1 \\
5 & 3 & 1 \\
6 & 2 & 1
\end{array}
\right] =\left[
\begin{array}{ccc}
52 & 21 & 9 \\
44 & 23 & 8 \\
5 & 4 & 1
\end{array}
\right]
\]
\end{hint}
\end{problem}

\begin{problem}\label{prb:10.95} Suppose $T$ is a linear transformation such that
\begin{eqnarray*}
T\left[
\begin{array}{r}
1 \\
3 \\
-7
\end{array}
\right] &=&\left[
\begin{array}{r}
-3 \\
1 \\
3
\end{array}
\right] \\
T\left[
\begin{array}{r}
-1 \\
-2 \\
6
\end{array}
\right] &=&\left[
\begin{array}{r}
1 \\
3 \\
-3
\end{array}
\right] \\
T\left[
\begin{array}{r}
0 \\
-1 \\
2
\end{array}
\right] &=&\left[
\begin{array}{r}
5 \\
3 \\
-3
\end{array}
\right]
\end{eqnarray*}
Find the matrix of $T$. That is find $A$ such that $T(\vec{x})=A\vec{x}$. \vspace{1mm}\vspace{1mm}
\begin{hint}
\[
\left[
\begin{array}{rrr}
-3 & 1 & 5 \\
1 & 3 & 3 \\
3 & -3 & -3
\end{array}
\right] \left[
\begin{array}{ccc}
2 & 2 & 1 \\
1 & 2 & 1 \\
4 & 1 & 1
\end{array}
\right] = \left[
\begin{array}{rrr}
15 & 1 & 3 \\
17 & 11 & 7 \\
-9 & -3 & -3
\end{array}
\right]
\]
\end{hint}
\end{problem}

\begin{problem}\label{prb:10.96} Suppose $T$ is a linear transformation such that
\begin{eqnarray*}
T\left[
\begin{array}{r}
1 \\
1 \\
-7
\end{array}
\right] &=&\left[
\begin{array}{r}
3 \\
3 \\
3
\end{array}
\right] \\
T\left[
\begin{array}{r}
-1 \\
0 \\
6
\end{array}
\right] &=&\left[
\begin{array}{r}
1 \\
2 \\
3
\end{array}
\right] \\
T\left[
\begin{array}{r}
0 \\
-1 \\
2
\end{array}
\right] &=&\left[
\begin{array}{r}
1 \\
3 \\
-1
\end{array}
\right]
\end{eqnarray*}
Find the matrix of $T$. That is find $A$ such that $T(\vec{x})=A\vec{x}$. \vspace{1mm}
\begin{hint}
\[
\left[
\begin{array}{rrr}
3 & 1 & 1 \\
3 & 2 & 3 \\
3 & 3 & -1
\end{array}
\right] \left[
\begin{array}{ccc}
6 & 2 & 1 \\
5 & 2 & 1 \\
6 & 1 & 1
\end{array}
\right] =\allowbreak \left[
\begin{array}{ccc}
29 & 9 & 5 \\
46 & 13 & 8 \\
27 & 11 & 5
\end{array}
\right]
\]
\end{hint}
\end{problem}

\begin{problem}\label{prb:10.97} Suppose $T$ is a linear transformation such that
\begin{eqnarray*}
T\left[
\begin{array}{r}
1 \\
2 \\
-18
\end{array}
\right] &=&\left[
\begin{array}{r}
5 \\
2 \\
5
\end{array}
\right] \\
T\left[
\begin{array}{r}
-1 \\
-1 \\
15
\end{array}
\right] &=&\left[
\begin{array}{r}
3 \\
3 \\
5
\end{array}
\right] \\
T\left[
\begin{array}{r}
0 \\
-1 \\
4
\end{array}
\right] &=&\left[
\begin{array}{r}
2 \\
5 \\
-2
\end{array}
\right]
\end{eqnarray*}
Find the matrix of $T$. That is find $A$ such that $T(\vec{x})=A\vec{x}$. \vspace{1mm}
\begin{hint}
\[
\left[
\begin{array}{rrr}
5 & 3 & 2 \\
2 & 3 & 5 \\
5 & 5 & -2
\end{array}
\right] \left[
\begin{array}{ccc}
11 & 4 & 1 \\
10 & 4 & 1 \\
12 & 3 & 1
\end{array}
\right] =\left[
\begin{array}{ccc}
109 & 38 & 10 \\
112 & 35 & 10 \\
81 & 34 & 8
\end{array}
\right]
\]
\end{hint}
\end{problem}


\begin{problem}\label{prb:10.98} Consider the following functions $T:\mathbb{R}^{3}\rightarrow \mathbb{R}^{2}$.
Show that each is a linear transformation and determine for each the matrix $A$ such that
$T(\vec{x})=A\vec{x}$.

\begin{enumerate}
\item $T\left[
\begin{array}{c}
x \\
y \\
z
\end{array}
\right] =\left[
\begin{array}{c}
x+2y+3z \\
2y-3x+z
\end{array}
\right] $

\item $T\left[
\begin{array}{c}
x \\
y \\
z
\end{array}
\right] =\left[
\begin{array}{c}
7x+2y+z \\
3x-11y+2z
\end{array}
\right] $

\item $T\left[
\begin{array}{c}
x \\
y \\
z
\end{array}
\right] =\left[
\begin{array}{c}
3x+2y+z \\
x+2y+6z
\end{array}
\right] $

\item $T\left[
\begin{array}{c}
x \\
y \\
z
\end{array}
\right] =\left[
\begin{array}{c}
2y-5x+z \\
x+y+z
\end{array}
\right] $
\end{enumerate}
%\begin{hint}
%\end{hint}
\end{problem}

\begin{problem}\label{prb:10.99} Consider the following functions $T:\mathbb{R}^{3}\rightarrow \mathbb{R}^{2}.$
Explain why each of these functions $T$ is not linear.

\begin{enumerate}
\item $T\left[
\begin{array}{c}
x \\
y \\
z
\end{array}
\right] =\left[
\begin{array}{c}
x+2y+3z+1 \\
2y-3x+z
\end{array}
\right] $

\item $T\left[
\begin{array}{c}
x \\
y \\
z
\end{array}
\right] =\left[
\begin{array}{c}
x+2y^{2}+3z \\
2y+3x+z
\end{array}
\right] $

\item $T\left[
\begin{array}{c}
x \\
y \\
z
\end{array}
\right] =\left[
\begin{array}{c}
\sin x+2y+3z \\
2y+3x+z
\end{array}
\right] $

\item $T\left[
\begin{array}{c}
x \\
y \\
z
\end{array}
\right] =\left[
\begin{array}{c}
x+2y+3z \\
2y+3x-\ln z
\end{array}
\right] $
\end{enumerate}
%\begin{hint}
%\end{hint}
\end{problem}


\begin{problem}\label{prb:10.100} Suppose
\begin{equation*}
\left[
\begin{array}{ccc}
A_{1} & \cdots & A_{n}
\end{array}
\right] ^{-1}
\end{equation*}
 exists where each $A_{j}\in \mathbb{R}^{n}$ and let
vectors  $\left\{ B_{1},\cdots ,B_{n}\right\} $ in $\mathbb{R}^{m}$ be given.
Show that there \textbf{always }exists a linear
transformation $T$ such that $T(A_{i})=B_{i}$.
%\begin{hint}
%\end{hint}
\end{problem}


\begin{problem}\label{prb:10.101}  Find the matrix for $T\left(\vec{w} \right) = \mbox{proj}_{\vec{v}}\left( \vec{w}\right) $
where $\vec{v}=\left[
\begin{array}{rrr}
1 & -2 & 3
\end{array}
\right] ^{T}.$
\begin{hint}
 Recall that $\mbox{proj}_{\vec{u}}\left( \vec{v}\right) =\frac{\vec{v}\dotp\vec{u} }{\norm{\vec{u}}^{2}}\vec{u}$ and so the desired matrix
has $i^{th}$ column equal to $\mbox{proj}_{\vec{u}}\left( \vec{e}_{i}\right) .$ Therefore, the matrix desired is
\[
\frac{1}{14}\left[
\begin{array}{rrr}
1 & -2 & 3 \\
-2 & 4 & -6 \\
3 & -6 & 9
\end{array}
\right]
\]
\end{hint}
\end{problem}

\begin{problem}\label{prb:10.102}  Find the matrix for $T\left(\vec{w} \right) = \mbox{proj}_{\vec{v}}\left( \vec{w}\right) $
where $\vec{v}=\left[
\begin{array}{rrr}
1 & 5 & 3
\end{array}
\right] ^{T}.$
\begin{hint}
\[
\frac{1}{35}\left[
\begin{array}{rrr}
1 & 5 & 3 \\
5 & 25 & 15 \\
3 & 15 & 9
\end{array}
\right]
\]
\end{hint}
\end{problem}

\begin{problem}\label{prb:10.103} Find the matrix for $T\left(\vec{w} \right) = \mbox{proj}_{\vec{v}}\left( \vec{w}\right) $
where $\vec{v}=\left[
\begin{array}{rrr}
1 & 0 & 3
\end{array}
\right] ^{T}.$
\begin{hint}
\[
\frac{1}{10}\left[
\begin{array}{ccc}
1 & 0 & 3 \\
0 & 0 & 0 \\
3 & 0 & 9
\end{array}
\right]
\]
\end{hint}
\end{problem}





\begin{problem}\label{prb:10.104}
Let $B = \left\{ \left[ \begin{array}{r}
2 \\
-1
\end{array} \right], \left[ \begin{array}{r}
3 \\
2
\end{array} \right] \right\}$ be a basis of $\mathbb{R}^2$ and let $\vec{x} = \left[
\begin{array}{r}
5 \\
-7
\end{array}
\right]$ be a vector in $\mathbb{R}^2$. Find $C_B(\vec{x})$.
\end{problem}

\begin{problem}\label{prb:10.105}
Let $B = \left\{ \left[ \begin{array}{r}
1 \\
-1 \\
2
\end{array} \right], \left[ \begin{array}{r}
2 \\
1 \\
2
 \end{array} \right], \left[ \begin{array}{r}
-1 \\
0 \\
2
\end{array} \right] \right\}$
be a basis of $\mathbb{R}^3$ and let $\vec{x} = \left[
\begin{array}{r}
5 \\
-1 \\
4
\end{array}
\right]$ be a vector in $\mathbb{R}^2$. Find $C_B(\vec{x})$.
\begin{hint}
 $C_B(\vec{x}) =
\left[ \begin{array}{r}
2 \\
1 \\
-1
 \end{array} \right] $.
\end{hint}
\end{problem}


\begin{problem}\label{prb:10.106}
Let $T: \mathbb{R}^2 \mapsto \mathbb{R}^2$ be a linear transformation defined by $T \left( \left[ \begin{array}{r}
a \\
b
\end{array} \right] \right) = \left[ \begin{array}{r}
a+b \\
a-b
\end{array} \right]$.

Consider the two bases
\[
B_1 = \left\{ \vec{v}_{1}, \vec{v}_{2} \right\} = \left\{ \left[ \begin{array}{r}
1 \\
0
\end{array}\right], \left[ \begin{array}{r}
-1 \\
1
\end{array}
\right]
\right\}
\]
 and
\[
B_2 = \left\{ \left[ \begin{array}{r}
1 \\
1
\end{array}
\right], \left[ \begin{array}{r}
1 \\
-1
\end{array}
\right]
\right\}
\]

Find the matrix $M_{B_2,B_1}$ of $T$ with respect to the bases $B_1$ and $B_2$.
\begin{hint}
$
M_{B_{2} B_{1}} = \left[
\begin{array}{rr}
 1 & 0 \\
 -1 & 1
\end{array}
\right] $
\end{hint}
\end{problem}

\begin{problem}\label{prob:inner_prod_7}
If $p = p(x)$ and $q = q(x)$ are polynomials in $\mathbb{P}^{n}$, define
\begin{equation*}
\langle p, q \rangle = p(0)q(0) + p(1)q(1) + \dots + p(n)q(n)
\end{equation*}
Show that this is an inner product on $\mathbb{P}^{n}$.
\end{problem}

\begin{problem}\label{prob:inner_prod_8}
Let $\mathcal{D}_{n}$ denote the space of all functions from the set
$\{1, 2, 3, \dots, n\}$ to $\RR$ with pointwise addition and
scalar multiplication. Show
that $\langle\ , \rangle$ is an inner product on $\mathcal{D}_{n}$ if \newline $\langle\vec{f}, \vec{g}\rangle = f(1)g(1) + f(2)g(2) + \dots + f(n)g(n)$.

\begin{hint}
P1 and P2 are clear since $f(i)$ and $g(i)$ are real numbers.

\begin{flalign*}
\mbox{P3: } \langle f + g, h \rangle &= \sum_{i}(f + g)(i) \dotp h(i) &\\
&= \sum_{i}(f(i) + g(i)) \dotp h(i) &\\
&= \sum_{i}[f(i)h(i) + g(i)h(i)] &\\
&= \sum_{i}f(i)h(i) + \sum_{i}g(i)h(i) &\\
&= \langle f, h \rangle + \langle g, h \rangle. &\\
\mbox{P4: } \hspace{1em}\langle rf, g \rangle &= \sum_{i}(rf)(i) \dotp g(i) &\\
&= \sum_{i}rf(i) \dotp g(i) &\\
&= r \sum_{i}f(i) \dotp g(i) &\\
&= r\langle f, g \rangle &\\
\end{flalign*}

P5: If $ f \neq 0 $, then $\langle f, f \rangle = \displaystyle \sum_{i}f(i)^2 > 0 $ because some $f(i) \neq 0$.
\end{hint}
\end{problem}



\section*{Practice Problem Source}
Problems \ref{prb:10.1} to \ref{prb:10.106} come from Chapter 9 of Ken Kuttler's \href{https://open.umn.edu/opentextbooks/textbooks/a-first-course-in-linear-algebra-2017}{\it A First Course in Linear Algebra}. (CC-BY)

Ken Kuttler, {\it  A First Course in Linear Algebra}, Lyryx 2017, Open Edition, pp. 469--535.

Problems \ref{prob:inner_prod_7} to \ref{} come from Section 10.1 of Keith Nicholson's \href{https://open.umn.edu/opentextbooks/textbooks/linear-algebra-with-applications}{\it Linear Algebra with Applications}. (CC-BY-NC-SA)

W. Keith Nicholson, {\it Linear Algebra with Applications}, Lyryx 2018, Open Edition, pp. 528--530.

\end{document}