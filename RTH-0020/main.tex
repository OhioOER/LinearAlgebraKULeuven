\documentclass{ximera}
%%% Begin Laad packages

\makeatletter
\@ifclassloaded{xourse}{%
    \typeout{Start loading preamble.tex (in a XOURSE)}%
    \def\isXourse{true}   % automatically defined; pre 112022 it had to be set 'manually' in a xourse
}{%
    \typeout{Start loading preamble.tex (NOT in a XOURSE)}%
}
\makeatother

\def\isEn\true 

\pgfplotsset{compat=1.16}

\usepackage{currfile}

% 201908/202301: PAS OP: babel en doclicense lijken problemen te veroorzaken in .jax bestand
% (wegens syntax error met toegevoegde \newcommands ...)
\pdfOnly{
    \usepackage[type={CC},modifier={by-nc-sa},version={4.0}]{doclicense}
    %\usepackage[hyperxmp=false,type={CC},modifier={by-nc-sa},version={4.0}]{doclicense}
    %%% \usepackage[dutch]{babel}
}



\usepackage[utf8]{inputenc}
\usepackage{morewrites}   % nav zomercursus (answer...?)
\usepackage{multirow}
\usepackage{multicol}
\usepackage{tikzsymbols}
\usepackage{ifthen}
%\usepackage{animate} BREAKS HTML STRUCTURE USED BY XIMERA
\usepackage{relsize}

\usepackage{eurosym}    % \euro  (€ werkt niet in xake ...?)
\usepackage{fontawesome} % smileys etc

% Nuttig als ook interactieve beamer slides worden voorzien:
\providecommand{\p}{} % default nothing ; potentially usefull for slides: redefine as \pause
%providecommand{\p}{\pause}

    % Layout-parameters voor het onderschrift bij figuren
\usepackage[margin=10pt,font=small,labelfont=bf, labelsep=endash,format=hang]{caption}
%\usepackage{caption} % captionof
%\usepackage{pdflscape}    % landscape environment

% Met "\newcommand\showtodonotes{}" kan je todonotes tonen (in pdf/online)
% 201908: online werkt het niet (goed)
\providecommand\showtodonotes{disable}
\providecommand\todo[1]{\typeout{TODO #1}}
%\usepackage[\showtodonotes]{todonotes}
%\usepackage{todonotes}

%
% Poging tot aanpassen layout
%
\usepackage{tcolorbox}
\tcbuselibrary{theorems}

%%% Einde laad packages

%%% Begin Ximera specifieke zaken

\graphicspath{
	{../../}
	{../}
	{./}
  	{../../pictures/}
   	{../pictures/}
   	{./pictures/}
	{./explog/}    % M05 in groeimodellen       
}

%%% Einde Ximera specifieke zaken

%
% define softer blue/red/green, use KU Leuven base colors for blue (and dark orange for red ?)
%
% todo: rather redefine blue/red/green ...?
%\definecolor{xmblue}{rgb}{0.01, 0.31, 0.59}
%\definecolor{xmred}{rgb}{0.89, 0.02, 0.17}
\definecolor{xmdarkblue}{rgb}{0.122, 0.671, 0.835}   % KU Leuven Blauw
\definecolor{xmblue}{rgb}{0.114, 0.553, 0.69}        % KU Leuven Blauw
\definecolor{xmgreen}{rgb}{0.13, 0.55, 0.13}         % No KULeuven variant for green found ...

\definecolor{xmaccent}{rgb}{0.867, 0.541, 0.18}      % KU Leuven Accent (orange ...)
\definecolor{kuaccent}{rgb}{0.867, 0.541, 0.18}      % KU Leuven Accent (orange ...)

\colorlet{xmred}{xmaccent!50!black}                  % Darker version of KU Leuven Accent

\providecommand{\blue}[1]{{\color{blue}#1}}    
\providecommand{\red}[1]{{\color{red}#1}}

\renewcommand\CancelColor{\color{xmaccent!50!black}}

% werkt in math en text mode om MATH met oranje (of grijze...)  achtergond te tonen (ook \important{\text{blabla}} lijkt te werken)
%\newcommand{\important}[1]{\ensuremath{\colorbox{xmaccent!50!white}{$#1$}}}   % werkt niet in Mathjax
%\newcommand{\important}[1]{\ensuremath{\colorbox{lightgray}{$#1$}}}
\newcommand{\important}[1]{\ensuremath{\colorbox{orange}{$#1$}}}   % TODO: kleur aanpassen voor mathjax; wordt overschreven infra!


% Uitzonderlijk kan met \pdfnl in de PDF een newline worden geforceerd, die online niet nodig/nuttig is omdat daar de regellengte hoe dan ook niet gekend is.
\ifdefined\HCode%
\providecommand{\pdfnl}{}%
\else%
\providecommand{\pdfnl}{%
  \\%
}%
\fi

% Uitzonderlijk kan met \handoutnl in de handout-PDF een newline worden geforceerd, die noch online noch in de PDF-met-antwoorden nuttig is.
\ifdefined\HCode
\providecommand{\handoutnl}{}
\else
\providecommand{\handoutnl}{%
\ifhandout%
  \nl%
\fi%
}
\fi



% \cellcolor IGNORED by tex4ht ?
% \begin{center} seems not to wordk
    % (missing margin-left: auto;   on tabular-inside-center ???)
%\newcommand{\importantcell}[1]{\ensuremath{\cellcolor{lightgray}#1}}  %  in tabular; usablility to be checked
\providecommand{\importantcell}[1]{\ensuremath{#1}}     % no mathjax2 support for colloring array cells

\pdfOnly{
  \renewcommand{\important}[1]{\ensuremath{\colorbox{kuaccent!50!white}{$#1$}}}
  \renewcommand{\importantcell}[1]{\ensuremath{\cellcolor{kuaccent!40!white}#1}}   
}

%%% Tikz styles


\pgfplotsset{compat=1.16}

\usetikzlibrary{trees,positioning,arrows,fit,shapes,math,calc,decorations.markings,through,intersections,patterns,matrix}

\usetikzlibrary{decorations.pathreplacing,backgrounds}    % 5/2023: from experimental


\usetikzlibrary{angles,quotes}

\usepgfplotslibrary{fillbetween} % bepaalde_integraal
\usepgfplotslibrary{polar}    % oa voor poolcoordinaten.tex

\pgfplotsset{ownstyle/.style={axis lines = center, axis equal image, xlabel = $x$, ylabel = $y$, enlargelimits}} 

\pgfplotsset{
	plot/.style={no marks,samples=50}
}

\newcommand{\xmPlotsColor}{
	\pgfplotsset{
		plot1/.style={darkgray,no marks,samples=100},
		plot2/.style={lightgray,no marks,samples=100},
		plotresult/.style={blue,no marks,samples=100},
		plotblue/.style={blue,no marks,samples=100},
		plotred/.style={red,no marks,samples=100},
		plotgreen/.style={green,no marks,samples=100},
		plotpurple/.style={purple,no marks,samples=100}
	}
}
\newcommand{\xmPlotsBlackWhite}{
	\pgfplotsset{
		plot1/.style={black,loosely dashed,no marks,samples=100},
		plot2/.style={black,loosely dotted,no marks,samples=100},
		plotresult/.style={black,no marks,samples=100},
		plotblue/.style={black,no marks,samples=100},
		plotred/.style={black,dotted,no marks,samples=100},
		plotgreen/.style={black,dashed,no marks,samples=100},
		plotpurple/.style={black,dashdotted,no marks,samples=100}
	}
}


\newcommand{\xmPlotsColorAndStyle}{
	\pgfplotsset{
		plot1/.style={darkgray,no marks,samples=100},
		plot2/.style={lightgray,no marks,samples=100},
		plotresult/.style={blue,no marks,samples=100},
		plotblue/.style={xmblue,no marks,samples=100},
		plotred/.style={xmred,dashed,thick,no marks,samples=100},
		plotgreen/.style={xmgreen,dotted,very thick,no marks,samples=100},
		plotpurple/.style={purple,no marks,samples=100}
	}
}


%\iftikzexport
\xmPlotsColorAndStyle
%\else
%\xmPlotsBlackWhite
%\fi
%%%


%
% Om venndiagrammen te arceren ...
%
\makeatletter
\pgfdeclarepatternformonly[\hatchdistance,\hatchthickness]{north east hatch}% name
{\pgfqpoint{-1pt}{-1pt}}% below left
{\pgfqpoint{\hatchdistance}{\hatchdistance}}% above right
{\pgfpoint{\hatchdistance-1pt}{\hatchdistance-1pt}}%
{
	\pgfsetcolor{\tikz@pattern@color}
	\pgfsetlinewidth{\hatchthickness}
	\pgfpathmoveto{\pgfqpoint{0pt}{0pt}}
	\pgfpathlineto{\pgfqpoint{\hatchdistance}{\hatchdistance}}
	\pgfusepath{stroke}
}
\pgfdeclarepatternformonly[\hatchdistance,\hatchthickness]{north west hatch}% name
{\pgfqpoint{-\hatchthickness}{-\hatchthickness}}% below left
{\pgfqpoint{\hatchdistance+\hatchthickness}{\hatchdistance+\hatchthickness}}% above right
{\pgfpoint{\hatchdistance}{\hatchdistance}}%
{
	\pgfsetcolor{\tikz@pattern@color}
	\pgfsetlinewidth{\hatchthickness}
	\pgfpathmoveto{\pgfqpoint{\hatchdistance+\hatchthickness}{-\hatchthickness}}
	\pgfpathlineto{\pgfqpoint{-\hatchthickness}{\hatchdistance+\hatchthickness}}
	\pgfusepath{stroke}
}
%\makeatother

\tikzset{
    hatch distance/.store in=\hatchdistance,
    hatch distance=10pt,
    hatch thickness/.store in=\hatchthickness,
   	hatch thickness=2pt
}

\colorlet{circle edge}{black}
\colorlet{circle area}{blue!20}


\tikzset{
    filled/.style={fill=green!30, draw=circle edge, thick},
    arceerl/.style={pattern=north east hatch, pattern color=blue!50, draw=circle edge},
    arceerr/.style={pattern=north west hatch, pattern color=yellow!50, draw=circle edge},
    outline/.style={draw=circle edge, thick}
}




%%% Updaten commando's
\def\hoofding #1#2#3{\maketitle}     % OBSOLETE ??

% we willen (bijna) altijd \geqslant ipv \geq ...!
\newcommand{\geqnoslant}{\geq}
\renewcommand{\geq}{\geqslant}
\newcommand{\leqnoslant}{\leq}
\renewcommand{\leq}{\leqslant}

% Todo: (201908) waarom komt er (soms) underlined voor emph ...?
\renewcommand{\emph}[1]{\textit{#1}}

% API commando's

\newcommand{\ds}{\displaystyle}
\newcommand{\ts}{\textstyle}  % tegenhanger van \ds   (Ximera zet PER  DEFAULT \ds!)

% uit Zomercursus-macro's: 
\newcommand{\bron}[1]{\begin{scriptsize} \emph{#1} \end{scriptsize}}     % deprecated ...?


%definities nieuwe commando's - afkortingen veel gebruikte symbolen
\newcommand{\R}{\ensuremath{\mathbb{R}}}
\newcommand{\Rnul}{\ensuremath{\mathbb{R}_0}}
\newcommand{\Reen}{\ensuremath{\mathbb{R}\setminus\{1\}}}
\newcommand{\Rnuleen}{\ensuremath{\mathbb{R}\setminus\{0,1\}}}
\newcommand{\Rplus}{\ensuremath{\mathbb{R}^+}}
\newcommand{\Rmin}{\ensuremath{\mathbb{R}^-}}
\newcommand{\Rnulplus}{\ensuremath{\mathbb{R}_0^+}}
\newcommand{\Rnulmin}{\ensuremath{\mathbb{R}_0^-}}
\newcommand{\Rnuleenplus}{\ensuremath{\mathbb{R}^+\setminus\{0,1\}}}
\newcommand{\N}{\ensuremath{\mathbb{N}}}
\newcommand{\Nnul}{\ensuremath{\mathbb{N}_0}}
\newcommand{\Z}{\ensuremath{\mathbb{Z}}}
\newcommand{\Znul}{\ensuremath{\mathbb{Z}_0}}
\newcommand{\Zplus}{\ensuremath{\mathbb{Z}^+}}
\newcommand{\Zmin}{\ensuremath{\mathbb{Z}^-}}
\newcommand{\Znulplus}{\ensuremath{\mathbb{Z}_0^+}}
\newcommand{\Znulmin}{\ensuremath{\mathbb{Z}_0^-}}
\newcommand{\C}{\ensuremath{\mathbb{C}}}
\newcommand{\Cnul}{\ensuremath{\mathbb{C}_0}}
\newcommand{\Cplus}{\ensuremath{\mathbb{C}^+}}
\newcommand{\Cmin}{\ensuremath{\mathbb{C}^-}}
\newcommand{\Cnulplus}{\ensuremath{\mathbb{C}_0^+}}
\newcommand{\Cnulmin}{\ensuremath{\mathbb{C}_0^-}}
\newcommand{\Q}{\ensuremath{\mathbb{Q}}}
\newcommand{\Qnul}{\ensuremath{\mathbb{Q}_0}}
\newcommand{\Qplus}{\ensuremath{\mathbb{Q}^+}}
\newcommand{\Qmin}{\ensuremath{\mathbb{Q}^-}}
\newcommand{\Qnulplus}{\ensuremath{\mathbb{Q}_0^+}}
\newcommand{\Qnulmin}{\ensuremath{\mathbb{Q}_0^-}}

\newcommand{\perdef}{\overset{\mathrm{def}}{=}}
\newcommand{\pernot}{\overset{\mathrm{notatie}}{=}}
\newcommand\perinderdaad{\overset{!}{=}}     % voorlopig gebruikt in limietenrekenregels
\newcommand\perhaps{\overset{?}{=}}          % voorlopig gebruikt in limietenrekenregels

\newcommand{\degree}{^\circ}


\DeclareMathOperator{\dom}{dom}     % domein
\DeclareMathOperator{\codom}{codom} % codomein
\DeclareMathOperator{\bld}{bld}     % beeld
\DeclareMathOperator{\graf}{graf}   % grafiek
\DeclareMathOperator{\rico}{rico}   % richtingcoëfficient
\DeclareMathOperator{\co}{co}       % coordinaat
\DeclareMathOperator{\gr}{gr}       % graad

\newcommand{\func}[5]{\ensuremath{#1: #2 \rightarrow #3: #4 \mapsto #5}} % Easy to write a function


% Operators
\DeclareMathOperator{\bgsin}{bgsin}
\DeclareMathOperator{\bgcos}{bgcos}
\DeclareMathOperator{\bgtan}{bgtan}
\DeclareMathOperator{\bgcot}{bgcot}
\DeclareMathOperator{\bgsinh}{bgsinh}
\DeclareMathOperator{\bgcosh}{bgcosh}
\DeclareMathOperator{\bgtanh}{bgtanh}
\DeclareMathOperator{\bgcoth}{bgcoth}

% Oude \Bgsin etc deprecated: gebruik \bgsin, en herdefinieer dat als je Bgsin wil!
%\DeclareMathOperator{\cosec}{cosec}    % not used? gebruik \csc en herdefinieer

% operatoren voor differentialen: to be verified; 1/2020: inconsequent gebruik bij afgeleiden/integralen
\renewcommand{\d}{\mathrm{d}}
\newcommand{\dx}{\d x}
\newcommand{\dd}[1]{\frac{\mathrm{d}}{\mathrm{d}#1}}
\newcommand{\ddx}{\dd{x}}

% om in voorbeelden/oefeningen de notatie voor afgeleiden te kunnen kiezen
% Usage: \afg{(2\sin(x))}  (en wordt d/dx, of accent, of D )
%\newcommand{\afg}[1]{{#1}'}
\newcommand{\afg}[1]{\left(#1\right)'}
%\renewcommand{\afg}[1]{\frac{\mathrm{d}#1}{\mathrm{d}x}}   % include in relevant exercises ...
%\renewcommand{\afg}[1]{D{#1}}

%
% \xmxxx commands: Extra KU Leuven functionaliteit van, boven of naast Ximera
%   ( Conventie 8/2019: xm+nederlandse omschrijving, maar is niet consequent gevolgd, en misschien ook niet erg handig !)
%
% (Met een minimale ximera.cls en preamble.tex zou een bruikbare .pdf moeten kunnen worden gemaakt van eender welke ximera)
%
% Usage: \xmtitle[Mijn korte abstract]{Mijn titel}{Mijn abstract}
% Eerste command na \begin{document}:
%  -> definieert de \title
%  -> definieert de abstract
%  -> doet \maketitle ( dus: print de hoofding als 'chapter' of 'sectie')
% Optionele parameter geeft eenn kort abstract (die met de globale setting \xmshortabstract{} al dan niet kan worden geprint.
% De optionele korte abstract kan worden gebruikt voor pseudo-grappige abtsarts, dus dus globaal al dan niet kunnen worden gebuikt...
% Globale settings:
%  de (optionele) 'korte abstract' wordt enkele getoond als \xmshortabstract is gezet
\providecommand\xmshortabstract{} % default: print (only!) short abstract if present
\newcommand{\xmtitle}[3][]{
	\title{#2}
	\begin{abstract}
		\ifdefined\xmshortabstract
		\ifstrempty{#1}{%
			#3
		}{%
			#1
		}%
		\else
		#3
		\fi
	\end{abstract}
	\maketitle
}

% 
% Kleine grapjes: moeten zonder verder gevolg kunnen worden verwijderd
%
%\newcommand{\xmopje}[1]{{\small#1{\reversemarginpar\marginpar{\Smiley}}}}   % probleem in floats!!
\newtoggle{showxmopje}
\toggletrue{showxmopje}

\newcommand{\xmopje}[1]{%
   \iftoggle{showxmopje}{#1}{}%
}


% -> geef een abstracte-formule-met-rechts-een-concreet-voorbeeld
% VB:  \formulevb{a^2+b^2=c^2}{3^2+4^2=5^2}
%
\ifdefined\HCode
\NewEnviron{xmdiv}[1]{\HCode{\Hnewline<div class="#1">\Hnewline}\BODY{\HCode{\Hnewline</div>\Hnewline}}}
\else
\NewEnviron{xmdiv}[1]{\BODY}
\fi

\providecommand{\formulevb}[2]{
	{\centering

    \begin{xmdiv}{xmformulevb}    % zie css voor online layout !!!
	\begin{tabular}{lcl}
		\important{#1}
		&  &
		Vb: $#2$
		\end{tabular}
	\end{xmdiv}

	}
}

\ifdefined\HCode
\providecommand{\vb}[1]{%
    \HCode{\Hnewline<span class="xmvb">}#1\HCode{</span>\Hnewline}%
}
\else
\providecommand{\vb}[1]{
    \colorbox{blue!10}{#1}
}
\fi

\ifdefined\HCode
\providecommand{\xmcolorbox}[2]{
	\HCode{\Hnewline<div class="xmcolorbox">\Hnewline}#2\HCode{\Hnewline</div>\Hnewline}
}
\else
\providecommand{\xmcolorbox}[2]{
  \cellcolor{#1}#2
}
\fi


\ifdefined\HCode
\providecommand{\xmopmerking}[1]{
 \HCode{\Hnewline<div class="xmopmerking">\Hnewline}#1\HCode{\Hnewline</div>\Hnewline}
}
\else
\providecommand{\xmopmerking}[1]{
	{\footnotesize #1}
}
\fi
% \providecommand{\voorbeeld}[1]{
% 	\colorbox{blue!10}{$#1$}
% }



% Hernoem Proof naar Bewijs, nodig voor HTML versie
\renewcommand*{\proofname}{Bewijs}

% Om opgave van oefening (wordt niet geprint bij oplossingenblad)
% (to be tested test)
\NewEnviron{statement}{\BODY}

% Environment 'oplossing' en 'uitkomst'
% voor resp. volledige 'uitwerking' dan wel 'enkel eindresultaat'
% geimplementeerd via feedback, omdat er in de ximera-server adhoc feedback-code is toegevoegd
%% Niet tonen indien handout
%% Te gebruiken om volledige oplossingen/uitwerkingen van oefeningen te tonen
%% \begin{oplossing}        De optelling is commutatief \end{oplossing}  : verschijnt online enkel 'op vraag'
%% \begin{oplossing}[toon]  De optelling is commutatief \end{oplossing}  : verschijnt steeds onmiddellijk online (bv te gebruiken bij voorbeelden) 

\ifhandout%
    \NewEnviron{oplossing}[1][onzichtbaar]%
    {%
    \ifthenelse{\equal{\detokenize{#1}}{\detokenize{toon}}}
    {
    \def\PH@Command{#1}% Use PH@Command to hold the content and be a target for "\expandafter" to expand once.

    \begin{trivlist}% Begin the trivlist to use formating of the "Feedback" label.
    \item[\hskip \labelsep\small\slshape\bfseries Oplossing% Format the "Feedback" label. Don't forget the space.
    %(\texttt{\detokenize\expandafter{\PH@Command}}):% Format (and detokenize) the condition for feedback to trigger
    \hspace{2ex}]\small%\slshape% Insert some space before the actual feedback given.
    \BODY
    \end{trivlist}
    }
    {  % \begin{feedback}[solution]   \BODY     \end{feedback}  }
    }
    }    
\else
% ONLY for HTML; xmoplossing is styled with css, and is not, and need not be a LaTeX environment
% THUS: it does NOT use feedback anymore ...
%    \NewEnviron{oplossing}{\begin{expandable}{xmoplossing}{\nlen{Toon uitwerking}{Show solution}}{\BODY}\end{expandable}}
    \newenvironment{oplossing}[1][onzichtbaar]
   {%
       \begin{expandable}{xmoplossing}{}
   }
   {%
   	   \end{expandable}
   } 
%     \newenvironment{oplossing}[1][onzichtbaar]
%    {%
%        \begin{feedback}[solution]   	
%    }
%    {%
%    	   \end{feedback}
%    } 
\fi

\ifhandout%
    \NewEnviron{uitkomst}[1][onzichtbaar]%
    {%
    \ifthenelse{\equal{\detokenize{#1}}{\detokenize{toon}}}
    {
    \def\PH@Command{#1}% Use PH@Command to hold the content and be a target for "\expandafter" to expand once.

    \begin{trivlist}% Begin the trivlist to use formating of the "Feedback" label.
    \item[\hskip \labelsep\small\slshape\bfseries Uitkomst:% Format the "Feedback" label. Don't forget the space.
    %(\texttt{\detokenize\expandafter{\PH@Command}}):% Format (and detokenize) the condition for feedback to trigger
    \hspace{2ex}]\small%\slshape% Insert some space before the actual feedback given.
    \BODY
    \end{trivlist}
    }
    {  % \begin{feedback}[solution]   \BODY     \end{feedback}  }
    }
    }    
\else
\ifdefined\HCode
   \newenvironment{uitkomst}[1][onzichtbaar]
    {%
        \begin{expandable}{xmuitkomst}{}%
    }
    {%
    	\end{expandable}%
    } 
\else
  % Do NOT print 'uitkomst' in non-handout
  %  (presumably, there is also an 'oplossing' ??)
  \newenvironment{uitkomst}[1][onzichtbaar]{}{}
\fi
\fi

%
% Uitweidingen zijn extra's die niet redelijkerwijze tot de leerstof behoren
% Uitbreidingen zijn extra's die wel redelijkerwijze tot de leerstof van bv meer geavanceerde versies kunnen behoren (B-programma/Wiskundestudenten/...?)
% Nog niet voorzien: design voor verschillende versies (A/B programma, BIO, voorkennis/ ...)
% Voor 'uitweidingen' is er een environment die online per default is ingeklapt, en in pdf al dan niet kan worden geincluded  (via \xmnouitweiding) 
%
% in een xourse, per default GEEN uitweidingen, tenzij \xmuitweiding expliciet ergens is gezet ...
\ifdefined\isXourse
   \ifdefined\xmuitweiding
   \else
       \def\xmnouitweiding{true}
   \fi
\fi

\ifdefined\xmnouitweiding
\newcounter{xmuitweiding}  % anders error undefined ...  
\excludecomment{xmuitweiding}
\else
\newtheoremstyle{dotless}{}{}{}{}{}{}{ }{}
\theoremstyle{dotless}
\newtheorem*{xmuitweidingnofrills}{}   % nofrills = no accordion; gebruikt dus de dotless theoremstyle!

\newcounter{xmuitweiding}
\newenvironment{xmuitweiding}[1][ ]%
{% 
	\refstepcounter{xmuitweiding}%
    \begin{expandable}{xmuitweiding}{\nlentext{Uitweiding \arabic{xmuitweiding}: #1}{Digression \arabic{xmuitweiding}: #1}}%
	\begin{xmuitweidingnofrills}%
}
{%
    \end{xmuitweidingnofrills}%
    \end{expandable}%
}   
% \newenvironment{xmuitweiding}[1][ ]%
% {% 
% 	\refstepcounter{xmuitweiding}
% 	\begin{accordion}\begin{accordion-item}[Uitweiding \arabic{xmuitweiding}: #1]%
% 			\begin{xmuitweidingnofrills}%
% 			}
% 			{\end{xmuitweidingnofrills}\end{accordion-item}\end{accordion}}   
\fi


\newenvironment{xmexpandable}[1][]{
	\begin{accordion}\begin{accordion-item}[#1]%
		}{\end{accordion-item}\end{accordion}}


% Command that gives a selection box online, but just prints the right answer in pdf
\newcommand{\xmonlineChoice}[1]{\pdfOnly{\wordchoicegiventrue}\wordChoice{#1}\pdfOnly{\wordchoicegivenfalse}}   % deprecated, gebruik onlineChoice ...
\newcommand{\onlineChoice}[1]{\pdfOnly{\wordchoicegiventrue}\wordChoice{#1}\pdfOnly{\wordchoicegivenfalse}}

\newcommand{\TJa}{\nlentext{ Ja }{ Yes }}
\newcommand{\TNee}{\nlentext{ Nee }{ No }}
\newcommand{\TJuist}{\nlentext{ Juist }{ True }}
\newcommand{\TFout}{\nlentext{ Fout }{ False }}

\newcommand{\choiceTrue }{{\renewcommand{\choiceminimumhorizontalsize}{4em}\wordChoice{\choice[correct]{\TJuist}\choice{\TFout}}}}
\newcommand{\choiceFalse}{{\renewcommand{\choiceminimumhorizontalsize}{4em}\wordChoice{\choice{\TJuist}\choice[correct]{\TFout}}}}

\newcommand{\choiceYes}{{\renewcommand{\choiceminimumhorizontalsize}{3em}\wordChoice{\choice[correct]{\TJa}\choice{\TNee}}}}
\newcommand{\choiceNo }{{\renewcommand{\choiceminimumhorizontalsize}{3em}\wordChoice{\choice{\TJa}\choice[correct]{\TNee}}}}

% Optional nicer formatting for wordChoice in PDF

\let\inlinechoiceorig\inlinechoice

%\makeatletter
%\providecommand{\choiceminimumverticalsize}{\vphantom{$\frac{\sqrt{2}}{2}$}}   % minimum height of boxes (cfr infra)
\providecommand{\choiceminimumverticalsize}{\vphantom{$\tfrac{2}{2}$}}   % minimum height of boxes (cfr infra)
\providecommand{\choiceminimumhorizontalsize}{1em}   % minimum width of boxes (cfr infra)

\newcommand{\inlinechoicesquares}[2][]{%
		\setkeys{choice}{#1}%
		\ifthenelse{\boolean{\choice@correct}}%
		{%
            \ifhandout%
               \fbox{\choiceminimumverticalsize #2}\allowbreak\ignorespaces%
            \else%
               \fcolorbox{blue}{blue!20}{\choiceminimumverticalsize #2}\allowbreak\ignorespaces\setkeys{choice}{correct=false}\ignorespaces%
            \fi%
		}%
		{% else
			\fbox{\choiceminimumverticalsize #2}\allowbreak\ignorespaces%  HACK: wat kleiner, zodat fits on line ... 	
		}%
}

\newcommand{\inlinechoicesquareX}[2][]{%
		\setkeys{choice}{#1}%
		\ifthenelse{\boolean{\choice@correct}}%
		{%
            \ifhandout%
               \framebox[\ifdim\choiceminimumhorizontalsize<\width\width\else\choiceminimumhorizontalsize\fi]{\choiceminimumverticalsize\ #2\ }\allowbreak\ignorespaces\setkeys{choice}{correct=false}\ignorespaces%
            \else%
               \fcolorbox{blue}{blue!20}{\makebox[\ifdim\choiceminimumhorizontalsize<\width\width\else\choiceminimumhorizontalsize\fi]{\choiceminimumverticalsize #2}}\allowbreak\ignorespaces\setkeys{choice}{correct=false}\ignorespaces%
            \fi%
		}%
		{% else
        \ifhandout%
			\framebox[\ifdim\choiceminimumhorizontalsize<\width\width\else\choiceminimumhorizontalsize\fi]{\choiceminimumverticalsize\ #2\ }\allowbreak\ignorespaces%  HACK: wat kleiner, zodat fits on line ... 	
        \fi
		}%
}


\newcommand{\inlinechoicepuntjes}[2][]{%
		\setkeys{choice}{#1}%
		\ifthenelse{\boolean{\choice@correct}}%
		{%
            \ifhandout%
               \dots\ldots\ignorespaces\setkeys{choice}{correct=false}\ignorespaces
            \else%
               \fcolorbox{blue}{blue!20}{\choiceminimumverticalsize #2}\allowbreak\ignorespaces\setkeys{choice}{correct=false}\ignorespaces%
            \fi%
		}%
		{% else
			%\fbox{\choiceminimumverticalsize #2}\allowbreak\ignorespaces%  HACK: wat kleiner, zodat fits on line ... 	
		}%
}

% print niets, maar definieer globale variable \myanswer
%  (gebruikt om oplossingsbladen te printen) 
\newcommand{\inlinechoicedefanswer}[2][]{%
		\setkeys{choice}{#1}%
		\ifthenelse{\boolean{\choice@correct}}%
		{%
               \gdef\myanswer{#2}\setkeys{choice}{correct=false}

		}%
		{% else
			%\fbox{\choiceminimumverticalsize #2}\allowbreak\ignorespaces%  HACK: wat kleiner, zodat fits on line ... 	
		}%
}



%\makeatother

\newcommand{\setchoicedefanswer}{
\ifdefined\HCode
\else
%    \renewenvironment{multipleChoice@}[1][]{}{} % remove trailing ')'
    \let\inlinechoice\inlinechoicedefanswer
\fi
}

\newcommand{\setchoicepuntjes}{
\ifdefined\HCode
\else
    \renewenvironment{multipleChoice@}[1][]{}{} % remove trailing ')'
    \let\inlinechoice\inlinechoicepuntjes
\fi
}
\newcommand{\setchoicesquares}{
\ifdefined\HCode
\else
    \renewenvironment{multipleChoice@}[1][]{}{} % remove trailing ')'
    \let\inlinechoice\inlinechoicesquares
\fi
}
%
\newcommand{\setchoicesquareX}{
\ifdefined\HCode
\else
    \renewenvironment{multipleChoice@}[1][]{}{} % remove trailing ')'
    \let\inlinechoice\inlinechoicesquareX
\fi
}
%
\newcommand{\setchoicelist}{
\ifdefined\HCode
\else
    \renewenvironment{multipleChoice@}[1][]{}{)}% re-add trailing ')'
    \let\inlinechoice\inlinechoiceorig
\fi
}

\setchoicesquareX  % by default list-of-squares with onlineChoice in PDF

% Omdat multicols niet werkt in html: enkel in pdf  (in html zijn langere pagina's misschien ook minder storend)
\newenvironment{xmmulticols}[1][2]{
 \pdfOnly{\begin{multicols}{#1}}%
}{ \pdfOnly{\end{multicols}}}

%
% Te gebruiken in plaats van \section\subsection
%  (in een printstyle kan dan het level worden aangepast
%    naargelang \chapter vs \section style )
% 3/2021: DO NOT USE \xmsubsection !
\newcommand\xmsection\subsection
\newcommand\xmsubsection\subsubsection

% Aanpassen printversie
%  (hier gedefinieerd, zodat ze in xourse kunnen worden gezet/overschreven)
\providebool{parttoc}
\providebool{printpartfrontpage}
\providebool{printactivitytitle}
\providebool{printactivityqrcode}
\providebool{printactivityurl}
\providebool{printcontinuouspagenumbers}
\providebool{numberactivitiesbysubpart}
\providebool{addtitlenumber}
\providebool{addsectiontitlenumber}
\addtitlenumbertrue
\addsectiontitlenumbertrue

% The following three commands are hardcoded in xake, you can't create other commands like these, without adding them to xake as well
%  ( gebruikt in xourses om juiste soort titelpagina te krijgen voor verschillende ximera's )
\newcommand{\activitychapter}[2][]{
    {    
    \ifstrequal{#1}{notnumbered}{
        \addtitlenumberfalse
    }{}
    \typeout{ACTIVITYCHAPTER #2}   % logging
	\chapterstyle
	\activity{#2}
    }
}
\newcommand{\activitysection}[2][]{
    {
    \ifstrequal{#1}{notnumbered}{
        \addsectiontitlenumberfalse
    }{}
	\typeout{ACTIVITYSECTION #2}   % logging
	\sectionstyle
	\activity{#2}
    }
}
% Practices worden als activity getoond om de grote blokken te krijgen online
\newcommand{\practicesection}[2][]{
    {
    \ifstrequal{#1}{notnumbered}{
        \addsectiontitlenumberfalse
    }{}
    \typeout{PRACTICESECTION #2}   % logging
	\sectionstyle
	\activity{#2}
    }
}
\newcommand{\activitychapterlink}[3][]{
    {
    \ifstrequal{#1}{notnumbered}{
        \addtitlenumberfalse
    }{}
    \typeout{ACTIVITYCHAPTERLINK #3}   % logging
	\chapterstyle
	\activitylink[#1]{#2}{#3}
    }
}

\newcommand{\activitysectionlink}[3][]{
    {
    \ifstrequal{#1}{notnumbered}{
        \addsectiontitlenumberfalse
    }{}
    \typeout{ACTIVITYSECTIONLINK #3}   % logging
	\sectionstyle
	\activitylink[#1]{#2}{#3}
    }
}


% Commando om de printstyle toe te voegen in ximera's. Zorgt ervoor dat er geen problemen zijn als je de xourses compileert
% hack om onhandige relative paden in TeX te omzeilen
% should work both in xourse and ximera (pre-112022 only in ximera; thus obsoletes adhoc setup in xourses)
% loads global.sty if present (cfr global.css for online settings!)
% use global.sty to overwrite settings in printstyle.sty ...
\newcommand{\addPrintStyle}[1]{
\iftikzexport\else   % only in PDF
  \makeatletter
  \ifx\@onlypreamble\@notprerr\else   % ONLY if in tex-preamble   (and e.g. not when included from xourse)
    \typeout{Loading printstyle}   % logging
    \usepackage{#1/printstyle} % mag enkel geinclude worden als je die apart compileert
    \IfFileExists{#1/global.sty}{
        \typeout{Loading printstyle-folder #1/global.sty}   % logging
        \usepackage{#1/global}
        }{
        \typeout{Info: No extra #1/global.sty}   % logging
    }   % load global.sty if present
    \IfFileExists{global.sty}{
        \typeout{Loading local-folder global.sty (or TEXINPUTPATH..)}   % logging
        \usepackage{global}
    }{
        \typeout{Info: No folder/global.sty}   % logging
    }   % load global.sty if present
    \IfFileExists{\currfilebase.sty}
    {
        \typeout{Loading \currfilebase.sty}
        \input{\currfilebase.sty}
    }{
        \typeout{Info: No local \currfilebase.sty}
    }
    \fi
  \makeatother
\fi
}

%
%  
% references: Ximera heeft adhoc logica	 om online labels te doen werken over verschillende files heen
% met \hyperref kan de getoonde tekst toch worden opgegeven, in plaats van af te hangen van de label-text
\ifdefined\HCode
% Link to standard \labels, but give your own description
% Usage:  Volg \hyperref[my_very_verbose_label]{deze link} voor wat tijdverlies
%   (01/2020: Ximera-server aangepast om bij class reference-keeptext de link-text NIET te vervangen door de label-text !!!) 
\renewcommand{\hyperref}[2][]{\HCode{<a class="reference reference-keeptext" href="\##1">}#2\HCode{</a>}}
%
%  Link to specific targets  (not tested ?)
\renewcommand{\hypertarget}[1]{\HCode{<a class="ximera-label" id="#1"></a>}}
\renewcommand{\hyperlink}[2]{\HCode{<a class="reference reference-keeptext" href="\##1">}#2\HCode{</a>}}
\fi

% Mmm, quid English ... (for keyword #1 !) ?
\newcommand{\wikilink}[2]{
    \hyperlink{https://nl.wikipedia.org/wiki/#1}{#2}
    \pdfOnly{\footnote{See \url{https://nl.wikipedia.org/wiki/#1}}
    }
}

\renewcommand{\figurename}{Figuur}
\renewcommand{\tablename}{Tabel}

%
% Gedoe om verschillende versies van xourse/ximera te maken afhankelijk van settings
%
% default: versie met antwoorden
% handout: versie voor de studenten, zonder antwoorden/oplossingen
% full: met alles erop en eraan, dus geschikt voor auteurs en/of lesgevers  (bevat in de pdf ook de 'online-only' stukken!)
%
%
% verder kunnen ook opties/variabele worden gezet voor hints/auteurs/uitweidingen/ etc
%
% 'Full' versie
\newtoggle{showonline}
\ifdefined\HCode   % zet default showOnline
    \toggletrue{showonline} 
\else
    \togglefalse{showonline}
\fi

% Full versie   % deprecated: see infra
\newcommand{\printFull}{
    \hintstrue
    \handoutfalse
    \toggletrue{showonline} 
}

\ifdefined\shouldPrintFull   % deprecated: see infra
    \printFull
\fi



% Overschrijf onlineOnly  (zoals gedefinieerd in ximera.cls)
\ifhandout   % in handout: gebruik de oorspronkelijke ximera.cls implementatie  (is dit wel nodig/nuttig?)
\else
    \iftoggle{showonline}{%
        \ifdefined\HCode
          \RenewEnviron{onlineOnly}{\bgroup\BODY\egroup}   % showOnline, en we zijn  online, dus toon de tekst
        \else
          \RenewEnviron{onlineOnly}{\bgroup\color{red!50!black}\BODY\egroup}   % showOnline, maar we zijn toch niet online: kleur de tekst rood 
        \fi
    }{%
      \RenewEnviron{onlineOnly}{}  % geen showOnline
    }
\fi

% hack om na hoofding van definition/proposition/... als dan niet op een nieuwe lijn te starten
% soms is dat goed en mooi, en soms niet; en in HTML is het nu (2/2020) anders dan in pdf
% vandaar suggestie om 
%     \begin{definition}[Nieuw concept] \nl
% te gebruiken als je zeker een newline wil na de hoofdig en titel
% (in het bijzonder itemize zonder \nl is 'lelijk' ...)
\ifdefined\HCode
\newcommand{\nl}{}
\else
\newcommand{\nl}{\ \par} % newline (achter heading van definition etc.)
\fi


% \nl enkel in handoutmode (ihb voor \wordChoice, die dan typisch veeeel langer wordt)
\ifdefined\HCode
\providecommand{\handoutnl}{}
\else
\providecommand{\handoutnl}{%
\ifhandout%
  \nl%
\fi%
}
\fi

% Could potentially replace \pdfOnline/\begin{onlineOnly} : 
% Usage= \ifonline{Hallo surfer}{Hallo PDFlezer}
\providecommand{\ifonline}[2]%
{
\begin{onlineOnly}#1\end{onlineOnly}%
\pdfOnly{#2}
}%


%
% Maak optionele 'basic' en 'extended' versies van een activity
%  met environment basicOnly, basicSkip en extendedOnly
%
%  (
%   Dit werkt ENKEL in de PDF; de online versies tonen (minstens voorklopig) steeds 
%   het default geval met printbasicversion en printextendversion beide FALSE
%  )
%
\providebool{printbasicversion}
\providebool{printextendedversion}   % not properly implemented
\providebool{printfullversion}       % presumably print everything (debug/auteur)
%
% only set these in xourses, and BEFORE loading this preamble
%
%\newif\ifshowbasic     \showbasictrue        % use this line in xourse to show 'basic' sections
%\newif\ifshowextended  \showextendedtrue     % use this line in xourse to show 'extended' sections
%
%
%\ifbool{showbasic}
%      { \NewEnviron{basicOnly}{\BODY} }    % if yes: just print contents
%      { \NewEnviron{basicOnly}{}      }    % if no:  completely ignore contents
%
%\ifbool{showbasic}
%      { \NewEnviron{basicSkip}{}      }
%      { \NewEnviron{basicSkip}{\BODY} }
%

\ifbool{printextendedversion}
      { \NewEnviron{extendedOnly}{\BODY} }
      { \NewEnviron{extendedOnly}{}      }
      


\ifdefined\HCode    % in html: always print
      {\newenvironment*{basicOnly}{}{}}    % if yes: just print contents
      {\newenvironment*{basicSkip}{}{}}    % if yes: just print contents
\else

\ifbool{printbasicversion}
      {\newenvironment*{basicOnly}{}{}}    % if yes: just print contents
      {\NewEnviron{basicOnly}{}      }    % if no:  completely ignore contents

\ifbool{printbasicversion}
      {\NewEnviron{basicSkip}{}      }
      {\newenvironment*{basicSkip}{}{}}

\fi

\usepackage{float}
\usepackage[rightbars,color]{changebar}

% Full versie
\ifbool{printfullversion}{
    \hintstrue
    \handoutfalse
    \toggletrue{showonline}
    \printbasicversionfalse
    \cbcolor{red}
    \renewenvironment*{basicOnly}{\cbstart}{\cbend}
    \renewenvironment*{basicSkip}{\cbstart}{\cbend}
    \def\xmtoonprintopties{FULL}   % will be printed in footer
}
{}
      
%
% Evalueer \ifhints IN de environment
%  
%
%\RenewEnviron{hint}
%{
%\ifhandout
%\ifhints\else\setbox0\vbox\fi%everything in een emty box
%\bgroup 
%\stepcounter{hintLevel}
%\BODY
%\egroup\ignorespacesafterend
%\addtocounter{hintLevel}{-1}
%\else
%\ifhints
%\begin{trivlist}\item[\hskip \labelsep\small\slshape\bfseries Hint:\hspace{2ex}]
%\small\slshape
%\stepcounter{hintLevel}
%\BODY
%\end{trivlist}
%\addtocounter{hintLevel}{-1}
%\fi
%\fi
%}

% Onafhankelijk van \ifhandout ...? TO BE VERIFIED
\RenewEnviron{hint}
{
\ifhints
\begin{trivlist}\item[\hskip \labelsep\small\bfseries Hint:\hspace{2ex}]
\small%\slshape
\stepcounter{hintLevel}
\BODY
\end{trivlist}
\addtocounter{hintLevel}{-1}
\else
\iftikzexport   % anders worden de tikz tekeningen in hints niet gegenereerd ?
\setbox0\vbox\bgroup
\stepcounter{hintLevel}
\BODY
\egroup\ignorespacesafterend
\addtocounter{hintLevel}{-1}
\fi % ifhandout
\fi %ifhints
}

%
% \tab sets typewriter-tabs (e.g. to format questions)
% (Has no effect in HTML :-( ))
%
\usepackage{tabto}
\ifdefined\HCode
  \renewcommand{\tab}{\quad}    % otherwise dummy .png's are generated ...?
\fi


% Also redefined in  preamble to get correct styling 
% for tikz images for (\tikzexport)
%

\theoremstyle{definition} % Bold titels
\makeatletter
\let\proposition\relax
\let\c@proposition\relax
\let\endproposition\relax
\makeatother
\newtheorem{proposition}{Eigenschap}


%\instructornotesfalse

% logic with \ifhandoutin ximera.cls unclear;so overwrite ...
\makeatletter
\@ifundefined{ifinstructornotes}{%
  \newif\ifinstructornotes
  \instructornotesfalse
  \newenvironment{instructorNotes}{}{}
}{}
\makeatother
\ifinstructornotes
\else
\renewenvironment{instructorNotes}%
{%
    \setbox0\vbox\bgroup
}
{%
    \egroup
}
\fi

% \RedeclareMathOperator
% from https://tex.stackexchange.com/questions/175251/how-to-redefine-a-command-using-declaremathoperator
\makeatletter
\newcommand\RedeclareMathOperator{%
    \@ifstar{\def\rmo@s{m}\rmo@redeclare}{\def\rmo@s{o}\rmo@redeclare}%
}
% this is taken from \renew@command
\newcommand\rmo@redeclare[2]{%
    \begingroup \escapechar\m@ne\xdef\@gtempa{{\string#1}}\endgroup
    \expandafter\@ifundefined\@gtempa
    {\@latex@error{\noexpand#1undefined}\@ehc}%
    \relax
    \expandafter\rmo@declmathop\rmo@s{#1}{#2}}
% This is just \@declmathop without \@ifdefinable
\newcommand\rmo@declmathop[3]{%
    \DeclareRobustCommand{#2}{\qopname\newmcodes@#1{#3}}%
}
\@onlypreamble\RedeclareMathOperator
\makeatother


%
% Engelse vertaling, vooral in mathmode
%
% 1. Algemene procedure
%
\ifdefined\isEn
 \newcommand{\nlen}[2]{#2}
 \newcommand{\nlentext}[2]{\text{#2}}
 \newcommand{\nlentextbf}[2]{\textbf{#2}}
\else
 \newcommand{\nlen}[2]{#1}
 \newcommand{\nlentext}[2]{\text{#1}}
 \newcommand{\nlentextbf}[2]{\textbf{#1}}
\fi

%
% 2. Lijst van erg veel gebruikte uitdrukkingen
%

% Ja/Nee/Fout/Juits etc
%\newcommand{\TJa}{\nlentext{ Ja }{ and }}
%\newcommand{\TNee}{\nlentext{ Nee }{ No }}
%\newcommand{\TJuist}{\nlentext{ Juist }{ Correct }
%\newcommand{\TFout}{\nlentext{ Fout }{ Wrong }
\newcommand{\TWaar}{\nlentext{ Waar }{ True }}
\newcommand{\TOnwaar}{\nlentext{ Vals }{ False }}
% Korte bindwoorden en, of, dus, ...
\newcommand{\Ten}{\nlentext{ en }{ and }}
\newcommand{\Tof}{\nlentext{ of }{ or }}
\newcommand{\Tdus}{\nlentext{ dus }{ so }}
\newcommand{\Tendus}{\nlentext{ en dus }{ and thus }}
\newcommand{\Tvooralle}{\nlentext{ voor alle }{ for all }}
\newcommand{\Took}{\nlentext{ ook }{ also }}
\newcommand{\Tals}{\nlentext{ als }{ when }} %of if?
\newcommand{\Twant}{\nlentext{ want }{ as }}
\newcommand{\Tmaal}{\nlentext{ maal }{ times }}
\newcommand{\Toptellen}{\nlentext{ optellen }{ add }}
\newcommand{\Tde}{\nlentext{ de }{ the }}
\newcommand{\Thet}{\nlentext{ het }{ the }}
\newcommand{\Tis}{\nlentext{ is }{ is }} %zodat is in text staat in mathmode (geen italics)
\newcommand{\Tmet}{\nlentext{ met }{ where }} % in situaties e.g met p < n --> where p < n
\newcommand{\Tnooit}{\nlentext{ nooit }{ never }}
\newcommand{\Tmaar}{\nlentext{ maar }{ but }}
\newcommand{\Tniet}{\nlentext{ niet }{ not }}
\newcommand{\Tuit}{\nlentext{ uit }{ from }}
\newcommand{\Ttov}{\nlentext{ t.o.v. }{ w.r.t. }}
\newcommand{\Tzodat}{\nlentext{ zodat }{ such that }}
\newcommand{\Tdeth}{\nlentext{de }{th }}
\newcommand{\Tomdat}{\nlentext{omdat }{because }} 


%
% Overschrijf addhoc commando's
%
\ifdefined\isEn
\renewcommand{\pernot}{\overset{\mathrm{notation}}{=}}
\RedeclareMathOperator{\bld}{im}     % beeld
\RedeclareMathOperator{\graf}{graph}   % grafiek
\RedeclareMathOperator{\rico}{slope}   % richtingcoëfficient
\RedeclareMathOperator{\co}{co}       % coordinaat
\RedeclareMathOperator{\gr}{deg}       % graad

% Operators
\RedeclareMathOperator{\bgsin}{arcsin}
\RedeclareMathOperator{\bgcos}{arccos}
\RedeclareMathOperator{\bgtan}{arctan}
\RedeclareMathOperator{\bgcot}{arccot}
\RedeclareMathOperator{\bgsinh}{arcsinh}
\RedeclareMathOperator{\bgcosh}{arccosh}
\RedeclareMathOperator{\bgtanh}{arctanh}
\RedeclareMathOperator{\bgcoth}{arccoth}

\fi


% HACK: use 'oplossing' for 'explanation' ...
\let\explanation\relax
\let\endexplanation\relax
% \newenvironment{explanation}{\begin{oplossing}}{\end{oplossing}}
\newcounter{explanation}

\ifhandout%
    \NewEnviron{explanation}[1][toon]%
    {%
    \RenewEnviron{verbatim}{ \red{VERBATIM CONTENT MISSING IN THIS PDF}} %% \expandafter\verb|\BODY|}

    \ifthenelse{\equal{\detokenize{#1}}{\detokenize{toon}}}
    {
    \def\PH@Command{#1}% Use PH@Command to hold the content and be a target for "\expandafter" to expand once.

    \begin{trivlist}% Begin the trivlist to use formating of the "Feedback" label.
    \item[\hskip \labelsep\small\slshape\bfseries Explanation:% Format the "Feedback" label. Don't forget the space.
    %(\texttt{\detokenize\expandafter{\PH@Command}}):% Format (and detokenize) the condition for feedback to trigger
    \hspace{2ex}]\small%\slshape% Insert some space before the actual feedback given.
    \BODY
    \end{trivlist}
    }
    {  % \begin{feedback}[solution]   \BODY     \end{feedback}  }
    }
    }    
\else
% ONLY for HTML; xmoplossing is styled with css, and is not, and need not be a LaTeX environment
% THUS: it does NOT use feedback anymore ...
%    \NewEnviron{oplossing}{\begin{expandable}{xmoplossing}{\nlen{Toon uitwerking}{Show solution}}{\BODY}\end{expandable}}
    \newenvironment{explanation}[1][toon]
   {%
       \begin{expandable}{xmoplossing}{}
   }
   {%
   	   \end{expandable}
   } 
\fi
 \title{Orthogonal Complements and Decompositions} \license{CC BY-NC-SA 4.0}

\begin{document}

\begin{abstract}
\end{abstract}
\maketitle

\section*{Orthogonal Complements and Decompositions}
\subsection*{Orthogonal Complements}
We will now consider the set of vectors that are orthogonal to every vector in a given subspace.  As a quick example, consider the $xy$-plane in $\RR^3$.  Clearly, every scalar multiple of the standard unit vector $\vec{k}$ in $\RR^3$ is orthogonal to every vector in the $xy$-plane.  We say that the set $\{c\vec{k} \mid c \in\RR\}$ is an orthogonal complement of $\{a\vec{i}+b\vec{j} \mid a, b \in\RR\}$.

  \begin{definition}[Orthogonal Complement of a Subspace of $\RR^n$]\label{def:023776}
If $W$ is a subspace of $\RR^n$, define the \dfn{orthogonal complement} $W^\perp$ of $W$ (pronounced ``$W$-perp'') by
\begin{equation*}
W^\perp = \{\vec{x} \mbox{ in } \RR^n \mid \vec{x} \dotp \vec{y} = 0 \mbox{ for all } \vec{y} \mbox{ in } W\}
\end{equation*}
\end{definition}

\begin{center}
\tdplotsetmaincoords{70}{130}
	\begin{tikzpicture}[scale=1]
\filldraw[blue, opacity=0.3] (2,0,2)--(2,0,-2)--(-2,0,-2)--(-2,0,2)--cycle;
    \draw[->,line width=0.4mm, -stealth, blue](0,-2,0)--(0,3,0) ;%normal to grey
 \node[label={below right:$W$}] at (0.8,0.1,-2) {};
    \node[label={below right:$W^\perp$}] at (0,3,0) {};   
   \node[fill,circle,inner sep=1.5pt] at (0,0,0) {};
     \end{tikzpicture}
\end{center}

The following theorem collects some
useful properties of the orthogonal complement; the proof of \ref{th:023783a} and \ref{th:023783b}
 is left as Practice Problem \ref{prob:8_1_6}.

\begin{theorem}\label{th:023783}
Let $W$ be a subspace of $\RR^n$.
\begin{enumerate}
\item\label{th:023783a} $W^\perp$ is a subspace of $\RR^n$.

\item\label{th:023783b} $\{\vec{0}\}^\perp = \RR^n$ and $(\RR^n)^\perp = \{\vec{0}\}$.

\item\label{th:023783c} If $W = \mbox{span}\left(\vec{x}_{1}, \vec{x}_{2}, \dots, \vec{x}_{k}\right)$ then $W^\perp = \{\vec{x} \mbox{ in } \RR^n \mid \vec{x} \dotp \vec{x}_{i} = 0 \mbox{ for } i = 1, 2, \dots, k\}$.

\end{enumerate}
\end{theorem}

\begin{proof}[Proof of Part~\ref{th:023783c}:]
We must show that $W^\perp = \{\vec{x} \mid \vec{x} \dotp \vec{x}_{i} = 0 \mbox{ for each } i\}$.  To show that two sets $\mathcal{S}$ and $\mathcal{T}$ are equal, a common strategy is to show that $\mathcal{S} \subseteq \mathcal{T}$ and also $\mathcal{T} \subseteq \mathcal{S}$.  We employ this strategy now.

Clearly, if $\vec{x}$ is in $W^\perp$ then $\vec{x} \dotp \vec{x}_{i} = 0$ for all $i$ because each $\vec{x}_{i}$ is in $W$. This shows $W^\perp \subseteq \{\vec{x} \mid \vec{x} \dotp \vec{x}_{i} = 0 \mbox{ for each } i\}$. For the reverse inclusion, suppose that $\vec{x} \dotp \vec{x}_{i} = 0$ for all $i$; we need to show that $\vec{x}$ is in $W^\perp$.  We need to show $\vec{x} \dotp \vec{y} = 0$ for each $\vec{y}$ in $W$. We can write $\vec{y} = c_{1}\vec{x}_{1} + c_{2}\vec{x}_{2} + \dots  + c_{k}\vec{x}_{k}$, where each $c_{i}$ is in $\RR$. Then
\begin{equation*}
\vec{x} \dotp \vec{y} = c_{1}(\vec{x} \dotp \vec{x}_{1}) + c_{2}(\vec{x} \dotp \vec{x}_{2})+ \dots +c_{k}(\vec{x} \dotp \vec{x}_{k}) = c_{1}0 + c_{2}0 + \dots + c_{k}0 = 0
\end{equation*}
 as required, and the proof of equality is complete.

\end{proof}

\begin{example}\label{ex:023829}
Find a basis for $W^\perp$ if $W = \mbox{span}\left\{\begin{bmatrix}
  1 \\ -1 \\ 2 \\ 0
  \end{bmatrix},
  \begin{bmatrix}
  1 \\ 0 \\ -2 \\ 3
  \end{bmatrix}\right\}$ in $\RR^4$.

\begin{explanation}
  By Theorem~\ref{th:023783}, $\vec{x} = \begin{bmatrix}
  x \\ y \\ z \\ w
  \end{bmatrix}$
  is in $W^\perp$ if and only if $\vec{x}$ is orthogonal to both 
  $\vec{v}_1 = \begin{bmatrix}
  1 \\ -1 \\ 2 \\ 0
  \end{bmatrix}$ and
  $\vec{v}_2 = \begin{bmatrix}
  1 \\ 0 \\ -2 \\ 3
  \end{bmatrix}$; that is,
  $\vec{x} \dotp \vec{v}_1 = 0$ and $\vec{x} \dotp \vec{v}_2 = 0$, or
\begin{equation*}
\begin{array}{rrrrrrrr}
x & - & y & + & 2z & & & =0\\
x & & & - & 2z & +& 3w & =0
\end{array}
\end{equation*}
Using Gaussian elimination on this system gives $W^\perp = \mbox{span}\left\{\begin{bmatrix}
  2 \\ 4 \\ 1 \\ 0
  \end{bmatrix},
  \begin{bmatrix}
  3 \\ 3 \\ 0 \\ -1
  \end{bmatrix}\right\}$.  You are asked to confirm this in Practice Problem \ref{prob:Uperp} (which serves as a wonderful review of concepts we covered earlier in the course!).  
\end{explanation}
\end{example}

Some of the important subspaces we studied earlier are orthogonal complements of each other.  Recall the following definitions associated with an $m \times n$ matrix $A$.
\begin{enumerate}
    \item Its \dfn{null space}, $\mbox{null}(A) = \{\vec{x}\in \RR^n \mid A\vec{x} = \vec{0}\}$, is a subspace of $\RR^n$.
    \item Its \dfn{row space}, $\mbox{row}(A) = \mbox{span} \{ \mbox{the rows of } A\}$, is a subspace of $\RR^n$.
    \item Its \dfn{column space}, $\mbox{col}(A) = \mbox{span} \{ \mbox{the columns of } A\}$, is a subspace of $\RR^m$.
\end{enumerate}

\begin{exploration}\label{exp:discoverortho}
In the following GeoGebra interactive, you can change the coordinates of the vectors $\vec{v}$ and $\vec{w}$ using the sliders.  (At this stage make sure that $\vec{v}$ and $\vec{w}$ are not collinear.) Then, since matrix $A$ is defined using $\vec{v}$ and $\vec{w}$ as its rows, we have $\mbox{row}(A) = \mbox{span}\{\vec{v},\vec{w}\}$.  
   
\begin{center}
\geogebra{f6eavqxs}{950}{800}
\end{center}

\begin{enumerate}
        \item Follow the prompts in the interactive to visualize $\mbox{row}(A)$ and $\mbox{null}(A)$.  What relationships do you observe between $\mbox{row}(A)$ and $\mbox{null}(A)$? 
        \end{enumerate}

  It is possible to ``break" this interactive (for certain choices of the sliders). If $\vec{v}$ and $\vec{w}$ are scalar multiples of each other, then $\mbox{row}(A)$ is a \wordChoice{\choice{Point}, \choice[correct]{Line},\choice{Plane}}, and the dimension of $\mbox{null}(A)$ is \wordChoice{\choice{1}, \choice[correct]{2},\choice{3}}.  The interactive does not accommodate this situation.  To see what happens when $\vec{v}$ and $\vec{w}$ are scalar multiples of each other, see Practice Problem \ref{prob:brokenInteractive}.
\end{exploration}


\begin{theorem}\label{th:4subspaces}
Let $A$ be an $m \times n$ matrix.  Then we have:
\begin{enumerate}
\item\label{th:4subspacesa} $\mbox{null}(A) = (\mbox{row}(A))^\perp$;
\item\label{th:4subspacesb} $\mbox{null}(A^T) = (\mbox{col}(A))^\perp$.
\end{enumerate}
\end{theorem}

Before proving this theorem, let's examine what it says about a couple of our examples.  In Example \ref{ex:023829}, we solved for the unknown vectors $\vec{x} = \begin{bmatrix}
  x \\ y \\ z \\ w
  \end{bmatrix}$. Notice that this is equivalent to creating a $2 \times 4$ matrix $A$ whose rows are $\vec{v}_1$ and $\vec{v}_2$, and then finding the null space of that matrix $A$.  You can check that a basis for $\mbox{null}\left(\begin{bmatrix}
  1 & -1 & 2 & 0 \\
  1 & 0 & -2 & 3
  \end{bmatrix}\right)$
is given by $\left\{\begin{bmatrix}
  2 \\ 4 \\ 1 \\ 0
  \end{bmatrix},
  \begin{bmatrix}
  3 \\ 3 \\ 0 \\ -1
  \end{bmatrix}\right\}$.

\begin{example}\label{ex:4subspaces}
Let
$$A=\begin{bmatrix}2&-1&1&-4&1\\1&0&3&3&0\\-2&1&-1&5&2\\4&-1&7&2&1\end{bmatrix}$$
Verify each of the statements in Theorem~\ref{th:4subspaces}.

\begin{explanation}
 We compute $\mbox{rref}(A)$ to find a basis for   $\mbox{null}(A)$, $\mbox{row}(A)$, and $\mbox{col}(A)$.  After some work we arrive at:
 $\mbox{null}(A) = \mbox{span}\left\{\begin{bmatrix}-3\\-5\\1\\0\\0\end{bmatrix}, \begin{bmatrix}9\\31\\0\\-3\\1\end{bmatrix}\right\}$ 
 and
 $$\mbox{row}(A)=\mbox{span}\Big(\begin{bmatrix}1&0&3&0&-9\end{bmatrix},
\begin{bmatrix}0&1&5&0&-31\end{bmatrix},
\begin{bmatrix}0&0&0&1&3\end{bmatrix}\Big).$$  (See examples in \link[Subspaces of $\RR^n$ Associated with Matrices]{https://ximera.osu.edu/oerlinalg/LinearAlgebra/VSP-0040/main} for the details.) It is easy to check that each of the basis vectors of $\mbox{null}(A)$ is orthogonal to each of the basis vectors of $\mbox{row}(A)$, demonstrating the first part of Theorem~\ref{th:4subspaces}.  You will be asked to demonstrate the second part of Theorem~\ref{th:4subspaces} for this example in Practice Problem \ref{prob:finishex4subspaces}.
\end{explanation}
\end{example}

We now return to the proof of Theorem \ref{th:4subspaces}.

\begin{proof}[Proof of Theorem~\ref{th:4subspaces}:]
Let $\vec{x}\in\RR^n$.  $\vec{x}\in\left(\mbox{row}(A)\right)^\perp$ if and only if x is orthogonal to every row of $A$.  But this is true if and only if $A\vec{x}=\vec{0}$, which is equivalent to saying $\vec{x}\in\mbox{null}(A)$, which proves \ref{th:4subspacesa}.  To prove \ref{th:4subspacesb}, we simply replace $A$ with $A^T$, and we may apply \ref{th:4subspacesa} since $\mbox{col}(A) = \mbox{row}(A^T)$.
\end{proof}

\subsection*{Orthogonal Decomposition Theorem}

Now that we have defined the orthogonal complement of a subspace, we are ready to state the main theorem of this section.  If you have studied physics or multi-variable calculus, you are familiar with the idea of expressing a vector in as the sum of its tangential and normal components. (If you haven't yet taken those courses, this section will help to prepare you for them!)  The following theorem is a generalization of that idea.

\begin{theorem}[Orthogonal Decomposition Theorem]\label{th:OrthoDecomp}
Let $W$ be a subspace of $\RR^n$ and let $\vec{x} \in \RR^n$.  Then there exist unique vectors $\vec{w} \in W$ and $\vec{w}^\perp \in W^\perp$ such that $\vec{x} = \vec{w} + \vec{w}^\perp$.
\end{theorem}

\begin{proof}
This is an example of an ``existence and uniqueness'' theorem, so there are two things to prove.  If we have an orthogonal basis $\{\vec{f}_{1}, \vec{f}_{2}, \dots, \vec{f}_{m}\}$ for $W$, then it is easy to show that our orthogonal decomposition exists for $\vec{x}$. We let $\vec{w}=\mbox{proj}_W(\vec{x})$, which is clearly in $W$, and we let  $\vec{w}^\perp = \vec{x} - \vec{w}$, and we have $\vec{w} + \vec{w}^\perp = \vec{w} + (\vec{x} - \vec{w}) = \vec{x}$, so we need to see that $\vec{w}^\perp \in W^\perp$.

By Theorem~\ref{th:023783}~\ref{th:023783c}, it suffices to show that $\vec{w}^\perp$ if orthogonal to each of the basis vectors $\vec{f}_i, i=1,\ldots,m$.  We compute for $i=1,\ldots,m$
\begin{align*}
    \vec{f}_i \dotp \vec{w}^\perp 
    &= \vec{f}_i \dotp (\vec{x} - \vec{w}) \\
    &= \vec{f}_i \dotp \vec{x} -  \vec{f}_i \dotp  \left(\frac{\vec{x} \dotp \vec{f}_{1}}{\norm{\vec{f}_{1}}^2}\vec{f}_{1} + \frac{\vec{x} \dotp \vec{f}_{2}}{\norm{\vec{f}_{2}}^2}\vec{f}_{2}+ \dots +\frac{\vec{x} \dotp \vec{f}_{m}}{\norm{\vec{f}_{m}}^2}\vec{f}_{m}\right) \\
    &= \vec{f}_i \dotp \vec{x} - \left(\frac{\vec{x} \dotp \vec{f}_{1}}{\norm{\vec{f}_{1}}^2}\vec{f}_i \dotp\vec{f}_{i} \right) = \vec{f}_i \dotp \vec{x} - (\vec{x} \dotp \vec{f}_i) = 0.
\end{align*}
This proves that $\vec{w}^\perp \in W^\perp$.

The reason we need to prove this decomposition is unique is because we started with the orthogonal basis $\{\vec{f}_{1}, \vec{f}_{2}, \dots, \vec{f}_{m}\}$ for $W$, but what would happen if we chose a different orthogonal basis?  

Suppose that $\{\vec{f}_1^\prime, \vec{f}_2^\prime, \dots, \vec{f}_m^\prime \}$  is another orthogonal basis of $W$, and let
\begin{equation*}
\vec{w}^{\prime} = \left(\frac{\vec{x} \dotp \vec{f}^{\prime}_{1}}{\norm{\vec{f}^{\prime}_{1}}^2}\right)\vec{f}^{\prime}_{1} + \left(\frac{\vec{x} \dotp \vec{f}^{\prime}_{2}}{\norm{\vec{f}^{\prime}_{2}}^2}\right)\vec{f}^{\prime}_{2} + \dots +\left(\frac{\vec{x} \dotp \vec{f}^{\prime}_{m}}{\norm{\vec{f}^{\prime}_{m}}^2}\right)\vec{f}^{\prime}_{m}
\end{equation*}
As before, $\vec{w}^{\prime} \in W$ and $\vec{x} - \vec{w}^{\prime} \in W^\perp$, and we must show that $\vec{w}^{\prime} = \vec{w}$. To see this, write the vector $\vec{w} - \vec{w}^\prime$ as follows:
\begin{equation*}
\vec{w} - \vec{w}^{\prime} = (\vec{x} - \vec{w}^{\prime}) - (\vec{x} - \vec{w})
\end{equation*}
This vector is in $W$ (because $\vec{w}$ and $\vec{w}^\prime$ are in $W$) and it is in $W^\perp$ (because $\vec{x} - \vec{w}^\prime$ and $\vec{x} - \vec{w}$ are in $W^\perp$), and so it must be the zero vector (it is orthogonal to itself!). This means $\vec{w}^\prime = \vec{w}$ as desired.
\end{proof}

\begin{example}\label{ex:OrthogDecomp}
Let $W$ be a subspace given by $W = \mbox{span}\left\{\begin{bmatrix}
  1 \\ 0 \\ 1 \\ 0
  \end{bmatrix},
  \begin{bmatrix}
  0 \\ 1 \\ 0 \\ 2
\end{bmatrix}\right\}$, and let $\vec{x}=\begin{bmatrix}
  1 \\ 2 \\ 3 \\ 4
  \end{bmatrix}$.  Write $\vec{x}$ as the sum of a vector in $W$ and a vector in $W^\perp$.
  
  \begin{explanation}
   Following the notation of Theorem \ref{th:OrthoDecomp}, we will write $\vec{x} = \vec{w} + \vec{w}^\perp$, where $\vec{w}=\mbox{proj}_W(\vec{x})$ and $\vec{w}^\perp = \vec{x} - \vec{w}$.  Let $\vec{f}_1=\begin{bmatrix}
  1 \\ 0 \\ 1 \\ 0
  \end{bmatrix}$ and let $\vec{f}_2=\begin{bmatrix}
  0 \\ 1 \\ 0 \\ 2
  \end{bmatrix}$.  We observe that we have the good fortune that $\vec{f}_1,\vec{f}_2$ is an orthogonal basis for $W$ (otherwise, our first step would be to use the Gram-Schmidt procedure to create an orthogonal basis for $W$).  We compute:
$$\vec{w}=\mbox{proj}_W(\vec{x})
      =\vec{x}-\mbox{proj}_{\vec{f}_1}(\vec{x})-\mbox{proj}_{\vec{f}_2}(\vec{x})
      = \begin{bmatrix}
  1 \\ 2 \\ 3 \\ 4
  \end{bmatrix} - \frac{4}{2}\begin{bmatrix}
  1 \\ 0 \\ 1 \\ 0
  \end{bmatrix} - \frac{10}{5}\begin{bmatrix}
  0 \\ 1 \\ 0 \\ 2
  \end{bmatrix} = \begin{bmatrix}
  2 \\ 2 \\ 2 \\ 4
  \end{bmatrix},$$
  and then $\vec{w}^\perp=\begin{bmatrix}
  1 \\ 2 \\ 3 \\ 4
  \end{bmatrix} - \begin{bmatrix}
  2 \\ 2 \\ 2 \\ 4
  \end{bmatrix} = \begin{bmatrix}
  -1 \\ 0 \\ 1 \\ 0
  \end{bmatrix}.$
  \end{explanation}
  \end{example}
  
The final theorem of this section shows that projection onto a subspace of $\RR^n$ is actually a linear transformation from $\RR^n$ to $\RR^n$.

\begin{theorem}\label{th:ProjLinTran}
Let $W$ be a fixed subspace of $\RR^n$. If we define $T : \RR^n \to \RR^n$ by
\begin{equation*}
T(\vec{x}) = \mbox{proj}_W(\vec{x}) \quad \mbox{ for all }\vec{x}\mbox{ in }\RR^n
\end{equation*}
\begin{enumerate}
\item\label{th:ProjLinTran_a} $T$ is a linear transformation.  (See \href{https://ximera.osu.edu/oerlinalg/LinearAlgebra/LTR-0010/main}{Introduction to Linear Transformations}.)

\item\label{th:ProjLinTran_b} The image of $T$ is $W$ and the kernel of $T$ is $ W^\perp$.  (See \href{https://ximera.osu.edu/oerlinalg/LinearAlgebra/LTR-0050/main}{Image and Kernel of a Linear Transformation}.)

\item\label{th:ProjLinTran_c} $\mbox{dim}(W) + \mbox{dim}(W^\perp) = n$.

\end{enumerate}
\end{theorem}

\begin{proof}
If $W = \{\vec{0}\}$, then $W^\perp = \RR^n$, and so $T(\vec{x}) = \vec{0}_\vec{x} = \vec{0}$ for all $\vec{x}$. Thus $T = 0$ is the zero (linear) operator, so \ref{th:ProjLinTran_a}, \ref{th:ProjLinTran_b}, and \ref{th:ProjLinTran_c} hold. Hence assume that $W \neq \{\vec{0}\}$.

\begin{enumerate}
\item If $\{\vec{q}_{1}, \vec{q}_{2}, \dots, \vec{q}_{m}\}$ is an orthonormal basis of $W$, then
\begin{equation}\label{orthonormalUeq}
T(\vec{x}) = (\vec{x} \dotp \vec{q}_{1})\vec{q}_{1} + (\vec{x} \dotp \vec{q}_{2})\vec{q}_{2} + \dots + (\vec{x} \dotp \vec{q}_{m})\vec{q}_{m} \quad \mbox{ for all }\vec{x} \mbox{ in } \RR^n
\end{equation}
by the definition of the projection. Thus $T$ is a linear transformation because
\begin{equation*}
(\vec{x} + \vec{y}) \dotp \vec{q}_{i} = \vec{x} \dotp \vec{q}_{i} + \vec{y} \dotp \vec{q}_{i} \quad \mbox{ and } \quad (r\vec{x}) \dotp \vec{q}_{i} = r(\vec{x} \dotp \vec{q}_{i}) \quad \mbox{ for each } i.
\end{equation*}

\item %This proof is correct, but perhaps written at a higher level than most of the proofs in our book.
We have the image of $T$ is a subset of $W$ by (\ref{orthonormalUeq}) because each $\vec{q}_{i}$ is in $W$. But if $\vec{x}$ is in $W$, then $\vec{x} = T(\vec{x})$ by (\ref{orthonormalUeq}) and Theorem \ref{th:fourierexpansion} applied to the space $W$. This shows that $W$ is a subset of the image of $T$, so the image of $T$ is $W$.

Now suppose that $\vec{x}$ is in $W^\perp$. Then $\vec{x} \dotp \vec{q}_{i} = 0$ for each $i$ (again because each $\vec{q}_{i}$ is in $W$) so $\vec{x}$ is in the kernel of $T$ by (\ref{th:023783}). Hence $W^\perp$ is in the kernel of $T$. On the other hand, Theorem~\ref{th:023783} shows that $\vec{x} - T(\vec{x})$ is in $W^\perp$ for all $\vec{x}$ in $\RR^n$, and it follows that the kernel of $T$ is in $W^\perp$. Hence the kernel of $T$ is $W^\perp$, proving \ref{th:ProjLinTran_b}.

\item This follows from \ref{th:ProjLinTran_a}, \ref{th:ProjLinTran_b}, and the Rank-Nullity theorem (Theorem~\ref{th:ranknullityforT}).
\end{enumerate}
\end{proof}

\section*{Practice Problems}
\begin{problem}\label{prob:Uperp}
Solve the linear system in Example \ref{ex:023829} and use your result to find a basis for $W^\perp$ if $W = \mbox{span}\left((1, -1, 2, 0), (1, 0, -2, 3)\right)$ in $\RR^4$.
\end{problem}

\begin{problem}\label{prob:brokenInteractive}
In this problem we return to the interactive in Exploration \ref{exp:discoverortho}, and we consider the case where the matrix $A$ is rank 1 (which that interactive could not handle).  This time, the sliders define row 1 of matrix $A$, and row 2 will be 2 times row 1.  Follow the prompts in the interactive to visualize $\mbox{row}(A)$ and $\mbox{null}(A)$.  What relationships do you observe between $\mbox{row}(A)$ and $\mbox{null}(A)$?

    \begin{center}
\geogebra{tyntjmdp}{950}{800}
\end{center}

\end{problem}

\begin{problem}\label{prob:finishex4subspaces}
In this problem you are asked to finish Example \ref{ex:4subspaces}.  More specifically, for the matrix $A=\begin{bmatrix}2&-1&1&-4&1\\1&0&3&3&0\\-2&1&-1&5&2\\4&-1&7&2&1\end{bmatrix}$, show that $\mbox{null}(A^T) = (\mbox{col}(A))^\perp$.  It may be helpful to consult \href{https://ximera.osu.edu/oerlinalg/LinearAlgebra/VSP-0040/main}{\underline{Subspaces of $\RR^n$ Associated with Matrices}}, where we found a basis for the column space of this matrix.
\end{problem}

\begin{problem}
In each case, write $\vec{x}$ as the sum of a vector in $W$ and a vector in $W^\perp$.

\begin{problem}\label{OrthoDecomp1}
$\vec{x} = (1, 5, 7)$, $W = \mbox{span}\left((1, -2, 3), (-1, 1, 1)\right)$
%ANSWER:  $\vec{x} = \frac{1}{182}(271,-221,1030)  + \frac{1}{182}(93,403,62)$
\end{problem}

\begin{problem}\label{OrthoDecomp2}
$\vec{x} = (2, 1, 6)$, $W = \mbox{span}\left((3, -1, 2), (2, 0, -3)\right)$
\end{problem}

\begin{problem}\label{OrthoDecomp3}
$\vec{x} = (3, 1, 5, 9)$, \\ $W = \mbox{span}\left((1, 0, 1, 1), (0, 1, -1, 1), (-2, 0, 1, 1)\right)$
%ANSWER:  $\vec{x}= \frac{1}{4}(1, 7, 11, 17) + \frac{1}{4}(7, -7, -7, 7)$
\end{problem}

\begin{problem}\label{OrthoDecomp4}
$\vec{x} = (2, 0, 1, 6)$, \\ \hspace*{-1em}$W = \mbox{span}\left\{(1, 1, 1, 1), (1, 1, -1, -1), (1, -1, 1, -1)\right\}$
\end{problem}

\begin{problem}\label{OrthoDecomp5}
$\vec{x} = (a, b, c, d)$, \\ $W = \mbox{span}\left((1, 0, 0, 0), (0, 1, 0, 0), (0, 0, 1, 0)\right)$
%ANSWER:  $\vec{x} = \frac{1}{12}(5a - 5b + c - 3d, -5a + 5b - c + 3d, a - b + 11c + 3d, -3a + 3b + 3c + 3d) + \frac{1}{12}(7a + 5b - c + 3d, 5a + 7b + c - 3d, -a + b + c -3d, 3a - 3b - 3c + 9d)$
\end{problem}

\begin{problem}\label{OrthoDecomp6}
$\vec{x} = (a, b, c, d)$, \\ $W = \mbox{span}\left((1, -1, 2, 0), (-1, 1, 1, 1)\right)$
\end{problem}
\end{problem}
	

\begin{problem}\label{Uperp}
Let $W = \mbox{span}\left(\vec{w}_{1}, \vec{w}_{2}, \dots, \vec{w}_{k}\right)$, $\vec{w}_{i}$ in $\RR^n$, and let $A$ be the $k \times n$ matrix with the $\vec{w}_{i}$ as rows.


\begin{enumerate}
\item Show that $W^\perp = \{\vec{x} \mid  \vec{x} \mbox{ in } \RR^n, A\vec{x}^{T} = \vec{0}\}$.

\item Use part (a) to find $W^\perp$ if \\ $W = \mbox{span}\left\{(1, -1, 2, 1), (1, 0, -1, 1)\right\}$.
%ANSWER:  $W^\perp = \mbox{span}\left((1, 3, 1, 0), (-1, 0, 0, 1)\right)$
\end{enumerate}

\end{problem}

\begin{problem}\label{prob:8_1_6}
\begin{enumerate}
\item Prove part \ref{th:023783a} of Theorem~\ref{th:023783}.

\item Prove part \ref{th:023783b} of Theorem~\ref{th:023783}.

\end{enumerate}
\end{problem}

\begin{problem} \label{prob:8_1_7}
Let $W$ be a subspace of $\RR^n$. If $\vec{x}$ in $\RR^n$ can be written in any way at all as $\vec{x} = \vec{p} + \vec{q}$ with $\vec{p}$ in $W$ and $\vec{q}$ in $W^\perp$, show that necessarily $\vec{p} = \mbox{proj}_W(\vec{x})$.
\end{problem}

\begin{problem}\label{prob:8_1_8}
Let $W$ be a subspace of $\RR^n$ and let $\vec{x}$ be a vector in $\RR^n$. Using Practice Problem \ref{prob:8_1_7}, or otherwise, show that $\vec{x}$ is in $W$ if and only if $\vec{x} = \mbox{proj}_W(\vec{x})$.

\begin{hint}
Write $\vec{w} = \mbox{proj}_W(\vec{x})$. Then $\vec{w}$ is in $W$ by definition. If $\vec{x}$ is $W$, then $\vec{x} - \vec{w}$ is in $W$. But $\vec{x} - \vec{w}$ is also in $W^\perp$ by Theorem~\ref{023885}, so $\vec{x} - \vec{w}$ is in $W \cap U^\perp = \{\vec{0}\}$. Thus $\vec{x} = \vec{w}$.
\end{hint}
\end{problem}


\begin{problem}\label{prob:8_1_10}
If $W$ is a subspace of $\RR^n$, show that $\mbox{proj}_W(\vec{x}) = \vec{x}$ for all $\vec{x}$ in $W$.

\begin{hint}
Let $\{\vec{q}_{1}, \vec{q}_{2}, \dots , \vec{q}_{m}\}$ be an orthonormal basis of $W$. If $\vec{x}$ is in $W$ the expansion theorem gives $\vec{x} = (\vec{x} \dotp \vec{q}_{1})\vec{q}_{1} + (\vec{x} \dotp \vec{q}_{2})\vec{q}_{2} + \dots  + (\vec{x} \dotp \vec{q}_{m})\vec{q}_{m} = \mbox{proj}_W(\vec{x})$.
\end{hint}
\end{problem}

\begin{problem}\label{prob:8_1_11}
If $W$ is a subspace of $\RR^n$, show that $\vec{x} = \mbox{proj}_W(\vec{x}) + \mbox{proj}_{W^\perp}\vec{x}$ for all $\vec{x}$ in $\RR^n$.
\end{problem}

\begin{problem}\label{prob:8_1_12}
If $\{\vec{v}_{1}, \dots, \vec{v}_{n}\}$ is an orthogonal basis of $\RR^n$ and $W = \mbox{span}\left(\vec{v}_{1}, \dots, \vec{v}_{m}\right)$, show that \\ $W^\perp = \mbox{span}\left(\vec{v}_{m + 1}, \dots, \vec{v}_{n}\right)$.
\end{problem}

\begin{problem}\label{prob:8_1_13}
If $W$ is a subspace of $\RR^n$, show that $\left(U^{\perp}\right)^\perp = U$. [\textit{Hint}: Show that $W \subseteq \left(U^{\perp}\right)^\perp$, then use Theorem~\ref{023953} (3) twice.]
\end{problem}

\begin{problem}\label{prob:8_1_14}
If $W$ is a subspace of $\RR^n$, show how to find an $n \times n$ matrix $A$ such that $W = \{\vec{x} \mid A\vec{x} = \vec{0}\}$. [\textit{Hint}: Practice Problem~\ref{prob:8_1_13}.]

\begin{hint}
Let $\{\vec{y}_{1}, \vec{y}_{2}, \dots, \vec{y}_{m}\}$ be a basis of $W^\perp$, and let $A$ be the $n \times n$ matrix with rows $\vec{y}^T_1, \vec{y}^T_2, \dots, \vec{y}^T_m, 0, \dots, 0$. Then $A\vec{x} = \vec{0}$ if and only if $\vec{y}_{i} \dotp \vec{x} = 0$ for each $i = 1, 2, \dots, m$; if and only if $\vec{x}$ is in $W^{\perp \perp} = U$.
\end{hint}
\end{problem}

\begin{problem}\label{prob:8_1_16}
If $U$ and $W$ are subspaces, show that $(U + W)^\perp = U^\perp \cap W^\perp$. [See Practice Problem \ref{prob:5_1_22}.]
\end{problem}

\begin{problem}
Think of $\RR^n$ as consisting of rows.

\begin{problem}\label{prob:8_1_17.1}
\item Let $E$ be an $n \times n$ matrix, and let \\ $W = \{\vec{x} E \mid \vec{x} \mbox{ in } \RR^n\}$. Show that the following are equivalent.


\begin{enumerate}[label={\roman*.}]
\item $E^{2} = E = E^{T}$ ($E$ is a \dfn{projection matrix}.

\item $(\vec{x} - \vec{x}E) \dotp (\vec{y}E) = 0$ for all $\vec{x}$ and $\vec{y}$ in $\RR^n$.

\item $\mbox{proj}_W(\vec{x}) = \vec{x}E$ for all $\vec{x}$ in $\RR^n$.
\begin{hint}
For (ii) implies (iii): Write $\vec{x} = \vec{x}E + (\vec{x} - \vec{x}E)$ and use the uniqueness argument preceding the definition of $\mbox{proj}_W(\vec{x})$. For (iii) implies (ii): $\vec{x} - \vec{x}E$ is in $W^\perp$ for all $\vec{x}$ in $\RR^n$.
\end{hint}
\end{enumerate}
\end{problem}

\begin{problem}\label{prob:8_1_17.2}
If $E$ is a projection matrix, show that $I - E$ is also a projection matrix.
\end{problem}

\begin{problem}\label{prob:8_1_17.3}
If $EF = 0 = FE$ and $E$ and $F$ are projection matrices, show that $E + F$ is also a projection matrix.

\end{problem}

\begin{problem}\label{prob:8_1_17.4}
If $A$ is $m \times n$ and $AA^{T}$ is invertible, show that $E = A^{T}(AA^{T})^{-1}A$ is a projection matrix.
%ANSWER $E^T = A^T[(AA^T)^-1]^T(A^T)^T  = A^T[(AA^T)^T]^{-1}A = A^T[AA^T]^{-1}A = E$
\end{problem}

\end{problem}




\section*{Text Source} This section was adapted from the second part of Section 8.1 of Keith Nicholson's \href{https://open.umn.edu/opentextbooks/textbooks/linear-algebra-with-applications}{\it Linear Algebra with Applications}. (CC-BY-NC-SA)

W. Keith Nicholson, {\it Linear Algebra with Applications}, Lyryx 2018, Open Edition, p. 415 

%\section*{Example Source}
Example \ref{ex:OrthogDecomp}  was adapted from Example 4.148  of Ken Kuttler's \href{https://open.umn.edu/opentextbooks/textbooks/a-first-course-in-linear-algebra-2017}{\it A First Course in Linear Algebra}. (CC-BY)

Ken Kuttler, {\it  A First Course in Linear Algebra}, Lyryx 2017, Open Edition, p. 249. 

%W. Keith Nicholson, {\it Linear Algebra with Applications}, Lyryx 2018, Open Edition, p. 346, 350

%\section*{Exercise Source}
%Practice Problems \ref{prob:linindabstractvsp1}, \ref{prob:linindabstractvsp2} and \ref{prob:linindabstractvsp3} are Exercises 6.3(a)(b)(c) from Keith Nicholson's \href{https://open.umn.edu/opentextbooks/textbooks/linear-algebra-with-applications}{\it Linear Algebra with Applications}. (CC-BY-NC-SA)

%W. Keith Nicholson, {\it Linear Algebra with Applications}, Lyryx 2018, Open Edition, p. 351



\end{document}