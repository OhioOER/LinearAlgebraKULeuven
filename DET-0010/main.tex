\documentclass{ximera}
%%% Begin Laad packages

\makeatletter
\@ifclassloaded{xourse}{%
    \typeout{Start loading preamble.tex (in a XOURSE)}%
    \def\isXourse{true}   % automatically defined; pre 112022 it had to be set 'manually' in a xourse
}{%
    \typeout{Start loading preamble.tex (NOT in a XOURSE)}%
}
\makeatother

\def\isEn\true 

\pgfplotsset{compat=1.16}

\usepackage{currfile}

% 201908/202301: PAS OP: babel en doclicense lijken problemen te veroorzaken in .jax bestand
% (wegens syntax error met toegevoegde \newcommands ...)
\pdfOnly{
    \usepackage[type={CC},modifier={by-nc-sa},version={4.0}]{doclicense}
    %\usepackage[hyperxmp=false,type={CC},modifier={by-nc-sa},version={4.0}]{doclicense}
    %%% \usepackage[dutch]{babel}
}



\usepackage[utf8]{inputenc}
\usepackage{morewrites}   % nav zomercursus (answer...?)
\usepackage{multirow}
\usepackage{multicol}
\usepackage{tikzsymbols}
\usepackage{ifthen}
%\usepackage{animate} BREAKS HTML STRUCTURE USED BY XIMERA
\usepackage{relsize}

\usepackage{eurosym}    % \euro  (€ werkt niet in xake ...?)
\usepackage{fontawesome} % smileys etc

% Nuttig als ook interactieve beamer slides worden voorzien:
\providecommand{\p}{} % default nothing ; potentially usefull for slides: redefine as \pause
%providecommand{\p}{\pause}

    % Layout-parameters voor het onderschrift bij figuren
\usepackage[margin=10pt,font=small,labelfont=bf, labelsep=endash,format=hang]{caption}
%\usepackage{caption} % captionof
%\usepackage{pdflscape}    % landscape environment

% Met "\newcommand\showtodonotes{}" kan je todonotes tonen (in pdf/online)
% 201908: online werkt het niet (goed)
\providecommand\showtodonotes{disable}
\providecommand\todo[1]{\typeout{TODO #1}}
%\usepackage[\showtodonotes]{todonotes}
%\usepackage{todonotes}

%
% Poging tot aanpassen layout
%
\usepackage{tcolorbox}
\tcbuselibrary{theorems}

%%% Einde laad packages

%%% Begin Ximera specifieke zaken

\graphicspath{
	{../../}
	{../}
	{./}
  	{../../pictures/}
   	{../pictures/}
   	{./pictures/}
	{./explog/}    % M05 in groeimodellen       
}

%%% Einde Ximera specifieke zaken

%
% define softer blue/red/green, use KU Leuven base colors for blue (and dark orange for red ?)
%
% todo: rather redefine blue/red/green ...?
%\definecolor{xmblue}{rgb}{0.01, 0.31, 0.59}
%\definecolor{xmred}{rgb}{0.89, 0.02, 0.17}
\definecolor{xmdarkblue}{rgb}{0.122, 0.671, 0.835}   % KU Leuven Blauw
\definecolor{xmblue}{rgb}{0.114, 0.553, 0.69}        % KU Leuven Blauw
\definecolor{xmgreen}{rgb}{0.13, 0.55, 0.13}         % No KULeuven variant for green found ...

\definecolor{xmaccent}{rgb}{0.867, 0.541, 0.18}      % KU Leuven Accent (orange ...)
\definecolor{kuaccent}{rgb}{0.867, 0.541, 0.18}      % KU Leuven Accent (orange ...)

\colorlet{xmred}{xmaccent!50!black}                  % Darker version of KU Leuven Accent

\providecommand{\blue}[1]{{\color{blue}#1}}    
\providecommand{\red}[1]{{\color{red}#1}}

\renewcommand\CancelColor{\color{xmaccent!50!black}}

% werkt in math en text mode om MATH met oranje (of grijze...)  achtergond te tonen (ook \important{\text{blabla}} lijkt te werken)
%\newcommand{\important}[1]{\ensuremath{\colorbox{xmaccent!50!white}{$#1$}}}   % werkt niet in Mathjax
%\newcommand{\important}[1]{\ensuremath{\colorbox{lightgray}{$#1$}}}
\newcommand{\important}[1]{\ensuremath{\colorbox{orange}{$#1$}}}   % TODO: kleur aanpassen voor mathjax; wordt overschreven infra!


% Uitzonderlijk kan met \pdfnl in de PDF een newline worden geforceerd, die online niet nodig/nuttig is omdat daar de regellengte hoe dan ook niet gekend is.
\ifdefined\HCode%
\providecommand{\pdfnl}{}%
\else%
\providecommand{\pdfnl}{%
  \\%
}%
\fi

% Uitzonderlijk kan met \handoutnl in de handout-PDF een newline worden geforceerd, die noch online noch in de PDF-met-antwoorden nuttig is.
\ifdefined\HCode
\providecommand{\handoutnl}{}
\else
\providecommand{\handoutnl}{%
\ifhandout%
  \nl%
\fi%
}
\fi



% \cellcolor IGNORED by tex4ht ?
% \begin{center} seems not to wordk
    % (missing margin-left: auto;   on tabular-inside-center ???)
%\newcommand{\importantcell}[1]{\ensuremath{\cellcolor{lightgray}#1}}  %  in tabular; usablility to be checked
\providecommand{\importantcell}[1]{\ensuremath{#1}}     % no mathjax2 support for colloring array cells

\pdfOnly{
  \renewcommand{\important}[1]{\ensuremath{\colorbox{kuaccent!50!white}{$#1$}}}
  \renewcommand{\importantcell}[1]{\ensuremath{\cellcolor{kuaccent!40!white}#1}}   
}

%%% Tikz styles


\pgfplotsset{compat=1.16}

\usetikzlibrary{trees,positioning,arrows,fit,shapes,math,calc,decorations.markings,through,intersections,patterns,matrix}

\usetikzlibrary{decorations.pathreplacing,backgrounds}    % 5/2023: from experimental


\usetikzlibrary{angles,quotes}

\usepgfplotslibrary{fillbetween} % bepaalde_integraal
\usepgfplotslibrary{polar}    % oa voor poolcoordinaten.tex

\pgfplotsset{ownstyle/.style={axis lines = center, axis equal image, xlabel = $x$, ylabel = $y$, enlargelimits}} 

\pgfplotsset{
	plot/.style={no marks,samples=50}
}

\newcommand{\xmPlotsColor}{
	\pgfplotsset{
		plot1/.style={darkgray,no marks,samples=100},
		plot2/.style={lightgray,no marks,samples=100},
		plotresult/.style={blue,no marks,samples=100},
		plotblue/.style={blue,no marks,samples=100},
		plotred/.style={red,no marks,samples=100},
		plotgreen/.style={green,no marks,samples=100},
		plotpurple/.style={purple,no marks,samples=100}
	}
}
\newcommand{\xmPlotsBlackWhite}{
	\pgfplotsset{
		plot1/.style={black,loosely dashed,no marks,samples=100},
		plot2/.style={black,loosely dotted,no marks,samples=100},
		plotresult/.style={black,no marks,samples=100},
		plotblue/.style={black,no marks,samples=100},
		plotred/.style={black,dotted,no marks,samples=100},
		plotgreen/.style={black,dashed,no marks,samples=100},
		plotpurple/.style={black,dashdotted,no marks,samples=100}
	}
}


\newcommand{\xmPlotsColorAndStyle}{
	\pgfplotsset{
		plot1/.style={darkgray,no marks,samples=100},
		plot2/.style={lightgray,no marks,samples=100},
		plotresult/.style={blue,no marks,samples=100},
		plotblue/.style={xmblue,no marks,samples=100},
		plotred/.style={xmred,dashed,thick,no marks,samples=100},
		plotgreen/.style={xmgreen,dotted,very thick,no marks,samples=100},
		plotpurple/.style={purple,no marks,samples=100}
	}
}


%\iftikzexport
\xmPlotsColorAndStyle
%\else
%\xmPlotsBlackWhite
%\fi
%%%


%
% Om venndiagrammen te arceren ...
%
\makeatletter
\pgfdeclarepatternformonly[\hatchdistance,\hatchthickness]{north east hatch}% name
{\pgfqpoint{-1pt}{-1pt}}% below left
{\pgfqpoint{\hatchdistance}{\hatchdistance}}% above right
{\pgfpoint{\hatchdistance-1pt}{\hatchdistance-1pt}}%
{
	\pgfsetcolor{\tikz@pattern@color}
	\pgfsetlinewidth{\hatchthickness}
	\pgfpathmoveto{\pgfqpoint{0pt}{0pt}}
	\pgfpathlineto{\pgfqpoint{\hatchdistance}{\hatchdistance}}
	\pgfusepath{stroke}
}
\pgfdeclarepatternformonly[\hatchdistance,\hatchthickness]{north west hatch}% name
{\pgfqpoint{-\hatchthickness}{-\hatchthickness}}% below left
{\pgfqpoint{\hatchdistance+\hatchthickness}{\hatchdistance+\hatchthickness}}% above right
{\pgfpoint{\hatchdistance}{\hatchdistance}}%
{
	\pgfsetcolor{\tikz@pattern@color}
	\pgfsetlinewidth{\hatchthickness}
	\pgfpathmoveto{\pgfqpoint{\hatchdistance+\hatchthickness}{-\hatchthickness}}
	\pgfpathlineto{\pgfqpoint{-\hatchthickness}{\hatchdistance+\hatchthickness}}
	\pgfusepath{stroke}
}
%\makeatother

\tikzset{
    hatch distance/.store in=\hatchdistance,
    hatch distance=10pt,
    hatch thickness/.store in=\hatchthickness,
   	hatch thickness=2pt
}

\colorlet{circle edge}{black}
\colorlet{circle area}{blue!20}


\tikzset{
    filled/.style={fill=green!30, draw=circle edge, thick},
    arceerl/.style={pattern=north east hatch, pattern color=blue!50, draw=circle edge},
    arceerr/.style={pattern=north west hatch, pattern color=yellow!50, draw=circle edge},
    outline/.style={draw=circle edge, thick}
}




%%% Updaten commando's
\def\hoofding #1#2#3{\maketitle}     % OBSOLETE ??

% we willen (bijna) altijd \geqslant ipv \geq ...!
\newcommand{\geqnoslant}{\geq}
\renewcommand{\geq}{\geqslant}
\newcommand{\leqnoslant}{\leq}
\renewcommand{\leq}{\leqslant}

% Todo: (201908) waarom komt er (soms) underlined voor emph ...?
\renewcommand{\emph}[1]{\textit{#1}}

% API commando's

\newcommand{\ds}{\displaystyle}
\newcommand{\ts}{\textstyle}  % tegenhanger van \ds   (Ximera zet PER  DEFAULT \ds!)

% uit Zomercursus-macro's: 
\newcommand{\bron}[1]{\begin{scriptsize} \emph{#1} \end{scriptsize}}     % deprecated ...?


%definities nieuwe commando's - afkortingen veel gebruikte symbolen
\newcommand{\R}{\ensuremath{\mathbb{R}}}
\newcommand{\Rnul}{\ensuremath{\mathbb{R}_0}}
\newcommand{\Reen}{\ensuremath{\mathbb{R}\setminus\{1\}}}
\newcommand{\Rnuleen}{\ensuremath{\mathbb{R}\setminus\{0,1\}}}
\newcommand{\Rplus}{\ensuremath{\mathbb{R}^+}}
\newcommand{\Rmin}{\ensuremath{\mathbb{R}^-}}
\newcommand{\Rnulplus}{\ensuremath{\mathbb{R}_0^+}}
\newcommand{\Rnulmin}{\ensuremath{\mathbb{R}_0^-}}
\newcommand{\Rnuleenplus}{\ensuremath{\mathbb{R}^+\setminus\{0,1\}}}
\newcommand{\N}{\ensuremath{\mathbb{N}}}
\newcommand{\Nnul}{\ensuremath{\mathbb{N}_0}}
\newcommand{\Z}{\ensuremath{\mathbb{Z}}}
\newcommand{\Znul}{\ensuremath{\mathbb{Z}_0}}
\newcommand{\Zplus}{\ensuremath{\mathbb{Z}^+}}
\newcommand{\Zmin}{\ensuremath{\mathbb{Z}^-}}
\newcommand{\Znulplus}{\ensuremath{\mathbb{Z}_0^+}}
\newcommand{\Znulmin}{\ensuremath{\mathbb{Z}_0^-}}
\newcommand{\C}{\ensuremath{\mathbb{C}}}
\newcommand{\Cnul}{\ensuremath{\mathbb{C}_0}}
\newcommand{\Cplus}{\ensuremath{\mathbb{C}^+}}
\newcommand{\Cmin}{\ensuremath{\mathbb{C}^-}}
\newcommand{\Cnulplus}{\ensuremath{\mathbb{C}_0^+}}
\newcommand{\Cnulmin}{\ensuremath{\mathbb{C}_0^-}}
\newcommand{\Q}{\ensuremath{\mathbb{Q}}}
\newcommand{\Qnul}{\ensuremath{\mathbb{Q}_0}}
\newcommand{\Qplus}{\ensuremath{\mathbb{Q}^+}}
\newcommand{\Qmin}{\ensuremath{\mathbb{Q}^-}}
\newcommand{\Qnulplus}{\ensuremath{\mathbb{Q}_0^+}}
\newcommand{\Qnulmin}{\ensuremath{\mathbb{Q}_0^-}}

\newcommand{\perdef}{\overset{\mathrm{def}}{=}}
\newcommand{\pernot}{\overset{\mathrm{notatie}}{=}}
\newcommand\perinderdaad{\overset{!}{=}}     % voorlopig gebruikt in limietenrekenregels
\newcommand\perhaps{\overset{?}{=}}          % voorlopig gebruikt in limietenrekenregels

\newcommand{\degree}{^\circ}


\DeclareMathOperator{\dom}{dom}     % domein
\DeclareMathOperator{\codom}{codom} % codomein
\DeclareMathOperator{\bld}{bld}     % beeld
\DeclareMathOperator{\graf}{graf}   % grafiek
\DeclareMathOperator{\rico}{rico}   % richtingcoëfficient
\DeclareMathOperator{\co}{co}       % coordinaat
\DeclareMathOperator{\gr}{gr}       % graad

\newcommand{\func}[5]{\ensuremath{#1: #2 \rightarrow #3: #4 \mapsto #5}} % Easy to write a function


% Operators
\DeclareMathOperator{\bgsin}{bgsin}
\DeclareMathOperator{\bgcos}{bgcos}
\DeclareMathOperator{\bgtan}{bgtan}
\DeclareMathOperator{\bgcot}{bgcot}
\DeclareMathOperator{\bgsinh}{bgsinh}
\DeclareMathOperator{\bgcosh}{bgcosh}
\DeclareMathOperator{\bgtanh}{bgtanh}
\DeclareMathOperator{\bgcoth}{bgcoth}

% Oude \Bgsin etc deprecated: gebruik \bgsin, en herdefinieer dat als je Bgsin wil!
%\DeclareMathOperator{\cosec}{cosec}    % not used? gebruik \csc en herdefinieer

% operatoren voor differentialen: to be verified; 1/2020: inconsequent gebruik bij afgeleiden/integralen
\renewcommand{\d}{\mathrm{d}}
\newcommand{\dx}{\d x}
\newcommand{\dd}[1]{\frac{\mathrm{d}}{\mathrm{d}#1}}
\newcommand{\ddx}{\dd{x}}

% om in voorbeelden/oefeningen de notatie voor afgeleiden te kunnen kiezen
% Usage: \afg{(2\sin(x))}  (en wordt d/dx, of accent, of D )
%\newcommand{\afg}[1]{{#1}'}
\newcommand{\afg}[1]{\left(#1\right)'}
%\renewcommand{\afg}[1]{\frac{\mathrm{d}#1}{\mathrm{d}x}}   % include in relevant exercises ...
%\renewcommand{\afg}[1]{D{#1}}

%
% \xmxxx commands: Extra KU Leuven functionaliteit van, boven of naast Ximera
%   ( Conventie 8/2019: xm+nederlandse omschrijving, maar is niet consequent gevolgd, en misschien ook niet erg handig !)
%
% (Met een minimale ximera.cls en preamble.tex zou een bruikbare .pdf moeten kunnen worden gemaakt van eender welke ximera)
%
% Usage: \xmtitle[Mijn korte abstract]{Mijn titel}{Mijn abstract}
% Eerste command na \begin{document}:
%  -> definieert de \title
%  -> definieert de abstract
%  -> doet \maketitle ( dus: print de hoofding als 'chapter' of 'sectie')
% Optionele parameter geeft eenn kort abstract (die met de globale setting \xmshortabstract{} al dan niet kan worden geprint.
% De optionele korte abstract kan worden gebruikt voor pseudo-grappige abtsarts, dus dus globaal al dan niet kunnen worden gebuikt...
% Globale settings:
%  de (optionele) 'korte abstract' wordt enkele getoond als \xmshortabstract is gezet
\providecommand\xmshortabstract{} % default: print (only!) short abstract if present
\newcommand{\xmtitle}[3][]{
	\title{#2}
	\begin{abstract}
		\ifdefined\xmshortabstract
		\ifstrempty{#1}{%
			#3
		}{%
			#1
		}%
		\else
		#3
		\fi
	\end{abstract}
	\maketitle
}

% 
% Kleine grapjes: moeten zonder verder gevolg kunnen worden verwijderd
%
%\newcommand{\xmopje}[1]{{\small#1{\reversemarginpar\marginpar{\Smiley}}}}   % probleem in floats!!
\newtoggle{showxmopje}
\toggletrue{showxmopje}

\newcommand{\xmopje}[1]{%
   \iftoggle{showxmopje}{#1}{}%
}


% -> geef een abstracte-formule-met-rechts-een-concreet-voorbeeld
% VB:  \formulevb{a^2+b^2=c^2}{3^2+4^2=5^2}
%
\ifdefined\HCode
\NewEnviron{xmdiv}[1]{\HCode{\Hnewline<div class="#1">\Hnewline}\BODY{\HCode{\Hnewline</div>\Hnewline}}}
\else
\NewEnviron{xmdiv}[1]{\BODY}
\fi

\providecommand{\formulevb}[2]{
	{\centering

    \begin{xmdiv}{xmformulevb}    % zie css voor online layout !!!
	\begin{tabular}{lcl}
		\important{#1}
		&  &
		Vb: $#2$
		\end{tabular}
	\end{xmdiv}

	}
}

\ifdefined\HCode
\providecommand{\vb}[1]{%
    \HCode{\Hnewline<span class="xmvb">}#1\HCode{</span>\Hnewline}%
}
\else
\providecommand{\vb}[1]{
    \colorbox{blue!10}{#1}
}
\fi

\ifdefined\HCode
\providecommand{\xmcolorbox}[2]{
	\HCode{\Hnewline<div class="xmcolorbox">\Hnewline}#2\HCode{\Hnewline</div>\Hnewline}
}
\else
\providecommand{\xmcolorbox}[2]{
  \cellcolor{#1}#2
}
\fi


\ifdefined\HCode
\providecommand{\xmopmerking}[1]{
 \HCode{\Hnewline<div class="xmopmerking">\Hnewline}#1\HCode{\Hnewline</div>\Hnewline}
}
\else
\providecommand{\xmopmerking}[1]{
	{\footnotesize #1}
}
\fi
% \providecommand{\voorbeeld}[1]{
% 	\colorbox{blue!10}{$#1$}
% }



% Hernoem Proof naar Bewijs, nodig voor HTML versie
\renewcommand*{\proofname}{Bewijs}

% Om opgave van oefening (wordt niet geprint bij oplossingenblad)
% (to be tested test)
\NewEnviron{statement}{\BODY}

% Environment 'oplossing' en 'uitkomst'
% voor resp. volledige 'uitwerking' dan wel 'enkel eindresultaat'
% geimplementeerd via feedback, omdat er in de ximera-server adhoc feedback-code is toegevoegd
%% Niet tonen indien handout
%% Te gebruiken om volledige oplossingen/uitwerkingen van oefeningen te tonen
%% \begin{oplossing}        De optelling is commutatief \end{oplossing}  : verschijnt online enkel 'op vraag'
%% \begin{oplossing}[toon]  De optelling is commutatief \end{oplossing}  : verschijnt steeds onmiddellijk online (bv te gebruiken bij voorbeelden) 

\ifhandout%
    \NewEnviron{oplossing}[1][onzichtbaar]%
    {%
    \ifthenelse{\equal{\detokenize{#1}}{\detokenize{toon}}}
    {
    \def\PH@Command{#1}% Use PH@Command to hold the content and be a target for "\expandafter" to expand once.

    \begin{trivlist}% Begin the trivlist to use formating of the "Feedback" label.
    \item[\hskip \labelsep\small\slshape\bfseries Oplossing% Format the "Feedback" label. Don't forget the space.
    %(\texttt{\detokenize\expandafter{\PH@Command}}):% Format (and detokenize) the condition for feedback to trigger
    \hspace{2ex}]\small%\slshape% Insert some space before the actual feedback given.
    \BODY
    \end{trivlist}
    }
    {  % \begin{feedback}[solution]   \BODY     \end{feedback}  }
    }
    }    
\else
% ONLY for HTML; xmoplossing is styled with css, and is not, and need not be a LaTeX environment
% THUS: it does NOT use feedback anymore ...
%    \NewEnviron{oplossing}{\begin{expandable}{xmoplossing}{\nlen{Toon uitwerking}{Show solution}}{\BODY}\end{expandable}}
    \newenvironment{oplossing}[1][onzichtbaar]
   {%
       \begin{expandable}{xmoplossing}{}
   }
   {%
   	   \end{expandable}
   } 
%     \newenvironment{oplossing}[1][onzichtbaar]
%    {%
%        \begin{feedback}[solution]   	
%    }
%    {%
%    	   \end{feedback}
%    } 
\fi

\ifhandout%
    \NewEnviron{uitkomst}[1][onzichtbaar]%
    {%
    \ifthenelse{\equal{\detokenize{#1}}{\detokenize{toon}}}
    {
    \def\PH@Command{#1}% Use PH@Command to hold the content and be a target for "\expandafter" to expand once.

    \begin{trivlist}% Begin the trivlist to use formating of the "Feedback" label.
    \item[\hskip \labelsep\small\slshape\bfseries Uitkomst:% Format the "Feedback" label. Don't forget the space.
    %(\texttt{\detokenize\expandafter{\PH@Command}}):% Format (and detokenize) the condition for feedback to trigger
    \hspace{2ex}]\small%\slshape% Insert some space before the actual feedback given.
    \BODY
    \end{trivlist}
    }
    {  % \begin{feedback}[solution]   \BODY     \end{feedback}  }
    }
    }    
\else
\ifdefined\HCode
   \newenvironment{uitkomst}[1][onzichtbaar]
    {%
        \begin{expandable}{xmuitkomst}{}%
    }
    {%
    	\end{expandable}%
    } 
\else
  % Do NOT print 'uitkomst' in non-handout
  %  (presumably, there is also an 'oplossing' ??)
  \newenvironment{uitkomst}[1][onzichtbaar]{}{}
\fi
\fi

%
% Uitweidingen zijn extra's die niet redelijkerwijze tot de leerstof behoren
% Uitbreidingen zijn extra's die wel redelijkerwijze tot de leerstof van bv meer geavanceerde versies kunnen behoren (B-programma/Wiskundestudenten/...?)
% Nog niet voorzien: design voor verschillende versies (A/B programma, BIO, voorkennis/ ...)
% Voor 'uitweidingen' is er een environment die online per default is ingeklapt, en in pdf al dan niet kan worden geincluded  (via \xmnouitweiding) 
%
% in een xourse, per default GEEN uitweidingen, tenzij \xmuitweiding expliciet ergens is gezet ...
\ifdefined\isXourse
   \ifdefined\xmuitweiding
   \else
       \def\xmnouitweiding{true}
   \fi
\fi

\ifdefined\xmnouitweiding
\newcounter{xmuitweiding}  % anders error undefined ...  
\excludecomment{xmuitweiding}
\else
\newtheoremstyle{dotless}{}{}{}{}{}{}{ }{}
\theoremstyle{dotless}
\newtheorem*{xmuitweidingnofrills}{}   % nofrills = no accordion; gebruikt dus de dotless theoremstyle!

\newcounter{xmuitweiding}
\newenvironment{xmuitweiding}[1][ ]%
{% 
	\refstepcounter{xmuitweiding}%
    \begin{expandable}{xmuitweiding}{\nlentext{Uitweiding \arabic{xmuitweiding}: #1}{Digression \arabic{xmuitweiding}: #1}}%
	\begin{xmuitweidingnofrills}%
}
{%
    \end{xmuitweidingnofrills}%
    \end{expandable}%
}   
% \newenvironment{xmuitweiding}[1][ ]%
% {% 
% 	\refstepcounter{xmuitweiding}
% 	\begin{accordion}\begin{accordion-item}[Uitweiding \arabic{xmuitweiding}: #1]%
% 			\begin{xmuitweidingnofrills}%
% 			}
% 			{\end{xmuitweidingnofrills}\end{accordion-item}\end{accordion}}   
\fi


\newenvironment{xmexpandable}[1][]{
	\begin{accordion}\begin{accordion-item}[#1]%
		}{\end{accordion-item}\end{accordion}}


% Command that gives a selection box online, but just prints the right answer in pdf
\newcommand{\xmonlineChoice}[1]{\pdfOnly{\wordchoicegiventrue}\wordChoice{#1}\pdfOnly{\wordchoicegivenfalse}}   % deprecated, gebruik onlineChoice ...
\newcommand{\onlineChoice}[1]{\pdfOnly{\wordchoicegiventrue}\wordChoice{#1}\pdfOnly{\wordchoicegivenfalse}}

\newcommand{\TJa}{\nlentext{ Ja }{ Yes }}
\newcommand{\TNee}{\nlentext{ Nee }{ No }}
\newcommand{\TJuist}{\nlentext{ Juist }{ True }}
\newcommand{\TFout}{\nlentext{ Fout }{ False }}

\newcommand{\choiceTrue }{{\renewcommand{\choiceminimumhorizontalsize}{4em}\wordChoice{\choice[correct]{\TJuist}\choice{\TFout}}}}
\newcommand{\choiceFalse}{{\renewcommand{\choiceminimumhorizontalsize}{4em}\wordChoice{\choice{\TJuist}\choice[correct]{\TFout}}}}

\newcommand{\choiceYes}{{\renewcommand{\choiceminimumhorizontalsize}{3em}\wordChoice{\choice[correct]{\TJa}\choice{\TNee}}}}
\newcommand{\choiceNo }{{\renewcommand{\choiceminimumhorizontalsize}{3em}\wordChoice{\choice{\TJa}\choice[correct]{\TNee}}}}

% Optional nicer formatting for wordChoice in PDF

\let\inlinechoiceorig\inlinechoice

%\makeatletter
%\providecommand{\choiceminimumverticalsize}{\vphantom{$\frac{\sqrt{2}}{2}$}}   % minimum height of boxes (cfr infra)
\providecommand{\choiceminimumverticalsize}{\vphantom{$\tfrac{2}{2}$}}   % minimum height of boxes (cfr infra)
\providecommand{\choiceminimumhorizontalsize}{1em}   % minimum width of boxes (cfr infra)

\newcommand{\inlinechoicesquares}[2][]{%
		\setkeys{choice}{#1}%
		\ifthenelse{\boolean{\choice@correct}}%
		{%
            \ifhandout%
               \fbox{\choiceminimumverticalsize #2}\allowbreak\ignorespaces%
            \else%
               \fcolorbox{blue}{blue!20}{\choiceminimumverticalsize #2}\allowbreak\ignorespaces\setkeys{choice}{correct=false}\ignorespaces%
            \fi%
		}%
		{% else
			\fbox{\choiceminimumverticalsize #2}\allowbreak\ignorespaces%  HACK: wat kleiner, zodat fits on line ... 	
		}%
}

\newcommand{\inlinechoicesquareX}[2][]{%
		\setkeys{choice}{#1}%
		\ifthenelse{\boolean{\choice@correct}}%
		{%
            \ifhandout%
               \framebox[\ifdim\choiceminimumhorizontalsize<\width\width\else\choiceminimumhorizontalsize\fi]{\choiceminimumverticalsize\ #2\ }\allowbreak\ignorespaces\setkeys{choice}{correct=false}\ignorespaces%
            \else%
               \fcolorbox{blue}{blue!20}{\makebox[\ifdim\choiceminimumhorizontalsize<\width\width\else\choiceminimumhorizontalsize\fi]{\choiceminimumverticalsize #2}}\allowbreak\ignorespaces\setkeys{choice}{correct=false}\ignorespaces%
            \fi%
		}%
		{% else
        \ifhandout%
			\framebox[\ifdim\choiceminimumhorizontalsize<\width\width\else\choiceminimumhorizontalsize\fi]{\choiceminimumverticalsize\ #2\ }\allowbreak\ignorespaces%  HACK: wat kleiner, zodat fits on line ... 	
        \fi
		}%
}


\newcommand{\inlinechoicepuntjes}[2][]{%
		\setkeys{choice}{#1}%
		\ifthenelse{\boolean{\choice@correct}}%
		{%
            \ifhandout%
               \dots\ldots\ignorespaces\setkeys{choice}{correct=false}\ignorespaces
            \else%
               \fcolorbox{blue}{blue!20}{\choiceminimumverticalsize #2}\allowbreak\ignorespaces\setkeys{choice}{correct=false}\ignorespaces%
            \fi%
		}%
		{% else
			%\fbox{\choiceminimumverticalsize #2}\allowbreak\ignorespaces%  HACK: wat kleiner, zodat fits on line ... 	
		}%
}

% print niets, maar definieer globale variable \myanswer
%  (gebruikt om oplossingsbladen te printen) 
\newcommand{\inlinechoicedefanswer}[2][]{%
		\setkeys{choice}{#1}%
		\ifthenelse{\boolean{\choice@correct}}%
		{%
               \gdef\myanswer{#2}\setkeys{choice}{correct=false}

		}%
		{% else
			%\fbox{\choiceminimumverticalsize #2}\allowbreak\ignorespaces%  HACK: wat kleiner, zodat fits on line ... 	
		}%
}



%\makeatother

\newcommand{\setchoicedefanswer}{
\ifdefined\HCode
\else
%    \renewenvironment{multipleChoice@}[1][]{}{} % remove trailing ')'
    \let\inlinechoice\inlinechoicedefanswer
\fi
}

\newcommand{\setchoicepuntjes}{
\ifdefined\HCode
\else
    \renewenvironment{multipleChoice@}[1][]{}{} % remove trailing ')'
    \let\inlinechoice\inlinechoicepuntjes
\fi
}
\newcommand{\setchoicesquares}{
\ifdefined\HCode
\else
    \renewenvironment{multipleChoice@}[1][]{}{} % remove trailing ')'
    \let\inlinechoice\inlinechoicesquares
\fi
}
%
\newcommand{\setchoicesquareX}{
\ifdefined\HCode
\else
    \renewenvironment{multipleChoice@}[1][]{}{} % remove trailing ')'
    \let\inlinechoice\inlinechoicesquareX
\fi
}
%
\newcommand{\setchoicelist}{
\ifdefined\HCode
\else
    \renewenvironment{multipleChoice@}[1][]{}{)}% re-add trailing ')'
    \let\inlinechoice\inlinechoiceorig
\fi
}

\setchoicesquareX  % by default list-of-squares with onlineChoice in PDF

% Omdat multicols niet werkt in html: enkel in pdf  (in html zijn langere pagina's misschien ook minder storend)
\newenvironment{xmmulticols}[1][2]{
 \pdfOnly{\begin{multicols}{#1}}%
}{ \pdfOnly{\end{multicols}}}

%
% Te gebruiken in plaats van \section\subsection
%  (in een printstyle kan dan het level worden aangepast
%    naargelang \chapter vs \section style )
% 3/2021: DO NOT USE \xmsubsection !
\newcommand\xmsection\subsection
\newcommand\xmsubsection\subsubsection

% Aanpassen printversie
%  (hier gedefinieerd, zodat ze in xourse kunnen worden gezet/overschreven)
\providebool{parttoc}
\providebool{printpartfrontpage}
\providebool{printactivitytitle}
\providebool{printactivityqrcode}
\providebool{printactivityurl}
\providebool{printcontinuouspagenumbers}
\providebool{numberactivitiesbysubpart}
\providebool{addtitlenumber}
\providebool{addsectiontitlenumber}
\addtitlenumbertrue
\addsectiontitlenumbertrue

% The following three commands are hardcoded in xake, you can't create other commands like these, without adding them to xake as well
%  ( gebruikt in xourses om juiste soort titelpagina te krijgen voor verschillende ximera's )
\newcommand{\activitychapter}[2][]{
    {    
    \ifstrequal{#1}{notnumbered}{
        \addtitlenumberfalse
    }{}
    \typeout{ACTIVITYCHAPTER #2}   % logging
	\chapterstyle
	\activity{#2}
    }
}
\newcommand{\activitysection}[2][]{
    {
    \ifstrequal{#1}{notnumbered}{
        \addsectiontitlenumberfalse
    }{}
	\typeout{ACTIVITYSECTION #2}   % logging
	\sectionstyle
	\activity{#2}
    }
}
% Practices worden als activity getoond om de grote blokken te krijgen online
\newcommand{\practicesection}[2][]{
    {
    \ifstrequal{#1}{notnumbered}{
        \addsectiontitlenumberfalse
    }{}
    \typeout{PRACTICESECTION #2}   % logging
	\sectionstyle
	\activity{#2}
    }
}
\newcommand{\activitychapterlink}[3][]{
    {
    \ifstrequal{#1}{notnumbered}{
        \addtitlenumberfalse
    }{}
    \typeout{ACTIVITYCHAPTERLINK #3}   % logging
	\chapterstyle
	\activitylink[#1]{#2}{#3}
    }
}

\newcommand{\activitysectionlink}[3][]{
    {
    \ifstrequal{#1}{notnumbered}{
        \addsectiontitlenumberfalse
    }{}
    \typeout{ACTIVITYSECTIONLINK #3}   % logging
	\sectionstyle
	\activitylink[#1]{#2}{#3}
    }
}


% Commando om de printstyle toe te voegen in ximera's. Zorgt ervoor dat er geen problemen zijn als je de xourses compileert
% hack om onhandige relative paden in TeX te omzeilen
% should work both in xourse and ximera (pre-112022 only in ximera; thus obsoletes adhoc setup in xourses)
% loads global.sty if present (cfr global.css for online settings!)
% use global.sty to overwrite settings in printstyle.sty ...
\newcommand{\addPrintStyle}[1]{
\iftikzexport\else   % only in PDF
  \makeatletter
  \ifx\@onlypreamble\@notprerr\else   % ONLY if in tex-preamble   (and e.g. not when included from xourse)
    \typeout{Loading printstyle}   % logging
    \usepackage{#1/printstyle} % mag enkel geinclude worden als je die apart compileert
    \IfFileExists{#1/global.sty}{
        \typeout{Loading printstyle-folder #1/global.sty}   % logging
        \usepackage{#1/global}
        }{
        \typeout{Info: No extra #1/global.sty}   % logging
    }   % load global.sty if present
    \IfFileExists{global.sty}{
        \typeout{Loading local-folder global.sty (or TEXINPUTPATH..)}   % logging
        \usepackage{global}
    }{
        \typeout{Info: No folder/global.sty}   % logging
    }   % load global.sty if present
    \IfFileExists{\currfilebase.sty}
    {
        \typeout{Loading \currfilebase.sty}
        \input{\currfilebase.sty}
    }{
        \typeout{Info: No local \currfilebase.sty}
    }
    \fi
  \makeatother
\fi
}

%
%  
% references: Ximera heeft adhoc logica	 om online labels te doen werken over verschillende files heen
% met \hyperref kan de getoonde tekst toch worden opgegeven, in plaats van af te hangen van de label-text
\ifdefined\HCode
% Link to standard \labels, but give your own description
% Usage:  Volg \hyperref[my_very_verbose_label]{deze link} voor wat tijdverlies
%   (01/2020: Ximera-server aangepast om bij class reference-keeptext de link-text NIET te vervangen door de label-text !!!) 
\renewcommand{\hyperref}[2][]{\HCode{<a class="reference reference-keeptext" href="\##1">}#2\HCode{</a>}}
%
%  Link to specific targets  (not tested ?)
\renewcommand{\hypertarget}[1]{\HCode{<a class="ximera-label" id="#1"></a>}}
\renewcommand{\hyperlink}[2]{\HCode{<a class="reference reference-keeptext" href="\##1">}#2\HCode{</a>}}
\fi

% Mmm, quid English ... (for keyword #1 !) ?
\newcommand{\wikilink}[2]{
    \hyperlink{https://nl.wikipedia.org/wiki/#1}{#2}
    \pdfOnly{\footnote{See \url{https://nl.wikipedia.org/wiki/#1}}
    }
}

\renewcommand{\figurename}{Figuur}
\renewcommand{\tablename}{Tabel}

%
% Gedoe om verschillende versies van xourse/ximera te maken afhankelijk van settings
%
% default: versie met antwoorden
% handout: versie voor de studenten, zonder antwoorden/oplossingen
% full: met alles erop en eraan, dus geschikt voor auteurs en/of lesgevers  (bevat in de pdf ook de 'online-only' stukken!)
%
%
% verder kunnen ook opties/variabele worden gezet voor hints/auteurs/uitweidingen/ etc
%
% 'Full' versie
\newtoggle{showonline}
\ifdefined\HCode   % zet default showOnline
    \toggletrue{showonline} 
\else
    \togglefalse{showonline}
\fi

% Full versie   % deprecated: see infra
\newcommand{\printFull}{
    \hintstrue
    \handoutfalse
    \toggletrue{showonline} 
}

\ifdefined\shouldPrintFull   % deprecated: see infra
    \printFull
\fi



% Overschrijf onlineOnly  (zoals gedefinieerd in ximera.cls)
\ifhandout   % in handout: gebruik de oorspronkelijke ximera.cls implementatie  (is dit wel nodig/nuttig?)
\else
    \iftoggle{showonline}{%
        \ifdefined\HCode
          \RenewEnviron{onlineOnly}{\bgroup\BODY\egroup}   % showOnline, en we zijn  online, dus toon de tekst
        \else
          \RenewEnviron{onlineOnly}{\bgroup\color{red!50!black}\BODY\egroup}   % showOnline, maar we zijn toch niet online: kleur de tekst rood 
        \fi
    }{%
      \RenewEnviron{onlineOnly}{}  % geen showOnline
    }
\fi

% hack om na hoofding van definition/proposition/... als dan niet op een nieuwe lijn te starten
% soms is dat goed en mooi, en soms niet; en in HTML is het nu (2/2020) anders dan in pdf
% vandaar suggestie om 
%     \begin{definition}[Nieuw concept] \nl
% te gebruiken als je zeker een newline wil na de hoofdig en titel
% (in het bijzonder itemize zonder \nl is 'lelijk' ...)
\ifdefined\HCode
\newcommand{\nl}{}
\else
\newcommand{\nl}{\ \par} % newline (achter heading van definition etc.)
\fi


% \nl enkel in handoutmode (ihb voor \wordChoice, die dan typisch veeeel langer wordt)
\ifdefined\HCode
\providecommand{\handoutnl}{}
\else
\providecommand{\handoutnl}{%
\ifhandout%
  \nl%
\fi%
}
\fi

% Could potentially replace \pdfOnline/\begin{onlineOnly} : 
% Usage= \ifonline{Hallo surfer}{Hallo PDFlezer}
\providecommand{\ifonline}[2]%
{
\begin{onlineOnly}#1\end{onlineOnly}%
\pdfOnly{#2}
}%


%
% Maak optionele 'basic' en 'extended' versies van een activity
%  met environment basicOnly, basicSkip en extendedOnly
%
%  (
%   Dit werkt ENKEL in de PDF; de online versies tonen (minstens voorklopig) steeds 
%   het default geval met printbasicversion en printextendversion beide FALSE
%  )
%
\providebool{printbasicversion}
\providebool{printextendedversion}   % not properly implemented
\providebool{printfullversion}       % presumably print everything (debug/auteur)
%
% only set these in xourses, and BEFORE loading this preamble
%
%\newif\ifshowbasic     \showbasictrue        % use this line in xourse to show 'basic' sections
%\newif\ifshowextended  \showextendedtrue     % use this line in xourse to show 'extended' sections
%
%
%\ifbool{showbasic}
%      { \NewEnviron{basicOnly}{\BODY} }    % if yes: just print contents
%      { \NewEnviron{basicOnly}{}      }    % if no:  completely ignore contents
%
%\ifbool{showbasic}
%      { \NewEnviron{basicSkip}{}      }
%      { \NewEnviron{basicSkip}{\BODY} }
%

\ifbool{printextendedversion}
      { \NewEnviron{extendedOnly}{\BODY} }
      { \NewEnviron{extendedOnly}{}      }
      


\ifdefined\HCode    % in html: always print
      {\newenvironment*{basicOnly}{}{}}    % if yes: just print contents
      {\newenvironment*{basicSkip}{}{}}    % if yes: just print contents
\else

\ifbool{printbasicversion}
      {\newenvironment*{basicOnly}{}{}}    % if yes: just print contents
      {\NewEnviron{basicOnly}{}      }    % if no:  completely ignore contents

\ifbool{printbasicversion}
      {\NewEnviron{basicSkip}{}      }
      {\newenvironment*{basicSkip}{}{}}

\fi

\usepackage{float}
\usepackage[rightbars,color]{changebar}

% Full versie
\ifbool{printfullversion}{
    \hintstrue
    \handoutfalse
    \toggletrue{showonline}
    \printbasicversionfalse
    \cbcolor{red}
    \renewenvironment*{basicOnly}{\cbstart}{\cbend}
    \renewenvironment*{basicSkip}{\cbstart}{\cbend}
    \def\xmtoonprintopties{FULL}   % will be printed in footer
}
{}
      
%
% Evalueer \ifhints IN de environment
%  
%
%\RenewEnviron{hint}
%{
%\ifhandout
%\ifhints\else\setbox0\vbox\fi%everything in een emty box
%\bgroup 
%\stepcounter{hintLevel}
%\BODY
%\egroup\ignorespacesafterend
%\addtocounter{hintLevel}{-1}
%\else
%\ifhints
%\begin{trivlist}\item[\hskip \labelsep\small\slshape\bfseries Hint:\hspace{2ex}]
%\small\slshape
%\stepcounter{hintLevel}
%\BODY
%\end{trivlist}
%\addtocounter{hintLevel}{-1}
%\fi
%\fi
%}

% Onafhankelijk van \ifhandout ...? TO BE VERIFIED
\RenewEnviron{hint}
{
\ifhints
\begin{trivlist}\item[\hskip \labelsep\small\bfseries Hint:\hspace{2ex}]
\small%\slshape
\stepcounter{hintLevel}
\BODY
\end{trivlist}
\addtocounter{hintLevel}{-1}
\else
\iftikzexport   % anders worden de tikz tekeningen in hints niet gegenereerd ?
\setbox0\vbox\bgroup
\stepcounter{hintLevel}
\BODY
\egroup\ignorespacesafterend
\addtocounter{hintLevel}{-1}
\fi % ifhandout
\fi %ifhints
}

%
% \tab sets typewriter-tabs (e.g. to format questions)
% (Has no effect in HTML :-( ))
%
\usepackage{tabto}
\ifdefined\HCode
  \renewcommand{\tab}{\quad}    % otherwise dummy .png's are generated ...?
\fi


% Also redefined in  preamble to get correct styling 
% for tikz images for (\tikzexport)
%

\theoremstyle{definition} % Bold titels
\makeatletter
\let\proposition\relax
\let\c@proposition\relax
\let\endproposition\relax
\makeatother
\newtheorem{proposition}{Eigenschap}


%\instructornotesfalse

% logic with \ifhandoutin ximera.cls unclear;so overwrite ...
\makeatletter
\@ifundefined{ifinstructornotes}{%
  \newif\ifinstructornotes
  \instructornotesfalse
  \newenvironment{instructorNotes}{}{}
}{}
\makeatother
\ifinstructornotes
\else
\renewenvironment{instructorNotes}%
{%
    \setbox0\vbox\bgroup
}
{%
    \egroup
}
\fi

% \RedeclareMathOperator
% from https://tex.stackexchange.com/questions/175251/how-to-redefine-a-command-using-declaremathoperator
\makeatletter
\newcommand\RedeclareMathOperator{%
    \@ifstar{\def\rmo@s{m}\rmo@redeclare}{\def\rmo@s{o}\rmo@redeclare}%
}
% this is taken from \renew@command
\newcommand\rmo@redeclare[2]{%
    \begingroup \escapechar\m@ne\xdef\@gtempa{{\string#1}}\endgroup
    \expandafter\@ifundefined\@gtempa
    {\@latex@error{\noexpand#1undefined}\@ehc}%
    \relax
    \expandafter\rmo@declmathop\rmo@s{#1}{#2}}
% This is just \@declmathop without \@ifdefinable
\newcommand\rmo@declmathop[3]{%
    \DeclareRobustCommand{#2}{\qopname\newmcodes@#1{#3}}%
}
\@onlypreamble\RedeclareMathOperator
\makeatother


%
% Engelse vertaling, vooral in mathmode
%
% 1. Algemene procedure
%
\ifdefined\isEn
 \newcommand{\nlen}[2]{#2}
 \newcommand{\nlentext}[2]{\text{#2}}
 \newcommand{\nlentextbf}[2]{\textbf{#2}}
\else
 \newcommand{\nlen}[2]{#1}
 \newcommand{\nlentext}[2]{\text{#1}}
 \newcommand{\nlentextbf}[2]{\textbf{#1}}
\fi

%
% 2. Lijst van erg veel gebruikte uitdrukkingen
%

% Ja/Nee/Fout/Juits etc
%\newcommand{\TJa}{\nlentext{ Ja }{ and }}
%\newcommand{\TNee}{\nlentext{ Nee }{ No }}
%\newcommand{\TJuist}{\nlentext{ Juist }{ Correct }
%\newcommand{\TFout}{\nlentext{ Fout }{ Wrong }
\newcommand{\TWaar}{\nlentext{ Waar }{ True }}
\newcommand{\TOnwaar}{\nlentext{ Vals }{ False }}
% Korte bindwoorden en, of, dus, ...
\newcommand{\Ten}{\nlentext{ en }{ and }}
\newcommand{\Tof}{\nlentext{ of }{ or }}
\newcommand{\Tdus}{\nlentext{ dus }{ so }}
\newcommand{\Tendus}{\nlentext{ en dus }{ and thus }}
\newcommand{\Tvooralle}{\nlentext{ voor alle }{ for all }}
\newcommand{\Took}{\nlentext{ ook }{ also }}
\newcommand{\Tals}{\nlentext{ als }{ when }} %of if?
\newcommand{\Twant}{\nlentext{ want }{ as }}
\newcommand{\Tmaal}{\nlentext{ maal }{ times }}
\newcommand{\Toptellen}{\nlentext{ optellen }{ add }}
\newcommand{\Tde}{\nlentext{ de }{ the }}
\newcommand{\Thet}{\nlentext{ het }{ the }}
\newcommand{\Tis}{\nlentext{ is }{ is }} %zodat is in text staat in mathmode (geen italics)
\newcommand{\Tmet}{\nlentext{ met }{ where }} % in situaties e.g met p < n --> where p < n
\newcommand{\Tnooit}{\nlentext{ nooit }{ never }}
\newcommand{\Tmaar}{\nlentext{ maar }{ but }}
\newcommand{\Tniet}{\nlentext{ niet }{ not }}
\newcommand{\Tuit}{\nlentext{ uit }{ from }}
\newcommand{\Ttov}{\nlentext{ t.o.v. }{ w.r.t. }}
\newcommand{\Tzodat}{\nlentext{ zodat }{ such that }}
\newcommand{\Tdeth}{\nlentext{de }{th }}
\newcommand{\Tomdat}{\nlentext{omdat }{because }} 


%
% Overschrijf addhoc commando's
%
\ifdefined\isEn
\renewcommand{\pernot}{\overset{\mathrm{notation}}{=}}
\RedeclareMathOperator{\bld}{im}     % beeld
\RedeclareMathOperator{\graf}{graph}   % grafiek
\RedeclareMathOperator{\rico}{slope}   % richtingcoëfficient
\RedeclareMathOperator{\co}{co}       % coordinaat
\RedeclareMathOperator{\gr}{deg}       % graad

% Operators
\RedeclareMathOperator{\bgsin}{arcsin}
\RedeclareMathOperator{\bgcos}{arccos}
\RedeclareMathOperator{\bgtan}{arctan}
\RedeclareMathOperator{\bgcot}{arccot}
\RedeclareMathOperator{\bgsinh}{arcsinh}
\RedeclareMathOperator{\bgcosh}{arccosh}
\RedeclareMathOperator{\bgtanh}{arctanh}
\RedeclareMathOperator{\bgcoth}{arccoth}

\fi


% HACK: use 'oplossing' for 'explanation' ...
\let\explanation\relax
\let\endexplanation\relax
% \newenvironment{explanation}{\begin{oplossing}}{\end{oplossing}}
\newcounter{explanation}

\ifhandout%
    \NewEnviron{explanation}[1][toon]%
    {%
    \RenewEnviron{verbatim}{ \red{VERBATIM CONTENT MISSING IN THIS PDF}} %% \expandafter\verb|\BODY|}

    \ifthenelse{\equal{\detokenize{#1}}{\detokenize{toon}}}
    {
    \def\PH@Command{#1}% Use PH@Command to hold the content and be a target for "\expandafter" to expand once.

    \begin{trivlist}% Begin the trivlist to use formating of the "Feedback" label.
    \item[\hskip \labelsep\small\slshape\bfseries Explanation:% Format the "Feedback" label. Don't forget the space.
    %(\texttt{\detokenize\expandafter{\PH@Command}}):% Format (and detokenize) the condition for feedback to trigger
    \hspace{2ex}]\small%\slshape% Insert some space before the actual feedback given.
    \BODY
    \end{trivlist}
    }
    {  % \begin{feedback}[solution]   \BODY     \end{feedback}  }
    }
    }    
\else
% ONLY for HTML; xmoplossing is styled with css, and is not, and need not be a LaTeX environment
% THUS: it does NOT use feedback anymore ...
%    \NewEnviron{oplossing}{\begin{expandable}{xmoplossing}{\nlen{Toon uitwerking}{Show solution}}{\BODY}\end{expandable}}
    \newenvironment{explanation}[1][toon]
   {%
       \begin{expandable}{xmoplossing}{}
   }
   {%
   	   \end{expandable}
   } 
\fi

 \title{Finding the Determinant} \license{CC BY-NC-SA 4.0}

\begin{document}

\begin{abstract}

\end{abstract}
\maketitle

\begin{onlineOnly}
\section*{Finding the Determinant}
\end{onlineOnly}

In this section we will define a function that assigns to each square matrix $A$ a scalar output called the \dfn{determinant of $A$}.  We will denote the determinant of $A$ by $\det{A}$.  For a matrix with real number entries, the output of the determinant function will always be a real number.

One important property of the determinant is its connection to matrix inverses.  We will find that a matrix $A$ is singular if and only if $\det{A}=0$.  For nonsingular matrices, we will establish a formula that gives the inverse of a matrix exclusively in terms of determinants.  This property will be addressed in detail in \href{https://ximera.osu.edu/oerlinalg/LinearAlgebra/DET-0040/main}{Poperties of the Determinant}, and \href{https://ximera.osu.edu/oerlinalg/LinearAlgebra/DET-0060/main}{Determinants and Inverses of Nonsingular Matrices}.

Geometrically speaking, the determinant of a matrix of a linear transformation is the factor by which the area (or volume or hypervolume) is scaled by the transformation.  This will be discussed in \href{https://ximera.osu.edu/oerlinalg/LinearAlgebra/DET-0070/main}{Determinants as Areas and Volumes}. 

\subsection*{Cofactor Expansion Along the Top Row}
To start from the beginning, let us define the determinant of a $1\times 1$ matrix.

\begin{definition}\label{def:onebyonedet} Let
$A=\begin{bmatrix}a\end{bmatrix}$.  Define the \dfn{determinant} of $A$ by $\det{A}=a$.
\end{definition}
It is important to note that this definition is consistent with our goal of making a connection between determinants and invertibility.  Observe that $A^{-1}=\begin{bmatrix}a^{-1}\end{bmatrix}$ exists if and only if $a\neq 0$.

Now we proceed to $2\times 2$ matrices.  According to Formula \ref{form:detinverse}, the inverse of a nonsingular matrix $A=\begin{bmatrix}a&b\\c&d\end{bmatrix}$ is given by
$$A^{-1}=\frac{1}{ad-bc}\begin{bmatrix}d&-b\\-c&a\end{bmatrix}$$
Observe that $A^{-1}$ exists if and only if $ad-bc\neq 0$.  We will call the number $ad-bc$ the determinant of $A$.

\begin{definition}\label{def:twobytwodet}
Let $A=\begin{bmatrix}a&b\\c&d\end{bmatrix}$.  The determinant of $A$ is defined by 
\begin{equation}\label{eq:twobytwodet}\det{A}=\det{\begin{bmatrix}a&b\\c&d\end{bmatrix}}=\begin{vmatrix}a&b\\c&d\end{vmatrix}=ad-bc\end{equation}
\end{definition}
Note the distinction between the square bracket notation associated with the matrix $\begin{bmatrix}a&b\\c&d\end{bmatrix}$ and the vertical bar notation $\begin{vmatrix}a&b\\c&d\end{vmatrix}$ used to denote  the determinant in expression (\ref{eq:twobytwodet}).

\begin{example}\label{ex:2x2det}
$$\det{\begin{bmatrix}1&2\\3&4\end{bmatrix}}=\begin{vmatrix}1&2\\3&4\end{vmatrix}=(1)(4)-(2)(3)=-2$$
\end{example}

The easiest way to understand the definition of the determinant for a $3\times 3$ matrix is to start with an example.

\begin{example}\label{ex:threebythreedet1}
Find $\det{A}$ if 
$$A=\begin{bmatrix}3&-2&1\\5&-1&2\\1&4&1\end{bmatrix}$$
\begin{explanation}
\begin{align*}
\det{A}&=(3)\begin{vmatrix}-1&2\\4&1\end{vmatrix}-(-2)\begin{vmatrix}5&2\\1&1\end{vmatrix}+(1)\begin{vmatrix}5&-1\\1&4\end{vmatrix}\\
&=(3)(-1-8)-(-2)(5-2)+(1)(20+1)\\
&=-27+6+21\\
&=0
\end{align*}

\youtube{aR2w-viFcvI}
\end{explanation}
\end{example}

We now formalize what we learned in Example \ref{ex:threebythreedet1}.

\begin{definition}\label{def:threebythreedet}
Let
$$A=\begin{bmatrix}a&b&c\\d&e&f\\g&h&i\end{bmatrix}$$
The \dfn{determinant} of $A$ is given by
\begin{align}\label{eq:det3by3}
\det{A}=|A|&=a\begin{vmatrix}e&f\\h&i\end{vmatrix}-b\begin{vmatrix}d&f\\g&i\end{vmatrix}+c\begin{vmatrix}d&e\\g&h\end{vmatrix}
\end{align}
\end{definition}

We will now reiterate several important features of this definition and introduce some vocabulary:
\begin{itemize}
\item The coefficients $a$, $b$ and $c$ are the entries of the first row of matrix $A$.  Coefficients in the formula follow an alternating sign pattern: $+a$, $-b$, $+c$.  This pattern will persist in the determinant formulas for determinants of larger matrices.
\item When using equation (\ref{eq:det3by3}), we compute determinants of three matrices:
$$\begin{bmatrix}e&f\\h&i\end{bmatrix},\quad \begin{bmatrix}d&f\\g&i\end{bmatrix},\quad \begin{bmatrix}d&e\\g&h\end{bmatrix}$$
These matrices are called \dfn{minor} matrices.  To form each minor matrix,
cross out the row and column that the corresponding coefficient is in.  For example, the minor matrix corresponding to coefficient $b$ is found by crossing out the row and column that $b$ is in.
\begin{center}
\begin{tikzpicture}
  \matrix (m)[
    matrix of math nodes,
    nodes in empty cells,
    left delimiter={[},right delimiter={]},minimum width=width("a"),minimum height=height("b")] {
    a    & {\color{red}b}  & c  \\
    d & e   & f   \\
    g   & h    & i     \\
  } ;

  %\draw (m-3-2.south west) rectangle (m-2-3.north east);
  \draw[blue](m-1-1.west) -- (m-1-1.east);
   \draw[blue](m-1-3.west) -- (m-1-3.east);
  \draw[blue](m-2-2.north) -- (m-3-2.south);
 \end{tikzpicture}
 \end{center} 
 \item The process for finding the determinant described in Definition \ref{def:threebythreedet} is referred to as a \dfn{cofactor expansion along the top row}. % we will explain this term in more detail in \href{https://ximera.osu.edu/oerlinalg/LinearAlgebra/DET-0050/main}{Proof of the Laplace Expansion Theorem}.
\end{itemize}


\begin{example}\label{ex:3x3det2}
Find $\det{A}$ if 
$$A=\begin{bmatrix}4&3&-2\\1&-5&3\\-4&1&1\end{bmatrix}$$
\begin{explanation}
$$
\det{A}=(4)\begin{vmatrix}\answer{-5}&\answer{3}\\\answer{1}&\answer{1}\end{vmatrix}-(3)\begin{vmatrix}\answer{1}&\answer{3}\\\answer{-4}&\answer{1}\end{vmatrix}+(-2)\begin{vmatrix}\answer{1}&\answer{-5}\\\answer{-4}&\answer{1}\end{vmatrix}
=\answer{-33}
$$
\end{explanation}
\end{example}

We are starting to observe a certain pattern in the process of computing the determinant.  This pattern will persist for larger matrices.  Let's take a look at a $4\times 4$ matrix.

\begin{example}\label{ex:4by4withVideo}
    Find $\det{A}$ if 
    $$A=\begin{bmatrix} 2 & 3 & -2 & -5\\0 & 1& -2& 0\\1& 3 & 0 &-1\\2&0& 1&1\end{bmatrix}$$

\begin{explanation}
As you watch the video below, pay particular attention to the same patterns as you saw in the case of $3\times 3$ matrices: the alternating sign pattern and the process of forming minor matrices.

\begin{align*}
\det{A}
&=2\begin{vmatrix}1& -2& 0\\3 & 0 &-1\\0& 1&1\end{vmatrix}-3\begin{vmatrix}0 & -2& 0\\1 & 0 &-1\\2& 1&1\end{vmatrix}-2\begin{vmatrix}0 & 1& 0\\1& 3 &-1\\2&0&1\end{vmatrix}-(-5)\begin{vmatrix}0 & 1& -2\\1& 3 & 0 \\2&0& 1\end{vmatrix}\\
&=2(7)-3(6)-2(-3)+5(11)\\
&=57
\end{align*}

\youtube{YIJq7TqncyU}
\end{explanation}
\end{example}

\begin{example}\label{ex:expansiontoprow}
Find $\det{A}$ if 
$$A=\begin{bmatrix}4&-1&2&1\\3&0&1&-2\\
2&1&5&1\\-2&1&3&-1\end{bmatrix}$$
\begin{explanation}
We will use the entries in the top row as coefficients in front of $3\times 3$ determinants.  As before, we will use the alternating sign pattern for the coefficients:
$$+(4), -(-1), +(2), -(1)$$
Just like in the case of a $3 \times 3$ matrix in Example \ref{ex:threebythreedet1}, each of the smaller determinants is obtained by crossing out the row and the column where the coefficient is located.
\begin{align*}
\det{A}
&=4\begin{vmatrix}0&1&-2\\1&5&1\\1&3&-1\end{vmatrix}+\begin{vmatrix}3&1&-2\\2&5&1\\-2&3&-1\end{vmatrix}+2\begin{vmatrix}3&0&-2\\2&1&1\\-2&1&-1\end{vmatrix}-\begin{vmatrix}3&0&1\\2&1&5\\-2&1&3\end{vmatrix}\\&=4(\answer{6})+(\answer{-56})+2(\answer{-14})-(\answer{-2})\\
&=\answer{-58}
\end{align*}
\end{explanation}
\end{example}

\subsection*{Cofactor Expansion Along the First Column}
We defined the determinant of a matrix in terms of cofactor expansion along the top row.  We will now see what happens when we expand along the first \textit{column} instead.  We will refer to this process as \dfn{cofactor expansion along the first column}.  Surprisingly, both expansions yield the same result.  To illustrate this, let's revisit Examples \ref{ex:threebythreedet1} and \ref{ex:expansiontoprow}. 

\begin{example}\label{init:expansionfirstcol1}
Let 
$$A=\begin{bmatrix}3&-2&1\\5&-1&2\\1&4&1\end{bmatrix}$$
In Example \ref{ex:threebythreedet1} we found that $\det{A}=0$.  Let's try to mimic what we did earlier, but instead of expanding along the first row, we will expand along the fist column.  
\begin{align*}
&(3)\begin{vmatrix}-1&2\\4&1\end{vmatrix}-(5)\begin{vmatrix}-2&1\\4&1\end{vmatrix}+(1)\begin{vmatrix}-2&1\\-1&2\end{vmatrix}\\
&=(3)(-1-8)-(5)(-2-4)+(1)(-4+1)\\
&=-27+30-3\\
&=0\\
&=\det{A}
\end{align*}
\end{example}

Let's go through this process again for a larger matrix.

\begin{exploration}\label{init:expansionfirstcol2}
Let
$$A=\begin{bmatrix}4&-1&2&1\\3&0&1&-2\\
2&1&5&1\\-2&1&3&-1\end{bmatrix}$$
In Example \ref{ex:expansiontoprow} we found that $\det{A}=-58$.  We will now try to expand along the fist column.  

When computing determinants of the four $3\times 3$ matrices below, try different approaches.  You might want to expand along the first row for some of them, and along the first column for others.  Looking for where zeros are located will help you decide what to try.
\begin{align*}
&4\begin{vmatrix}0&1&-2\\1&5&1\\1&3&-1\end{vmatrix}-3\begin{vmatrix}-1&2&1\\1&5&1\\1&3&-1\end{vmatrix}+2\begin{vmatrix}-1&2&1\\0&1&-2\\1&3&-1\end{vmatrix}-(-2)\begin{vmatrix}-1&2&1\\0&1&-2\\1&5&1\end{vmatrix}\\
&=4(\answer{6})-3(\answer{10})+2(\answer{-10})+2(\answer{-16})\\
&=\answer{-58}\\
&=\det{A}
\end{align*}
\end{exploration}

In Example \ref{init:expansionfirstcol1} and Exploration \ref{init:expansionfirstcol2} we were careful not to claim at the outset that we were finding the determinant of the matrix by cofactor expansion along the first column; we merely observed that the resulting value was equal to the determinant.  It is possible to prove that both expansions produce the same result (Theorem \ref{th:rowcolexpequivalence}).  Therefore the determinant of a matrix can be defined in terms of cofactor expansion along the first row or column.  

\subsection*{Cofactor Expansion Along Any Row or Column}

We originally defined the determinant of a matrix via expansion along the top row of the matrix.  We later observed that expansion along the first column produces the same result.  It turns out that the value of the determinant can be computed by expanding along any row or column.  This result is known as the \dfn{Laplace Expansion Theorem} (\ref{th:laplace1}).

When expanding along an arbitrary row or column, we will continue to follow the two patterns we observed earlier.
\begin{itemize}
    \item The alternating sign pattern for coefficients will follow the checkerboard pattern below.
 $$\begin{bmatrix}+&-&+&-&+&\ldots\\-&+&-&+&-&\ldots\\
 +&-&+&-&+&\ldots\\-&+&-&+&-&\ldots\\\vdots &\vdots  & \vdots & \vdots &\vdots &\ddots \end{bmatrix}$$
 \item Minor matrices will be formed by eliminating the row and column that the corresponding coefficient is in.
\end{itemize}

To illustrate this, let's take another look at matrix $A$ from Example \ref{ex:expansiontoprow}. 


 \begin{example}\label{ex:laplace1}
Let  
$$A=\begin{bmatrix}4&-1&2&1\\3&0&1&-2\\
2&1&5&1\\-2&1&3&-1\end{bmatrix}$$
Follow the rules described above to expand along the second row.  Compare your result with the determinant you found in Example \ref{ex:expansiontoprow}.
\begin{explanation}

 The second row has the advantage over other rows in that it contains a zero.  This will simplify our calculations.  Following the checkerboard sign pattern along the second row we get

\begin{align*}
\det{A}
&=-3\begin{vmatrix}-1&2&1\\1&5&1\\1&3&-1\end{vmatrix}-\begin{vmatrix}4&-1&1\\2&1&1\\-2&1&-1\end{vmatrix}+(-2)\begin{vmatrix}4&-1&2\\2&1&5\\-2&1&3\end{vmatrix}\\&=-3(\answer{10})-(\answer{-4})-2(\answer{16})\\
&=\answer{-58}
\end{align*}
This answer is the same as the answer we got using expansion along the first row in Example \ref{ex:expansiontoprow}.
    
\end{explanation}
 \end{example}
 
It is clear that having zeros as entries in the matrix reduces the number of computations necessary to find the determinant.  The following example demonstrates how to use zeros to our advantage.

\begin{example}\label{ex:laplace2}
Find $\det{A}$ if
$$A=\begin{bmatrix}4&0&0&0&2\\0&-1&1&0&0\\2&0&0&-5&3\\0&1&4&0&-1\\1&1&5&0&0\end{bmatrix}$$

\begin{explanation}
The fourth column contains the most zeros, so we will expand along that column.  The  $(3, 4)$-entry is the only non-zero entry in the fourth column.  Following the checkerboard pattern, we see that the sign in front of $-5$ is a minus.  
$$\det{A}=-(-5)\begin{vmatrix}4&0&0&2\\0&-1&1&0\\0&1&4&-1\\1&1&5&0\end{vmatrix}
$$
Next we will expand the minor matrix along the top row.
$$\det{A}=5\left(4\begin{vmatrix}-1&1&0\\1&4&-1\\1&5&0\end{vmatrix}-2\begin{vmatrix}0&-1&1\\0&1&4&\\1&1&5\end{vmatrix}\right)$$
Try the next step on your own.  We suggest that you expand the first matrix along the last column and expand the second matrix along the first column.
$$\det(A)=5\big(4(\answer{-6})-2(\answer{-5})\big)=\answer{-70}$$
\end{explanation}
\end{example}

\subsection*{A Note on Equivalency}
We initially introduced the determinant of a matrix via cofactor expansion along the top row.  We later observed that cofactor expansion along any row or column produces the same result.  We have to be careful, however, not to use a few examples as ``proof" that all cofactor expansions are equivalent.  Such claims need to be carefully supported with general proofs.  Unfortunately, in this case, the proofs are tedious and conceptually unenlightening.  An interested reader can find them in \href{https://ximera.osu.edu/oerlinalg/LinearAlgebra/DET-0050/main}{Tedious Proofs Concerning Determinants}.

\subsection*{Determinants of Some Special Matrices}
We know that we can find the determinant of a matrix by cofactor expansion along the top row or the first column.  (See Theorem \ref{th:rowcolexpequivalence} of \href{https://ximera.osu.edu/oerlinalg/LinearAlgebra/DET-0050/main}{Tedious Proofs Concerning Determinants} for proof.)  This property gives rise to a useful result.
\begin{theorem}\label{th:detoftrans}
Let $A$ be a square matrix, then
$$\det{A^T}=\det{A}$$
\end{theorem}
\begin{proof}
    See Practice Problem \ref{prob:detOfTrans}.
\end{proof}

As we observed earlier, having zeros in a matrix makes it easier for us to compute its determinant.  Recall that that a square matrix is \dfn{upper-triangular} if all of the entries below the main diagonal are zero.  Similarly, a square matrix is called \dfn{lower-triangular} if all of the entries above the main diagonal are zero.  Together, upper and lower triangular matrices are categorized as \dfn{triangular} matrices.

\begin{theorem}\label{lemma:triangulardet}
If $A$ is a triangular matrix, then $\det{A}$ is equal to the product of its diagonal entries.
\end{theorem}
\begin{proof}
We proceed by induction on $n$, where $A$ is an $n\times n$ matrix.  It is easy to see that this result holds for $n=1, 2$.  Suppose that the result holds for $(n-1)\times (n-1)$ triangular matrices.  We need to show that it holds for $n\times n$ triangular matrices.  

Suppose $A=[a_{ij}]$ is a triangular matrix. Then, with the exception of $a_{11}$, the entries in the first row (or column) of $A$ are guaranteed to be zeros.
  We will take advantage of these zeros and expand along the first row (or column) of $A$.  As we do so, we obtain a single product of $a_{11}$ and the determinant of a minor matrix obtained by crossing out the first row and column of $A$.  But this minor $(n-1)\times (n-1)$ matrix is also a triangular matrix with diagonal etries $a_{22}, a_{33},\ldots, a_{nn}$.  By induction hypothesis, its determinant is equal to the product of its diagonal entries, $a_{22}\cdot a_{33}\cdot\ldots\cdot a_{nn}$.  Therefore $$\det{A}=a_{11}(a_{22}\cdot a_{33}\cdot\ldots\cdot a_{nn})$$
 This completes the proof.
\end{proof}

As an immediate consequence of this theorem, we have the following result.

\begin{corollary}\label{lemma:detofid} Let $I$ be the identity matrix, then
 $$\det{I}=1$$
 \end{corollary}

 We first introduced block matrices in \href{https://ximera.osu.edu/oerlinalg/LinearAlgebra/MAT-0023/main}{Block Matrix Multiplication}.  Matrices of the form $\begin{bmatrix}A & C\\O& B\end{bmatrix}$ and $\begin{bmatrix}A & O\\D& B\end{bmatrix}$, where $A$, $B$ are square matrices and $O$ is the zero matrix, are said to be \dfn{block triangular}.  The following theorem makes it easy to compute determinants of such matrices.

 \begin{theorem}\label{th:blockTriDet}
     Consider block triangular matrices $\begin{bmatrix}A & C\\O& B\end{bmatrix}$ and $\begin{bmatrix}A & O\\D& B\end{bmatrix}$, where $A$ and $B$ are square matrices.  Then
     $$\det{\begin{bmatrix}A & C\\O& B\end{bmatrix}}=\det{A}\det{B}\quad\mbox{and}\quad\det{\begin{bmatrix}A & O\\D& B\end{bmatrix}}=\det{A}\det{B}$$
 \end{theorem}
\begin{proof}
    Write $T=\begin{bmatrix}A & C\\O& B\end{bmatrix}$ and proceed by induction on $k$, where $A$ is $k\times k$.  If $k=1$, then the result follows from cofactor expansion along the first column.  In general, let $S_i(T)$ denote the matrix obtained from $T$ by deleting row $i$ and column 1. Then the cofactor expansion along the first column is
    $$\det{T}=a_{11}\det{S_1(T)}-a_{21}\det{S_2(T)}+\dots + (-1)^{k+1}\det{S_k(T)}$$
    where $a_{11}, a_{21},\dots , a_{k1}$ are the entries in the first column of $A$.  Observe that 
    $$S_i(T)=\begin{bmatrix}S_i(A) & C_i\\O& B\end{bmatrix}$$
    where $i=1,2,\dots , k$, $S_i(A)$ denotes matrix $A$ with column 1 and row $i$ deleted, and $C_i$ denotes matrix $C$ with with row $i$ deleted.
    
    Since $S_i(A)$ is a $(k-1)\times (k-1)$ matrix, by the induction hypothesis,
    $$\det{S_i(T)}=\det{S_i(A)}\cdot \det{B}$$
    This gives us
    \begin{align*}
        \det{T}=&a_{11}\det{S_1(T)}-a_{21}\det{S_2(T)}+\dots + (-1)^{k+1}a_{k1}\det{S_k(T)}\\
        =&a_{11}\det{S_1(A)}\cdot\det{B}-a_{21}\det{S_2(A)}\cdot\det{B}+\dots \\
        &\dots+ (-1)^{k+1}a_{k1}\det{S_k(A)}\cdot\det{B}\\
        =&\Big(a_{11}\det{S_1(A)}-a_{21}\det{S_2(A)}+\dots + (-1)^{k+1}a_{k1}\det{S_k(A)}\Big)\det{B}\\
        =&\det{A}\det{B}
    \end{align*}
    The lower triangular case is similar.
\end{proof}

\begin{example}\label{ex:blockTriDet}
    Find $\det{A}$ if 
    $$A=\begin{bmatrix}2&1&3&3\\1&-1&-2&1\\0&0&1&1\\0&0&4&1\end{bmatrix}$$
  \begin{explanation}
      $$\begin{vmatrix}2&1&3&3\\1&-1&-2&1\\0&0&1&1\\0&0&4&1\end{vmatrix}=\begin{vmatrix}2&1\\1&-1\end{vmatrix}\begin{vmatrix}1&1\\4&1\end{vmatrix}=\answer{9}$$
  \end{explanation}  
\end{example}


\section*{Practice Problems}

\emph{Problems \ref{prob:2x2det1}-\ref{prob:laplace}}
Find the determinant of each matrix.

  \begin{problem}\label{prob:2x2det1}
  $$A=\begin{bmatrix}4&-2\\3&7\end{bmatrix}$$
  Answer:
  $$\det{A}=\answer{34}$$
  \end{problem}
  
  \begin{problem}\label{prob:2x2det2}
  $$B=\begin{bmatrix}5&-1&0\\0&3&-2\\1&-1&2\end{bmatrix}$$
  Answer:
  $$\text{det}(B)=\answer{22}$$
  \end{problem}

  \begin{problem}\label{prob:laplace}
  $$C=\begin{bmatrix}1&-2&0&0&0\\0&-4&1&1&0\\3&0&-1&0&1\\0&0&4&1&0\\-1&-2&0&0&0\end{bmatrix}$$
   Answer:
  $$\det(C)=\answer{12}$$
 \end{problem}


\begin{problem}\label{prob:toprowexp2x2}
Show that Definition \ref{def:toprowexpansion} is consistent with Definition \ref{def:twobytwodet} by verifying that both produce the same result when applied to a $2\times 2$ matrix.
\end{problem}

\begin{problem}\label{prob:detOfTrans}
    Prove Theorem \ref{th:detoftrans}.
\end{problem}

\begin{problem}\label{prob:detrowswitch}
Let $B'$ be a matrix obtained from $B$ of Problem \ref{prob:2x2det2} by switching the first and the second row of $B$.  Compute the determinant of $B'$.  What do you observe?
\end{problem}

\begin{problem}\label{prob:2x2rowswitchproof}
Make a conjecture about what happens to the determinant of a matrix if two rows of a matrix are switched.  Prove your conjecture for a $2\times 2$ matrix.
\end{problem}

\begin{problem}\label{prob:scalarmultrowdet} Let $B'$ be a matrix obtained from $B$ of Problem \ref{prob:2x2det2} by multiplying the middle row by $-3$.  Compute the determinant of $B'$.  What do you observe?
\end{problem}

\begin{problem}\label{prob:rowtimesconstant2x2proof}
Make a conjecture about what happens to the determinant of a matrix if one of the rows is multiplied by a constant.  Prove your conjecture for a $2\times 2$ matrix.
\end{problem}

\begin{problem}\label{prob:matrixtimesconst}
Let $B'$ be a matrix obtained from $B$ of Problem \ref{prob:2x2det2} by multiplying $B$ by $2$.  Compute the determinant of $B'$.  What do you observe?
\end{problem}

\begin{problem}\label{prob:matrixtimesconstant2x2proof}
Make a conjecture about what happens to the determinant of a matrix if the matrix is multiplied by a constant.  Prove your conjecture for a $2\times 2$ matrix.
\end{problem}

\begin{problem}\label{prob:scalarmultofrow}
Let $B'$ be a matrix obtained from $B$ of Problem \ref{prob:2x2det2} by adding twice the third row to the first.  Compute the determinant of $B'$.  What do you observe?
\end{problem}

\begin{problem}\label{prob:scalarmultofrow2x2}
Make a conjecture about what happens to the determinant of a matrix if a multiple of one row is added to another row.  Prove your conjecture for a $2\times 2$ matrix.
\end{problem}

\begin{problem}\label{prob:detsumsumdetquestion}
Is it true that 
$\det{(A+B)}=\det{A}+\det{B}$?
\end{problem}

\section*{Text Source}
The text in this section partially comes from Section 3.1  of Keith Nicholson's \href{https://open.umn.edu/opentextbooks/textbooks/linear-algebra-with-applications}{\it Linear Algebra with Applications} (CC-BY-NC-SA).

\end{document} 