\documentclass{ximera}
%%% Begin Laad packages

\makeatletter
\@ifclassloaded{xourse}{%
    \typeout{Start loading preamble.tex (in a XOURSE)}%
    \def\isXourse{true}   % automatically defined; pre 112022 it had to be set 'manually' in a xourse
}{%
    \typeout{Start loading preamble.tex (NOT in a XOURSE)}%
}
\makeatother

\def\isEn\true 

\pgfplotsset{compat=1.16}

\usepackage{currfile}

% 201908/202301: PAS OP: babel en doclicense lijken problemen te veroorzaken in .jax bestand
% (wegens syntax error met toegevoegde \newcommands ...)
\pdfOnly{
    \usepackage[type={CC},modifier={by-nc-sa},version={4.0}]{doclicense}
    %\usepackage[hyperxmp=false,type={CC},modifier={by-nc-sa},version={4.0}]{doclicense}
    %%% \usepackage[dutch]{babel}
}



\usepackage[utf8]{inputenc}
\usepackage{morewrites}   % nav zomercursus (answer...?)
\usepackage{multirow}
\usepackage{multicol}
\usepackage{tikzsymbols}
\usepackage{ifthen}
%\usepackage{animate} BREAKS HTML STRUCTURE USED BY XIMERA
\usepackage{relsize}

\usepackage{eurosym}    % \euro  (€ werkt niet in xake ...?)
\usepackage{fontawesome} % smileys etc

% Nuttig als ook interactieve beamer slides worden voorzien:
\providecommand{\p}{} % default nothing ; potentially usefull for slides: redefine as \pause
%providecommand{\p}{\pause}

    % Layout-parameters voor het onderschrift bij figuren
\usepackage[margin=10pt,font=small,labelfont=bf, labelsep=endash,format=hang]{caption}
%\usepackage{caption} % captionof
%\usepackage{pdflscape}    % landscape environment

% Met "\newcommand\showtodonotes{}" kan je todonotes tonen (in pdf/online)
% 201908: online werkt het niet (goed)
\providecommand\showtodonotes{disable}
\providecommand\todo[1]{\typeout{TODO #1}}
%\usepackage[\showtodonotes]{todonotes}
%\usepackage{todonotes}

%
% Poging tot aanpassen layout
%
\usepackage{tcolorbox}
\tcbuselibrary{theorems}

%%% Einde laad packages

%%% Begin Ximera specifieke zaken

\graphicspath{
	{../../}
	{../}
	{./}
  	{../../pictures/}
   	{../pictures/}
   	{./pictures/}
	{./explog/}    % M05 in groeimodellen       
}

%%% Einde Ximera specifieke zaken

%
% define softer blue/red/green, use KU Leuven base colors for blue (and dark orange for red ?)
%
% todo: rather redefine blue/red/green ...?
%\definecolor{xmblue}{rgb}{0.01, 0.31, 0.59}
%\definecolor{xmred}{rgb}{0.89, 0.02, 0.17}
\definecolor{xmdarkblue}{rgb}{0.122, 0.671, 0.835}   % KU Leuven Blauw
\definecolor{xmblue}{rgb}{0.114, 0.553, 0.69}        % KU Leuven Blauw
\definecolor{xmgreen}{rgb}{0.13, 0.55, 0.13}         % No KULeuven variant for green found ...

\definecolor{xmaccent}{rgb}{0.867, 0.541, 0.18}      % KU Leuven Accent (orange ...)
\definecolor{kuaccent}{rgb}{0.867, 0.541, 0.18}      % KU Leuven Accent (orange ...)

\colorlet{xmred}{xmaccent!50!black}                  % Darker version of KU Leuven Accent

\providecommand{\blue}[1]{{\color{blue}#1}}    
\providecommand{\red}[1]{{\color{red}#1}}

\renewcommand\CancelColor{\color{xmaccent!50!black}}

% werkt in math en text mode om MATH met oranje (of grijze...)  achtergond te tonen (ook \important{\text{blabla}} lijkt te werken)
%\newcommand{\important}[1]{\ensuremath{\colorbox{xmaccent!50!white}{$#1$}}}   % werkt niet in Mathjax
%\newcommand{\important}[1]{\ensuremath{\colorbox{lightgray}{$#1$}}}
\newcommand{\important}[1]{\ensuremath{\colorbox{orange}{$#1$}}}   % TODO: kleur aanpassen voor mathjax; wordt overschreven infra!


% Uitzonderlijk kan met \pdfnl in de PDF een newline worden geforceerd, die online niet nodig/nuttig is omdat daar de regellengte hoe dan ook niet gekend is.
\ifdefined\HCode%
\providecommand{\pdfnl}{}%
\else%
\providecommand{\pdfnl}{%
  \\%
}%
\fi

% Uitzonderlijk kan met \handoutnl in de handout-PDF een newline worden geforceerd, die noch online noch in de PDF-met-antwoorden nuttig is.
\ifdefined\HCode
\providecommand{\handoutnl}{}
\else
\providecommand{\handoutnl}{%
\ifhandout%
  \nl%
\fi%
}
\fi



% \cellcolor IGNORED by tex4ht ?
% \begin{center} seems not to wordk
    % (missing margin-left: auto;   on tabular-inside-center ???)
%\newcommand{\importantcell}[1]{\ensuremath{\cellcolor{lightgray}#1}}  %  in tabular; usablility to be checked
\providecommand{\importantcell}[1]{\ensuremath{#1}}     % no mathjax2 support for colloring array cells

\pdfOnly{
  \renewcommand{\important}[1]{\ensuremath{\colorbox{kuaccent!50!white}{$#1$}}}
  \renewcommand{\importantcell}[1]{\ensuremath{\cellcolor{kuaccent!40!white}#1}}   
}

%%% Tikz styles


\pgfplotsset{compat=1.16}

\usetikzlibrary{trees,positioning,arrows,fit,shapes,math,calc,decorations.markings,through,intersections,patterns,matrix}

\usetikzlibrary{decorations.pathreplacing,backgrounds}    % 5/2023: from experimental


\usetikzlibrary{angles,quotes}

\usepgfplotslibrary{fillbetween} % bepaalde_integraal
\usepgfplotslibrary{polar}    % oa voor poolcoordinaten.tex

\pgfplotsset{ownstyle/.style={axis lines = center, axis equal image, xlabel = $x$, ylabel = $y$, enlargelimits}} 

\pgfplotsset{
	plot/.style={no marks,samples=50}
}

\newcommand{\xmPlotsColor}{
	\pgfplotsset{
		plot1/.style={darkgray,no marks,samples=100},
		plot2/.style={lightgray,no marks,samples=100},
		plotresult/.style={blue,no marks,samples=100},
		plotblue/.style={blue,no marks,samples=100},
		plotred/.style={red,no marks,samples=100},
		plotgreen/.style={green,no marks,samples=100},
		plotpurple/.style={purple,no marks,samples=100}
	}
}
\newcommand{\xmPlotsBlackWhite}{
	\pgfplotsset{
		plot1/.style={black,loosely dashed,no marks,samples=100},
		plot2/.style={black,loosely dotted,no marks,samples=100},
		plotresult/.style={black,no marks,samples=100},
		plotblue/.style={black,no marks,samples=100},
		plotred/.style={black,dotted,no marks,samples=100},
		plotgreen/.style={black,dashed,no marks,samples=100},
		plotpurple/.style={black,dashdotted,no marks,samples=100}
	}
}


\newcommand{\xmPlotsColorAndStyle}{
	\pgfplotsset{
		plot1/.style={darkgray,no marks,samples=100},
		plot2/.style={lightgray,no marks,samples=100},
		plotresult/.style={blue,no marks,samples=100},
		plotblue/.style={xmblue,no marks,samples=100},
		plotred/.style={xmred,dashed,thick,no marks,samples=100},
		plotgreen/.style={xmgreen,dotted,very thick,no marks,samples=100},
		plotpurple/.style={purple,no marks,samples=100}
	}
}


%\iftikzexport
\xmPlotsColorAndStyle
%\else
%\xmPlotsBlackWhite
%\fi
%%%


%
% Om venndiagrammen te arceren ...
%
\makeatletter
\pgfdeclarepatternformonly[\hatchdistance,\hatchthickness]{north east hatch}% name
{\pgfqpoint{-1pt}{-1pt}}% below left
{\pgfqpoint{\hatchdistance}{\hatchdistance}}% above right
{\pgfpoint{\hatchdistance-1pt}{\hatchdistance-1pt}}%
{
	\pgfsetcolor{\tikz@pattern@color}
	\pgfsetlinewidth{\hatchthickness}
	\pgfpathmoveto{\pgfqpoint{0pt}{0pt}}
	\pgfpathlineto{\pgfqpoint{\hatchdistance}{\hatchdistance}}
	\pgfusepath{stroke}
}
\pgfdeclarepatternformonly[\hatchdistance,\hatchthickness]{north west hatch}% name
{\pgfqpoint{-\hatchthickness}{-\hatchthickness}}% below left
{\pgfqpoint{\hatchdistance+\hatchthickness}{\hatchdistance+\hatchthickness}}% above right
{\pgfpoint{\hatchdistance}{\hatchdistance}}%
{
	\pgfsetcolor{\tikz@pattern@color}
	\pgfsetlinewidth{\hatchthickness}
	\pgfpathmoveto{\pgfqpoint{\hatchdistance+\hatchthickness}{-\hatchthickness}}
	\pgfpathlineto{\pgfqpoint{-\hatchthickness}{\hatchdistance+\hatchthickness}}
	\pgfusepath{stroke}
}
%\makeatother

\tikzset{
    hatch distance/.store in=\hatchdistance,
    hatch distance=10pt,
    hatch thickness/.store in=\hatchthickness,
   	hatch thickness=2pt
}

\colorlet{circle edge}{black}
\colorlet{circle area}{blue!20}


\tikzset{
    filled/.style={fill=green!30, draw=circle edge, thick},
    arceerl/.style={pattern=north east hatch, pattern color=blue!50, draw=circle edge},
    arceerr/.style={pattern=north west hatch, pattern color=yellow!50, draw=circle edge},
    outline/.style={draw=circle edge, thick}
}




%%% Updaten commando's
\def\hoofding #1#2#3{\maketitle}     % OBSOLETE ??

% we willen (bijna) altijd \geqslant ipv \geq ...!
\newcommand{\geqnoslant}{\geq}
\renewcommand{\geq}{\geqslant}
\newcommand{\leqnoslant}{\leq}
\renewcommand{\leq}{\leqslant}

% Todo: (201908) waarom komt er (soms) underlined voor emph ...?
\renewcommand{\emph}[1]{\textit{#1}}

% API commando's

\newcommand{\ds}{\displaystyle}
\newcommand{\ts}{\textstyle}  % tegenhanger van \ds   (Ximera zet PER  DEFAULT \ds!)

% uit Zomercursus-macro's: 
\newcommand{\bron}[1]{\begin{scriptsize} \emph{#1} \end{scriptsize}}     % deprecated ...?


%definities nieuwe commando's - afkortingen veel gebruikte symbolen
\newcommand{\R}{\ensuremath{\mathbb{R}}}
\newcommand{\Rnul}{\ensuremath{\mathbb{R}_0}}
\newcommand{\Reen}{\ensuremath{\mathbb{R}\setminus\{1\}}}
\newcommand{\Rnuleen}{\ensuremath{\mathbb{R}\setminus\{0,1\}}}
\newcommand{\Rplus}{\ensuremath{\mathbb{R}^+}}
\newcommand{\Rmin}{\ensuremath{\mathbb{R}^-}}
\newcommand{\Rnulplus}{\ensuremath{\mathbb{R}_0^+}}
\newcommand{\Rnulmin}{\ensuremath{\mathbb{R}_0^-}}
\newcommand{\Rnuleenplus}{\ensuremath{\mathbb{R}^+\setminus\{0,1\}}}
\newcommand{\N}{\ensuremath{\mathbb{N}}}
\newcommand{\Nnul}{\ensuremath{\mathbb{N}_0}}
\newcommand{\Z}{\ensuremath{\mathbb{Z}}}
\newcommand{\Znul}{\ensuremath{\mathbb{Z}_0}}
\newcommand{\Zplus}{\ensuremath{\mathbb{Z}^+}}
\newcommand{\Zmin}{\ensuremath{\mathbb{Z}^-}}
\newcommand{\Znulplus}{\ensuremath{\mathbb{Z}_0^+}}
\newcommand{\Znulmin}{\ensuremath{\mathbb{Z}_0^-}}
\newcommand{\C}{\ensuremath{\mathbb{C}}}
\newcommand{\Cnul}{\ensuremath{\mathbb{C}_0}}
\newcommand{\Cplus}{\ensuremath{\mathbb{C}^+}}
\newcommand{\Cmin}{\ensuremath{\mathbb{C}^-}}
\newcommand{\Cnulplus}{\ensuremath{\mathbb{C}_0^+}}
\newcommand{\Cnulmin}{\ensuremath{\mathbb{C}_0^-}}
\newcommand{\Q}{\ensuremath{\mathbb{Q}}}
\newcommand{\Qnul}{\ensuremath{\mathbb{Q}_0}}
\newcommand{\Qplus}{\ensuremath{\mathbb{Q}^+}}
\newcommand{\Qmin}{\ensuremath{\mathbb{Q}^-}}
\newcommand{\Qnulplus}{\ensuremath{\mathbb{Q}_0^+}}
\newcommand{\Qnulmin}{\ensuremath{\mathbb{Q}_0^-}}

\newcommand{\perdef}{\overset{\mathrm{def}}{=}}
\newcommand{\pernot}{\overset{\mathrm{notatie}}{=}}
\newcommand\perinderdaad{\overset{!}{=}}     % voorlopig gebruikt in limietenrekenregels
\newcommand\perhaps{\overset{?}{=}}          % voorlopig gebruikt in limietenrekenregels

\newcommand{\degree}{^\circ}


\DeclareMathOperator{\dom}{dom}     % domein
\DeclareMathOperator{\codom}{codom} % codomein
\DeclareMathOperator{\bld}{bld}     % beeld
\DeclareMathOperator{\graf}{graf}   % grafiek
\DeclareMathOperator{\rico}{rico}   % richtingcoëfficient
\DeclareMathOperator{\co}{co}       % coordinaat
\DeclareMathOperator{\gr}{gr}       % graad

\newcommand{\func}[5]{\ensuremath{#1: #2 \rightarrow #3: #4 \mapsto #5}} % Easy to write a function


% Operators
\DeclareMathOperator{\bgsin}{bgsin}
\DeclareMathOperator{\bgcos}{bgcos}
\DeclareMathOperator{\bgtan}{bgtan}
\DeclareMathOperator{\bgcot}{bgcot}
\DeclareMathOperator{\bgsinh}{bgsinh}
\DeclareMathOperator{\bgcosh}{bgcosh}
\DeclareMathOperator{\bgtanh}{bgtanh}
\DeclareMathOperator{\bgcoth}{bgcoth}

% Oude \Bgsin etc deprecated: gebruik \bgsin, en herdefinieer dat als je Bgsin wil!
%\DeclareMathOperator{\cosec}{cosec}    % not used? gebruik \csc en herdefinieer

% operatoren voor differentialen: to be verified; 1/2020: inconsequent gebruik bij afgeleiden/integralen
\renewcommand{\d}{\mathrm{d}}
\newcommand{\dx}{\d x}
\newcommand{\dd}[1]{\frac{\mathrm{d}}{\mathrm{d}#1}}
\newcommand{\ddx}{\dd{x}}

% om in voorbeelden/oefeningen de notatie voor afgeleiden te kunnen kiezen
% Usage: \afg{(2\sin(x))}  (en wordt d/dx, of accent, of D )
%\newcommand{\afg}[1]{{#1}'}
\newcommand{\afg}[1]{\left(#1\right)'}
%\renewcommand{\afg}[1]{\frac{\mathrm{d}#1}{\mathrm{d}x}}   % include in relevant exercises ...
%\renewcommand{\afg}[1]{D{#1}}

%
% \xmxxx commands: Extra KU Leuven functionaliteit van, boven of naast Ximera
%   ( Conventie 8/2019: xm+nederlandse omschrijving, maar is niet consequent gevolgd, en misschien ook niet erg handig !)
%
% (Met een minimale ximera.cls en preamble.tex zou een bruikbare .pdf moeten kunnen worden gemaakt van eender welke ximera)
%
% Usage: \xmtitle[Mijn korte abstract]{Mijn titel}{Mijn abstract}
% Eerste command na \begin{document}:
%  -> definieert de \title
%  -> definieert de abstract
%  -> doet \maketitle ( dus: print de hoofding als 'chapter' of 'sectie')
% Optionele parameter geeft eenn kort abstract (die met de globale setting \xmshortabstract{} al dan niet kan worden geprint.
% De optionele korte abstract kan worden gebruikt voor pseudo-grappige abtsarts, dus dus globaal al dan niet kunnen worden gebuikt...
% Globale settings:
%  de (optionele) 'korte abstract' wordt enkele getoond als \xmshortabstract is gezet
\providecommand\xmshortabstract{} % default: print (only!) short abstract if present
\newcommand{\xmtitle}[3][]{
	\title{#2}
	\begin{abstract}
		\ifdefined\xmshortabstract
		\ifstrempty{#1}{%
			#3
		}{%
			#1
		}%
		\else
		#3
		\fi
	\end{abstract}
	\maketitle
}

% 
% Kleine grapjes: moeten zonder verder gevolg kunnen worden verwijderd
%
%\newcommand{\xmopje}[1]{{\small#1{\reversemarginpar\marginpar{\Smiley}}}}   % probleem in floats!!
\newtoggle{showxmopje}
\toggletrue{showxmopje}

\newcommand{\xmopje}[1]{%
   \iftoggle{showxmopje}{#1}{}%
}


% -> geef een abstracte-formule-met-rechts-een-concreet-voorbeeld
% VB:  \formulevb{a^2+b^2=c^2}{3^2+4^2=5^2}
%
\ifdefined\HCode
\NewEnviron{xmdiv}[1]{\HCode{\Hnewline<div class="#1">\Hnewline}\BODY{\HCode{\Hnewline</div>\Hnewline}}}
\else
\NewEnviron{xmdiv}[1]{\BODY}
\fi

\providecommand{\formulevb}[2]{
	{\centering

    \begin{xmdiv}{xmformulevb}    % zie css voor online layout !!!
	\begin{tabular}{lcl}
		\important{#1}
		&  &
		Vb: $#2$
		\end{tabular}
	\end{xmdiv}

	}
}

\ifdefined\HCode
\providecommand{\vb}[1]{%
    \HCode{\Hnewline<span class="xmvb">}#1\HCode{</span>\Hnewline}%
}
\else
\providecommand{\vb}[1]{
    \colorbox{blue!10}{#1}
}
\fi

\ifdefined\HCode
\providecommand{\xmcolorbox}[2]{
	\HCode{\Hnewline<div class="xmcolorbox">\Hnewline}#2\HCode{\Hnewline</div>\Hnewline}
}
\else
\providecommand{\xmcolorbox}[2]{
  \cellcolor{#1}#2
}
\fi


\ifdefined\HCode
\providecommand{\xmopmerking}[1]{
 \HCode{\Hnewline<div class="xmopmerking">\Hnewline}#1\HCode{\Hnewline</div>\Hnewline}
}
\else
\providecommand{\xmopmerking}[1]{
	{\footnotesize #1}
}
\fi
% \providecommand{\voorbeeld}[1]{
% 	\colorbox{blue!10}{$#1$}
% }



% Hernoem Proof naar Bewijs, nodig voor HTML versie
\renewcommand*{\proofname}{Bewijs}

% Om opgave van oefening (wordt niet geprint bij oplossingenblad)
% (to be tested test)
\NewEnviron{statement}{\BODY}

% Environment 'oplossing' en 'uitkomst'
% voor resp. volledige 'uitwerking' dan wel 'enkel eindresultaat'
% geimplementeerd via feedback, omdat er in de ximera-server adhoc feedback-code is toegevoegd
%% Niet tonen indien handout
%% Te gebruiken om volledige oplossingen/uitwerkingen van oefeningen te tonen
%% \begin{oplossing}        De optelling is commutatief \end{oplossing}  : verschijnt online enkel 'op vraag'
%% \begin{oplossing}[toon]  De optelling is commutatief \end{oplossing}  : verschijnt steeds onmiddellijk online (bv te gebruiken bij voorbeelden) 

\ifhandout%
    \NewEnviron{oplossing}[1][onzichtbaar]%
    {%
    \ifthenelse{\equal{\detokenize{#1}}{\detokenize{toon}}}
    {
    \def\PH@Command{#1}% Use PH@Command to hold the content and be a target for "\expandafter" to expand once.

    \begin{trivlist}% Begin the trivlist to use formating of the "Feedback" label.
    \item[\hskip \labelsep\small\slshape\bfseries Oplossing% Format the "Feedback" label. Don't forget the space.
    %(\texttt{\detokenize\expandafter{\PH@Command}}):% Format (and detokenize) the condition for feedback to trigger
    \hspace{2ex}]\small%\slshape% Insert some space before the actual feedback given.
    \BODY
    \end{trivlist}
    }
    {  % \begin{feedback}[solution]   \BODY     \end{feedback}  }
    }
    }    
\else
% ONLY for HTML; xmoplossing is styled with css, and is not, and need not be a LaTeX environment
% THUS: it does NOT use feedback anymore ...
%    \NewEnviron{oplossing}{\begin{expandable}{xmoplossing}{\nlen{Toon uitwerking}{Show solution}}{\BODY}\end{expandable}}
    \newenvironment{oplossing}[1][onzichtbaar]
   {%
       \begin{expandable}{xmoplossing}{}
   }
   {%
   	   \end{expandable}
   } 
%     \newenvironment{oplossing}[1][onzichtbaar]
%    {%
%        \begin{feedback}[solution]   	
%    }
%    {%
%    	   \end{feedback}
%    } 
\fi

\ifhandout%
    \NewEnviron{uitkomst}[1][onzichtbaar]%
    {%
    \ifthenelse{\equal{\detokenize{#1}}{\detokenize{toon}}}
    {
    \def\PH@Command{#1}% Use PH@Command to hold the content and be a target for "\expandafter" to expand once.

    \begin{trivlist}% Begin the trivlist to use formating of the "Feedback" label.
    \item[\hskip \labelsep\small\slshape\bfseries Uitkomst:% Format the "Feedback" label. Don't forget the space.
    %(\texttt{\detokenize\expandafter{\PH@Command}}):% Format (and detokenize) the condition for feedback to trigger
    \hspace{2ex}]\small%\slshape% Insert some space before the actual feedback given.
    \BODY
    \end{trivlist}
    }
    {  % \begin{feedback}[solution]   \BODY     \end{feedback}  }
    }
    }    
\else
\ifdefined\HCode
   \newenvironment{uitkomst}[1][onzichtbaar]
    {%
        \begin{expandable}{xmuitkomst}{}%
    }
    {%
    	\end{expandable}%
    } 
\else
  % Do NOT print 'uitkomst' in non-handout
  %  (presumably, there is also an 'oplossing' ??)
  \newenvironment{uitkomst}[1][onzichtbaar]{}{}
\fi
\fi

%
% Uitweidingen zijn extra's die niet redelijkerwijze tot de leerstof behoren
% Uitbreidingen zijn extra's die wel redelijkerwijze tot de leerstof van bv meer geavanceerde versies kunnen behoren (B-programma/Wiskundestudenten/...?)
% Nog niet voorzien: design voor verschillende versies (A/B programma, BIO, voorkennis/ ...)
% Voor 'uitweidingen' is er een environment die online per default is ingeklapt, en in pdf al dan niet kan worden geincluded  (via \xmnouitweiding) 
%
% in een xourse, per default GEEN uitweidingen, tenzij \xmuitweiding expliciet ergens is gezet ...
\ifdefined\isXourse
   \ifdefined\xmuitweiding
   \else
       \def\xmnouitweiding{true}
   \fi
\fi

\ifdefined\xmnouitweiding
\newcounter{xmuitweiding}  % anders error undefined ...  
\excludecomment{xmuitweiding}
\else
\newtheoremstyle{dotless}{}{}{}{}{}{}{ }{}
\theoremstyle{dotless}
\newtheorem*{xmuitweidingnofrills}{}   % nofrills = no accordion; gebruikt dus de dotless theoremstyle!

\newcounter{xmuitweiding}
\newenvironment{xmuitweiding}[1][ ]%
{% 
	\refstepcounter{xmuitweiding}%
    \begin{expandable}{xmuitweiding}{\nlentext{Uitweiding \arabic{xmuitweiding}: #1}{Digression \arabic{xmuitweiding}: #1}}%
	\begin{xmuitweidingnofrills}%
}
{%
    \end{xmuitweidingnofrills}%
    \end{expandable}%
}   
% \newenvironment{xmuitweiding}[1][ ]%
% {% 
% 	\refstepcounter{xmuitweiding}
% 	\begin{accordion}\begin{accordion-item}[Uitweiding \arabic{xmuitweiding}: #1]%
% 			\begin{xmuitweidingnofrills}%
% 			}
% 			{\end{xmuitweidingnofrills}\end{accordion-item}\end{accordion}}   
\fi


\newenvironment{xmexpandable}[1][]{
	\begin{accordion}\begin{accordion-item}[#1]%
		}{\end{accordion-item}\end{accordion}}


% Command that gives a selection box online, but just prints the right answer in pdf
\newcommand{\xmonlineChoice}[1]{\pdfOnly{\wordchoicegiventrue}\wordChoice{#1}\pdfOnly{\wordchoicegivenfalse}}   % deprecated, gebruik onlineChoice ...
\newcommand{\onlineChoice}[1]{\pdfOnly{\wordchoicegiventrue}\wordChoice{#1}\pdfOnly{\wordchoicegivenfalse}}

\newcommand{\TJa}{\nlentext{ Ja }{ Yes }}
\newcommand{\TNee}{\nlentext{ Nee }{ No }}
\newcommand{\TJuist}{\nlentext{ Juist }{ True }}
\newcommand{\TFout}{\nlentext{ Fout }{ False }}

\newcommand{\choiceTrue }{{\renewcommand{\choiceminimumhorizontalsize}{4em}\wordChoice{\choice[correct]{\TJuist}\choice{\TFout}}}}
\newcommand{\choiceFalse}{{\renewcommand{\choiceminimumhorizontalsize}{4em}\wordChoice{\choice{\TJuist}\choice[correct]{\TFout}}}}

\newcommand{\choiceYes}{{\renewcommand{\choiceminimumhorizontalsize}{3em}\wordChoice{\choice[correct]{\TJa}\choice{\TNee}}}}
\newcommand{\choiceNo }{{\renewcommand{\choiceminimumhorizontalsize}{3em}\wordChoice{\choice{\TJa}\choice[correct]{\TNee}}}}

% Optional nicer formatting for wordChoice in PDF

\let\inlinechoiceorig\inlinechoice

%\makeatletter
%\providecommand{\choiceminimumverticalsize}{\vphantom{$\frac{\sqrt{2}}{2}$}}   % minimum height of boxes (cfr infra)
\providecommand{\choiceminimumverticalsize}{\vphantom{$\tfrac{2}{2}$}}   % minimum height of boxes (cfr infra)
\providecommand{\choiceminimumhorizontalsize}{1em}   % minimum width of boxes (cfr infra)

\newcommand{\inlinechoicesquares}[2][]{%
		\setkeys{choice}{#1}%
		\ifthenelse{\boolean{\choice@correct}}%
		{%
            \ifhandout%
               \fbox{\choiceminimumverticalsize #2}\allowbreak\ignorespaces%
            \else%
               \fcolorbox{blue}{blue!20}{\choiceminimumverticalsize #2}\allowbreak\ignorespaces\setkeys{choice}{correct=false}\ignorespaces%
            \fi%
		}%
		{% else
			\fbox{\choiceminimumverticalsize #2}\allowbreak\ignorespaces%  HACK: wat kleiner, zodat fits on line ... 	
		}%
}

\newcommand{\inlinechoicesquareX}[2][]{%
		\setkeys{choice}{#1}%
		\ifthenelse{\boolean{\choice@correct}}%
		{%
            \ifhandout%
               \framebox[\ifdim\choiceminimumhorizontalsize<\width\width\else\choiceminimumhorizontalsize\fi]{\choiceminimumverticalsize\ #2\ }\allowbreak\ignorespaces\setkeys{choice}{correct=false}\ignorespaces%
            \else%
               \fcolorbox{blue}{blue!20}{\makebox[\ifdim\choiceminimumhorizontalsize<\width\width\else\choiceminimumhorizontalsize\fi]{\choiceminimumverticalsize #2}}\allowbreak\ignorespaces\setkeys{choice}{correct=false}\ignorespaces%
            \fi%
		}%
		{% else
        \ifhandout%
			\framebox[\ifdim\choiceminimumhorizontalsize<\width\width\else\choiceminimumhorizontalsize\fi]{\choiceminimumverticalsize\ #2\ }\allowbreak\ignorespaces%  HACK: wat kleiner, zodat fits on line ... 	
        \fi
		}%
}


\newcommand{\inlinechoicepuntjes}[2][]{%
		\setkeys{choice}{#1}%
		\ifthenelse{\boolean{\choice@correct}}%
		{%
            \ifhandout%
               \dots\ldots\ignorespaces\setkeys{choice}{correct=false}\ignorespaces
            \else%
               \fcolorbox{blue}{blue!20}{\choiceminimumverticalsize #2}\allowbreak\ignorespaces\setkeys{choice}{correct=false}\ignorespaces%
            \fi%
		}%
		{% else
			%\fbox{\choiceminimumverticalsize #2}\allowbreak\ignorespaces%  HACK: wat kleiner, zodat fits on line ... 	
		}%
}

% print niets, maar definieer globale variable \myanswer
%  (gebruikt om oplossingsbladen te printen) 
\newcommand{\inlinechoicedefanswer}[2][]{%
		\setkeys{choice}{#1}%
		\ifthenelse{\boolean{\choice@correct}}%
		{%
               \gdef\myanswer{#2}\setkeys{choice}{correct=false}

		}%
		{% else
			%\fbox{\choiceminimumverticalsize #2}\allowbreak\ignorespaces%  HACK: wat kleiner, zodat fits on line ... 	
		}%
}



%\makeatother

\newcommand{\setchoicedefanswer}{
\ifdefined\HCode
\else
%    \renewenvironment{multipleChoice@}[1][]{}{} % remove trailing ')'
    \let\inlinechoice\inlinechoicedefanswer
\fi
}

\newcommand{\setchoicepuntjes}{
\ifdefined\HCode
\else
    \renewenvironment{multipleChoice@}[1][]{}{} % remove trailing ')'
    \let\inlinechoice\inlinechoicepuntjes
\fi
}
\newcommand{\setchoicesquares}{
\ifdefined\HCode
\else
    \renewenvironment{multipleChoice@}[1][]{}{} % remove trailing ')'
    \let\inlinechoice\inlinechoicesquares
\fi
}
%
\newcommand{\setchoicesquareX}{
\ifdefined\HCode
\else
    \renewenvironment{multipleChoice@}[1][]{}{} % remove trailing ')'
    \let\inlinechoice\inlinechoicesquareX
\fi
}
%
\newcommand{\setchoicelist}{
\ifdefined\HCode
\else
    \renewenvironment{multipleChoice@}[1][]{}{)}% re-add trailing ')'
    \let\inlinechoice\inlinechoiceorig
\fi
}

\setchoicesquareX  % by default list-of-squares with onlineChoice in PDF

% Omdat multicols niet werkt in html: enkel in pdf  (in html zijn langere pagina's misschien ook minder storend)
\newenvironment{xmmulticols}[1][2]{
 \pdfOnly{\begin{multicols}{#1}}%
}{ \pdfOnly{\end{multicols}}}

%
% Te gebruiken in plaats van \section\subsection
%  (in een printstyle kan dan het level worden aangepast
%    naargelang \chapter vs \section style )
% 3/2021: DO NOT USE \xmsubsection !
\newcommand\xmsection\subsection
\newcommand\xmsubsection\subsubsection

% Aanpassen printversie
%  (hier gedefinieerd, zodat ze in xourse kunnen worden gezet/overschreven)
\providebool{parttoc}
\providebool{printpartfrontpage}
\providebool{printactivitytitle}
\providebool{printactivityqrcode}
\providebool{printactivityurl}
\providebool{printcontinuouspagenumbers}
\providebool{numberactivitiesbysubpart}
\providebool{addtitlenumber}
\providebool{addsectiontitlenumber}
\addtitlenumbertrue
\addsectiontitlenumbertrue

% The following three commands are hardcoded in xake, you can't create other commands like these, without adding them to xake as well
%  ( gebruikt in xourses om juiste soort titelpagina te krijgen voor verschillende ximera's )
\newcommand{\activitychapter}[2][]{
    {    
    \ifstrequal{#1}{notnumbered}{
        \addtitlenumberfalse
    }{}
    \typeout{ACTIVITYCHAPTER #2}   % logging
	\chapterstyle
	\activity{#2}
    }
}
\newcommand{\activitysection}[2][]{
    {
    \ifstrequal{#1}{notnumbered}{
        \addsectiontitlenumberfalse
    }{}
	\typeout{ACTIVITYSECTION #2}   % logging
	\sectionstyle
	\activity{#2}
    }
}
% Practices worden als activity getoond om de grote blokken te krijgen online
\newcommand{\practicesection}[2][]{
    {
    \ifstrequal{#1}{notnumbered}{
        \addsectiontitlenumberfalse
    }{}
    \typeout{PRACTICESECTION #2}   % logging
	\sectionstyle
	\activity{#2}
    }
}
\newcommand{\activitychapterlink}[3][]{
    {
    \ifstrequal{#1}{notnumbered}{
        \addtitlenumberfalse
    }{}
    \typeout{ACTIVITYCHAPTERLINK #3}   % logging
	\chapterstyle
	\activitylink[#1]{#2}{#3}
    }
}

\newcommand{\activitysectionlink}[3][]{
    {
    \ifstrequal{#1}{notnumbered}{
        \addsectiontitlenumberfalse
    }{}
    \typeout{ACTIVITYSECTIONLINK #3}   % logging
	\sectionstyle
	\activitylink[#1]{#2}{#3}
    }
}


% Commando om de printstyle toe te voegen in ximera's. Zorgt ervoor dat er geen problemen zijn als je de xourses compileert
% hack om onhandige relative paden in TeX te omzeilen
% should work both in xourse and ximera (pre-112022 only in ximera; thus obsoletes adhoc setup in xourses)
% loads global.sty if present (cfr global.css for online settings!)
% use global.sty to overwrite settings in printstyle.sty ...
\newcommand{\addPrintStyle}[1]{
\iftikzexport\else   % only in PDF
  \makeatletter
  \ifx\@onlypreamble\@notprerr\else   % ONLY if in tex-preamble   (and e.g. not when included from xourse)
    \typeout{Loading printstyle}   % logging
    \usepackage{#1/printstyle} % mag enkel geinclude worden als je die apart compileert
    \IfFileExists{#1/global.sty}{
        \typeout{Loading printstyle-folder #1/global.sty}   % logging
        \usepackage{#1/global}
        }{
        \typeout{Info: No extra #1/global.sty}   % logging
    }   % load global.sty if present
    \IfFileExists{global.sty}{
        \typeout{Loading local-folder global.sty (or TEXINPUTPATH..)}   % logging
        \usepackage{global}
    }{
        \typeout{Info: No folder/global.sty}   % logging
    }   % load global.sty if present
    \IfFileExists{\currfilebase.sty}
    {
        \typeout{Loading \currfilebase.sty}
        \input{\currfilebase.sty}
    }{
        \typeout{Info: No local \currfilebase.sty}
    }
    \fi
  \makeatother
\fi
}

%
%  
% references: Ximera heeft adhoc logica	 om online labels te doen werken over verschillende files heen
% met \hyperref kan de getoonde tekst toch worden opgegeven, in plaats van af te hangen van de label-text
\ifdefined\HCode
% Link to standard \labels, but give your own description
% Usage:  Volg \hyperref[my_very_verbose_label]{deze link} voor wat tijdverlies
%   (01/2020: Ximera-server aangepast om bij class reference-keeptext de link-text NIET te vervangen door de label-text !!!) 
\renewcommand{\hyperref}[2][]{\HCode{<a class="reference reference-keeptext" href="\##1">}#2\HCode{</a>}}
%
%  Link to specific targets  (not tested ?)
\renewcommand{\hypertarget}[1]{\HCode{<a class="ximera-label" id="#1"></a>}}
\renewcommand{\hyperlink}[2]{\HCode{<a class="reference reference-keeptext" href="\##1">}#2\HCode{</a>}}
\fi

% Mmm, quid English ... (for keyword #1 !) ?
\newcommand{\wikilink}[2]{
    \hyperlink{https://nl.wikipedia.org/wiki/#1}{#2}
    \pdfOnly{\footnote{See \url{https://nl.wikipedia.org/wiki/#1}}
    }
}

\renewcommand{\figurename}{Figuur}
\renewcommand{\tablename}{Tabel}

%
% Gedoe om verschillende versies van xourse/ximera te maken afhankelijk van settings
%
% default: versie met antwoorden
% handout: versie voor de studenten, zonder antwoorden/oplossingen
% full: met alles erop en eraan, dus geschikt voor auteurs en/of lesgevers  (bevat in de pdf ook de 'online-only' stukken!)
%
%
% verder kunnen ook opties/variabele worden gezet voor hints/auteurs/uitweidingen/ etc
%
% 'Full' versie
\newtoggle{showonline}
\ifdefined\HCode   % zet default showOnline
    \toggletrue{showonline} 
\else
    \togglefalse{showonline}
\fi

% Full versie   % deprecated: see infra
\newcommand{\printFull}{
    \hintstrue
    \handoutfalse
    \toggletrue{showonline} 
}

\ifdefined\shouldPrintFull   % deprecated: see infra
    \printFull
\fi



% Overschrijf onlineOnly  (zoals gedefinieerd in ximera.cls)
\ifhandout   % in handout: gebruik de oorspronkelijke ximera.cls implementatie  (is dit wel nodig/nuttig?)
\else
    \iftoggle{showonline}{%
        \ifdefined\HCode
          \RenewEnviron{onlineOnly}{\bgroup\BODY\egroup}   % showOnline, en we zijn  online, dus toon de tekst
        \else
          \RenewEnviron{onlineOnly}{\bgroup\color{red!50!black}\BODY\egroup}   % showOnline, maar we zijn toch niet online: kleur de tekst rood 
        \fi
    }{%
      \RenewEnviron{onlineOnly}{}  % geen showOnline
    }
\fi

% hack om na hoofding van definition/proposition/... als dan niet op een nieuwe lijn te starten
% soms is dat goed en mooi, en soms niet; en in HTML is het nu (2/2020) anders dan in pdf
% vandaar suggestie om 
%     \begin{definition}[Nieuw concept] \nl
% te gebruiken als je zeker een newline wil na de hoofdig en titel
% (in het bijzonder itemize zonder \nl is 'lelijk' ...)
\ifdefined\HCode
\newcommand{\nl}{}
\else
\newcommand{\nl}{\ \par} % newline (achter heading van definition etc.)
\fi


% \nl enkel in handoutmode (ihb voor \wordChoice, die dan typisch veeeel langer wordt)
\ifdefined\HCode
\providecommand{\handoutnl}{}
\else
\providecommand{\handoutnl}{%
\ifhandout%
  \nl%
\fi%
}
\fi

% Could potentially replace \pdfOnline/\begin{onlineOnly} : 
% Usage= \ifonline{Hallo surfer}{Hallo PDFlezer}
\providecommand{\ifonline}[2]%
{
\begin{onlineOnly}#1\end{onlineOnly}%
\pdfOnly{#2}
}%


%
% Maak optionele 'basic' en 'extended' versies van een activity
%  met environment basicOnly, basicSkip en extendedOnly
%
%  (
%   Dit werkt ENKEL in de PDF; de online versies tonen (minstens voorklopig) steeds 
%   het default geval met printbasicversion en printextendversion beide FALSE
%  )
%
\providebool{printbasicversion}
\providebool{printextendedversion}   % not properly implemented
\providebool{printfullversion}       % presumably print everything (debug/auteur)
%
% only set these in xourses, and BEFORE loading this preamble
%
%\newif\ifshowbasic     \showbasictrue        % use this line in xourse to show 'basic' sections
%\newif\ifshowextended  \showextendedtrue     % use this line in xourse to show 'extended' sections
%
%
%\ifbool{showbasic}
%      { \NewEnviron{basicOnly}{\BODY} }    % if yes: just print contents
%      { \NewEnviron{basicOnly}{}      }    % if no:  completely ignore contents
%
%\ifbool{showbasic}
%      { \NewEnviron{basicSkip}{}      }
%      { \NewEnviron{basicSkip}{\BODY} }
%

\ifbool{printextendedversion}
      { \NewEnviron{extendedOnly}{\BODY} }
      { \NewEnviron{extendedOnly}{}      }
      


\ifdefined\HCode    % in html: always print
      {\newenvironment*{basicOnly}{}{}}    % if yes: just print contents
      {\newenvironment*{basicSkip}{}{}}    % if yes: just print contents
\else

\ifbool{printbasicversion}
      {\newenvironment*{basicOnly}{}{}}    % if yes: just print contents
      {\NewEnviron{basicOnly}{}      }    % if no:  completely ignore contents

\ifbool{printbasicversion}
      {\NewEnviron{basicSkip}{}      }
      {\newenvironment*{basicSkip}{}{}}

\fi

\usepackage{float}
\usepackage[rightbars,color]{changebar}

% Full versie
\ifbool{printfullversion}{
    \hintstrue
    \handoutfalse
    \toggletrue{showonline}
    \printbasicversionfalse
    \cbcolor{red}
    \renewenvironment*{basicOnly}{\cbstart}{\cbend}
    \renewenvironment*{basicSkip}{\cbstart}{\cbend}
    \def\xmtoonprintopties{FULL}   % will be printed in footer
}
{}
      
%
% Evalueer \ifhints IN de environment
%  
%
%\RenewEnviron{hint}
%{
%\ifhandout
%\ifhints\else\setbox0\vbox\fi%everything in een emty box
%\bgroup 
%\stepcounter{hintLevel}
%\BODY
%\egroup\ignorespacesafterend
%\addtocounter{hintLevel}{-1}
%\else
%\ifhints
%\begin{trivlist}\item[\hskip \labelsep\small\slshape\bfseries Hint:\hspace{2ex}]
%\small\slshape
%\stepcounter{hintLevel}
%\BODY
%\end{trivlist}
%\addtocounter{hintLevel}{-1}
%\fi
%\fi
%}

% Onafhankelijk van \ifhandout ...? TO BE VERIFIED
\RenewEnviron{hint}
{
\ifhints
\begin{trivlist}\item[\hskip \labelsep\small\bfseries Hint:\hspace{2ex}]
\small%\slshape
\stepcounter{hintLevel}
\BODY
\end{trivlist}
\addtocounter{hintLevel}{-1}
\else
\iftikzexport   % anders worden de tikz tekeningen in hints niet gegenereerd ?
\setbox0\vbox\bgroup
\stepcounter{hintLevel}
\BODY
\egroup\ignorespacesafterend
\addtocounter{hintLevel}{-1}
\fi % ifhandout
\fi %ifhints
}

%
% \tab sets typewriter-tabs (e.g. to format questions)
% (Has no effect in HTML :-( ))
%
\usepackage{tabto}
\ifdefined\HCode
  \renewcommand{\tab}{\quad}    % otherwise dummy .png's are generated ...?
\fi


% Also redefined in  preamble to get correct styling 
% for tikz images for (\tikzexport)
%

\theoremstyle{definition} % Bold titels
\makeatletter
\let\proposition\relax
\let\c@proposition\relax
\let\endproposition\relax
\makeatother
\newtheorem{proposition}{Eigenschap}


%\instructornotesfalse

% logic with \ifhandoutin ximera.cls unclear;so overwrite ...
\makeatletter
\@ifundefined{ifinstructornotes}{%
  \newif\ifinstructornotes
  \instructornotesfalse
  \newenvironment{instructorNotes}{}{}
}{}
\makeatother
\ifinstructornotes
\else
\renewenvironment{instructorNotes}%
{%
    \setbox0\vbox\bgroup
}
{%
    \egroup
}
\fi

% \RedeclareMathOperator
% from https://tex.stackexchange.com/questions/175251/how-to-redefine-a-command-using-declaremathoperator
\makeatletter
\newcommand\RedeclareMathOperator{%
    \@ifstar{\def\rmo@s{m}\rmo@redeclare}{\def\rmo@s{o}\rmo@redeclare}%
}
% this is taken from \renew@command
\newcommand\rmo@redeclare[2]{%
    \begingroup \escapechar\m@ne\xdef\@gtempa{{\string#1}}\endgroup
    \expandafter\@ifundefined\@gtempa
    {\@latex@error{\noexpand#1undefined}\@ehc}%
    \relax
    \expandafter\rmo@declmathop\rmo@s{#1}{#2}}
% This is just \@declmathop without \@ifdefinable
\newcommand\rmo@declmathop[3]{%
    \DeclareRobustCommand{#2}{\qopname\newmcodes@#1{#3}}%
}
\@onlypreamble\RedeclareMathOperator
\makeatother


%
% Engelse vertaling, vooral in mathmode
%
% 1. Algemene procedure
%
\ifdefined\isEn
 \newcommand{\nlen}[2]{#2}
 \newcommand{\nlentext}[2]{\text{#2}}
 \newcommand{\nlentextbf}[2]{\textbf{#2}}
\else
 \newcommand{\nlen}[2]{#1}
 \newcommand{\nlentext}[2]{\text{#1}}
 \newcommand{\nlentextbf}[2]{\textbf{#1}}
\fi

%
% 2. Lijst van erg veel gebruikte uitdrukkingen
%

% Ja/Nee/Fout/Juits etc
%\newcommand{\TJa}{\nlentext{ Ja }{ and }}
%\newcommand{\TNee}{\nlentext{ Nee }{ No }}
%\newcommand{\TJuist}{\nlentext{ Juist }{ Correct }
%\newcommand{\TFout}{\nlentext{ Fout }{ Wrong }
\newcommand{\TWaar}{\nlentext{ Waar }{ True }}
\newcommand{\TOnwaar}{\nlentext{ Vals }{ False }}
% Korte bindwoorden en, of, dus, ...
\newcommand{\Ten}{\nlentext{ en }{ and }}
\newcommand{\Tof}{\nlentext{ of }{ or }}
\newcommand{\Tdus}{\nlentext{ dus }{ so }}
\newcommand{\Tendus}{\nlentext{ en dus }{ and thus }}
\newcommand{\Tvooralle}{\nlentext{ voor alle }{ for all }}
\newcommand{\Took}{\nlentext{ ook }{ also }}
\newcommand{\Tals}{\nlentext{ als }{ when }} %of if?
\newcommand{\Twant}{\nlentext{ want }{ as }}
\newcommand{\Tmaal}{\nlentext{ maal }{ times }}
\newcommand{\Toptellen}{\nlentext{ optellen }{ add }}
\newcommand{\Tde}{\nlentext{ de }{ the }}
\newcommand{\Thet}{\nlentext{ het }{ the }}
\newcommand{\Tis}{\nlentext{ is }{ is }} %zodat is in text staat in mathmode (geen italics)
\newcommand{\Tmet}{\nlentext{ met }{ where }} % in situaties e.g met p < n --> where p < n
\newcommand{\Tnooit}{\nlentext{ nooit }{ never }}
\newcommand{\Tmaar}{\nlentext{ maar }{ but }}
\newcommand{\Tniet}{\nlentext{ niet }{ not }}
\newcommand{\Tuit}{\nlentext{ uit }{ from }}
\newcommand{\Ttov}{\nlentext{ t.o.v. }{ w.r.t. }}
\newcommand{\Tzodat}{\nlentext{ zodat }{ such that }}
\newcommand{\Tdeth}{\nlentext{de }{th }}
\newcommand{\Tomdat}{\nlentext{omdat }{because }} 


%
% Overschrijf addhoc commando's
%
\ifdefined\isEn
\renewcommand{\pernot}{\overset{\mathrm{notation}}{=}}
\RedeclareMathOperator{\bld}{im}     % beeld
\RedeclareMathOperator{\graf}{graph}   % grafiek
\RedeclareMathOperator{\rico}{slope}   % richtingcoëfficient
\RedeclareMathOperator{\co}{co}       % coordinaat
\RedeclareMathOperator{\gr}{deg}       % graad

% Operators
\RedeclareMathOperator{\bgsin}{arcsin}
\RedeclareMathOperator{\bgcos}{arccos}
\RedeclareMathOperator{\bgtan}{arctan}
\RedeclareMathOperator{\bgcot}{arccot}
\RedeclareMathOperator{\bgsinh}{arcsinh}
\RedeclareMathOperator{\bgcosh}{arccosh}
\RedeclareMathOperator{\bgtanh}{arctanh}
\RedeclareMathOperator{\bgcoth}{arccoth}

\fi


% HACK: use 'oplossing' for 'explanation' ...
\let\explanation\relax
\let\endexplanation\relax
% \newenvironment{explanation}{\begin{oplossing}}{\end{oplossing}}
\newcounter{explanation}

\ifhandout%
    \NewEnviron{explanation}[1][toon]%
    {%
    \RenewEnviron{verbatim}{ \red{VERBATIM CONTENT MISSING IN THIS PDF}} %% \expandafter\verb|\BODY|}

    \ifthenelse{\equal{\detokenize{#1}}{\detokenize{toon}}}
    {
    \def\PH@Command{#1}% Use PH@Command to hold the content and be a target for "\expandafter" to expand once.

    \begin{trivlist}% Begin the trivlist to use formating of the "Feedback" label.
    \item[\hskip \labelsep\small\slshape\bfseries Explanation:% Format the "Feedback" label. Don't forget the space.
    %(\texttt{\detokenize\expandafter{\PH@Command}}):% Format (and detokenize) the condition for feedback to trigger
    \hspace{2ex}]\small%\slshape% Insert some space before the actual feedback given.
    \BODY
    \end{trivlist}
    }
    {  % \begin{feedback}[solution]   \BODY     \end{feedback}  }
    }
    }    
\else
% ONLY for HTML; xmoplossing is styled with css, and is not, and need not be a LaTeX environment
% THUS: it does NOT use feedback anymore ...
%    \NewEnviron{oplossing}{\begin{expandable}{xmoplossing}{\nlen{Toon uitwerking}{Show solution}}{\BODY}\end{expandable}}
    \newenvironment{explanation}[1][toon]
   {%
       \begin{expandable}{xmoplossing}{}
   }
   {%
   	   \end{expandable}
   } 
\fi

 \title{The Characteristic Equation} \license{CC BY-NC-SA 4.0}


\begin{document}
\begin{abstract}

\end{abstract}
\maketitle
\section*{The Characteristic Equation}
Let $A$ be an $n \times n$ matrix.  In \href{https://ximera.osu.edu/oerlinalg/LinearAlgebra/EIG-0010/main}{Describing Eigenvalues and Eigenvectors Algebraically and Geometrically} we learned that the eigenvectors and eigenvalues of $A$ are vectors $\vec{x}$ and scalars $\lambda$ that satisfy the equation  
\begin{align}\label{def:eigen} A \vec{x} = \lambda \vec{x}\end{align}
We listed a few reasons why we are interested in finding eigenvalues and eigenvectors, but we did not give any process for finding them.  In this section we will focus on a process which can be used for small matrices.  For larger matrices, the best methods we have are iterative methods, and we will explore some of these in \href{https://ximera.osu.edu/oerlinalg/LinearAlgebra/EIG-0070/main}{The Power Method and the Dominant Eigenvalue}.

For an $n \times n$ matrix, we will see that the eigenvalues are the roots of a polynomial called the \dfn{characteristic polynomial}.  So finding eigenvalues is equivalent to solving a polynomial equation of degree $n$.  Finding the corresponding eigenvectors turns out to be a matter of computing the null space of a matrix, as the following exploration demonstrates.

\begin{exploration}\label{exp:slowdown}
If a vector $\vec{x}$ is an eigenvector satisfying Equation (\ref{def:eigen}), then clearly it also satisfies  $A\vec{x}-\lambda \vec{x} =$ \wordChoice{\choice{0} \choice[correct]{$\vec{0}$}}.

It seems natural at this point to try to factor.  We would love to ``factor out'' $\vec{x}$.  Here is the procedure:
\begin{align*}
A\vec{x}-\lambda \vec{x} &= \vec{0} \\
A\vec{x}-\lambda I\vec{x} &= \vec{0} \\
(A-\lambda I)\vec{x} &= \vec{0}
\end{align*}
The middle step was necessary before factoring because \wordChoice{\choice[correct]{we cannot subtract a $1 \times 1$ scalar $\lambda$ from an $n \times n$ matrix $A$} \choice{$\lambda$ is a Greek letter}}.

This shows that any eigenvector $\vec{x}$ of $A$ is in the \wordChoice{\choice{row space}\choice{column space}\choice[correct]{null space}}  of the related matrix, $A-\lambda I$.

Since eigenvectors are non-zero vectors, this means that $A$ will have eigenvectors if and only if the null space of $A-\lambda I$ is nontrivial.  The only way that $\mbox{null}(A-\lambda I)$ can be nontrivial is if $\mbox{rank}(A-\lambda I)$ \wordChoice{\choice{$=$}\choice[correct]{$<$}\choice{$>$}} $n$.
\end{exploration}

If the rank of an $n \times n$ matrix is less than $n$, then the matrix is singular.  Since $(A-\lambda I)$ must be singular for any eigenvalue $\lambda$, we see that $\lambda$ is an eigenvalue of $A$ if and only if \begin{align}\label{eqn:chareqn}\mbox{det}(A-\lambda I) = 0\end{align}

\subsection{Eigenvalues}
In theory, then, to find the eigenvalues of $A$, one can solve Equation (\ref{eqn:chareqn}) for $\lambda$.  

\begin{definition}\label{def:chareqcharpoly}
The equation 
$$\mbox{det}(A-\lambda I) = 0$$ is called the \dfn{characteristic equation} of $A$.  The left-hand side of the equation is a polynomial in $\lambda$ and is called the \dfn{characteristic polynomial} of $A$.
\end{definition}

\begin{example}\label{ex:2x2eig}
Let $A=\begin{bmatrix} 2& 1\\ 1&2
\end{bmatrix}$.  Compute the eigenvalues of this matrix using the characteristic equation.
\begin{explanation}
\begin{align*}\det(A-\lambda I)=\begin{vmatrix}2-\lambda&1\\1&2-\lambda\end{vmatrix}&=(2-\lambda)^2-1\\
&=\lambda^2-4\lambda+3\\
&=(\lambda-1)(\lambda-3)
\end{align*}
The characteristic equation $(\lambda-1)(\lambda-3)=0$ has solutions $\lambda_1=1$ and $\lambda_2=3$.  These are the eigenvalues of $A$.
\end{explanation}
\end{example}

\begin{example}\label{ex:2x2eig2}
Let $B=\begin{bmatrix} 2& 1\\ 4&2
\end{bmatrix}$.  Compute the eigenvalues of $B$ using the characteristic equation.

$\lambda_1=\answer{0}$ and $\lambda_2=\answer{4}$
\end{example}

\begin{example}\label{ex:3x3eig}
Let $C=\begin{bmatrix} 2 & 1 & 1\\ 1 & 2 & 1\\ 1 & 1 & 2\end{bmatrix}$.  Compute the eigenvalues of $C$ using the characteristic equation.
\begin{explanation}
\begin{align*}\det(C-\lambda I)&=\begin{vmatrix}2-\lambda & 1 & 1\\ 1 & 2-\lambda & 1\\ 1 & 1 & 2-\lambda\end{vmatrix}\\
&=(2-\lambda)\begin{vmatrix}2-\lambda&1\\1&2-\lambda\end{vmatrix}-1\begin{vmatrix}1&1\\1&2-\lambda\end{vmatrix}+1\begin{vmatrix}1&2-\lambda\\1&1\end{vmatrix}\\
&=(2-\lambda)^3-(2-\lambda)-((2-\lambda)-1)+1-(2-\lambda)\\
&=8-12\lambda+6\lambda^2-\lambda^3-2+\lambda-2+\lambda+1+1-2+\lambda\\
&=4-9\lambda+6\lambda^2-\lambda^3\\
&=-(\lambda-4)(\lambda-1)^2
\end{align*}
Matrix $C$ has eigenvalues $\lambda_1=1$ and $\lambda_2=4$.
\end{explanation}
\end{example}

In Example \ref{ex:3x3eig}, the factor  $(\lambda-1)$ appears twice in the characteristic polynomial.  This repeated factor gives rise to the eigenvalue $\lambda_1=1$.  We say that $\lambda_1=1$ has \dfn{algebraic multiplicity} $2$. 

The three examples above are a bit contrived.  It is not always possible to completely factor the characteristic polynomial.  However, a fundamental fact from algebra is that every degree $n$ polynomial has $n$ roots (counting multiplicity) provided that we allow complex numbers.  This is why sometimes eigenvalues and their corresponding eigenvectors involve complex numbers.  The next example illustrates this point.

\begin{example}\label{ex:3x3_complex_eig}
Let $D=\begin{bmatrix} 0&0&0\\ 0 &1&1\\ 0 & -1&1\end{bmatrix}$.  Compute the eigenvalues of this matrix.

\begin{explanation}
\begin{align*}\det(D-\lambda I)&=\begin{vmatrix} -\lambda&0&0\\ 0 &1-\lambda&1\\ 0 & -1&1-\lambda\end{vmatrix}\\
&=-\lambda\begin{vmatrix}1-\lambda&1\\-1&1-\lambda\end{vmatrix}\\
&=-\lambda((1-\lambda)^2+1)\\
&=-\lambda(\lambda^2-2\lambda+2)
\end{align*}
So one of the eigenvalues of $D$ is $\lambda=0$.  To get the other eigenvalues we must solve $\lambda^2-2\lambda+2=0$.  Using the quadratic formula, we compute that $\lambda_1=1+i$ and $\lambda_2=1-i$ are also eigenvalues of $D$.
\end{explanation}
\end{example}

\begin{exploration}\label{init:3x3tri}
Let $T=\begin{bmatrix} 1 & 2 & 3\\ 0 & 5 & 6\\ 0 & 0 & 9\end{bmatrix}$.  Compute the eigenvalues of this matrix.

$$\lambda_1=\answer{1},\quad\lambda_2=\answer{5},\quad\text{and}\quad\lambda_3=\answer{9}$$

What do you observe about the eigenvalues?
\begin{hint}
The eigenvalues are the diagonal entries of the matrix.
\end{hint}

What property of the matrix makes this ``coincidence" possible?

\begin{hint}
$T$ is a triangular matrix.
\end{hint}
\end{exploration}

The matrix in Exploration Problem \ref{init:3x3tri} is a triangular matrix, and the property you observed holds in general.

\begin{theorem}\label{th:eigtri}
Let $T$ be a triangular matrix.  Then the eigenvalues of $T$ are the entries on the main diagonal.
\end{theorem}

\begin{proof}
See Practice Problem \ref{prob:eigtri}.
\end{proof}

\begin{corollary}\label{th:eigdiag}
Let $D$ be a diagonal matrix.  Then the eigenvalues of $D$ are the entries on the main diagonal.
\end{corollary}

One final note about eigenvalues.  We began this section with the sentence, "In theory, then, to find the eigenvalues of $A$, one can solve Equation (\ref{eqn:chareqn}) for $\lambda$."  In general, one does not attempt to compute eigenvalues by solving the characteristic equation of a matrix, as there is no simple way to solve such an equation for $n>4$.  Instead, one can often approximate the eigenvalues using \dfn{iterative methods}.  

\subsection{Eigenvectors}
Once we have computed an eigenvalue $\lambda$ of an $n \times n$ matrix $A$, the next step is to compute the associated eigenvectors.  In other words, we seek vectors $\vec{x}$ such that $A\vec{x}=\lambda \vec{x}$, or equivalently,
\begin{align}\label{eqn:nullspace}
 (A-\lambda I) \vec{x}=\vec{0}   
\end{align} 
For any given eigenvalue $\lambda$ there are infinitely many eigenvectors associated with it.  In fact, the eigenvectors associated with $\lambda$ form a subspace of $\RR^n$. (see Practice Problems \ref{prob:eigenspace1} and \ref{prob:eigenspace2})  This motivates the following definition.

\begin{definition}\label{def:eigspace}
The set of all eigenvectors associated with a given eigenvalue of a matrix is known as the \dfn{eigenspace} associated with that eigenvalue.
\end{definition}

So given an eigenvalue $\lambda$, there is an associated eigenspace $\mathcal{S}$, and our goal is to find a basis of $\mathcal{S}$, for then any eigenvector $\vec{x}$ will be a linear combination of the vectors in that basis.  Moreover, we are trying to find a basis for the set of vectors that satisfy Equation \ref{eqn:nullspace}, which means we seek a basis for $\mbox{null}(A-\lambda I)$.  We have already learned how to compute a basis of a null space - see \href{https://ximera.osu.edu/oerlinalg/LinearAlgebra/VSP-0040/main}{Subspaces Associated with Matrices}.

Let's return to the examples we did in the first part of this section.

\begin{example}\label{ex:eigvect2x2eig} (Finding eigenvectors for Example \ref{ex:2x2eig} ) 

Recall that $A=\begin{bmatrix} 2& 1\\ 1&2
\end{bmatrix}$ has eigenvalues $\lambda_1=1$ and $\lambda_2=3$.  Compute a basis for the eigenspace associated with each of these eigenvalues.
\begin{explanation}
Eigenvectors associated with the eigenvalue $\lambda_1=1$ are in the null space of $A-I$.  So we seek a basis for $\mbox{null}(A-I)$.  We compute:
\begin{align*}\mbox{rref}(A-I)=\mbox{rref}\left(\begin{bmatrix}1&1\\1&1\end{bmatrix}\right)&=\begin{bmatrix}1&1\\0&0\end{bmatrix},
\end{align*}
From this we see that the eigenspace $\mathcal{S}_1$ associated with $\lambda_1=1$ consists of vectors of the form $\begin{bmatrix}-1\\1\end{bmatrix}t$.
%for any eigenvector $\begin{bmatrix}x_1\\x_2\end{bmatrix}$, we have $x_1+x_2=0$, so that $x_1=-x_2$  
This means that $\left\{\begin{bmatrix}-1\\1\end{bmatrix}\right\}$ is one possible basis for $\mathcal{S}_1$.

In a similar way, we compute a basis for $\mathcal{S}_3$, the subspace of all eigenvectors associated with the eigenvalue $\lambda_2=3$.  Now we compute:
\begin{align*}\mbox{rref}(A-3I)=\mbox{rref}\left(\begin{bmatrix}-1&1\\1&-1\end{bmatrix}\right)&=\begin{bmatrix}1&-1\\0&0\end{bmatrix},
\end{align*}
Vectors in the null space have the form $\begin{bmatrix}1\\1\end{bmatrix}t$ This means that $\left\{\begin{bmatrix}1\\1\end{bmatrix}\right\}$ is one possible basis for the eigenspace $\mathcal{S}_3$.
\end{explanation}
\end{example}

\begin{example}\label{ex:eigvectors2x2eig2} (Finding eigenvectors for Example \ref{ex:2x2eig2}) 
We know from Example \ref{ex:2x2eig2} that $B=\begin{bmatrix} 2& 1\\ 4&2
\end{bmatrix}$ has eigenvalues $\lambda_1=0$ and $\lambda_2=4$.  Compute a basis for the eigenspace associated with each of these eigenvalues.
\begin{explanation}
Let's begin by finding a basis for the eigenspace $\mathcal{S}_0$, which is the subspace of $\RR^n$ consisting of eigenvectors corresponding to the eigenvalue $\lambda_1=0$.  We need to compute a basis for $\mbox{null}(B-0I) = \mbox{null}(B)$.  We compute:
\begin{align*}\mbox{rref}(B)=\mbox{rref}\left(\begin{bmatrix}2&1\\4&2\end{bmatrix}\right)&=\begin{bmatrix}1&\frac{1}{2}\\0&0\end{bmatrix},
\end{align*}
From this we see that an eigenvector in $\mathcal{S}_0$ has the form $\begin{bmatrix}-1/2\\1\end{bmatrix}t$. %$\begin{bmatrix}x_1\\x_2\end{bmatrix}$ in $\mathcal{S}_0$ satisfies $x_1+\frac{1}{2} x_2=0$, so that $2x_1=-x_2$. 
This means that $\left\{\begin{bmatrix}-1/2\\1\end{bmatrix}\right\}$ is one possible basis for the eigenspace $\mathcal{S}_0$.  By letting $t=-2$, we obtain an arguably nicer-looking basis: $\left\{\begin{bmatrix}1\\-2\end{bmatrix}\right\}$. 

See if you can compute a basis for $\mathcal{S}_4$.  Click on the arrow if you need help.

\begin{hint}
To compute a basis for $\mathcal{S}_4$, the subspace of all eigenvectors associated to the eigenvalue $\lambda_2=4$, we compute:
\begin{align*}\mbox{rref}(B-4I)=\mbox{rref}\left(\begin{bmatrix}-2&1\\4&-2\end{bmatrix}\right)&=\begin{bmatrix}1&-\frac{1}{2}\\0&0\end{bmatrix},
\end{align*}
\end{hint}

From this we find that $\left\{\begin{bmatrix}1\\\answer{2}\end{bmatrix}\right\}$ is one possible basis for the eigenspace $\mathcal{S}_4$.
\end{explanation}
\end{example}

\begin{example}\label{ex:eigvectors3x3eig} (Finding eigenvectors for Example \ref{ex:3x3eig})
We know from Example \ref{ex:3x3eig} that $C=\begin{bmatrix} 2 & 1 & 1\\ 1 & 2 & 1\\ 1 & 1 & 2\end{bmatrix}$ has eigenvalues $\lambda_1=1$ and $\lambda_2=4$.  Compute a basis for the eigenspace associated to each of these eigenvalues.
\end{example}

\begin{explanation}
We first find a basis for the eigenspace $\mathcal{S}_1$.  We need to compute a basis for $\mbox{null}(C-I)$.  We compute:
\begin{align*}\mbox{rref}(C-I)=\mbox{rref}\left(\begin{bmatrix} 1 & 1 & 1\\ 1 & 1 & 1\\ 1 & 1 & 1\end{bmatrix}\right)&=\begin{bmatrix} 1 & 1 & 1\\ 0 & 0 & 0\\ 0 & 0 & 0\end{bmatrix},
\end{align*}
%From this we see for any eigenvector $\begin{bmatrix}x_1\\x_2\\x_3\end{bmatrix}$ in $\mathcal{S}_1$ satisfies $x_1+x_2+x_3=0$.  
Notice that there are two free variables.  %Let $x_2=s$ and $x_3=t$.
The eigenvectors in $\mathcal{S}_1$ have the form
$$\begin{bmatrix}-s-t\\s\\t\end{bmatrix} = s\begin{bmatrix}-1\\1\\0\end{bmatrix} + t\begin{bmatrix}-1\\0\\1\end{bmatrix}$$

So one possible basis for the eigenspace $\mathcal{S}_1$ is given by $\left\{\begin{bmatrix}-1\\1\\0\end{bmatrix}, \begin{bmatrix}-1\\0\\1\end{bmatrix}\right\}$.

Next we find a basis for the eigenspace $\mathcal{S}_4$.  We need to compute a basis for $\mbox{null}(C-4I)$.  We compute:
\begin{align*}\mbox{rref}(C-4I)=\mbox{rref}\left(\begin{bmatrix} -2 & 1 & 1\\ 1 & -2 & 1\\ 1 & 1 & -2\end{bmatrix}\right)&=\begin{bmatrix} 1 & 0 & -1\\ 0 & 1 & -1\\ 0 & 0 & 0\end{bmatrix}
\end{align*}
This time there is one free variable.  %Setting $x_3=t$, we also get $x_1=t$ and $x_2=t$.  From this we see
The eigenvectors in $\mathcal{S}_4$ have the form $\begin{bmatrix}t\\t\\t\end{bmatrix}$, so a possible basis for the eigenspace $\mathcal{S}_4$ is given by $\left\{\begin{bmatrix}1\\1\\1\end{bmatrix}\right\}$.
\end{explanation}

\begin{example}\label{ex:3x3_complex_ev} (Finding eigenvectors for Example \ref{ex:3x3_complex_eig})
We know from Example \ref{ex:3x3_complex_eig} that $D=\begin{bmatrix} 0&0&0\\ 0 &1&1\\ 0 & -1&1\end{bmatrix}$ has eigenvalues $\lambda=0$, $\lambda_1=1+i$, and $\lambda_2=1-i$.  Compute a basis for the eigenspace associated with each eigenvalue.
\begin{explanation}
We first find a basis for the eigenspace $\mathcal{S}_0$.  We need to compute a basis for $\mbox{null}(D-0I)=\mbox{null}(D)$.  We compute:
\begin{align*}\mbox{rref}(D)=\mbox{rref}\left(\begin{bmatrix} 0&0&0\\ 0 &1&1\\ 0 & -1&1\end{bmatrix}\right)&=\begin{bmatrix} 0 & 1 & 1\\ 0 & 0 & 1\\ 0 & 0 & 0\end{bmatrix},
\end{align*}
From this we see that for any eigenvector $\begin{bmatrix}x_1\\x_2\\x_3\end{bmatrix}$ in $\mathcal{S}_0$ we have $x_2=0$ and $x_3=0$, but $x_1$ is a free variable. 
So one possible basis for the eigenspace $\mathcal{S}_0$ is given by $$\left\{\begin{bmatrix}1\\0\\0\end{bmatrix}\right\}$$
Next we find a basis for the eigenspace $\mathcal{S}_{1+i}$.  We need to compute a basis for $\mbox{null}(D-(1+i)I)$.  We compute:
\begin{align*}\mbox{rref}(D-(1+i)I)&=\mbox{rref}\left(\begin{bmatrix} -(1+i)&0&0\\ 0 &1-(1+i)&1\\ 0 & -1&1-(1+i)\end{bmatrix}\right) \\
&=\begin{bmatrix} 1 & 0 &0\\ 0 & 1 & -i\\ 0 & 0 & 0\end{bmatrix}
\end{align*}
There is one free variable.  Setting $x_3=t$, we get $x_1=0$ and $x_2=ti$.  From this we see that eigenvectors in $\mathcal{S}_4$ have the form $\begin{bmatrix}0\\i\\1\end{bmatrix}t$, so a possible basis for the eigenspace $\mathcal{S}_{1+i}$ is given by $\left\{\begin{bmatrix}0\\i\\1\end{bmatrix}\right\}$.
We ask you in Practice Problem \ref{prob:3x3_complex_ev} to show that $\left\{\begin{bmatrix}0\\-i\\1\end{bmatrix}\right\}$ is a basis for $\mathcal{S}_{1-i}$.
\end{explanation}
\end{example}

We conclude this section by establishing the significance of a matrix having an eigenvalue of zero.

\begin{theorem}\label{th:zero_ew}
A square matrix has an eigenvalue of zero if and only if it is singular.
\end{theorem}

\begin{proof}
A square matrix $A$ is singular if and only if $\det{A}=0$.(see  \ref{th:detofsingularmatrix}).  But $\det{A}=0$ if and only if $\det{A-0I}=0$, which is true if and only if zero is an eigenvalue of $A$.
\end{proof}

\section*{Practice Problems}

\begin{problem}
In this exercise we will prove that the eigenvectors associated with an eigenvalue $\lambda$ of an $n \times n$ matrix $A$ form a subspace of $\RR^n$.
\begin{problem}\label{prob:eigenspace1}
Let $\vec{x}$ and $\vec{y}$ be eigenvectors of $A$ associated with $\lambda$.  Show that $\vec{x}+\vec{y}$ is also an eigenvector of $A$ associated with $\lambda$.  (This shows that the set of eigenvectors of $A$ associated with $\lambda$ is closed under addition).
\end{problem}
\begin{problem}\label{prob:eigenspace2}
Show that the set of eigenvectors of $A$ associated with $\lambda$ is closed under scalar multiplication.
\end{problem}
\end{problem}

\begin{problem}
Compute the eigenvalues of the given matrix and find the corresponding eigenspaces.
\begin{problem}\label{prob:eigenspace3}
$$\begin{bmatrix}4&1\\8&-3\end{bmatrix}$$
Answer:
(List the eigenvalues in an increasing order.)
$$\lambda_1=\answer{-4},\quad\lambda_2=\answer{5}$$

A basis for $\lambda_1$ is $\left\{\begin{bmatrix}\answer{-1/8}\\1\end{bmatrix}\right\}$.  A basis for $\lambda_2$ is $\left\{\begin{bmatrix}\answer{1}\\1\end{bmatrix}\right\}$.
\end{problem}
\begin{problem}\label{prob:eigenspace4}
$$\begin{bmatrix}1&-2\\2&1\end{bmatrix}$$
Answer:

$$\lambda_1=\answer{1}+\answer{2}i,\quad\lambda_2=\answer{1}-\answer{2}i$$

A basis for $\lambda_1$ is $\left\{\begin{bmatrix}i\\\answer{1}\end{bmatrix}\right\}$.  A basis for $\lambda_2$ is $\left\{\begin{bmatrix}-i\\\answer{1}\end{bmatrix}\right\}$.
\end{problem}
\end{problem}

\begin{problem}\label{prob:3x3tri_ev}
Let $T=\begin{bmatrix} 1 & 2 & 3\\ 0 & 5 & 6\\ 0 & 0 & 9\end{bmatrix}$.  Compute a basis for each of the eigenspaces of this matrix, $\mathcal{S}_1$, $\mathcal{S}_5$, and $\mathcal{S}_9$.
\end{problem}

\begin{problem}
Let $A=\begin{bmatrix} 9 & 2 & 8\\ 2 & -6 & -2\\ -8 & 2 & -5\end{bmatrix}$.  
\begin{problem}\label{prob:3x3fromKuttler1}
Compute the eigenvalues of this matrix. 
\begin{hint}
One of the eigenvalues of $A$ is -3.
\end{hint}

Answer:

(List your answers in an increasing order.)
$$\lambda_1 = \answer{-3},\quad \lambda_2 = \answer{-1},\quad \lambda_3 = \answer{2}$$
\end{problem}

\begin{problem}\label{prob:3x3fromKuttler2}
Compute a basis for each of the eigenspaces of this matrix,  $\mathcal{S}_{\lambda_1}$, $\mathcal{S}_{\lambda_2}$, and $\mathcal{S}_{\lambda_3}$.

Answer:
A basis for $\mathcal{S}_{\lambda_1}$ is $\left\{\begin{bmatrix}1\\\answer{2}\\\answer{-2}\end{bmatrix}\right\}$, 
a basis for $\mathcal{S}_{\lambda_2}$ is $\left\{\begin{bmatrix}-2\\\answer{-2}\\\answer{3}\end{bmatrix}\right\}$,

and a basis for $\mathcal{S}_{\lambda_3}$ is $\left\{\begin{bmatrix}2\\\answer{3}\\\answer{-2}\end{bmatrix}\right\}$.
\end{problem}
\end{problem}

\begin{problem}\label{prob:3x3_complex_ev}
Complete Example \ref{ex:3x3_complex_ev} by showing that a basis for $\mathcal{S}_{1-i}$ is given by $\left\{\begin{bmatrix}0\\-i\\1\end{bmatrix}\right\}$, where  $\mathcal{S}_{1-i}$ is the eigenspace associated with the eigenvalue $\lambda=1-i$ of the matrix $D=\begin{bmatrix} 0&0&0\\ 0 &1&1\\ 0 & -1&1\end{bmatrix}$.
\end{problem}


\begin{problem}\label{prob:eigtri}
Prove Theorem \ref{th:eigtri}.  (HINT:  Proceed by induction on the dimension n.  For the inductive step, compute $\det(A-\lambda I)$ by expanding along the first column (or row) if $T$ is upper (lower) triangular.)
\end{problem}

\begin{problem}
The following set of problems deals with geometric interpretation of eigenvalues and eigenvectors, as well as linear transformations of the plane.  Please use \href{https://ximera.osu.edu/oerlinalg/LinearAlgebra/EIG-0010/main}{Describing Eigenvalues and Eigenvectors Algebraically and Geometrically} and \href{https://ximera.osu.edu/oerlinalg/LinearAlgebra/LTR-0070/main}{Geometric Transformations of the Plane} for reference.
\begin{problem}\label{prob:eigvectorstransfr2_1}
Recall that a vertical stretch/compression of the plane is a linear transformation whose standard matrix is $$M_v=\begin{bmatrix}1&0\\0&k\end{bmatrix}$$
Find the eigenvalues of $M_v$.  Find a basis for the eigenspace corresponding to each eigenvalue.

Answer:  A basis for $\mathcal{S}_1$ is $\left\{\begin{bmatrix}1\\\answer{0}\end{bmatrix}\right\}$
and a basis for $\mathcal{S}_k$ is $\left\{\begin{bmatrix}\answer{0}\\1\end{bmatrix}\right\}$

Sketch several vectors in each eigenspace and use geometry to explain why the eigenvectors you sketched make sense.
\end{problem}

\begin{problem}\label{prob:eigvectorstransfr2_2}
Recall that a horizontal shear of the plane is a linear transformation whose standard matrix is $$M_{hs}=\begin{bmatrix}1&k\\0&1\end{bmatrix}$$
Find the eigenvalue of $M_{hs}$.  

Answer: $\lambda=\answer{1}$

Find a basis for the eigenspace corresponding to $\lambda$.

Answer:  A basis for $\mathcal{S}$ is $\left\{\begin{bmatrix}1\\\answer{0}\end{bmatrix}\right\}$

Sketch several vectors in the eigenspace and use geometry to explain why the eigenvectors you sketched make sense.
\end{problem}

\begin{problem}\label{prob:rotmatrixrealeig2}
Recall that a counterclockwise rotation of the plane through angle $\theta$ is a linear transformation whose standard matrix is $$M_{\theta}=\begin{bmatrix}\cos\theta&-\sin\theta\\\sin\theta&\cos\theta\end{bmatrix}$$
Verify that the eigenvalues of $M_{\theta}$ are
$$\lambda=\cos\theta\pm\sqrt{\cos^2\theta-1}$$
Explain why $\lambda$ is real number if and only if $\theta$ is a multiple of $\pi$.  (Compare this to Practice Problem \ref{prob:rotmatrixrealeig1} of \href{https://ximera.osu.edu/oerlinalg/LinearAlgebra/EIG-0010/main}{Describing Eigenvalues and Eigenvectors Algebraically and Geometrically}.)

Suppose $\theta$ is a muliple of $\pi$.  Then the eigenspaces corresponding to the two eigenvalues are the same.  Which of the following describes the eigenspace?
\begin{multipleChoice}
    \choice[correct]{All vectors in $\RR^2$.}
    \choice{All vectors along the $x$-axis.}
    \choice{All vectors along the $y$-axis.}
    \choice{All vectors along the line $y=x$.}
\end{multipleChoice}

\end{problem}

\begin{problem}\label{prob:eigvectorstransfr2_3}
Recall that a reflection of the plane about the line $y=mx$ is a linear transformation whose standard matrix is
$$M_{y=mx}=\frac{1}{1+m^2}\begin{bmatrix}
1-m^2 & 2m \\
2m & m^2-1
\end{bmatrix}$$
Verify that the eigenvalues of $M_{y=mx}$ are
$$\lambda_1=m^2+1\quad\text{and}\quad\lambda_2=-m^2-1$$
Find a basis for eigenspaces $\mathcal{S}_{\lambda_1}$ and $\mathcal{S}_{\lambda_2}$.  (For simplicity, assume that $m\neq 0$.)

Answer:  A basis for $\mathcal{S}_{\lambda_1}$ is $\left\{\begin{bmatrix}1/m\\\answer{1}\end{bmatrix}\right\}$
and a basis for $\mathcal{S}_{\lambda_2}$ is $\left\{\begin{bmatrix}-m\\\answer{1}\end{bmatrix}\right\}$

Choose the best description of $\mathcal{S}_{\lambda_1}$.
\begin{multipleChoice}
    \choice{All vectors in $\RR^2$.}
    \choice[correct]{All vectors with ``slope" $m$.}
    \choice{All vectors with ``slope" $1/m$.}
    \choice{All vectors with ``slope" $-1/m$.}
\end{multipleChoice}

Choose the best description of $\mathcal{S}_{\lambda_2}$.
\begin{multipleChoice}
    \choice{All vectors along the line $y=mx$.}
    \choice{All vectors parallel to the $x$-axis.}
    \choice{All vectors parallel to the $y$-axis.}
    \choice[correct]{All vectors perpendicular to the line $y=mx$.}
\end{multipleChoice}

Use geometry to explain why the eigenspaces you found make sense.

\end{problem}
\end{problem}

\section*{Exercise Source}
Practice Problem \ref{prob:3x3fromKuttler1} is adopted from Problem 7.1.11 of Ken Kuttler's \href{https://open.umn.edu/opentextbooks/textbooks/a-first-course-in-linear-algebra-2017}{\it A First Course in Linear Algebra}. (CC-BY)

Ken Kuttler, {\it  A First Course in Linear Algebra}, Lyryx 2017, Open Edition, p. 361.

\end{document}
