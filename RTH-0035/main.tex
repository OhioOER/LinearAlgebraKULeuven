\documentclass{ximera}
%%% Begin Laad packages

\makeatletter
\@ifclassloaded{xourse}{%
    \typeout{Start loading preamble.tex (in a XOURSE)}%
    \def\isXourse{true}   % automatically defined; pre 112022 it had to be set 'manually' in a xourse
}{%
    \typeout{Start loading preamble.tex (NOT in a XOURSE)}%
}
\makeatother

\def\isEn\true 

\pgfplotsset{compat=1.16}

\usepackage{currfile}

% 201908/202301: PAS OP: babel en doclicense lijken problemen te veroorzaken in .jax bestand
% (wegens syntax error met toegevoegde \newcommands ...)
\pdfOnly{
    \usepackage[type={CC},modifier={by-nc-sa},version={4.0}]{doclicense}
    %\usepackage[hyperxmp=false,type={CC},modifier={by-nc-sa},version={4.0}]{doclicense}
    %%% \usepackage[dutch]{babel}
}



\usepackage[utf8]{inputenc}
\usepackage{morewrites}   % nav zomercursus (answer...?)
\usepackage{multirow}
\usepackage{multicol}
\usepackage{tikzsymbols}
\usepackage{ifthen}
%\usepackage{animate} BREAKS HTML STRUCTURE USED BY XIMERA
\usepackage{relsize}

\usepackage{eurosym}    % \euro  (€ werkt niet in xake ...?)
\usepackage{fontawesome} % smileys etc

% Nuttig als ook interactieve beamer slides worden voorzien:
\providecommand{\p}{} % default nothing ; potentially usefull for slides: redefine as \pause
%providecommand{\p}{\pause}

    % Layout-parameters voor het onderschrift bij figuren
\usepackage[margin=10pt,font=small,labelfont=bf, labelsep=endash,format=hang]{caption}
%\usepackage{caption} % captionof
%\usepackage{pdflscape}    % landscape environment

% Met "\newcommand\showtodonotes{}" kan je todonotes tonen (in pdf/online)
% 201908: online werkt het niet (goed)
\providecommand\showtodonotes{disable}
\providecommand\todo[1]{\typeout{TODO #1}}
%\usepackage[\showtodonotes]{todonotes}
%\usepackage{todonotes}

%
% Poging tot aanpassen layout
%
\usepackage{tcolorbox}
\tcbuselibrary{theorems}

%%% Einde laad packages

%%% Begin Ximera specifieke zaken

\graphicspath{
	{../../}
	{../}
	{./}
  	{../../pictures/}
   	{../pictures/}
   	{./pictures/}
	{./explog/}    % M05 in groeimodellen       
}

%%% Einde Ximera specifieke zaken

%
% define softer blue/red/green, use KU Leuven base colors for blue (and dark orange for red ?)
%
% todo: rather redefine blue/red/green ...?
%\definecolor{xmblue}{rgb}{0.01, 0.31, 0.59}
%\definecolor{xmred}{rgb}{0.89, 0.02, 0.17}
\definecolor{xmdarkblue}{rgb}{0.122, 0.671, 0.835}   % KU Leuven Blauw
\definecolor{xmblue}{rgb}{0.114, 0.553, 0.69}        % KU Leuven Blauw
\definecolor{xmgreen}{rgb}{0.13, 0.55, 0.13}         % No KULeuven variant for green found ...

\definecolor{xmaccent}{rgb}{0.867, 0.541, 0.18}      % KU Leuven Accent (orange ...)
\definecolor{kuaccent}{rgb}{0.867, 0.541, 0.18}      % KU Leuven Accent (orange ...)

\colorlet{xmred}{xmaccent!50!black}                  % Darker version of KU Leuven Accent

\providecommand{\blue}[1]{{\color{blue}#1}}    
\providecommand{\red}[1]{{\color{red}#1}}

\renewcommand\CancelColor{\color{xmaccent!50!black}}

% werkt in math en text mode om MATH met oranje (of grijze...)  achtergond te tonen (ook \important{\text{blabla}} lijkt te werken)
%\newcommand{\important}[1]{\ensuremath{\colorbox{xmaccent!50!white}{$#1$}}}   % werkt niet in Mathjax
%\newcommand{\important}[1]{\ensuremath{\colorbox{lightgray}{$#1$}}}
\newcommand{\important}[1]{\ensuremath{\colorbox{orange}{$#1$}}}   % TODO: kleur aanpassen voor mathjax; wordt overschreven infra!


% Uitzonderlijk kan met \pdfnl in de PDF een newline worden geforceerd, die online niet nodig/nuttig is omdat daar de regellengte hoe dan ook niet gekend is.
\ifdefined\HCode%
\providecommand{\pdfnl}{}%
\else%
\providecommand{\pdfnl}{%
  \\%
}%
\fi

% Uitzonderlijk kan met \handoutnl in de handout-PDF een newline worden geforceerd, die noch online noch in de PDF-met-antwoorden nuttig is.
\ifdefined\HCode
\providecommand{\handoutnl}{}
\else
\providecommand{\handoutnl}{%
\ifhandout%
  \nl%
\fi%
}
\fi



% \cellcolor IGNORED by tex4ht ?
% \begin{center} seems not to wordk
    % (missing margin-left: auto;   on tabular-inside-center ???)
%\newcommand{\importantcell}[1]{\ensuremath{\cellcolor{lightgray}#1}}  %  in tabular; usablility to be checked
\providecommand{\importantcell}[1]{\ensuremath{#1}}     % no mathjax2 support for colloring array cells

\pdfOnly{
  \renewcommand{\important}[1]{\ensuremath{\colorbox{kuaccent!50!white}{$#1$}}}
  \renewcommand{\importantcell}[1]{\ensuremath{\cellcolor{kuaccent!40!white}#1}}   
}

%%% Tikz styles


\pgfplotsset{compat=1.16}

\usetikzlibrary{trees,positioning,arrows,fit,shapes,math,calc,decorations.markings,through,intersections,patterns,matrix}

\usetikzlibrary{decorations.pathreplacing,backgrounds}    % 5/2023: from experimental


\usetikzlibrary{angles,quotes}

\usepgfplotslibrary{fillbetween} % bepaalde_integraal
\usepgfplotslibrary{polar}    % oa voor poolcoordinaten.tex

\pgfplotsset{ownstyle/.style={axis lines = center, axis equal image, xlabel = $x$, ylabel = $y$, enlargelimits}} 

\pgfplotsset{
	plot/.style={no marks,samples=50}
}

\newcommand{\xmPlotsColor}{
	\pgfplotsset{
		plot1/.style={darkgray,no marks,samples=100},
		plot2/.style={lightgray,no marks,samples=100},
		plotresult/.style={blue,no marks,samples=100},
		plotblue/.style={blue,no marks,samples=100},
		plotred/.style={red,no marks,samples=100},
		plotgreen/.style={green,no marks,samples=100},
		plotpurple/.style={purple,no marks,samples=100}
	}
}
\newcommand{\xmPlotsBlackWhite}{
	\pgfplotsset{
		plot1/.style={black,loosely dashed,no marks,samples=100},
		plot2/.style={black,loosely dotted,no marks,samples=100},
		plotresult/.style={black,no marks,samples=100},
		plotblue/.style={black,no marks,samples=100},
		plotred/.style={black,dotted,no marks,samples=100},
		plotgreen/.style={black,dashed,no marks,samples=100},
		plotpurple/.style={black,dashdotted,no marks,samples=100}
	}
}


\newcommand{\xmPlotsColorAndStyle}{
	\pgfplotsset{
		plot1/.style={darkgray,no marks,samples=100},
		plot2/.style={lightgray,no marks,samples=100},
		plotresult/.style={blue,no marks,samples=100},
		plotblue/.style={xmblue,no marks,samples=100},
		plotred/.style={xmred,dashed,thick,no marks,samples=100},
		plotgreen/.style={xmgreen,dotted,very thick,no marks,samples=100},
		plotpurple/.style={purple,no marks,samples=100}
	}
}


%\iftikzexport
\xmPlotsColorAndStyle
%\else
%\xmPlotsBlackWhite
%\fi
%%%


%
% Om venndiagrammen te arceren ...
%
\makeatletter
\pgfdeclarepatternformonly[\hatchdistance,\hatchthickness]{north east hatch}% name
{\pgfqpoint{-1pt}{-1pt}}% below left
{\pgfqpoint{\hatchdistance}{\hatchdistance}}% above right
{\pgfpoint{\hatchdistance-1pt}{\hatchdistance-1pt}}%
{
	\pgfsetcolor{\tikz@pattern@color}
	\pgfsetlinewidth{\hatchthickness}
	\pgfpathmoveto{\pgfqpoint{0pt}{0pt}}
	\pgfpathlineto{\pgfqpoint{\hatchdistance}{\hatchdistance}}
	\pgfusepath{stroke}
}
\pgfdeclarepatternformonly[\hatchdistance,\hatchthickness]{north west hatch}% name
{\pgfqpoint{-\hatchthickness}{-\hatchthickness}}% below left
{\pgfqpoint{\hatchdistance+\hatchthickness}{\hatchdistance+\hatchthickness}}% above right
{\pgfpoint{\hatchdistance}{\hatchdistance}}%
{
	\pgfsetcolor{\tikz@pattern@color}
	\pgfsetlinewidth{\hatchthickness}
	\pgfpathmoveto{\pgfqpoint{\hatchdistance+\hatchthickness}{-\hatchthickness}}
	\pgfpathlineto{\pgfqpoint{-\hatchthickness}{\hatchdistance+\hatchthickness}}
	\pgfusepath{stroke}
}
%\makeatother

\tikzset{
    hatch distance/.store in=\hatchdistance,
    hatch distance=10pt,
    hatch thickness/.store in=\hatchthickness,
   	hatch thickness=2pt
}

\colorlet{circle edge}{black}
\colorlet{circle area}{blue!20}


\tikzset{
    filled/.style={fill=green!30, draw=circle edge, thick},
    arceerl/.style={pattern=north east hatch, pattern color=blue!50, draw=circle edge},
    arceerr/.style={pattern=north west hatch, pattern color=yellow!50, draw=circle edge},
    outline/.style={draw=circle edge, thick}
}




%%% Updaten commando's
\def\hoofding #1#2#3{\maketitle}     % OBSOLETE ??

% we willen (bijna) altijd \geqslant ipv \geq ...!
\newcommand{\geqnoslant}{\geq}
\renewcommand{\geq}{\geqslant}
\newcommand{\leqnoslant}{\leq}
\renewcommand{\leq}{\leqslant}

% Todo: (201908) waarom komt er (soms) underlined voor emph ...?
\renewcommand{\emph}[1]{\textit{#1}}

% API commando's

\newcommand{\ds}{\displaystyle}
\newcommand{\ts}{\textstyle}  % tegenhanger van \ds   (Ximera zet PER  DEFAULT \ds!)

% uit Zomercursus-macro's: 
\newcommand{\bron}[1]{\begin{scriptsize} \emph{#1} \end{scriptsize}}     % deprecated ...?


%definities nieuwe commando's - afkortingen veel gebruikte symbolen
\newcommand{\R}{\ensuremath{\mathbb{R}}}
\newcommand{\Rnul}{\ensuremath{\mathbb{R}_0}}
\newcommand{\Reen}{\ensuremath{\mathbb{R}\setminus\{1\}}}
\newcommand{\Rnuleen}{\ensuremath{\mathbb{R}\setminus\{0,1\}}}
\newcommand{\Rplus}{\ensuremath{\mathbb{R}^+}}
\newcommand{\Rmin}{\ensuremath{\mathbb{R}^-}}
\newcommand{\Rnulplus}{\ensuremath{\mathbb{R}_0^+}}
\newcommand{\Rnulmin}{\ensuremath{\mathbb{R}_0^-}}
\newcommand{\Rnuleenplus}{\ensuremath{\mathbb{R}^+\setminus\{0,1\}}}
\newcommand{\N}{\ensuremath{\mathbb{N}}}
\newcommand{\Nnul}{\ensuremath{\mathbb{N}_0}}
\newcommand{\Z}{\ensuremath{\mathbb{Z}}}
\newcommand{\Znul}{\ensuremath{\mathbb{Z}_0}}
\newcommand{\Zplus}{\ensuremath{\mathbb{Z}^+}}
\newcommand{\Zmin}{\ensuremath{\mathbb{Z}^-}}
\newcommand{\Znulplus}{\ensuremath{\mathbb{Z}_0^+}}
\newcommand{\Znulmin}{\ensuremath{\mathbb{Z}_0^-}}
\newcommand{\C}{\ensuremath{\mathbb{C}}}
\newcommand{\Cnul}{\ensuremath{\mathbb{C}_0}}
\newcommand{\Cplus}{\ensuremath{\mathbb{C}^+}}
\newcommand{\Cmin}{\ensuremath{\mathbb{C}^-}}
\newcommand{\Cnulplus}{\ensuremath{\mathbb{C}_0^+}}
\newcommand{\Cnulmin}{\ensuremath{\mathbb{C}_0^-}}
\newcommand{\Q}{\ensuremath{\mathbb{Q}}}
\newcommand{\Qnul}{\ensuremath{\mathbb{Q}_0}}
\newcommand{\Qplus}{\ensuremath{\mathbb{Q}^+}}
\newcommand{\Qmin}{\ensuremath{\mathbb{Q}^-}}
\newcommand{\Qnulplus}{\ensuremath{\mathbb{Q}_0^+}}
\newcommand{\Qnulmin}{\ensuremath{\mathbb{Q}_0^-}}

\newcommand{\perdef}{\overset{\mathrm{def}}{=}}
\newcommand{\pernot}{\overset{\mathrm{notatie}}{=}}
\newcommand\perinderdaad{\overset{!}{=}}     % voorlopig gebruikt in limietenrekenregels
\newcommand\perhaps{\overset{?}{=}}          % voorlopig gebruikt in limietenrekenregels

\newcommand{\degree}{^\circ}


\DeclareMathOperator{\dom}{dom}     % domein
\DeclareMathOperator{\codom}{codom} % codomein
\DeclareMathOperator{\bld}{bld}     % beeld
\DeclareMathOperator{\graf}{graf}   % grafiek
\DeclareMathOperator{\rico}{rico}   % richtingcoëfficient
\DeclareMathOperator{\co}{co}       % coordinaat
\DeclareMathOperator{\gr}{gr}       % graad

\newcommand{\func}[5]{\ensuremath{#1: #2 \rightarrow #3: #4 \mapsto #5}} % Easy to write a function


% Operators
\DeclareMathOperator{\bgsin}{bgsin}
\DeclareMathOperator{\bgcos}{bgcos}
\DeclareMathOperator{\bgtan}{bgtan}
\DeclareMathOperator{\bgcot}{bgcot}
\DeclareMathOperator{\bgsinh}{bgsinh}
\DeclareMathOperator{\bgcosh}{bgcosh}
\DeclareMathOperator{\bgtanh}{bgtanh}
\DeclareMathOperator{\bgcoth}{bgcoth}

% Oude \Bgsin etc deprecated: gebruik \bgsin, en herdefinieer dat als je Bgsin wil!
%\DeclareMathOperator{\cosec}{cosec}    % not used? gebruik \csc en herdefinieer

% operatoren voor differentialen: to be verified; 1/2020: inconsequent gebruik bij afgeleiden/integralen
\renewcommand{\d}{\mathrm{d}}
\newcommand{\dx}{\d x}
\newcommand{\dd}[1]{\frac{\mathrm{d}}{\mathrm{d}#1}}
\newcommand{\ddx}{\dd{x}}

% om in voorbeelden/oefeningen de notatie voor afgeleiden te kunnen kiezen
% Usage: \afg{(2\sin(x))}  (en wordt d/dx, of accent, of D )
%\newcommand{\afg}[1]{{#1}'}
\newcommand{\afg}[1]{\left(#1\right)'}
%\renewcommand{\afg}[1]{\frac{\mathrm{d}#1}{\mathrm{d}x}}   % include in relevant exercises ...
%\renewcommand{\afg}[1]{D{#1}}

%
% \xmxxx commands: Extra KU Leuven functionaliteit van, boven of naast Ximera
%   ( Conventie 8/2019: xm+nederlandse omschrijving, maar is niet consequent gevolgd, en misschien ook niet erg handig !)
%
% (Met een minimale ximera.cls en preamble.tex zou een bruikbare .pdf moeten kunnen worden gemaakt van eender welke ximera)
%
% Usage: \xmtitle[Mijn korte abstract]{Mijn titel}{Mijn abstract}
% Eerste command na \begin{document}:
%  -> definieert de \title
%  -> definieert de abstract
%  -> doet \maketitle ( dus: print de hoofding als 'chapter' of 'sectie')
% Optionele parameter geeft eenn kort abstract (die met de globale setting \xmshortabstract{} al dan niet kan worden geprint.
% De optionele korte abstract kan worden gebruikt voor pseudo-grappige abtsarts, dus dus globaal al dan niet kunnen worden gebuikt...
% Globale settings:
%  de (optionele) 'korte abstract' wordt enkele getoond als \xmshortabstract is gezet
\providecommand\xmshortabstract{} % default: print (only!) short abstract if present
\newcommand{\xmtitle}[3][]{
	\title{#2}
	\begin{abstract}
		\ifdefined\xmshortabstract
		\ifstrempty{#1}{%
			#3
		}{%
			#1
		}%
		\else
		#3
		\fi
	\end{abstract}
	\maketitle
}

% 
% Kleine grapjes: moeten zonder verder gevolg kunnen worden verwijderd
%
%\newcommand{\xmopje}[1]{{\small#1{\reversemarginpar\marginpar{\Smiley}}}}   % probleem in floats!!
\newtoggle{showxmopje}
\toggletrue{showxmopje}

\newcommand{\xmopje}[1]{%
   \iftoggle{showxmopje}{#1}{}%
}


% -> geef een abstracte-formule-met-rechts-een-concreet-voorbeeld
% VB:  \formulevb{a^2+b^2=c^2}{3^2+4^2=5^2}
%
\ifdefined\HCode
\NewEnviron{xmdiv}[1]{\HCode{\Hnewline<div class="#1">\Hnewline}\BODY{\HCode{\Hnewline</div>\Hnewline}}}
\else
\NewEnviron{xmdiv}[1]{\BODY}
\fi

\providecommand{\formulevb}[2]{
	{\centering

    \begin{xmdiv}{xmformulevb}    % zie css voor online layout !!!
	\begin{tabular}{lcl}
		\important{#1}
		&  &
		Vb: $#2$
		\end{tabular}
	\end{xmdiv}

	}
}

\ifdefined\HCode
\providecommand{\vb}[1]{%
    \HCode{\Hnewline<span class="xmvb">}#1\HCode{</span>\Hnewline}%
}
\else
\providecommand{\vb}[1]{
    \colorbox{blue!10}{#1}
}
\fi

\ifdefined\HCode
\providecommand{\xmcolorbox}[2]{
	\HCode{\Hnewline<div class="xmcolorbox">\Hnewline}#2\HCode{\Hnewline</div>\Hnewline}
}
\else
\providecommand{\xmcolorbox}[2]{
  \cellcolor{#1}#2
}
\fi


\ifdefined\HCode
\providecommand{\xmopmerking}[1]{
 \HCode{\Hnewline<div class="xmopmerking">\Hnewline}#1\HCode{\Hnewline</div>\Hnewline}
}
\else
\providecommand{\xmopmerking}[1]{
	{\footnotesize #1}
}
\fi
% \providecommand{\voorbeeld}[1]{
% 	\colorbox{blue!10}{$#1$}
% }



% Hernoem Proof naar Bewijs, nodig voor HTML versie
\renewcommand*{\proofname}{Bewijs}

% Om opgave van oefening (wordt niet geprint bij oplossingenblad)
% (to be tested test)
\NewEnviron{statement}{\BODY}

% Environment 'oplossing' en 'uitkomst'
% voor resp. volledige 'uitwerking' dan wel 'enkel eindresultaat'
% geimplementeerd via feedback, omdat er in de ximera-server adhoc feedback-code is toegevoegd
%% Niet tonen indien handout
%% Te gebruiken om volledige oplossingen/uitwerkingen van oefeningen te tonen
%% \begin{oplossing}        De optelling is commutatief \end{oplossing}  : verschijnt online enkel 'op vraag'
%% \begin{oplossing}[toon]  De optelling is commutatief \end{oplossing}  : verschijnt steeds onmiddellijk online (bv te gebruiken bij voorbeelden) 

\ifhandout%
    \NewEnviron{oplossing}[1][onzichtbaar]%
    {%
    \ifthenelse{\equal{\detokenize{#1}}{\detokenize{toon}}}
    {
    \def\PH@Command{#1}% Use PH@Command to hold the content and be a target for "\expandafter" to expand once.

    \begin{trivlist}% Begin the trivlist to use formating of the "Feedback" label.
    \item[\hskip \labelsep\small\slshape\bfseries Oplossing% Format the "Feedback" label. Don't forget the space.
    %(\texttt{\detokenize\expandafter{\PH@Command}}):% Format (and detokenize) the condition for feedback to trigger
    \hspace{2ex}]\small%\slshape% Insert some space before the actual feedback given.
    \BODY
    \end{trivlist}
    }
    {  % \begin{feedback}[solution]   \BODY     \end{feedback}  }
    }
    }    
\else
% ONLY for HTML; xmoplossing is styled with css, and is not, and need not be a LaTeX environment
% THUS: it does NOT use feedback anymore ...
%    \NewEnviron{oplossing}{\begin{expandable}{xmoplossing}{\nlen{Toon uitwerking}{Show solution}}{\BODY}\end{expandable}}
    \newenvironment{oplossing}[1][onzichtbaar]
   {%
       \begin{expandable}{xmoplossing}{}
   }
   {%
   	   \end{expandable}
   } 
%     \newenvironment{oplossing}[1][onzichtbaar]
%    {%
%        \begin{feedback}[solution]   	
%    }
%    {%
%    	   \end{feedback}
%    } 
\fi

\ifhandout%
    \NewEnviron{uitkomst}[1][onzichtbaar]%
    {%
    \ifthenelse{\equal{\detokenize{#1}}{\detokenize{toon}}}
    {
    \def\PH@Command{#1}% Use PH@Command to hold the content and be a target for "\expandafter" to expand once.

    \begin{trivlist}% Begin the trivlist to use formating of the "Feedback" label.
    \item[\hskip \labelsep\small\slshape\bfseries Uitkomst:% Format the "Feedback" label. Don't forget the space.
    %(\texttt{\detokenize\expandafter{\PH@Command}}):% Format (and detokenize) the condition for feedback to trigger
    \hspace{2ex}]\small%\slshape% Insert some space before the actual feedback given.
    \BODY
    \end{trivlist}
    }
    {  % \begin{feedback}[solution]   \BODY     \end{feedback}  }
    }
    }    
\else
\ifdefined\HCode
   \newenvironment{uitkomst}[1][onzichtbaar]
    {%
        \begin{expandable}{xmuitkomst}{}%
    }
    {%
    	\end{expandable}%
    } 
\else
  % Do NOT print 'uitkomst' in non-handout
  %  (presumably, there is also an 'oplossing' ??)
  \newenvironment{uitkomst}[1][onzichtbaar]{}{}
\fi
\fi

%
% Uitweidingen zijn extra's die niet redelijkerwijze tot de leerstof behoren
% Uitbreidingen zijn extra's die wel redelijkerwijze tot de leerstof van bv meer geavanceerde versies kunnen behoren (B-programma/Wiskundestudenten/...?)
% Nog niet voorzien: design voor verschillende versies (A/B programma, BIO, voorkennis/ ...)
% Voor 'uitweidingen' is er een environment die online per default is ingeklapt, en in pdf al dan niet kan worden geincluded  (via \xmnouitweiding) 
%
% in een xourse, per default GEEN uitweidingen, tenzij \xmuitweiding expliciet ergens is gezet ...
\ifdefined\isXourse
   \ifdefined\xmuitweiding
   \else
       \def\xmnouitweiding{true}
   \fi
\fi

\ifdefined\xmnouitweiding
\newcounter{xmuitweiding}  % anders error undefined ...  
\excludecomment{xmuitweiding}
\else
\newtheoremstyle{dotless}{}{}{}{}{}{}{ }{}
\theoremstyle{dotless}
\newtheorem*{xmuitweidingnofrills}{}   % nofrills = no accordion; gebruikt dus de dotless theoremstyle!

\newcounter{xmuitweiding}
\newenvironment{xmuitweiding}[1][ ]%
{% 
	\refstepcounter{xmuitweiding}%
    \begin{expandable}{xmuitweiding}{\nlentext{Uitweiding \arabic{xmuitweiding}: #1}{Digression \arabic{xmuitweiding}: #1}}%
	\begin{xmuitweidingnofrills}%
}
{%
    \end{xmuitweidingnofrills}%
    \end{expandable}%
}   
% \newenvironment{xmuitweiding}[1][ ]%
% {% 
% 	\refstepcounter{xmuitweiding}
% 	\begin{accordion}\begin{accordion-item}[Uitweiding \arabic{xmuitweiding}: #1]%
% 			\begin{xmuitweidingnofrills}%
% 			}
% 			{\end{xmuitweidingnofrills}\end{accordion-item}\end{accordion}}   
\fi


\newenvironment{xmexpandable}[1][]{
	\begin{accordion}\begin{accordion-item}[#1]%
		}{\end{accordion-item}\end{accordion}}


% Command that gives a selection box online, but just prints the right answer in pdf
\newcommand{\xmonlineChoice}[1]{\pdfOnly{\wordchoicegiventrue}\wordChoice{#1}\pdfOnly{\wordchoicegivenfalse}}   % deprecated, gebruik onlineChoice ...
\newcommand{\onlineChoice}[1]{\pdfOnly{\wordchoicegiventrue}\wordChoice{#1}\pdfOnly{\wordchoicegivenfalse}}

\newcommand{\TJa}{\nlentext{ Ja }{ Yes }}
\newcommand{\TNee}{\nlentext{ Nee }{ No }}
\newcommand{\TJuist}{\nlentext{ Juist }{ True }}
\newcommand{\TFout}{\nlentext{ Fout }{ False }}

\newcommand{\choiceTrue }{{\renewcommand{\choiceminimumhorizontalsize}{4em}\wordChoice{\choice[correct]{\TJuist}\choice{\TFout}}}}
\newcommand{\choiceFalse}{{\renewcommand{\choiceminimumhorizontalsize}{4em}\wordChoice{\choice{\TJuist}\choice[correct]{\TFout}}}}

\newcommand{\choiceYes}{{\renewcommand{\choiceminimumhorizontalsize}{3em}\wordChoice{\choice[correct]{\TJa}\choice{\TNee}}}}
\newcommand{\choiceNo }{{\renewcommand{\choiceminimumhorizontalsize}{3em}\wordChoice{\choice{\TJa}\choice[correct]{\TNee}}}}

% Optional nicer formatting for wordChoice in PDF

\let\inlinechoiceorig\inlinechoice

%\makeatletter
%\providecommand{\choiceminimumverticalsize}{\vphantom{$\frac{\sqrt{2}}{2}$}}   % minimum height of boxes (cfr infra)
\providecommand{\choiceminimumverticalsize}{\vphantom{$\tfrac{2}{2}$}}   % minimum height of boxes (cfr infra)
\providecommand{\choiceminimumhorizontalsize}{1em}   % minimum width of boxes (cfr infra)

\newcommand{\inlinechoicesquares}[2][]{%
		\setkeys{choice}{#1}%
		\ifthenelse{\boolean{\choice@correct}}%
		{%
            \ifhandout%
               \fbox{\choiceminimumverticalsize #2}\allowbreak\ignorespaces%
            \else%
               \fcolorbox{blue}{blue!20}{\choiceminimumverticalsize #2}\allowbreak\ignorespaces\setkeys{choice}{correct=false}\ignorespaces%
            \fi%
		}%
		{% else
			\fbox{\choiceminimumverticalsize #2}\allowbreak\ignorespaces%  HACK: wat kleiner, zodat fits on line ... 	
		}%
}

\newcommand{\inlinechoicesquareX}[2][]{%
		\setkeys{choice}{#1}%
		\ifthenelse{\boolean{\choice@correct}}%
		{%
            \ifhandout%
               \framebox[\ifdim\choiceminimumhorizontalsize<\width\width\else\choiceminimumhorizontalsize\fi]{\choiceminimumverticalsize\ #2\ }\allowbreak\ignorespaces\setkeys{choice}{correct=false}\ignorespaces%
            \else%
               \fcolorbox{blue}{blue!20}{\makebox[\ifdim\choiceminimumhorizontalsize<\width\width\else\choiceminimumhorizontalsize\fi]{\choiceminimumverticalsize #2}}\allowbreak\ignorespaces\setkeys{choice}{correct=false}\ignorespaces%
            \fi%
		}%
		{% else
        \ifhandout%
			\framebox[\ifdim\choiceminimumhorizontalsize<\width\width\else\choiceminimumhorizontalsize\fi]{\choiceminimumverticalsize\ #2\ }\allowbreak\ignorespaces%  HACK: wat kleiner, zodat fits on line ... 	
        \fi
		}%
}


\newcommand{\inlinechoicepuntjes}[2][]{%
		\setkeys{choice}{#1}%
		\ifthenelse{\boolean{\choice@correct}}%
		{%
            \ifhandout%
               \dots\ldots\ignorespaces\setkeys{choice}{correct=false}\ignorespaces
            \else%
               \fcolorbox{blue}{blue!20}{\choiceminimumverticalsize #2}\allowbreak\ignorespaces\setkeys{choice}{correct=false}\ignorespaces%
            \fi%
		}%
		{% else
			%\fbox{\choiceminimumverticalsize #2}\allowbreak\ignorespaces%  HACK: wat kleiner, zodat fits on line ... 	
		}%
}

% print niets, maar definieer globale variable \myanswer
%  (gebruikt om oplossingsbladen te printen) 
\newcommand{\inlinechoicedefanswer}[2][]{%
		\setkeys{choice}{#1}%
		\ifthenelse{\boolean{\choice@correct}}%
		{%
               \gdef\myanswer{#2}\setkeys{choice}{correct=false}

		}%
		{% else
			%\fbox{\choiceminimumverticalsize #2}\allowbreak\ignorespaces%  HACK: wat kleiner, zodat fits on line ... 	
		}%
}



%\makeatother

\newcommand{\setchoicedefanswer}{
\ifdefined\HCode
\else
%    \renewenvironment{multipleChoice@}[1][]{}{} % remove trailing ')'
    \let\inlinechoice\inlinechoicedefanswer
\fi
}

\newcommand{\setchoicepuntjes}{
\ifdefined\HCode
\else
    \renewenvironment{multipleChoice@}[1][]{}{} % remove trailing ')'
    \let\inlinechoice\inlinechoicepuntjes
\fi
}
\newcommand{\setchoicesquares}{
\ifdefined\HCode
\else
    \renewenvironment{multipleChoice@}[1][]{}{} % remove trailing ')'
    \let\inlinechoice\inlinechoicesquares
\fi
}
%
\newcommand{\setchoicesquareX}{
\ifdefined\HCode
\else
    \renewenvironment{multipleChoice@}[1][]{}{} % remove trailing ')'
    \let\inlinechoice\inlinechoicesquareX
\fi
}
%
\newcommand{\setchoicelist}{
\ifdefined\HCode
\else
    \renewenvironment{multipleChoice@}[1][]{}{)}% re-add trailing ')'
    \let\inlinechoice\inlinechoiceorig
\fi
}

\setchoicesquareX  % by default list-of-squares with onlineChoice in PDF

% Omdat multicols niet werkt in html: enkel in pdf  (in html zijn langere pagina's misschien ook minder storend)
\newenvironment{xmmulticols}[1][2]{
 \pdfOnly{\begin{multicols}{#1}}%
}{ \pdfOnly{\end{multicols}}}

%
% Te gebruiken in plaats van \section\subsection
%  (in een printstyle kan dan het level worden aangepast
%    naargelang \chapter vs \section style )
% 3/2021: DO NOT USE \xmsubsection !
\newcommand\xmsection\subsection
\newcommand\xmsubsection\subsubsection

% Aanpassen printversie
%  (hier gedefinieerd, zodat ze in xourse kunnen worden gezet/overschreven)
\providebool{parttoc}
\providebool{printpartfrontpage}
\providebool{printactivitytitle}
\providebool{printactivityqrcode}
\providebool{printactivityurl}
\providebool{printcontinuouspagenumbers}
\providebool{numberactivitiesbysubpart}
\providebool{addtitlenumber}
\providebool{addsectiontitlenumber}
\addtitlenumbertrue
\addsectiontitlenumbertrue

% The following three commands are hardcoded in xake, you can't create other commands like these, without adding them to xake as well
%  ( gebruikt in xourses om juiste soort titelpagina te krijgen voor verschillende ximera's )
\newcommand{\activitychapter}[2][]{
    {    
    \ifstrequal{#1}{notnumbered}{
        \addtitlenumberfalse
    }{}
    \typeout{ACTIVITYCHAPTER #2}   % logging
	\chapterstyle
	\activity{#2}
    }
}
\newcommand{\activitysection}[2][]{
    {
    \ifstrequal{#1}{notnumbered}{
        \addsectiontitlenumberfalse
    }{}
	\typeout{ACTIVITYSECTION #2}   % logging
	\sectionstyle
	\activity{#2}
    }
}
% Practices worden als activity getoond om de grote blokken te krijgen online
\newcommand{\practicesection}[2][]{
    {
    \ifstrequal{#1}{notnumbered}{
        \addsectiontitlenumberfalse
    }{}
    \typeout{PRACTICESECTION #2}   % logging
	\sectionstyle
	\activity{#2}
    }
}
\newcommand{\activitychapterlink}[3][]{
    {
    \ifstrequal{#1}{notnumbered}{
        \addtitlenumberfalse
    }{}
    \typeout{ACTIVITYCHAPTERLINK #3}   % logging
	\chapterstyle
	\activitylink[#1]{#2}{#3}
    }
}

\newcommand{\activitysectionlink}[3][]{
    {
    \ifstrequal{#1}{notnumbered}{
        \addsectiontitlenumberfalse
    }{}
    \typeout{ACTIVITYSECTIONLINK #3}   % logging
	\sectionstyle
	\activitylink[#1]{#2}{#3}
    }
}


% Commando om de printstyle toe te voegen in ximera's. Zorgt ervoor dat er geen problemen zijn als je de xourses compileert
% hack om onhandige relative paden in TeX te omzeilen
% should work both in xourse and ximera (pre-112022 only in ximera; thus obsoletes adhoc setup in xourses)
% loads global.sty if present (cfr global.css for online settings!)
% use global.sty to overwrite settings in printstyle.sty ...
\newcommand{\addPrintStyle}[1]{
\iftikzexport\else   % only in PDF
  \makeatletter
  \ifx\@onlypreamble\@notprerr\else   % ONLY if in tex-preamble   (and e.g. not when included from xourse)
    \typeout{Loading printstyle}   % logging
    \usepackage{#1/printstyle} % mag enkel geinclude worden als je die apart compileert
    \IfFileExists{#1/global.sty}{
        \typeout{Loading printstyle-folder #1/global.sty}   % logging
        \usepackage{#1/global}
        }{
        \typeout{Info: No extra #1/global.sty}   % logging
    }   % load global.sty if present
    \IfFileExists{global.sty}{
        \typeout{Loading local-folder global.sty (or TEXINPUTPATH..)}   % logging
        \usepackage{global}
    }{
        \typeout{Info: No folder/global.sty}   % logging
    }   % load global.sty if present
    \IfFileExists{\currfilebase.sty}
    {
        \typeout{Loading \currfilebase.sty}
        \input{\currfilebase.sty}
    }{
        \typeout{Info: No local \currfilebase.sty}
    }
    \fi
  \makeatother
\fi
}

%
%  
% references: Ximera heeft adhoc logica	 om online labels te doen werken over verschillende files heen
% met \hyperref kan de getoonde tekst toch worden opgegeven, in plaats van af te hangen van de label-text
\ifdefined\HCode
% Link to standard \labels, but give your own description
% Usage:  Volg \hyperref[my_very_verbose_label]{deze link} voor wat tijdverlies
%   (01/2020: Ximera-server aangepast om bij class reference-keeptext de link-text NIET te vervangen door de label-text !!!) 
\renewcommand{\hyperref}[2][]{\HCode{<a class="reference reference-keeptext" href="\##1">}#2\HCode{</a>}}
%
%  Link to specific targets  (not tested ?)
\renewcommand{\hypertarget}[1]{\HCode{<a class="ximera-label" id="#1"></a>}}
\renewcommand{\hyperlink}[2]{\HCode{<a class="reference reference-keeptext" href="\##1">}#2\HCode{</a>}}
\fi

% Mmm, quid English ... (for keyword #1 !) ?
\newcommand{\wikilink}[2]{
    \hyperlink{https://nl.wikipedia.org/wiki/#1}{#2}
    \pdfOnly{\footnote{See \url{https://nl.wikipedia.org/wiki/#1}}
    }
}

\renewcommand{\figurename}{Figuur}
\renewcommand{\tablename}{Tabel}

%
% Gedoe om verschillende versies van xourse/ximera te maken afhankelijk van settings
%
% default: versie met antwoorden
% handout: versie voor de studenten, zonder antwoorden/oplossingen
% full: met alles erop en eraan, dus geschikt voor auteurs en/of lesgevers  (bevat in de pdf ook de 'online-only' stukken!)
%
%
% verder kunnen ook opties/variabele worden gezet voor hints/auteurs/uitweidingen/ etc
%
% 'Full' versie
\newtoggle{showonline}
\ifdefined\HCode   % zet default showOnline
    \toggletrue{showonline} 
\else
    \togglefalse{showonline}
\fi

% Full versie   % deprecated: see infra
\newcommand{\printFull}{
    \hintstrue
    \handoutfalse
    \toggletrue{showonline} 
}

\ifdefined\shouldPrintFull   % deprecated: see infra
    \printFull
\fi



% Overschrijf onlineOnly  (zoals gedefinieerd in ximera.cls)
\ifhandout   % in handout: gebruik de oorspronkelijke ximera.cls implementatie  (is dit wel nodig/nuttig?)
\else
    \iftoggle{showonline}{%
        \ifdefined\HCode
          \RenewEnviron{onlineOnly}{\bgroup\BODY\egroup}   % showOnline, en we zijn  online, dus toon de tekst
        \else
          \RenewEnviron{onlineOnly}{\bgroup\color{red!50!black}\BODY\egroup}   % showOnline, maar we zijn toch niet online: kleur de tekst rood 
        \fi
    }{%
      \RenewEnviron{onlineOnly}{}  % geen showOnline
    }
\fi

% hack om na hoofding van definition/proposition/... als dan niet op een nieuwe lijn te starten
% soms is dat goed en mooi, en soms niet; en in HTML is het nu (2/2020) anders dan in pdf
% vandaar suggestie om 
%     \begin{definition}[Nieuw concept] \nl
% te gebruiken als je zeker een newline wil na de hoofdig en titel
% (in het bijzonder itemize zonder \nl is 'lelijk' ...)
\ifdefined\HCode
\newcommand{\nl}{}
\else
\newcommand{\nl}{\ \par} % newline (achter heading van definition etc.)
\fi


% \nl enkel in handoutmode (ihb voor \wordChoice, die dan typisch veeeel langer wordt)
\ifdefined\HCode
\providecommand{\handoutnl}{}
\else
\providecommand{\handoutnl}{%
\ifhandout%
  \nl%
\fi%
}
\fi

% Could potentially replace \pdfOnline/\begin{onlineOnly} : 
% Usage= \ifonline{Hallo surfer}{Hallo PDFlezer}
\providecommand{\ifonline}[2]%
{
\begin{onlineOnly}#1\end{onlineOnly}%
\pdfOnly{#2}
}%


%
% Maak optionele 'basic' en 'extended' versies van een activity
%  met environment basicOnly, basicSkip en extendedOnly
%
%  (
%   Dit werkt ENKEL in de PDF; de online versies tonen (minstens voorklopig) steeds 
%   het default geval met printbasicversion en printextendversion beide FALSE
%  )
%
\providebool{printbasicversion}
\providebool{printextendedversion}   % not properly implemented
\providebool{printfullversion}       % presumably print everything (debug/auteur)
%
% only set these in xourses, and BEFORE loading this preamble
%
%\newif\ifshowbasic     \showbasictrue        % use this line in xourse to show 'basic' sections
%\newif\ifshowextended  \showextendedtrue     % use this line in xourse to show 'extended' sections
%
%
%\ifbool{showbasic}
%      { \NewEnviron{basicOnly}{\BODY} }    % if yes: just print contents
%      { \NewEnviron{basicOnly}{}      }    % if no:  completely ignore contents
%
%\ifbool{showbasic}
%      { \NewEnviron{basicSkip}{}      }
%      { \NewEnviron{basicSkip}{\BODY} }
%

\ifbool{printextendedversion}
      { \NewEnviron{extendedOnly}{\BODY} }
      { \NewEnviron{extendedOnly}{}      }
      


\ifdefined\HCode    % in html: always print
      {\newenvironment*{basicOnly}{}{}}    % if yes: just print contents
      {\newenvironment*{basicSkip}{}{}}    % if yes: just print contents
\else

\ifbool{printbasicversion}
      {\newenvironment*{basicOnly}{}{}}    % if yes: just print contents
      {\NewEnviron{basicOnly}{}      }    % if no:  completely ignore contents

\ifbool{printbasicversion}
      {\NewEnviron{basicSkip}{}      }
      {\newenvironment*{basicSkip}{}{}}

\fi

\usepackage{float}
\usepackage[rightbars,color]{changebar}

% Full versie
\ifbool{printfullversion}{
    \hintstrue
    \handoutfalse
    \toggletrue{showonline}
    \printbasicversionfalse
    \cbcolor{red}
    \renewenvironment*{basicOnly}{\cbstart}{\cbend}
    \renewenvironment*{basicSkip}{\cbstart}{\cbend}
    \def\xmtoonprintopties{FULL}   % will be printed in footer
}
{}
      
%
% Evalueer \ifhints IN de environment
%  
%
%\RenewEnviron{hint}
%{
%\ifhandout
%\ifhints\else\setbox0\vbox\fi%everything in een emty box
%\bgroup 
%\stepcounter{hintLevel}
%\BODY
%\egroup\ignorespacesafterend
%\addtocounter{hintLevel}{-1}
%\else
%\ifhints
%\begin{trivlist}\item[\hskip \labelsep\small\slshape\bfseries Hint:\hspace{2ex}]
%\small\slshape
%\stepcounter{hintLevel}
%\BODY
%\end{trivlist}
%\addtocounter{hintLevel}{-1}
%\fi
%\fi
%}

% Onafhankelijk van \ifhandout ...? TO BE VERIFIED
\RenewEnviron{hint}
{
\ifhints
\begin{trivlist}\item[\hskip \labelsep\small\bfseries Hint:\hspace{2ex}]
\small%\slshape
\stepcounter{hintLevel}
\BODY
\end{trivlist}
\addtocounter{hintLevel}{-1}
\else
\iftikzexport   % anders worden de tikz tekeningen in hints niet gegenereerd ?
\setbox0\vbox\bgroup
\stepcounter{hintLevel}
\BODY
\egroup\ignorespacesafterend
\addtocounter{hintLevel}{-1}
\fi % ifhandout
\fi %ifhints
}

%
% \tab sets typewriter-tabs (e.g. to format questions)
% (Has no effect in HTML :-( ))
%
\usepackage{tabto}
\ifdefined\HCode
  \renewcommand{\tab}{\quad}    % otherwise dummy .png's are generated ...?
\fi


% Also redefined in  preamble to get correct styling 
% for tikz images for (\tikzexport)
%

\theoremstyle{definition} % Bold titels
\makeatletter
\let\proposition\relax
\let\c@proposition\relax
\let\endproposition\relax
\makeatother
\newtheorem{proposition}{Eigenschap}


%\instructornotesfalse

% logic with \ifhandoutin ximera.cls unclear;so overwrite ...
\makeatletter
\@ifundefined{ifinstructornotes}{%
  \newif\ifinstructornotes
  \instructornotesfalse
  \newenvironment{instructorNotes}{}{}
}{}
\makeatother
\ifinstructornotes
\else
\renewenvironment{instructorNotes}%
{%
    \setbox0\vbox\bgroup
}
{%
    \egroup
}
\fi

% \RedeclareMathOperator
% from https://tex.stackexchange.com/questions/175251/how-to-redefine-a-command-using-declaremathoperator
\makeatletter
\newcommand\RedeclareMathOperator{%
    \@ifstar{\def\rmo@s{m}\rmo@redeclare}{\def\rmo@s{o}\rmo@redeclare}%
}
% this is taken from \renew@command
\newcommand\rmo@redeclare[2]{%
    \begingroup \escapechar\m@ne\xdef\@gtempa{{\string#1}}\endgroup
    \expandafter\@ifundefined\@gtempa
    {\@latex@error{\noexpand#1undefined}\@ehc}%
    \relax
    \expandafter\rmo@declmathop\rmo@s{#1}{#2}}
% This is just \@declmathop without \@ifdefinable
\newcommand\rmo@declmathop[3]{%
    \DeclareRobustCommand{#2}{\qopname\newmcodes@#1{#3}}%
}
\@onlypreamble\RedeclareMathOperator
\makeatother


%
% Engelse vertaling, vooral in mathmode
%
% 1. Algemene procedure
%
\ifdefined\isEn
 \newcommand{\nlen}[2]{#2}
 \newcommand{\nlentext}[2]{\text{#2}}
 \newcommand{\nlentextbf}[2]{\textbf{#2}}
\else
 \newcommand{\nlen}[2]{#1}
 \newcommand{\nlentext}[2]{\text{#1}}
 \newcommand{\nlentextbf}[2]{\textbf{#1}}
\fi

%
% 2. Lijst van erg veel gebruikte uitdrukkingen
%

% Ja/Nee/Fout/Juits etc
%\newcommand{\TJa}{\nlentext{ Ja }{ and }}
%\newcommand{\TNee}{\nlentext{ Nee }{ No }}
%\newcommand{\TJuist}{\nlentext{ Juist }{ Correct }
%\newcommand{\TFout}{\nlentext{ Fout }{ Wrong }
\newcommand{\TWaar}{\nlentext{ Waar }{ True }}
\newcommand{\TOnwaar}{\nlentext{ Vals }{ False }}
% Korte bindwoorden en, of, dus, ...
\newcommand{\Ten}{\nlentext{ en }{ and }}
\newcommand{\Tof}{\nlentext{ of }{ or }}
\newcommand{\Tdus}{\nlentext{ dus }{ so }}
\newcommand{\Tendus}{\nlentext{ en dus }{ and thus }}
\newcommand{\Tvooralle}{\nlentext{ voor alle }{ for all }}
\newcommand{\Took}{\nlentext{ ook }{ also }}
\newcommand{\Tals}{\nlentext{ als }{ when }} %of if?
\newcommand{\Twant}{\nlentext{ want }{ as }}
\newcommand{\Tmaal}{\nlentext{ maal }{ times }}
\newcommand{\Toptellen}{\nlentext{ optellen }{ add }}
\newcommand{\Tde}{\nlentext{ de }{ the }}
\newcommand{\Thet}{\nlentext{ het }{ the }}
\newcommand{\Tis}{\nlentext{ is }{ is }} %zodat is in text staat in mathmode (geen italics)
\newcommand{\Tmet}{\nlentext{ met }{ where }} % in situaties e.g met p < n --> where p < n
\newcommand{\Tnooit}{\nlentext{ nooit }{ never }}
\newcommand{\Tmaar}{\nlentext{ maar }{ but }}
\newcommand{\Tniet}{\nlentext{ niet }{ not }}
\newcommand{\Tuit}{\nlentext{ uit }{ from }}
\newcommand{\Ttov}{\nlentext{ t.o.v. }{ w.r.t. }}
\newcommand{\Tzodat}{\nlentext{ zodat }{ such that }}
\newcommand{\Tdeth}{\nlentext{de }{th }}
\newcommand{\Tomdat}{\nlentext{omdat }{because }} 


%
% Overschrijf addhoc commando's
%
\ifdefined\isEn
\renewcommand{\pernot}{\overset{\mathrm{notation}}{=}}
\RedeclareMathOperator{\bld}{im}     % beeld
\RedeclareMathOperator{\graf}{graph}   % grafiek
\RedeclareMathOperator{\rico}{slope}   % richtingcoëfficient
\RedeclareMathOperator{\co}{co}       % coordinaat
\RedeclareMathOperator{\gr}{deg}       % graad

% Operators
\RedeclareMathOperator{\bgsin}{arcsin}
\RedeclareMathOperator{\bgcos}{arccos}
\RedeclareMathOperator{\bgtan}{arctan}
\RedeclareMathOperator{\bgcot}{arccot}
\RedeclareMathOperator{\bgsinh}{arcsinh}
\RedeclareMathOperator{\bgcosh}{arccosh}
\RedeclareMathOperator{\bgtanh}{arctanh}
\RedeclareMathOperator{\bgcoth}{arccoth}

\fi


% HACK: use 'oplossing' for 'explanation' ...
\let\explanation\relax
\let\endexplanation\relax
% \newenvironment{explanation}{\begin{oplossing}}{\end{oplossing}}
\newcounter{explanation}

\ifhandout%
    \NewEnviron{explanation}[1][toon]%
    {%
    \RenewEnviron{verbatim}{ \red{VERBATIM CONTENT MISSING IN THIS PDF}} %% \expandafter\verb|\BODY|}

    \ifthenelse{\equal{\detokenize{#1}}{\detokenize{toon}}}
    {
    \def\PH@Command{#1}% Use PH@Command to hold the content and be a target for "\expandafter" to expand once.

    \begin{trivlist}% Begin the trivlist to use formating of the "Feedback" label.
    \item[\hskip \labelsep\small\slshape\bfseries Explanation:% Format the "Feedback" label. Don't forget the space.
    %(\texttt{\detokenize\expandafter{\PH@Command}}):% Format (and detokenize) the condition for feedback to trigger
    \hspace{2ex}]\small%\slshape% Insert some space before the actual feedback given.
    \BODY
    \end{trivlist}
    }
    {  % \begin{feedback}[solution]   \BODY     \end{feedback}  }
    }
    }    
\else
% ONLY for HTML; xmoplossing is styled with css, and is not, and need not be a LaTeX environment
% THUS: it does NOT use feedback anymore ...
%    \NewEnviron{oplossing}{\begin{expandable}{xmoplossing}{\nlen{Toon uitwerking}{Show solution}}{\BODY}\end{expandable}}
    \newenvironment{explanation}[1][toon]
   {%
       \begin{expandable}{xmoplossing}{}
   }
   {%
   	   \end{expandable}
   } 
\fi

\title{Orthogonal Matrices and Symmetric Matrices} \license{CC BY-NC-SA 4.0}

\begin{document}

\begin{abstract}

\end{abstract}
\maketitle

\section*{Orthogonal Matrices and Symmetric Matrices}
Recall that an $n \times n$ matrix $A$ is diagonalizable if and only if it has $n$ linearly independent eigenvectors.  (see \href{\xmbaseurl/EIG-0050/main}{Diagonalizable Matrices and Multiplicity}) Moreover, the matrix $P$ with these eigenvectors as columns is a diagonalizing matrix for $A$, that is
\begin{equation*}
P^{-1}AP \mbox{ is diagonal.}
\end{equation*}
As we have seen, the nice bases of $\RR^n$ are the orthogonal ones, so a natural question is: which $n \times n$ matrices have $n$ orthogonal eigenvectors, so that columns of $P$ form an orthogonal basis for $\RR^n$? These turn out to be precisely the \dfn{symmetric matrices} (matrices for which $A=A^T$), and this is the main result of this section.

\section*{Orthogonal Matrices}
Recall that an orthogonal set of vectors is called \dfn{orthonormal} if $\norm{\vec{q}} = 1$ for each vector $\vec{q}$ in the set, and that any orthogonal set $\{\vec{v}_{1}, \vec{v}_{2}, \dots, \vec{v}_{k}\}$ can be ``\textit{normalized}'', i.e. converted into an orthonormal set $\left\{ \frac{1}{\norm{\vec{v}_{1}}}\vec{v}_{1}, \frac{1}{\norm{\vec{v}_{2}}}\vec{v}_{2}, \dots, \frac{1}{\norm{\vec{v}_{k}}}\vec{v}_{k} \right\}$. In particular, if a matrix $A$ has $n$ orthogonal eigenvectors, they can (by normalizing) be taken to be orthonormal. The corresponding diagonalizing matrix (we will use $Q$ instead of $P$) has orthonormal columns, and such matrices are very easy to invert.


\begin{theorem}\label{th:orthogonal_matrices}
The following conditions are equivalent for an $n \times n$ matrix $Q$.

\begin{enumerate}
\item\label{th:orthogonal_matrices_a} $Q$ is invertible and $Q^{-1} = Q^{T}$.

\item\label{th:orthogonal_matrices_b} The rows of $Q$ are orthonormal.

\item\label{th:orthogonal_matrices_c} The columns of $Q$ are orthonormal.

\end{enumerate}
\end{theorem}

\begin{proof}
First note that condition \ref{th:orthogonal_matrices_a} is equivalent to $Q^{T}Q = I$. Let $\vec{q}_{1}, \vec{q}_{2}, \dots, \vec{q}_{n}$ denote the columns of $Q$. Then $\vec{q}_{i}^{T}$ is the $i$th row of $Q^{T}$, so the $(i, j)$-entry of $Q^{T}Q$ is $\vec{q}_{i} \dotp \vec{q}_{j}$. Thus $Q^{T}Q = I$ means that $\vec{q}_{i} \dotp \vec{q}_{j} = 0$ if $i \neq j$ and $\vec{q}_{i} \dotp \vec{q}_{j} = 1$ if $i = j$. Hence condition \ref{th:orthogonal_matrices_a} is equivalent to \ref{th:orthogonal_matrices_c}. The proof of the equivalence of \ref{th:orthogonal_matrices_a} and \ref{th:orthogonal_matrices_b} is similar.
\end{proof}

\begin{definition}[Orthogonal Matrices]\label{def:orthogonal matrices}
An $n \times n$ matrix $Q$ is called an \dfn{orthogonal matrix} if it satisfies one (and hence all) of the conditions in Theorem~\ref{th:orthogonal_matrices}.
\end{definition}


\begin{example}\label{ex:rotation_ortho}
The rotation matrix
$\begin{bmatrix}
\cos\theta & -\sin\theta \\
\sin\theta & \cos\theta
\end{bmatrix}$ is orthogonal for any angle $\theta$.
\begin{explanation}
See Practice Problem \ref{prob:rotation_ortho}.
\end{explanation}
\end{example}

\begin{remark}\label{rem:orthVsOrthnormMat}
In view of \ref{th:orthogonal_matrices_b} and \ref{th:orthogonal_matrices_c} of Theorem~\ref{th:orthogonal_matrices}, \dfn{orthonormal matrix} might be a better name. But \dfn{orthogonal matrix} is standard.
\end{remark}

It is not enough that the rows of a matrix $A$ are merely orthogonal for $A$ to be an orthogonal matrix. Here is an example.
%\begin{exploration}\label{exp:make_orthogonal}
%\begin{octave} 
%A=[2 1 1; -1 1 1; 0 -1 1]
%\end{octave}
%\begin{enumerate}
 %   \item Check that matrix $A$ in the Octave window has rows that are orthogonal.
  %  \begin{hint}
  %  A(1,:)*transpose(A(2,:)) 
    
  %  A(2,:)*transpose(A(3,:)) 
    
  %  A(1,:)*transpose(A(3,:)) 
    
  %  \end{hint}
 %   \item Check that matrix $A$ in the Octave window has columns that are NOT orthogonal.
 %   \begin{hint}
 %   transpose(A(:,1))*A(:,2) 
    
 %   etc.
    
 %   \end{hint}
 %   \item Check that matrix $A$ in the Octave window has rows that are NOT orthonormal.
 %   \begin{hint}
 %   A(1,:)*transpose(A(1,:)) 
    
 %   A(2,:)*transpose(A(2,:)) 
    
 %   A(3,:)*transpose(A(3,:)) 
    
 %   \end{hint}
%    \item Create a matrix $Q$ by normalizing each of the rows of $A$.
%    \begin{hint}
%    q1=A(1,:)/norm(A(1,:)); 
    
%    q2=A(2,:)/norm(A(2,:));
    
%    q3=A(3,:)/norm(A(3,:));
    
%    Q = [q1;q2;q3]
    
%    \end{hint}
%    \item Check that $Q$ is an orthogonal matrix.
%    \begin{hint}
%    You should get $Q = \begin{bmatrix}
%\frac{2}{\sqrt{6}} & \frac{1}{\sqrt{6}} & %\frac{1}{\sqrt{6}} \\
%\frac{-1}{\sqrt{3}} & \frac{1}{\sqrt{3}} & %\frac{1}{\sqrt{3}} \\
%0 & \frac{-1}{\sqrt{2}} & \frac{1}{\sqrt{2}}
%\end{bmatrix}$, and one can check that this is orthogonal in a number of ways.
 %   \end{hint}
%\end{enumerate}

%\end{exploration}

\begin{exploration}\label{exp:make_orthogonal}
Let $A=\begin{bmatrix}
    2&1&1\\-1&1&1\\0&-1&1
\end{bmatrix}$
\begin{enumerate}
    \item Check that matrix $A$ has rows that are orthogonal.
    \item Check that matrix $A$ has columns that are NOT orthogonal.
    \item Check that matrix $A$ has rows that are NOT orthonormal.
    \item Create a matrix $Q$ by normalizing each of the rows of $A$.
    \item Check that $Q$ is an orthogonal matrix.
\end{enumerate}

Click the arrow to see the answer.
\begin{expandable}
You should get $Q = \begin{bmatrix}
\frac{2}{\sqrt{6}} & \frac{1}{\sqrt{6}} & \frac{1}{\sqrt{6}} \\
\frac{-1}{\sqrt{3}} & \frac{1}{\sqrt{3}} & \frac{1}{\sqrt{3}} \\
0 & \frac{-1}{\sqrt{2}} & \frac{1}{\sqrt{2}}
\end{bmatrix}$, and one can check that this is orthogonal in a number of ways.
\end{expandable}

This exploration can certainly be done by hand (although it takes some time), but it also makes for a very nice Octave exercise.

To use Octave, go to the \href{https://sagecell.sagemath.org/}{Sage Math Cell Webpage}, copy the code below into the cell, select OCTAVE as the language, and press EVALUATE.

\begin{verbatim}
    %Exploration from Section 9.4 Orthogonal Matrices and Symmetric Matrices
    A=[2 1 1; -1 1 1; 0 -1 1]
    %Check that matrix A has rows that are orthogonal.
    A(1,:)*transpose(A(2,:)) 
    A(2,:)*transpose(A(3,:)) 
    A(1,:)*transpose(A(3,:)) 
    %Check that matrix A has columns that are NOT orthogonal.
    transpose(A(:,1))*A(:,2) 
    %(This is 1 of 3 calculations to do.)
    %Check that matrix A in the Octave window has rows that are NOT orthonormal.
    %(See the results from the first question.)
    %Create a matrix Q by normalizing each of the rows of A.
    q1=A(1,:)/norm(A(1,:)); 
    q2=A(2,:)/norm(A(2,:));
    q3=A(3,:)/norm(A(3,:));
    Q = [q1;q2;q3]
    %Check that Q is an orthogonal matrix.
    Q*transpose(Q)
    %(You may get numbers close to zero in some places you expect to get zero due to rounding error)
\end{verbatim}
    

\end{exploration}

We studied the idea of closure when we studied \href{\xmbaseurl/VSP-0020/main}{Subspaces of $\RR^n$}.  The next theorem tells us that orthogonal matrices are closed under matrix multiplication.

\begin{theorem}\label{th:orthogonal_product_inverse}
\begin{enumerate}
    \item\label{th:orthogonal_product}
    If $P$ and $Q$ are orthogonal matrices, then $PQ$ is also orthogonal. (We say that the set of orthogonal matices is closed under matrix multiplication.)
    \item\label{th:orthogonal_inverse}
    If $P$ is an orthogonal matrix, then so is $P^{-1} = P^{T}$.
\end{enumerate}

\begin{proof}
For \ref{th:orthogonal_product}, $P$ and $Q$ are invertible, so $PQ$ is also invertible and
\begin{equation*}
(PQ)^{-1} = Q^{-1}P^{-1} = Q^{T}P^{T} = (PQ)^{T}
\end{equation*}
Hence $PQ$ is orthogonal. For \ref{th:orthogonal_inverse},
\begin{equation*}
(P^{-1})^{-1} = P = (P^{T})^{T} = (P^{-1})^{T}
\end{equation*}
shows that $P^{-1}$ is orthogonal.
\end{proof}
\end{theorem}

\section*{Symmetric Matrices}

We now shift our focus from orthogonal matrices to another important class of $n \times n$ matrices called \dfn{symmetric} matrices.  A \dfn{symmetric matrix} is a matrix which is equal to its transpose.  We saw a few examples of such matrices in \href{\xmbaseurl/MAT-0025/main}{Transpose of a Matrix}.

When we began our study of eigenvalues and eigenvectors, we saw numerous examples of matrices with entries that were real numbers with eigenvalues that were complex numbers.  It can be shown that symmetric matrices only have real eigenvalues.  We also learned that some matrices are diagonalizable while other matrices are not.  It turns out that every symmetric matrix is diagonalizable.  In fact, we can say more, but first we need the following definition.

\begin{definition}\label{def:orthDiag}
An $n \times n$ matrix $A$ is said to be \dfn{orthogonally diagonalizable} if an orthogonal matrix $Q$ can be found such that  $Q^{-1}AQ = Q^{T}AQ$ is diagonal.
\end{definition}

We have learned earlier that when we diagonalize a matrix $A$, we write $P^{-1}AP=D$ for some matrix $P$ where $D$ is diagonal, and the diagonal entries are the eigenvalues of $A$.  We have also learned that the columns of the matrix $P$ are the corresponding eigenvectors of $A$.  So when a matrix is orthogonally diagonalizable, we are able to accomplish the diagonalization using a matrix $Q$ consisting of $n$ eigenvectors that form an orthonormal basis for $\RR^n$.  The following remarkable theorem shows that the matrices that have this property are precisely the symmetric matrices.

\begin{theorem}[Real Spectral Theorem]\label{th:PrinAxes}
Let  $A$ be an $n \times n$ matrix.  Then $A$ is symmetric if and only if $A$ is orthogonally diagonalizable.
\end{theorem}
\begin{proof}
If $A$ is orthogonally diagonalizable, then it is an easy exercise to prove that it is symmetric.  You are asked to do this in Practice Problem \ref{prob:ortho_diag_implies_symmetric}.

To prove the ``only if'' part of this theorem, we assume $A$ is symmetric, and we need to show it is orthogonally diagonalizable.  We proceed by induction on $n$, the size of the symmetric matrix. If $n = 1$, $A$ is already diagonal. If $n > 1$, assume that we know the ``only if'' statement holds for $(n - 1) \times (n - 1)$ symmetric matrices. Let $\lambda_{1}$ be an eigenvalue of $A$, and let $A\vec{x}_{1} = \lambda_{1}\vec{x}_{1}$, where $\norm{\vec{x}_{1}} = 1$. Next, set $\vec{q}_{1}=\vec{x}_{1}$, and use the Gram-Schmidt algorithm to find an orthonormal basis $\{\vec{q}_{1}, \vec{q}_{2}, \dots, \vec{q}_{n}\}$ for $\RR^n$. Let $Q_{1} = \begin{bmatrix}
| & | &   & | \\
\vec{q}_1 & \vec{q}_2  & \cdots & \vec{q}_n \\
| & | &   & |
\end{bmatrix}$, so that $Q_{1}$ is an orthogonal matrix.  We have
\begin{align*}
Q_{1}^TAQ_{1} &= \begin{bmatrix}
-- & \vec{q}_{1}^T & -- \\ -- & \vec{q}_{2}^T & -- \\ & \vdots & \\ -- & \vec{q}_{n}^T & --
\end{bmatrix} A \begin{bmatrix}
| & | &   & | \\
\vec{q}_1 & \vec{q}_2  & \cdots & \vec{q}_n \\
| & | &   & |
\end{bmatrix} \\
&= \begin{bmatrix}
-- & \vec{q}_{1}^T & -- \\ -- & \vec{q}_{2}^T & -- \\ & \vdots & \\ -- & \vec{q}_{n}^T & --
\end{bmatrix} \begin{bmatrix}
| & | &   & | \\
A\vec{q}_1 & A\vec{q}_2  & \cdots & A\vec{q}_n \\
| & | &   & |
\end{bmatrix} \\
&=
\begin{bmatrix}
\lambda_{1} & B \\
\vec{0} & A_{1}
\end{bmatrix},\\
\end{align*} 
where the block $B$ has dimensions $1 \times (n-1)$, and the block under $\lambda_1$ is a $(n-1) \times 1$ zero matrix, because of the orthogonality of the basis vectors.

Next, using the fact that $A$ is symmetric, we notice that
$$(Q_{1}^TAQ_{1})^T = Q_{1}^T A^T (Q_{1}^T)^T = Q_{1}^TAQ_{1},$$ so $Q_{1}^TAQ_{1}$ is symmetric.  It follows that $B$ is also a zero matrix and that $A_{1}$ is symmetric.  Since $A_{1}$ is an $(n - 1) \times (n - 1)$ symmetric matrix, we may apply the inductive hypothesis, so there exists an $(n - 1) \times (n - 1)$ orthogonal matrix $Q$ such that $Q^{T}A_{1}Q = D_{1}$ is diagonal. We observe that $Q_{2} = \begin{bmatrix}
 1 & 0\\
 0 & Q
 \end{bmatrix}$
 is orthogonal, and we compute:
\begin{align*}
(Q_{1}Q_{2})^TA(Q_{1}Q_{2}) &= Q_{2}^T(Q_{1}^TAQ_{1})Q_{2} \\
&= \begin{bmatrix}
1 & 0 \\
0 & Q^T
\end{bmatrix} \begin{bmatrix}
\lambda_{1} & 0 \\
0 & A_{1}
\end{bmatrix}\begin{bmatrix}
1 & 0 \\
0 & Q
\end{bmatrix}\\
&= \begin{bmatrix}
\lambda_{1} & 0 \\
0 & D_{1}
\end{bmatrix}
\end{align*}
is diagonal. Because $Q_{1}Q_{2}$ is orthogonal by Theorem \ref{th:orthogonal_product_inverse}\ref{th:orthogonal_product}, this completes the proof.
\end{proof}


Because the eigenvalues of a real symmetric matrix are real, Theorem~\ref{th:PrinAxes} is also called the \dfn{Real Spectral Theorem}, and the set of distinct eigenvalues is called the \dfn{spectrum} of the matrix. A similar result holds for matrices with complex entries (Theorem \ref{th:025890}).


\begin{example}\label{ex:DiagonalizeSymmetricMatrix}
Find an orthogonal matrix $Q$ such that $Q^{-1}AQ$ is diagonal, where $A = \begin{bmatrix}
1 & 0 & -1 \\
0 & 1 & 2 \\
-1 & 2 & 5
\end{bmatrix}$.

\begin{explanation}
 The characteristic polynomial of $A$ is (adding twice row 1 to row 2):
\begin{equation*}
c_{A}(z) = \det \begin{bmatrix}
z - 1 & 0 & 1 \\
0 & z - 1 & -2 \\
1 & -2 & z - 5
\end{bmatrix} = z(z - 1)(z - 6)
\end{equation*}
Thus the eigenvalues are $\lambda = 0$, $1$, and $6$, and corresponding eigenvectors are
\begin{equation*}
\vec{x}_{1} = \begin{bmatrix}
1 \\
-2 \\
1
\end{bmatrix} \;
\vec{x}_{2} = \begin{bmatrix}
2 \\
1 \\
0
\end{bmatrix} \;
\vec{x}_{3} = \begin{bmatrix}
-1 \\
2 \\
5
\end{bmatrix}
\end{equation*}
respectively. Moreover, by what at first appears to be remarkably good luck, these eigenvectors are \textit{orthogonal}. We have $\norm{\vec{x}_{1}}^{2} = 6$, $\norm{\vec{x}_{2}}^{2} = 5$, and $\norm{\vec{x}_{3}}^{2} = 30$, so
\begin{equation*}
P = \begin{bmatrix}
| & | &  | \\
\frac{1}{\sqrt{6}}\vec{x}_{1} & \frac{1}{\sqrt{5}}\vec{x}_{2} & \frac{1}{\sqrt{30}}\vec{x}_{3} \\
| & | &  | 
\end{bmatrix} = \frac{1}{\sqrt{30}} \begin{bmatrix}
\sqrt{5} & 2\sqrt{6} & -1 \\
-2\sqrt{5} & \sqrt{6} & 2 \\
\sqrt{5} & 0 & 5
\end{bmatrix}
\end{equation*}
is an orthogonal matrix. Thus $P^{-1} = P^{T}$ and
\begin{equation*}
P^TAP = \begin{bmatrix}
0 & 0 & 0 \\
0 & 1 & 0 \\
0 & 0 & 6
\end{bmatrix}.
\end{equation*}

\end{explanation}
\end{example}

Actually, the fact that the eigenvectors in Example~\ref{ex:DiagonalizeSymmetricMatrix} are orthogonal is no coincidence. These vectors certainly must be linearly independent (they correspond to distinct eigenvalues).  We will see that the fact that the matrix is \textit{symmetric} implies that the eigenvectors are orthogonal. To prove this we need the following useful fact about symmetric matrices.

\begin{theorem}\label{th:dotpSymmetric}
If A is an $n \times n$ symmetric matrix, then
\begin{equation*}
(A\vec{x}) \dotp \vec{y} = \vec{x} \dotp (A\vec{y})
\end{equation*}
for all columns $\vec{x}$ and $\vec{y}$ in $\RR^n$.
\end{theorem}
\begin{remark}
The converse also holds (see Practice Problem \ref{ex:8_2_15}).
\end{remark}

\begin{proof}
Recall that $\vec{x} \dotp \vec{y} = \vec{x}^{T} \vec{y}$ for all columns $\vec{x}$ and $\vec{y}$. Because $A^{T} = A$, we get
\begin{equation*}
(A\vec{x}) \dotp \vec{y} = (A\vec{x})^T\vec{y} = \vec{x}^TA^T\vec{y} =  \vec{x}^TA\vec{y} = \vec{x} \dotp (A\vec{y})
\end{equation*}
\end{proof}

\begin{theorem}\label{th:symmetric_has_ortho_ev}
If $A$ is a symmetric matrix, then eigenvectors of $A$ corresponding to distinct eigenvalues are orthogonal.
\end{theorem}

\begin{proof}
Let $A\vec{x} = \lambda \vec{x}$ and $A\vec{y} = \mu \vec{y}$, where $\lambda \neq \mu$. We compute
\begin{equation*}
\lambda(\vec{x} \dotp \vec{y}) = (\lambda\vec{x}) \dotp \vec{y} = (A\vec{x}) \dotp \vec{y} = \vec{x} \dotp (A\vec{y}) = \vec{x} \dotp (\mu\vec{y}) = \mu(\vec{x} \dotp \vec{y})
\end{equation*}
Hence $(\lambda - \mu)(\vec{x} \dotp \vec{y}) = 0$, and so $\vec{x} \dotp \vec{y} = 0$ because $\lambda \neq \mu$.
\end{proof}

Now the procedure for diagonalizing a symmetric $n \times n$ matrix is clear. Find the distinct eigenvalues
 and find orthonormal bases for each eigenspace (the Gram-Schmidt
algorithm may be needed when there is a repeated eigenvalue). Then the set of all these basis vectors is
orthonormal (by Theorem~\ref{th:symmetric_has_ortho_ev}) and contains $n$ vectors. Here is an example.

\begin{example}\label{exa:ortho_diag_symm}
Orthogonally diagonalize the symmetric matrix $A = \begin{bmatrix}
8 & -2 & 2 \\
-2 & 5 & 4 \\
2 & 4 & 5
\end{bmatrix}$.


\begin{explanation}
  The characteristic polynomial is
\begin{equation*}
c_{A}(z) = \det  \begin{bmatrix}
z-8 & 2 & -2 \\
2 & z-5 & -4 \\
-2 & -4 & z-5
\end{bmatrix} = z(z-9)^2
\end{equation*}
Hence the distinct eigenvalues are $0$ and $9$ are of algebraic multiplicity $1$ and $2$, respectively.  The geometric multiplicities must be the same, for $A$ is diagonalizable, being symmetric. It follows that $\mbox{dim}(\mathcal{S}_0) = 1$ and $\mbox{dim}(\mathcal{S}_9) = 2$.  Gaussian elimination gives
\begin{equation*}
\mathcal{S}_{0}(A) = \mbox{span}\{\vec{x}_{1}\}, \quad \vec{x}_{1} = \begin{bmatrix}
1 \\
2 \\
-2
\end{bmatrix}, \quad \mbox{ and } \quad \mathcal{S}_{9}(A) = \mbox{span} \left\lbrace \begin{bmatrix}
	-2 \\
	1 \\
	0
	\end{bmatrix}, \begin{bmatrix}
2 \\
0 \\
1
\end{bmatrix} \right\rbrace
\end{equation*}
The eigenvectors in $\mathcal{S}_{9}$ are both orthogonal to $\vec{x}_{1}$ as Theorem~\ref{th:symmetric_has_ortho_ev} guarantees, but not to each other. However, the Gram-Schmidt process yields an orthogonal basis
\begin{equation*}
\{\vec{f}_{2}, \vec{f}_{3}\} \mbox{ of } \mathcal{S}_{9}(A) \quad \mbox{ where } \quad \vec{f}_{2} = \begin{bmatrix}
-2 \\
1 \\
0
\end{bmatrix} \mbox{ and }  \vec{f}_{3} = \begin{bmatrix}
2 \\
4 \\
5
\end{bmatrix}
\end{equation*}
Normalizing gives orthonormal vectors $\{\frac{1}{3}\vec{x}_{1}, \frac{1}{\sqrt{5}}\vec{f}_{2}, \frac{1}{3\sqrt{5}}\vec{f}_{3}\}$, so
\begin{equation*}
Q = \begin{bmatrix}
| & | &  | \\
\frac{1}{3}\vec{x}_{1} & \frac{1}{\sqrt{5}}\vec{f}_{2} & \frac{1}{3\sqrt{5}}\vec{f}_{3} \\
| & | &  | 
\end{bmatrix} = \frac{1}{3\sqrt{5}}\begin{bmatrix}
\sqrt{5} & -6 & 2 \\
2\sqrt{5} & 3 & 4 \\
-2\sqrt{5} & 0 & 5
\end{bmatrix}
\end{equation*}
is an orthogonal matrix such that $Q^{-1}AQ$ is diagonal.


It is worth noting that other, more convenient, diagonalizing matrices $Q$ exist. For example, $\vec{y}_{2} = \begin{bmatrix}
2 \\
1 \\
2
\end{bmatrix}$ and $\vec{y}_{3} = \begin{bmatrix}
-2 \\
2 \\
1
\end{bmatrix}$
 lie in $\mathcal{S}_{9}(A)$ and they are orthogonal. Moreover, they both have norm $3$ (as does $\vec{x}_{1}$), so
\begin{equation*}
\hat{Q} = \begin{bmatrix}
| & | &  | \\
\frac{1}{3}\vec{x}_{1} & \frac{1}{3}\vec{y}_{2} & \frac{1}{3}\vec{y}_{3} \\
| & | &  |
\end{bmatrix} = \frac{1}{3}\begin{bmatrix}
1 & 2 & -2 \\
2 & 1 & 2 \\
-2 & 2 & 1
\end{bmatrix}
\end{equation*}
is a nicer orthogonal matrix with the property that $\hat{Q}^{-1}A\hat{Q}$ is diagonal.
\end{explanation}
\end{example}

\begin{theorem}\label{th:PrinAxesOtherStuff}
Let $A$ be an $n \times n$ matrix.  $A$ has an orthonormal set of $n$ eigenvectors if and only if $A$ is orthogonally diagonalizable.

\begin{proof}
Let $\vec{q}_{1}, \vec{q}_{2}, \dots, \vec{q}_{n}$ be orthonormal eigenvectors of $A$ with corresponding eigenvalues $\lambda_1, \lambda_2, \ldots, \lambda_n$ .  We must show $A$ is orthogonally diagonalizable. Let $Q = \begin{bmatrix}
| & | &   & | \\
\vec{q}_1 & \vec{q}_2  & \cdots & \vec{q}_n \\
| & | &   & |
\end{bmatrix}$ so that $Q$ is orthogonal.  We have
$$AQ = \begin{bmatrix}
| & | &   & | \\
A\vec{q}_1 & A\vec{q}_2  & \cdots & A\vec{q}_n \\
| & | &   & |
\end{bmatrix} = \begin{bmatrix}
| & | &   & | \\
\lambda_1 \vec{q}_{1} & \lambda_2 \vec{q}_{2} & \dots & \lambda_n \vec{q}_{n} \\
| & | &   & |
\end{bmatrix}=QD, $$
where $D$ is the diagonal matrix with diagonal entries $\lambda_1, \lambda_2, \ldots, \lambda_n$.  But then $Q^TAQ=D$, proving this half of the theorem.

For the converse, if $A$ is orthogonally diagonalizable, then by Theorem \ref{th:PrinAxes} it is symmetric.  But then Theorem~\ref{th:symmetric_has_ortho_ev} tells us that eigenvectors corresponding to distinct eigenvalues are orthogonal.  Because $A$ is (orthogonally) diagonalizable, we know the geometric multiplicity of each eigenvalue is equal to its algebraic multiplicity.  This implies that we can use Gram-Schmidt on each eigenspace of dimension $> 1$ to get a full set of $n$ orthogonal eigenvectors.
\end{proof}
\end{theorem}

If we are willing to replace ``diagonal'' by ``upper triangular'' in the real spectral theorem, we can weaken the requirement that $A$ is symmetric to insisting only that $A$ has real eigenvalues.

\begin{theorem}[Schur Triangularization Theorem]\label{th:Schur}
If $A$ is an $n \times n$ matrix with $n$ real eigenvalues, an orthogonal matrix $Q$ exists such that $Q^{T}AQ$ is upper triangular.
\end{theorem}
\remark{There is also a lower triangular version of this theorem.}

\begin{proof}
See Practice Problem \ref{prob:SchurChallenge}
\end{proof}

The eigenvalues of an upper triangular matrix are displayed along the main diagonal. Because $A$ and $Q^{T}AQ$ have the same determinant and trace whenever $Q$ is orthogonal (for they are similar matrices), Theorem~\ref{th:Schur} gives:

\begin{corollary}\label{cor:det_and_tr}
If $A$ is an $n \times n$ matrix with real eigenvalues $\lambda_{1}, \lambda_{2}, \dots, \lambda_{n}$ (possibly not all distinct), then $\det A = \lambda_{1}\lambda_{2} \dots \lambda_{n}$ and $\mbox{tr} A = \lambda_{1} + \lambda_{2} + \dots  + \lambda_{n}$.
\end{corollary}
\remark{This corollary remains true even if the eigenvalues are not real.} 

\section*{Practice Problems}

\begin{problem}\label{prob:ortho_diag_implies_symmetric}
Suppose $A$ is orthogonally diagonalizable.  Prove that $A$ is symmetric.  (This is the easy direction of the "if and only if" in Theorem \ref{th:PrinAxes}.)
\end{problem}

\begin{problem}%\label{prob:make_ortho_matrix}
Normalize the rows to make each of the following matrices orthogonal.

% 
\begin{problem}\label{prob:make_ortho_matrix1} 
$A = \begin{bmatrix}
1 & 1 \\
-1 & 1
\end{bmatrix}$
\end{problem}

\begin{problem}\label{prob:make_ortho_matrix3}
$A = \begin{bmatrix}
1 & 2 \\
-4 & 2
\end{bmatrix}$
\end{problem}

\begin{problem}\label{prob:make_ortho_matrix5}
$A = \begin{bmatrix}
\cos\theta & -\sin\theta & 0 \\
\sin\theta & \cos\theta & 0 \\
0 & 0 & 2
\end{bmatrix}$
\end{problem}

\begin{problem}\label{prob:make_ortho_matrix7}
$A = \begin{bmatrix}
-1 & 2 & 2 \\
2 & -1 & 2 \\
2 & 2 & -1
\end{bmatrix}$
\end{problem}
\end{problem}

\begin{problem}\label{prob:triag_orthogonal}
If $Q$ is a triangular orthogonal matrix, show that $Q$ is diagonal and that all diagonal entries are $1$ or $-1$.

\begin{hint}
We have $Q^{T} = Q^{-1}$; the first step is to show that $Q$ is lower triangular and also upper triangular, and so is diagonal. But then $Q = Q^{T} = Q^{-1}$, so $Q^{2} = I$. This implies that the diagonal entries of $Q$ are all $\pm 1$.
\end{hint}
\end{problem}

\begin{problem}\label{prob:scalar_othogonal}
If $Q$ is orthogonal, show that $kQ$ is orthogonal if and only if $k = 1$ or $k = -1$.
\end{problem}

\begin{problem}\label{prob:thirdrow}
If the first two rows of an orthogonal matrix are $[\frac{1}{3}, \frac{2}{3}, \frac{2}{3}]$ and $[\frac{2}{3}, \frac{1}{3}, \frac{-2}{3}]$, find all possible third rows.
\end{problem}

\begin{problem}\label{prob:findQ}
For each matrix $A$, find an orthogonal matrix $Q$ such that $Q^{-1}AQ$ is diagonal.

\begin{enumerate}
\item\label{prob:findQa} $A = \begin{bmatrix}
0 & 1 \\
1 & 0
\end{bmatrix}$

\item\label{prob:findQb} $A = \begin{bmatrix}
1 & -1 \\
-1 & 1
\end{bmatrix}$

\item\label{prob:findQc} $A = \begin{bmatrix}
3 & 0 & 0 \\
0 & 2 & 2 \\
0 & 2 & 5
\end{bmatrix}$

\item\label{prob:findQd} $A = \begin{bmatrix}
3 & 0 & 7 \\
0 & 5 & 0 \\
7 & 0 & 3
\end{bmatrix}$

\item\label{prob:findQe} $A = \begin{bmatrix}
1 & 1 & 0 \\
1 & 1 & 0 \\
0 & 0 & 2
\end{bmatrix}$

\item $A = \begin{bmatrix}
5 & -2 & -4 \\
-2 & 8 & -2\\
-4 & -2 & 5
\end{bmatrix}$
\item\label{prob:findQf} (challenging problem) $A = \begin{bmatrix}
5 & 3 & 0 & 0 \\
3 & 5 & 0 & 0 \\
0 & 0 & 7 & 1 \\
0 & 0 & 1 & 7
\end{bmatrix}$

\item\label{prob:findQg} (challenging problem) $A = \begin{bmatrix}
3 & 5 & -1 & 1 \\
5 & 3 & 1 & -1 \\
-1 & 1 & 3 & 5 \\
1 & -1 & 5 & 3
\end{bmatrix}$
\end{enumerate}

%\begin{sol}
%\begin{enumerate} 
%\setcounter{enumi}{1}
%\item  $\frac{1}{\sqrt{2}}\begin{bmatrix}
%1 & -1 \\
%1 & 1
%\end{bmatrix}$

%\setcounter{enumi}{3}
%\item  $\frac{1}{\sqrt{2}}\begin{bmatrix}
%0 & 1 & 1\\
%\sqrt{2} & 0 & 0\\
%0 & 1 & -1
%\end{bmatrix}$

%\setcounter{enumi}{5}
%\item  $\frac{1}{3\sqrt{2}}\begin{bmatrix}
%2\sqrt{2} & 3 & 1\\
%\sqrt{2} & 0 & -4\\
%2\sqrt{2} & -3 & 1
%\end{bmatrix}$ or %$\frac{1}{3}\begin{bmatrix}
%2 & -2 & 1\\
%1 & 2 & 2\\
%2 & 1 & -2
%\end{bmatrix}$

%\setcounter{enumi}{7}
%\item  $\frac{1}{2}\begin{bmatrix}
%1 & -1 & \sqrt{2} & 0\\
%-1 & 1 & \sqrt{2} & 0\\
%-1 & -1 & 0 & \sqrt{2}\\
%1 & 1 & 0 & \sqrt{2}
%\end{bmatrix}$

%\end{enumerate}
%\end{sol}
\end{problem}

\begin{problem}\label{prob:ortho15a}
Show that the following are equivalent for a symmetric matrix $A$.


\begin{enumerate}
\item $A$ is orthogonal.
\item $A^{2} = I$.
\item All eigenvalues of $A$ are $\pm 1$.
\end{enumerate}
\begin{hint}
For (b) if and only if (c), use Theorem~\ref{th:detofproduct}.
\end{hint}


%\begin{sol}
%\begin{enumerate} 
%\setcounter{enumi}{2}
%\item $\Rightarrow$ a. By Theorem~\ref{thm:024227} let $P^{-1}AP = D = \func{diag}(\lambda_{1}, \dots, \lambda_{n})$ where the $\lambda_{i}$ are the eigenvalues of $A$. By c. we have $\lambda_{i} = \pm 1$ for each $i$, whence $D^{2} = I$. But then $A^{2} = (PDP^{-1})^{2} = PD^{2}P^{-1} = I$. Since $A$ is symmetric this is $AA^{T} = I$, proving a.

%\end{enumerate}
%\end{sol}
\end{problem}


\begin{problem}\label{ex:8_2_12}
We call matrices $A$ and $B$ \dfn{orthogonally similar} (and write $A \stackrel{\circ}{\sim} B$) if $B = P^{T}AP$ for an orthogonal matrix $P$.


\begin{enumerate}
\item Show that $A \stackrel{\circ}{\sim} A$ for all $A$; $A \stackrel{\circ}{\sim} B \Rightarrow B \stackrel{\circ}{\sim} A$; and $A \stackrel{\circ}{\sim} B$ and $B \stackrel{\circ}{\sim} C \Rightarrow A \stackrel{\circ}{\sim} C$. (This means that ``orthogonally similar" is an \dfn{equivalence relation}.)

\item Show that the following are equivalent for two symmetric matrices $A$ and $B$.


\begin{enumerate}
\item $A$ and $B$ are similar.

\item $A$ and $B$ are orthogonally similar.

\item $A$ and $B$ have the same eigenvalues.

\end{enumerate}
\end{enumerate}
\end{problem}

\begin{problem}\label{prob:ortho14a}
Assume that $A$ and $B$ are orthogonally similar (Problem \ref{ex:8_2_12}).


\begin{enumerate} 
\item If $A$ and $B$ are invertible, show that $A^{-1}$ and $B^{-1}$ are orthogonally similar.

\item Show that $A^{2}$ and $B^{2}$ are orthogonally similar.

\item Show that, if $A$ is symmetric, so is $B$.

\end{enumerate}

%\begin{sol}
%\begin{enumerate}
%\setcounter{enumi}{1}
%\item  If $B = P^{T}AP = P^{-1}$, then $B^{2} = P^{T}APP^{T}AP = P^{T}A^{2}P$.

%\end{enumerate}
%\end{sol}
\end{problem}

\begin{problem}\label{prob:ortho15}
If $A$ is symmetric, show that every eigenvalue of $A$ is nonnegative if and only if $A = B^{2}$ for some symmetric matrix $B$.
\end{problem}

\begin{problem}\label{ex:8_2_15}
Prove the converse of Theorem~\ref{th:dotpSymmetric}:

If $(A\vec{x}) \dotp \vec{y} = \vec{x} \dotp (A\vec{y})$ for all $n$-columns $\vec{x}$ and $\vec{y}$, then $A$ is symmetric.

%\begin{sol}
%If $\vec{x}$ and $\vec{y}$ are respectively columns $i$ and $j$ of $I_{n}$, then $\vec{x}^{T}A^{T}\vec{y} = \vec{x}^{T}A\vec{y}$ shows that the $(i, j)$-entries of $A^{T}$ and $A$ are equal.
%\end{sol}
\end{problem}

\begin{problem}\label{prob:ortho17}
Show that every eigenvalue of $A$ is zero if and only if $A$ is nilpotent ($A^{k} = 0$ for some $k \geq 1$).
\end{problem}

\begin{problem}\label{prob:ortho18}
If $A$ has real eigenvalues, show that $A = B + C$ where $B$ is symmetric and $C$ is nilpotent. 
%\begin{hint}
%Theorem~\ref{th:024503}
%\end{hint}
\end{problem}

\begin{problem}\label{prob:ortho19}
Let $Q$ be an orthogonal matrix.

\begin{enumerate}
\item Show that $\det Q = 1$ or $\det Q = -1$.

\item Give $2 \times 2$ examples of $Q$ such that $\det Q = 1$ and $\det  Q = -1$.

\item If $\det  Q = -1$, show that $I + Q$ has no inverse.
\begin{hint}
$Q^{T}(I + Q) = (I + Q)^{T}$.
\end{hint}

\item If $P$ is $n \times n$ and $\det P \neq (-1)^{n}$, show that $I - P$ has no inverse.

\begin{hint}
$P^{T}(I - P) = -(I - P)^{T}$
\end{hint}
\end{enumerate}
%\begin{sol}
%\begin{enumerate} 
%\setcounter{enumi}{1}
%\item $\det \begin{bmatrix}
%\cos\theta & -\sin\theta \\
%\sin\theta & \cos\theta
%\end{bmatrix} = 1$ \\ and $\det \begin{bmatrix}
%\cos\theta & \sin\theta \\
%\sin\theta & -\cos\theta
%\end{bmatrix} = -1$
%\begin{remark}
%These are the \textit{only} $2 \times 2$ examples.
%\end{remark}

%\setcounter{enumi}{3}
%\item Use the fact that $P^{-1} = P^{T}$ to show that $P^{T}(I - P) = -(I - P)^{T}$. Now take determinants and use the hypothesis that $\det P \neq (-1)^{n}$.

%\end{enumerate}
%\end{sol}
\end{problem}

\begin{problem}\label{prob:ortho20}
We call a square matrix $E$ a \dfn{projection matrix} if $E^{2} = E = E^{T}$.


\begin{enumerate} 
\item If $E$ is a projection matrix, show that $Q = I - 2E$ is orthogonal and symmetric.

\item If $Q$ is orthogonal and symmetric, show that \\ $E = \frac{1}{2}(I - Q)$ is a projection matrix.

\item If $Q$ is $m \times n$ and $Q^{T}Q = I$ (for example, a unit column in $\RR^n$), show that $E = QQ^{T}$ is a projection matrix.

\end{enumerate}
\end{problem}

\begin{problem}\label{prob:ortho21}
A matrix that we obtain from the identity matrix by writing its rows in a different order is called a \dfn{permutation matrix} (see Theorem \ref{th:LUPA}). Show that every permutation matrix is orthogonal.
\end{problem}

\begin{problem}\label{prob:ortho22}
If the rows $\vec{r}_{1}, \dots, \vec{r}_{n}$ of the $n \times n$ matrix $A = \begin{bmatrix}
a_{ij}
\end{bmatrix}$ are orthogonal, show that the $(i, j)$-entry of $A^{-1}$ is $\frac{a_{ji}}{\norm{\vec{r}_{j}}^2}$.

%\begin{sol}
%We have $AA^{T} = D$, where $D$ is diagonal with main diagonal entries $\norm{R_{1}}^{2}, \dots, \norm{R_{n}}^{2}$. Hence $A^{-1} = A^{T}D^{-1}$, and the result follows because $D^{-1}$ has diagonal entries $1 / \norm{R_{1}}^{2}, \dots, 1 / \norm{R_{n}}^{2}$.
%\end{sol}
\end{problem}

\begin{problem}\label{prob:ortho23}
\begin{enumerate} 
\item Let $A$ be an $m \times n$ matrix. Show that the following are equivalent.


\begin{enumerate}[label={\roman*.}]
\item $A$ has orthogonal rows.

\item $A$ can be factored as $A = DP$, where $D$ is invertible and diagonal and $P$ has orthonormal rows.

\item $AA^{T}$ is an invertible, diagonal matrix.

\end{enumerate}
\item Show that an $n \times n$ matrix $A$ has orthogonal rows if and only if $A$ can be factored as $A = DQ$, where $Q$ is orthogonal and $D$ is diagonal and invertible.

\end{enumerate}
\end{problem}

\begin{problem}\label{prob:ortho24}
Let $A$ be a skew-symmetric matrix; that is, $A^{T} = -A$. Assume that $A$ is an $n \times n$ matrix.


\begin{enumerate} 
\item Show that $I + A$ is invertible. 
\begin{hint}
By Theorem~\ref{thm:004553}, it suffices to show that $(I + A)\vec{x} = \vec{0}$, $\vec{x}$ in $\RR^n$, implies $\vec{x} = \vec{0}$. Compute $\vec{x} \dotp \vec{x} = \vec{x}^{T}\vec{x}$, and use the fact that $A\vec{x} = -\vec{x}$ and $A^{2}\vec{x} = \vec{x}$.
\end{hint}

\item Show that $Q = (I - A)(I + A)^{-1}$ is orthogonal.

\item Show that every orthogonal matrix $P$ such that $I + P$ is invertible arises as in part (b) from some skew-symmetric matrix $A$. 
\begin{hint}
Solve $P = (I - A)(I + A)^{-1}$ for $A$.
\end{hint}

\end{enumerate}
%\begin{sol}
%\begin{enumerate} 
%\setcounter{enumi}{1}
%\item  Because $I - A$ and $I + A$ commute, $PP^{T} = (I - A)(I + A)^{-1}[(I + A)^{-1}]^{T}(I - A)^{T} = (I - A)(I + A)^{-1}(I - A)^{-1}(I + A) = I$.

%\end{enumerate}
%\end{sol}
\end{problem}

\begin{problem}\label{prob:ortho25}
Show that the following are equivalent for an $n \times n$ matrix $Q$.


\begin{enumerate} 
\item $Q$ is orthogonal.

\item $\norm{Q\vec{x}} = \norm{\vec{x}}$ for all $\vec{x}\in\RR^n$.

\item $\norm{ Q\vec{x} - Q\vec{y}} = \norm{\vec{x} - \vec{y}}$ for all $\vec{x}$, $\vec{y}\in \RR^n$.

\item $(Q\vec{x}) \dotp (Q\vec{y}) = \vec{x} \dotp \vec{y}$ for all columns $\vec{x}$, $\vec{y}\in\RR^n$.

\begin{hint}
%For (c) $\Rightarrow$ (d), see Exercise \ref{ex:5_3_14}(a). 
For (d) $\Rightarrow$ (a), show that column $i$ of $Q$ equals $Q\vec{e}_{i}$, where $\vec{e}_{i}$ is column $i$ of the identity matrix.
\end{hint}
\end{enumerate}
\begin{remark}
    This exercise shows that linear transformations with orthogonal standard matrices are distance-preserving (b,c) and angle-preserving (d).
\end{remark}

\end{problem}



\begin{problem}\label{prob:rotation_ortho}
\begin{enumerate}
    \item Show that $\begin{bmatrix}
\cos\theta & -\sin\theta \\
\sin\theta & \cos\theta
\end{bmatrix}$ is an orthogonal matrix.

    \item Show that every $2 \times 2$ orthogonal matrix has the form $\begin{bmatrix}
\cos\theta & -\sin\theta \\
\sin\theta & \cos\theta
\end{bmatrix}$ or $\begin{bmatrix}
\cos\theta & \sin\theta \\
\sin\theta & -\cos\theta
\end{bmatrix}$
 for some angle $\theta$.
 \begin{hint}
 If $a^{2} + b^{2} = 1$, then $a = \cos\theta$ and $b = \sin\theta$ for some angle $\theta$.
 \end{hint}
\end{enumerate}

\end{problem}

\begin{problem}\label{prob:SchurChallenge}
Modify the proof of Theorem~\ref{th:PrinAxes} to prove Theorem \ref{th:Schur}.
%If $A\vec{x}_{1} = \lambda_{1}\vec{x}_{1}$ where $\norm{\vec{x}_{1}} = 1$, let $\{\vec{x}_{1}, \vec{x}_{2}, \dots, \vec{x}_{n}\}$ be an orthonormal basis of $\RR^n$, and let $P_{1} = \begin{bmatrix}
%\vec{x}_{1} & \vec{x}_{2} & \cdots &  \vec{x}_{n}
%\end{bmatrix}$. Then $P_{1}$ is orthogonal and $P_{1}^TAP_{1} = \begin{bmatrix}
%\lambda_{1} & B \\
%0 & A_{1}
%\end{bmatrix}$ in block form. By induction, let $Q^{T}A_{1}Q = T_{1}$ be upper triangular where $Q$ is of size $(n-1)\times(n-1)$ and orthogonal. Then $P_{2} = \begin{bmatrix}
%1 & 0 \\
%0 & Q
%\end{bmatrix}$ is orthogonal, so $P = P_{1}P_{2}$ is also orthogonal and $P^TAP = \begin{bmatrix}
%\lambda_{1} & BQ \\
%0 & T_{1}
%\end{bmatrix}$
% is upper triangular.
\end{problem}

\section*{Text Source} This section was adapted from Section 8.2 of Keith Nicholson's \href{https://open.umn.edu/opentextbooks/textbooks/linear-algebra-with-applications}{\it Linear Algebra with Applications}. (CC-BY-NC-SA)

W. Keith Nicholson, {\it Linear Algebra with Applications}, Lyryx 2018, Open Edition, p. 424


\end{document}
