\documentclass{ximera}
%%% Begin Laad packages

\makeatletter
\@ifclassloaded{xourse}{%
    \typeout{Start loading preamble.tex (in a XOURSE)}%
    \def\isXourse{true}   % automatically defined; pre 112022 it had to be set 'manually' in a xourse
}{%
    \typeout{Start loading preamble.tex (NOT in a XOURSE)}%
}
\makeatother

\def\isEn\true 

\pgfplotsset{compat=1.16}

\usepackage{currfile}

% 201908/202301: PAS OP: babel en doclicense lijken problemen te veroorzaken in .jax bestand
% (wegens syntax error met toegevoegde \newcommands ...)
\pdfOnly{
    \usepackage[type={CC},modifier={by-nc-sa},version={4.0}]{doclicense}
    %\usepackage[hyperxmp=false,type={CC},modifier={by-nc-sa},version={4.0}]{doclicense}
    %%% \usepackage[dutch]{babel}
}



\usepackage[utf8]{inputenc}
\usepackage{morewrites}   % nav zomercursus (answer...?)
\usepackage{multirow}
\usepackage{multicol}
\usepackage{tikzsymbols}
\usepackage{ifthen}
%\usepackage{animate} BREAKS HTML STRUCTURE USED BY XIMERA
\usepackage{relsize}

\usepackage{eurosym}    % \euro  (€ werkt niet in xake ...?)
\usepackage{fontawesome} % smileys etc

% Nuttig als ook interactieve beamer slides worden voorzien:
\providecommand{\p}{} % default nothing ; potentially usefull for slides: redefine as \pause
%providecommand{\p}{\pause}

    % Layout-parameters voor het onderschrift bij figuren
\usepackage[margin=10pt,font=small,labelfont=bf, labelsep=endash,format=hang]{caption}
%\usepackage{caption} % captionof
%\usepackage{pdflscape}    % landscape environment

% Met "\newcommand\showtodonotes{}" kan je todonotes tonen (in pdf/online)
% 201908: online werkt het niet (goed)
\providecommand\showtodonotes{disable}
\providecommand\todo[1]{\typeout{TODO #1}}
%\usepackage[\showtodonotes]{todonotes}
%\usepackage{todonotes}

%
% Poging tot aanpassen layout
%
\usepackage{tcolorbox}
\tcbuselibrary{theorems}

%%% Einde laad packages

%%% Begin Ximera specifieke zaken

\graphicspath{
	{../../}
	{../}
	{./}
  	{../../pictures/}
   	{../pictures/}
   	{./pictures/}
	{./explog/}    % M05 in groeimodellen       
}

%%% Einde Ximera specifieke zaken

%
% define softer blue/red/green, use KU Leuven base colors for blue (and dark orange for red ?)
%
% todo: rather redefine blue/red/green ...?
%\definecolor{xmblue}{rgb}{0.01, 0.31, 0.59}
%\definecolor{xmred}{rgb}{0.89, 0.02, 0.17}
\definecolor{xmdarkblue}{rgb}{0.122, 0.671, 0.835}   % KU Leuven Blauw
\definecolor{xmblue}{rgb}{0.114, 0.553, 0.69}        % KU Leuven Blauw
\definecolor{xmgreen}{rgb}{0.13, 0.55, 0.13}         % No KULeuven variant for green found ...

\definecolor{xmaccent}{rgb}{0.867, 0.541, 0.18}      % KU Leuven Accent (orange ...)
\definecolor{kuaccent}{rgb}{0.867, 0.541, 0.18}      % KU Leuven Accent (orange ...)

\colorlet{xmred}{xmaccent!50!black}                  % Darker version of KU Leuven Accent

\providecommand{\blue}[1]{{\color{blue}#1}}    
\providecommand{\red}[1]{{\color{red}#1}}

\renewcommand\CancelColor{\color{xmaccent!50!black}}

% werkt in math en text mode om MATH met oranje (of grijze...)  achtergond te tonen (ook \important{\text{blabla}} lijkt te werken)
%\newcommand{\important}[1]{\ensuremath{\colorbox{xmaccent!50!white}{$#1$}}}   % werkt niet in Mathjax
%\newcommand{\important}[1]{\ensuremath{\colorbox{lightgray}{$#1$}}}
\newcommand{\important}[1]{\ensuremath{\colorbox{orange}{$#1$}}}   % TODO: kleur aanpassen voor mathjax; wordt overschreven infra!


% Uitzonderlijk kan met \pdfnl in de PDF een newline worden geforceerd, die online niet nodig/nuttig is omdat daar de regellengte hoe dan ook niet gekend is.
\ifdefined\HCode%
\providecommand{\pdfnl}{}%
\else%
\providecommand{\pdfnl}{%
  \\%
}%
\fi

% Uitzonderlijk kan met \handoutnl in de handout-PDF een newline worden geforceerd, die noch online noch in de PDF-met-antwoorden nuttig is.
\ifdefined\HCode
\providecommand{\handoutnl}{}
\else
\providecommand{\handoutnl}{%
\ifhandout%
  \nl%
\fi%
}
\fi



% \cellcolor IGNORED by tex4ht ?
% \begin{center} seems not to wordk
    % (missing margin-left: auto;   on tabular-inside-center ???)
%\newcommand{\importantcell}[1]{\ensuremath{\cellcolor{lightgray}#1}}  %  in tabular; usablility to be checked
\providecommand{\importantcell}[1]{\ensuremath{#1}}     % no mathjax2 support for colloring array cells

\pdfOnly{
  \renewcommand{\important}[1]{\ensuremath{\colorbox{kuaccent!50!white}{$#1$}}}
  \renewcommand{\importantcell}[1]{\ensuremath{\cellcolor{kuaccent!40!white}#1}}   
}

%%% Tikz styles


\pgfplotsset{compat=1.16}

\usetikzlibrary{trees,positioning,arrows,fit,shapes,math,calc,decorations.markings,through,intersections,patterns,matrix}

\usetikzlibrary{decorations.pathreplacing,backgrounds}    % 5/2023: from experimental


\usetikzlibrary{angles,quotes}

\usepgfplotslibrary{fillbetween} % bepaalde_integraal
\usepgfplotslibrary{polar}    % oa voor poolcoordinaten.tex

\pgfplotsset{ownstyle/.style={axis lines = center, axis equal image, xlabel = $x$, ylabel = $y$, enlargelimits}} 

\pgfplotsset{
	plot/.style={no marks,samples=50}
}

\newcommand{\xmPlotsColor}{
	\pgfplotsset{
		plot1/.style={darkgray,no marks,samples=100},
		plot2/.style={lightgray,no marks,samples=100},
		plotresult/.style={blue,no marks,samples=100},
		plotblue/.style={blue,no marks,samples=100},
		plotred/.style={red,no marks,samples=100},
		plotgreen/.style={green,no marks,samples=100},
		plotpurple/.style={purple,no marks,samples=100}
	}
}
\newcommand{\xmPlotsBlackWhite}{
	\pgfplotsset{
		plot1/.style={black,loosely dashed,no marks,samples=100},
		plot2/.style={black,loosely dotted,no marks,samples=100},
		plotresult/.style={black,no marks,samples=100},
		plotblue/.style={black,no marks,samples=100},
		plotred/.style={black,dotted,no marks,samples=100},
		plotgreen/.style={black,dashed,no marks,samples=100},
		plotpurple/.style={black,dashdotted,no marks,samples=100}
	}
}


\newcommand{\xmPlotsColorAndStyle}{
	\pgfplotsset{
		plot1/.style={darkgray,no marks,samples=100},
		plot2/.style={lightgray,no marks,samples=100},
		plotresult/.style={blue,no marks,samples=100},
		plotblue/.style={xmblue,no marks,samples=100},
		plotred/.style={xmred,dashed,thick,no marks,samples=100},
		plotgreen/.style={xmgreen,dotted,very thick,no marks,samples=100},
		plotpurple/.style={purple,no marks,samples=100}
	}
}


%\iftikzexport
\xmPlotsColorAndStyle
%\else
%\xmPlotsBlackWhite
%\fi
%%%


%
% Om venndiagrammen te arceren ...
%
\makeatletter
\pgfdeclarepatternformonly[\hatchdistance,\hatchthickness]{north east hatch}% name
{\pgfqpoint{-1pt}{-1pt}}% below left
{\pgfqpoint{\hatchdistance}{\hatchdistance}}% above right
{\pgfpoint{\hatchdistance-1pt}{\hatchdistance-1pt}}%
{
	\pgfsetcolor{\tikz@pattern@color}
	\pgfsetlinewidth{\hatchthickness}
	\pgfpathmoveto{\pgfqpoint{0pt}{0pt}}
	\pgfpathlineto{\pgfqpoint{\hatchdistance}{\hatchdistance}}
	\pgfusepath{stroke}
}
\pgfdeclarepatternformonly[\hatchdistance,\hatchthickness]{north west hatch}% name
{\pgfqpoint{-\hatchthickness}{-\hatchthickness}}% below left
{\pgfqpoint{\hatchdistance+\hatchthickness}{\hatchdistance+\hatchthickness}}% above right
{\pgfpoint{\hatchdistance}{\hatchdistance}}%
{
	\pgfsetcolor{\tikz@pattern@color}
	\pgfsetlinewidth{\hatchthickness}
	\pgfpathmoveto{\pgfqpoint{\hatchdistance+\hatchthickness}{-\hatchthickness}}
	\pgfpathlineto{\pgfqpoint{-\hatchthickness}{\hatchdistance+\hatchthickness}}
	\pgfusepath{stroke}
}
%\makeatother

\tikzset{
    hatch distance/.store in=\hatchdistance,
    hatch distance=10pt,
    hatch thickness/.store in=\hatchthickness,
   	hatch thickness=2pt
}

\colorlet{circle edge}{black}
\colorlet{circle area}{blue!20}


\tikzset{
    filled/.style={fill=green!30, draw=circle edge, thick},
    arceerl/.style={pattern=north east hatch, pattern color=blue!50, draw=circle edge},
    arceerr/.style={pattern=north west hatch, pattern color=yellow!50, draw=circle edge},
    outline/.style={draw=circle edge, thick}
}




%%% Updaten commando's
\def\hoofding #1#2#3{\maketitle}     % OBSOLETE ??

% we willen (bijna) altijd \geqslant ipv \geq ...!
\newcommand{\geqnoslant}{\geq}
\renewcommand{\geq}{\geqslant}
\newcommand{\leqnoslant}{\leq}
\renewcommand{\leq}{\leqslant}

% Todo: (201908) waarom komt er (soms) underlined voor emph ...?
\renewcommand{\emph}[1]{\textit{#1}}

% API commando's

\newcommand{\ds}{\displaystyle}
\newcommand{\ts}{\textstyle}  % tegenhanger van \ds   (Ximera zet PER  DEFAULT \ds!)

% uit Zomercursus-macro's: 
\newcommand{\bron}[1]{\begin{scriptsize} \emph{#1} \end{scriptsize}}     % deprecated ...?


%definities nieuwe commando's - afkortingen veel gebruikte symbolen
\newcommand{\R}{\ensuremath{\mathbb{R}}}
\newcommand{\Rnul}{\ensuremath{\mathbb{R}_0}}
\newcommand{\Reen}{\ensuremath{\mathbb{R}\setminus\{1\}}}
\newcommand{\Rnuleen}{\ensuremath{\mathbb{R}\setminus\{0,1\}}}
\newcommand{\Rplus}{\ensuremath{\mathbb{R}^+}}
\newcommand{\Rmin}{\ensuremath{\mathbb{R}^-}}
\newcommand{\Rnulplus}{\ensuremath{\mathbb{R}_0^+}}
\newcommand{\Rnulmin}{\ensuremath{\mathbb{R}_0^-}}
\newcommand{\Rnuleenplus}{\ensuremath{\mathbb{R}^+\setminus\{0,1\}}}
\newcommand{\N}{\ensuremath{\mathbb{N}}}
\newcommand{\Nnul}{\ensuremath{\mathbb{N}_0}}
\newcommand{\Z}{\ensuremath{\mathbb{Z}}}
\newcommand{\Znul}{\ensuremath{\mathbb{Z}_0}}
\newcommand{\Zplus}{\ensuremath{\mathbb{Z}^+}}
\newcommand{\Zmin}{\ensuremath{\mathbb{Z}^-}}
\newcommand{\Znulplus}{\ensuremath{\mathbb{Z}_0^+}}
\newcommand{\Znulmin}{\ensuremath{\mathbb{Z}_0^-}}
\newcommand{\C}{\ensuremath{\mathbb{C}}}
\newcommand{\Cnul}{\ensuremath{\mathbb{C}_0}}
\newcommand{\Cplus}{\ensuremath{\mathbb{C}^+}}
\newcommand{\Cmin}{\ensuremath{\mathbb{C}^-}}
\newcommand{\Cnulplus}{\ensuremath{\mathbb{C}_0^+}}
\newcommand{\Cnulmin}{\ensuremath{\mathbb{C}_0^-}}
\newcommand{\Q}{\ensuremath{\mathbb{Q}}}
\newcommand{\Qnul}{\ensuremath{\mathbb{Q}_0}}
\newcommand{\Qplus}{\ensuremath{\mathbb{Q}^+}}
\newcommand{\Qmin}{\ensuremath{\mathbb{Q}^-}}
\newcommand{\Qnulplus}{\ensuremath{\mathbb{Q}_0^+}}
\newcommand{\Qnulmin}{\ensuremath{\mathbb{Q}_0^-}}

\newcommand{\perdef}{\overset{\mathrm{def}}{=}}
\newcommand{\pernot}{\overset{\mathrm{notatie}}{=}}
\newcommand\perinderdaad{\overset{!}{=}}     % voorlopig gebruikt in limietenrekenregels
\newcommand\perhaps{\overset{?}{=}}          % voorlopig gebruikt in limietenrekenregels

\newcommand{\degree}{^\circ}


\DeclareMathOperator{\dom}{dom}     % domein
\DeclareMathOperator{\codom}{codom} % codomein
\DeclareMathOperator{\bld}{bld}     % beeld
\DeclareMathOperator{\graf}{graf}   % grafiek
\DeclareMathOperator{\rico}{rico}   % richtingcoëfficient
\DeclareMathOperator{\co}{co}       % coordinaat
\DeclareMathOperator{\gr}{gr}       % graad

\newcommand{\func}[5]{\ensuremath{#1: #2 \rightarrow #3: #4 \mapsto #5}} % Easy to write a function


% Operators
\DeclareMathOperator{\bgsin}{bgsin}
\DeclareMathOperator{\bgcos}{bgcos}
\DeclareMathOperator{\bgtan}{bgtan}
\DeclareMathOperator{\bgcot}{bgcot}
\DeclareMathOperator{\bgsinh}{bgsinh}
\DeclareMathOperator{\bgcosh}{bgcosh}
\DeclareMathOperator{\bgtanh}{bgtanh}
\DeclareMathOperator{\bgcoth}{bgcoth}

% Oude \Bgsin etc deprecated: gebruik \bgsin, en herdefinieer dat als je Bgsin wil!
%\DeclareMathOperator{\cosec}{cosec}    % not used? gebruik \csc en herdefinieer

% operatoren voor differentialen: to be verified; 1/2020: inconsequent gebruik bij afgeleiden/integralen
\renewcommand{\d}{\mathrm{d}}
\newcommand{\dx}{\d x}
\newcommand{\dd}[1]{\frac{\mathrm{d}}{\mathrm{d}#1}}
\newcommand{\ddx}{\dd{x}}

% om in voorbeelden/oefeningen de notatie voor afgeleiden te kunnen kiezen
% Usage: \afg{(2\sin(x))}  (en wordt d/dx, of accent, of D )
%\newcommand{\afg}[1]{{#1}'}
\newcommand{\afg}[1]{\left(#1\right)'}
%\renewcommand{\afg}[1]{\frac{\mathrm{d}#1}{\mathrm{d}x}}   % include in relevant exercises ...
%\renewcommand{\afg}[1]{D{#1}}

%
% \xmxxx commands: Extra KU Leuven functionaliteit van, boven of naast Ximera
%   ( Conventie 8/2019: xm+nederlandse omschrijving, maar is niet consequent gevolgd, en misschien ook niet erg handig !)
%
% (Met een minimale ximera.cls en preamble.tex zou een bruikbare .pdf moeten kunnen worden gemaakt van eender welke ximera)
%
% Usage: \xmtitle[Mijn korte abstract]{Mijn titel}{Mijn abstract}
% Eerste command na \begin{document}:
%  -> definieert de \title
%  -> definieert de abstract
%  -> doet \maketitle ( dus: print de hoofding als 'chapter' of 'sectie')
% Optionele parameter geeft eenn kort abstract (die met de globale setting \xmshortabstract{} al dan niet kan worden geprint.
% De optionele korte abstract kan worden gebruikt voor pseudo-grappige abtsarts, dus dus globaal al dan niet kunnen worden gebuikt...
% Globale settings:
%  de (optionele) 'korte abstract' wordt enkele getoond als \xmshortabstract is gezet
\providecommand\xmshortabstract{} % default: print (only!) short abstract if present
\newcommand{\xmtitle}[3][]{
	\title{#2}
	\begin{abstract}
		\ifdefined\xmshortabstract
		\ifstrempty{#1}{%
			#3
		}{%
			#1
		}%
		\else
		#3
		\fi
	\end{abstract}
	\maketitle
}

% 
% Kleine grapjes: moeten zonder verder gevolg kunnen worden verwijderd
%
%\newcommand{\xmopje}[1]{{\small#1{\reversemarginpar\marginpar{\Smiley}}}}   % probleem in floats!!
\newtoggle{showxmopje}
\toggletrue{showxmopje}

\newcommand{\xmopje}[1]{%
   \iftoggle{showxmopje}{#1}{}%
}


% -> geef een abstracte-formule-met-rechts-een-concreet-voorbeeld
% VB:  \formulevb{a^2+b^2=c^2}{3^2+4^2=5^2}
%
\ifdefined\HCode
\NewEnviron{xmdiv}[1]{\HCode{\Hnewline<div class="#1">\Hnewline}\BODY{\HCode{\Hnewline</div>\Hnewline}}}
\else
\NewEnviron{xmdiv}[1]{\BODY}
\fi

\providecommand{\formulevb}[2]{
	{\centering

    \begin{xmdiv}{xmformulevb}    % zie css voor online layout !!!
	\begin{tabular}{lcl}
		\important{#1}
		&  &
		Vb: $#2$
		\end{tabular}
	\end{xmdiv}

	}
}

\ifdefined\HCode
\providecommand{\vb}[1]{%
    \HCode{\Hnewline<span class="xmvb">}#1\HCode{</span>\Hnewline}%
}
\else
\providecommand{\vb}[1]{
    \colorbox{blue!10}{#1}
}
\fi

\ifdefined\HCode
\providecommand{\xmcolorbox}[2]{
	\HCode{\Hnewline<div class="xmcolorbox">\Hnewline}#2\HCode{\Hnewline</div>\Hnewline}
}
\else
\providecommand{\xmcolorbox}[2]{
  \cellcolor{#1}#2
}
\fi


\ifdefined\HCode
\providecommand{\xmopmerking}[1]{
 \HCode{\Hnewline<div class="xmopmerking">\Hnewline}#1\HCode{\Hnewline</div>\Hnewline}
}
\else
\providecommand{\xmopmerking}[1]{
	{\footnotesize #1}
}
\fi
% \providecommand{\voorbeeld}[1]{
% 	\colorbox{blue!10}{$#1$}
% }



% Hernoem Proof naar Bewijs, nodig voor HTML versie
\renewcommand*{\proofname}{Bewijs}

% Om opgave van oefening (wordt niet geprint bij oplossingenblad)
% (to be tested test)
\NewEnviron{statement}{\BODY}

% Environment 'oplossing' en 'uitkomst'
% voor resp. volledige 'uitwerking' dan wel 'enkel eindresultaat'
% geimplementeerd via feedback, omdat er in de ximera-server adhoc feedback-code is toegevoegd
%% Niet tonen indien handout
%% Te gebruiken om volledige oplossingen/uitwerkingen van oefeningen te tonen
%% \begin{oplossing}        De optelling is commutatief \end{oplossing}  : verschijnt online enkel 'op vraag'
%% \begin{oplossing}[toon]  De optelling is commutatief \end{oplossing}  : verschijnt steeds onmiddellijk online (bv te gebruiken bij voorbeelden) 

\ifhandout%
    \NewEnviron{oplossing}[1][onzichtbaar]%
    {%
    \ifthenelse{\equal{\detokenize{#1}}{\detokenize{toon}}}
    {
    \def\PH@Command{#1}% Use PH@Command to hold the content and be a target for "\expandafter" to expand once.

    \begin{trivlist}% Begin the trivlist to use formating of the "Feedback" label.
    \item[\hskip \labelsep\small\slshape\bfseries Oplossing% Format the "Feedback" label. Don't forget the space.
    %(\texttt{\detokenize\expandafter{\PH@Command}}):% Format (and detokenize) the condition for feedback to trigger
    \hspace{2ex}]\small%\slshape% Insert some space before the actual feedback given.
    \BODY
    \end{trivlist}
    }
    {  % \begin{feedback}[solution]   \BODY     \end{feedback}  }
    }
    }    
\else
% ONLY for HTML; xmoplossing is styled with css, and is not, and need not be a LaTeX environment
% THUS: it does NOT use feedback anymore ...
%    \NewEnviron{oplossing}{\begin{expandable}{xmoplossing}{\nlen{Toon uitwerking}{Show solution}}{\BODY}\end{expandable}}
    \newenvironment{oplossing}[1][onzichtbaar]
   {%
       \begin{expandable}{xmoplossing}{}
   }
   {%
   	   \end{expandable}
   } 
%     \newenvironment{oplossing}[1][onzichtbaar]
%    {%
%        \begin{feedback}[solution]   	
%    }
%    {%
%    	   \end{feedback}
%    } 
\fi

\ifhandout%
    \NewEnviron{uitkomst}[1][onzichtbaar]%
    {%
    \ifthenelse{\equal{\detokenize{#1}}{\detokenize{toon}}}
    {
    \def\PH@Command{#1}% Use PH@Command to hold the content and be a target for "\expandafter" to expand once.

    \begin{trivlist}% Begin the trivlist to use formating of the "Feedback" label.
    \item[\hskip \labelsep\small\slshape\bfseries Uitkomst:% Format the "Feedback" label. Don't forget the space.
    %(\texttt{\detokenize\expandafter{\PH@Command}}):% Format (and detokenize) the condition for feedback to trigger
    \hspace{2ex}]\small%\slshape% Insert some space before the actual feedback given.
    \BODY
    \end{trivlist}
    }
    {  % \begin{feedback}[solution]   \BODY     \end{feedback}  }
    }
    }    
\else
\ifdefined\HCode
   \newenvironment{uitkomst}[1][onzichtbaar]
    {%
        \begin{expandable}{xmuitkomst}{}%
    }
    {%
    	\end{expandable}%
    } 
\else
  % Do NOT print 'uitkomst' in non-handout
  %  (presumably, there is also an 'oplossing' ??)
  \newenvironment{uitkomst}[1][onzichtbaar]{}{}
\fi
\fi

%
% Uitweidingen zijn extra's die niet redelijkerwijze tot de leerstof behoren
% Uitbreidingen zijn extra's die wel redelijkerwijze tot de leerstof van bv meer geavanceerde versies kunnen behoren (B-programma/Wiskundestudenten/...?)
% Nog niet voorzien: design voor verschillende versies (A/B programma, BIO, voorkennis/ ...)
% Voor 'uitweidingen' is er een environment die online per default is ingeklapt, en in pdf al dan niet kan worden geincluded  (via \xmnouitweiding) 
%
% in een xourse, per default GEEN uitweidingen, tenzij \xmuitweiding expliciet ergens is gezet ...
\ifdefined\isXourse
   \ifdefined\xmuitweiding
   \else
       \def\xmnouitweiding{true}
   \fi
\fi

\ifdefined\xmnouitweiding
\newcounter{xmuitweiding}  % anders error undefined ...  
\excludecomment{xmuitweiding}
\else
\newtheoremstyle{dotless}{}{}{}{}{}{}{ }{}
\theoremstyle{dotless}
\newtheorem*{xmuitweidingnofrills}{}   % nofrills = no accordion; gebruikt dus de dotless theoremstyle!

\newcounter{xmuitweiding}
\newenvironment{xmuitweiding}[1][ ]%
{% 
	\refstepcounter{xmuitweiding}%
    \begin{expandable}{xmuitweiding}{\nlentext{Uitweiding \arabic{xmuitweiding}: #1}{Digression \arabic{xmuitweiding}: #1}}%
	\begin{xmuitweidingnofrills}%
}
{%
    \end{xmuitweidingnofrills}%
    \end{expandable}%
}   
% \newenvironment{xmuitweiding}[1][ ]%
% {% 
% 	\refstepcounter{xmuitweiding}
% 	\begin{accordion}\begin{accordion-item}[Uitweiding \arabic{xmuitweiding}: #1]%
% 			\begin{xmuitweidingnofrills}%
% 			}
% 			{\end{xmuitweidingnofrills}\end{accordion-item}\end{accordion}}   
\fi


\newenvironment{xmexpandable}[1][]{
	\begin{accordion}\begin{accordion-item}[#1]%
		}{\end{accordion-item}\end{accordion}}


% Command that gives a selection box online, but just prints the right answer in pdf
\newcommand{\xmonlineChoice}[1]{\pdfOnly{\wordchoicegiventrue}\wordChoice{#1}\pdfOnly{\wordchoicegivenfalse}}   % deprecated, gebruik onlineChoice ...
\newcommand{\onlineChoice}[1]{\pdfOnly{\wordchoicegiventrue}\wordChoice{#1}\pdfOnly{\wordchoicegivenfalse}}

\newcommand{\TJa}{\nlentext{ Ja }{ Yes }}
\newcommand{\TNee}{\nlentext{ Nee }{ No }}
\newcommand{\TJuist}{\nlentext{ Juist }{ True }}
\newcommand{\TFout}{\nlentext{ Fout }{ False }}

\newcommand{\choiceTrue }{{\renewcommand{\choiceminimumhorizontalsize}{4em}\wordChoice{\choice[correct]{\TJuist}\choice{\TFout}}}}
\newcommand{\choiceFalse}{{\renewcommand{\choiceminimumhorizontalsize}{4em}\wordChoice{\choice{\TJuist}\choice[correct]{\TFout}}}}

\newcommand{\choiceYes}{{\renewcommand{\choiceminimumhorizontalsize}{3em}\wordChoice{\choice[correct]{\TJa}\choice{\TNee}}}}
\newcommand{\choiceNo }{{\renewcommand{\choiceminimumhorizontalsize}{3em}\wordChoice{\choice{\TJa}\choice[correct]{\TNee}}}}

% Optional nicer formatting for wordChoice in PDF

\let\inlinechoiceorig\inlinechoice

%\makeatletter
%\providecommand{\choiceminimumverticalsize}{\vphantom{$\frac{\sqrt{2}}{2}$}}   % minimum height of boxes (cfr infra)
\providecommand{\choiceminimumverticalsize}{\vphantom{$\tfrac{2}{2}$}}   % minimum height of boxes (cfr infra)
\providecommand{\choiceminimumhorizontalsize}{1em}   % minimum width of boxes (cfr infra)

\newcommand{\inlinechoicesquares}[2][]{%
		\setkeys{choice}{#1}%
		\ifthenelse{\boolean{\choice@correct}}%
		{%
            \ifhandout%
               \fbox{\choiceminimumverticalsize #2}\allowbreak\ignorespaces%
            \else%
               \fcolorbox{blue}{blue!20}{\choiceminimumverticalsize #2}\allowbreak\ignorespaces\setkeys{choice}{correct=false}\ignorespaces%
            \fi%
		}%
		{% else
			\fbox{\choiceminimumverticalsize #2}\allowbreak\ignorespaces%  HACK: wat kleiner, zodat fits on line ... 	
		}%
}

\newcommand{\inlinechoicesquareX}[2][]{%
		\setkeys{choice}{#1}%
		\ifthenelse{\boolean{\choice@correct}}%
		{%
            \ifhandout%
               \framebox[\ifdim\choiceminimumhorizontalsize<\width\width\else\choiceminimumhorizontalsize\fi]{\choiceminimumverticalsize\ #2\ }\allowbreak\ignorespaces\setkeys{choice}{correct=false}\ignorespaces%
            \else%
               \fcolorbox{blue}{blue!20}{\makebox[\ifdim\choiceminimumhorizontalsize<\width\width\else\choiceminimumhorizontalsize\fi]{\choiceminimumverticalsize #2}}\allowbreak\ignorespaces\setkeys{choice}{correct=false}\ignorespaces%
            \fi%
		}%
		{% else
        \ifhandout%
			\framebox[\ifdim\choiceminimumhorizontalsize<\width\width\else\choiceminimumhorizontalsize\fi]{\choiceminimumverticalsize\ #2\ }\allowbreak\ignorespaces%  HACK: wat kleiner, zodat fits on line ... 	
        \fi
		}%
}


\newcommand{\inlinechoicepuntjes}[2][]{%
		\setkeys{choice}{#1}%
		\ifthenelse{\boolean{\choice@correct}}%
		{%
            \ifhandout%
               \dots\ldots\ignorespaces\setkeys{choice}{correct=false}\ignorespaces
            \else%
               \fcolorbox{blue}{blue!20}{\choiceminimumverticalsize #2}\allowbreak\ignorespaces\setkeys{choice}{correct=false}\ignorespaces%
            \fi%
		}%
		{% else
			%\fbox{\choiceminimumverticalsize #2}\allowbreak\ignorespaces%  HACK: wat kleiner, zodat fits on line ... 	
		}%
}

% print niets, maar definieer globale variable \myanswer
%  (gebruikt om oplossingsbladen te printen) 
\newcommand{\inlinechoicedefanswer}[2][]{%
		\setkeys{choice}{#1}%
		\ifthenelse{\boolean{\choice@correct}}%
		{%
               \gdef\myanswer{#2}\setkeys{choice}{correct=false}

		}%
		{% else
			%\fbox{\choiceminimumverticalsize #2}\allowbreak\ignorespaces%  HACK: wat kleiner, zodat fits on line ... 	
		}%
}



%\makeatother

\newcommand{\setchoicedefanswer}{
\ifdefined\HCode
\else
%    \renewenvironment{multipleChoice@}[1][]{}{} % remove trailing ')'
    \let\inlinechoice\inlinechoicedefanswer
\fi
}

\newcommand{\setchoicepuntjes}{
\ifdefined\HCode
\else
    \renewenvironment{multipleChoice@}[1][]{}{} % remove trailing ')'
    \let\inlinechoice\inlinechoicepuntjes
\fi
}
\newcommand{\setchoicesquares}{
\ifdefined\HCode
\else
    \renewenvironment{multipleChoice@}[1][]{}{} % remove trailing ')'
    \let\inlinechoice\inlinechoicesquares
\fi
}
%
\newcommand{\setchoicesquareX}{
\ifdefined\HCode
\else
    \renewenvironment{multipleChoice@}[1][]{}{} % remove trailing ')'
    \let\inlinechoice\inlinechoicesquareX
\fi
}
%
\newcommand{\setchoicelist}{
\ifdefined\HCode
\else
    \renewenvironment{multipleChoice@}[1][]{}{)}% re-add trailing ')'
    \let\inlinechoice\inlinechoiceorig
\fi
}

\setchoicesquareX  % by default list-of-squares with onlineChoice in PDF

% Omdat multicols niet werkt in html: enkel in pdf  (in html zijn langere pagina's misschien ook minder storend)
\newenvironment{xmmulticols}[1][2]{
 \pdfOnly{\begin{multicols}{#1}}%
}{ \pdfOnly{\end{multicols}}}

%
% Te gebruiken in plaats van \section\subsection
%  (in een printstyle kan dan het level worden aangepast
%    naargelang \chapter vs \section style )
% 3/2021: DO NOT USE \xmsubsection !
\newcommand\xmsection\subsection
\newcommand\xmsubsection\subsubsection

% Aanpassen printversie
%  (hier gedefinieerd, zodat ze in xourse kunnen worden gezet/overschreven)
\providebool{parttoc}
\providebool{printpartfrontpage}
\providebool{printactivitytitle}
\providebool{printactivityqrcode}
\providebool{printactivityurl}
\providebool{printcontinuouspagenumbers}
\providebool{numberactivitiesbysubpart}
\providebool{addtitlenumber}
\providebool{addsectiontitlenumber}
\addtitlenumbertrue
\addsectiontitlenumbertrue

% The following three commands are hardcoded in xake, you can't create other commands like these, without adding them to xake as well
%  ( gebruikt in xourses om juiste soort titelpagina te krijgen voor verschillende ximera's )
\newcommand{\activitychapter}[2][]{
    {    
    \ifstrequal{#1}{notnumbered}{
        \addtitlenumberfalse
    }{}
    \typeout{ACTIVITYCHAPTER #2}   % logging
	\chapterstyle
	\activity{#2}
    }
}
\newcommand{\activitysection}[2][]{
    {
    \ifstrequal{#1}{notnumbered}{
        \addsectiontitlenumberfalse
    }{}
	\typeout{ACTIVITYSECTION #2}   % logging
	\sectionstyle
	\activity{#2}
    }
}
% Practices worden als activity getoond om de grote blokken te krijgen online
\newcommand{\practicesection}[2][]{
    {
    \ifstrequal{#1}{notnumbered}{
        \addsectiontitlenumberfalse
    }{}
    \typeout{PRACTICESECTION #2}   % logging
	\sectionstyle
	\activity{#2}
    }
}
\newcommand{\activitychapterlink}[3][]{
    {
    \ifstrequal{#1}{notnumbered}{
        \addtitlenumberfalse
    }{}
    \typeout{ACTIVITYCHAPTERLINK #3}   % logging
	\chapterstyle
	\activitylink[#1]{#2}{#3}
    }
}

\newcommand{\activitysectionlink}[3][]{
    {
    \ifstrequal{#1}{notnumbered}{
        \addsectiontitlenumberfalse
    }{}
    \typeout{ACTIVITYSECTIONLINK #3}   % logging
	\sectionstyle
	\activitylink[#1]{#2}{#3}
    }
}


% Commando om de printstyle toe te voegen in ximera's. Zorgt ervoor dat er geen problemen zijn als je de xourses compileert
% hack om onhandige relative paden in TeX te omzeilen
% should work both in xourse and ximera (pre-112022 only in ximera; thus obsoletes adhoc setup in xourses)
% loads global.sty if present (cfr global.css for online settings!)
% use global.sty to overwrite settings in printstyle.sty ...
\newcommand{\addPrintStyle}[1]{
\iftikzexport\else   % only in PDF
  \makeatletter
  \ifx\@onlypreamble\@notprerr\else   % ONLY if in tex-preamble   (and e.g. not when included from xourse)
    \typeout{Loading printstyle}   % logging
    \usepackage{#1/printstyle} % mag enkel geinclude worden als je die apart compileert
    \IfFileExists{#1/global.sty}{
        \typeout{Loading printstyle-folder #1/global.sty}   % logging
        \usepackage{#1/global}
        }{
        \typeout{Info: No extra #1/global.sty}   % logging
    }   % load global.sty if present
    \IfFileExists{global.sty}{
        \typeout{Loading local-folder global.sty (or TEXINPUTPATH..)}   % logging
        \usepackage{global}
    }{
        \typeout{Info: No folder/global.sty}   % logging
    }   % load global.sty if present
    \IfFileExists{\currfilebase.sty}
    {
        \typeout{Loading \currfilebase.sty}
        \input{\currfilebase.sty}
    }{
        \typeout{Info: No local \currfilebase.sty}
    }
    \fi
  \makeatother
\fi
}

%
%  
% references: Ximera heeft adhoc logica	 om online labels te doen werken over verschillende files heen
% met \hyperref kan de getoonde tekst toch worden opgegeven, in plaats van af te hangen van de label-text
\ifdefined\HCode
% Link to standard \labels, but give your own description
% Usage:  Volg \hyperref[my_very_verbose_label]{deze link} voor wat tijdverlies
%   (01/2020: Ximera-server aangepast om bij class reference-keeptext de link-text NIET te vervangen door de label-text !!!) 
\renewcommand{\hyperref}[2][]{\HCode{<a class="reference reference-keeptext" href="\##1">}#2\HCode{</a>}}
%
%  Link to specific targets  (not tested ?)
\renewcommand{\hypertarget}[1]{\HCode{<a class="ximera-label" id="#1"></a>}}
\renewcommand{\hyperlink}[2]{\HCode{<a class="reference reference-keeptext" href="\##1">}#2\HCode{</a>}}
\fi

% Mmm, quid English ... (for keyword #1 !) ?
\newcommand{\wikilink}[2]{
    \hyperlink{https://nl.wikipedia.org/wiki/#1}{#2}
    \pdfOnly{\footnote{See \url{https://nl.wikipedia.org/wiki/#1}}
    }
}

\renewcommand{\figurename}{Figuur}
\renewcommand{\tablename}{Tabel}

%
% Gedoe om verschillende versies van xourse/ximera te maken afhankelijk van settings
%
% default: versie met antwoorden
% handout: versie voor de studenten, zonder antwoorden/oplossingen
% full: met alles erop en eraan, dus geschikt voor auteurs en/of lesgevers  (bevat in de pdf ook de 'online-only' stukken!)
%
%
% verder kunnen ook opties/variabele worden gezet voor hints/auteurs/uitweidingen/ etc
%
% 'Full' versie
\newtoggle{showonline}
\ifdefined\HCode   % zet default showOnline
    \toggletrue{showonline} 
\else
    \togglefalse{showonline}
\fi

% Full versie   % deprecated: see infra
\newcommand{\printFull}{
    \hintstrue
    \handoutfalse
    \toggletrue{showonline} 
}

\ifdefined\shouldPrintFull   % deprecated: see infra
    \printFull
\fi



% Overschrijf onlineOnly  (zoals gedefinieerd in ximera.cls)
\ifhandout   % in handout: gebruik de oorspronkelijke ximera.cls implementatie  (is dit wel nodig/nuttig?)
\else
    \iftoggle{showonline}{%
        \ifdefined\HCode
          \RenewEnviron{onlineOnly}{\bgroup\BODY\egroup}   % showOnline, en we zijn  online, dus toon de tekst
        \else
          \RenewEnviron{onlineOnly}{\bgroup\color{red!50!black}\BODY\egroup}   % showOnline, maar we zijn toch niet online: kleur de tekst rood 
        \fi
    }{%
      \RenewEnviron{onlineOnly}{}  % geen showOnline
    }
\fi

% hack om na hoofding van definition/proposition/... als dan niet op een nieuwe lijn te starten
% soms is dat goed en mooi, en soms niet; en in HTML is het nu (2/2020) anders dan in pdf
% vandaar suggestie om 
%     \begin{definition}[Nieuw concept] \nl
% te gebruiken als je zeker een newline wil na de hoofdig en titel
% (in het bijzonder itemize zonder \nl is 'lelijk' ...)
\ifdefined\HCode
\newcommand{\nl}{}
\else
\newcommand{\nl}{\ \par} % newline (achter heading van definition etc.)
\fi


% \nl enkel in handoutmode (ihb voor \wordChoice, die dan typisch veeeel langer wordt)
\ifdefined\HCode
\providecommand{\handoutnl}{}
\else
\providecommand{\handoutnl}{%
\ifhandout%
  \nl%
\fi%
}
\fi

% Could potentially replace \pdfOnline/\begin{onlineOnly} : 
% Usage= \ifonline{Hallo surfer}{Hallo PDFlezer}
\providecommand{\ifonline}[2]%
{
\begin{onlineOnly}#1\end{onlineOnly}%
\pdfOnly{#2}
}%


%
% Maak optionele 'basic' en 'extended' versies van een activity
%  met environment basicOnly, basicSkip en extendedOnly
%
%  (
%   Dit werkt ENKEL in de PDF; de online versies tonen (minstens voorklopig) steeds 
%   het default geval met printbasicversion en printextendversion beide FALSE
%  )
%
\providebool{printbasicversion}
\providebool{printextendedversion}   % not properly implemented
\providebool{printfullversion}       % presumably print everything (debug/auteur)
%
% only set these in xourses, and BEFORE loading this preamble
%
%\newif\ifshowbasic     \showbasictrue        % use this line in xourse to show 'basic' sections
%\newif\ifshowextended  \showextendedtrue     % use this line in xourse to show 'extended' sections
%
%
%\ifbool{showbasic}
%      { \NewEnviron{basicOnly}{\BODY} }    % if yes: just print contents
%      { \NewEnviron{basicOnly}{}      }    % if no:  completely ignore contents
%
%\ifbool{showbasic}
%      { \NewEnviron{basicSkip}{}      }
%      { \NewEnviron{basicSkip}{\BODY} }
%

\ifbool{printextendedversion}
      { \NewEnviron{extendedOnly}{\BODY} }
      { \NewEnviron{extendedOnly}{}      }
      


\ifdefined\HCode    % in html: always print
      {\newenvironment*{basicOnly}{}{}}    % if yes: just print contents
      {\newenvironment*{basicSkip}{}{}}    % if yes: just print contents
\else

\ifbool{printbasicversion}
      {\newenvironment*{basicOnly}{}{}}    % if yes: just print contents
      {\NewEnviron{basicOnly}{}      }    % if no:  completely ignore contents

\ifbool{printbasicversion}
      {\NewEnviron{basicSkip}{}      }
      {\newenvironment*{basicSkip}{}{}}

\fi

\usepackage{float}
\usepackage[rightbars,color]{changebar}

% Full versie
\ifbool{printfullversion}{
    \hintstrue
    \handoutfalse
    \toggletrue{showonline}
    \printbasicversionfalse
    \cbcolor{red}
    \renewenvironment*{basicOnly}{\cbstart}{\cbend}
    \renewenvironment*{basicSkip}{\cbstart}{\cbend}
    \def\xmtoonprintopties{FULL}   % will be printed in footer
}
{}
      
%
% Evalueer \ifhints IN de environment
%  
%
%\RenewEnviron{hint}
%{
%\ifhandout
%\ifhints\else\setbox0\vbox\fi%everything in een emty box
%\bgroup 
%\stepcounter{hintLevel}
%\BODY
%\egroup\ignorespacesafterend
%\addtocounter{hintLevel}{-1}
%\else
%\ifhints
%\begin{trivlist}\item[\hskip \labelsep\small\slshape\bfseries Hint:\hspace{2ex}]
%\small\slshape
%\stepcounter{hintLevel}
%\BODY
%\end{trivlist}
%\addtocounter{hintLevel}{-1}
%\fi
%\fi
%}

% Onafhankelijk van \ifhandout ...? TO BE VERIFIED
\RenewEnviron{hint}
{
\ifhints
\begin{trivlist}\item[\hskip \labelsep\small\bfseries Hint:\hspace{2ex}]
\small%\slshape
\stepcounter{hintLevel}
\BODY
\end{trivlist}
\addtocounter{hintLevel}{-1}
\else
\iftikzexport   % anders worden de tikz tekeningen in hints niet gegenereerd ?
\setbox0\vbox\bgroup
\stepcounter{hintLevel}
\BODY
\egroup\ignorespacesafterend
\addtocounter{hintLevel}{-1}
\fi % ifhandout
\fi %ifhints
}

%
% \tab sets typewriter-tabs (e.g. to format questions)
% (Has no effect in HTML :-( ))
%
\usepackage{tabto}
\ifdefined\HCode
  \renewcommand{\tab}{\quad}    % otherwise dummy .png's are generated ...?
\fi


% Also redefined in  preamble to get correct styling 
% for tikz images for (\tikzexport)
%

\theoremstyle{definition} % Bold titels
\makeatletter
\let\proposition\relax
\let\c@proposition\relax
\let\endproposition\relax
\makeatother
\newtheorem{proposition}{Eigenschap}


%\instructornotesfalse

% logic with \ifhandoutin ximera.cls unclear;so overwrite ...
\makeatletter
\@ifundefined{ifinstructornotes}{%
  \newif\ifinstructornotes
  \instructornotesfalse
  \newenvironment{instructorNotes}{}{}
}{}
\makeatother
\ifinstructornotes
\else
\renewenvironment{instructorNotes}%
{%
    \setbox0\vbox\bgroup
}
{%
    \egroup
}
\fi

% \RedeclareMathOperator
% from https://tex.stackexchange.com/questions/175251/how-to-redefine-a-command-using-declaremathoperator
\makeatletter
\newcommand\RedeclareMathOperator{%
    \@ifstar{\def\rmo@s{m}\rmo@redeclare}{\def\rmo@s{o}\rmo@redeclare}%
}
% this is taken from \renew@command
\newcommand\rmo@redeclare[2]{%
    \begingroup \escapechar\m@ne\xdef\@gtempa{{\string#1}}\endgroup
    \expandafter\@ifundefined\@gtempa
    {\@latex@error{\noexpand#1undefined}\@ehc}%
    \relax
    \expandafter\rmo@declmathop\rmo@s{#1}{#2}}
% This is just \@declmathop without \@ifdefinable
\newcommand\rmo@declmathop[3]{%
    \DeclareRobustCommand{#2}{\qopname\newmcodes@#1{#3}}%
}
\@onlypreamble\RedeclareMathOperator
\makeatother


%
% Engelse vertaling, vooral in mathmode
%
% 1. Algemene procedure
%
\ifdefined\isEn
 \newcommand{\nlen}[2]{#2}
 \newcommand{\nlentext}[2]{\text{#2}}
 \newcommand{\nlentextbf}[2]{\textbf{#2}}
\else
 \newcommand{\nlen}[2]{#1}
 \newcommand{\nlentext}[2]{\text{#1}}
 \newcommand{\nlentextbf}[2]{\textbf{#1}}
\fi

%
% 2. Lijst van erg veel gebruikte uitdrukkingen
%

% Ja/Nee/Fout/Juits etc
%\newcommand{\TJa}{\nlentext{ Ja }{ and }}
%\newcommand{\TNee}{\nlentext{ Nee }{ No }}
%\newcommand{\TJuist}{\nlentext{ Juist }{ Correct }
%\newcommand{\TFout}{\nlentext{ Fout }{ Wrong }
\newcommand{\TWaar}{\nlentext{ Waar }{ True }}
\newcommand{\TOnwaar}{\nlentext{ Vals }{ False }}
% Korte bindwoorden en, of, dus, ...
\newcommand{\Ten}{\nlentext{ en }{ and }}
\newcommand{\Tof}{\nlentext{ of }{ or }}
\newcommand{\Tdus}{\nlentext{ dus }{ so }}
\newcommand{\Tendus}{\nlentext{ en dus }{ and thus }}
\newcommand{\Tvooralle}{\nlentext{ voor alle }{ for all }}
\newcommand{\Took}{\nlentext{ ook }{ also }}
\newcommand{\Tals}{\nlentext{ als }{ when }} %of if?
\newcommand{\Twant}{\nlentext{ want }{ as }}
\newcommand{\Tmaal}{\nlentext{ maal }{ times }}
\newcommand{\Toptellen}{\nlentext{ optellen }{ add }}
\newcommand{\Tde}{\nlentext{ de }{ the }}
\newcommand{\Thet}{\nlentext{ het }{ the }}
\newcommand{\Tis}{\nlentext{ is }{ is }} %zodat is in text staat in mathmode (geen italics)
\newcommand{\Tmet}{\nlentext{ met }{ where }} % in situaties e.g met p < n --> where p < n
\newcommand{\Tnooit}{\nlentext{ nooit }{ never }}
\newcommand{\Tmaar}{\nlentext{ maar }{ but }}
\newcommand{\Tniet}{\nlentext{ niet }{ not }}
\newcommand{\Tuit}{\nlentext{ uit }{ from }}
\newcommand{\Ttov}{\nlentext{ t.o.v. }{ w.r.t. }}
\newcommand{\Tzodat}{\nlentext{ zodat }{ such that }}
\newcommand{\Tdeth}{\nlentext{de }{th }}
\newcommand{\Tomdat}{\nlentext{omdat }{because }} 


%
% Overschrijf addhoc commando's
%
\ifdefined\isEn
\renewcommand{\pernot}{\overset{\mathrm{notation}}{=}}
\RedeclareMathOperator{\bld}{im}     % beeld
\RedeclareMathOperator{\graf}{graph}   % grafiek
\RedeclareMathOperator{\rico}{slope}   % richtingcoëfficient
\RedeclareMathOperator{\co}{co}       % coordinaat
\RedeclareMathOperator{\gr}{deg}       % graad

% Operators
\RedeclareMathOperator{\bgsin}{arcsin}
\RedeclareMathOperator{\bgcos}{arccos}
\RedeclareMathOperator{\bgtan}{arctan}
\RedeclareMathOperator{\bgcot}{arccot}
\RedeclareMathOperator{\bgsinh}{arcsinh}
\RedeclareMathOperator{\bgcosh}{arccosh}
\RedeclareMathOperator{\bgtanh}{arctanh}
\RedeclareMathOperator{\bgcoth}{arccoth}

\fi


% HACK: use 'oplossing' for 'explanation' ...
\let\explanation\relax
\let\endexplanation\relax
% \newenvironment{explanation}{\begin{oplossing}}{\end{oplossing}}
\newcounter{explanation}

\ifhandout%
    \NewEnviron{explanation}[1][toon]%
    {%
    \RenewEnviron{verbatim}{ \red{VERBATIM CONTENT MISSING IN THIS PDF}} %% \expandafter\verb|\BODY|}

    \ifthenelse{\equal{\detokenize{#1}}{\detokenize{toon}}}
    {
    \def\PH@Command{#1}% Use PH@Command to hold the content and be a target for "\expandafter" to expand once.

    \begin{trivlist}% Begin the trivlist to use formating of the "Feedback" label.
    \item[\hskip \labelsep\small\slshape\bfseries Explanation:% Format the "Feedback" label. Don't forget the space.
    %(\texttt{\detokenize\expandafter{\PH@Command}}):% Format (and detokenize) the condition for feedback to trigger
    \hspace{2ex}]\small%\slshape% Insert some space before the actual feedback given.
    \BODY
    \end{trivlist}
    }
    {  % \begin{feedback}[solution]   \BODY     \end{feedback}  }
    }
    }    
\else
% ONLY for HTML; xmoplossing is styled with css, and is not, and need not be a LaTeX environment
% THUS: it does NOT use feedback anymore ...
%    \NewEnviron{oplossing}{\begin{expandable}{xmoplossing}{\nlen{Toon uitwerking}{Show solution}}{\BODY}\end{expandable}}
    \newenvironment{explanation}[1][toon]
   {%
       \begin{expandable}{xmoplossing}{}
   }
   {%
   	   \end{expandable}
   } 
\fi

\title{Least-Squares Approximation} \license{CC BY-NC-SA 4.0}

\begin{document}

\begin{abstract}
\end{abstract}
\maketitle

\begin{onlineOnly}
\section*{Least-Squares Approximation}
\end{onlineOnly}

% \begin{center}
% \begin{tikzpicture}[scale=1]
%   \filldraw[orange](-0.25,3.5)--(0.25,3.5)--(1.5,0)--(-1.5,0)--cycle;
%   \filldraw[orange] (0,0) ellipse (2cm and 1cm);
%   \filldraw[orange] (0,3.5) ellipse (0.25cm and 0.15cm);
% \end{tikzpicture}
 
% UNDER CONSTRUCTION -- COMING SOON
% \end{center}

Often an exact solution to a problem in applied mathematics is difficult or impossible to obtain. However, it is usually just as useful to find an approximation to a solution. In particular, finding ``linear approximations" is a potent technique in applied mathematics. One basic case is the situation where a system
of linear equations has no solution, and it is desirable to find a ``best approximation" to a solution to the
system. In this section best approximations are defined and a method for finding them is described. 
\subsection*{Best Approximate Solutions}
\begin{exploration}\label{exp:leastSq1}
    Let
    $$A=\begin{bmatrix}3 & 1\\1 & 2\\1 & 2\end{bmatrix}\quad\text{and}\quad \vec{b}=\begin{bmatrix}2\\1\\3\end{bmatrix}$$
    Consider the matrix equation $A\vec{x}=\vec{b}$.
    A quick examination of the last two rows should convince you that this equation has no solutions.  In other words, $\vec{b}$ is not in the span of the columns of $A$.

    If $\vec{z}$ were an exact solution to $A\vec{x}=\vec{b}$, then $\vec{b}-A\vec{z}$ would be $\vec{0}$.  Since the equation does not have a solution, we will attempt to find the next best thing to a solution by finding $\vec{z}$ such that $\norm{\vec{b}-A\vec{z}}$ is as small as possible.  The quantity $\norm{\vec{b}-A\vec{z}}$ is called the \emph{error}.

    The following GeoGebra interactive will help you understand the geometry behind finding $\vec{z}$.  
    
    RIGHT-CLICK and DRAG to rotate the image for a better view.

\begin{onlineOnly}
\begin{center}
\geogebra{mm7wauhw}{950}{650}
\end{center}
\end{onlineOnly}

 Record your best guess for $\vec{z}$ -- you will have a chance to check your answer in Example \ref{ex:leastSquares1}.

What did you discover about the geometry of minimizing $\norm{\vec{b}-A\vec{z}}$? 
Select all that apply.

\begin{selectAll}
    \choice{$\vec{z}$ is orthogonal to the plane spanned by the columns of $A$.}
    \choice{$\norm{\vec{b}-A\vec{z}}$ is orthogonal to $\text{col}(A)$.}
    \choice[correct]{$\vec{b}-A\vec{z}$ is orthogonal to $\text{col}(A)$.}
    \choice{$A\vec{z}$ is orthogonal to $\text{col}(A)$.}
     \choice[correct]{$A\vec{z}$ is an orthogonal projection of $\vec{b}$ onto $\text{col}(A)$.}
\end{selectAll}

Our geometric observations will help us develop a method for finding $\vec{z}$ .  
\end{exploration}

  Suppose $A$ is an $m\times n$ matrix, and $\vec{b}$ is a column vector in $\RR^m$.  Consider the matrix equation $A\vec{x}=\vec{b}$. If this equation does not have a solution, we can attempt to find a best approximation by finding $\vec{z}$ which minimizes the error $\norm{\vec{b}-A\vec{z}}$.  
  
  In the case when $\text{col}(A)$ is a subspace of $\RR^3$, 
  we can see geometrically that $\vec{z}$ is the best approximation if and only if $A\vec{z}$ is an orthogonal projection of $\vec{b}$ onto $\text{col}(A)$, and the error is the magnitude of $\left(A\vec{z}\right)^{\perp}=\vec{b}-A\vec{z}$, as shown below.

\begin{center}
\tdplotsetmaincoords{70}{130}
	\begin{tikzpicture}[scale=0.8]
\filldraw[blue, opacity=0.2] (0,0,0)--(5,0,0)--(5,0,5)--(0,0,5)--cycle;
\node[label={left:$\vec{b}-A\vec{z}$}] at (4.5,0.6,0) {};
\node[label={above:$\vec{b}$}] at (2,1.5,2) {};
\node[label={above:$A\vec{z}$}] at (2.2,0,3.5) {};
\draw[->,line width=1.5mm, -stealth, blue,opacity=0.5](4,0,4)--(4,3,4) ;
   \draw[->,line width=0.8mm, -stealth, black](0,0,0)--(1,0,5) ;
    \draw[->,line width=0.8mm, -stealth, black](0,0,0)--(5,0,1) ;
   \draw[->,line width=0.8mm, -stealth, red](0,0,0)--(4,3,4) ;
    \draw[->,line width=1.5mm, -stealth, red,opacity=0.2](0,0,0)--(4,0,4) ;
    %\node[label={below: $\vec{w}=\mbox{proj}_W\vec{x}=\mbox{proj}_{\vec{f}_1}\vec{x}+\mbox{proj}_{\vec{f}_2}\vec{x}$}] at (3,-1,3) {};
      \end{tikzpicture}
\end{center}

What we observed for %a
the 
$3\times 2$ matrix $A$, holds in general.  We will use this fact to find $\vec{z}$.

Every vector in $\text{col}(A)$ can be written in the form $A\vec{x}$ for some $\vec{x}$ in $\RR^m$.  Our goal is to find $\vec{z}$ such that $A\vec{z}$ is the orthogonal projection of $\vec{b}$ onto $\text{col}(A)$. By Corollary \ref{cor:orthProjOntoW}, every vector $A\vec{x}$ in $\text{col}(A)$ is orthogonal to $\vec{b}-A\vec{z}$.  Therefore, we have
\begin{eqnarray*}
    0=(A\vec{x})\dotp(\vec{b}-A\vec{x})&=&\left(A\vec{x}\right)^T(\vec{b}-A\vec{z})\\
    &=&\vec{x}^TA^T(\vec{b}-A\vec{x})\\
    &=&\vec{x}\dotp\left(A^T(\vec{b}-A\vec{z})\right)
\end{eqnarray*}
But this means that $A^T(\vec{b}-A\vec{z})$ is orthogonal to \emph{every} vector of $\RR^m$, including itself.  Therefore,
\begin{eqnarray}
A^T(\vec{b}-A\vec{z})&=&\vec{0}\nonumber\\
A^T\vec{b}-A^TA\vec{z}&=&\vec{0}\nonumber\\
A^TA\vec{z}&=&A^T\vec{b}\label{eq:normalForZ}
\end{eqnarray}

The system of linear equations in (\ref{eq:normalForZ}) is called \emph{normal equations for} $\vec{z}$.  If $A^TA$ is invertable, then we have the following formula for finding $\vec{z}$

\begin{equation}\label{eq:leastSquaresZ}
    \vec{z}=(A^TA)^{-1}A^T\vec{b}
\end{equation}

We will return to the question of invertability of $A^TA$ in Theorem \ref{th:ATAinverse}.  For now, let's revisit the problem posed in Exploration \ref{exp:leastSq1}.

\begin{example}\label{ex:leastSquares1}
We now return to the matrix equation $A\vec{x}=\vec{b}$ of Exploration \ref{exp:leastSq1} to find $\vec{z}$ that best approximates a solution.

\begin{explanation}
    Recall that $$A=\begin{bmatrix}3 & 1\\1 & 2\\1 & 2\end{bmatrix}\quad\text{and}\quad \vec{b}=\begin{bmatrix}2\\1\\3\end{bmatrix}$$

In this case, $(A^TA)^{-1}$ exists.  Applying equation (\ref{eq:leastSquaresZ}), we compute
\begin{eqnarray*}
\vec{z}&=&\left(\begin{bmatrix}3 & 1 & 1\\1 & 2 & 2\end{bmatrix}\begin{bmatrix}3 & 1\\1 & 2\\1& 2\end{bmatrix}\right)^{-1}\begin{bmatrix}3 & 1 & 1\\1 & 2 & 2\end{bmatrix}\begin{bmatrix}2\\1\\3\end{bmatrix}\\
&=&\begin{bmatrix}11 & 7\\7 & 9\end{bmatrix}^{-1}\begin{bmatrix}10\\10\end{bmatrix}=\begin{bmatrix}0.18 & -0.14\\-0.14 & 0.22\end{bmatrix}\begin{bmatrix}10\\10\end{bmatrix}\\
&=&\begin{bmatrix}0.4\\0.8\end{bmatrix}
    \end{eqnarray*}    
\end{explanation}
Compare this answer to your guess in Exploration \ref{exp:leastSq1}.  If your guess was correct, nice job!  If your guess was different, try setting $\vec{z}$ to the correct answer and examine the geometry of the problem.    
\end{example}

We now come back to the question of when $A^TA$ is invertable.

\begin{theorem}\label{th:ATAinverse}
If columns of matrix $A$ are linearly independent, then $A^TA$ is invertable.    
\end{theorem}

\begin{proof}
    Let $A$ be a matrix with linearly independent columns.  We will show that $\left(A^TA\right)\vec{x}=\vec{0}$ has only the trivial solution.  For $\vec{x}$, a solution of $A^TA\vec{x}=\vec{0}$, we have
    \begin{eqnarray*}
        \norm{A\vec{x}}^2&=&(A\vec{x})\dotp(A\vec{x})\\
        &=&\left(A\vec{x}\right)^TA\vec{x}\\
        &=&\vec{x}^TA^TA\vec{x}\\
        &=&\vec{x}^T\dotp\vec{0}=0
    \end{eqnarray*}
    Therefore $A\vec{x}=\vec{0}$.  By linear independence of the columns of $A$ we conclude that $\vec{x}=\vec{0}$.
\end{proof}    

    We summarize our findings in the following theorem.

    \begin{theorem}\label{th:bestApprox}
    Let $A$ be an $m\times n$ matrix, let $\vec{b}$ be a column vector in $\RR^m$.  Consider the matrix equation
    $$A\vec{x}=\vec{b}$$
    \begin{enumerate}
        \item Any solution $\vec{z}$ to the normal equations 
        $$\left(A^TA\right)\vec{z}=A^T\vec{b}$$
        is a best approximation to a solution to $A\vec{x}=\vec{b}$ in the sense that $\norm{\vec{b}-A\vec{z}}$ is minimized.
        \item If the columns of $A$ are linearly independent, then $A^TA$ is invertible and $\vec{z}$ is given uniquely by 
        $$\vec{z}=\left(A^TA\right)^{-1}A^T\vec{b}$$
    \end{enumerate}
    \end{theorem}

    \begin{example}\label{ex:leastSq2}
        The sytem of linear equations
        $$\begin{matrix}3x & - & y&=&4\\x&+&2y&=&0\\2x&+&y&=&1\end{matrix}$$
        has no solution.  Find the vector $\vec{z}$ that best approximates a solution.

        \begin{explanation}
            In this case,
            $$A=\begin{bmatrix}3&-1\\1&2\\2&1\end{bmatrix}$$            $$A^TA=\begin{bmatrix}3&1&2\\-1&2&1\end{bmatrix}\begin{bmatrix}3&-1\\1&2\\2&1\end{bmatrix}=\begin{bmatrix}14&1\\1&6\end{bmatrix}$$
            is invertible.  Using the normal equation $\left(A^TA\right)\vec{z}=A^T\vec{z}$ we compute, $$\begin{bmatrix}14&1\\1&6\end{bmatrix}\vec{z}=\begin{bmatrix}14\\-3\end{bmatrix}$$ $$\vec{z}=\frac{1}{83}\begin{bmatrix}87\\-56\end{bmatrix}$$
            With this vector $\vec{z}$, the left sides of the equations become
            $$\begin{matrix}3(87/83) & - & (-56/83)&\approx&3.82\\(87/83)&+&2(-56/83)&\approx&-0.30\\2(87/83)&+&(-56/83)&\approx&1.42\end{matrix}$$
            This is as close as possible to a solution.
        \end{explanation}
    \end{example}

\begin{example}\label{ex:leastSquares3}
The average number $g$ of goals per game scored by a hockey player seems to be related linearly to
two factors: the number $x_1$ of years of experience and the number $x_2$ of goals in the preceding 10
games. The data on the following page were collected on four players. Find the linear function $g=a_0+a_1x_1+a_2x_2$ that best fits the data.
$$\begin{array}{|c|c|c|} 
 \hline g&x_1&x_2\\ \hline 0.8&5&3\\  0.8&3&4\\
 0.6 &1&5\\
 0.4 &2&1\\
 \hline 
 \end{array}$$

\begin{explanation}
    If the relationship is given by $g=a_0+a_1x_1+a_2x_2$, then the data can be described as follows:
    $$\begin{bmatrix}1&5&3\\1&3&4\\1&1&5\\1&2&1\end{bmatrix}\begin{bmatrix}a_1\\a_2\\a_3\end{bmatrix}=\begin{bmatrix}0.8\\0.8\\0.6\\0.4\end{bmatrix}$$
    Using Theorem \ref{th:bestApprox}, we get
    $$\vec{z}=\underbrace{\frac{1}{42}\begin{bmatrix}
        119&-17&-19\\-17&5&1\\-19&1&5
    \end{bmatrix}}_{\left(A^TA\right)^{-1}}\underbrace{\begin{bmatrix}
        1&1&1&1\\5&3&1&2\\3&4&5&1
    \end{bmatrix}}_{A^T}\begin{bmatrix}
        0.8\\0.8\\0.6\\0.4
    \end{bmatrix}=\begin{bmatrix}
        0.14\\0.09\\0.08
    \end{bmatrix}$$
    Hence the best-fitting function is $g=0.14+0.09x_1+0.08x_2$.  
\end{explanation}    
\end{example}

\subsection*{Polynomial Regression}
In \href{https://ximera.osu.edu/oerlinalg/LinearAlgebra/APP-0070/main}{Curve Fitting}, we discussed how to fit a function to a set of data points, so that the graph of the function passes through each of the points.  We also discussed why doing so is sometimes impossible (two points lie one above the other), and may not even be desirable.  In this section we will learn how to approximate a collection of data points with a line (or a curve) that fits the ``trend" of the points.  We will start with data that fit a linear pattern.  

\begin{exploration}\label{exp:leastSq2}
    Consider the points $(1,1)$, $(2, 3)$ and $(4,4)$.  These points do not lie on a straight line, but they have a general upward linear trend.  (Typically there would be many more points to consider, but we will limit our exploration to what we can do by hand.)  Our goal is to find a line that fits these points as closely as possible.  

    We are looking for a function $f$ of the form $f(x)=ax+b$ such that the following infeasible system is satisfied as closely as possible
    $$\begin{matrix}a(1)&+&b&=&1\\a(2)&+&b&=&3\\a(4)&+&b&=&4\end{matrix}$$

    From the first part of this section we know how to find a best approximation.  By Theorem \ref{th:bestApprox}, we have
    \begin{eqnarray*}
\vec{z}=\begin{bmatrix}a\\b\end{bmatrix}&=&\left(A^TA\right)^{-1}A^T\vec{b}\\
&=&\left(\begin{bmatrix}1&2&4\\1&1&1\end{bmatrix}\begin{bmatrix}
1&1\\2&1\\4&1\end{bmatrix}\right)^{-1}\begin{bmatrix}1&2&4\\1&1&1\end{bmatrix}\begin{bmatrix}
    1\\3\\4
\end{bmatrix}\\
&=&\begin{bmatrix}13/14\\1/2\end{bmatrix}
\end{eqnarray*}
According to our computations, the line that best fits the data is given by $$f(x)=\frac{13}{14}x+\frac{1}{2}$$
Let's take a look.
\begin{center}
\begin{tikzpicture}[scale=1]
\draw[thin,gray!40] (-1,-1) grid (5,5);
  \draw[<->] (-1,0)--(5,0);
  \draw[<->] (0,-1)--(0,5);
  \draw[line width=2pt](-1,-0.43)--(5,5.14) ;
\fill[blue] (1,1)  circle (0.08cm);
\fill[blue] (2,3) circle (0.08cm);
\fill[blue] (4,4) circle (0.08cm);
 \end{tikzpicture}
 \end{center}

We found this fit by minimizing $\norm{\vec{b}-A\vec{z}}$.  We will investigate the meaning of this expression in relation to the line and the data points.
\begin{equation}
    \vec{b}-A\vec{z}=\begin{bmatrix}1\\3\\4\end{bmatrix}-\begin{bmatrix}(13/14)(1)+1/2\\(13/14)(2)+1/2\\{(13/14)(4)+1/2}\end{bmatrix}\approx\begin{bmatrix}
        -0.43\\0.64\\-0.21
    \end{bmatrix}
\end{equation}
Observe that each entry of $\vec{b}-A\vec{z}$ is the signed vertical distance between a particular point and the line.

\begin{equation}
    \norm{\vec{b}-A\vec{z}}^2=(-0.43)^2+0.64^2+(-0.21)^2\approx 0.64
\end{equation}

Minimizing $\norm{\vec{b}-A\vec{z}}$ also minimizes $\norm{\vec{b}-A\vec{z}}^2$.  Therefore, what we have minimized is the sum of squares of the vertical distances between the data points and the line.
The following GeoGebra interactive will help you explore this idea.

\begin{onlineOnly}
\begin{center}
\geogebra{crgw4usb}{950}{650}
\end{center}
\end{onlineOnly}
\end{exploration}


In Exploration \ref{exp:leastSq2} we discovered that $\norm{\vec{b}-A\vec{z}}^2$ is the sum of vertical distances between the given data points and the proposed line.  By minimizing $\norm{\vec{b}-A\vec{z}}$, we minimize the sum of squares of vertical distances.  This observation holds in general.  Given a collection of points $(x_1, y_1), (x_2, y_2),\dots ,(x_n, y_n)$, finding a linear function of the form $f(x)=ax+b$ that best fits the points we would find a best solution to the system
$$\begin{bmatrix}x_1&1\\x_2&1\\\vdots&\vdots\\x_n&1\end{bmatrix}\begin{bmatrix}a\\b\end{bmatrix}=\begin{bmatrix}y_1\\y_2\\\vdots\\y_n\end{bmatrix}$$
by minimizing
$$\norm{\begin{bmatrix}y_1-(ax_1+b)\\y_2-(ax_2+b)\\\vdots\\y_n-(ax_n+b)\end{bmatrix}}^2=(y_1-(ax_1+b))^2+(y_2-(ax_2+b))^2+\dots +(y_n-(ax_n+b))^2$$
A geometric interpretation of $y_i-(ax_i+b)$ is shown below.


\begin{center}
\begin{tikzpicture}[scale=1.5]
\draw[thin,gray!40] (-1,-1) grid (4,4);
   \draw[line width=2pt](-1,-1)--(2.75,4) ;
\fill[blue] (2,1) node[below right]{$(x_i, y_i)$}  circle (0.08cm);
\fill[red] (2,3) node[above left] {$(x_i, ax_i+b)$}circle (0.08cm);
\draw[line width=0.5pt, dashed](2,1)--(2,3);
\node[] at (2.8, 2)   {$y_i-(ax_i+b)$};
 \end{tikzpicture}
 \end{center}

 The line we obtain in this fashion is called a \emph{line of best fit} or a \emph{trendline}, and the method we used is referred to as the  \emph{method of least squares}.

 We can apply the method of least squares to find best fitting non-linear functions.  

 \begin{example}\label{ex:leastSquaresPoly}
Find the least squares approximating quadratic polynomial of the form $f(x)=ax^2+bx+c$ for the following points.
$$(-3, 3), (-1, 1), (0, 1), (1, 2), (3, 4)$$

\begin{explanation}
    We are looking for an approximate solution to the system of equations
    $$\begin{matrix}a(-3)^2&+&b(-3)&+&c&=&3\\a(-1)^2&+&b(-1)&+&c&=&1\\a(0)^2&+&b(0)&+&c&=&1\\a(1)^2&+&b(1)&+&c&=&2\\a(3)^2&+&b(3)&+&c&=&4\end{matrix}$$
    This corresponds to the matrix equation
    $$\begin{bmatrix}9&-3&1\\1&-1&1\\0&0&1\\1&1&1\\9&3&1\end{bmatrix}\begin{bmatrix}a\\b\\c\end{bmatrix}=\begin{bmatrix}3\\1\\1\\2\\4\end{bmatrix}$$
    Using the normal equations, we obtain
    $$A^TA\vec{z}=A^T\vec{b}$$
    $$\begin{bmatrix}164&0&20\\0&20&0\\20&0&5\end{bmatrix}\vec{z}=\begin{bmatrix}66\\11\\4\end{bmatrix}$$
    Solving for $\vec{z}$ yields
    $$\vec{z}=\begin{bmatrix}0.26\\0.20\\1.15\end{bmatrix}$$

    Therefore, the quadratic function of best fit is given by $f(x)=0.26x^2+0.2x+1.15$.  You can see the graph and the points shown below.

\begin{onlineOnly}
\begin{center}
\desmos{kxbnnn5cwr}{950}{650}
\end{center}
\end{onlineOnly}

 
\end{explanation}
 \end{example}

In this section we chose not to dwell on the tedious process of finding $\left(A^TA\right)^{-1}$.  You should be able to perform this computation by hand for smaller matrices.  For larger matrices, using technology is the only reasonable option.  Below, we go through the process of finding the solution for Example \ref{ex:leastSquaresPoly} using Octave.

To use Octave, go to the \href{https://sagecell.sagemath.org/}{Sage Math Cell Webpage}, copy the code below into the cell, select OCTAVE as the language, and press EVALUATE.

\begin{verbatim}
% Define matrix A
A=[9 -3 1;1 -1 1;0 0 1;1 1 1; 9 3 1];
% Define vector b
b=[3;1;1;2;4];
% Let B be the inverse of A^TA.  
B= inv(transpose(A)*A);
% Now we solve for z
ans=B*transpose(A)*b
\end{verbatim}

You will be able to re-use this code to complete some of the more tedious homework problems.

 \begin{example}\label{ex:leastSq3}
     Given the data points $(-1, 0)$, $(0,1)$, and $(1,4)$, find the least squares approximating function of the form $f(x)=ax+b2^x$.
\begin{explanation}
    We are looking for an approximate solution to the system of equations
    $$\begin{matrix}a(-1)&+&b(2^{-1})&=&0\\a(0)&+&b(2^{0})&=&1\\a(1)&+&b(2^{1})&=&4\end{matrix}\\$$
    This corresponds to the matrix equation
    $$\begin{bmatrix}-1&1/2\\0&1\\1&2\end{bmatrix}\begin{bmatrix}a\\b\end{bmatrix}=\begin{bmatrix}0\\1\\4\end{bmatrix}$$

    Using the normal equations, we obtain
    $$A^TA\vec{z}=A^T\vec{b}$$
    $$\frac{1}{4}\begin{bmatrix}8&6\\6&21\end{bmatrix}\vec{z}=\begin{bmatrix}4\\9\end{bmatrix}$$
Solving for $\vec{z}$ yields
$$\vec{z}=\begin{bmatrix}10/11\\16/11\end{bmatrix}$$
Therefore, the function of best fit (of the given form) is given by $$f(x)=\frac{10}{11}x+\frac{16}{11}(2^x)$$

\begin{onlineOnly}
\begin{center}
\desmos{y2fgw13hki}{950}{650}
\end{center}
\end{onlineOnly}
\end{explanation}
 \end{example}

\section*{Practice Problems}

\begin{problem}\label{prob:leastSq1}
Find the best approximation to a solution to the system of equations.
$$\begin{matrix}3x&+&y&+&z&=&6\\2x&+&3y&-&z&=&1\\2x&-&y&+&z&=&0\\3x&-&3y&+&3z&=&8\end{matrix}$$
Enter answers in fraction form below.
$$x=\answer{\frac{-20}{12}},\quad y=\answer{\frac{46}{12}},\quad z=\answer{\frac{95}{12}}$$
\end{problem}

\begin{problem}
Find a linear function of best fit for each of the following sets of data points.  Examine how well your line fits the points by typing the equation of the line into the Desmos window.

\begin{problem}\label{prob:leastSq2a}
$$(2,4), (4,3), (7,2), (8,1)$$
Enter your answers in fraction form.
$$f(x)=\answer{\frac{-6}{13}}x+\answer{\frac{64}{13}}$$
\begin{onlineOnly}
\begin{center}
\desmos{tktztr0nvo}{950}{650}
\end{center}
\end{onlineOnly}
\end{problem}

\begin{problem}\label{prob:leastSq2b}
$$(-2, 3), (-1,1), (0,0), (1, -2), (2, -4)$$

$$f(x)=\answer{-1.7}x+\answer{-0.4}$$
\begin{onlineOnly}
\begin{center}
\desmos{oom9xumepk}{950}{650}
\end{center}
\end{onlineOnly}
\end{problem}
\end{problem}

\begin{problem}\label{prob:leastSq3}
Use Octave to find the least squares approximating quadratic function for the following data points.
$$(-2,1),(0,0),(3,2),(4,3)$$
Round your answers to three decimal places.
$$f(x)=\answer{0.194}x^2+\answer{-0.024}x+\answer{0.127}$$
\end{problem}


% In many scientific investigations, data are collected that relate two variables. For example, if $x$ is the
% number of dollars spent on advertising by a manufacturer and $y$ is the value of sales in the region in
% question, the manufacturer could generate data by spending $x_1,\,x_2,\, \dots,\, x_n$ dollars at different times and
% measuring the corresponding sales values $y_1,\, y_2,\,\dots ,\, y_n$.

% Suppose it is known that a linear relationship exists between the variables
% $x$ and $y$ -- in other words, that $y = a+bx$ for some constants $a$ and
% $b$. If the data are plotted, the points $(x_1, y_1),\, (x_2, y_2),\dots ,\, (x_n, y_n)$ may
% appear to lie on a straight line and estimating $a$ and $b$ requires finding
% the “best-fitting” line through these data points. For example, if five data
% points occur as shown in the diagram, line $L_1$ is clearly a better fit than line
% $L_2$. In general, the problem is to find the values of the constants $a$ and $b$
% such that the line $y = a+bx$ best approximates the data in question. Note
% that an exact fit would be obtained if $a$ and $b$ were such that $y_i = a+bx_i$
% were true for each data point $(x_i, y_i)$. But this is too much to expect. So, the choice of $a$ and $b$ should be made in such a
% way that the errors between the observed values $y_i$ and the corresponding fitted values $a+bx_i$ are in some
% sense minimized. 



% Least squares approximation is a way to do this.
% The first thing we must do is explain exactly what we mean by the best fit of a line y = a+bx to an
% observed set of data points (x1, y1), (x2, y2), . . . , (xn, yn). For convenience, write the linear function
% r0+r1x as
% f





% This is from Nicholson's section on QR, and may be helpful:

% In Section~\ref{sec:5_6} we found how to find a best approximation $\vec{z}$ to a solution of a (possibly inconsistent) system $A\vec{x} = \vec{b}$ of linear equations: take $\vec{z}$ to be any solution of the ``normal'' equations $(A^{T}A)\vec{z} = A^{T}\vec{b}$. If $A$ has independent columns this $\vec{z}$ is unique ($A^{T}A$ is invertible by Theorem~\ref{thm:015672}), so it is often desirable to compute $(A^{T}A)^{-1}$. This is particularly useful in least squares approximation (Section~\ref{sec:5_6}). This is simplified if we have a QR-factorization of $A$ (and is one of the main reasons for the importance of Theorem~\ref{th:QR-025133}). For if $A = QR$ is such a factorization, then $Q^{T}Q = I_{n}$ because $Q$ has orthonormal columns (verify), so we obtain
% \begin{equation*}
% A^TA = R^TQ^TQR = R^TR
% \end{equation*}
% Hence computing $(A^{T}A)^{-1}$ amounts to finding $R^{-1}$, and this is a routine matter because $R$ is upper triangular. Thus the difficulty in computing $(A^{T}A)^{-1}$ lies in obtaining the QR-factorization of $A$.

\section*{Text Source}
A portion of this section has been adapted from
W. Keith Nicholson, {\it Linear Algebra with Applications}, Lyryx 2021-A, Open Edition, pp. 308-319.

\end{document}