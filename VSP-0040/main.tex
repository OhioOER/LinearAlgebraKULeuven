\documentclass{ximera}

%%% Begin Laad packages

\makeatletter
\@ifclassloaded{xourse}{%
    \typeout{Start loading preamble.tex (in a XOURSE)}%
    \def\isXourse{true}   % automatically defined; pre 112022 it had to be set 'manually' in a xourse
}{%
    \typeout{Start loading preamble.tex (NOT in a XOURSE)}%
}
\makeatother

\def\isEn\true 

\pgfplotsset{compat=1.16}

\usepackage{currfile}

% 201908/202301: PAS OP: babel en doclicense lijken problemen te veroorzaken in .jax bestand
% (wegens syntax error met toegevoegde \newcommands ...)
\pdfOnly{
    \usepackage[type={CC},modifier={by-nc-sa},version={4.0}]{doclicense}
    %\usepackage[hyperxmp=false,type={CC},modifier={by-nc-sa},version={4.0}]{doclicense}
    %%% \usepackage[dutch]{babel}
}



\usepackage[utf8]{inputenc}
\usepackage{morewrites}   % nav zomercursus (answer...?)
\usepackage{multirow}
\usepackage{multicol}
\usepackage{tikzsymbols}
\usepackage{ifthen}
%\usepackage{animate} BREAKS HTML STRUCTURE USED BY XIMERA
\usepackage{relsize}

\usepackage{eurosym}    % \euro  (€ werkt niet in xake ...?)
\usepackage{fontawesome} % smileys etc

% Nuttig als ook interactieve beamer slides worden voorzien:
\providecommand{\p}{} % default nothing ; potentially usefull for slides: redefine as \pause
%providecommand{\p}{\pause}

    % Layout-parameters voor het onderschrift bij figuren
\usepackage[margin=10pt,font=small,labelfont=bf, labelsep=endash,format=hang]{caption}
%\usepackage{caption} % captionof
%\usepackage{pdflscape}    % landscape environment

% Met "\newcommand\showtodonotes{}" kan je todonotes tonen (in pdf/online)
% 201908: online werkt het niet (goed)
\providecommand\showtodonotes{disable}
\providecommand\todo[1]{\typeout{TODO #1}}
%\usepackage[\showtodonotes]{todonotes}
%\usepackage{todonotes}

%
% Poging tot aanpassen layout
%
\usepackage{tcolorbox}
\tcbuselibrary{theorems}

%%% Einde laad packages

%%% Begin Ximera specifieke zaken

\graphicspath{
	{../../}
	{../}
	{./}
  	{../../pictures/}
   	{../pictures/}
   	{./pictures/}
	{./explog/}    % M05 in groeimodellen       
}

%%% Einde Ximera specifieke zaken

%
% define softer blue/red/green, use KU Leuven base colors for blue (and dark orange for red ?)
%
% todo: rather redefine blue/red/green ...?
%\definecolor{xmblue}{rgb}{0.01, 0.31, 0.59}
%\definecolor{xmred}{rgb}{0.89, 0.02, 0.17}
\definecolor{xmdarkblue}{rgb}{0.122, 0.671, 0.835}   % KU Leuven Blauw
\definecolor{xmblue}{rgb}{0.114, 0.553, 0.69}        % KU Leuven Blauw
\definecolor{xmgreen}{rgb}{0.13, 0.55, 0.13}         % No KULeuven variant for green found ...

\definecolor{xmaccent}{rgb}{0.867, 0.541, 0.18}      % KU Leuven Accent (orange ...)
\definecolor{kuaccent}{rgb}{0.867, 0.541, 0.18}      % KU Leuven Accent (orange ...)

\colorlet{xmred}{xmaccent!50!black}                  % Darker version of KU Leuven Accent

\providecommand{\blue}[1]{{\color{blue}#1}}    
\providecommand{\red}[1]{{\color{red}#1}}

\renewcommand\CancelColor{\color{xmaccent!50!black}}

% werkt in math en text mode om MATH met oranje (of grijze...)  achtergond te tonen (ook \important{\text{blabla}} lijkt te werken)
%\newcommand{\important}[1]{\ensuremath{\colorbox{xmaccent!50!white}{$#1$}}}   % werkt niet in Mathjax
%\newcommand{\important}[1]{\ensuremath{\colorbox{lightgray}{$#1$}}}
\newcommand{\important}[1]{\ensuremath{\colorbox{orange}{$#1$}}}   % TODO: kleur aanpassen voor mathjax; wordt overschreven infra!


% Uitzonderlijk kan met \pdfnl in de PDF een newline worden geforceerd, die online niet nodig/nuttig is omdat daar de regellengte hoe dan ook niet gekend is.
\ifdefined\HCode%
\providecommand{\pdfnl}{}%
\else%
\providecommand{\pdfnl}{%
  \\%
}%
\fi

% Uitzonderlijk kan met \handoutnl in de handout-PDF een newline worden geforceerd, die noch online noch in de PDF-met-antwoorden nuttig is.
\ifdefined\HCode
\providecommand{\handoutnl}{}
\else
\providecommand{\handoutnl}{%
\ifhandout%
  \nl%
\fi%
}
\fi



% \cellcolor IGNORED by tex4ht ?
% \begin{center} seems not to wordk
    % (missing margin-left: auto;   on tabular-inside-center ???)
%\newcommand{\importantcell}[1]{\ensuremath{\cellcolor{lightgray}#1}}  %  in tabular; usablility to be checked
\providecommand{\importantcell}[1]{\ensuremath{#1}}     % no mathjax2 support for colloring array cells

\pdfOnly{
  \renewcommand{\important}[1]{\ensuremath{\colorbox{kuaccent!50!white}{$#1$}}}
  \renewcommand{\importantcell}[1]{\ensuremath{\cellcolor{kuaccent!40!white}#1}}   
}

%%% Tikz styles


\pgfplotsset{compat=1.16}

\usetikzlibrary{trees,positioning,arrows,fit,shapes,math,calc,decorations.markings,through,intersections,patterns,matrix}

\usetikzlibrary{decorations.pathreplacing,backgrounds}    % 5/2023: from experimental


\usetikzlibrary{angles,quotes}

\usepgfplotslibrary{fillbetween} % bepaalde_integraal
\usepgfplotslibrary{polar}    % oa voor poolcoordinaten.tex

\pgfplotsset{ownstyle/.style={axis lines = center, axis equal image, xlabel = $x$, ylabel = $y$, enlargelimits}} 

\pgfplotsset{
	plot/.style={no marks,samples=50}
}

\newcommand{\xmPlotsColor}{
	\pgfplotsset{
		plot1/.style={darkgray,no marks,samples=100},
		plot2/.style={lightgray,no marks,samples=100},
		plotresult/.style={blue,no marks,samples=100},
		plotblue/.style={blue,no marks,samples=100},
		plotred/.style={red,no marks,samples=100},
		plotgreen/.style={green,no marks,samples=100},
		plotpurple/.style={purple,no marks,samples=100}
	}
}
\newcommand{\xmPlotsBlackWhite}{
	\pgfplotsset{
		plot1/.style={black,loosely dashed,no marks,samples=100},
		plot2/.style={black,loosely dotted,no marks,samples=100},
		plotresult/.style={black,no marks,samples=100},
		plotblue/.style={black,no marks,samples=100},
		plotred/.style={black,dotted,no marks,samples=100},
		plotgreen/.style={black,dashed,no marks,samples=100},
		plotpurple/.style={black,dashdotted,no marks,samples=100}
	}
}


\newcommand{\xmPlotsColorAndStyle}{
	\pgfplotsset{
		plot1/.style={darkgray,no marks,samples=100},
		plot2/.style={lightgray,no marks,samples=100},
		plotresult/.style={blue,no marks,samples=100},
		plotblue/.style={xmblue,no marks,samples=100},
		plotred/.style={xmred,dashed,thick,no marks,samples=100},
		plotgreen/.style={xmgreen,dotted,very thick,no marks,samples=100},
		plotpurple/.style={purple,no marks,samples=100}
	}
}


%\iftikzexport
\xmPlotsColorAndStyle
%\else
%\xmPlotsBlackWhite
%\fi
%%%


%
% Om venndiagrammen te arceren ...
%
\makeatletter
\pgfdeclarepatternformonly[\hatchdistance,\hatchthickness]{north east hatch}% name
{\pgfqpoint{-1pt}{-1pt}}% below left
{\pgfqpoint{\hatchdistance}{\hatchdistance}}% above right
{\pgfpoint{\hatchdistance-1pt}{\hatchdistance-1pt}}%
{
	\pgfsetcolor{\tikz@pattern@color}
	\pgfsetlinewidth{\hatchthickness}
	\pgfpathmoveto{\pgfqpoint{0pt}{0pt}}
	\pgfpathlineto{\pgfqpoint{\hatchdistance}{\hatchdistance}}
	\pgfusepath{stroke}
}
\pgfdeclarepatternformonly[\hatchdistance,\hatchthickness]{north west hatch}% name
{\pgfqpoint{-\hatchthickness}{-\hatchthickness}}% below left
{\pgfqpoint{\hatchdistance+\hatchthickness}{\hatchdistance+\hatchthickness}}% above right
{\pgfpoint{\hatchdistance}{\hatchdistance}}%
{
	\pgfsetcolor{\tikz@pattern@color}
	\pgfsetlinewidth{\hatchthickness}
	\pgfpathmoveto{\pgfqpoint{\hatchdistance+\hatchthickness}{-\hatchthickness}}
	\pgfpathlineto{\pgfqpoint{-\hatchthickness}{\hatchdistance+\hatchthickness}}
	\pgfusepath{stroke}
}
%\makeatother

\tikzset{
    hatch distance/.store in=\hatchdistance,
    hatch distance=10pt,
    hatch thickness/.store in=\hatchthickness,
   	hatch thickness=2pt
}

\colorlet{circle edge}{black}
\colorlet{circle area}{blue!20}


\tikzset{
    filled/.style={fill=green!30, draw=circle edge, thick},
    arceerl/.style={pattern=north east hatch, pattern color=blue!50, draw=circle edge},
    arceerr/.style={pattern=north west hatch, pattern color=yellow!50, draw=circle edge},
    outline/.style={draw=circle edge, thick}
}




%%% Updaten commando's
\def\hoofding #1#2#3{\maketitle}     % OBSOLETE ??

% we willen (bijna) altijd \geqslant ipv \geq ...!
\newcommand{\geqnoslant}{\geq}
\renewcommand{\geq}{\geqslant}
\newcommand{\leqnoslant}{\leq}
\renewcommand{\leq}{\leqslant}

% Todo: (201908) waarom komt er (soms) underlined voor emph ...?
\renewcommand{\emph}[1]{\textit{#1}}

% API commando's

\newcommand{\ds}{\displaystyle}
\newcommand{\ts}{\textstyle}  % tegenhanger van \ds   (Ximera zet PER  DEFAULT \ds!)

% uit Zomercursus-macro's: 
\newcommand{\bron}[1]{\begin{scriptsize} \emph{#1} \end{scriptsize}}     % deprecated ...?


%definities nieuwe commando's - afkortingen veel gebruikte symbolen
\newcommand{\R}{\ensuremath{\mathbb{R}}}
\newcommand{\Rnul}{\ensuremath{\mathbb{R}_0}}
\newcommand{\Reen}{\ensuremath{\mathbb{R}\setminus\{1\}}}
\newcommand{\Rnuleen}{\ensuremath{\mathbb{R}\setminus\{0,1\}}}
\newcommand{\Rplus}{\ensuremath{\mathbb{R}^+}}
\newcommand{\Rmin}{\ensuremath{\mathbb{R}^-}}
\newcommand{\Rnulplus}{\ensuremath{\mathbb{R}_0^+}}
\newcommand{\Rnulmin}{\ensuremath{\mathbb{R}_0^-}}
\newcommand{\Rnuleenplus}{\ensuremath{\mathbb{R}^+\setminus\{0,1\}}}
\newcommand{\N}{\ensuremath{\mathbb{N}}}
\newcommand{\Nnul}{\ensuremath{\mathbb{N}_0}}
\newcommand{\Z}{\ensuremath{\mathbb{Z}}}
\newcommand{\Znul}{\ensuremath{\mathbb{Z}_0}}
\newcommand{\Zplus}{\ensuremath{\mathbb{Z}^+}}
\newcommand{\Zmin}{\ensuremath{\mathbb{Z}^-}}
\newcommand{\Znulplus}{\ensuremath{\mathbb{Z}_0^+}}
\newcommand{\Znulmin}{\ensuremath{\mathbb{Z}_0^-}}
\newcommand{\C}{\ensuremath{\mathbb{C}}}
\newcommand{\Cnul}{\ensuremath{\mathbb{C}_0}}
\newcommand{\Cplus}{\ensuremath{\mathbb{C}^+}}
\newcommand{\Cmin}{\ensuremath{\mathbb{C}^-}}
\newcommand{\Cnulplus}{\ensuremath{\mathbb{C}_0^+}}
\newcommand{\Cnulmin}{\ensuremath{\mathbb{C}_0^-}}
\newcommand{\Q}{\ensuremath{\mathbb{Q}}}
\newcommand{\Qnul}{\ensuremath{\mathbb{Q}_0}}
\newcommand{\Qplus}{\ensuremath{\mathbb{Q}^+}}
\newcommand{\Qmin}{\ensuremath{\mathbb{Q}^-}}
\newcommand{\Qnulplus}{\ensuremath{\mathbb{Q}_0^+}}
\newcommand{\Qnulmin}{\ensuremath{\mathbb{Q}_0^-}}

\newcommand{\perdef}{\overset{\mathrm{def}}{=}}
\newcommand{\pernot}{\overset{\mathrm{notatie}}{=}}
\newcommand\perinderdaad{\overset{!}{=}}     % voorlopig gebruikt in limietenrekenregels
\newcommand\perhaps{\overset{?}{=}}          % voorlopig gebruikt in limietenrekenregels

\newcommand{\degree}{^\circ}


\DeclareMathOperator{\dom}{dom}     % domein
\DeclareMathOperator{\codom}{codom} % codomein
\DeclareMathOperator{\bld}{bld}     % beeld
\DeclareMathOperator{\graf}{graf}   % grafiek
\DeclareMathOperator{\rico}{rico}   % richtingcoëfficient
\DeclareMathOperator{\co}{co}       % coordinaat
\DeclareMathOperator{\gr}{gr}       % graad

\newcommand{\func}[5]{\ensuremath{#1: #2 \rightarrow #3: #4 \mapsto #5}} % Easy to write a function


% Operators
\DeclareMathOperator{\bgsin}{bgsin}
\DeclareMathOperator{\bgcos}{bgcos}
\DeclareMathOperator{\bgtan}{bgtan}
\DeclareMathOperator{\bgcot}{bgcot}
\DeclareMathOperator{\bgsinh}{bgsinh}
\DeclareMathOperator{\bgcosh}{bgcosh}
\DeclareMathOperator{\bgtanh}{bgtanh}
\DeclareMathOperator{\bgcoth}{bgcoth}

% Oude \Bgsin etc deprecated: gebruik \bgsin, en herdefinieer dat als je Bgsin wil!
%\DeclareMathOperator{\cosec}{cosec}    % not used? gebruik \csc en herdefinieer

% operatoren voor differentialen: to be verified; 1/2020: inconsequent gebruik bij afgeleiden/integralen
\renewcommand{\d}{\mathrm{d}}
\newcommand{\dx}{\d x}
\newcommand{\dd}[1]{\frac{\mathrm{d}}{\mathrm{d}#1}}
\newcommand{\ddx}{\dd{x}}

% om in voorbeelden/oefeningen de notatie voor afgeleiden te kunnen kiezen
% Usage: \afg{(2\sin(x))}  (en wordt d/dx, of accent, of D )
%\newcommand{\afg}[1]{{#1}'}
\newcommand{\afg}[1]{\left(#1\right)'}
%\renewcommand{\afg}[1]{\frac{\mathrm{d}#1}{\mathrm{d}x}}   % include in relevant exercises ...
%\renewcommand{\afg}[1]{D{#1}}

%
% \xmxxx commands: Extra KU Leuven functionaliteit van, boven of naast Ximera
%   ( Conventie 8/2019: xm+nederlandse omschrijving, maar is niet consequent gevolgd, en misschien ook niet erg handig !)
%
% (Met een minimale ximera.cls en preamble.tex zou een bruikbare .pdf moeten kunnen worden gemaakt van eender welke ximera)
%
% Usage: \xmtitle[Mijn korte abstract]{Mijn titel}{Mijn abstract}
% Eerste command na \begin{document}:
%  -> definieert de \title
%  -> definieert de abstract
%  -> doet \maketitle ( dus: print de hoofding als 'chapter' of 'sectie')
% Optionele parameter geeft eenn kort abstract (die met de globale setting \xmshortabstract{} al dan niet kan worden geprint.
% De optionele korte abstract kan worden gebruikt voor pseudo-grappige abtsarts, dus dus globaal al dan niet kunnen worden gebuikt...
% Globale settings:
%  de (optionele) 'korte abstract' wordt enkele getoond als \xmshortabstract is gezet
\providecommand\xmshortabstract{} % default: print (only!) short abstract if present
\newcommand{\xmtitle}[3][]{
	\title{#2}
	\begin{abstract}
		\ifdefined\xmshortabstract
		\ifstrempty{#1}{%
			#3
		}{%
			#1
		}%
		\else
		#3
		\fi
	\end{abstract}
	\maketitle
}

% 
% Kleine grapjes: moeten zonder verder gevolg kunnen worden verwijderd
%
%\newcommand{\xmopje}[1]{{\small#1{\reversemarginpar\marginpar{\Smiley}}}}   % probleem in floats!!
\newtoggle{showxmopje}
\toggletrue{showxmopje}

\newcommand{\xmopje}[1]{%
   \iftoggle{showxmopje}{#1}{}%
}


% -> geef een abstracte-formule-met-rechts-een-concreet-voorbeeld
% VB:  \formulevb{a^2+b^2=c^2}{3^2+4^2=5^2}
%
\ifdefined\HCode
\NewEnviron{xmdiv}[1]{\HCode{\Hnewline<div class="#1">\Hnewline}\BODY{\HCode{\Hnewline</div>\Hnewline}}}
\else
\NewEnviron{xmdiv}[1]{\BODY}
\fi

\providecommand{\formulevb}[2]{
	{\centering

    \begin{xmdiv}{xmformulevb}    % zie css voor online layout !!!
	\begin{tabular}{lcl}
		\important{#1}
		&  &
		Vb: $#2$
		\end{tabular}
	\end{xmdiv}

	}
}

\ifdefined\HCode
\providecommand{\vb}[1]{%
    \HCode{\Hnewline<span class="xmvb">}#1\HCode{</span>\Hnewline}%
}
\else
\providecommand{\vb}[1]{
    \colorbox{blue!10}{#1}
}
\fi

\ifdefined\HCode
\providecommand{\xmcolorbox}[2]{
	\HCode{\Hnewline<div class="xmcolorbox">\Hnewline}#2\HCode{\Hnewline</div>\Hnewline}
}
\else
\providecommand{\xmcolorbox}[2]{
  \cellcolor{#1}#2
}
\fi


\ifdefined\HCode
\providecommand{\xmopmerking}[1]{
 \HCode{\Hnewline<div class="xmopmerking">\Hnewline}#1\HCode{\Hnewline</div>\Hnewline}
}
\else
\providecommand{\xmopmerking}[1]{
	{\footnotesize #1}
}
\fi
% \providecommand{\voorbeeld}[1]{
% 	\colorbox{blue!10}{$#1$}
% }



% Hernoem Proof naar Bewijs, nodig voor HTML versie
\renewcommand*{\proofname}{Bewijs}

% Om opgave van oefening (wordt niet geprint bij oplossingenblad)
% (to be tested test)
\NewEnviron{statement}{\BODY}

% Environment 'oplossing' en 'uitkomst'
% voor resp. volledige 'uitwerking' dan wel 'enkel eindresultaat'
% geimplementeerd via feedback, omdat er in de ximera-server adhoc feedback-code is toegevoegd
%% Niet tonen indien handout
%% Te gebruiken om volledige oplossingen/uitwerkingen van oefeningen te tonen
%% \begin{oplossing}        De optelling is commutatief \end{oplossing}  : verschijnt online enkel 'op vraag'
%% \begin{oplossing}[toon]  De optelling is commutatief \end{oplossing}  : verschijnt steeds onmiddellijk online (bv te gebruiken bij voorbeelden) 

\ifhandout%
    \NewEnviron{oplossing}[1][onzichtbaar]%
    {%
    \ifthenelse{\equal{\detokenize{#1}}{\detokenize{toon}}}
    {
    \def\PH@Command{#1}% Use PH@Command to hold the content and be a target for "\expandafter" to expand once.

    \begin{trivlist}% Begin the trivlist to use formating of the "Feedback" label.
    \item[\hskip \labelsep\small\slshape\bfseries Oplossing% Format the "Feedback" label. Don't forget the space.
    %(\texttt{\detokenize\expandafter{\PH@Command}}):% Format (and detokenize) the condition for feedback to trigger
    \hspace{2ex}]\small%\slshape% Insert some space before the actual feedback given.
    \BODY
    \end{trivlist}
    }
    {  % \begin{feedback}[solution]   \BODY     \end{feedback}  }
    }
    }    
\else
% ONLY for HTML; xmoplossing is styled with css, and is not, and need not be a LaTeX environment
% THUS: it does NOT use feedback anymore ...
%    \NewEnviron{oplossing}{\begin{expandable}{xmoplossing}{\nlen{Toon uitwerking}{Show solution}}{\BODY}\end{expandable}}
    \newenvironment{oplossing}[1][onzichtbaar]
   {%
       \begin{expandable}{xmoplossing}{}
   }
   {%
   	   \end{expandable}
   } 
%     \newenvironment{oplossing}[1][onzichtbaar]
%    {%
%        \begin{feedback}[solution]   	
%    }
%    {%
%    	   \end{feedback}
%    } 
\fi

\ifhandout%
    \NewEnviron{uitkomst}[1][onzichtbaar]%
    {%
    \ifthenelse{\equal{\detokenize{#1}}{\detokenize{toon}}}
    {
    \def\PH@Command{#1}% Use PH@Command to hold the content and be a target for "\expandafter" to expand once.

    \begin{trivlist}% Begin the trivlist to use formating of the "Feedback" label.
    \item[\hskip \labelsep\small\slshape\bfseries Uitkomst:% Format the "Feedback" label. Don't forget the space.
    %(\texttt{\detokenize\expandafter{\PH@Command}}):% Format (and detokenize) the condition for feedback to trigger
    \hspace{2ex}]\small%\slshape% Insert some space before the actual feedback given.
    \BODY
    \end{trivlist}
    }
    {  % \begin{feedback}[solution]   \BODY     \end{feedback}  }
    }
    }    
\else
\ifdefined\HCode
   \newenvironment{uitkomst}[1][onzichtbaar]
    {%
        \begin{expandable}{xmuitkomst}{}%
    }
    {%
    	\end{expandable}%
    } 
\else
  % Do NOT print 'uitkomst' in non-handout
  %  (presumably, there is also an 'oplossing' ??)
  \newenvironment{uitkomst}[1][onzichtbaar]{}{}
\fi
\fi

%
% Uitweidingen zijn extra's die niet redelijkerwijze tot de leerstof behoren
% Uitbreidingen zijn extra's die wel redelijkerwijze tot de leerstof van bv meer geavanceerde versies kunnen behoren (B-programma/Wiskundestudenten/...?)
% Nog niet voorzien: design voor verschillende versies (A/B programma, BIO, voorkennis/ ...)
% Voor 'uitweidingen' is er een environment die online per default is ingeklapt, en in pdf al dan niet kan worden geincluded  (via \xmnouitweiding) 
%
% in een xourse, per default GEEN uitweidingen, tenzij \xmuitweiding expliciet ergens is gezet ...
\ifdefined\isXourse
   \ifdefined\xmuitweiding
   \else
       \def\xmnouitweiding{true}
   \fi
\fi

\ifdefined\xmnouitweiding
\newcounter{xmuitweiding}  % anders error undefined ...  
\excludecomment{xmuitweiding}
\else
\newtheoremstyle{dotless}{}{}{}{}{}{}{ }{}
\theoremstyle{dotless}
\newtheorem*{xmuitweidingnofrills}{}   % nofrills = no accordion; gebruikt dus de dotless theoremstyle!

\newcounter{xmuitweiding}
\newenvironment{xmuitweiding}[1][ ]%
{% 
	\refstepcounter{xmuitweiding}%
    \begin{expandable}{xmuitweiding}{\nlentext{Uitweiding \arabic{xmuitweiding}: #1}{Digression \arabic{xmuitweiding}: #1}}%
	\begin{xmuitweidingnofrills}%
}
{%
    \end{xmuitweidingnofrills}%
    \end{expandable}%
}   
% \newenvironment{xmuitweiding}[1][ ]%
% {% 
% 	\refstepcounter{xmuitweiding}
% 	\begin{accordion}\begin{accordion-item}[Uitweiding \arabic{xmuitweiding}: #1]%
% 			\begin{xmuitweidingnofrills}%
% 			}
% 			{\end{xmuitweidingnofrills}\end{accordion-item}\end{accordion}}   
\fi


\newenvironment{xmexpandable}[1][]{
	\begin{accordion}\begin{accordion-item}[#1]%
		}{\end{accordion-item}\end{accordion}}


% Command that gives a selection box online, but just prints the right answer in pdf
\newcommand{\xmonlineChoice}[1]{\pdfOnly{\wordchoicegiventrue}\wordChoice{#1}\pdfOnly{\wordchoicegivenfalse}}   % deprecated, gebruik onlineChoice ...
\newcommand{\onlineChoice}[1]{\pdfOnly{\wordchoicegiventrue}\wordChoice{#1}\pdfOnly{\wordchoicegivenfalse}}

\newcommand{\TJa}{\nlentext{ Ja }{ Yes }}
\newcommand{\TNee}{\nlentext{ Nee }{ No }}
\newcommand{\TJuist}{\nlentext{ Juist }{ True }}
\newcommand{\TFout}{\nlentext{ Fout }{ False }}

\newcommand{\choiceTrue }{{\renewcommand{\choiceminimumhorizontalsize}{4em}\wordChoice{\choice[correct]{\TJuist}\choice{\TFout}}}}
\newcommand{\choiceFalse}{{\renewcommand{\choiceminimumhorizontalsize}{4em}\wordChoice{\choice{\TJuist}\choice[correct]{\TFout}}}}

\newcommand{\choiceYes}{{\renewcommand{\choiceminimumhorizontalsize}{3em}\wordChoice{\choice[correct]{\TJa}\choice{\TNee}}}}
\newcommand{\choiceNo }{{\renewcommand{\choiceminimumhorizontalsize}{3em}\wordChoice{\choice{\TJa}\choice[correct]{\TNee}}}}

% Optional nicer formatting for wordChoice in PDF

\let\inlinechoiceorig\inlinechoice

%\makeatletter
%\providecommand{\choiceminimumverticalsize}{\vphantom{$\frac{\sqrt{2}}{2}$}}   % minimum height of boxes (cfr infra)
\providecommand{\choiceminimumverticalsize}{\vphantom{$\tfrac{2}{2}$}}   % minimum height of boxes (cfr infra)
\providecommand{\choiceminimumhorizontalsize}{1em}   % minimum width of boxes (cfr infra)

\newcommand{\inlinechoicesquares}[2][]{%
		\setkeys{choice}{#1}%
		\ifthenelse{\boolean{\choice@correct}}%
		{%
            \ifhandout%
               \fbox{\choiceminimumverticalsize #2}\allowbreak\ignorespaces%
            \else%
               \fcolorbox{blue}{blue!20}{\choiceminimumverticalsize #2}\allowbreak\ignorespaces\setkeys{choice}{correct=false}\ignorespaces%
            \fi%
		}%
		{% else
			\fbox{\choiceminimumverticalsize #2}\allowbreak\ignorespaces%  HACK: wat kleiner, zodat fits on line ... 	
		}%
}

\newcommand{\inlinechoicesquareX}[2][]{%
		\setkeys{choice}{#1}%
		\ifthenelse{\boolean{\choice@correct}}%
		{%
            \ifhandout%
               \framebox[\ifdim\choiceminimumhorizontalsize<\width\width\else\choiceminimumhorizontalsize\fi]{\choiceminimumverticalsize\ #2\ }\allowbreak\ignorespaces\setkeys{choice}{correct=false}\ignorespaces%
            \else%
               \fcolorbox{blue}{blue!20}{\makebox[\ifdim\choiceminimumhorizontalsize<\width\width\else\choiceminimumhorizontalsize\fi]{\choiceminimumverticalsize #2}}\allowbreak\ignorespaces\setkeys{choice}{correct=false}\ignorespaces%
            \fi%
		}%
		{% else
        \ifhandout%
			\framebox[\ifdim\choiceminimumhorizontalsize<\width\width\else\choiceminimumhorizontalsize\fi]{\choiceminimumverticalsize\ #2\ }\allowbreak\ignorespaces%  HACK: wat kleiner, zodat fits on line ... 	
        \fi
		}%
}


\newcommand{\inlinechoicepuntjes}[2][]{%
		\setkeys{choice}{#1}%
		\ifthenelse{\boolean{\choice@correct}}%
		{%
            \ifhandout%
               \dots\ldots\ignorespaces\setkeys{choice}{correct=false}\ignorespaces
            \else%
               \fcolorbox{blue}{blue!20}{\choiceminimumverticalsize #2}\allowbreak\ignorespaces\setkeys{choice}{correct=false}\ignorespaces%
            \fi%
		}%
		{% else
			%\fbox{\choiceminimumverticalsize #2}\allowbreak\ignorespaces%  HACK: wat kleiner, zodat fits on line ... 	
		}%
}

% print niets, maar definieer globale variable \myanswer
%  (gebruikt om oplossingsbladen te printen) 
\newcommand{\inlinechoicedefanswer}[2][]{%
		\setkeys{choice}{#1}%
		\ifthenelse{\boolean{\choice@correct}}%
		{%
               \gdef\myanswer{#2}\setkeys{choice}{correct=false}

		}%
		{% else
			%\fbox{\choiceminimumverticalsize #2}\allowbreak\ignorespaces%  HACK: wat kleiner, zodat fits on line ... 	
		}%
}



%\makeatother

\newcommand{\setchoicedefanswer}{
\ifdefined\HCode
\else
%    \renewenvironment{multipleChoice@}[1][]{}{} % remove trailing ')'
    \let\inlinechoice\inlinechoicedefanswer
\fi
}

\newcommand{\setchoicepuntjes}{
\ifdefined\HCode
\else
    \renewenvironment{multipleChoice@}[1][]{}{} % remove trailing ')'
    \let\inlinechoice\inlinechoicepuntjes
\fi
}
\newcommand{\setchoicesquares}{
\ifdefined\HCode
\else
    \renewenvironment{multipleChoice@}[1][]{}{} % remove trailing ')'
    \let\inlinechoice\inlinechoicesquares
\fi
}
%
\newcommand{\setchoicesquareX}{
\ifdefined\HCode
\else
    \renewenvironment{multipleChoice@}[1][]{}{} % remove trailing ')'
    \let\inlinechoice\inlinechoicesquareX
\fi
}
%
\newcommand{\setchoicelist}{
\ifdefined\HCode
\else
    \renewenvironment{multipleChoice@}[1][]{}{)}% re-add trailing ')'
    \let\inlinechoice\inlinechoiceorig
\fi
}

\setchoicesquareX  % by default list-of-squares with onlineChoice in PDF

% Omdat multicols niet werkt in html: enkel in pdf  (in html zijn langere pagina's misschien ook minder storend)
\newenvironment{xmmulticols}[1][2]{
 \pdfOnly{\begin{multicols}{#1}}%
}{ \pdfOnly{\end{multicols}}}

%
% Te gebruiken in plaats van \section\subsection
%  (in een printstyle kan dan het level worden aangepast
%    naargelang \chapter vs \section style )
% 3/2021: DO NOT USE \xmsubsection !
\newcommand\xmsection\subsection
\newcommand\xmsubsection\subsubsection

% Aanpassen printversie
%  (hier gedefinieerd, zodat ze in xourse kunnen worden gezet/overschreven)
\providebool{parttoc}
\providebool{printpartfrontpage}
\providebool{printactivitytitle}
\providebool{printactivityqrcode}
\providebool{printactivityurl}
\providebool{printcontinuouspagenumbers}
\providebool{numberactivitiesbysubpart}
\providebool{addtitlenumber}
\providebool{addsectiontitlenumber}
\addtitlenumbertrue
\addsectiontitlenumbertrue

% The following three commands are hardcoded in xake, you can't create other commands like these, without adding them to xake as well
%  ( gebruikt in xourses om juiste soort titelpagina te krijgen voor verschillende ximera's )
\newcommand{\activitychapter}[2][]{
    {    
    \ifstrequal{#1}{notnumbered}{
        \addtitlenumberfalse
    }{}
    \typeout{ACTIVITYCHAPTER #2}   % logging
	\chapterstyle
	\activity{#2}
    }
}
\newcommand{\activitysection}[2][]{
    {
    \ifstrequal{#1}{notnumbered}{
        \addsectiontitlenumberfalse
    }{}
	\typeout{ACTIVITYSECTION #2}   % logging
	\sectionstyle
	\activity{#2}
    }
}
% Practices worden als activity getoond om de grote blokken te krijgen online
\newcommand{\practicesection}[2][]{
    {
    \ifstrequal{#1}{notnumbered}{
        \addsectiontitlenumberfalse
    }{}
    \typeout{PRACTICESECTION #2}   % logging
	\sectionstyle
	\activity{#2}
    }
}
\newcommand{\activitychapterlink}[3][]{
    {
    \ifstrequal{#1}{notnumbered}{
        \addtitlenumberfalse
    }{}
    \typeout{ACTIVITYCHAPTERLINK #3}   % logging
	\chapterstyle
	\activitylink[#1]{#2}{#3}
    }
}

\newcommand{\activitysectionlink}[3][]{
    {
    \ifstrequal{#1}{notnumbered}{
        \addsectiontitlenumberfalse
    }{}
    \typeout{ACTIVITYSECTIONLINK #3}   % logging
	\sectionstyle
	\activitylink[#1]{#2}{#3}
    }
}


% Commando om de printstyle toe te voegen in ximera's. Zorgt ervoor dat er geen problemen zijn als je de xourses compileert
% hack om onhandige relative paden in TeX te omzeilen
% should work both in xourse and ximera (pre-112022 only in ximera; thus obsoletes adhoc setup in xourses)
% loads global.sty if present (cfr global.css for online settings!)
% use global.sty to overwrite settings in printstyle.sty ...
\newcommand{\addPrintStyle}[1]{
\iftikzexport\else   % only in PDF
  \makeatletter
  \ifx\@onlypreamble\@notprerr\else   % ONLY if in tex-preamble   (and e.g. not when included from xourse)
    \typeout{Loading printstyle}   % logging
    \usepackage{#1/printstyle} % mag enkel geinclude worden als je die apart compileert
    \IfFileExists{#1/global.sty}{
        \typeout{Loading printstyle-folder #1/global.sty}   % logging
        \usepackage{#1/global}
        }{
        \typeout{Info: No extra #1/global.sty}   % logging
    }   % load global.sty if present
    \IfFileExists{global.sty}{
        \typeout{Loading local-folder global.sty (or TEXINPUTPATH..)}   % logging
        \usepackage{global}
    }{
        \typeout{Info: No folder/global.sty}   % logging
    }   % load global.sty if present
    \IfFileExists{\currfilebase.sty}
    {
        \typeout{Loading \currfilebase.sty}
        \input{\currfilebase.sty}
    }{
        \typeout{Info: No local \currfilebase.sty}
    }
    \fi
  \makeatother
\fi
}

%
%  
% references: Ximera heeft adhoc logica	 om online labels te doen werken over verschillende files heen
% met \hyperref kan de getoonde tekst toch worden opgegeven, in plaats van af te hangen van de label-text
\ifdefined\HCode
% Link to standard \labels, but give your own description
% Usage:  Volg \hyperref[my_very_verbose_label]{deze link} voor wat tijdverlies
%   (01/2020: Ximera-server aangepast om bij class reference-keeptext de link-text NIET te vervangen door de label-text !!!) 
\renewcommand{\hyperref}[2][]{\HCode{<a class="reference reference-keeptext" href="\##1">}#2\HCode{</a>}}
%
%  Link to specific targets  (not tested ?)
\renewcommand{\hypertarget}[1]{\HCode{<a class="ximera-label" id="#1"></a>}}
\renewcommand{\hyperlink}[2]{\HCode{<a class="reference reference-keeptext" href="\##1">}#2\HCode{</a>}}
\fi

% Mmm, quid English ... (for keyword #1 !) ?
\newcommand{\wikilink}[2]{
    \hyperlink{https://nl.wikipedia.org/wiki/#1}{#2}
    \pdfOnly{\footnote{See \url{https://nl.wikipedia.org/wiki/#1}}
    }
}

\renewcommand{\figurename}{Figuur}
\renewcommand{\tablename}{Tabel}

%
% Gedoe om verschillende versies van xourse/ximera te maken afhankelijk van settings
%
% default: versie met antwoorden
% handout: versie voor de studenten, zonder antwoorden/oplossingen
% full: met alles erop en eraan, dus geschikt voor auteurs en/of lesgevers  (bevat in de pdf ook de 'online-only' stukken!)
%
%
% verder kunnen ook opties/variabele worden gezet voor hints/auteurs/uitweidingen/ etc
%
% 'Full' versie
\newtoggle{showonline}
\ifdefined\HCode   % zet default showOnline
    \toggletrue{showonline} 
\else
    \togglefalse{showonline}
\fi

% Full versie   % deprecated: see infra
\newcommand{\printFull}{
    \hintstrue
    \handoutfalse
    \toggletrue{showonline} 
}

\ifdefined\shouldPrintFull   % deprecated: see infra
    \printFull
\fi



% Overschrijf onlineOnly  (zoals gedefinieerd in ximera.cls)
\ifhandout   % in handout: gebruik de oorspronkelijke ximera.cls implementatie  (is dit wel nodig/nuttig?)
\else
    \iftoggle{showonline}{%
        \ifdefined\HCode
          \RenewEnviron{onlineOnly}{\bgroup\BODY\egroup}   % showOnline, en we zijn  online, dus toon de tekst
        \else
          \RenewEnviron{onlineOnly}{\bgroup\color{red!50!black}\BODY\egroup}   % showOnline, maar we zijn toch niet online: kleur de tekst rood 
        \fi
    }{%
      \RenewEnviron{onlineOnly}{}  % geen showOnline
    }
\fi

% hack om na hoofding van definition/proposition/... als dan niet op een nieuwe lijn te starten
% soms is dat goed en mooi, en soms niet; en in HTML is het nu (2/2020) anders dan in pdf
% vandaar suggestie om 
%     \begin{definition}[Nieuw concept] \nl
% te gebruiken als je zeker een newline wil na de hoofdig en titel
% (in het bijzonder itemize zonder \nl is 'lelijk' ...)
\ifdefined\HCode
\newcommand{\nl}{}
\else
\newcommand{\nl}{\ \par} % newline (achter heading van definition etc.)
\fi


% \nl enkel in handoutmode (ihb voor \wordChoice, die dan typisch veeeel langer wordt)
\ifdefined\HCode
\providecommand{\handoutnl}{}
\else
\providecommand{\handoutnl}{%
\ifhandout%
  \nl%
\fi%
}
\fi

% Could potentially replace \pdfOnline/\begin{onlineOnly} : 
% Usage= \ifonline{Hallo surfer}{Hallo PDFlezer}
\providecommand{\ifonline}[2]%
{
\begin{onlineOnly}#1\end{onlineOnly}%
\pdfOnly{#2}
}%


%
% Maak optionele 'basic' en 'extended' versies van een activity
%  met environment basicOnly, basicSkip en extendedOnly
%
%  (
%   Dit werkt ENKEL in de PDF; de online versies tonen (minstens voorklopig) steeds 
%   het default geval met printbasicversion en printextendversion beide FALSE
%  )
%
\providebool{printbasicversion}
\providebool{printextendedversion}   % not properly implemented
\providebool{printfullversion}       % presumably print everything (debug/auteur)
%
% only set these in xourses, and BEFORE loading this preamble
%
%\newif\ifshowbasic     \showbasictrue        % use this line in xourse to show 'basic' sections
%\newif\ifshowextended  \showextendedtrue     % use this line in xourse to show 'extended' sections
%
%
%\ifbool{showbasic}
%      { \NewEnviron{basicOnly}{\BODY} }    % if yes: just print contents
%      { \NewEnviron{basicOnly}{}      }    % if no:  completely ignore contents
%
%\ifbool{showbasic}
%      { \NewEnviron{basicSkip}{}      }
%      { \NewEnviron{basicSkip}{\BODY} }
%

\ifbool{printextendedversion}
      { \NewEnviron{extendedOnly}{\BODY} }
      { \NewEnviron{extendedOnly}{}      }
      


\ifdefined\HCode    % in html: always print
      {\newenvironment*{basicOnly}{}{}}    % if yes: just print contents
      {\newenvironment*{basicSkip}{}{}}    % if yes: just print contents
\else

\ifbool{printbasicversion}
      {\newenvironment*{basicOnly}{}{}}    % if yes: just print contents
      {\NewEnviron{basicOnly}{}      }    % if no:  completely ignore contents

\ifbool{printbasicversion}
      {\NewEnviron{basicSkip}{}      }
      {\newenvironment*{basicSkip}{}{}}

\fi

\usepackage{float}
\usepackage[rightbars,color]{changebar}

% Full versie
\ifbool{printfullversion}{
    \hintstrue
    \handoutfalse
    \toggletrue{showonline}
    \printbasicversionfalse
    \cbcolor{red}
    \renewenvironment*{basicOnly}{\cbstart}{\cbend}
    \renewenvironment*{basicSkip}{\cbstart}{\cbend}
    \def\xmtoonprintopties{FULL}   % will be printed in footer
}
{}
      
%
% Evalueer \ifhints IN de environment
%  
%
%\RenewEnviron{hint}
%{
%\ifhandout
%\ifhints\else\setbox0\vbox\fi%everything in een emty box
%\bgroup 
%\stepcounter{hintLevel}
%\BODY
%\egroup\ignorespacesafterend
%\addtocounter{hintLevel}{-1}
%\else
%\ifhints
%\begin{trivlist}\item[\hskip \labelsep\small\slshape\bfseries Hint:\hspace{2ex}]
%\small\slshape
%\stepcounter{hintLevel}
%\BODY
%\end{trivlist}
%\addtocounter{hintLevel}{-1}
%\fi
%\fi
%}

% Onafhankelijk van \ifhandout ...? TO BE VERIFIED
\RenewEnviron{hint}
{
\ifhints
\begin{trivlist}\item[\hskip \labelsep\small\bfseries Hint:\hspace{2ex}]
\small%\slshape
\stepcounter{hintLevel}
\BODY
\end{trivlist}
\addtocounter{hintLevel}{-1}
\else
\iftikzexport   % anders worden de tikz tekeningen in hints niet gegenereerd ?
\setbox0\vbox\bgroup
\stepcounter{hintLevel}
\BODY
\egroup\ignorespacesafterend
\addtocounter{hintLevel}{-1}
\fi % ifhandout
\fi %ifhints
}

%
% \tab sets typewriter-tabs (e.g. to format questions)
% (Has no effect in HTML :-( ))
%
\usepackage{tabto}
\ifdefined\HCode
  \renewcommand{\tab}{\quad}    % otherwise dummy .png's are generated ...?
\fi


% Also redefined in  preamble to get correct styling 
% for tikz images for (\tikzexport)
%

\theoremstyle{definition} % Bold titels
\makeatletter
\let\proposition\relax
\let\c@proposition\relax
\let\endproposition\relax
\makeatother
\newtheorem{proposition}{Eigenschap}


%\instructornotesfalse

% logic with \ifhandoutin ximera.cls unclear;so overwrite ...
\makeatletter
\@ifundefined{ifinstructornotes}{%
  \newif\ifinstructornotes
  \instructornotesfalse
  \newenvironment{instructorNotes}{}{}
}{}
\makeatother
\ifinstructornotes
\else
\renewenvironment{instructorNotes}%
{%
    \setbox0\vbox\bgroup
}
{%
    \egroup
}
\fi

% \RedeclareMathOperator
% from https://tex.stackexchange.com/questions/175251/how-to-redefine-a-command-using-declaremathoperator
\makeatletter
\newcommand\RedeclareMathOperator{%
    \@ifstar{\def\rmo@s{m}\rmo@redeclare}{\def\rmo@s{o}\rmo@redeclare}%
}
% this is taken from \renew@command
\newcommand\rmo@redeclare[2]{%
    \begingroup \escapechar\m@ne\xdef\@gtempa{{\string#1}}\endgroup
    \expandafter\@ifundefined\@gtempa
    {\@latex@error{\noexpand#1undefined}\@ehc}%
    \relax
    \expandafter\rmo@declmathop\rmo@s{#1}{#2}}
% This is just \@declmathop without \@ifdefinable
\newcommand\rmo@declmathop[3]{%
    \DeclareRobustCommand{#2}{\qopname\newmcodes@#1{#3}}%
}
\@onlypreamble\RedeclareMathOperator
\makeatother


%
% Engelse vertaling, vooral in mathmode
%
% 1. Algemene procedure
%
\ifdefined\isEn
 \newcommand{\nlen}[2]{#2}
 \newcommand{\nlentext}[2]{\text{#2}}
 \newcommand{\nlentextbf}[2]{\textbf{#2}}
\else
 \newcommand{\nlen}[2]{#1}
 \newcommand{\nlentext}[2]{\text{#1}}
 \newcommand{\nlentextbf}[2]{\textbf{#1}}
\fi

%
% 2. Lijst van erg veel gebruikte uitdrukkingen
%

% Ja/Nee/Fout/Juits etc
%\newcommand{\TJa}{\nlentext{ Ja }{ and }}
%\newcommand{\TNee}{\nlentext{ Nee }{ No }}
%\newcommand{\TJuist}{\nlentext{ Juist }{ Correct }
%\newcommand{\TFout}{\nlentext{ Fout }{ Wrong }
\newcommand{\TWaar}{\nlentext{ Waar }{ True }}
\newcommand{\TOnwaar}{\nlentext{ Vals }{ False }}
% Korte bindwoorden en, of, dus, ...
\newcommand{\Ten}{\nlentext{ en }{ and }}
\newcommand{\Tof}{\nlentext{ of }{ or }}
\newcommand{\Tdus}{\nlentext{ dus }{ so }}
\newcommand{\Tendus}{\nlentext{ en dus }{ and thus }}
\newcommand{\Tvooralle}{\nlentext{ voor alle }{ for all }}
\newcommand{\Took}{\nlentext{ ook }{ also }}
\newcommand{\Tals}{\nlentext{ als }{ when }} %of if?
\newcommand{\Twant}{\nlentext{ want }{ as }}
\newcommand{\Tmaal}{\nlentext{ maal }{ times }}
\newcommand{\Toptellen}{\nlentext{ optellen }{ add }}
\newcommand{\Tde}{\nlentext{ de }{ the }}
\newcommand{\Thet}{\nlentext{ het }{ the }}
\newcommand{\Tis}{\nlentext{ is }{ is }} %zodat is in text staat in mathmode (geen italics)
\newcommand{\Tmet}{\nlentext{ met }{ where }} % in situaties e.g met p < n --> where p < n
\newcommand{\Tnooit}{\nlentext{ nooit }{ never }}
\newcommand{\Tmaar}{\nlentext{ maar }{ but }}
\newcommand{\Tniet}{\nlentext{ niet }{ not }}
\newcommand{\Tuit}{\nlentext{ uit }{ from }}
\newcommand{\Ttov}{\nlentext{ t.o.v. }{ w.r.t. }}
\newcommand{\Tzodat}{\nlentext{ zodat }{ such that }}
\newcommand{\Tdeth}{\nlentext{de }{th }}
\newcommand{\Tomdat}{\nlentext{omdat }{because }} 


%
% Overschrijf addhoc commando's
%
\ifdefined\isEn
\renewcommand{\pernot}{\overset{\mathrm{notation}}{=}}
\RedeclareMathOperator{\bld}{im}     % beeld
\RedeclareMathOperator{\graf}{graph}   % grafiek
\RedeclareMathOperator{\rico}{slope}   % richtingcoëfficient
\RedeclareMathOperator{\co}{co}       % coordinaat
\RedeclareMathOperator{\gr}{deg}       % graad

% Operators
\RedeclareMathOperator{\bgsin}{arcsin}
\RedeclareMathOperator{\bgcos}{arccos}
\RedeclareMathOperator{\bgtan}{arctan}
\RedeclareMathOperator{\bgcot}{arccot}
\RedeclareMathOperator{\bgsinh}{arcsinh}
\RedeclareMathOperator{\bgcosh}{arccosh}
\RedeclareMathOperator{\bgtanh}{arctanh}
\RedeclareMathOperator{\bgcoth}{arccoth}

\fi


% HACK: use 'oplossing' for 'explanation' ...
\let\explanation\relax
\let\endexplanation\relax
% \newenvironment{explanation}{\begin{oplossing}}{\end{oplossing}}
\newcounter{explanation}

\ifhandout%
    \NewEnviron{explanation}[1][toon]%
    {%
    \RenewEnviron{verbatim}{ \red{VERBATIM CONTENT MISSING IN THIS PDF}} %% \expandafter\verb|\BODY|}

    \ifthenelse{\equal{\detokenize{#1}}{\detokenize{toon}}}
    {
    \def\PH@Command{#1}% Use PH@Command to hold the content and be a target for "\expandafter" to expand once.

    \begin{trivlist}% Begin the trivlist to use formating of the "Feedback" label.
    \item[\hskip \labelsep\small\slshape\bfseries Explanation:% Format the "Feedback" label. Don't forget the space.
    %(\texttt{\detokenize\expandafter{\PH@Command}}):% Format (and detokenize) the condition for feedback to trigger
    \hspace{2ex}]\small%\slshape% Insert some space before the actual feedback given.
    \BODY
    \end{trivlist}
    }
    {  % \begin{feedback}[solution]   \BODY     \end{feedback}  }
    }
    }    
\else
% ONLY for HTML; xmoplossing is styled with css, and is not, and need not be a LaTeX environment
% THUS: it does NOT use feedback anymore ...
%    \NewEnviron{oplossing}{\begin{expandable}{xmoplossing}{\nlen{Toon uitwerking}{Show solution}}{\BODY}\end{expandable}}
    \newenvironment{explanation}[1][toon]
   {%
       \begin{expandable}{xmoplossing}{}
   }
   {%
   	   \end{expandable}
   } 
\fi

\title{Subspaces of $\RR^n$ Associated with Matrices} \license{CC BY-NC-SA 4.0}



\begin{document}
\begin{abstract}

\end{abstract}
\maketitle
\section*{Subspaces of $\RR^n$ Associated with Matrices}
\subsection*{Row Space of a Matrix}
Recall that in \href{https://ximera.osu.edu/oerlinalg/LinearAlgebra/SYS-0030/main}{Gaussian Elimination and Rank}, we claimed that every row-echelon form of a given matrix has the same number of nonzero rows.  This result suggests that there are certain characteristics associated with the rows of a matrix that are not affected by elementary row operations.  We are now in the position to examine this question and to supply the proof we omitted earlier.
\begin{definition}\label{def:rowspace} Let $A$ be an $m\times n$ matrix.  The \dfn{row space} of $A$, denoted by $\mbox{row}(A)$, is the subspace of $\RR^n$ spanned by the rows of $A$.
\end{definition}

\begin{exploration}\label{init:rowspace}
Consider the matrix
$$A=\begin{bmatrix}-2&2&1\\4&-2&1\end{bmatrix}$$
Let $\vec{r}_1$ and $\vec{r}_2$ be the rows of $A$: 
$$\vec{r}_1=\begin{bmatrix}-2&2&1\end{bmatrix},\quad \vec{r}_2=\begin{bmatrix}4&-2&1\end{bmatrix}$$

Then 
$\mbox{row}(A)=\mbox{span}(\vec{r}_1, \vec{r}_2)$
is a plane through the origin containing $\vec{r}_1$ and  $\vec{r}_2$.  

\begin{center}
\tdplotsetmaincoords{70}{130}
\begin{tikzpicture}[rotate around x=0]
	\draw[->](-2,0,0)--(3,0,0) node[below left]{$y$};
    \draw[->](0,-1,0)--(0,3,0) node[below left]{$z$};
    \draw[->](0,0,-2)--(0,0,5) node[below left]{$x$};
    \filldraw[blue, opacity=0.3] (0,0,0)--(2,1,-2)--(0,2,2)--(-2,1,4)--cycle;
    
    \draw[->, line width=2pt,blue, -stealth](0,0,0)--(2,1,-2)node[below right]{$\vec{r}_1$};
      
    \draw[->, line width=2pt,blue, -stealth](0,0,0)--(-2,1,4)node[below left]{$\vec{r}_2$};
    \end{tikzpicture}
\end{center}

We will use elementary row operations to reduce $A$ to $\mbox{rref}(A)$.
$$\begin{bmatrix}-2&2&1\\4&-2&1\end{bmatrix}\rightsquigarrow\begin{bmatrix}1&0&1\\0&1&3/2\end{bmatrix}$$
Let $\vec{\rho}_1$ and $\vec{\rho}_2$ be the rows of $\mbox{rref}(A)$:
$$\vec{\rho}_1=\begin{bmatrix}1&0&1\end{bmatrix},\quad \vec{\rho}_2=\begin{bmatrix}0&1&3/2\end{bmatrix}$$
What do you think $\mbox{span}(\vec{\rho}_1, \vec{\rho}_2)$ looks like?  

The following video will help us visualize $\mbox{span}(\vec{\rho}_1, \vec{\rho}_2)$ and compare it to $\mbox{span}(\vec{r}_1, \vec{r}_2)$.

\youtube{6wz-5G14jo8}

Based on what we observed in the video, we may conjecture that 
$$\mbox{span}(\vec{\rho}_1, \vec{\rho}_2)=\mbox{span}(\vec{r}_1, \vec{r}_2)$$

\begin{center}
\tdplotsetmaincoords{70}{130}
\begin{tikzpicture}[rotate around x=0]
	\draw[->](-2,0,0)--(3,0,0) node[below left]{$y$};
    \draw[->](0,-1,0)--(0,3,0) node[below left]{$z$};
    \draw[->](0,0,-2)--(0,0,5) node[below left]{$x$};
    \filldraw[blue, opacity=0.3] (0,0,0)--(2,1,-2)--(0,2,2)--(-2,1,4)--cycle;
    
    \draw[->, line width=2pt,blue, -stealth](0,0,0)--(2,1,-2)node[below right]{$\vec{r}_1$};
      
    \draw[->, line width=2pt,blue, -stealth](0,0,0)--(-2,1,4)node[below left]{$\vec{r}_2$};
    
    \draw[->, line width=2pt,red, -stealth](0,0,0)--(0,1,1)node[above left]{$\vec{\rho}_1$};
      
    \draw[->, line width=2pt,red, -stealth](0,0,0)--(1,3/2,0)node[above right]{$\vec{\rho}_2$};
    
    \end{tikzpicture}
\end{center}
But why does this make sense?  Vectors $\vec{\rho}_1$ and $\vec{\rho}_2$ were obtained from $\vec{r}_1$ and $\vec{r}_2$ by repeated applications of elementary row operations.  At every stage of the row reduction process, the rows of the matrix are linear combinations of $\vec{r}_1$ and $\vec{r}_2$.  Thus, at every stage of the row reduction process, the rows of the matrix lie in the span of $\vec{r}_1$ and $\vec{r}_2$.  Our next video shows a step-by-step row reduction process accompanied by sketches of vectors.

\youtube{KpoeWUQ3wkY}

\end{exploration}

Exploration \ref{init:rowspace} makes a convincing case for the following theorem.

\begin{theorem}\label{th:rowBrowA} If matrix $B$ was obtained from matrix $A$ by applying an elementary row operation to $A$ then
$$\mbox{row}(B)=\mbox{row}(A)$$
\end{theorem}
\begin{proof}
Let $\vec{r}_1,\ldots ,\vec{r}_m$ be the rows of $A$. 

There are three elementary row operations.  Clearly, switching the order of vectors in $\mbox{span}(\vec{r}_1,\ldots  ,\vec{r}_m)$ will not affect the span.  

Suppose that $B$ was obtained from $A$ by multiplying the $i^{th}$ row of $A$ by a non-zero constant $k$.  We need to show that 
$$\mbox{span}(\vec{r}_1,\ldots ,k\vec{r}_i,\ldots ,\vec{r}_m)=\mbox{span}(\vec{r}_1,\ldots ,\vec{r}_i,\ldots ,\vec{r}_m)$$

To do this we will assume that some vector $\vec{v}$ is in $\mbox{span}(\vec{r}_1,\ldots ,k\vec{r}_i,\ldots ,\vec{r}_m)$, and show that $\vec{v}$ is in $\mbox{span}(\vec{r}_1,\ldots ,\vec{r}_i,\ldots ,\vec{r}_m)$.  We will then assume that some vector $\vec{w}$ is in $\mbox{span}(\vec{r}_1,\ldots ,\vec{r}_i,\ldots ,\vec{r}_m)$ and show that $\vec{w}$ must be in $\mbox{span}(\vec{r}_1,\ldots ,k\vec{r}_i,\ldots ,\vec{r}_m)$.

Suppose that $\vec{v}$ is in $\mbox{span}(\vec{r}_1,\ldots ,k\vec{r}_i,\ldots ,\vec{r}_m)$.  Then 
$$\vec{v}=a_1\vec{r}_1+\ldots +a_i(k\vec{r}_i)+\ldots +a_m\vec{r}_m$$
But then 
$$\vec{v}=a_1\vec{r}_1+\ldots +(a_ik)\vec{r}_i+\ldots +a_m\vec{r}_m$$
So $\vec{v}$ is in $\mbox{span}(\vec{r}_1,\ldots ,\vec{r}_i,\ldots ,\vec{r}_m)$.

Now suppose $\vec{w}$ is in $\mbox{span}(\vec{r}_1,\ldots ,\vec{r}_i,\ldots ,\vec{r}_m)$, then
$$\vec{w}=b_1\vec{r}_1+\ldots +b_i\vec{r}_i+\ldots +b_m\vec{r}_m$$
But because $k\neq 0$, we can do the following:
$$\vec{w}=b_1\vec{r}_1+\ldots +\frac{b_i}{k}(k\vec{r}_i)+\ldots +b_m\vec{r}_m$$
So $\vec{w}$ is in $\mbox{span}(\vec{r}_1,\ldots ,k\vec{r}_i,\ldots ,\vec{r}_m)$.
  
We leave it to the reader to verify that adding a multiple of one row of $A$ to another does not change the row space.  (See Practice Problem \ref{prob:proofofrowBrowA}.)  
%  Now suppose that $B$ was obtained from $A$ by adding $k$ times row $j$ to row $i$.  We need to show that 
%  $$\mbox{span}(\vec{r}_1,\ldots ,\vec{r}_i+k\vec{r}_j,\ldots ,\vec{r}_m)=\mbox{span}(\vec{r}_1,\ldots ,\vec{r}_i,\ldots ,\vec{r}_m)$$
  
%  To do this, we will assume that some vector $\vec{v}$ is in $\mbox{span}(\vec{r}_1,\ldots ,\vec{r}_i+k\vec{r}_j,\ldots ,\vec{r}_m)$ and show that $\vec{v}$ is in $\mbox{span}(\vec{r}_1,\ldots ,\vec{r}_i,\ldots ,\vec{r}_m)$.  We will then assume that some vector $\vec{w}$ is in $\mbox{span}(\vec{r}_1,\ldots ,\vec{r}_i,\ldots ,\vec{r}_m)$, and show that $\vec{w}$ is in $\mbox{span}(\vec{r}_1,\ldots ,\vec{r}_i+k\vec{r}_j,\ldots ,\vec{r}_m)$.
  
 % Suppose $\vec{v}$ is in $\mbox{span}(\vec{r}_1,\ldots ,\vec{r}_i+k\vec{r}_j,\ldots ,\vec{r}_m)$.  Then

%$$\vec{v}=a_1\vec{r}_1+\ldots +a_i(\vec{r}_i+k\vec{r}_j)+\ldots +a_j\vec{r}_j+\ldots +a_m\vec{r}_m$$
%But then
%$$\vec{v}=a_1\vec{r}_1+\ldots +a_i\vec{r}_i+\ldots +(a_ik+a_j)\vec{r}_j+\ldots +a_m\vec{r}_m$$
%Thus $\vec{v}$ is in $\mbox{span}(\vec{r}_1,\ldots ,\vec{r}_i,\ldots ,\vec{r}_m)$

\end{proof}

\begin{corollary}\label{cor:rowequiv}
If matrix $B$ was obtained from matrix $A$ by applying a sequence of elementary row operations to $A$ then
$$\mbox{row}(B)=\mbox{row}(A)$$
\end{corollary}
\begin{proof}
This follows from repeated applications of Theorem \ref{th:rowBrowA}.
\end{proof}
\begin{corollary}\label{cor:rowArowrrefA}
$$\mbox{row}(A)=\mbox{row}(\mbox{rref}(A))$$
\end{corollary}

\begin{example}\label{ex:basisrowspace}
Let
$$A=\begin{bmatrix}2&-1&1&-4&1\\1&0&3&3&0\\-2&1&-1&5&2\\4&-1&7&2&1\end{bmatrix}$$
Find two distinct bases for $\mbox{row}(A)$.
\begin{explanation}
By Corollary \ref{cor:rowArowrrefA} a basis for $\mbox{row}(\mbox{rref}(A))$ will also be a basis for $\mbox{row}(A)$. Row reduction gives us:
$$\begin{bmatrix}2&-1&1&-4&1\\1&0&3&3&0\\-2&1&-1&5&2\\4&-1&7&2&1\end{bmatrix}\rightsquigarrow\begin{bmatrix}1&0&3&0&-9\\0&1&5&0&-31\\0&0&0&1&3\\0&0&0&0&0\end{bmatrix}=\mbox{rref}(A)$$

Since the zero row contributes nothing to the span, we conclude that the nonzero rows of $\mbox{rref}(A)$ span $\mbox{row}(\mbox{rref}(A))$.  Therefore
$$\mbox{row}(A)=\mbox{span}\Big(\begin{bmatrix}1&0&3&0&-9\end{bmatrix},
\begin{bmatrix}0&1&5&0&-31\end{bmatrix},
\begin{bmatrix}0&0&0&1&3\end{bmatrix}\Big)$$
By Theorem \ref{th:rowsrreflinind} of VEC-0110, the nonzero rows of $\mbox{rref}(A)$ are linearly independent.
%Observe that the leading $1's$ have $0's$ above and below them.  Thus, the non-zero rows of $\mbox{rref}(A)$ are linearly independent.
It follows that the nonzero rows of $\mbox{rref}(A)$ form a basis for $\mbox{row}(A)$.

To find a second basis for $\mbox{row}(A)$, observe that by Corollary \ref{cor:rowequiv} the row space of {\it any} row-echelon form of $A$ will be equal to $\mbox{row}(A)$.  Matrix $A$ has many row-echelon forms.  Here is one of them:
$$B=\begin{bmatrix}1&0&3&3&0\\0&-1&-5&-10&1\\0&0&0&1&3\\0&0&0&0&0 \end{bmatrix}$$
The nonzero rows of $B$ span $\mbox{row}(A)$.  By Theorem \ref{th:rowsofreflinind} of VEC-0110 the nonzero rows of $B$ are linearly independent.  Thus the nonzero rows of $B$ form a basis for $\mbox{row}(A)$.
\end{explanation}
\end{example}

Our observations in Example \ref{ex:basisrowspace} can be generalized to all matrices.  Given any matrix $A$,
\begin{enumerate}
    \item The nonzero rows of $\mbox{rref}(A)$ are linearly independent and span $\mbox{row}(A)$.  (Theorem \ref{th:rowsrreflinind} of VEC-0110, and Corollary \ref{cor:rowArowrrefA})
    \item The nonzero rows of any row-echelon form of $A$ are linearly independent and span $\mbox{row}(A)$.  (Theorem \ref{th:rowsreflinind} of VEC-0110, and Corollary \ref{cor:rowequiv})
\end{enumerate}
Therefore nonzero rows of $\mbox{rref}(A)$ or the nonzero rows of any row-echelon form of $A$ constitute a basis of $\mbox{row}(A)$.  Since all bases for $\mbox{row}(A)$ must have the same number of elements (Theorem \ref{th:dimwelldefined}), we have just proved the following theorem.

\begin{theorem}\label{th:samenumberofnonzerorows}
All row-echelon forms of a given matrix have the same number of nonzero rows.
\end{theorem}

This result was first introduced without proof in \href{https://ximera.osu.edu/oerlinalg/LinearAlgebra/SYS-0030/main}{Gaussian Elimination and Rank} where we used it to define the \dfn{rank} of a matrix as the number of nonzero rows in its row-echelon forms.

\begin{theorem}\label{th:dimofrowA}
Let $A$ be a matrix.  
%\begin{enumerate}
%\item The non-zero rows of $\mbox{rref}(A)$ form a basis of $\mbox{row}(A)$.
%\item 
$\mbox{dim}\Big(\mbox{row}(A)\Big)=\mbox{rank}(A)$.
%\end{enumerate}
\end{theorem}


\subsection*{Column Space of a Matrix}
\begin{definition}\label{def:colspace} Let $A$ be an $m\times n$ matrix.  The \dfn{column space} of $A$, denoted by $\mbox{col}(A)$, is the subspace of $\RR^m$ spanned by the columns of $A$.
\end{definition}

\begin{exploration}\label{init:colspace}
Let
$$B=\begin{bmatrix}2&-1&3&1\\1&-1&2&2\\1&3&-2&-3\end{bmatrix}$$
Our goal is to find a basis for $\mbox{col}(B)$.  To do this we need to find a linearly independent subset of the columns of $B$ that spans $\mbox{col}(B)$.

Consider the linear relation:
\begin{equation}\label{eq:init:colspaceB} a_1\begin{bmatrix}2\\1\\1\end{bmatrix}+a_2\begin{bmatrix}-1\\-1\\3\end{bmatrix}+a_3\begin{bmatrix}3\\2\\-2\end{bmatrix}+a_4\begin{bmatrix}1\\2\\-3\end{bmatrix}=\vec{0}\end{equation}

Solving this homogeneous equation amounts to finding $\mbox{rref}(B)$.
$$\begin{bmatrix}2&-1&3&1\\1&-1&2&2\\1&3&-2&-3\end{bmatrix}\rightsquigarrow\begin{bmatrix}1&0&1&0\\0&1&-1&0\\0&0&0&1\end{bmatrix}=\mbox{rref}(B)$$
We now see that (\ref{eq:init:colspaceB}) has infinitely many solutions.  

Observe that the homogeneous equation

\begin{equation}\label{eq:init:colspaceR} a_1\begin{bmatrix}1\\0\\0\end{bmatrix}+a_2\begin{bmatrix}0\\1\\0\end{bmatrix}+a_3\begin{bmatrix}1\\-1\\0\end{bmatrix}+a_4\begin{bmatrix}0\\0\\1\end{bmatrix}=\vec{0}\end{equation}

has the same solution set as (\ref{eq:init:colspaceB}).  In particular, $a_1=1$, $a_2=-1$, $a_3=-1$, $a_4=0$ is a non-trivial solution of (\ref{eq:init:colspaceB}) and (\ref{eq:init:colspaceR}).  This means that the third column of $B$ and the third column of $\mbox{rref}(B)$ can be expressed as the first column minus the second column of their respective matrices.  We conclude that the third column of $B$ can be eliminated from the spanning set for $\mbox{col}(B)$ and 
$$\mbox{col}(B)=\mbox{span}\left(\begin{bmatrix}2\\1\\1\end{bmatrix},\begin{bmatrix}-1\\-1\\3\end{bmatrix}, \begin{bmatrix}3\\2\\-2\end{bmatrix}, \begin{bmatrix}1\\2\\-3\end{bmatrix}\right)=\mbox{span}\left(\begin{bmatrix}2\\1\\1\end{bmatrix},\begin{bmatrix}-1\\-1\\3\end{bmatrix}, \begin{bmatrix}1\\2\\-3\end{bmatrix}\right)$$
Having gotten rid of one of the vectors, we need to determine whether the remaining three vectors are linearly independent.  To do this we need to find all solutions of 

\begin{equation}\label{eq:init:colspaceB2} b_1\begin{bmatrix}2\\1\\1\end{bmatrix}+b_2\begin{bmatrix}-1\\-1\\3\end{bmatrix}+b_3\begin{bmatrix}1\\2\\-3\end{bmatrix}=\vec{0}\end{equation}
Fortunately, we do not have to start from scratch.  Observe that crossing out the third column in the previous row reduction process yields the desired reduced row-echelon form.
\begin{center}
\begin{tikzpicture}
  \matrix (m)[
    matrix of math nodes,
    nodes in empty cells,
    left delimiter={[},right delimiter={]}] {
    2    & -1  & 3 & 1 \\
    1 & -1   & 2  & 2  \\
    1   & 3    & -2 & -3     \\
  } ;
  
  \matrix (s)[
    matrix of math nodes,
    nodes in empty cells,
    ] at ($(m.east)+(0.5,0)$)
    {
     \\
    \rightsquigarrow \\
    \\
  } ;
  
  \matrix (r)[
    matrix of math nodes,
    nodes in empty cells,
    left delimiter={[},right delimiter={]}] at ($(m.east)+(2.2,0)$)
    {
     1&0&1&0\\
    0&1&-1&0 \\
    0&0&0&1\\
  } ;
  
  \draw[blue](m-1-3.north) -- (m-3-3.south);
   \draw[blue](r-1-3.north) -- (r-3-3.south);
 \end{tikzpicture}
 \end{center} 
 This time the reduced row-echelon form tells us that (\ref{eq:init:colspaceB2}) has only the trivial solution.  We conclude that the three vectors are linearly independent and 
 $$\left\{\begin{bmatrix}2\\1\\1\end{bmatrix},\begin{bmatrix}-1\\-1\\3\end{bmatrix}, \begin{bmatrix}1\\2\\-3\end{bmatrix}\right\}$$
 is a basis for $\mbox{col}(B)$.
\end{exploration}

The approach we took to find a basis for $\mbox{col}(B)$ in Exploration \ref{init:colspace} uses the reduced row-echelon form of $B$. It is true, however, that {\it any} row-echelon form of $B$ could have been used in place of $\mbox{rref}(B)$.  (Why?). We generalize the steps as follows:
\begin{procedure}\label{proc:colspace}
Given a matrix $B$, a basis for $\mbox{col}(B)$ can be found as follows:
\begin{enumerate}
\item Find $\mbox{rref}(B)$ (or any row-echelon form $B'$ of $B$.)
\item Identify the pivot columns of $\mbox{rref}(B)$ (or $B'$).
\item The columns of $B$ corresponding to the pivot columns of $\mbox{rref}(B)$ (or $B'$) form a basis for $\mbox{col}(B)$.
\end{enumerate}
\begin{proof}  Let $\vec{b}_1,\ldots ,\vec{b}_n$ be the columns of $B$, and let $\vec{b}'_1,\ldots ,\vec{b}'_n$ be the columns of $\mbox{rref}(B)$ (or $B'$).
Observe that the equations
\begin{equation}a_1\vec{b}_1+\ldots +a_n\vec{b}_n=\vec{0}\end{equation}
\begin{equation}a_1\vec{b}'_1+\ldots +a_n\vec{b}'_n=\vec{0}\end{equation}
have the same solution set.  This means that any non-trivial relation among the columns of $\mbox{rref}(B)$ (or $B'$) translates into a non-trivial relation among the columns of $B$.  Likewise, any collection of linearly independent columns of $\mbox{rref}(B)$ (or $B'$) corresponds to linearly independent columns of $B$.

%Because the leading $1's$ are the only non-zero entries in the pivot columns of $\mbox{rref}(B)$ and because no two leading $1's$ appears in the same row, it is easy to see that the pivot columns of $\mbox{rref}(B)$ are linearly independent.  
By Theorems \ref{th:rowsrreflinind} and \ref{th:rowsofreflinind} of VEC-0110, the pivot columns of $\mbox{rref}(B)$ (or $B'$) are linearly independent.  Therefore the corresponding columns of $B$ are linearly independent.  Non-pivot columns can be expressed as linear combinations of the pivot columns, therefore they contribute nothing to the span and can be removed from the spanning set. 
\end{proof}
\end{procedure}

The proof of Procedure \ref{proc:colspace} shows that the number of basis elements for the column space of a matrix is equal to the number of pivot columns.  But the number of pivot columns is the same as the number of pivots in a row-echelon form, which is equal to the number of nonzero rows and the rank of the matrix.  This gives us the following important result.
\begin{theorem}\label{th:dimroweqdimcoleqrank}
Let $A$ be a matrix.
$$\mbox{dim}\Big(\mbox{row}(A)\Big)=\mbox{rank}(A)=\mbox{dim}\Big(\mbox{col}(A)\Big)$$
\end{theorem}

\begin{example}\label{ex:basiscolspace}
We will return to matrix $A$ of Example \ref{ex:basisrowspace} and find a basis for $\mbox{col}(A)$.
\begin{explanation}
We begin by finding $\mbox{rref}(A)$.
$$\begin{bmatrix}2&-1&1&-4&1\\1&0&3&3&0\\-2&1&-1&5&2\\4&-1&7&2&1\end{bmatrix}\rightsquigarrow\begin{bmatrix}1&0&3&0&-9\\0&1&5&0&-31\\0&0&0&1&3\\0&0&0&0&0\end{bmatrix}=\mbox{rref}(A)$$
Columns $1$, $2$ and $4$ of $\mbox{rref}(A)$ contain leading $1's$.  Therefore columns $1$, $2$ and $4$ of $A$ form a basis for $\mbox{col}(A)$.
\end{explanation}

\end{example}

\subsection*{The Null Space}
\begin{definition}\label{def:nullspace} Let $A$ be an $m\times n$ matrix.  The \dfn{null space} of $A$, denoted by $\mbox{null}(A)$, is the set of all vectors $\vec{x}$ in $\RR^n$ such that $A\vec{x}=\vec{0}$.
\end{definition}

\begin{example}\label{ex:nullintro}Find $\mbox{null}(A)$ if
$$A=\begin{bmatrix}3&-1\\-6&2\end{bmatrix}$$
\begin{explanation}We need to solve the equation $A\vec{x}=\vec{0}$.  Row reduction gives us
$$\begin{bmatrix}3&-1\\-6&2\end{bmatrix}\rightsquigarrow\begin{bmatrix}1&-1/3\\0&0\end{bmatrix}=\mbox{rref}(A)$$
We conclude that $\vec{x}=\begin{bmatrix}1/3\\1\end{bmatrix}t$.  Thus $\mbox{null}(A)$ consists of all vectors of the form $\begin{bmatrix}1/3\\1\end{bmatrix}t$.  We might write
$$\mbox{null}(A)=\left\{\begin{bmatrix}1/3\\1\end{bmatrix}t\right\}$$ or
$$\mbox{null}(A)=\mbox{span}\left(\begin{bmatrix}1/3\\1\end{bmatrix}\right)$$
\end{explanation}
\end{example}
Example \ref{ex:nullintro} allows us to make an important observation. Note that every scalar multiple of $\begin{bmatrix}1/3\\1\end{bmatrix}$ is contained in $\mbox{null}(A)$.  This means that $\mbox{null}(A)$ is closed under vector addition and scalar multiplication.  Recall that this property makes $\mbox{null}(A)$ a \dfn{subspace} of $\RR^n$.  This result was first presented as Practice Problem \ref{prob:null(A)_is_subspace}. We now formalize it as a theorem.

\begin{theorem}\label{th:nullsubspacern} Let $A$ be an $m\times n$ matrix.  Then $\mbox{null}(A)$ is a subspace of $\RR^n$.
\end{theorem}
\begin{proof}To show that $\mbox{null}(A)$ is closed under vector addition and scalar multiplication we will show that a linear combination of any two elements of $\mbox{null}(A)$ is contained in $\mbox{null}(A)$.

Suppose $\vec{x}_1$ and $\vec{x}_2$ are in $\mbox{null}(A)$.  Then $A\vec{x}_1=\vec{0}$ and $A\vec{x}_2=\vec{0}$.  But then
$$A(a_1\vec{x}_1+a_2\vec{x}_2)=a_1A\vec{x}_1+a_2A\vec{x}_2=\vec{0}$$
We conclude that $a_1\vec{x}_1+a_2\vec{x}_2$ is also in $\mbox{null}(A)$.
\end{proof}

\begin{example}\label{ex:dimnull} Find a basis for $\mbox{null}(A)$, where $A$ is the matrix in Example \ref{ex:basisrowspace}.
% \begin{example}\label{ex:dimnull} We will return to matrix $A$ of Example \ref{ex:basisrowspace} and find $\mbox{null}(A)$, a basis for $\mbox{null}(A)$, and $\mbox{dim}\Big(\mbox{null}(A)\Big)$.
\begin{explanation}
Elements in the null space of $A$ are solutions to the equation
$$\begin{bmatrix}2&-1&1&-4&1\\1&0&3&3&0\\-2&1&-1&5&2\\4&-1&7&2&1\end{bmatrix}\vec{x}=\vec{0}$$
Row reduction yields $\mbox{rref}(A)$
$$\begin{bmatrix}2&-1&1&-4&1\\1&0&3&3&0\\-2&1&-1&5&2\\4&-1&7&2&1\end{bmatrix}\rightsquigarrow\begin{bmatrix}1&0&3&0&-9\\0&1&5&0&-31\\0&0&0&1&3\\0&0&0&0&0\end{bmatrix}$$
Therefore, elements of $\mbox{null}(A)$ are of the form
$$\vec{x}=\begin{bmatrix}9t-3s\\31t-5s\\s\\-3t\\t\end{bmatrix}=\begin{bmatrix}-3\\-5\\1\\0\\0\end{bmatrix}s+\begin{bmatrix}9\\31\\0\\-3\\1\end{bmatrix}t$$
Thus
$$\mbox{null}(A)=\mbox{span}\left( \begin{bmatrix}-3\\-5\\1\\0\\0\end{bmatrix}, \begin{bmatrix}9\\31\\0\\-3\\1\end{bmatrix}\right)$$

Now we need to find a basis for $\mbox{null}(A)$ we need to find linearly independent vectors that span $\mbox{null}(A)$.  Take a closer look at the vectors
$$\begin{bmatrix}-3\\-5\\{\color{red}\fbox{$1$}}\\0\\{\color{blue}\fbox{$0$}}\end{bmatrix}, \begin{bmatrix}9\\31\\{\color{red}\fbox{$0$}}\\-3\\{\color{blue}\fbox{$1$}}\end{bmatrix}$$
Because of the locations of $1's$ and $0's$, it is clear that one vector is not a scalar multiple of the other.  Therefore the two vectors are linearly independent.  We conclude that 
$$\left\{\begin{bmatrix}-3\\-5\\1\\0\\0\end{bmatrix}, \begin{bmatrix}9\\31\\0\\-3\\1\end{bmatrix}\right\}$$
is a basis of $\mbox{null}(A)$, and $\mbox{dim}\Big(\mbox{null}(A)\Big)=2$.
\end{explanation}
\end{example}
It is not a coincidence that the steps we used in Example \ref{ex:dimnull} produced linearly independent vectors, and it is worth while to try to understand why this procedure will always produce linearly independent vectors.

Take a closer look at the elements of the null space:
$$\vec{x}=\begin{bmatrix}9t-3s\\31t-5s\\{\color{red}\fbox{$s$}}\\-3t\\{\color{blue}\fbox{$t$}}\end{bmatrix}=\begin{bmatrix}-3\\-5\\{\color{red}\fbox{$1$}}\\0\\{\color{blue}\fbox{$0$}}\end{bmatrix}s+\begin{bmatrix}9\\31\\{\color{red}\fbox{$0$}}\\-3\\{\color{blue}\fbox{$1$}}\end{bmatrix}t$$
The parameter $s$ in the third component of $\vec{x}$ produces a $1$ in the third component of the first vector and a $0$ in the third component of the second vector, while parameter $t$ in the fifth component of $\vec{x}$ produces a $1$ in the fifth component of the second vector and a $0$ in the fifth component of the first vector. This makes it clear that the two vectors are linearly independent.  

This pattern will hold for any number of parameters, each parameter producing a $1$ in exactly one vector and $0's$ in the corresponding components of the other vectors.

$$\begin{bmatrix}\vdots \\t_1\\\vdots\\t_2\\\vdots\\t_3\\\vdots\\t_n\\\vdots \end{bmatrix}=\begin{bmatrix}\vdots \\1\\\vdots\\0\\\vdots\\0\\\vdots\\0\\\vdots \end{bmatrix}t_1+\ldots +\begin{bmatrix}\vdots \\0\\\vdots\\1\\\vdots\\0\\\vdots\\0\\\vdots \end{bmatrix}t_2+\ldots+\begin{bmatrix}\vdots \\0\\\vdots\\0\\\vdots\\1\\\vdots\\0\\\vdots \end{bmatrix}t_3+\ldots+\begin{bmatrix}\vdots \\0\\\vdots\\0\\\vdots\\0\\\vdots\\1\\\vdots \end{bmatrix}t_n$$
Therefore, vectors obtained in this way will always be linearly independent.

%Each free variable corresponds to a separate parameter, therefore the dimension of any matrix (number of elements in a basis) is equal to the number of free variables in the solution of the homogeneous equation associated with the matrix.


\subsection*{Rank and Nullity Theorem}
\begin{definition}\label{def:matrixnullity}
Let $A$ be a matrix.  The dimension of the null space of $A$ is called the \dfn{nullity} of $A$.
$$\mbox{dim}\Big(\mbox{null}(A)\Big)=\mbox{nullity}(A)$$
\end{definition}

We know that the dimension of the row space and the dimension of the column space of a matrix are the same and are equal to the rank of the matrix (or the number of nonzero rows in any row-echelon form of the matrix).

As we observed in Example \ref{ex:dimnull}, the dimension of the null space of a matrix is equal to the number of free variables in the solution vector of the homogeneous system associated with the matrix.  Since the number of pivots and the number of free variables add up to the number of columns in a matrix (Theorem \ref{th:rankandsolutions}) we have the following significant result.

\begin{theorem}\label{th:matrixranknullity} Let $A$ be an $m\times n$ matrix.  Then 
$$\mbox{rank}(A)+\mbox{nullity}(A)=n$$
\end{theorem}
We will see the geometric implications of this theorem when we study linear transformations.

\section*{Practice Problems}
\begin{problem}
Let
$$A=\begin{bmatrix}2&0&2&4\\1&3&-2&-1\\-1&-2&1&0\end{bmatrix}$$

\begin{problem}\label{prob:colrowmatrixA1}
Find $\mbox{rref}(A)$.
$$\mbox{rref}(A)=\begin{bmatrix}\answer{1}&\answer{0}&\answer{1}&\answer{2}\\\answer{0}&\answer{1}&\answer{-1}&\answer{-1}\\\answer{0}&\answer{0}&\answer{0}&\answer{0}\end{bmatrix}$$
\end{problem}

\begin{problem}\label{prob:colrowmatrixA2}
$$\mbox{rank}(A)=\mbox{dim}(\mbox{row}(A))=\mbox{dim}(\mbox{col}(A))=\answer{2}$$
\end{problem}

\begin{problem}\label{prob:colrowmatrixA3}
Use $\mbox{rref}(A)$ and the procedure outlined in Example \ref{ex:basisrowspace} to find a basis for $\mbox{row}(A)$.

Basis for $\mbox{row}(A):\quad
\left\{\begin{bmatrix}\answer{1}& \answer{0}& \answer{1} & \answer{2}\end{bmatrix},\begin{bmatrix}\answer{0} & \answer{1}& \answer{-1} &\answer{-1}\end{bmatrix} \right\}$
\end{problem}

\begin{problem}\label{prob:colrowmatrixA4}
Use Procedure \ref{proc:colspace} to find a basis for $\mbox{col}(A)$.

Basis for $\mbox{col}(A):\quad
\left\{ \begin{bmatrix}\answer{2}\\\answer{1}\\\answer{-1}\end{bmatrix}, \begin{bmatrix}\answer{0}\\\answer{3}\\\answer{-2}\end{bmatrix}\right\}$
\end{problem}

\end{problem}

\begin{problem} 
Let
$$B=\begin{bmatrix}1&2&3\\-1&1&3\\2&0&-2\\1&-2&-5\\0&1&2\end{bmatrix}$$

\begin{problem}\label{prob:colrowmatrixB1}
Find $\mbox{rref}(B)$.
$$\mbox{rref}(B)=\begin{bmatrix}\answer{1}&\answer{0}&\answer{-1}\\\answer{0}&\answer{1}&\answer{2}\\\answer{0}&\answer{0}&\answer{0}\\\answer{0}&\answer{0}&\answer{0}\\\answer{0}&\answer{0}&\answer{0}\end{bmatrix}$$
\end{problem}

\begin{problem}\label{prob:colrowmatrixB2}
$$\mbox{rank}(B)=\mbox{dim}(\mbox{row}(B))=\mbox{dim}(\mbox{col}(B))=\answer{2}$$
\end{problem}

\begin{problem}\label{prob:colrowmatrixB3}
Use $\mbox{rref}(B)$ and the procedure outlined in Example \ref{ex:basisrowspace} to find a basis for $\mbox{row}(B)$.

Basis for $\mbox{row}(B):\quad
\left\{\begin{bmatrix}\answer{1}& \answer{0}& \answer{-1}\end{bmatrix},\begin{bmatrix}\answer{0} & \answer{1}& \answer{2}\end{bmatrix} \right\}$
\end{problem}

\begin{problem}\label{prob:colrowmatrixB4}
Use Procedure \ref{proc:colspace} to find a basis for $\mbox{col}(B)$.

Basis for $\mbox{col}(B):\quad
\left\{ \begin{bmatrix}\answer{1}\\\answer{-1}\\\answer{2}\\\answer{1}\\\answer{0}\end{bmatrix}, \begin{bmatrix}\answer{2}\\\answer{1}\\\answer{0}\\\answer{-2}\\\answer{1}\end{bmatrix}\right\}$
\end{problem}

\end{problem}





\begin{problem}\label{prob:rankoftranspose}
Prove that $\mbox{rank}(A)=\mbox{rank}(A^T)$
\end{problem}

\begin{problem}\label{prob:basisforV}
Find a basis for $V$ if 
$$V=\mbox{span}\left( \begin{bmatrix}1\\0\\2\end{bmatrix}, \begin{bmatrix}-1\\2\\-1\end{bmatrix}, \begin{bmatrix}1\\2\\3\end{bmatrix}, \begin{bmatrix}3\\-1\\0\end{bmatrix}, \begin{bmatrix}3\\1\\1\end{bmatrix}\right)$$
\begin{hint}
Find a basis for the column space of a matrix whose columns are the given vectors.
\end{hint}
\end{problem}

\begin{problem}
This problem will refer to matrices $A$ and $B$ of Problems \ref{prob:colrowmatrixA1} and \ref{prob:colrowmatrixB1}.

\begin{problem}\label{prob:nullABC1}
Find a basis for $\mbox{null}(A)$.

Basis for $\mbox{null}(A):\quad \left\{ \begin{bmatrix}\answer{-1}\\\answer{1}\\1\\0\end{bmatrix}, \begin{bmatrix}\answer{-2}\\\answer{1}\\0\\1\end{bmatrix}\right\}$

Demonstrate that the Rank-Nullity Theorem (Theorem \ref{th:matrixranknullity}) holds for $A$.

Explain how you can quickly tell that the two vectors you selected for your basis are linearly independent.
\end{problem}

\begin{problem}\label{prob:nullABC2}
Find a basis for $\mbox{null}(B)$.

Basis for $\mbox{null}(B):\quad \left\{ \begin{bmatrix}\answer{1}\\\answer{-2}\\1\end{bmatrix}\right\}$

Demonstrate that the Rank-Nullity Theorem (Theorem \ref{th:matrixranknullity}) holds for $B$.
\end{problem}

\end{problem}


\begin{problem}
Suppose matrix $M$ is such that 
$$\mbox{rref}(M)=\begin{bmatrix}1&0&2&0&3&1\\0&1&-1&0&1&-2\\0&0&0&1&-2&1\\0&0&0&0&0&0\end{bmatrix}$$

\begin{problem}\label{prob:nullM1}
Follow the process used in Example \ref{ex:dimnull} to find a basis for $\mbox{null}(M)$.  Explain why the basis elements obtained in this way are linearly independent.

Basis of $\mbox{null}(M):\quad\left\{\begin{bmatrix}\answer{-2}\\\answer{1}\\1\\\answer{0}\\0\\0\end{bmatrix}, \begin{bmatrix}\answer{-3}\\\answer{-1}\\0\\\answer{2}\\1\\0\end{bmatrix}, \begin{bmatrix}\answer{-1}\\\answer{2}\\0\\\answer{-1}\\0\\1\end{bmatrix} \right\}$
\end{problem}

\begin{problem}\label{prob:nullM2}
Let $\vec{v}_1,\ldots,\vec{v}_6$ denote the columns of $M$.  Express $\vec{v}_3$ as a linear combination of $\vec{v}_1$ and $\vec{v}_2$.

Answer:
$$\vec{v}_3=\answer{2}\vec{v}_1+\answer{-1}\vec{v}_2$$
\end{problem}
\end{problem}

\begin{problem}\label{prob:truefalse3by5}
Suppose $A$ is a $3\times 5$ matrix.  Which of the following statements could be true?
\begin{selectAll}
 \choice{$\mbox{dim}(\mbox{col}(A))=5$}
 \choice[correct]{$\mbox{dim}(\mbox{row}(A))=3$}
 \choice{$\mbox{dim}(\mbox{null}(A))=1$}
 \choice[correct]{$\mbox{dim}(\mbox{null}(A))=2$}
 \choice[correct]{$\mbox{dim}(\mbox{null}(A))=3$}
 \end{selectAll}
\end{problem}

\begin{problem}\label{prob:truefalse7by3}
Suppose $A$ is a $7\times 3$ matrix.  Which of the following statements could be true?
\begin{selectAll}
 \choice[correct]{$\mbox{dim}(\mbox{col}(A))=3$}
 \choice[correct]{$\mbox{dim}(\mbox{row}(A))=3$}
 \choice{$\mbox{dim}(\mbox{row}(A))=7$}
 \choice[correct]{$\mbox{dim}(\mbox{null}(A))=0$}
 \choice{$\mbox{dim}(\mbox{null}(A))=4$}
 \end{selectAll}
\end{problem}

\begin{problem}\label{prob:proofofrowBrowA}
Complete the proof of Theorem \ref{th:rowBrowA} by showing that adding a scalar multiple of one row of a matrix to another row does not change the row space.
\end{problem}
\end{document}
