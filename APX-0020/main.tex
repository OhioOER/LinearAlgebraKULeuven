\documentclass{ximera}
%%% Begin Laad packages

\makeatletter
\@ifclassloaded{xourse}{%
    \typeout{Start loading preamble.tex (in a XOURSE)}%
    \def\isXourse{true}   % automatically defined; pre 112022 it had to be set 'manually' in a xourse
}{%
    \typeout{Start loading preamble.tex (NOT in a XOURSE)}%
}
\makeatother

\def\isEn\true 

\pgfplotsset{compat=1.16}

\usepackage{currfile}

% 201908/202301: PAS OP: babel en doclicense lijken problemen te veroorzaken in .jax bestand
% (wegens syntax error met toegevoegde \newcommands ...)
\pdfOnly{
    \usepackage[type={CC},modifier={by-nc-sa},version={4.0}]{doclicense}
    %\usepackage[hyperxmp=false,type={CC},modifier={by-nc-sa},version={4.0}]{doclicense}
    %%% \usepackage[dutch]{babel}
}



\usepackage[utf8]{inputenc}
\usepackage{morewrites}   % nav zomercursus (answer...?)
\usepackage{multirow}
\usepackage{multicol}
\usepackage{tikzsymbols}
\usepackage{ifthen}
%\usepackage{animate} BREAKS HTML STRUCTURE USED BY XIMERA
\usepackage{relsize}

\usepackage{eurosym}    % \euro  (€ werkt niet in xake ...?)
\usepackage{fontawesome} % smileys etc

% Nuttig als ook interactieve beamer slides worden voorzien:
\providecommand{\p}{} % default nothing ; potentially usefull for slides: redefine as \pause
%providecommand{\p}{\pause}

    % Layout-parameters voor het onderschrift bij figuren
\usepackage[margin=10pt,font=small,labelfont=bf, labelsep=endash,format=hang]{caption}
%\usepackage{caption} % captionof
%\usepackage{pdflscape}    % landscape environment

% Met "\newcommand\showtodonotes{}" kan je todonotes tonen (in pdf/online)
% 201908: online werkt het niet (goed)
\providecommand\showtodonotes{disable}
\providecommand\todo[1]{\typeout{TODO #1}}
%\usepackage[\showtodonotes]{todonotes}
%\usepackage{todonotes}

%
% Poging tot aanpassen layout
%
\usepackage{tcolorbox}
\tcbuselibrary{theorems}

%%% Einde laad packages

%%% Begin Ximera specifieke zaken

\graphicspath{
	{../../}
	{../}
	{./}
  	{../../pictures/}
   	{../pictures/}
   	{./pictures/}
	{./explog/}    % M05 in groeimodellen       
}

%%% Einde Ximera specifieke zaken

%
% define softer blue/red/green, use KU Leuven base colors for blue (and dark orange for red ?)
%
% todo: rather redefine blue/red/green ...?
%\definecolor{xmblue}{rgb}{0.01, 0.31, 0.59}
%\definecolor{xmred}{rgb}{0.89, 0.02, 0.17}
\definecolor{xmdarkblue}{rgb}{0.122, 0.671, 0.835}   % KU Leuven Blauw
\definecolor{xmblue}{rgb}{0.114, 0.553, 0.69}        % KU Leuven Blauw
\definecolor{xmgreen}{rgb}{0.13, 0.55, 0.13}         % No KULeuven variant for green found ...

\definecolor{xmaccent}{rgb}{0.867, 0.541, 0.18}      % KU Leuven Accent (orange ...)
\definecolor{kuaccent}{rgb}{0.867, 0.541, 0.18}      % KU Leuven Accent (orange ...)

\colorlet{xmred}{xmaccent!50!black}                  % Darker version of KU Leuven Accent

\providecommand{\blue}[1]{{\color{blue}#1}}    
\providecommand{\red}[1]{{\color{red}#1}}

\renewcommand\CancelColor{\color{xmaccent!50!black}}

% werkt in math en text mode om MATH met oranje (of grijze...)  achtergond te tonen (ook \important{\text{blabla}} lijkt te werken)
%\newcommand{\important}[1]{\ensuremath{\colorbox{xmaccent!50!white}{$#1$}}}   % werkt niet in Mathjax
%\newcommand{\important}[1]{\ensuremath{\colorbox{lightgray}{$#1$}}}
\newcommand{\important}[1]{\ensuremath{\colorbox{orange}{$#1$}}}   % TODO: kleur aanpassen voor mathjax; wordt overschreven infra!


% Uitzonderlijk kan met \pdfnl in de PDF een newline worden geforceerd, die online niet nodig/nuttig is omdat daar de regellengte hoe dan ook niet gekend is.
\ifdefined\HCode%
\providecommand{\pdfnl}{}%
\else%
\providecommand{\pdfnl}{%
  \\%
}%
\fi

% Uitzonderlijk kan met \handoutnl in de handout-PDF een newline worden geforceerd, die noch online noch in de PDF-met-antwoorden nuttig is.
\ifdefined\HCode
\providecommand{\handoutnl}{}
\else
\providecommand{\handoutnl}{%
\ifhandout%
  \nl%
\fi%
}
\fi



% \cellcolor IGNORED by tex4ht ?
% \begin{center} seems not to wordk
    % (missing margin-left: auto;   on tabular-inside-center ???)
%\newcommand{\importantcell}[1]{\ensuremath{\cellcolor{lightgray}#1}}  %  in tabular; usablility to be checked
\providecommand{\importantcell}[1]{\ensuremath{#1}}     % no mathjax2 support for colloring array cells

\pdfOnly{
  \renewcommand{\important}[1]{\ensuremath{\colorbox{kuaccent!50!white}{$#1$}}}
  \renewcommand{\importantcell}[1]{\ensuremath{\cellcolor{kuaccent!40!white}#1}}   
}

%%% Tikz styles


\pgfplotsset{compat=1.16}

\usetikzlibrary{trees,positioning,arrows,fit,shapes,math,calc,decorations.markings,through,intersections,patterns,matrix}

\usetikzlibrary{decorations.pathreplacing,backgrounds}    % 5/2023: from experimental


\usetikzlibrary{angles,quotes}

\usepgfplotslibrary{fillbetween} % bepaalde_integraal
\usepgfplotslibrary{polar}    % oa voor poolcoordinaten.tex

\pgfplotsset{ownstyle/.style={axis lines = center, axis equal image, xlabel = $x$, ylabel = $y$, enlargelimits}} 

\pgfplotsset{
	plot/.style={no marks,samples=50}
}

\newcommand{\xmPlotsColor}{
	\pgfplotsset{
		plot1/.style={darkgray,no marks,samples=100},
		plot2/.style={lightgray,no marks,samples=100},
		plotresult/.style={blue,no marks,samples=100},
		plotblue/.style={blue,no marks,samples=100},
		plotred/.style={red,no marks,samples=100},
		plotgreen/.style={green,no marks,samples=100},
		plotpurple/.style={purple,no marks,samples=100}
	}
}
\newcommand{\xmPlotsBlackWhite}{
	\pgfplotsset{
		plot1/.style={black,loosely dashed,no marks,samples=100},
		plot2/.style={black,loosely dotted,no marks,samples=100},
		plotresult/.style={black,no marks,samples=100},
		plotblue/.style={black,no marks,samples=100},
		plotred/.style={black,dotted,no marks,samples=100},
		plotgreen/.style={black,dashed,no marks,samples=100},
		plotpurple/.style={black,dashdotted,no marks,samples=100}
	}
}


\newcommand{\xmPlotsColorAndStyle}{
	\pgfplotsset{
		plot1/.style={darkgray,no marks,samples=100},
		plot2/.style={lightgray,no marks,samples=100},
		plotresult/.style={blue,no marks,samples=100},
		plotblue/.style={xmblue,no marks,samples=100},
		plotred/.style={xmred,dashed,thick,no marks,samples=100},
		plotgreen/.style={xmgreen,dotted,very thick,no marks,samples=100},
		plotpurple/.style={purple,no marks,samples=100}
	}
}


%\iftikzexport
\xmPlotsColorAndStyle
%\else
%\xmPlotsBlackWhite
%\fi
%%%


%
% Om venndiagrammen te arceren ...
%
\makeatletter
\pgfdeclarepatternformonly[\hatchdistance,\hatchthickness]{north east hatch}% name
{\pgfqpoint{-1pt}{-1pt}}% below left
{\pgfqpoint{\hatchdistance}{\hatchdistance}}% above right
{\pgfpoint{\hatchdistance-1pt}{\hatchdistance-1pt}}%
{
	\pgfsetcolor{\tikz@pattern@color}
	\pgfsetlinewidth{\hatchthickness}
	\pgfpathmoveto{\pgfqpoint{0pt}{0pt}}
	\pgfpathlineto{\pgfqpoint{\hatchdistance}{\hatchdistance}}
	\pgfusepath{stroke}
}
\pgfdeclarepatternformonly[\hatchdistance,\hatchthickness]{north west hatch}% name
{\pgfqpoint{-\hatchthickness}{-\hatchthickness}}% below left
{\pgfqpoint{\hatchdistance+\hatchthickness}{\hatchdistance+\hatchthickness}}% above right
{\pgfpoint{\hatchdistance}{\hatchdistance}}%
{
	\pgfsetcolor{\tikz@pattern@color}
	\pgfsetlinewidth{\hatchthickness}
	\pgfpathmoveto{\pgfqpoint{\hatchdistance+\hatchthickness}{-\hatchthickness}}
	\pgfpathlineto{\pgfqpoint{-\hatchthickness}{\hatchdistance+\hatchthickness}}
	\pgfusepath{stroke}
}
%\makeatother

\tikzset{
    hatch distance/.store in=\hatchdistance,
    hatch distance=10pt,
    hatch thickness/.store in=\hatchthickness,
   	hatch thickness=2pt
}

\colorlet{circle edge}{black}
\colorlet{circle area}{blue!20}


\tikzset{
    filled/.style={fill=green!30, draw=circle edge, thick},
    arceerl/.style={pattern=north east hatch, pattern color=blue!50, draw=circle edge},
    arceerr/.style={pattern=north west hatch, pattern color=yellow!50, draw=circle edge},
    outline/.style={draw=circle edge, thick}
}




%%% Updaten commando's
\def\hoofding #1#2#3{\maketitle}     % OBSOLETE ??

% we willen (bijna) altijd \geqslant ipv \geq ...!
\newcommand{\geqnoslant}{\geq}
\renewcommand{\geq}{\geqslant}
\newcommand{\leqnoslant}{\leq}
\renewcommand{\leq}{\leqslant}

% Todo: (201908) waarom komt er (soms) underlined voor emph ...?
\renewcommand{\emph}[1]{\textit{#1}}

% API commando's

\newcommand{\ds}{\displaystyle}
\newcommand{\ts}{\textstyle}  % tegenhanger van \ds   (Ximera zet PER  DEFAULT \ds!)

% uit Zomercursus-macro's: 
\newcommand{\bron}[1]{\begin{scriptsize} \emph{#1} \end{scriptsize}}     % deprecated ...?


%definities nieuwe commando's - afkortingen veel gebruikte symbolen
\newcommand{\R}{\ensuremath{\mathbb{R}}}
\newcommand{\Rnul}{\ensuremath{\mathbb{R}_0}}
\newcommand{\Reen}{\ensuremath{\mathbb{R}\setminus\{1\}}}
\newcommand{\Rnuleen}{\ensuremath{\mathbb{R}\setminus\{0,1\}}}
\newcommand{\Rplus}{\ensuremath{\mathbb{R}^+}}
\newcommand{\Rmin}{\ensuremath{\mathbb{R}^-}}
\newcommand{\Rnulplus}{\ensuremath{\mathbb{R}_0^+}}
\newcommand{\Rnulmin}{\ensuremath{\mathbb{R}_0^-}}
\newcommand{\Rnuleenplus}{\ensuremath{\mathbb{R}^+\setminus\{0,1\}}}
\newcommand{\N}{\ensuremath{\mathbb{N}}}
\newcommand{\Nnul}{\ensuremath{\mathbb{N}_0}}
\newcommand{\Z}{\ensuremath{\mathbb{Z}}}
\newcommand{\Znul}{\ensuremath{\mathbb{Z}_0}}
\newcommand{\Zplus}{\ensuremath{\mathbb{Z}^+}}
\newcommand{\Zmin}{\ensuremath{\mathbb{Z}^-}}
\newcommand{\Znulplus}{\ensuremath{\mathbb{Z}_0^+}}
\newcommand{\Znulmin}{\ensuremath{\mathbb{Z}_0^-}}
\newcommand{\C}{\ensuremath{\mathbb{C}}}
\newcommand{\Cnul}{\ensuremath{\mathbb{C}_0}}
\newcommand{\Cplus}{\ensuremath{\mathbb{C}^+}}
\newcommand{\Cmin}{\ensuremath{\mathbb{C}^-}}
\newcommand{\Cnulplus}{\ensuremath{\mathbb{C}_0^+}}
\newcommand{\Cnulmin}{\ensuremath{\mathbb{C}_0^-}}
\newcommand{\Q}{\ensuremath{\mathbb{Q}}}
\newcommand{\Qnul}{\ensuremath{\mathbb{Q}_0}}
\newcommand{\Qplus}{\ensuremath{\mathbb{Q}^+}}
\newcommand{\Qmin}{\ensuremath{\mathbb{Q}^-}}
\newcommand{\Qnulplus}{\ensuremath{\mathbb{Q}_0^+}}
\newcommand{\Qnulmin}{\ensuremath{\mathbb{Q}_0^-}}

\newcommand{\perdef}{\overset{\mathrm{def}}{=}}
\newcommand{\pernot}{\overset{\mathrm{notatie}}{=}}
\newcommand\perinderdaad{\overset{!}{=}}     % voorlopig gebruikt in limietenrekenregels
\newcommand\perhaps{\overset{?}{=}}          % voorlopig gebruikt in limietenrekenregels

\newcommand{\degree}{^\circ}


\DeclareMathOperator{\dom}{dom}     % domein
\DeclareMathOperator{\codom}{codom} % codomein
\DeclareMathOperator{\bld}{bld}     % beeld
\DeclareMathOperator{\graf}{graf}   % grafiek
\DeclareMathOperator{\rico}{rico}   % richtingcoëfficient
\DeclareMathOperator{\co}{co}       % coordinaat
\DeclareMathOperator{\gr}{gr}       % graad

\newcommand{\func}[5]{\ensuremath{#1: #2 \rightarrow #3: #4 \mapsto #5}} % Easy to write a function


% Operators
\DeclareMathOperator{\bgsin}{bgsin}
\DeclareMathOperator{\bgcos}{bgcos}
\DeclareMathOperator{\bgtan}{bgtan}
\DeclareMathOperator{\bgcot}{bgcot}
\DeclareMathOperator{\bgsinh}{bgsinh}
\DeclareMathOperator{\bgcosh}{bgcosh}
\DeclareMathOperator{\bgtanh}{bgtanh}
\DeclareMathOperator{\bgcoth}{bgcoth}

% Oude \Bgsin etc deprecated: gebruik \bgsin, en herdefinieer dat als je Bgsin wil!
%\DeclareMathOperator{\cosec}{cosec}    % not used? gebruik \csc en herdefinieer

% operatoren voor differentialen: to be verified; 1/2020: inconsequent gebruik bij afgeleiden/integralen
\renewcommand{\d}{\mathrm{d}}
\newcommand{\dx}{\d x}
\newcommand{\dd}[1]{\frac{\mathrm{d}}{\mathrm{d}#1}}
\newcommand{\ddx}{\dd{x}}

% om in voorbeelden/oefeningen de notatie voor afgeleiden te kunnen kiezen
% Usage: \afg{(2\sin(x))}  (en wordt d/dx, of accent, of D )
%\newcommand{\afg}[1]{{#1}'}
\newcommand{\afg}[1]{\left(#1\right)'}
%\renewcommand{\afg}[1]{\frac{\mathrm{d}#1}{\mathrm{d}x}}   % include in relevant exercises ...
%\renewcommand{\afg}[1]{D{#1}}

%
% \xmxxx commands: Extra KU Leuven functionaliteit van, boven of naast Ximera
%   ( Conventie 8/2019: xm+nederlandse omschrijving, maar is niet consequent gevolgd, en misschien ook niet erg handig !)
%
% (Met een minimale ximera.cls en preamble.tex zou een bruikbare .pdf moeten kunnen worden gemaakt van eender welke ximera)
%
% Usage: \xmtitle[Mijn korte abstract]{Mijn titel}{Mijn abstract}
% Eerste command na \begin{document}:
%  -> definieert de \title
%  -> definieert de abstract
%  -> doet \maketitle ( dus: print de hoofding als 'chapter' of 'sectie')
% Optionele parameter geeft eenn kort abstract (die met de globale setting \xmshortabstract{} al dan niet kan worden geprint.
% De optionele korte abstract kan worden gebruikt voor pseudo-grappige abtsarts, dus dus globaal al dan niet kunnen worden gebuikt...
% Globale settings:
%  de (optionele) 'korte abstract' wordt enkele getoond als \xmshortabstract is gezet
\providecommand\xmshortabstract{} % default: print (only!) short abstract if present
\newcommand{\xmtitle}[3][]{
	\title{#2}
	\begin{abstract}
		\ifdefined\xmshortabstract
		\ifstrempty{#1}{%
			#3
		}{%
			#1
		}%
		\else
		#3
		\fi
	\end{abstract}
	\maketitle
}

% 
% Kleine grapjes: moeten zonder verder gevolg kunnen worden verwijderd
%
%\newcommand{\xmopje}[1]{{\small#1{\reversemarginpar\marginpar{\Smiley}}}}   % probleem in floats!!
\newtoggle{showxmopje}
\toggletrue{showxmopje}

\newcommand{\xmopje}[1]{%
   \iftoggle{showxmopje}{#1}{}%
}


% -> geef een abstracte-formule-met-rechts-een-concreet-voorbeeld
% VB:  \formulevb{a^2+b^2=c^2}{3^2+4^2=5^2}
%
\ifdefined\HCode
\NewEnviron{xmdiv}[1]{\HCode{\Hnewline<div class="#1">\Hnewline}\BODY{\HCode{\Hnewline</div>\Hnewline}}}
\else
\NewEnviron{xmdiv}[1]{\BODY}
\fi

\providecommand{\formulevb}[2]{
	{\centering

    \begin{xmdiv}{xmformulevb}    % zie css voor online layout !!!
	\begin{tabular}{lcl}
		\important{#1}
		&  &
		Vb: $#2$
		\end{tabular}
	\end{xmdiv}

	}
}

\ifdefined\HCode
\providecommand{\vb}[1]{%
    \HCode{\Hnewline<span class="xmvb">}#1\HCode{</span>\Hnewline}%
}
\else
\providecommand{\vb}[1]{
    \colorbox{blue!10}{#1}
}
\fi

\ifdefined\HCode
\providecommand{\xmcolorbox}[2]{
	\HCode{\Hnewline<div class="xmcolorbox">\Hnewline}#2\HCode{\Hnewline</div>\Hnewline}
}
\else
\providecommand{\xmcolorbox}[2]{
  \cellcolor{#1}#2
}
\fi


\ifdefined\HCode
\providecommand{\xmopmerking}[1]{
 \HCode{\Hnewline<div class="xmopmerking">\Hnewline}#1\HCode{\Hnewline</div>\Hnewline}
}
\else
\providecommand{\xmopmerking}[1]{
	{\footnotesize #1}
}
\fi
% \providecommand{\voorbeeld}[1]{
% 	\colorbox{blue!10}{$#1$}
% }



% Hernoem Proof naar Bewijs, nodig voor HTML versie
\renewcommand*{\proofname}{Bewijs}

% Om opgave van oefening (wordt niet geprint bij oplossingenblad)
% (to be tested test)
\NewEnviron{statement}{\BODY}

% Environment 'oplossing' en 'uitkomst'
% voor resp. volledige 'uitwerking' dan wel 'enkel eindresultaat'
% geimplementeerd via feedback, omdat er in de ximera-server adhoc feedback-code is toegevoegd
%% Niet tonen indien handout
%% Te gebruiken om volledige oplossingen/uitwerkingen van oefeningen te tonen
%% \begin{oplossing}        De optelling is commutatief \end{oplossing}  : verschijnt online enkel 'op vraag'
%% \begin{oplossing}[toon]  De optelling is commutatief \end{oplossing}  : verschijnt steeds onmiddellijk online (bv te gebruiken bij voorbeelden) 

\ifhandout%
    \NewEnviron{oplossing}[1][onzichtbaar]%
    {%
    \ifthenelse{\equal{\detokenize{#1}}{\detokenize{toon}}}
    {
    \def\PH@Command{#1}% Use PH@Command to hold the content and be a target for "\expandafter" to expand once.

    \begin{trivlist}% Begin the trivlist to use formating of the "Feedback" label.
    \item[\hskip \labelsep\small\slshape\bfseries Oplossing% Format the "Feedback" label. Don't forget the space.
    %(\texttt{\detokenize\expandafter{\PH@Command}}):% Format (and detokenize) the condition for feedback to trigger
    \hspace{2ex}]\small%\slshape% Insert some space before the actual feedback given.
    \BODY
    \end{trivlist}
    }
    {  % \begin{feedback}[solution]   \BODY     \end{feedback}  }
    }
    }    
\else
% ONLY for HTML; xmoplossing is styled with css, and is not, and need not be a LaTeX environment
% THUS: it does NOT use feedback anymore ...
%    \NewEnviron{oplossing}{\begin{expandable}{xmoplossing}{\nlen{Toon uitwerking}{Show solution}}{\BODY}\end{expandable}}
    \newenvironment{oplossing}[1][onzichtbaar]
   {%
       \begin{expandable}{xmoplossing}{}
   }
   {%
   	   \end{expandable}
   } 
%     \newenvironment{oplossing}[1][onzichtbaar]
%    {%
%        \begin{feedback}[solution]   	
%    }
%    {%
%    	   \end{feedback}
%    } 
\fi

\ifhandout%
    \NewEnviron{uitkomst}[1][onzichtbaar]%
    {%
    \ifthenelse{\equal{\detokenize{#1}}{\detokenize{toon}}}
    {
    \def\PH@Command{#1}% Use PH@Command to hold the content and be a target for "\expandafter" to expand once.

    \begin{trivlist}% Begin the trivlist to use formating of the "Feedback" label.
    \item[\hskip \labelsep\small\slshape\bfseries Uitkomst:% Format the "Feedback" label. Don't forget the space.
    %(\texttt{\detokenize\expandafter{\PH@Command}}):% Format (and detokenize) the condition for feedback to trigger
    \hspace{2ex}]\small%\slshape% Insert some space before the actual feedback given.
    \BODY
    \end{trivlist}
    }
    {  % \begin{feedback}[solution]   \BODY     \end{feedback}  }
    }
    }    
\else
\ifdefined\HCode
   \newenvironment{uitkomst}[1][onzichtbaar]
    {%
        \begin{expandable}{xmuitkomst}{}%
    }
    {%
    	\end{expandable}%
    } 
\else
  % Do NOT print 'uitkomst' in non-handout
  %  (presumably, there is also an 'oplossing' ??)
  \newenvironment{uitkomst}[1][onzichtbaar]{}{}
\fi
\fi

%
% Uitweidingen zijn extra's die niet redelijkerwijze tot de leerstof behoren
% Uitbreidingen zijn extra's die wel redelijkerwijze tot de leerstof van bv meer geavanceerde versies kunnen behoren (B-programma/Wiskundestudenten/...?)
% Nog niet voorzien: design voor verschillende versies (A/B programma, BIO, voorkennis/ ...)
% Voor 'uitweidingen' is er een environment die online per default is ingeklapt, en in pdf al dan niet kan worden geincluded  (via \xmnouitweiding) 
%
% in een xourse, per default GEEN uitweidingen, tenzij \xmuitweiding expliciet ergens is gezet ...
\ifdefined\isXourse
   \ifdefined\xmuitweiding
   \else
       \def\xmnouitweiding{true}
   \fi
\fi

\ifdefined\xmnouitweiding
\newcounter{xmuitweiding}  % anders error undefined ...  
\excludecomment{xmuitweiding}
\else
\newtheoremstyle{dotless}{}{}{}{}{}{}{ }{}
\theoremstyle{dotless}
\newtheorem*{xmuitweidingnofrills}{}   % nofrills = no accordion; gebruikt dus de dotless theoremstyle!

\newcounter{xmuitweiding}
\newenvironment{xmuitweiding}[1][ ]%
{% 
	\refstepcounter{xmuitweiding}%
    \begin{expandable}{xmuitweiding}{\nlentext{Uitweiding \arabic{xmuitweiding}: #1}{Digression \arabic{xmuitweiding}: #1}}%
	\begin{xmuitweidingnofrills}%
}
{%
    \end{xmuitweidingnofrills}%
    \end{expandable}%
}   
% \newenvironment{xmuitweiding}[1][ ]%
% {% 
% 	\refstepcounter{xmuitweiding}
% 	\begin{accordion}\begin{accordion-item}[Uitweiding \arabic{xmuitweiding}: #1]%
% 			\begin{xmuitweidingnofrills}%
% 			}
% 			{\end{xmuitweidingnofrills}\end{accordion-item}\end{accordion}}   
\fi


\newenvironment{xmexpandable}[1][]{
	\begin{accordion}\begin{accordion-item}[#1]%
		}{\end{accordion-item}\end{accordion}}


% Command that gives a selection box online, but just prints the right answer in pdf
\newcommand{\xmonlineChoice}[1]{\pdfOnly{\wordchoicegiventrue}\wordChoice{#1}\pdfOnly{\wordchoicegivenfalse}}   % deprecated, gebruik onlineChoice ...
\newcommand{\onlineChoice}[1]{\pdfOnly{\wordchoicegiventrue}\wordChoice{#1}\pdfOnly{\wordchoicegivenfalse}}

\newcommand{\TJa}{\nlentext{ Ja }{ Yes }}
\newcommand{\TNee}{\nlentext{ Nee }{ No }}
\newcommand{\TJuist}{\nlentext{ Juist }{ True }}
\newcommand{\TFout}{\nlentext{ Fout }{ False }}

\newcommand{\choiceTrue }{{\renewcommand{\choiceminimumhorizontalsize}{4em}\wordChoice{\choice[correct]{\TJuist}\choice{\TFout}}}}
\newcommand{\choiceFalse}{{\renewcommand{\choiceminimumhorizontalsize}{4em}\wordChoice{\choice{\TJuist}\choice[correct]{\TFout}}}}

\newcommand{\choiceYes}{{\renewcommand{\choiceminimumhorizontalsize}{3em}\wordChoice{\choice[correct]{\TJa}\choice{\TNee}}}}
\newcommand{\choiceNo }{{\renewcommand{\choiceminimumhorizontalsize}{3em}\wordChoice{\choice{\TJa}\choice[correct]{\TNee}}}}

% Optional nicer formatting for wordChoice in PDF

\let\inlinechoiceorig\inlinechoice

%\makeatletter
%\providecommand{\choiceminimumverticalsize}{\vphantom{$\frac{\sqrt{2}}{2}$}}   % minimum height of boxes (cfr infra)
\providecommand{\choiceminimumverticalsize}{\vphantom{$\tfrac{2}{2}$}}   % minimum height of boxes (cfr infra)
\providecommand{\choiceminimumhorizontalsize}{1em}   % minimum width of boxes (cfr infra)

\newcommand{\inlinechoicesquares}[2][]{%
		\setkeys{choice}{#1}%
		\ifthenelse{\boolean{\choice@correct}}%
		{%
            \ifhandout%
               \fbox{\choiceminimumverticalsize #2}\allowbreak\ignorespaces%
            \else%
               \fcolorbox{blue}{blue!20}{\choiceminimumverticalsize #2}\allowbreak\ignorespaces\setkeys{choice}{correct=false}\ignorespaces%
            \fi%
		}%
		{% else
			\fbox{\choiceminimumverticalsize #2}\allowbreak\ignorespaces%  HACK: wat kleiner, zodat fits on line ... 	
		}%
}

\newcommand{\inlinechoicesquareX}[2][]{%
		\setkeys{choice}{#1}%
		\ifthenelse{\boolean{\choice@correct}}%
		{%
            \ifhandout%
               \framebox[\ifdim\choiceminimumhorizontalsize<\width\width\else\choiceminimumhorizontalsize\fi]{\choiceminimumverticalsize\ #2\ }\allowbreak\ignorespaces\setkeys{choice}{correct=false}\ignorespaces%
            \else%
               \fcolorbox{blue}{blue!20}{\makebox[\ifdim\choiceminimumhorizontalsize<\width\width\else\choiceminimumhorizontalsize\fi]{\choiceminimumverticalsize #2}}\allowbreak\ignorespaces\setkeys{choice}{correct=false}\ignorespaces%
            \fi%
		}%
		{% else
        \ifhandout%
			\framebox[\ifdim\choiceminimumhorizontalsize<\width\width\else\choiceminimumhorizontalsize\fi]{\choiceminimumverticalsize\ #2\ }\allowbreak\ignorespaces%  HACK: wat kleiner, zodat fits on line ... 	
        \fi
		}%
}


\newcommand{\inlinechoicepuntjes}[2][]{%
		\setkeys{choice}{#1}%
		\ifthenelse{\boolean{\choice@correct}}%
		{%
            \ifhandout%
               \dots\ldots\ignorespaces\setkeys{choice}{correct=false}\ignorespaces
            \else%
               \fcolorbox{blue}{blue!20}{\choiceminimumverticalsize #2}\allowbreak\ignorespaces\setkeys{choice}{correct=false}\ignorespaces%
            \fi%
		}%
		{% else
			%\fbox{\choiceminimumverticalsize #2}\allowbreak\ignorespaces%  HACK: wat kleiner, zodat fits on line ... 	
		}%
}

% print niets, maar definieer globale variable \myanswer
%  (gebruikt om oplossingsbladen te printen) 
\newcommand{\inlinechoicedefanswer}[2][]{%
		\setkeys{choice}{#1}%
		\ifthenelse{\boolean{\choice@correct}}%
		{%
               \gdef\myanswer{#2}\setkeys{choice}{correct=false}

		}%
		{% else
			%\fbox{\choiceminimumverticalsize #2}\allowbreak\ignorespaces%  HACK: wat kleiner, zodat fits on line ... 	
		}%
}



%\makeatother

\newcommand{\setchoicedefanswer}{
\ifdefined\HCode
\else
%    \renewenvironment{multipleChoice@}[1][]{}{} % remove trailing ')'
    \let\inlinechoice\inlinechoicedefanswer
\fi
}

\newcommand{\setchoicepuntjes}{
\ifdefined\HCode
\else
    \renewenvironment{multipleChoice@}[1][]{}{} % remove trailing ')'
    \let\inlinechoice\inlinechoicepuntjes
\fi
}
\newcommand{\setchoicesquares}{
\ifdefined\HCode
\else
    \renewenvironment{multipleChoice@}[1][]{}{} % remove trailing ')'
    \let\inlinechoice\inlinechoicesquares
\fi
}
%
\newcommand{\setchoicesquareX}{
\ifdefined\HCode
\else
    \renewenvironment{multipleChoice@}[1][]{}{} % remove trailing ')'
    \let\inlinechoice\inlinechoicesquareX
\fi
}
%
\newcommand{\setchoicelist}{
\ifdefined\HCode
\else
    \renewenvironment{multipleChoice@}[1][]{}{)}% re-add trailing ')'
    \let\inlinechoice\inlinechoiceorig
\fi
}

\setchoicesquareX  % by default list-of-squares with onlineChoice in PDF

% Omdat multicols niet werkt in html: enkel in pdf  (in html zijn langere pagina's misschien ook minder storend)
\newenvironment{xmmulticols}[1][2]{
 \pdfOnly{\begin{multicols}{#1}}%
}{ \pdfOnly{\end{multicols}}}

%
% Te gebruiken in plaats van \section\subsection
%  (in een printstyle kan dan het level worden aangepast
%    naargelang \chapter vs \section style )
% 3/2021: DO NOT USE \xmsubsection !
\newcommand\xmsection\subsection
\newcommand\xmsubsection\subsubsection

% Aanpassen printversie
%  (hier gedefinieerd, zodat ze in xourse kunnen worden gezet/overschreven)
\providebool{parttoc}
\providebool{printpartfrontpage}
\providebool{printactivitytitle}
\providebool{printactivityqrcode}
\providebool{printactivityurl}
\providebool{printcontinuouspagenumbers}
\providebool{numberactivitiesbysubpart}
\providebool{addtitlenumber}
\providebool{addsectiontitlenumber}
\addtitlenumbertrue
\addsectiontitlenumbertrue

% The following three commands are hardcoded in xake, you can't create other commands like these, without adding them to xake as well
%  ( gebruikt in xourses om juiste soort titelpagina te krijgen voor verschillende ximera's )
\newcommand{\activitychapter}[2][]{
    {    
    \ifstrequal{#1}{notnumbered}{
        \addtitlenumberfalse
    }{}
    \typeout{ACTIVITYCHAPTER #2}   % logging
	\chapterstyle
	\activity{#2}
    }
}
\newcommand{\activitysection}[2][]{
    {
    \ifstrequal{#1}{notnumbered}{
        \addsectiontitlenumberfalse
    }{}
	\typeout{ACTIVITYSECTION #2}   % logging
	\sectionstyle
	\activity{#2}
    }
}
% Practices worden als activity getoond om de grote blokken te krijgen online
\newcommand{\practicesection}[2][]{
    {
    \ifstrequal{#1}{notnumbered}{
        \addsectiontitlenumberfalse
    }{}
    \typeout{PRACTICESECTION #2}   % logging
	\sectionstyle
	\activity{#2}
    }
}
\newcommand{\activitychapterlink}[3][]{
    {
    \ifstrequal{#1}{notnumbered}{
        \addtitlenumberfalse
    }{}
    \typeout{ACTIVITYCHAPTERLINK #3}   % logging
	\chapterstyle
	\activitylink[#1]{#2}{#3}
    }
}

\newcommand{\activitysectionlink}[3][]{
    {
    \ifstrequal{#1}{notnumbered}{
        \addsectiontitlenumberfalse
    }{}
    \typeout{ACTIVITYSECTIONLINK #3}   % logging
	\sectionstyle
	\activitylink[#1]{#2}{#3}
    }
}


% Commando om de printstyle toe te voegen in ximera's. Zorgt ervoor dat er geen problemen zijn als je de xourses compileert
% hack om onhandige relative paden in TeX te omzeilen
% should work both in xourse and ximera (pre-112022 only in ximera; thus obsoletes adhoc setup in xourses)
% loads global.sty if present (cfr global.css for online settings!)
% use global.sty to overwrite settings in printstyle.sty ...
\newcommand{\addPrintStyle}[1]{
\iftikzexport\else   % only in PDF
  \makeatletter
  \ifx\@onlypreamble\@notprerr\else   % ONLY if in tex-preamble   (and e.g. not when included from xourse)
    \typeout{Loading printstyle}   % logging
    \usepackage{#1/printstyle} % mag enkel geinclude worden als je die apart compileert
    \IfFileExists{#1/global.sty}{
        \typeout{Loading printstyle-folder #1/global.sty}   % logging
        \usepackage{#1/global}
        }{
        \typeout{Info: No extra #1/global.sty}   % logging
    }   % load global.sty if present
    \IfFileExists{global.sty}{
        \typeout{Loading local-folder global.sty (or TEXINPUTPATH..)}   % logging
        \usepackage{global}
    }{
        \typeout{Info: No folder/global.sty}   % logging
    }   % load global.sty if present
    \IfFileExists{\currfilebase.sty}
    {
        \typeout{Loading \currfilebase.sty}
        \input{\currfilebase.sty}
    }{
        \typeout{Info: No local \currfilebase.sty}
    }
    \fi
  \makeatother
\fi
}

%
%  
% references: Ximera heeft adhoc logica	 om online labels te doen werken over verschillende files heen
% met \hyperref kan de getoonde tekst toch worden opgegeven, in plaats van af te hangen van de label-text
\ifdefined\HCode
% Link to standard \labels, but give your own description
% Usage:  Volg \hyperref[my_very_verbose_label]{deze link} voor wat tijdverlies
%   (01/2020: Ximera-server aangepast om bij class reference-keeptext de link-text NIET te vervangen door de label-text !!!) 
\renewcommand{\hyperref}[2][]{\HCode{<a class="reference reference-keeptext" href="\##1">}#2\HCode{</a>}}
%
%  Link to specific targets  (not tested ?)
\renewcommand{\hypertarget}[1]{\HCode{<a class="ximera-label" id="#1"></a>}}
\renewcommand{\hyperlink}[2]{\HCode{<a class="reference reference-keeptext" href="\##1">}#2\HCode{</a>}}
\fi

% Mmm, quid English ... (for keyword #1 !) ?
\newcommand{\wikilink}[2]{
    \hyperlink{https://nl.wikipedia.org/wiki/#1}{#2}
    \pdfOnly{\footnote{See \url{https://nl.wikipedia.org/wiki/#1}}
    }
}

\renewcommand{\figurename}{Figuur}
\renewcommand{\tablename}{Tabel}

%
% Gedoe om verschillende versies van xourse/ximera te maken afhankelijk van settings
%
% default: versie met antwoorden
% handout: versie voor de studenten, zonder antwoorden/oplossingen
% full: met alles erop en eraan, dus geschikt voor auteurs en/of lesgevers  (bevat in de pdf ook de 'online-only' stukken!)
%
%
% verder kunnen ook opties/variabele worden gezet voor hints/auteurs/uitweidingen/ etc
%
% 'Full' versie
\newtoggle{showonline}
\ifdefined\HCode   % zet default showOnline
    \toggletrue{showonline} 
\else
    \togglefalse{showonline}
\fi

% Full versie   % deprecated: see infra
\newcommand{\printFull}{
    \hintstrue
    \handoutfalse
    \toggletrue{showonline} 
}

\ifdefined\shouldPrintFull   % deprecated: see infra
    \printFull
\fi



% Overschrijf onlineOnly  (zoals gedefinieerd in ximera.cls)
\ifhandout   % in handout: gebruik de oorspronkelijke ximera.cls implementatie  (is dit wel nodig/nuttig?)
\else
    \iftoggle{showonline}{%
        \ifdefined\HCode
          \RenewEnviron{onlineOnly}{\bgroup\BODY\egroup}   % showOnline, en we zijn  online, dus toon de tekst
        \else
          \RenewEnviron{onlineOnly}{\bgroup\color{red!50!black}\BODY\egroup}   % showOnline, maar we zijn toch niet online: kleur de tekst rood 
        \fi
    }{%
      \RenewEnviron{onlineOnly}{}  % geen showOnline
    }
\fi

% hack om na hoofding van definition/proposition/... als dan niet op een nieuwe lijn te starten
% soms is dat goed en mooi, en soms niet; en in HTML is het nu (2/2020) anders dan in pdf
% vandaar suggestie om 
%     \begin{definition}[Nieuw concept] \nl
% te gebruiken als je zeker een newline wil na de hoofdig en titel
% (in het bijzonder itemize zonder \nl is 'lelijk' ...)
\ifdefined\HCode
\newcommand{\nl}{}
\else
\newcommand{\nl}{\ \par} % newline (achter heading van definition etc.)
\fi


% \nl enkel in handoutmode (ihb voor \wordChoice, die dan typisch veeeel langer wordt)
\ifdefined\HCode
\providecommand{\handoutnl}{}
\else
\providecommand{\handoutnl}{%
\ifhandout%
  \nl%
\fi%
}
\fi

% Could potentially replace \pdfOnline/\begin{onlineOnly} : 
% Usage= \ifonline{Hallo surfer}{Hallo PDFlezer}
\providecommand{\ifonline}[2]%
{
\begin{onlineOnly}#1\end{onlineOnly}%
\pdfOnly{#2}
}%


%
% Maak optionele 'basic' en 'extended' versies van een activity
%  met environment basicOnly, basicSkip en extendedOnly
%
%  (
%   Dit werkt ENKEL in de PDF; de online versies tonen (minstens voorklopig) steeds 
%   het default geval met printbasicversion en printextendversion beide FALSE
%  )
%
\providebool{printbasicversion}
\providebool{printextendedversion}   % not properly implemented
\providebool{printfullversion}       % presumably print everything (debug/auteur)
%
% only set these in xourses, and BEFORE loading this preamble
%
%\newif\ifshowbasic     \showbasictrue        % use this line in xourse to show 'basic' sections
%\newif\ifshowextended  \showextendedtrue     % use this line in xourse to show 'extended' sections
%
%
%\ifbool{showbasic}
%      { \NewEnviron{basicOnly}{\BODY} }    % if yes: just print contents
%      { \NewEnviron{basicOnly}{}      }    % if no:  completely ignore contents
%
%\ifbool{showbasic}
%      { \NewEnviron{basicSkip}{}      }
%      { \NewEnviron{basicSkip}{\BODY} }
%

\ifbool{printextendedversion}
      { \NewEnviron{extendedOnly}{\BODY} }
      { \NewEnviron{extendedOnly}{}      }
      


\ifdefined\HCode    % in html: always print
      {\newenvironment*{basicOnly}{}{}}    % if yes: just print contents
      {\newenvironment*{basicSkip}{}{}}    % if yes: just print contents
\else

\ifbool{printbasicversion}
      {\newenvironment*{basicOnly}{}{}}    % if yes: just print contents
      {\NewEnviron{basicOnly}{}      }    % if no:  completely ignore contents

\ifbool{printbasicversion}
      {\NewEnviron{basicSkip}{}      }
      {\newenvironment*{basicSkip}{}{}}

\fi

\usepackage{float}
\usepackage[rightbars,color]{changebar}

% Full versie
\ifbool{printfullversion}{
    \hintstrue
    \handoutfalse
    \toggletrue{showonline}
    \printbasicversionfalse
    \cbcolor{red}
    \renewenvironment*{basicOnly}{\cbstart}{\cbend}
    \renewenvironment*{basicSkip}{\cbstart}{\cbend}
    \def\xmtoonprintopties{FULL}   % will be printed in footer
}
{}
      
%
% Evalueer \ifhints IN de environment
%  
%
%\RenewEnviron{hint}
%{
%\ifhandout
%\ifhints\else\setbox0\vbox\fi%everything in een emty box
%\bgroup 
%\stepcounter{hintLevel}
%\BODY
%\egroup\ignorespacesafterend
%\addtocounter{hintLevel}{-1}
%\else
%\ifhints
%\begin{trivlist}\item[\hskip \labelsep\small\slshape\bfseries Hint:\hspace{2ex}]
%\small\slshape
%\stepcounter{hintLevel}
%\BODY
%\end{trivlist}
%\addtocounter{hintLevel}{-1}
%\fi
%\fi
%}

% Onafhankelijk van \ifhandout ...? TO BE VERIFIED
\RenewEnviron{hint}
{
\ifhints
\begin{trivlist}\item[\hskip \labelsep\small\bfseries Hint:\hspace{2ex}]
\small%\slshape
\stepcounter{hintLevel}
\BODY
\end{trivlist}
\addtocounter{hintLevel}{-1}
\else
\iftikzexport   % anders worden de tikz tekeningen in hints niet gegenereerd ?
\setbox0\vbox\bgroup
\stepcounter{hintLevel}
\BODY
\egroup\ignorespacesafterend
\addtocounter{hintLevel}{-1}
\fi % ifhandout
\fi %ifhints
}

%
% \tab sets typewriter-tabs (e.g. to format questions)
% (Has no effect in HTML :-( ))
%
\usepackage{tabto}
\ifdefined\HCode
  \renewcommand{\tab}{\quad}    % otherwise dummy .png's are generated ...?
\fi


% Also redefined in  preamble to get correct styling 
% for tikz images for (\tikzexport)
%

\theoremstyle{definition} % Bold titels
\makeatletter
\let\proposition\relax
\let\c@proposition\relax
\let\endproposition\relax
\makeatother
\newtheorem{proposition}{Eigenschap}


%\instructornotesfalse

% logic with \ifhandoutin ximera.cls unclear;so overwrite ...
\makeatletter
\@ifundefined{ifinstructornotes}{%
  \newif\ifinstructornotes
  \instructornotesfalse
  \newenvironment{instructorNotes}{}{}
}{}
\makeatother
\ifinstructornotes
\else
\renewenvironment{instructorNotes}%
{%
    \setbox0\vbox\bgroup
}
{%
    \egroup
}
\fi

% \RedeclareMathOperator
% from https://tex.stackexchange.com/questions/175251/how-to-redefine-a-command-using-declaremathoperator
\makeatletter
\newcommand\RedeclareMathOperator{%
    \@ifstar{\def\rmo@s{m}\rmo@redeclare}{\def\rmo@s{o}\rmo@redeclare}%
}
% this is taken from \renew@command
\newcommand\rmo@redeclare[2]{%
    \begingroup \escapechar\m@ne\xdef\@gtempa{{\string#1}}\endgroup
    \expandafter\@ifundefined\@gtempa
    {\@latex@error{\noexpand#1undefined}\@ehc}%
    \relax
    \expandafter\rmo@declmathop\rmo@s{#1}{#2}}
% This is just \@declmathop without \@ifdefinable
\newcommand\rmo@declmathop[3]{%
    \DeclareRobustCommand{#2}{\qopname\newmcodes@#1{#3}}%
}
\@onlypreamble\RedeclareMathOperator
\makeatother


%
% Engelse vertaling, vooral in mathmode
%
% 1. Algemene procedure
%
\ifdefined\isEn
 \newcommand{\nlen}[2]{#2}
 \newcommand{\nlentext}[2]{\text{#2}}
 \newcommand{\nlentextbf}[2]{\textbf{#2}}
\else
 \newcommand{\nlen}[2]{#1}
 \newcommand{\nlentext}[2]{\text{#1}}
 \newcommand{\nlentextbf}[2]{\textbf{#1}}
\fi

%
% 2. Lijst van erg veel gebruikte uitdrukkingen
%

% Ja/Nee/Fout/Juits etc
%\newcommand{\TJa}{\nlentext{ Ja }{ and }}
%\newcommand{\TNee}{\nlentext{ Nee }{ No }}
%\newcommand{\TJuist}{\nlentext{ Juist }{ Correct }
%\newcommand{\TFout}{\nlentext{ Fout }{ Wrong }
\newcommand{\TWaar}{\nlentext{ Waar }{ True }}
\newcommand{\TOnwaar}{\nlentext{ Vals }{ False }}
% Korte bindwoorden en, of, dus, ...
\newcommand{\Ten}{\nlentext{ en }{ and }}
\newcommand{\Tof}{\nlentext{ of }{ or }}
\newcommand{\Tdus}{\nlentext{ dus }{ so }}
\newcommand{\Tendus}{\nlentext{ en dus }{ and thus }}
\newcommand{\Tvooralle}{\nlentext{ voor alle }{ for all }}
\newcommand{\Took}{\nlentext{ ook }{ also }}
\newcommand{\Tals}{\nlentext{ als }{ when }} %of if?
\newcommand{\Twant}{\nlentext{ want }{ as }}
\newcommand{\Tmaal}{\nlentext{ maal }{ times }}
\newcommand{\Toptellen}{\nlentext{ optellen }{ add }}
\newcommand{\Tde}{\nlentext{ de }{ the }}
\newcommand{\Thet}{\nlentext{ het }{ the }}
\newcommand{\Tis}{\nlentext{ is }{ is }} %zodat is in text staat in mathmode (geen italics)
\newcommand{\Tmet}{\nlentext{ met }{ where }} % in situaties e.g met p < n --> where p < n
\newcommand{\Tnooit}{\nlentext{ nooit }{ never }}
\newcommand{\Tmaar}{\nlentext{ maar }{ but }}
\newcommand{\Tniet}{\nlentext{ niet }{ not }}
\newcommand{\Tuit}{\nlentext{ uit }{ from }}
\newcommand{\Ttov}{\nlentext{ t.o.v. }{ w.r.t. }}
\newcommand{\Tzodat}{\nlentext{ zodat }{ such that }}
\newcommand{\Tdeth}{\nlentext{de }{th }}
\newcommand{\Tomdat}{\nlentext{omdat }{because }} 


%
% Overschrijf addhoc commando's
%
\ifdefined\isEn
\renewcommand{\pernot}{\overset{\mathrm{notation}}{=}}
\RedeclareMathOperator{\bld}{im}     % beeld
\RedeclareMathOperator{\graf}{graph}   % grafiek
\RedeclareMathOperator{\rico}{slope}   % richtingcoëfficient
\RedeclareMathOperator{\co}{co}       % coordinaat
\RedeclareMathOperator{\gr}{deg}       % graad

% Operators
\RedeclareMathOperator{\bgsin}{arcsin}
\RedeclareMathOperator{\bgcos}{arccos}
\RedeclareMathOperator{\bgtan}{arctan}
\RedeclareMathOperator{\bgcot}{arccot}
\RedeclareMathOperator{\bgsinh}{arcsinh}
\RedeclareMathOperator{\bgcosh}{arccosh}
\RedeclareMathOperator{\bgtanh}{arctanh}
\RedeclareMathOperator{\bgcoth}{arccoth}

\fi


% HACK: use 'oplossing' for 'explanation' ...
\let\explanation\relax
\let\endexplanation\relax
% \newenvironment{explanation}{\begin{oplossing}}{\end{oplossing}}
\newcounter{explanation}

\ifhandout%
    \NewEnviron{explanation}[1][toon]%
    {%
    \RenewEnviron{verbatim}{ \red{VERBATIM CONTENT MISSING IN THIS PDF}} %% \expandafter\verb|\BODY|}

    \ifthenelse{\equal{\detokenize{#1}}{\detokenize{toon}}}
    {
    \def\PH@Command{#1}% Use PH@Command to hold the content and be a target for "\expandafter" to expand once.

    \begin{trivlist}% Begin the trivlist to use formating of the "Feedback" label.
    \item[\hskip \labelsep\small\slshape\bfseries Explanation:% Format the "Feedback" label. Don't forget the space.
    %(\texttt{\detokenize\expandafter{\PH@Command}}):% Format (and detokenize) the condition for feedback to trigger
    \hspace{2ex}]\small%\slshape% Insert some space before the actual feedback given.
    \BODY
    \end{trivlist}
    }
    {  % \begin{feedback}[solution]   \BODY     \end{feedback}  }
    }
    }    
\else
% ONLY for HTML; xmoplossing is styled with css, and is not, and need not be a LaTeX environment
% THUS: it does NOT use feedback anymore ...
%    \NewEnviron{oplossing}{\begin{expandable}{xmoplossing}{\nlen{Toon uitwerking}{Show solution}}{\BODY}\end{expandable}}
    \newenvironment{explanation}[1][toon]
   {%
       \begin{expandable}{xmoplossing}{}
   }
   {%
   	   \end{expandable}
   } 
\fi

\title{Complex Numbers} \license{CC BY-SA-NC 4.0}

\begin{document}

\begin{abstract}
\end{abstract}
\maketitle

\section*{Complex Numbers}
The fact that the square of every real number is non-negative shows that the equation $x^{2} + 1 = 0$ has no real root; in other words, there is no real number $u$ such that $u^{2} = -1$. So the set of real numbers is inadequate for finding all roots of all polynomials. This kind of problem arises with other number systems as well. The set of integers contains no solution of the equation $3x + 2 = 0$, and the rational numbers had to be invented to solve such
equations. But the set of rational numbers is also incomplete because,
for example, it contains no root of the polynomial $x^{2} - 2$. Hence the real numbers were invented. In the same way, the set of
complex numbers was invented, which contains all real numbers together
with a root of the equation $x^{2} + 1 = 0$. However, the process ends here: the complex numbers have the property that \textit{every} polynomial with complex coefficients has a (complex) root. This fact is known as the fundamental theorem of algebra.

One pleasant aspect of the complex
numbers is that, whereas describing the real numbers in terms of the
rationals is a rather complicated business, the complex numbers are
quite easy to describe in terms of real numbers. Every \dfn{complex number} has the form
\begin{equation*}
a + bi
\end{equation*}
where $a$ and $b$ are real numbers, and $i$ is a root of the polynomial $x^{2} + 1$. Here $a$ and $b$ are called the \dfn{real part} and the \dfn{imaginary part}of the complex number, respectively. The real numbers are now regarded as special complex numbers of the form $a + 0i = a$, with zero imaginary part. The complex numbers of the form $0 + bi = bi$ with zero real part are called \dfn{pure imaginary} numbers. The complex number $i$ itself is called the \dfn{imaginary unit} and is distinguished by the fact that
\begin{equation*}
i^2 = -1
\end{equation*}
As the terms \textit{complex} and \textit{imaginary}
 suggest, these numbers met with some resistance when they were first
used. This has changed; now they are essential in science and
engineering as well as mathematics, and they are used extensively. The
names persist, however, and continue to be a bit misleading: These
numbers are no more ``\textit{complex}'' than the real numbers, and the number $i$ is no more ``\textit{imaginary}'' than $-1$.

Much as for polynomials, two complex numbers are declared to be \dfn{equal} if and only if they have the same real parts and the same imaginary parts. In symbols,
\begin{equation*}
a+bi = a^{\prime} + b^{\prime} i \quad \mbox{ if and only if } a = a^{\prime} \mbox{ and } b = b^{\prime}
\end{equation*}
The addition and subtraction of complex numbers is accomplished by adding and subtracting real and imaginary parts:
\begin{align*}
(a+bi) + (a^{\prime} + b^{\prime}i) & = (a + a^{\prime}) + (b + b^{\prime})i\\
(a+bi) - (a^{\prime} + b^{\prime}i) & = (a - a^{\prime}) + (b - b^{\prime})i
\end{align*}
This is analogous to these operations for linear polynomials $a + bx$ and $a^{\prime} + b^{\prime}x$, and the multiplication of complex numbers is also analogous with one difference: $i^{2} = -1$. The definition is
\begin{equation*}
(a+bi)(a^{\prime} + b^{\prime}i) = (a a^{\prime} - b b^{\prime}) + (a b^{\prime} + b a^{\prime})i
\end{equation*}
With these definitions of equality, addition, and multiplication, the complex numbers \textit{satisfy all the basic arithmetical axioms adhered to by the real numbers} (the verifications are omitted). One consequence of this is that they can be manipulated in the obvious fashion, except that $i^{2}$ is replaced by $-1$ wherever it occurs, and the rule for equality must be observed.

\begin{example}\label{ex:033865}
If $z = 2 - 3i$ and $w = -1 + i$, write each of the following in the form $a + bi$: $z + w$, $z - w$, $zw$, $\frac{1}{3}z$, and $z^{2}$.

\begin{explanation}
\begin{align*}
z+w & = (2-3i) + (-1+i) = (2-1) + (-3+1)i = 1-2i \\
z-w & = (2-3i) - (-1+i) = (2+1) + (-3-1)i = 3-4i \\
zw &= (2-3i)(-1+i) = (-2-3i^2) + (2+3)i = 1+5i \\
\frac{1}{3}z &= \frac{1}{3}(2-3i) = \frac{2}{3}-i \\
z^2 &=(2-3i)(2-3i) = (4+9i^2) + (-6-6)i = -5-12i
\end{align*}
\end{explanation}
\end{example}

\begin{example}\label{ex:033872}
Find all complex numbers $z$ such as that $z^{2} = i$.

\begin{explanation}
  Write $z = a + bi$; we must determine $a$ and $b$. Now $z^{2} = (a^{2} - b^{2}) + (2ab)i$, so the condition $z^{2} = i$ becomes
\begin{equation*}
(a^2 - b^2) + (2ab)i = 0+i
\end{equation*}
Equating real and imaginary parts, we find that $a^{2} = b^{2}$ and $2ab = 1$. The solution is $a = b = \pm \frac{1}{\sqrt{2}}$, so the complex numbers required are $z = \frac{1}{\sqrt{2}} + \frac{1}{\sqrt{2}}i$  and  $z = -\frac{1}{\sqrt{2}} - \frac{1}{\sqrt{2}}i$.
\end{explanation}
\end{example}

As for real numbers, it is possible to divide by every nonzero complex number $z$. That is, there exists a complex number $w$ such that $wz = 1$. As in the real case, this number $w$ is called the \dfn{inverse} of $z$ and is denoted by $z^{-1}$ or $\frac{1}{z}$. Moreover, if $z = a + bi$, the fact that $z \neq 0$ means that $a \neq 0$ or $b \neq 0$. Hence $a^{2} + b^{2} \neq 0$, and an explicit formula for the inverse is
\begin{equation*}
\frac{1}{z} = \frac{a}{a^2 + b^2} - \frac{b}{a^2+b^2}{i}
\end{equation*}
In actual calculations, the work is
facilitated by two useful notions: the conjugate and the absolute value
of a complex number. The next example illustrates the technique.

\begin{example}\label{ex:033897}
Write $\frac{3+2i}{2+5i}$ in the form $a + bi$.

\begin{explanation}
  Multiply top and bottom by the complex number $2 - 5i$ (obtained from the denominator by negating the imaginary part). The result is
\begin{equation*}
\frac{3+2i}{2+5i} = \frac{(2-5i)(3+2i)}{(2-5i)(2+5i)} = \frac{(6+10)+(4-15)i}{2^2-(5i)^2} = \frac{16}{29} - \frac{11}{29} i
\end{equation*}
Hence the simplified form is $\frac{16}{29} - \frac{11}{29} i $, as required.
\end{explanation}
\end{example}

The key to this technique is that the product $(2 - 5i)(2 + 5i) = 29$ in the denominator turned out to be a \textit{real} number. The situation in general leads to the following notation: If $z = a + bi$ is a complex number, the \dfn{conjugate} of $z$ is the complex number, denoted $\overline{z}$, given by
\begin{equation*}
\overline{z} = a-bi \quad \mbox{ where } z = a+bi
\end{equation*}
Hence $\overline{z}$ is obtained from $z$ by negating the imaginary part. Thus $\overline{(2+3i)} = 2-3i$ and  $\overline{(1-i)} = 1+i$. If we multiply $z = a + bi$ by $\overline{z}$, we obtain
\begin{equation*}
z \overline{z} = a^2 + b^2 \quad \mbox{ where } z = a+bi
\end{equation*}

The real number $a^{2} + b^{2}$ is always nonnegative, so we can state the following definition: The \dfn{absolute value} or \dfn{modulus} of a complex number $z = a + bi$, denoted by $|z|$, is the positive square root $\sqrt{a^2 + b^2}$; that is,
\begin{equation*}
|z| = \sqrt{a^2 + b^2} \quad \mbox{ where } z = a+bi
\end{equation*}
For example, $| 2-3i| = \sqrt{2^2 + (-3)^2} = \sqrt{13}$
 and $| 1+i| = \sqrt{1^2 + 1^2} = \sqrt{2}$.

Note that if a real number $a$ is viewed as the complex number $a + 0i$, its absolute value (as a complex number) is $|a| = \sqrt{a^2}$, which agrees with its absolute value as a \textit{real} number.

With these notions in hand, we can describe the technique applied in Example~\ref{ex:033897}  as follows: When converting a quotient $\frac{z}{w}$
 of complex numbers to the form $a + bi$, multiply top and bottom by the conjugate  $\overline{w}$ of the denominator.

The following list contains the most important properties of conjugates and absolute values. Throughout, $z$ and $w$ denote complex numbers.
\begin{equation*}
\begin{array}{llcll}
C1. & \overline{z \pm w} = \overline{z} \pm \overline{w} & \quad & C7. & \frac{1}{z} = \frac{1}{|z|^2}\overline{z} \\
C2. & \overline{zw} = \overline{z}~\overline{w} & \quad & C8. & |z| \geq 0 \mbox{ for all complex numbers } z \\
C3. & \overline{\left(\frac{z}{w}\right)} = \frac{\overline{z}}{\hspace{0.05em}\overline{w}\hspace{0.05em}} & \quad & C9. & |z| = 0 \mbox{ if and only if } z=0 \\
C4. & \overline{(\overline{z})} = z & \quad & C10. & |zw| = |z||w| \\
C5. & z \mbox{ is real if and only if } \overline{z} =z  & \quad & C11. & |\frac{z}{w}| = \frac{|z|}{|w|} \\
C6. & z\overline{z} = |z|^2 & \quad & C12. & |z+w| \leq |z|+|w| \mbox{ (\dfn{triangle inequality})} \\
\end{array}
\end{equation*}
All these properties (except property C12) can (and should) be verified by the reader for arbitrary complex numbers $z = a + bi$ and $w = c + di$. They are not independent; for example, property C10 follows from properties C2 and C6.

The triangle inequality, as its name
suggests, comes from a geometric representation of the complex numbers
analogous to identification of the real numbers with the points of a
line. The representation is achieved as follows:



Introduce a rectangular coordinate system in the plane, and identify the complex number $a + bi$ with the point $(a, b)$, as shown in the figure below.
\begin{center}
\begin{tikzpicture}[scale=1]
  \draw[<->] (-1,0)--(4,0);
  \draw[<->] (0,-1)--(0,3);
%\draw[line width=1pt,blue](0,0)--(3,2);
%\draw[line width=1pt,blue](0,0)--(3,-2);
\draw[line width=0.5pt, dashed](3,2)--(0,2);
\draw[line width=0.5pt, dashed](3,2)--(3,0);
\fill[blue] (3,2)node[above]{$(a,b)=a+bi$} circle (0.1cm);
\fill[black] (0,2)node[left]{$(0,b)=bi$} circle (0.1cm);
\fill[black] (3,0)node[below right]{$(a,0)=a$} circle (0.1cm);
%\fill[blue] (3,-2)node[below]{$(a,-b)=a-bi$} circle (0.1cm);
 \end{tikzpicture}
\end{center}

When this is done, the plane is called the \dfn{complex plane}. Note that the point $(a, 0)$ on the $x$ axis now represents the \textit{real} number $a = a + 0i$, and for this reason, the $x$ axis is called the \dfn{real axis}. Similarly, the $y$ axis is called the \dfn{imaginary axis}. The identification $(a, b) = a + bi$ of the geometric point $(a, b)$ and the complex number $a + bi$ will be used in what follows without comment. For example, the origin will be referred to as $0$.

This representation of the complex
numbers in the complex plane gives a useful way of describing the
absolute value and conjugate of a complex number $z = a + bi$. The absolute value $|z| = \sqrt{a^2+b^2}$
 is just the distance from $z$ to the origin. This makes properties C8 and C9 quite obvious. The conjugate $\overline{z} = a-bi$ of $z$ is just the reflection of $z$ in the real axis ($x$ axis), a fact that makes properties C4 and C5 clear.

 \begin{center}
\begin{tikzpicture}[scale=1]
  \draw[<->] (-1,0)--(4,0);
  \draw[<->] (0,-3)--(0,3);
\draw[line width=1pt,blue](0,0)--(3,2);
%\draw[line width=1pt,blue](0,0)--(3,-2);
\draw[line width=0.5pt, dashed](3,2)--(0,2);
\draw[line width=0.5pt, dashed](3,2)--(3,-2);
\fill[blue] (3,2)node[above]{$z=a+bi$} circle (0.1cm);
\fill[black] (0,2)node[left]{$bi$} circle (0.1cm);
\fill[black] (3,0)node[below right]{$a$} circle (0.1cm);
\fill[blue] (3,-2)node[below]{$\overline{z}=a-bi$} circle (0.1cm);
\fill[blue] (1.6,1.3)node[left]{$|z|$};
 \end{tikzpicture}
\end{center}

Given two complex numbers $z_{1} = a_{1} + b_{1}i = (a_{1}, b_{1})$ and $z_{2} = a_{2} + b_{2}i = (a_{2}, b_{2})$, the absolute value of their difference
\begin{equation*}
|z_1 - z_2| = \sqrt{(a_1-a_2)^2 + (b_1 - b_2)^2}
\end{equation*}
is just the distance between them. This gives the \dfn{complex distance formula}:

\begin{equation*}
|z_1 - z_2| \mbox{ is the distance between } z_1 \mbox{ and } z_2
\end{equation*}

This useful fact yields a simple verification of the triangle inequality, property C12. Suppose $z$ and $w$ are given complex numbers. Consider the triangle in the figure below whose vertices are $0$, $w$, and $z + w$. 

\begin{center}
\begin{tikzpicture}[scale=1]
  \draw[<->] (-2,0)--(3,0);
  \draw[<->] (0,-1)--(0,3);
%\draw[line width=1pt,blue](0,0)--(3,2);
%\draw[line width=1pt,blue](0,0)--(3,-2);
\draw[line width=0.5pt, blue](3,1)--(0,0);
\draw[line width=0.5pt, blue](-1,2)--(3,1);
\draw[line width=0.5pt, blue](-1,2)--(0,0);
\fill[blue] (3,1)node[above]{$w$} circle (0.1cm);
\fill[blue] (-1,2)node[above]{$z+w$} circle (0.1cm);
\fill[blue] (0,0)node[below right]{$0$} circle (0.1cm);
\fill[blue] (-0.5,0.5)node[above left]{$|z+w|$};
\fill[blue] (0.6,2)node[right]{$|(z+w)-w|=|z|$};
\fill[blue] (1.2,0.2)node[right]{$|w|$};
%\fill[blue] (3,-2)node[below]{$(a,-b)=a-bi$} circle (0.1cm);
 \end{tikzpicture}
\end{center}


The three sides have lengths $|z|$, $|w|$, and $|z + w|$ by the complex distance formula, so the inequality
\begin{equation*}
|z+w| \leq |z| + |w|
\end{equation*}
expresses the obvious geometric fact that the sum of the lengths of two
sides of a triangle is at least as great as the length of the third
side.

The representation of complex numbers
as points in the complex plane has another very useful property: It
enables us to give a geometric description of the sum and product of two
 complex numbers. To obtain the description for the sum, let
\begin{align*}
z & = a+bi = (a,b) \\
w & = c+di = (c,d)
\end{align*}

denote two complex numbers. We claim that the four points $0$, $z$, $w$, and $z + w$ form the vertices of a parallelogram. In fact, in the figure below, the lines from $0$ to $z$ and from $w$ to $z + w$ have slopes
\begin{equation*}
\frac{b-0}{a-0} = \frac{b}{a} \quad \mbox{ and } \quad \frac{(b+d)-d}{(a+c)-c} = \frac{b}{a}
\end{equation*}
respectively, so these lines are parallel. (If it happens that $a = 0$, then both these lines are vertical.) 

\begin{center}
\begin{tikzpicture}[scale=1]
  \draw[<->] (-1,0)--(4,0);
  \draw[<->] (0,-1)--(0,3);
\draw[line width=0.5pt,blue,dashed](1,2)--(4,3);
\draw[line width=0.5pt,blue, dashed](3, 1)--(4,3);
\draw[line width=0.5pt, blue](1,2)--(0,0);
\draw[line width=0.5pt, blue](0,0)--(3,1);
\fill[blue] (3,1)node[below right]{$w=(c,d)$} circle (0.1cm);
\fill[red] (4,3)node[above]{$z+w=(a+c, b+d)$} circle (0.1cm);
\fill[blue] (1,2)node[below right]{$z=(a, b)$} circle (0.1cm);
 \end{tikzpicture}
\end{center}


Similarly, the lines from $z$ to $z + w$ and from $0$ to $w$ are also parallel, so the figure with vertices $0$, $z$, $w$, and $z + w$ is indeed a parallelogram. Hence, the complex number $z + w$ can be obtained geometrically from $z$ and $w$ by \textit{completing} the parallelogram. This is sometimes called the \dfn{parallelogram law}
 of complex addition. Readers who have studied mechanics will recall
that velocities and accelerations add in the same way; in fact, these
are all special cases of \textit{vector} addition.

\subsection*{Polar Form}
The geometric description of what
happens when two complex numbers are multiplied is at least as elegant
as the parallelogram law of addition, but it requires that the complex
numbers be represented in polar form. Before discussing this, we pause
to recall the general definition of the trigonometric functions sine and
 cosine. An angle $\theta$ in the complex plane is in \dfn{standard position} if it is measured counterclockwise from the positive real axis as indicated in the figure below.



\begin{center}
\begin{tikzpicture}[scale=2]
\node[blue] at (0.85, 0.8)   (c) {$P$};
\node[] at (0.55, 0.25)   (b) {$\theta$};
\draw[<->] (-1.3,0)--(1.3,0);
  \draw[<->] (0,-1.3)--(0,1.3);
  \fill[blue] (0.7,0.7) circle (0.03cm); 
\fill[] (1, 0) circle (0.03cm);
\fill[] (0, 1) circle (0.03cm);
\fill[] (-1, 0) circle (0.03cm);
\fill[] (0, -1) circle (0.03cm);
   \draw[line width=1pt,blue](0,0)--(0.7,0.7);
    \draw (0.5,0) arc (0:45:0.5) ;
   \draw[dashed] (0,0) circle (1);
   \fill[black] (1,0)node[below right]{$1$};
   \fill[black] (-1,0)node[below left]{$-1$};
   \fill[black] (0,1)node[above right]{$i$};
   \fill[black] (0,-1)node[below right]{$-i$};
 \end{tikzpicture}
 \end{center}

 
 Rather than using degrees to measure angles, it is more natural to use
radian measure. This is defined as follows: The circle with its centre
at the origin and radius $1$ (called the \dfn{unit circle}) is drawn as in the above figure. It has circumference $2\pi$, and the \dfn{radian measure} of $\theta$ is the length of the arc on the unit circle counterclockwise from $1$ to the point $P$ on the unit circle determined by $\theta$. Hence $90^{\circ} = \frac{\pi}{2}$, $45^{\circ}  = \frac{\pi}{4}$, $180^{\circ}  = \pi$, and a full circle has the angle $360^{\circ}  = 2\pi$. Angles measured clockwise from $1$ are negative; for example, $-i$ corresponds to $-\frac{\pi}{2}$ (or to $\frac{3\pi}{2}$).

Consider an angle $\theta$ in the range $0 \leq \theta \leq \frac{\pi}{2}$. If $\theta$ is plotted in standard position as in the above figure, it determines a unique point $P$ on the unit circle, and $P$ has coordinates ($\cos \theta$, $\sin \theta$) by elementary trigonometry. However, \textit{any} angle $\theta$ (acute or not) determines a unique point on the unit circle, so we \textit{define} the \dfn{cosine} and \dfn{sine} of $\theta$ (written $\cos \theta$ and $\sin \theta$) to be the $x$ and $y$ coordinates of this point. For example, the points
\begin{equation*}
\begin{array}{llll}
1=(1,0) & i=(0,1) & -1=(-1,0) & -i=(0,-1)
\end{array}
\end{equation*}
plotted in the figure are determined by the angles $0$, $\frac{\pi}{2}$, $\pi$, $\frac{3\pi}{2}$, respectively. Hence
\begin{equation*}
\begin{array}{lllllll}
\cos 0 = 1 & \quad & \cos \frac{\pi}{2} = 0 & \quad & \cos \pi = -1 & \quad & \cos \frac{3\pi}{2} = 0\\
\sin 0 = 0 & & \sin \frac{\pi}{2} = 1 & & \sin \pi = 0 & &\sin \frac{3\pi}{2} = -1
\end{array}
\end{equation*}

Now we can describe the polar form of a complex number. Let $z = a + bi$ be a complex number, and write the absolute value of $z$ as
\begin{equation*}
r = |z| = \sqrt{a^2+b^2}
\end{equation*}

If $z \neq 0$, the angle $\theta$ shown in the figure below is called an \dfn{argument} of $z$ and is denoted
\begin{equation*}
\theta = \mbox{arg} z
\end{equation*}

\begin{center}
\begin{tikzpicture}[scale=1]
  \draw[<->] (-2,0)--(3,0);
  \draw[<->] (0,-1)--(0,3);
\draw[line width=0.5pt, blue](2.5,2)--(0,0);
\fill[blue] (2.5,2)node[above]{$z=(a,b)$} circle (0.1cm);
\fill[blue] (0.9,0.4)node[right]{$\theta=\mbox{arg}(z)$};
\fill[blue] (0.8,1)node[above]{$r=|z|$};
\draw[blue] (1,0) arc (0:38.66:1) ;
 \end{tikzpicture}
\end{center}

This angle is not unique ($\theta + 2\pi k$ would do as well for any \newline $k = 0, \pm 1, \pm 2, \dots$ ). However, there is only one argument $\theta$ in the range $-\pi < \theta \leq \pi$, and this is sometimes called the \dfn{principal argument} of $z$.

Referring to the figure below, we find that the real and imaginary parts $a$ and $b$ of $z$ are related to $r$ and $\theta$ by
\begin{align*}
a &= r \cos \theta \\
b &= r \sin \theta
\end{align*}

\begin{center}
\begin{tikzpicture}[scale=1]
  \draw[<->] (-2,0)--(3,0);
  \draw[<->] (0,-1)--(0,3);
%\draw[line width=1pt,blue](0,0)--(3,2);
%\draw[line width=1pt,blue](0,0)--(3,-2);
\draw[line width=0.5pt, blue](2.5,2)--(0,0);
\draw[line width=0.5pt, dashed](2.5,2)--(2.5,0);
\fill[blue] (2.5,2)node[above]{$z=(a,b)$} circle (0.1cm);
%\fill[blue] (-1,2)node[above]{$z+w$} circle (0.1cm);
%\fill[blue] (0,0)node[below right]{$0$} circle (0.1cm);
\fill[blue] (0.9,0.4)node[right]{$\theta$};
\fill[blue] (0.8,1)node[above]{$r=|z|$};
\fill[] (1.5,0)node[below]{$a=r\cos\theta$};
\fill[] (2.5,1)node[right]{$b=r\sin\theta$};
\draw[blue] (1,0) arc (0:38.66:1) ;
%\fill[blue] (3,-2)node[below]{$(a,-b)=a-bi$} circle (0.1cm);
 \end{tikzpicture}
\end{center}

Hence the complex number $z = a + bi$ has the form
\begin{equation*}
z = r(\cos \theta + i \sin \theta) \quad\mbox{where}\quad r = |z|\quad\mbox{and}\quad \theta = \mbox{arg}(z)
\end{equation*}
The combination $\cos \theta + i \sin \theta$ is so important that a special notation is used:
\begin{equation*}
e^{i\theta} = \cos \theta + i \sin \theta
\end{equation*}
is called \dfn{Euler's formula} after the great Swiss mathematician Leonhard Euler (1707--1783). With this notation, $z$ is written
\begin{equation*}
z = r e^{i \theta} \quad\mbox{where}\quad r = |z|\quad\mbox{and}\quad \theta = \mbox{arg}(z)
\end{equation*}
This is a \dfn{polar form} of the complex number $z$. Of course it is not unique, because the argument can be changed by adding a multiple of $2\pi$.

\begin{example}\label{ex:033987}
Write $z_{1} = -2 + 2i$ and $z_{2} = -i$ in polar form.

\begin{explanation}

The two numbers are plotted in the complex plane, as shown in the figure below. 

\begin{center}
\begin{tikzpicture}[scale=1]
  \draw[<->] (-2.5,0)--(2.5,0);
  \draw[<->] (0,-2.5)--(0,2.5);
\draw[line width=1pt, blue](-2,2)--(0,0);
\draw[line width=1pt, red](0,-1)--(0,0);
\fill[blue] (-2,2)node[above]{$z_1=-2+2i$} circle (0.1cm);
\fill[red] (0,-1)node[below right]{$z_2=-i$} circle (0.1cm);
\fill[blue] (0.6,1)node[right]{$\theta_1$};
\fill[red] (-0.6,-0.8)node[above]{$\theta_2$};
\draw[blue] (1,0) arc (0:135:1) ;
\draw[red] (0.5,0) arc (0:270:0.5) ;
 \end{tikzpicture}
\end{center}




The absolute values are
\begin{align*}
r_1 &= |-2 + 2i| = \sqrt{(-2)^2 + 2^2} = 2\sqrt{2}\\
r_2 &= |-i| = \sqrt{0^2 + (-1)^2} = 1
\end{align*}
By inspection, arguments of $z_{1}$ and $z_{2}$ are
\begin{align*}
\theta_1 &= \mbox{arg}(-2+2i) = \frac{3\pi}{4}\\
\theta_2 &= \mbox{arg}(-i) = \frac{3\pi}{2}
\end{align*}
The corresponding polar forms are $z_{1} = -2 + 2i = 2\sqrt{2} e^{3\pi i/4}$
 and $z_{2} = -i = e^{3\pi i/2}$. Of course, we could have taken the argument $-\frac{\pi}{2}$ for $z_{2}$ and obtained the polar form $z_{2} = e^{-\pi i/2}$.
\end{explanation}
\end{example}

In Euler's formula $e^{i\theta}= \cos \theta + i \sin \theta$, the number $e$ is the familiar constant $e = 2.71828\dots$  from calculus. The reason for using $e$ will not be given here; the reason why $\cos \theta + i \sin \theta$ is written as an \textit{exponential} function of $\theta$ is that the \dfn{law of exponents} holds:
\begin{equation*}
e^{i\theta} \cdot e^{i\phi} = e^{i (\theta + \phi)}
\end{equation*}
where $\theta$ and $\phi$ are any two angles. In fact, this is an immediate consequence of the addition identities for $\sin(\theta + \phi)$ and $\cos(\theta + \phi)$:
\newpage
\begin{align*}
e^{i\theta} e^{i\phi} &= (\cos \theta + i \sin \theta) (\cos \phi + i \sin \phi) \\
&= (\cos \theta \cos \phi - \sin \theta \sin \phi) + i (\cos \theta \sin \phi + \sin \theta \cos \phi) \\
&= \cos (\theta +  \phi) +i \sin (\theta + \phi) \\
& =e^{i (\theta + \phi)}
\end{align*}
This is analogous to the rule $e^{a}e^{b} = e^{a+b}$, which holds for real numbers $a$ and $b$, so it is not unnatural to use the exponential notation $e^{i\theta}$ for the expression $\cos \theta + i \sin \theta$. In fact, a whole theory exists wherein functions such as $e^{z}$, $\sin z$, and $\cos z$ are studied, where $z$ is a \textit{complex}
 variable. Many deep and beautiful theorems can be proved in this
theory, one of which is the so-called fundamental theorem of algebra
mentioned later (Theorem~\ref{thm:034196}). We shall not pursue this here.

The geometric description of the multiplication of two complex numbers follows from the law of exponents.

\begin{theorem}\label{th:034029}
If $z_{1} = r_{1}e^{i{\theta}_1}$ and $z_{2} = r_{2}e^{i{\theta}_2}$ are complex numbers in polar form, then
\begin{equation*}
z_1z_2 = r_1r_2e^{i (\theta_1 + \theta_2)}
\end{equation*}
\end{theorem}

In other words, to multiply two complex
 numbers, simply multiply the absolute values and add the arguments.
This simplifies calculations considerably, particularly when we observe
that it is valid for \textit{any} arguments $\theta_{1}$ and $\theta_{2}$.

\begin{example}\label{ex:034047}
Multiply $(1-i)(1+\sqrt{3}i)$ in two ways.

\begin{explanation}
We have $|1 - i| = \sqrt{2}$ and $|1 + \sqrt{3}i| = 2$.  Consider the figure below.

\begin{center}
\begin{tikzpicture}[scale=1.5]
  \draw[<->] (-0.5,0)--(3,0);
  \draw[<->] (0,-1.5)--(0,2);
\draw[line width=1pt, blue](1,1.73)--(0,0);
\draw[line width=1pt, red](1,-1)--(0,0);
\draw[line width=0.5pt, dashed](2.73,0.73)--(0,0);
\fill[blue] (1,1.73)node[above right]{$1+\sqrt{3}i$} circle (0.05cm);
\fill[red] (1,-1)node[below right]{$1-i$} circle (0.05cm);
\fill[] (2.73,0.73)node[right]{$(1-i)(1+\sqrt{3}i)$} circle (0.05cm);
\fill[blue] (0.6,0.5)node[right]{$\pi/3$};
\fill[red] (0.8,-0.5)node[above]{$-\pi/4$};
\fill[] (1.25,0.2)node[right]{$\pi/12$};
\draw[blue] (0.75,0) arc (0:60:0.75) ;
\draw[red] (0.5,0) arc (0:-45:0.5) ;
\draw[] (1.25,0) arc (0:15:1.25) ;
 \end{tikzpicture}
\end{center}

We have,
\begin{align*}
& 1-i = \sqrt{2} e^{-i\pi /4} \\
& 1+ \sqrt{3}i = 2e^{i\pi /3}
\end{align*}

Hence, by Theorem~\ref{th:034029},
\begin{align*}
(1-i)(1+\sqrt{3}i) &= (\sqrt{2} e^{-i\pi /4})(2e^{i\pi /3}) \\
&= 2\sqrt{2} e^{i(-\pi/4 + \pi/3)} \\
&= 2 \sqrt{2} e^{i\pi/12}
\end{align*}

This gives the required product in polar form. Of course, direct multiplication gives $(1 - i)(1 + \sqrt{3}i) = (\sqrt{3} + 1) + (\sqrt{3} - 1)i$. Hence, equating real and imaginary parts gives the formulas $\cos (\frac{\pi}{12}) = \frac{\sqrt{3}+1}{2\sqrt{2}}$ and $\sin (\frac{\pi}{12}) = \frac{\sqrt{3}-1}{2\sqrt{2}}$.
\end{explanation}
\end{example}

\subsection*{Roots of Unity}

If a complex number $z = re^{i\theta}$ is given in polar form, the powers assume a particularly simple form. In fact, $z^{2} = (re^{i\theta})(re^{i\theta}) = r^{2}e^{2i\theta}$, $z^{3} = z^{2} \cdot z = (r^{2}e^{2i\theta})(re^{i\theta}) = r^{3}e^{3i\theta}$, and so on. Continuing in this way, it follows by induction that the following theorem holds for any positive integer $n$. The name honors Abraham De Moivre (1667--1754).


\begin{theorem}[De Moivre's Theorem]\label{th:034080}
If $\theta$ is any angle, then $(e^{i\theta})^{n} = e^{in\theta}$ holds for all integers $n$.
\end{theorem}

\begin{proof}
The case $n > 0$ has been discussed, and the reader can verify the result for $n = 0$. To derive it for $n < 0$, first observe that
\begin{equation*}
\mbox{if } \quad z = re^{i\theta}\neq 0 \quad \mbox{ then } \quad z^{-1} = \frac{1}{r}~e^{-i\theta}
\end{equation*}
In fact, $(re^{i\theta})(\frac{1}{r} e^{-i\theta}) = 1e^{i0} = 1$ by the multiplication rule. Now assume that $n$ is negative and write it as $n = -m$, $m > 0$. Then
\begin{equation*}
(re^{i\theta})^n = [(re^{i\theta})^{-1}]^m = (\frac{1}{r}~e^{-i\theta})^m = r^{-m} e^{i(-m\theta)}=r^ne^{in\theta}
\end{equation*}
If $r = 1$, this is De Moivre's theorem for negative $n$.
\end{proof}

\begin{example}\label{ex:034096}
Verify that $(-1+\sqrt{3}i)^3 = 8$.

\begin{explanation}
  We have $| -1 + \sqrt{3}i| =2$, so $-1 + \sqrt{3}i = 2e^{2\pi i /3}$ (see the figure below). 
  
\begin{center}
\begin{tikzpicture}[scale=1.5]
  \draw[<->] (-1,0)--(1,0);
  \draw[<->] (0,-0.5)--(0,2);
\draw[line width=1pt, blue](-1,1.73)--(0,0);
\fill[blue] (-1,1.73)node[above]{$-1+\sqrt{3}i$} circle (0.05cm);
\fill[blue] (0.1,0.4)node[right]{$2\pi/3$};
\draw[blue] (0.25,0) arc (0:120:0.25) ;
 \end{tikzpicture}
\end{center}  
  
  
  
De Moivre's theorem gives
\begin{equation*}
(-1+\sqrt{3}i)^3 = (2e^{2\pi i /3})^3 = 8e^{3(2\pi i /3)} = 8e^{2\pi i} = 8
\end{equation*}
\end{explanation}
\end{example}

De Moivre's theorem can be used to find $n$th roots of complex numbers where $n$ is positive. The next example illustrates this technique.


\begin{example}\label{ex:034107}
Find the cube roots of unity; that is, find all complex numbers $z$ such that $z^{3} = 1$.

\begin{explanation}
  First write $z = re^{i\theta}$ and $1 = 1e^{i0}$ in polar form. We must use the condition $z^{3} = 1$ to determine $r$ and $\theta$. Because $z^{3} = r^{3}e^{3i\theta}$ by De Moivre's theorem, this requirement becomes
\begin{equation*}
r^3 e^{3i\theta} = 1 e^{0i}
\end{equation*}
These two complex numbers are equal, so
 their absolute values must be equal and the arguments must either be
equal or differ by an integral multiple of $2\pi$:
\begin{align*}
r^3 & = 1 \\
3 \theta &= 0 +2k\pi, \quad k \mbox{ some integer}
\end{align*}
Because $r$ is real and positive, the condition $r^{3} = 1$ implies that $r = 1$. However,
\begin{equation*}
\theta = \frac{2k\pi}{3}, \quad k \mbox{ some integer}
\end{equation*}

seems at first glance to yield infinitely many different angles for $z$. However, choosing $k = 0, 1, 2$ gives three possible arguments $\theta$ (where $0 \leq \theta < 2\pi$), and the corresponding roots are
\begin{align*}
1e^{0i} & = 1 \\
1e^{2\pi i/3} & = -\frac{1}{2} + \frac{\sqrt{3}}{2} i \\
1e^{4\pi i/3} & = -\frac{1}{2} - \frac{\sqrt{3}}{2} i
\end{align*}
These are displayed in the figure below. 

\begin{center}
\begin{tikzpicture}[scale=1.5]
  \draw[<->] (-1.5,0)--(1.5,0);
  \draw[<->] (0,-1.5)--(0,1.5);
\draw[line width=1pt, blue](-1/2,1.73/2)--(0,0);
\draw[line width=1pt, red](-1/2,-1.73/2)--(0,0);
\draw[line width=2pt](1,0)--(0,0);
\fill[blue] (-1/2,1.73/2)node[above left]{$-\frac{1}{2}+\frac{\sqrt{3}}{2}i$} circle (0.05cm);
\fill[red] (-1/2,-1.73/2)node[below left]{$-\frac{1}{2}-\frac{\sqrt{3}}{2}i$} circle (0.05cm);
\fill[] (1,0)node[above right]{$1$} circle (0.05cm);
\fill[blue] (0.1,0.6)node[right]{$2\pi/3$};
\fill[red] (-0.15,-0.5)node[above left]{$4\pi/3$};
\draw[blue] (0.5,0) arc (0:120:0.5) ;
\draw[red] (0.25,0) arc (0:240:0.25) ;
\draw[dashed] (0,0) circle (1);
 \end{tikzpicture}
\end{center}

All other values of $k$
 yield values of $\theta$ that differ from one of these by a multiple of $2\pi$---and
 so do not give new roots. Hence we have found all the roots.
\end{explanation}
\end{example}

The same type of calculation gives all complex $n$\dfn{th roots of unity}; that is, all complex numbers $z$ such that $z^n = 1$. As before, write $1 = 1e^{0i}$ and
\begin{equation*}
z = re^{i\theta}
\end{equation*}
in polar form. Then $z^n = 1$ takes the form
\begin{equation*}
r^ne^{ni\theta} = 1e^{0i}
\end{equation*}
using De Moivre's theorem. Comparing absolute values and arguments yields
\begin{align*}
r^n &= 1 \\
n\theta & = 0 + 2k\pi, \quad k \mbox{ some integer}
\end{align*}
Hence $r = 1$, and the $n$ values
\begin{equation*}
\theta = \frac{2k\pi}{n}, \quad k=0, 1, 2, \dots, n-1
\end{equation*}
of $\theta$ all lie in the range $0 \leq \theta < 2\pi$. As in Example~\ref{ex:034107}, \textit{every} choice of $k$ yields a value of $\theta$ that differs from one of these by a multiple of $2\pi$, so these give the arguments of \textit{all} the possible roots.


\begin{theorem}[$n$th Roots of Unity]\label{th:034138}
If $n \geq 1$ is an integer, the $n$th roots of unity (that is, the solutions to $z^n = 1$) are given by
\begin{equation*}
z = e^{2\pi ki/n}, \quad k = 0, 1, 2, \dots, n-1
\end{equation*}
\end{theorem}

The $n$th roots of unity can be found geometrically as the points on the unit circle that cut the circle into $n$ equal sectors, starting at $1$. The case $n = 5$ is shown in the figure below, where the five fifth roots of unity are plotted.

\begin{center}
\begin{tikzpicture}[scale=1.5]
  \draw[<->] (-1.5,0)--(1.5,0);
  \draw[<->] (0,-1.5)--(0,1.5);
\draw[line width=1pt, blue](0.31,0.95)--(0,0);
\draw[line width=1pt, blue](0.31,-0.95)--(0,0);
\draw[line width=1pt, blue](-0.81,-0.59)--(0,0);
\draw[line width=1pt, blue](-0.81,0.59)--(0,0);
\draw[line width=2pt, blue](1,0)--(0,0);
\fill[blue] (0.31,0.95)node[above right]{$e^{2\pi i/5}$} circle (0.05cm);
\fill[blue] (0.31,-0.95)node[below right]{$e^{8\pi i/5}$} circle (0.05cm);
\fill[blue] (1,0)node[above right]{$1=e^{0i}$} circle (0.05cm);
\fill[blue] (-0.81,0.59)node[above left]{$e^{4\pi i/5}$} circle (0.05cm);
\fill[blue] (-0.81,-0.59)node[below left]{$e^{6\pi i/5}$} circle (0.05cm);
\draw[dashed] (0,0) circle (1);
 \end{tikzpicture}
\end{center}

The method just used to find the $n$th roots of unity works equally well to find the $n$th roots of any complex number in polar form. We give one example.

\begin{example}\label{ex:034148}
Find the fourth roots of $\sqrt{2} + \sqrt{2}i$.

\begin{explanation}
  First write $\sqrt{2} + \sqrt{2}i = 2e^{\pi i/4}$ in polar form. If $z = re^{i\theta}$ satisfies $z^{4} = \sqrt{2} + \sqrt{2}i$, then De Moivre's theorem gives
\begin{equation*}
r^4e^{i(4\theta)} = 2e^{\pi i/4}
\end{equation*}
Hence $r^{4} = 2$ and $4\theta = \frac{\pi}{4} + 2k\pi$, $k$ an integer. We obtain four distinct roots (and hence all) by
\begin{equation*}
r = \sqrt[4]{2}, \quad \theta = \frac{\pi}{16} = \frac{2k\pi}{16}, k=0, 1, 2, 3
\end{equation*}
Thus the four roots are
\begin{equation*}
\sqrt[4]{2} e^{\pi i/16} \quad  \sqrt[4]{2} e^{9\pi i/16} \quad \sqrt[4]{2} e^{17 \pi i/16} \quad \sqrt[4]{2} e^{25\pi i/16}
\end{equation*}
Of course, reducing these roots to the form $a + bi$ would require the computation of $\sqrt[4]{2}$
 and the sine and cosine of the various angles.
\end{explanation}
\end{example}

An expression of the form $ax^{2} + bx + c$, where the coefficients $a \neq 0$, $b$, and $c$ are real numbers, is called a \dfn{real quadratic}. A complex number $u$ is called a \dfn{root} of the quadratic if $au^{2} + bu + c = 0$. The roots are given by the famous \dfn{quadratic formula}:
\begin{equation*}
u = \frac{-b \pm \sqrt{b^2 - 4ac}}{2a}
\end{equation*}
The quantity $d = b^{2} - 4ac$ is called the \dfn{discriminant} of the quadratic $ax^{2} + bx + c$, and there is no real root if and only if $d < 0$. In this case the quadratic is said to be \dfn{irreducible}. Moreover, the fact that $d < 0$ means that $\sqrt{d} = i\sqrt{|d|}$, so the two (complex) roots are conjugates of each other:
\begin{equation*}
u = \frac{1}{2a}(-b+i\sqrt{|d|}) \quad \mbox{ and } \quad \overline{u} = \frac{1}{2a}(-b-i\sqrt{|d|})
\end{equation*}
The converse of this is true too: Given any nonreal complex number $u$, then $u$ and $\overline{u}$
 are the roots of some real irreducible quadratic. Indeed, the quadratic
\begin{equation*}
x^2 - (u + \overline{u})x + u \overline{u} = (x-u)(x-\overline{u})
\end{equation*}
has real coefficients ($u\overline{u} = |u|^{2}$ and $u + \overline{u}$
 is twice the real part of $u$) and so is irreducible because its roots $u$ and $\overline{u}$
 are not real.

\begin{example}\label{ex:034182}
Find a real irreducible quadratic with $u = 3 - 4i$ as a root.

\begin{explanation}
  We have $u + \overline{u} = 6$ and $|u|^{2} = 25$, so $x^{2} - 6x + 25$ is irreducible with $u$ and $\overline{u} = 3 + 4i$ as roots.
\end{explanation}
\end{example}

\subsection*{Fundamental Theorem of Algebra}

As we mentioned earlier, the complex
numbers are the culmination of a long search by mathematicians to find a
 set of numbers large enough to contain a root of every polynomial. The
fact that the complex numbers have this property was first proved by
Gauss in 1797 when he was 20 years old. The proof is omitted.

\begin{theorem}[Fundamental Theorem of Algebra]{th:034196}
Every polynomial of positive degree with complex coefficients has a complex root.
\end{theorem}

If $f(x)$ is a polynomial with complex coefficients, and if $u_{1}$ is a root, then the Factor Theorem (see \href{https://www.stitz-zeager.com/}{Precalculus} by Stitz-Zeager, for instance) asserts that
\begin{equation*}
f(x) = (x-u_1)g(x)
\end{equation*}
where $g(x)$ is a polynomial with complex coefficients and with degree one less than the degree of $f(x)$. Suppose that $u_{2}$ is a root of $g(x)$, again by the fundamental theorem. Then $g(x) = (x - u_{2})h(x)$, so
\begin{equation*}
f(x) = (x-u_1)(x-u_2)h(x)
\end{equation*}
This process continues until the last polynomial to appear is linear. Thus $f(x)$ has been expressed as a product of linear factors. The last of these factors can be written in the form $u(x - u_{n})$, where $u$ and $u_{n}$ are complex (verify this), so the fundamental theorem takes the following form.


\begin{theorem}\label{th:034210}
Every complex polynomial $f(x)$ of degree $n \geq 1$ has the form
\begin{equation*}
f(x) = u(x-u_1)(x-u_2)\cdots (x-u_n)
\end{equation*}
where $u, u_{1}, \dots, u_{n}$ are complex numbers and $u \neq 0$. The numbers $u_{1}, u_{2}, \dots, u_{n}$ are the roots of $f(x)$ (and need not all be distinct), and $u$ is the coefficient of $x^{n}$.
\end{theorem}

This form of the fundamental theorem, when applied to a polynomial $f(x)$ with \textit{real} coefficients, can be used to deduce the following result.

\begin{theorem}\label{th:034221}
Every polynomial $f(x)$ of positive
degree with real coefficients can be factored as a product of linear and
 irreducible quadratic factors.
\end{theorem}

In fact, suppose $f(x)$ has the form
\begin{equation*}
f(x) = a_nx^n + a_{n-1}x^{n-1} + \ldots + a_1x + a_0
\end{equation*}
where the coefficients $a_{i}$ are real. If $u$ is a complex root of $f(x)$, then we claim first that $\overline{u}$
 is also a root. In fact, we have $f(u) = 0$, so
\begin{align*}
0 = \overline{0} = \overline{f(u)} & = \overline{a_nu^n + a_{n-1}u^{n-1} + \ldots + a_1u + a_0 } \\
& = \overline{a_nu^n} + \overline{a_{n-1}u^{n-1}} + \ldots + \overline{a_1u} + \overline{a_0 } \\
& = \overline{a}_n\overline{u}^n + \overline{a}_{n-1}\overline{u}^{n-1} + \ldots + \overline{a}_1\overline{u} + \overline{a}_0 \\
& = a_n\overline{u}^n + a_{n-1}\overline{u}^{n-1} + \ldots + a_1\overline{u} + a_0 \\
&= f(\overline{u})
\end{align*}
where $\overline{a}_i = a_i$
 for each $i$ because the coefficients $a_{i}$ are real. Thus if $u$ is a root of $f(x)$, so is its conjugate $\overline{u}$. Of course some of the roots of $f(x)$ may be real (and so equal their conjugates), but the nonreal roots come in pairs, $u$ and $\overline{u}$. By Theorem~\ref{thm:034221}, we can thus write $f(x)$ as a product:
\begin{equation}\label{eq:complexproduct}
f(x) = a_n(x-r_1)\cdots(x-r_k)(x-u_1)(x-\overline{u}_1)\cdots (x-u_m)(x-\overline{u}_m)
\end{equation}
where $a_{n}$ is the coefficient of $x^{n}$ in $f(x)$; $r_{1}, r_{2}, \dots , r_{k}$ are the real roots; and $u_{1}, \overline{u}_{1}, u_{2}, \overline{u}_{2}, \dots , u_{m}, \overline{u}_{m}$ are the nonreal roots. But the product
\begin{equation*}
(x-u_j)(x-\overline{u}_j) = x^2 - (u_j + \overline{u}_j)x +(u_j \overline{u}_j)
\end{equation*}
is a real irreducible quadratic for each $j$ (see the discussion preceding Example~\ref{ex:034182}). Hence (\ref{eq:complexproduct}) shows that $f(x)$ is a product of linear and irreducible quadratic factors, each with real coefficients. This is the conclusion in Theorem~\ref{th:034221}.


\section*{Practice Problems}

\begin{problem}\label{prb:A.1}
Solve each of the following for the real number $x$.

\begin{enumerate}
\item $x-4i = (2-i)^2$
$$x=\answer{3}$$
\item $(2+xi)(3-2i) = 12+5i$
\item $(2+xi)^2=4$
$$x=\answer{1} \mbox{ or } x=\answer{-1}$$
\item $(2+xi)(2-xi)=5$
\end{enumerate}
\end{problem}

\begin{problem}\label{prb:A.2}
Convert each of the following to the form $a + bi$.

\begin{enumerate}
\item (challenge problem) $(2-3i)-2(2-3i)+9$
\item (challenge problem)  $(3-2i)(1+i)+|3+4i|$
$$\answer{10} + i$$
\item $\frac{1+i}{2-3i} + \frac{1-i}{-2+3i}$
\item $\frac{3-2i}{1-i} + \frac{3-7i}{2-3i}$
$$\frac{\answer{11}}{\answer{26}} + \frac{\answer{23}}{\answer{26}}i$$
\item $i^{131}$
\item $(2 - i)^{3}$
$$\answer{2} + \answer{-11}i$$
\item $(1 + i)^{4}$
$$\answer{8} +\answer{-6}i$$
\item $(1 - i)^{2}(2 + i)^{2}$
\item (challenge problem) $\frac{3\sqrt{3}-i}{\sqrt{3}+i} + \frac{\sqrt{3}+7i}{\sqrt{3}-i}$
\end{enumerate}
\end{problem}

\begin{problem}\label{prb:A.3}
In each case, find the complex number $z$.

\begin{enumerate}
\item $ iz - (1 + i)^{2} = 3 - i$
\item (challenge problem) $(i + z) - 3i(2 - z) = iz + 1$ 

$$\frac{\answer{11}}{\answer{5}} + \frac{\answer{3}}{\answer{5}}i$$

\item $z^{2} = -i$
\item $z^{2} = 3 - 4i$
$$\pm(\answer{2}-i)$$
\item $z(1+i) = \overline{z} + (3+2i)$
\item (challenge problem) $z(2-i) = (\overline{z}+1)(1+i)$

Click the arrow to see the answer.
\begin{expandable}
$$ 1 + i$$
\end{expandable}
\end{enumerate}
\end{problem}

\begin{problem}\label{prb:A.4}
In each case, find the roots of the real quadratic equation.

\begin{enumerate}
\item  $x^{2} - 2x + 3 = 0$
\item  $x^{2} - x + 1 = 0$

Click the arrow to see the answer.
\begin{expandable}
$$ \frac{1}{2} \pm \frac{\sqrt{3}}{2}i$$
\end{expandable}

\item $3x^{2} - 4x + 2 = 0$
\item  $2x^{2} - 5x + 2 = 0$

Click the arrow to see the answer.
\begin{expandable}
$$ 2, \frac{1}{2}$$
\end{expandable}
\end{enumerate}
\end{problem}

\begin{problem}\label{prb:A.5}
Find all numbers $x$ in each case.

\begin{enumerate}
\item $x^{3} = 8$
\item $x^{3} = -8$

Click the arrow to see the answer.
\begin{expandable}
$$ -2, 1 \pm \sqrt{3}i$$
\end{expandable}

\item  $x^{4} = 16$
\item  $x^{4} = 64$

Click the arrow to see the answer.
\begin{expandable}
$$ \pm 2\sqrt{2}, \pm 2\sqrt{i}$$
\end{expandable}
\end{enumerate}
\end{problem}

\begin{problem}\label{prb:A.6}
In each case, find a real quadratic with $u$ as a root, and find the other root.

\begin{enumerate}
\item  $u = 1 + i$
\item  $u = 2 - 3i$

Click the arrow to see the answer.
\begin{expandable}
$$x^{2} - 4x + 13; 2 + 3i$$
\end{expandable}

\item $u = -i$
\item $u = 3 - 4i$

Click the arrow to see the answer.
\begin{expandable}
$$ x^{2} - 6x + 25; 3 + 4i$$
\end{expandable}
\end{enumerate}
\end{problem}

\begin{problem}\label{prb:A.7}
Find the roots of $x^{2} - 2\cos \theta x + 1 = 0$, $\theta$ any angle.
\end{problem}

\begin{problem}\label{prb:A.8}
Find a real polynomial of degree $4$ with $2 - i$ and $3 - 2i$ as roots.

Click the arrow to see the answer.
\begin{expandable}
$$x^{4} - 10x^{3} + 42x^{2} - 82x + 65$$
\end{expandable}
\end{problem}

\begin{problem}\label{prb:A.9}
Let $\mbox{re }z$ and $\mbox{im }z$ denote, respectively, the real and imaginary parts of $z$. Show that:

\begin{enumerate}
\item $\mbox{im}(iz) = \mbox{re }z$
\item $\mbox{re}(iz) = -\mbox{im }z$
\item $z + \overline{z} = 2 \mbox{re}z$
\item $z - \overline{z} = 2i \mbox{im} z$
\item (challenge problem) $\mbox{re}(z + w) = \mbox{re }z + \mbox{re }w$, and $\mbox{re}(tz) = t \cdot \mbox{re }z$ if $t$ is real
\item (challenge problem) $\mbox{im}(z + w) = \mbox{im }z + \mbox{im }w$, and $\mbox{im}(tz) = t \cdot \mbox{im }z$ if $t$ is real
\end{enumerate}
\end{problem}

\begin{problem}\label{prb:A.10}
In each case, show that $u$ is a root of the quadratic equation, and find the other root.

\begin{enumerate}
\item (challenge problem) $x^{2} - 3ix + (-3 + i) = 0$; $u = 1 + i$
\item (challenge problem) $x^{2} + ix - (4 - 2i) = 0$; $u = -2$
\begin{hint}
$$(-2)^{2} - 2i - (4 - 2i) = 0; 2 - i$$
\end{hint}
\item (challenge problem) $x^{2} - (3 - 2i)x + (5 - i) = 0$; $u = 2 - 3i$
\item (challenge problem) $x^{2} + 3(1 - i)x - 5i = 0$; $u = -2 + i$
\begin{hint}
$$(-2 + i)^{2} + 3(1 - i)(-1 + 2i) - 5i = 0; -1 + 2i$$
\end{hint}
\end{enumerate}
\end{problem}

\begin{problem}\label{prb:A.11}
Find the roots of each of the following complex quadratic equations.

\begin{enumerate}
\item $x^{2} + 2x + (1 + i) = 0$
\item  $x^{2} - x + (1 - i) = 0$

Click the arrow to see the answer.
\begin{expandable}
$$-i, 1 + i$$
\end{expandable}

\item (challenge problem) $x^{2} - (2 - i)x + (3 - i) = 0$
\item (challenge problem) $x^{2} - 3(1 - i)x - 5i = 0$

Click the arrow to see the answer.
\begin{expandable}
$$2 - i, 1 - 2i$$
\end{expandable}
\end{enumerate}
\end{problem}

\begin{problem}\label{prb:A.12}
In each case, describe the graph of the equation (where $z$ denotes a complex number).

\begin{enumerate}
\item $|z| = 1$
\item $|z - 1| = 2$

Click the arrow to see the answer.
\begin{expandable}
Circle, centre at $1$, radius $2$
\end{expandable}

\item $z = i \overline{z}$
\item $z = -\overline{z}$

Click the arrow to see the answer.
\begin{expandable}
Imaginary axis
\end{expandable}

\item $z = |z|$
\item $\mbox{im }z = m \cdot \mbox{re }z$, $m$ a real number

Click the arrow to see the answer.
\begin{expandable}
 Line $y = mx$
\end{expandable}

\end{enumerate}
\end{problem}

\begin{problem}\label{prb:A.13}
\begin{enumerate}[label={\alph*.}]
\item Verify $|zw| = |z||w|$ directly for $z = a + bi$ and $w = c + di$.

\item Deduce (a) from properties C2 and C6.

\end{enumerate}
\end{problem}

\begin{problem}\label{prb:A.14}
Prove that $|z+w| = |z|^2 + |w|^2 + w\overline{z} + \overline{w}z$
 for all complex numbers $w$ and $z$.
\end{problem}

\begin{problem}\label{prb:A.15}
If $zw$ is real and $z \neq 0$, show that $w = a \overline{z}$
 for some real number $a$.
\end{problem}

\begin{problem}\label{prb:A.16}
If $zw = \overline{z}v$
 and $z \neq 0$, show that $w = uv$ for some $u$ in $\mathbb{C}$  with $|u| = 1$.
\end{problem}

\begin{problem}\label{prb:A.17}
Show that $(1 + i)^{n} + (1 - i)^{n}$ is real for all $n$, using property C5.
\end{problem}

\begin{problem}\label{prb:A.18}
Express each of the following in polar form (use the principal argument).

\begin{enumerate}
\item $3 - 3i$
\item $-4i$

Click the arrow to see the answer.
\begin{expandable}
 $$4e^{-\pi i/2}$$
\end{expandable}

\item $-\sqrt{3} + i$
\item $-4 + 4\sqrt{3}i$

Click the arrow to see the answer.
\begin{expandable}
$$8e^{2\pi i/3}$$
\end{expandable}

\item $-7i$
\item $-6 + 6i$

Click the arrow to see the answer.
\begin{expandable}
 $$6\sqrt{2}e^{3\pi i/4}$$
\end{expandable}

\end{enumerate}
\end{problem}

\begin{problem}\label{prb:A.19}
Express each of the following in the form $a + bi$.

\begin{enumerate}
\item $3e^{\pi i}$
\item $e^{7\pi i/3}$

Click the arrow to see the answer.
\begin{expandable}
 $$\frac{1}{2} + \frac{\sqrt{3}}{2} i$$
\end{expandable}

\item $2e^{3 \pi i/4}$
\item $\sqrt{2}e^{-\pi i/4}$

Click the arrow to see the answer.
\begin{expandable}
 $$1 - i$$
\end{expandable}

\item $e^{5\pi i/4}$
\item $2\sqrt{3}e^{-2\pi i/6}$

Click the arrow to see the answer.
\begin{expandable}
 $$\sqrt{3} - 3i$$
\end{expandable}
\end{enumerate}
\end{problem}

\begin{problem}\label{prb:A.20}
Express each of the following in the form $a + bi$.

\begin{enumerate}
\item $(-1 + \sqrt{3}i)^2$
\item $(1 + \sqrt{3}i)^{-4}$

Click the arrow to see the answer.
\begin{expandable}
 $$-\frac{1}{32} + \frac{\sqrt{3}}{32}i$$
\end{expandable}

\item $(1 + i)^8$
\item $(1 - i)^{10}$

Click the arrow to see the answer.
\begin{expandable}
 $$-32i$$
\end{expandable}

\item $(1 - i)^{6}(\sqrt{3} + i)^{3}$
\item $(\sqrt{3} - i)^{9}(2 - 2i)^{5}$

Click the arrow to see the answer.
\begin{expandable}
 $$-2^{16}(1 + i)$$
\end{expandable}
\end{enumerate}
\end{problem}

\begin{problem}\label{prb:A.21}
Use De Moivre's theorem to show that:

\begin{enumerate}[label={\alph*.}]
\item $\cos 2\theta = \cos^{2} \theta - \sin^{2} \theta$; $\sin 2\theta = 2 \cos \theta \sin \theta$

\item $\cos 3\theta = \cos^{3} \theta - 3 \cos \theta \sin^{2} \theta$; \\
$\sin 3\theta = 3 \cos^2 \theta \sin \theta - \sin^3 \theta$

\end{enumerate}
\end{problem}

\begin{problem}\label{prb:A.22}
\begin{enumerate}[label={\alph*.}]
\item Find the fourth roots of unity.

\item Find the sixth roots of unity.

\end{enumerate}
\end{problem}

\begin{problem}\label{prb:A.23}
Find all complex numbers $z$ such that:

\begin{enumerate}
\item $z^{4} = -1$
\item $z^{4} = 2(\sqrt{3}i - 1)$

Click the arrow to see the answer.
\begin{expandable}
 $$\pm \frac{\sqrt{2}}{2}(\sqrt{3}+i), \pm \frac{\sqrt{2}}{2}(-1 + \sqrt{3}i)$$
\end{expandable}

\item $z^{3} = -27i$
\item $z^{6} = -64$

Click the arrow to see the answer.
\begin{expandable}
$$\pm 2i, \pm (\sqrt{3} +i), \pm (\sqrt{3}-i)$$
\end{expandable}

\end{enumerate}
\end{problem}

\begin{problem}\label{prb:A.24}
If $z = re^{i\theta}$ in polar form, show that:

\begin{enumerate}
\item $\overline{z} = re^{-i\theta}$
\item $z^{-1}  = \frac{1}{r} e^{-i\theta}$ if  $z \neq 0 $
\end{enumerate}
\end{problem}

\begin{problem}\label{prb:A.25}
Show that the sum of the $n$th roots of unity is zero. 

\begin{hint}
$1 - z^{n} = (1 - z)(1 + z + z^{2} + \dots  + z^{n-1})$ for any complex number $z$.
\end{hint}
\end{problem}

\begin{problem}\label{prb:A.26}
\begin{enumerate}
\item Let $z_{1}$, $z_{2}$, $z_{3}$, $z_{4}$, and $z_{5}$ be equally spaced around the unit circle. Show that $z_{1} + z_{2} + z_{3} + z_{4} + z_{5} = 0$.

\begin{hint}
$(1 - z)(1 + z + z^{2} + z^{3} + z^{4}) = 1 - z^{5}$ for any complex number $z$.
\end{hint}

\item Repeat (a) for any $n \geq 2$ points equally spaced around the unit circle.

\begin{hint}
The argument in (a) applies using $\beta = \frac{2\pi}{n}$.
 Then $ 1 + z + \ldots + z^{n-1} = \frac{1-z^n}{1-z}=0$.
\end{hint}

\item If $|w| = 1$, show that the sum of the roots of $z^n = w$ is zero.

\end{enumerate}
\end{problem}

\begin{problem}\label{prb:A.27}
If $z^n$ is real, $n \geq 1$, show that $(\overline{z})^{n}$ is real.
\end{problem}

\begin{problem}\label{prb:A.28}
If $\overline{z}^2 = z^{2}$, show that $z$ is real or pure imaginary.
\end{problem}

\begin{problem}\label{prb:A.29}
If $a$ and $b$ are \textit{rational} numbers, let $p$ and $q$ denote numbers of the form $a + b\sqrt{2}$. If $p = a + b\sqrt{2}$, define $\tilde{p} = a-b\sqrt{2}$ and $[p] = a^{2} - 2b^{2}$. Show that each of the following holds.

\begin{enumerate}
\item (challenge problem) $a + b\sqrt{2} = a_{1} + b_{1}\sqrt{2}$ only if $a = a_{1}$ and $b = b_{1}$
\item $\widetilde{p \pm q} = \tilde{p} \pm \tilde{q}$
\item $\widetilde{pq} = \tilde{p}\tilde{q}$
\item $[p] = p \tilde{p}$
\item $[pq] = [p][q]$
\item (challenge problem) If $f(x)$ is a polynomial with rational coefficients and $p = a + b\sqrt{2}$ is a root of $f(x)$, then $\tilde{p}$
 is also a root of $f(x)$.
\end{enumerate}
\end{problem}

\section*{Text Source} This section was adapted from Appendix A of Keith Nicholson's \href{https://open.umn.edu/opentextbooks/textbooks/linear-algebra-with-applications}{\it Linear Algebra with Applications}. (CC-BY-NC-SA)

W. Keith Nicholson, {\it Linear Algebra with Applications}, Lyryx 2018, Open Edition, pp. 581--594.

\end{document}