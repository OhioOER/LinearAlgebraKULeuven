\documentclass{ximera}
%%% Begin Laad packages

\makeatletter
\@ifclassloaded{xourse}{%
    \typeout{Start loading preamble.tex (in a XOURSE)}%
    \def\isXourse{true}   % automatically defined; pre 112022 it had to be set 'manually' in a xourse
}{%
    \typeout{Start loading preamble.tex (NOT in a XOURSE)}%
}
\makeatother

\def\isEn\true 

\pgfplotsset{compat=1.16}

\usepackage{currfile}

% 201908/202301: PAS OP: babel en doclicense lijken problemen te veroorzaken in .jax bestand
% (wegens syntax error met toegevoegde \newcommands ...)
\pdfOnly{
    \usepackage[type={CC},modifier={by-nc-sa},version={4.0}]{doclicense}
    %\usepackage[hyperxmp=false,type={CC},modifier={by-nc-sa},version={4.0}]{doclicense}
    %%% \usepackage[dutch]{babel}
}



\usepackage[utf8]{inputenc}
\usepackage{morewrites}   % nav zomercursus (answer...?)
\usepackage{multirow}
\usepackage{multicol}
\usepackage{tikzsymbols}
\usepackage{ifthen}
%\usepackage{animate} BREAKS HTML STRUCTURE USED BY XIMERA
\usepackage{relsize}

\usepackage{eurosym}    % \euro  (€ werkt niet in xake ...?)
\usepackage{fontawesome} % smileys etc

% Nuttig als ook interactieve beamer slides worden voorzien:
\providecommand{\p}{} % default nothing ; potentially usefull for slides: redefine as \pause
%providecommand{\p}{\pause}

    % Layout-parameters voor het onderschrift bij figuren
\usepackage[margin=10pt,font=small,labelfont=bf, labelsep=endash,format=hang]{caption}
%\usepackage{caption} % captionof
%\usepackage{pdflscape}    % landscape environment

% Met "\newcommand\showtodonotes{}" kan je todonotes tonen (in pdf/online)
% 201908: online werkt het niet (goed)
\providecommand\showtodonotes{disable}
\providecommand\todo[1]{\typeout{TODO #1}}
%\usepackage[\showtodonotes]{todonotes}
%\usepackage{todonotes}

%
% Poging tot aanpassen layout
%
\usepackage{tcolorbox}
\tcbuselibrary{theorems}

%%% Einde laad packages

%%% Begin Ximera specifieke zaken

\graphicspath{
	{../../}
	{../}
	{./}
  	{../../pictures/}
   	{../pictures/}
   	{./pictures/}
	{./explog/}    % M05 in groeimodellen       
}

%%% Einde Ximera specifieke zaken

%
% define softer blue/red/green, use KU Leuven base colors for blue (and dark orange for red ?)
%
% todo: rather redefine blue/red/green ...?
%\definecolor{xmblue}{rgb}{0.01, 0.31, 0.59}
%\definecolor{xmred}{rgb}{0.89, 0.02, 0.17}
\definecolor{xmdarkblue}{rgb}{0.122, 0.671, 0.835}   % KU Leuven Blauw
\definecolor{xmblue}{rgb}{0.114, 0.553, 0.69}        % KU Leuven Blauw
\definecolor{xmgreen}{rgb}{0.13, 0.55, 0.13}         % No KULeuven variant for green found ...

\definecolor{xmaccent}{rgb}{0.867, 0.541, 0.18}      % KU Leuven Accent (orange ...)
\definecolor{kuaccent}{rgb}{0.867, 0.541, 0.18}      % KU Leuven Accent (orange ...)

\colorlet{xmred}{xmaccent!50!black}                  % Darker version of KU Leuven Accent

\providecommand{\blue}[1]{{\color{blue}#1}}    
\providecommand{\red}[1]{{\color{red}#1}}

\renewcommand\CancelColor{\color{xmaccent!50!black}}

% werkt in math en text mode om MATH met oranje (of grijze...)  achtergond te tonen (ook \important{\text{blabla}} lijkt te werken)
%\newcommand{\important}[1]{\ensuremath{\colorbox{xmaccent!50!white}{$#1$}}}   % werkt niet in Mathjax
%\newcommand{\important}[1]{\ensuremath{\colorbox{lightgray}{$#1$}}}
\newcommand{\important}[1]{\ensuremath{\colorbox{orange}{$#1$}}}   % TODO: kleur aanpassen voor mathjax; wordt overschreven infra!


% Uitzonderlijk kan met \pdfnl in de PDF een newline worden geforceerd, die online niet nodig/nuttig is omdat daar de regellengte hoe dan ook niet gekend is.
\ifdefined\HCode%
\providecommand{\pdfnl}{}%
\else%
\providecommand{\pdfnl}{%
  \\%
}%
\fi

% Uitzonderlijk kan met \handoutnl in de handout-PDF een newline worden geforceerd, die noch online noch in de PDF-met-antwoorden nuttig is.
\ifdefined\HCode
\providecommand{\handoutnl}{}
\else
\providecommand{\handoutnl}{%
\ifhandout%
  \nl%
\fi%
}
\fi



% \cellcolor IGNORED by tex4ht ?
% \begin{center} seems not to wordk
    % (missing margin-left: auto;   on tabular-inside-center ???)
%\newcommand{\importantcell}[1]{\ensuremath{\cellcolor{lightgray}#1}}  %  in tabular; usablility to be checked
\providecommand{\importantcell}[1]{\ensuremath{#1}}     % no mathjax2 support for colloring array cells

\pdfOnly{
  \renewcommand{\important}[1]{\ensuremath{\colorbox{kuaccent!50!white}{$#1$}}}
  \renewcommand{\importantcell}[1]{\ensuremath{\cellcolor{kuaccent!40!white}#1}}   
}

%%% Tikz styles


\pgfplotsset{compat=1.16}

\usetikzlibrary{trees,positioning,arrows,fit,shapes,math,calc,decorations.markings,through,intersections,patterns,matrix}

\usetikzlibrary{decorations.pathreplacing,backgrounds}    % 5/2023: from experimental


\usetikzlibrary{angles,quotes}

\usepgfplotslibrary{fillbetween} % bepaalde_integraal
\usepgfplotslibrary{polar}    % oa voor poolcoordinaten.tex

\pgfplotsset{ownstyle/.style={axis lines = center, axis equal image, xlabel = $x$, ylabel = $y$, enlargelimits}} 

\pgfplotsset{
	plot/.style={no marks,samples=50}
}

\newcommand{\xmPlotsColor}{
	\pgfplotsset{
		plot1/.style={darkgray,no marks,samples=100},
		plot2/.style={lightgray,no marks,samples=100},
		plotresult/.style={blue,no marks,samples=100},
		plotblue/.style={blue,no marks,samples=100},
		plotred/.style={red,no marks,samples=100},
		plotgreen/.style={green,no marks,samples=100},
		plotpurple/.style={purple,no marks,samples=100}
	}
}
\newcommand{\xmPlotsBlackWhite}{
	\pgfplotsset{
		plot1/.style={black,loosely dashed,no marks,samples=100},
		plot2/.style={black,loosely dotted,no marks,samples=100},
		plotresult/.style={black,no marks,samples=100},
		plotblue/.style={black,no marks,samples=100},
		plotred/.style={black,dotted,no marks,samples=100},
		plotgreen/.style={black,dashed,no marks,samples=100},
		plotpurple/.style={black,dashdotted,no marks,samples=100}
	}
}


\newcommand{\xmPlotsColorAndStyle}{
	\pgfplotsset{
		plot1/.style={darkgray,no marks,samples=100},
		plot2/.style={lightgray,no marks,samples=100},
		plotresult/.style={blue,no marks,samples=100},
		plotblue/.style={xmblue,no marks,samples=100},
		plotred/.style={xmred,dashed,thick,no marks,samples=100},
		plotgreen/.style={xmgreen,dotted,very thick,no marks,samples=100},
		plotpurple/.style={purple,no marks,samples=100}
	}
}


%\iftikzexport
\xmPlotsColorAndStyle
%\else
%\xmPlotsBlackWhite
%\fi
%%%


%
% Om venndiagrammen te arceren ...
%
\makeatletter
\pgfdeclarepatternformonly[\hatchdistance,\hatchthickness]{north east hatch}% name
{\pgfqpoint{-1pt}{-1pt}}% below left
{\pgfqpoint{\hatchdistance}{\hatchdistance}}% above right
{\pgfpoint{\hatchdistance-1pt}{\hatchdistance-1pt}}%
{
	\pgfsetcolor{\tikz@pattern@color}
	\pgfsetlinewidth{\hatchthickness}
	\pgfpathmoveto{\pgfqpoint{0pt}{0pt}}
	\pgfpathlineto{\pgfqpoint{\hatchdistance}{\hatchdistance}}
	\pgfusepath{stroke}
}
\pgfdeclarepatternformonly[\hatchdistance,\hatchthickness]{north west hatch}% name
{\pgfqpoint{-\hatchthickness}{-\hatchthickness}}% below left
{\pgfqpoint{\hatchdistance+\hatchthickness}{\hatchdistance+\hatchthickness}}% above right
{\pgfpoint{\hatchdistance}{\hatchdistance}}%
{
	\pgfsetcolor{\tikz@pattern@color}
	\pgfsetlinewidth{\hatchthickness}
	\pgfpathmoveto{\pgfqpoint{\hatchdistance+\hatchthickness}{-\hatchthickness}}
	\pgfpathlineto{\pgfqpoint{-\hatchthickness}{\hatchdistance+\hatchthickness}}
	\pgfusepath{stroke}
}
%\makeatother

\tikzset{
    hatch distance/.store in=\hatchdistance,
    hatch distance=10pt,
    hatch thickness/.store in=\hatchthickness,
   	hatch thickness=2pt
}

\colorlet{circle edge}{black}
\colorlet{circle area}{blue!20}


\tikzset{
    filled/.style={fill=green!30, draw=circle edge, thick},
    arceerl/.style={pattern=north east hatch, pattern color=blue!50, draw=circle edge},
    arceerr/.style={pattern=north west hatch, pattern color=yellow!50, draw=circle edge},
    outline/.style={draw=circle edge, thick}
}




%%% Updaten commando's
\def\hoofding #1#2#3{\maketitle}     % OBSOLETE ??

% we willen (bijna) altijd \geqslant ipv \geq ...!
\newcommand{\geqnoslant}{\geq}
\renewcommand{\geq}{\geqslant}
\newcommand{\leqnoslant}{\leq}
\renewcommand{\leq}{\leqslant}

% Todo: (201908) waarom komt er (soms) underlined voor emph ...?
\renewcommand{\emph}[1]{\textit{#1}}

% API commando's

\newcommand{\ds}{\displaystyle}
\newcommand{\ts}{\textstyle}  % tegenhanger van \ds   (Ximera zet PER  DEFAULT \ds!)

% uit Zomercursus-macro's: 
\newcommand{\bron}[1]{\begin{scriptsize} \emph{#1} \end{scriptsize}}     % deprecated ...?


%definities nieuwe commando's - afkortingen veel gebruikte symbolen
\newcommand{\R}{\ensuremath{\mathbb{R}}}
\newcommand{\Rnul}{\ensuremath{\mathbb{R}_0}}
\newcommand{\Reen}{\ensuremath{\mathbb{R}\setminus\{1\}}}
\newcommand{\Rnuleen}{\ensuremath{\mathbb{R}\setminus\{0,1\}}}
\newcommand{\Rplus}{\ensuremath{\mathbb{R}^+}}
\newcommand{\Rmin}{\ensuremath{\mathbb{R}^-}}
\newcommand{\Rnulplus}{\ensuremath{\mathbb{R}_0^+}}
\newcommand{\Rnulmin}{\ensuremath{\mathbb{R}_0^-}}
\newcommand{\Rnuleenplus}{\ensuremath{\mathbb{R}^+\setminus\{0,1\}}}
\newcommand{\N}{\ensuremath{\mathbb{N}}}
\newcommand{\Nnul}{\ensuremath{\mathbb{N}_0}}
\newcommand{\Z}{\ensuremath{\mathbb{Z}}}
\newcommand{\Znul}{\ensuremath{\mathbb{Z}_0}}
\newcommand{\Zplus}{\ensuremath{\mathbb{Z}^+}}
\newcommand{\Zmin}{\ensuremath{\mathbb{Z}^-}}
\newcommand{\Znulplus}{\ensuremath{\mathbb{Z}_0^+}}
\newcommand{\Znulmin}{\ensuremath{\mathbb{Z}_0^-}}
\newcommand{\C}{\ensuremath{\mathbb{C}}}
\newcommand{\Cnul}{\ensuremath{\mathbb{C}_0}}
\newcommand{\Cplus}{\ensuremath{\mathbb{C}^+}}
\newcommand{\Cmin}{\ensuremath{\mathbb{C}^-}}
\newcommand{\Cnulplus}{\ensuremath{\mathbb{C}_0^+}}
\newcommand{\Cnulmin}{\ensuremath{\mathbb{C}_0^-}}
\newcommand{\Q}{\ensuremath{\mathbb{Q}}}
\newcommand{\Qnul}{\ensuremath{\mathbb{Q}_0}}
\newcommand{\Qplus}{\ensuremath{\mathbb{Q}^+}}
\newcommand{\Qmin}{\ensuremath{\mathbb{Q}^-}}
\newcommand{\Qnulplus}{\ensuremath{\mathbb{Q}_0^+}}
\newcommand{\Qnulmin}{\ensuremath{\mathbb{Q}_0^-}}

\newcommand{\perdef}{\overset{\mathrm{def}}{=}}
\newcommand{\pernot}{\overset{\mathrm{notatie}}{=}}
\newcommand\perinderdaad{\overset{!}{=}}     % voorlopig gebruikt in limietenrekenregels
\newcommand\perhaps{\overset{?}{=}}          % voorlopig gebruikt in limietenrekenregels

\newcommand{\degree}{^\circ}


\DeclareMathOperator{\dom}{dom}     % domein
\DeclareMathOperator{\codom}{codom} % codomein
\DeclareMathOperator{\bld}{bld}     % beeld
\DeclareMathOperator{\graf}{graf}   % grafiek
\DeclareMathOperator{\rico}{rico}   % richtingcoëfficient
\DeclareMathOperator{\co}{co}       % coordinaat
\DeclareMathOperator{\gr}{gr}       % graad

\newcommand{\func}[5]{\ensuremath{#1: #2 \rightarrow #3: #4 \mapsto #5}} % Easy to write a function


% Operators
\DeclareMathOperator{\bgsin}{bgsin}
\DeclareMathOperator{\bgcos}{bgcos}
\DeclareMathOperator{\bgtan}{bgtan}
\DeclareMathOperator{\bgcot}{bgcot}
\DeclareMathOperator{\bgsinh}{bgsinh}
\DeclareMathOperator{\bgcosh}{bgcosh}
\DeclareMathOperator{\bgtanh}{bgtanh}
\DeclareMathOperator{\bgcoth}{bgcoth}

% Oude \Bgsin etc deprecated: gebruik \bgsin, en herdefinieer dat als je Bgsin wil!
%\DeclareMathOperator{\cosec}{cosec}    % not used? gebruik \csc en herdefinieer

% operatoren voor differentialen: to be verified; 1/2020: inconsequent gebruik bij afgeleiden/integralen
\renewcommand{\d}{\mathrm{d}}
\newcommand{\dx}{\d x}
\newcommand{\dd}[1]{\frac{\mathrm{d}}{\mathrm{d}#1}}
\newcommand{\ddx}{\dd{x}}

% om in voorbeelden/oefeningen de notatie voor afgeleiden te kunnen kiezen
% Usage: \afg{(2\sin(x))}  (en wordt d/dx, of accent, of D )
%\newcommand{\afg}[1]{{#1}'}
\newcommand{\afg}[1]{\left(#1\right)'}
%\renewcommand{\afg}[1]{\frac{\mathrm{d}#1}{\mathrm{d}x}}   % include in relevant exercises ...
%\renewcommand{\afg}[1]{D{#1}}

%
% \xmxxx commands: Extra KU Leuven functionaliteit van, boven of naast Ximera
%   ( Conventie 8/2019: xm+nederlandse omschrijving, maar is niet consequent gevolgd, en misschien ook niet erg handig !)
%
% (Met een minimale ximera.cls en preamble.tex zou een bruikbare .pdf moeten kunnen worden gemaakt van eender welke ximera)
%
% Usage: \xmtitle[Mijn korte abstract]{Mijn titel}{Mijn abstract}
% Eerste command na \begin{document}:
%  -> definieert de \title
%  -> definieert de abstract
%  -> doet \maketitle ( dus: print de hoofding als 'chapter' of 'sectie')
% Optionele parameter geeft eenn kort abstract (die met de globale setting \xmshortabstract{} al dan niet kan worden geprint.
% De optionele korte abstract kan worden gebruikt voor pseudo-grappige abtsarts, dus dus globaal al dan niet kunnen worden gebuikt...
% Globale settings:
%  de (optionele) 'korte abstract' wordt enkele getoond als \xmshortabstract is gezet
\providecommand\xmshortabstract{} % default: print (only!) short abstract if present
\newcommand{\xmtitle}[3][]{
	\title{#2}
	\begin{abstract}
		\ifdefined\xmshortabstract
		\ifstrempty{#1}{%
			#3
		}{%
			#1
		}%
		\else
		#3
		\fi
	\end{abstract}
	\maketitle
}

% 
% Kleine grapjes: moeten zonder verder gevolg kunnen worden verwijderd
%
%\newcommand{\xmopje}[1]{{\small#1{\reversemarginpar\marginpar{\Smiley}}}}   % probleem in floats!!
\newtoggle{showxmopje}
\toggletrue{showxmopje}

\newcommand{\xmopje}[1]{%
   \iftoggle{showxmopje}{#1}{}%
}


% -> geef een abstracte-formule-met-rechts-een-concreet-voorbeeld
% VB:  \formulevb{a^2+b^2=c^2}{3^2+4^2=5^2}
%
\ifdefined\HCode
\NewEnviron{xmdiv}[1]{\HCode{\Hnewline<div class="#1">\Hnewline}\BODY{\HCode{\Hnewline</div>\Hnewline}}}
\else
\NewEnviron{xmdiv}[1]{\BODY}
\fi

\providecommand{\formulevb}[2]{
	{\centering

    \begin{xmdiv}{xmformulevb}    % zie css voor online layout !!!
	\begin{tabular}{lcl}
		\important{#1}
		&  &
		Vb: $#2$
		\end{tabular}
	\end{xmdiv}

	}
}

\ifdefined\HCode
\providecommand{\vb}[1]{%
    \HCode{\Hnewline<span class="xmvb">}#1\HCode{</span>\Hnewline}%
}
\else
\providecommand{\vb}[1]{
    \colorbox{blue!10}{#1}
}
\fi

\ifdefined\HCode
\providecommand{\xmcolorbox}[2]{
	\HCode{\Hnewline<div class="xmcolorbox">\Hnewline}#2\HCode{\Hnewline</div>\Hnewline}
}
\else
\providecommand{\xmcolorbox}[2]{
  \cellcolor{#1}#2
}
\fi


\ifdefined\HCode
\providecommand{\xmopmerking}[1]{
 \HCode{\Hnewline<div class="xmopmerking">\Hnewline}#1\HCode{\Hnewline</div>\Hnewline}
}
\else
\providecommand{\xmopmerking}[1]{
	{\footnotesize #1}
}
\fi
% \providecommand{\voorbeeld}[1]{
% 	\colorbox{blue!10}{$#1$}
% }



% Hernoem Proof naar Bewijs, nodig voor HTML versie
\renewcommand*{\proofname}{Bewijs}

% Om opgave van oefening (wordt niet geprint bij oplossingenblad)
% (to be tested test)
\NewEnviron{statement}{\BODY}

% Environment 'oplossing' en 'uitkomst'
% voor resp. volledige 'uitwerking' dan wel 'enkel eindresultaat'
% geimplementeerd via feedback, omdat er in de ximera-server adhoc feedback-code is toegevoegd
%% Niet tonen indien handout
%% Te gebruiken om volledige oplossingen/uitwerkingen van oefeningen te tonen
%% \begin{oplossing}        De optelling is commutatief \end{oplossing}  : verschijnt online enkel 'op vraag'
%% \begin{oplossing}[toon]  De optelling is commutatief \end{oplossing}  : verschijnt steeds onmiddellijk online (bv te gebruiken bij voorbeelden) 

\ifhandout%
    \NewEnviron{oplossing}[1][onzichtbaar]%
    {%
    \ifthenelse{\equal{\detokenize{#1}}{\detokenize{toon}}}
    {
    \def\PH@Command{#1}% Use PH@Command to hold the content and be a target for "\expandafter" to expand once.

    \begin{trivlist}% Begin the trivlist to use formating of the "Feedback" label.
    \item[\hskip \labelsep\small\slshape\bfseries Oplossing% Format the "Feedback" label. Don't forget the space.
    %(\texttt{\detokenize\expandafter{\PH@Command}}):% Format (and detokenize) the condition for feedback to trigger
    \hspace{2ex}]\small%\slshape% Insert some space before the actual feedback given.
    \BODY
    \end{trivlist}
    }
    {  % \begin{feedback}[solution]   \BODY     \end{feedback}  }
    }
    }    
\else
% ONLY for HTML; xmoplossing is styled with css, and is not, and need not be a LaTeX environment
% THUS: it does NOT use feedback anymore ...
%    \NewEnviron{oplossing}{\begin{expandable}{xmoplossing}{\nlen{Toon uitwerking}{Show solution}}{\BODY}\end{expandable}}
    \newenvironment{oplossing}[1][onzichtbaar]
   {%
       \begin{expandable}{xmoplossing}{}
   }
   {%
   	   \end{expandable}
   } 
%     \newenvironment{oplossing}[1][onzichtbaar]
%    {%
%        \begin{feedback}[solution]   	
%    }
%    {%
%    	   \end{feedback}
%    } 
\fi

\ifhandout%
    \NewEnviron{uitkomst}[1][onzichtbaar]%
    {%
    \ifthenelse{\equal{\detokenize{#1}}{\detokenize{toon}}}
    {
    \def\PH@Command{#1}% Use PH@Command to hold the content and be a target for "\expandafter" to expand once.

    \begin{trivlist}% Begin the trivlist to use formating of the "Feedback" label.
    \item[\hskip \labelsep\small\slshape\bfseries Uitkomst:% Format the "Feedback" label. Don't forget the space.
    %(\texttt{\detokenize\expandafter{\PH@Command}}):% Format (and detokenize) the condition for feedback to trigger
    \hspace{2ex}]\small%\slshape% Insert some space before the actual feedback given.
    \BODY
    \end{trivlist}
    }
    {  % \begin{feedback}[solution]   \BODY     \end{feedback}  }
    }
    }    
\else
\ifdefined\HCode
   \newenvironment{uitkomst}[1][onzichtbaar]
    {%
        \begin{expandable}{xmuitkomst}{}%
    }
    {%
    	\end{expandable}%
    } 
\else
  % Do NOT print 'uitkomst' in non-handout
  %  (presumably, there is also an 'oplossing' ??)
  \newenvironment{uitkomst}[1][onzichtbaar]{}{}
\fi
\fi

%
% Uitweidingen zijn extra's die niet redelijkerwijze tot de leerstof behoren
% Uitbreidingen zijn extra's die wel redelijkerwijze tot de leerstof van bv meer geavanceerde versies kunnen behoren (B-programma/Wiskundestudenten/...?)
% Nog niet voorzien: design voor verschillende versies (A/B programma, BIO, voorkennis/ ...)
% Voor 'uitweidingen' is er een environment die online per default is ingeklapt, en in pdf al dan niet kan worden geincluded  (via \xmnouitweiding) 
%
% in een xourse, per default GEEN uitweidingen, tenzij \xmuitweiding expliciet ergens is gezet ...
\ifdefined\isXourse
   \ifdefined\xmuitweiding
   \else
       \def\xmnouitweiding{true}
   \fi
\fi

\ifdefined\xmnouitweiding
\newcounter{xmuitweiding}  % anders error undefined ...  
\excludecomment{xmuitweiding}
\else
\newtheoremstyle{dotless}{}{}{}{}{}{}{ }{}
\theoremstyle{dotless}
\newtheorem*{xmuitweidingnofrills}{}   % nofrills = no accordion; gebruikt dus de dotless theoremstyle!

\newcounter{xmuitweiding}
\newenvironment{xmuitweiding}[1][ ]%
{% 
	\refstepcounter{xmuitweiding}%
    \begin{expandable}{xmuitweiding}{\nlentext{Uitweiding \arabic{xmuitweiding}: #1}{Digression \arabic{xmuitweiding}: #1}}%
	\begin{xmuitweidingnofrills}%
}
{%
    \end{xmuitweidingnofrills}%
    \end{expandable}%
}   
% \newenvironment{xmuitweiding}[1][ ]%
% {% 
% 	\refstepcounter{xmuitweiding}
% 	\begin{accordion}\begin{accordion-item}[Uitweiding \arabic{xmuitweiding}: #1]%
% 			\begin{xmuitweidingnofrills}%
% 			}
% 			{\end{xmuitweidingnofrills}\end{accordion-item}\end{accordion}}   
\fi


\newenvironment{xmexpandable}[1][]{
	\begin{accordion}\begin{accordion-item}[#1]%
		}{\end{accordion-item}\end{accordion}}


% Command that gives a selection box online, but just prints the right answer in pdf
\newcommand{\xmonlineChoice}[1]{\pdfOnly{\wordchoicegiventrue}\wordChoice{#1}\pdfOnly{\wordchoicegivenfalse}}   % deprecated, gebruik onlineChoice ...
\newcommand{\onlineChoice}[1]{\pdfOnly{\wordchoicegiventrue}\wordChoice{#1}\pdfOnly{\wordchoicegivenfalse}}

\newcommand{\TJa}{\nlentext{ Ja }{ Yes }}
\newcommand{\TNee}{\nlentext{ Nee }{ No }}
\newcommand{\TJuist}{\nlentext{ Juist }{ True }}
\newcommand{\TFout}{\nlentext{ Fout }{ False }}

\newcommand{\choiceTrue }{{\renewcommand{\choiceminimumhorizontalsize}{4em}\wordChoice{\choice[correct]{\TJuist}\choice{\TFout}}}}
\newcommand{\choiceFalse}{{\renewcommand{\choiceminimumhorizontalsize}{4em}\wordChoice{\choice{\TJuist}\choice[correct]{\TFout}}}}

\newcommand{\choiceYes}{{\renewcommand{\choiceminimumhorizontalsize}{3em}\wordChoice{\choice[correct]{\TJa}\choice{\TNee}}}}
\newcommand{\choiceNo }{{\renewcommand{\choiceminimumhorizontalsize}{3em}\wordChoice{\choice{\TJa}\choice[correct]{\TNee}}}}

% Optional nicer formatting for wordChoice in PDF

\let\inlinechoiceorig\inlinechoice

%\makeatletter
%\providecommand{\choiceminimumverticalsize}{\vphantom{$\frac{\sqrt{2}}{2}$}}   % minimum height of boxes (cfr infra)
\providecommand{\choiceminimumverticalsize}{\vphantom{$\tfrac{2}{2}$}}   % minimum height of boxes (cfr infra)
\providecommand{\choiceminimumhorizontalsize}{1em}   % minimum width of boxes (cfr infra)

\newcommand{\inlinechoicesquares}[2][]{%
		\setkeys{choice}{#1}%
		\ifthenelse{\boolean{\choice@correct}}%
		{%
            \ifhandout%
               \fbox{\choiceminimumverticalsize #2}\allowbreak\ignorespaces%
            \else%
               \fcolorbox{blue}{blue!20}{\choiceminimumverticalsize #2}\allowbreak\ignorespaces\setkeys{choice}{correct=false}\ignorespaces%
            \fi%
		}%
		{% else
			\fbox{\choiceminimumverticalsize #2}\allowbreak\ignorespaces%  HACK: wat kleiner, zodat fits on line ... 	
		}%
}

\newcommand{\inlinechoicesquareX}[2][]{%
		\setkeys{choice}{#1}%
		\ifthenelse{\boolean{\choice@correct}}%
		{%
            \ifhandout%
               \framebox[\ifdim\choiceminimumhorizontalsize<\width\width\else\choiceminimumhorizontalsize\fi]{\choiceminimumverticalsize\ #2\ }\allowbreak\ignorespaces\setkeys{choice}{correct=false}\ignorespaces%
            \else%
               \fcolorbox{blue}{blue!20}{\makebox[\ifdim\choiceminimumhorizontalsize<\width\width\else\choiceminimumhorizontalsize\fi]{\choiceminimumverticalsize #2}}\allowbreak\ignorespaces\setkeys{choice}{correct=false}\ignorespaces%
            \fi%
		}%
		{% else
        \ifhandout%
			\framebox[\ifdim\choiceminimumhorizontalsize<\width\width\else\choiceminimumhorizontalsize\fi]{\choiceminimumverticalsize\ #2\ }\allowbreak\ignorespaces%  HACK: wat kleiner, zodat fits on line ... 	
        \fi
		}%
}


\newcommand{\inlinechoicepuntjes}[2][]{%
		\setkeys{choice}{#1}%
		\ifthenelse{\boolean{\choice@correct}}%
		{%
            \ifhandout%
               \dots\ldots\ignorespaces\setkeys{choice}{correct=false}\ignorespaces
            \else%
               \fcolorbox{blue}{blue!20}{\choiceminimumverticalsize #2}\allowbreak\ignorespaces\setkeys{choice}{correct=false}\ignorespaces%
            \fi%
		}%
		{% else
			%\fbox{\choiceminimumverticalsize #2}\allowbreak\ignorespaces%  HACK: wat kleiner, zodat fits on line ... 	
		}%
}

% print niets, maar definieer globale variable \myanswer
%  (gebruikt om oplossingsbladen te printen) 
\newcommand{\inlinechoicedefanswer}[2][]{%
		\setkeys{choice}{#1}%
		\ifthenelse{\boolean{\choice@correct}}%
		{%
               \gdef\myanswer{#2}\setkeys{choice}{correct=false}

		}%
		{% else
			%\fbox{\choiceminimumverticalsize #2}\allowbreak\ignorespaces%  HACK: wat kleiner, zodat fits on line ... 	
		}%
}



%\makeatother

\newcommand{\setchoicedefanswer}{
\ifdefined\HCode
\else
%    \renewenvironment{multipleChoice@}[1][]{}{} % remove trailing ')'
    \let\inlinechoice\inlinechoicedefanswer
\fi
}

\newcommand{\setchoicepuntjes}{
\ifdefined\HCode
\else
    \renewenvironment{multipleChoice@}[1][]{}{} % remove trailing ')'
    \let\inlinechoice\inlinechoicepuntjes
\fi
}
\newcommand{\setchoicesquares}{
\ifdefined\HCode
\else
    \renewenvironment{multipleChoice@}[1][]{}{} % remove trailing ')'
    \let\inlinechoice\inlinechoicesquares
\fi
}
%
\newcommand{\setchoicesquareX}{
\ifdefined\HCode
\else
    \renewenvironment{multipleChoice@}[1][]{}{} % remove trailing ')'
    \let\inlinechoice\inlinechoicesquareX
\fi
}
%
\newcommand{\setchoicelist}{
\ifdefined\HCode
\else
    \renewenvironment{multipleChoice@}[1][]{}{)}% re-add trailing ')'
    \let\inlinechoice\inlinechoiceorig
\fi
}

\setchoicesquareX  % by default list-of-squares with onlineChoice in PDF

% Omdat multicols niet werkt in html: enkel in pdf  (in html zijn langere pagina's misschien ook minder storend)
\newenvironment{xmmulticols}[1][2]{
 \pdfOnly{\begin{multicols}{#1}}%
}{ \pdfOnly{\end{multicols}}}

%
% Te gebruiken in plaats van \section\subsection
%  (in een printstyle kan dan het level worden aangepast
%    naargelang \chapter vs \section style )
% 3/2021: DO NOT USE \xmsubsection !
\newcommand\xmsection\subsection
\newcommand\xmsubsection\subsubsection

% Aanpassen printversie
%  (hier gedefinieerd, zodat ze in xourse kunnen worden gezet/overschreven)
\providebool{parttoc}
\providebool{printpartfrontpage}
\providebool{printactivitytitle}
\providebool{printactivityqrcode}
\providebool{printactivityurl}
\providebool{printcontinuouspagenumbers}
\providebool{numberactivitiesbysubpart}
\providebool{addtitlenumber}
\providebool{addsectiontitlenumber}
\addtitlenumbertrue
\addsectiontitlenumbertrue

% The following three commands are hardcoded in xake, you can't create other commands like these, without adding them to xake as well
%  ( gebruikt in xourses om juiste soort titelpagina te krijgen voor verschillende ximera's )
\newcommand{\activitychapter}[2][]{
    {    
    \ifstrequal{#1}{notnumbered}{
        \addtitlenumberfalse
    }{}
    \typeout{ACTIVITYCHAPTER #2}   % logging
	\chapterstyle
	\activity{#2}
    }
}
\newcommand{\activitysection}[2][]{
    {
    \ifstrequal{#1}{notnumbered}{
        \addsectiontitlenumberfalse
    }{}
	\typeout{ACTIVITYSECTION #2}   % logging
	\sectionstyle
	\activity{#2}
    }
}
% Practices worden als activity getoond om de grote blokken te krijgen online
\newcommand{\practicesection}[2][]{
    {
    \ifstrequal{#1}{notnumbered}{
        \addsectiontitlenumberfalse
    }{}
    \typeout{PRACTICESECTION #2}   % logging
	\sectionstyle
	\activity{#2}
    }
}
\newcommand{\activitychapterlink}[3][]{
    {
    \ifstrequal{#1}{notnumbered}{
        \addtitlenumberfalse
    }{}
    \typeout{ACTIVITYCHAPTERLINK #3}   % logging
	\chapterstyle
	\activitylink[#1]{#2}{#3}
    }
}

\newcommand{\activitysectionlink}[3][]{
    {
    \ifstrequal{#1}{notnumbered}{
        \addsectiontitlenumberfalse
    }{}
    \typeout{ACTIVITYSECTIONLINK #3}   % logging
	\sectionstyle
	\activitylink[#1]{#2}{#3}
    }
}


% Commando om de printstyle toe te voegen in ximera's. Zorgt ervoor dat er geen problemen zijn als je de xourses compileert
% hack om onhandige relative paden in TeX te omzeilen
% should work both in xourse and ximera (pre-112022 only in ximera; thus obsoletes adhoc setup in xourses)
% loads global.sty if present (cfr global.css for online settings!)
% use global.sty to overwrite settings in printstyle.sty ...
\newcommand{\addPrintStyle}[1]{
\iftikzexport\else   % only in PDF
  \makeatletter
  \ifx\@onlypreamble\@notprerr\else   % ONLY if in tex-preamble   (and e.g. not when included from xourse)
    \typeout{Loading printstyle}   % logging
    \usepackage{#1/printstyle} % mag enkel geinclude worden als je die apart compileert
    \IfFileExists{#1/global.sty}{
        \typeout{Loading printstyle-folder #1/global.sty}   % logging
        \usepackage{#1/global}
        }{
        \typeout{Info: No extra #1/global.sty}   % logging
    }   % load global.sty if present
    \IfFileExists{global.sty}{
        \typeout{Loading local-folder global.sty (or TEXINPUTPATH..)}   % logging
        \usepackage{global}
    }{
        \typeout{Info: No folder/global.sty}   % logging
    }   % load global.sty if present
    \IfFileExists{\currfilebase.sty}
    {
        \typeout{Loading \currfilebase.sty}
        \input{\currfilebase.sty}
    }{
        \typeout{Info: No local \currfilebase.sty}
    }
    \fi
  \makeatother
\fi
}

%
%  
% references: Ximera heeft adhoc logica	 om online labels te doen werken over verschillende files heen
% met \hyperref kan de getoonde tekst toch worden opgegeven, in plaats van af te hangen van de label-text
\ifdefined\HCode
% Link to standard \labels, but give your own description
% Usage:  Volg \hyperref[my_very_verbose_label]{deze link} voor wat tijdverlies
%   (01/2020: Ximera-server aangepast om bij class reference-keeptext de link-text NIET te vervangen door de label-text !!!) 
\renewcommand{\hyperref}[2][]{\HCode{<a class="reference reference-keeptext" href="\##1">}#2\HCode{</a>}}
%
%  Link to specific targets  (not tested ?)
\renewcommand{\hypertarget}[1]{\HCode{<a class="ximera-label" id="#1"></a>}}
\renewcommand{\hyperlink}[2]{\HCode{<a class="reference reference-keeptext" href="\##1">}#2\HCode{</a>}}
\fi

% Mmm, quid English ... (for keyword #1 !) ?
\newcommand{\wikilink}[2]{
    \hyperlink{https://nl.wikipedia.org/wiki/#1}{#2}
    \pdfOnly{\footnote{See \url{https://nl.wikipedia.org/wiki/#1}}
    }
}

\renewcommand{\figurename}{Figuur}
\renewcommand{\tablename}{Tabel}

%
% Gedoe om verschillende versies van xourse/ximera te maken afhankelijk van settings
%
% default: versie met antwoorden
% handout: versie voor de studenten, zonder antwoorden/oplossingen
% full: met alles erop en eraan, dus geschikt voor auteurs en/of lesgevers  (bevat in de pdf ook de 'online-only' stukken!)
%
%
% verder kunnen ook opties/variabele worden gezet voor hints/auteurs/uitweidingen/ etc
%
% 'Full' versie
\newtoggle{showonline}
\ifdefined\HCode   % zet default showOnline
    \toggletrue{showonline} 
\else
    \togglefalse{showonline}
\fi

% Full versie   % deprecated: see infra
\newcommand{\printFull}{
    \hintstrue
    \handoutfalse
    \toggletrue{showonline} 
}

\ifdefined\shouldPrintFull   % deprecated: see infra
    \printFull
\fi



% Overschrijf onlineOnly  (zoals gedefinieerd in ximera.cls)
\ifhandout   % in handout: gebruik de oorspronkelijke ximera.cls implementatie  (is dit wel nodig/nuttig?)
\else
    \iftoggle{showonline}{%
        \ifdefined\HCode
          \RenewEnviron{onlineOnly}{\bgroup\BODY\egroup}   % showOnline, en we zijn  online, dus toon de tekst
        \else
          \RenewEnviron{onlineOnly}{\bgroup\color{red!50!black}\BODY\egroup}   % showOnline, maar we zijn toch niet online: kleur de tekst rood 
        \fi
    }{%
      \RenewEnviron{onlineOnly}{}  % geen showOnline
    }
\fi

% hack om na hoofding van definition/proposition/... als dan niet op een nieuwe lijn te starten
% soms is dat goed en mooi, en soms niet; en in HTML is het nu (2/2020) anders dan in pdf
% vandaar suggestie om 
%     \begin{definition}[Nieuw concept] \nl
% te gebruiken als je zeker een newline wil na de hoofdig en titel
% (in het bijzonder itemize zonder \nl is 'lelijk' ...)
\ifdefined\HCode
\newcommand{\nl}{}
\else
\newcommand{\nl}{\ \par} % newline (achter heading van definition etc.)
\fi


% \nl enkel in handoutmode (ihb voor \wordChoice, die dan typisch veeeel langer wordt)
\ifdefined\HCode
\providecommand{\handoutnl}{}
\else
\providecommand{\handoutnl}{%
\ifhandout%
  \nl%
\fi%
}
\fi

% Could potentially replace \pdfOnline/\begin{onlineOnly} : 
% Usage= \ifonline{Hallo surfer}{Hallo PDFlezer}
\providecommand{\ifonline}[2]%
{
\begin{onlineOnly}#1\end{onlineOnly}%
\pdfOnly{#2}
}%


%
% Maak optionele 'basic' en 'extended' versies van een activity
%  met environment basicOnly, basicSkip en extendedOnly
%
%  (
%   Dit werkt ENKEL in de PDF; de online versies tonen (minstens voorklopig) steeds 
%   het default geval met printbasicversion en printextendversion beide FALSE
%  )
%
\providebool{printbasicversion}
\providebool{printextendedversion}   % not properly implemented
\providebool{printfullversion}       % presumably print everything (debug/auteur)
%
% only set these in xourses, and BEFORE loading this preamble
%
%\newif\ifshowbasic     \showbasictrue        % use this line in xourse to show 'basic' sections
%\newif\ifshowextended  \showextendedtrue     % use this line in xourse to show 'extended' sections
%
%
%\ifbool{showbasic}
%      { \NewEnviron{basicOnly}{\BODY} }    % if yes: just print contents
%      { \NewEnviron{basicOnly}{}      }    % if no:  completely ignore contents
%
%\ifbool{showbasic}
%      { \NewEnviron{basicSkip}{}      }
%      { \NewEnviron{basicSkip}{\BODY} }
%

\ifbool{printextendedversion}
      { \NewEnviron{extendedOnly}{\BODY} }
      { \NewEnviron{extendedOnly}{}      }
      


\ifdefined\HCode    % in html: always print
      {\newenvironment*{basicOnly}{}{}}    % if yes: just print contents
      {\newenvironment*{basicSkip}{}{}}    % if yes: just print contents
\else

\ifbool{printbasicversion}
      {\newenvironment*{basicOnly}{}{}}    % if yes: just print contents
      {\NewEnviron{basicOnly}{}      }    % if no:  completely ignore contents

\ifbool{printbasicversion}
      {\NewEnviron{basicSkip}{}      }
      {\newenvironment*{basicSkip}{}{}}

\fi

\usepackage{float}
\usepackage[rightbars,color]{changebar}

% Full versie
\ifbool{printfullversion}{
    \hintstrue
    \handoutfalse
    \toggletrue{showonline}
    \printbasicversionfalse
    \cbcolor{red}
    \renewenvironment*{basicOnly}{\cbstart}{\cbend}
    \renewenvironment*{basicSkip}{\cbstart}{\cbend}
    \def\xmtoonprintopties{FULL}   % will be printed in footer
}
{}
      
%
% Evalueer \ifhints IN de environment
%  
%
%\RenewEnviron{hint}
%{
%\ifhandout
%\ifhints\else\setbox0\vbox\fi%everything in een emty box
%\bgroup 
%\stepcounter{hintLevel}
%\BODY
%\egroup\ignorespacesafterend
%\addtocounter{hintLevel}{-1}
%\else
%\ifhints
%\begin{trivlist}\item[\hskip \labelsep\small\slshape\bfseries Hint:\hspace{2ex}]
%\small\slshape
%\stepcounter{hintLevel}
%\BODY
%\end{trivlist}
%\addtocounter{hintLevel}{-1}
%\fi
%\fi
%}

% Onafhankelijk van \ifhandout ...? TO BE VERIFIED
\RenewEnviron{hint}
{
\ifhints
\begin{trivlist}\item[\hskip \labelsep\small\bfseries Hint:\hspace{2ex}]
\small%\slshape
\stepcounter{hintLevel}
\BODY
\end{trivlist}
\addtocounter{hintLevel}{-1}
\else
\iftikzexport   % anders worden de tikz tekeningen in hints niet gegenereerd ?
\setbox0\vbox\bgroup
\stepcounter{hintLevel}
\BODY
\egroup\ignorespacesafterend
\addtocounter{hintLevel}{-1}
\fi % ifhandout
\fi %ifhints
}

%
% \tab sets typewriter-tabs (e.g. to format questions)
% (Has no effect in HTML :-( ))
%
\usepackage{tabto}
\ifdefined\HCode
  \renewcommand{\tab}{\quad}    % otherwise dummy .png's are generated ...?
\fi


% Also redefined in  preamble to get correct styling 
% for tikz images for (\tikzexport)
%

\theoremstyle{definition} % Bold titels
\makeatletter
\let\proposition\relax
\let\c@proposition\relax
\let\endproposition\relax
\makeatother
\newtheorem{proposition}{Eigenschap}


%\instructornotesfalse

% logic with \ifhandoutin ximera.cls unclear;so overwrite ...
\makeatletter
\@ifundefined{ifinstructornotes}{%
  \newif\ifinstructornotes
  \instructornotesfalse
  \newenvironment{instructorNotes}{}{}
}{}
\makeatother
\ifinstructornotes
\else
\renewenvironment{instructorNotes}%
{%
    \setbox0\vbox\bgroup
}
{%
    \egroup
}
\fi

% \RedeclareMathOperator
% from https://tex.stackexchange.com/questions/175251/how-to-redefine-a-command-using-declaremathoperator
\makeatletter
\newcommand\RedeclareMathOperator{%
    \@ifstar{\def\rmo@s{m}\rmo@redeclare}{\def\rmo@s{o}\rmo@redeclare}%
}
% this is taken from \renew@command
\newcommand\rmo@redeclare[2]{%
    \begingroup \escapechar\m@ne\xdef\@gtempa{{\string#1}}\endgroup
    \expandafter\@ifundefined\@gtempa
    {\@latex@error{\noexpand#1undefined}\@ehc}%
    \relax
    \expandafter\rmo@declmathop\rmo@s{#1}{#2}}
% This is just \@declmathop without \@ifdefinable
\newcommand\rmo@declmathop[3]{%
    \DeclareRobustCommand{#2}{\qopname\newmcodes@#1{#3}}%
}
\@onlypreamble\RedeclareMathOperator
\makeatother


%
% Engelse vertaling, vooral in mathmode
%
% 1. Algemene procedure
%
\ifdefined\isEn
 \newcommand{\nlen}[2]{#2}
 \newcommand{\nlentext}[2]{\text{#2}}
 \newcommand{\nlentextbf}[2]{\textbf{#2}}
\else
 \newcommand{\nlen}[2]{#1}
 \newcommand{\nlentext}[2]{\text{#1}}
 \newcommand{\nlentextbf}[2]{\textbf{#1}}
\fi

%
% 2. Lijst van erg veel gebruikte uitdrukkingen
%

% Ja/Nee/Fout/Juits etc
%\newcommand{\TJa}{\nlentext{ Ja }{ and }}
%\newcommand{\TNee}{\nlentext{ Nee }{ No }}
%\newcommand{\TJuist}{\nlentext{ Juist }{ Correct }
%\newcommand{\TFout}{\nlentext{ Fout }{ Wrong }
\newcommand{\TWaar}{\nlentext{ Waar }{ True }}
\newcommand{\TOnwaar}{\nlentext{ Vals }{ False }}
% Korte bindwoorden en, of, dus, ...
\newcommand{\Ten}{\nlentext{ en }{ and }}
\newcommand{\Tof}{\nlentext{ of }{ or }}
\newcommand{\Tdus}{\nlentext{ dus }{ so }}
\newcommand{\Tendus}{\nlentext{ en dus }{ and thus }}
\newcommand{\Tvooralle}{\nlentext{ voor alle }{ for all }}
\newcommand{\Took}{\nlentext{ ook }{ also }}
\newcommand{\Tals}{\nlentext{ als }{ when }} %of if?
\newcommand{\Twant}{\nlentext{ want }{ as }}
\newcommand{\Tmaal}{\nlentext{ maal }{ times }}
\newcommand{\Toptellen}{\nlentext{ optellen }{ add }}
\newcommand{\Tde}{\nlentext{ de }{ the }}
\newcommand{\Thet}{\nlentext{ het }{ the }}
\newcommand{\Tis}{\nlentext{ is }{ is }} %zodat is in text staat in mathmode (geen italics)
\newcommand{\Tmet}{\nlentext{ met }{ where }} % in situaties e.g met p < n --> where p < n
\newcommand{\Tnooit}{\nlentext{ nooit }{ never }}
\newcommand{\Tmaar}{\nlentext{ maar }{ but }}
\newcommand{\Tniet}{\nlentext{ niet }{ not }}
\newcommand{\Tuit}{\nlentext{ uit }{ from }}
\newcommand{\Ttov}{\nlentext{ t.o.v. }{ w.r.t. }}
\newcommand{\Tzodat}{\nlentext{ zodat }{ such that }}
\newcommand{\Tdeth}{\nlentext{de }{th }}
\newcommand{\Tomdat}{\nlentext{omdat }{because }} 


%
% Overschrijf addhoc commando's
%
\ifdefined\isEn
\renewcommand{\pernot}{\overset{\mathrm{notation}}{=}}
\RedeclareMathOperator{\bld}{im}     % beeld
\RedeclareMathOperator{\graf}{graph}   % grafiek
\RedeclareMathOperator{\rico}{slope}   % richtingcoëfficient
\RedeclareMathOperator{\co}{co}       % coordinaat
\RedeclareMathOperator{\gr}{deg}       % graad

% Operators
\RedeclareMathOperator{\bgsin}{arcsin}
\RedeclareMathOperator{\bgcos}{arccos}
\RedeclareMathOperator{\bgtan}{arctan}
\RedeclareMathOperator{\bgcot}{arccot}
\RedeclareMathOperator{\bgsinh}{arcsinh}
\RedeclareMathOperator{\bgcosh}{arccosh}
\RedeclareMathOperator{\bgtanh}{arctanh}
\RedeclareMathOperator{\bgcoth}{arccoth}

\fi


% HACK: use 'oplossing' for 'explanation' ...
\let\explanation\relax
\let\endexplanation\relax
% \newenvironment{explanation}{\begin{oplossing}}{\end{oplossing}}
\newcounter{explanation}

\ifhandout%
    \NewEnviron{explanation}[1][toon]%
    {%
    \RenewEnviron{verbatim}{ \red{VERBATIM CONTENT MISSING IN THIS PDF}} %% \expandafter\verb|\BODY|}

    \ifthenelse{\equal{\detokenize{#1}}{\detokenize{toon}}}
    {
    \def\PH@Command{#1}% Use PH@Command to hold the content and be a target for "\expandafter" to expand once.

    \begin{trivlist}% Begin the trivlist to use formating of the "Feedback" label.
    \item[\hskip \labelsep\small\slshape\bfseries Explanation:% Format the "Feedback" label. Don't forget the space.
    %(\texttt{\detokenize\expandafter{\PH@Command}}):% Format (and detokenize) the condition for feedback to trigger
    \hspace{2ex}]\small%\slshape% Insert some space before the actual feedback given.
    \BODY
    \end{trivlist}
    }
    {  % \begin{feedback}[solution]   \BODY     \end{feedback}  }
    }
    }    
\else
% ONLY for HTML; xmoplossing is styled with css, and is not, and need not be a LaTeX environment
% THUS: it does NOT use feedback anymore ...
%    \NewEnviron{oplossing}{\begin{expandable}{xmoplossing}{\nlen{Toon uitwerking}{Show solution}}{\BODY}\end{expandable}}
    \newenvironment{explanation}[1][toon]
   {%
       \begin{expandable}{xmoplossing}{}
   }
   {%
   	   \end{expandable}
   } 
\fi

 \title{Orthogonality and Projections} \license{CC BY-NC-SA 4.0}

\begin{document}
\begin{abstract}
\end{abstract}
\maketitle

\begin{onlineOnly}
\section*{Orthogonality and Projections}
\end{onlineOnly}

\subsection*{Orthogonal and Orthonormal Sets}
In this section, we examine what it means for vectors (and sets of
vectors) to be orthogonal and orthonormal. Recall that two non-zero vectors are orthogonal if their dot product is zero.  A collection of non-zero vectors in $\RR^n$ is called \dfn{orthogonal} if the vectors are pair-wise orthogonal.  The diagram below shows two orthogonal vectors in $\RR^2$ and three orthogonal vectors in $\RR^3$.
\begin{center}
\begin{tikzpicture}[scale=0.5]
\draw[line width=0.5pt, dashed](-2,-0.5)--(2,-0.5)--(2,3.5)--(-2,3.5)--cycle;
\draw[line width=2pt,red,-stealth](0,0)--(1.5,0.5);  
\draw[line width=2pt,blue,-stealth](0,0)--(-1,3);
\node[] at (-0.5, -1.5)  (p2)    {Orthogonal vectors in $\RR^2$};
 \end{tikzpicture}
 \quad\quad\quad
\begin{tikzpicture}[scale=0.5]
	\draw[line width=2pt,red,-stealth](0,0,0)--(4,1,0);
    \draw[line width=2pt,blue,-stealth](0,0,0)--(-0.5,2,0);
    \draw[line width=2pt,black,-stealth](0,0,0)--(0,0,5);
    \node[] at (3, -3,1.5)  (p2)    {Orthogonal vectors in $\RR^3$};
        \draw[-,line width=0.2mm, dashed](6.5,3.5,5.5)--(6.5,-1,5.5) ;
    \draw[-,line width=0.2mm, dashed](6.5,3.5,5.5)--(-1,3.5,5.5) ;
    \draw[-,line width=0.2mm, dashed](6.5,3.5,5.5)--(6.5,3.5,-1) ;
    \draw[-,line width=0.2mm, dashed](-1,-1,5.5)--(6.5,-1,5.5) ;
    \draw[-,line width=0.2mm, dashed](-1,-1,5.5)--(-1,3.5,5.5) ;
    \draw[-,line width=0.2mm, dashed](6.5,-1,5.5)--(6.5,-1,-1) ;
    \draw[-,line width=0.2mm, dashed](6.5,3.5,-1)--(6.5,-1,-1) ;
    \draw[-,line width=0.2mm, dashed](6.5,3.5,-1)--(-1,3.5,-1) ;
    \draw[-,line width=0.2mm, dashed](-1,3.5,5.5)--(-1,3.5,-1) ;
        \end{tikzpicture}
\end{center}

If every vector in an orthogonal set of vectors is also a unit vector, then we say that the given set of vectors is \dfn{orthonormal}.

\begin{center}
\begin{tikzpicture}[scale=0.5]
\draw[line width=2pt,red,-stealth](0,0)--(3,4);  
\draw[line width=2pt,blue,-stealth](0,0)--(-4,3);
\node[] at (-1, -1)  (p2)    {An orthonormal set of two vectors};
\node[] at (-1.8, 2)  (p2)    {1};
\node[] at (1.3, 2.5)  (p2)    {1};
 \end{tikzpicture}
 \end{center}

Formally, we can define orthogonal and orthonormal vectors as follows.

\begin{definition}\label{orthset}
Let $\{ \vec{v}_1, \vec{v}_2, \cdots, \vec{v}_k \}$ be a set of nonzero
vectors in $\RR^n$. Then this set is called an
\dfn{orthogonal set} if 
$\vec{v}_i \dotp \vec{v}_j = 0$ for all $i \neq j$.
Moreover, if $\norm{\vec{v}_i}=1$ for $i=1,\ldots,m$ (i.e. each vector in the set is a unit vector), we say the set of vectors is an \dfn{orthonormal set}.
\end{definition}

An orthogonal set of vectors may not be orthonormal.  To convert an orthogonal set to an orthonormal set, we need to divide each vector by its own length.

\begin{definition}\label{normalizing}
\dfn{Normalizing} an orthogonal set is the process of turning an orthogonal set into an orthonormal set.
If $\{ \vec{v}_1, \vec{v}_2, \ldots, \vec{v}_k\}$
is an orthogonal subset of $\RR^n$,
then
\[ \left\{
\frac{1}{\norm{\vec{v}_1}}\vec{v}_1,
\frac{1}{\norm{\vec{v}_2}}\vec{v}_2, \ldots,
\frac{1}{\norm{\vec{v}_k}}\vec{v}_k \right\}
\]
is an orthonormal set.
\end{definition}

We illustrate this concept in the following example.

\begin{example}\label{ex:orthonormalset}
Consider the vectors
\[
\vec{v}_1=\begin{bmatrix}
1 \\
1
\end{bmatrix},\quad \vec{v}_2  =
\begin{bmatrix}
-1 \\
1
\end{bmatrix}
\]
Show that $\{\vec{v}_1,\vec{v}_2\}$ is an orthogonal set of vectors  but not an orthonormal one. Find the corresponding orthonormal set.

\begin{explanation}
One easily verifies that $\vec{v}_1 \dotp \vec{v}_2 = 0$ and
$\left\{ \vec{v}_1, \vec{v}_2 \right\}$ is an orthogonal set of
vectors. On the other hand one can compute that ${\norm{\vec{v}_1}}= {\norm{\vec{v}_2}} =
\sqrt{2} \neq 1$ and so the set is not orthonormal.

To find a corresponding orthonormal set, we need to
normalize each vector. 
\begin{eqnarray*}
\vec{q}_1 &=& \frac{1}{\norm{\vec{v}_1}}\vec{v}_1\\
&=& \frac{1}{\sqrt{2}} \begin{bmatrix}
1 \\
1
\end{bmatrix} \\
&=&
\begin{bmatrix}
\frac{1}{\sqrt{2}} \\
\frac{1}{\sqrt{2}}
\end{bmatrix}
\end{eqnarray*}

Similarly,
\begin{eqnarray*}
\vec{q}_2 &=& \frac{1}{\norm{\vec{v}_2}}\vec{v}_2\\
&=& \frac{1}{\sqrt{2}} \begin{bmatrix}
-1 \\
1
\end{bmatrix} \\
&=&
\begin{bmatrix}
-\frac{1}{\sqrt{2}} \\
\frac{1}{\sqrt{2}}
\end{bmatrix}
\end{eqnarray*}

Therefore the corresponding orthonormal set is
\[
\left\{ \vec{q}_1, \vec{q}_2 \right\} =
\left\{
\begin{bmatrix}
\frac{1}{\sqrt{2}} \\
\frac{1}{\sqrt{2}}
\end{bmatrix},
\begin{bmatrix}
-\frac{1}{\sqrt{2}} \\
\frac{1}{\sqrt{2}}
\end{bmatrix}
\right\}
\]

You can verify that this set is orthonormal.
\end{explanation}
\end{example}

\subsection*{Orthogonal and Orthonormal Bases}
Recall that every basis of $\RR^n$ (or a subspace $W$ of $\RR^n$) imposes a coordinate system on $\RR^n$ (or $W$) that can be used to express any vector of $\RR^n$ (or $W$) as a linear combination of the elements of the basis.  For example, vectors $\vec{v}_1$ and $\vec{v}_2$ impose a coordinate system onto the plane, as shown in the figure below.  We readily see that $\vec{x}$, contained in the plane, can be written as $\vec{x}=\vec{v}_1+2\vec{v}_2$.

\begin{center}
\begin{tikzpicture}[scale=1]
\draw[line width=0.5pt, gray](-2,1)--(0.5,6);  
\draw[line width=0.5pt, gray](-1,-2)--(3,6);
\draw[line width=0.5pt, gray](1.5,-2)--(5.5,6);
\draw[line width=0.5pt, gray](4,-2)--(8,6);
\draw[line width=0.5pt, gray](6.5,-2)--(8,1);
\draw[line width=0.5pt, gray](-2, 4.33)--(3,6);
\draw[line width=0.5pt, gray](-2, 2.66)--(8,6);
\draw[line width=0.5pt, gray](-2, 1)--(8,4.33);
\draw[line width=0.5pt, gray](-2, -0.66)--(8,2.66);
\draw[line width=0.5pt, gray](-1, -2)--(8,1);
\draw[line width=0.5pt, gray](4,-2)--(8,-0.66);
 \draw[line width=2pt,blue,-stealth](0,0)--(1,2);
\draw[line width=2pt,red,-stealth](0,0)--(3,1);
\draw[line width=2pt,-stealth](0,0)--(7,4);
\node[blue] at (0.5, 1.5)  (p2)    {$\vec{v}_1$};
\node[red] at (1.6, 0.2)  (p2)    {$\vec{v}_2$};
\node[] at (3.5, 2.3)  (p2)    {$\vec{x}$};
%\node[] at (-0.2, 0.2)  (p2)    {$\vec{O}$};
 \end{tikzpicture}
 \end{center}
Vector $\vec{x}$ is visually easy to work with.  In general, one way to express an arbitrary vector as a linear combination of the basis vectors is to solve a system of linear equations, which can be costly.  One reason we like $\{\vec{i},\vec{j}\}$ as a basis of $\RR^2$ is because any vector $\vec{x}$ of $\RR^2$ can be easily expressed as the sum of the orthogonal projections of $\vec{x}$ onto the basis vectors $\vec{i}$ and $\vec{j}$, as shown below.
\begin{center}
\begin{tikzpicture}[scale=1.4]
 \draw[<->] (-1,0)--(3.5,0);
  \draw[<->] (0,-1)--(0,3.5);
  \draw[line width=6pt,-stealth, black!20!white](0,0)--(2,0);
  \draw[line width=6pt,-stealth, black!20!white](0,0)--(0,3);
\draw[line width=2pt,red,-stealth](0,0)--(1,0);  
\draw[line width=2pt,blue,-stealth](0,0)--(0,1);
\draw[line width=2pt,-stealth](0,0)--(2,3);  
\draw[line width=0.5pt,dashed](2,3)--(2,0);  
\draw[line width=0.5pt,dashed](2,3)--(0,3); 
\node[] at (1.7, -0.4)  (p2)    {$\mbox{proj}_{\vec{i}}\vec{x}$};
\node[] at (-0.7, 2.5)  (p2)    {$\mbox{proj}_{\vec{j}}\vec{x}$};
\node[] at (3, 3.1)  (p2)    {$\vec{x}=\mbox{proj}_{\vec{i}}\vec{x}+\mbox{proj}_{\vec{j}}\vec{x}$};
\node[red] at (0.5, -0.3)  (p2)    {$\vec{i}$};
\node[blue] at (-0.3, 0.5)  (p2)    {$\vec{j}$};
 \end{tikzpicture}
\end{center}

We can see why an ``upright" coordinate system with basis $\{\vec{i},\vec{j}\}$ works well.  What if we tilt this coordinate system while preserving the orthogonal relationship between the basis vectors?  The following exploration allows you to investigate the consequences.

\begin{exploration}\label{exp:orth1a}
    In the following GeoGebra interactive, vectors $\vec{v}_1$ and $\vec{v}_2$ are orthogonal (slopes of the lines containing them are negative reciprocals of each other).  These vectors are clearly linearly independent and span $\RR^2$.  Therefore $\{\vec{v}_1,\vec{v}_2\}$ is a basis of $\RR^2$.  
    
    Let $\vec{x}$ be an arbitrary vector.  Orthogonal projections of $\vec{x}$ onto $\vec{v}_1$ and $\vec{v}_2$ are depicted in light grey.
    \begin{itemize}
           \item Use the tip of vector $\vec{x}$ to manipulate the vector and convince yourself that $\vec{x}$ is always the diagonal of the parallelogram (a rectangle!) determined by the projections.
        \item Use the tips of $\vec{v}_1$ and $\vec{v}_2$ to change the basis vectors.  What happens when $\vec{v}_1$ and $\vec{v}_2$ are no longer orthogonal?
        \item Pick another pair of orthogonal vectors $\vec{v}_1$ and $\vec{v}_2$.  Verify that $\vec{x}$ is the sum of its projections.
    \end{itemize}

\pdfOnly{
Access GeoGebra interactives through the online version of this text at 

\href{https://ximera.osu.edu/oerlinalg}{https://ximera.osu.edu/oerlinalg}.
}

\begin{onlineOnly}
    \begin{center}
\geogebra{nsqzhsxv}{800}{600}
\end{center}
\end{onlineOnly}
\end{exploration}


As you have just discovered in Exploration \ref{exp:orth1a}, we can express an arbitrary vector of $\RR^2$ as the sum of its projections onto the basis vectors, provided that the basis is orthogonal. It turns out that this result holds for any subspace of $\RR^n$, making a basis consisting of orthogonal vectors especially useful. 

If an orthogonal set is a basis, we call it an
\dfn{orthogonal basis}. Similarly, if an orthonormal set is a basis, we call it an \dfn{orthonormal basis}.


The following theorem generalizes our observation in Exploration \ref{exp:orth1a}.  As you read the statement of the theorem, it will be helpful to recall that the orthogonal projection of vector $\vec{x}$ onto a non-zero vector $\vec{d}$ is given by
\begin{equation}\label{eq:orthProj}
\mbox{proj}_{\vec{d}}\vec{x}=\left(\frac{\vec{x}\cdot\vec{d}}{\norm{\vec{d}}^2}\right)\vec{d}
\end{equation}
(See \href{https://ximera.osu.edu/oerlinalg/LinearAlgebra/VEC-0070/main}{Orthogonal Projections}.)

\begin{theorem}\label{th:fourierexpansion}
Let $W$ be a subspace of $\RR^n$ and suppose $\{ \vec{f}_1, \vec{f}_2, \ldots, \vec{f}_m \}$
is an orthogonal basis of $W$.
Then for every $\vec{x}$ in $W$,
\begin{equation}\label{FourierEqn}
\vec{x} =
\left(\frac{\vec{x}\dotp \vec{f}_1}{\norm{\vec{f}_1}^2}\right) \vec{f}_1 +
\left(\frac{\vec{x}\dotp \vec{f}_2}{\norm{\vec{f}_2}^2}\right) \vec{f}_2 +
\cdots +
\left(\frac{\vec{x}\dotp \vec{f}_m}{\norm{\vec{f}_m}^2}\right) \vec{f}_m.
\end{equation}
\end{theorem}

\begin{proof}
We may express $\vec{x}$ as a linear combination of the basis elements:
\[ \vec{x} =
c_1 \vec{f}_1 +
c_2 \vec{f}_2 +
\cdots +
c_m \vec{f}_m.
\]
We claim that $c_i = \frac{\vec{x}\dotp \vec{f}_i}{\norm{\vec{f}_i}^2}$ for $i=1,\ldots,m$. To see this, we take the dot product of
each side with the vector $\vec{f}_i$ and obtain the following.

\begin{equation*}
  \vec{x} \dotp \vec{f}_i =  \left(c_1\vec{f}_1 +
c_2\vec{f}_2 +
\cdots +
c_m\vec{f}_m\right) \dotp \vec{f}_i 
\end{equation*}
Our basis is orthogonal, so $\vec{f}_j \dotp \vec{f}_i = 0$ for all $j \neq i$, which means after we distribute the dot product, only one term will remain on the right-hand side.  We have 
\begin{equation*}
  \vec{x} \dotp \vec{f}_i =  c_i\vec{f}_i \dotp \vec{f}_i 
\end{equation*}

We now divide both sides by $\vec{f}_i \dotp \vec{f}_i = \norm{\vec{f}_i}^2$, and since our claim holds for $i=1,\ldots,m$, the proof is complete.
\end{proof}

Theorem~\ref{th:fourierexpansion} shows one important benefit of a basis being orthogonal.  With an orthogonal basis it is easy to represent any vector in terms of the basis vectors.  

\begin{example}\label{fourier}
Let
$\vec{f}_1= \begin{bmatrix}
1 \\ -1 \\ 2
\end{bmatrix},
\vec{f}_2= \begin{bmatrix}
0 \\ 2 \\ 1 
\end{bmatrix},
\vec{f}_3 =\begin{bmatrix}
5 \\ 1 \\ -2
\end{bmatrix}$,
and let
$\vec{x} =\begin{bmatrix}
1 \\ 1 \\ 1
\end{bmatrix}$.  

Notice that $\mathcal{B}=\{ \vec{f}_1, \vec{f}_2, \vec{f}_3\}$
is an orthogonal set of vectors, and $\mathcal{B}$ spans $\RR^3$.  Use this fact to write $\vec{x}$ as  a linear combination of the vectors of $\mathcal{B}$.

\begin{explanation}
We first observe that $\mathcal{B}$ is a linearly independent set of vectors, and so $\mathcal{B}$ is a basis for $\RR^3$. Next we apply Theorem~\ref{th:fourierexpansion} to express $\vec{x}$ as  a linear combination of the vectors of $\mathcal{B}$.  We wish to write:

\[
\vec{x}   =
\left(\frac{\vec{x}\dotp \vec{f}_1}{\norm{\vec{f}_1}^2}\right) \vec{f}_1 +
\left(\frac{\vec{x}\dotp \vec{f}_2}{\norm{\vec{f}_2}^2}\right) \vec{f}_2 +
\left(\frac{\vec{x}\dotp \vec{f}_3}{\norm{\vec{f}_3}^2}\right) \vec{f}_3.
\]

We readily compute:

\[
\frac{\vec{x}\dotp\vec{f}_1}{\norm{\vec{f}_1}^2} = \frac{2}{6}, \;
\frac{\vec{x}\dotp\vec{f}_2}{\norm{\vec{f}_2}^2} = \frac{3}{5},
\mbox{ and }
\frac{\vec{x}\dotp\vec{f}_3}{\norm{\vec{f}_3}^2} = \frac{4}{30}.\]

Therefore,
\[ \begin{bmatrix}
1 \\ 1 \\ 1
\end{bmatrix}
= \frac{1}{3}\begin{bmatrix}
1 \\ -1 \\ 2
\end{bmatrix}
+\frac{3}{5}\begin{bmatrix}
0 \\ 2 \\ 1
\end{bmatrix}
+\frac{2}{15}\begin{bmatrix}
5 \\ 1 \\ -2
\end{bmatrix}.\]
\end{explanation} 
\end{example}

The formula from Theorem~\ref{th:fourierexpansion} is easy to use, and it becomes even easier when our basis is \emph{orthonormal}.

\begin{corollary}\label{cor:orthonormal}
Let $W$ be a subspace of $\RR^n$ and suppose $\{ \vec{q}_1, \vec{q}_2, \ldots, \vec{q}_m \}$
is an orthonormal basis of $W$.
Then for any $\vec{x}$ in $W$,
\[ \vec{x} =
\left(\vec{x}\dotp \vec{q}_1\right) \vec{q}_1 +
\left(\vec{x}\dotp \vec{q}_2\right) \vec{q}_2 +
\cdots +
\left(\vec{x}\dotp \vec{q}_m\right)  \vec{q}_m.
\]
\end{corollary}
\begin{proof}
This is a special case of Theorem \ref{th:fourierexpansion}.  Because $\norm{\vec{u_i}} = 1$ for $i=1,\ldots,m$, %where we can compute the coefficients of $x$ with respect to the basis by simply taking the dot product with each basis vector, for in this case 
the terms are given by 
$$\left(\frac{\vec{x}\cdot \vec{q}_i}{\norm{\vec{q}_i}^2}\right)\vec{q}_i=\left(\vec{x}\dotp \vec{q}_i\right) \vec{q}_i.$$

\end{proof}

\subsection*{Orthogonal Projection onto a Subspace}
In the previous section we found that given a subspace $W$ of $\RR^n$ with an orthogonal basis $\mathcal{B}$, every vector $\vec{x}$ in $W$ can be expressed as the sum of the orthogonal projections of $\vec{x}$ onto the elements of $\mathcal{B}$.  Note that our premise was that $\vec{x}$ is in $W$.  In this section, we look into the meaning of the sum of orthogonal projections of $\vec{x}$ onto the elements of an orthogonal basis of $W$ for those vectors $\vec{x}$ of $\RR^n$ that are \emph{not} in $W$.

\begin{exploration}\label{exp:orthProjSub}
In the GeoGebra interactive below, $W$ is a plane spanned by $\vec{v}_1$ and $\vec{v}_2$, in $\RR^3$.  $W$ is subspace of $\RR^3$.  In the initial set up, $\vec{v}_1$ and $\vec{v}_2$ are orthogonal.  Vector $\vec{x}$ is not in $W$.  

Use check-boxes to construct the sum of orthogonal projections of $\vec{x}$ onto $\vec{v}_1$ and $\vec{v}_2$.  RIGHT-CLICK and DRAG to rotate the image.   

\pdfOnly{
Access GeoGebra interactives through the online version of this text at 

\href{https://ximera.osu.edu/oerlinalg}{https://ximera.osu.edu/oerlinalg}.
}

\begin{onlineOnly}
\begin{center}
\geogebra{hehqyayz}{950}{800}
\end{center}
\end{onlineOnly}

\begin{question}
If moved, return the basis vectors $\vec{v}_1$ and $\vec{v}_2$ to their default position (set $s_1=s_2=0$) to ensure that they are orthogonal.  

\begin{itemize}
\item Rotate the image to convince yourself that the perpendiculars dropped from the tip of $\vec{x}$ to $\vec{v}_1$ and $\vec{v}_2$ are indeed perpendicular to $\vec{v}_1$ and $\vec{v}_2$ in the diagram. (You'll have to look at it just right to convince yourself of this.)  Are both of these perpendiculars also necessarily perpendicular to the plane? \wordChoice{\choice{Yes}, \choice[correct]{No}}

\item Use sliders $x_1, x_2$ and $x_3$ to manipulate $\vec{x}$.  Rotate the figure for a better view.  What is true about about vector $\vec{p}$?
    
    \begin{multipleChoice}
 \choice{$\vec{p}=\vec{x}-(\mbox{proj}_{\vec{v}_1}\vec{x}+\mbox{proj}_{\vec{v}_2}\vec{x})$.}
 \choice{Vector $\vec{p}$ is orthogonal to $W$.}
 \choice[correct]{All of the above.}
 \end{multipleChoice}
 
  \item Rotate the figure so that you're looking directly down at the plane.  If you're looking at it correctly, you will notice that (1) the parallelogram determined by the projections of $\vec{x}$ onto $\vec{v}_1$ and $\vec{v}_2$ is a rectangle; (2) the sum of projections, $\mbox{proj}_{\vec{v}_1}\vec{x}+\mbox{proj}_{\vec{v}_2}\vec{x}$, is located directly underneath $\vec{x}$, like a shadow at midday.
 \end{itemize} 
 \end{question}

 \begin{question}
 Use sliders $s_1$ and $s_2$ to manipulate the basis vectors $\vec{v}_1$ and $\vec{v}_2$ so that they are no longer orthogonal.  

\begin{itemize}
 \item 
 Rotate the figure for a better view.  Which of the following is true?
 \begin{multipleChoice}
 \choice[correct]{$\vec{p}=\vec{x}-(\mbox{proj}_{\vec{v}_1}\vec{x}+\mbox{proj}_{\vec{v}_2}\vec{x})$.}
 \choice{Vector $\vec{p}$ is orthogonal to $W$.}
  \choice{All of the above.}
 \end{multipleChoice}
 \item
 Rotate your figure so that you're looking directly down at the plane. Which of the following is true?
 \begin{multipleChoice}
 \choice{Parallelogram determined by $\vec{v}_1$ and $\vec{v}_2$ is a rectangle.}
 \choice{$\mbox{proj}_{\vec{v}_1}\vec{x}+\mbox{proj}_{\vec{v}_2}\vec{x}$ is located directly underneath $\vec{x}$.}
  \choice[correct]{None of the above.}
 \end{multipleChoice}
\end{itemize}
\end{question}
\end{exploration}

In Exploration \ref{exp:orthProjSub}, you discovered that given a plane, spanned by orthogonal vectors $\vec{v}_1,\vec{v}_2$, in $\RR^3$, and a vector $\vec{x}$, not in the plane, we can interpret the sum of orthogonal projections of $\vec{x}$ onto $\vec{v}_1$ and $\vec{v}_2$ as a ``shadow" of $\vec{x}$ that lies in the plane directly underneath the vector $\vec{x}$. We say that this ``shadow" is an \dfn{orthogonal projection of $\vec{x}$ onto $W$}. You have also found that if $\vec{v}_1,\vec{v}_2$ are not orthogonal, the parallelogram representing the sum of the orthogonal projections of $\vec{x}$ onto $\vec{v}_1$ and $\vec{v}_2$ will not be a rectangle.  In this case, $\vec{x}$ minus this sum will NOT be orthogonal to the plane.  It is essential that $\vec{v}_1,\vec{v}_2$ are orthogonal for $\mbox{proj}_{\vec{v}_1}\vec{x}+\mbox{proj}_{\vec{v}_2}\vec{x}$ to be considered an orthogonal projection.  

In general, we can define an orthogonal projection of $\vec{x}$ in $\RR^n$ onto a subspace $W$ of $\RR^n$ as the sum of the orthogonal projections of $\vec{x}$ onto the elements of an orthogonal basis of $W$.  %We denote such a projection by $\mbox{proj}_V(\vec{x})$. An important aspect of this definition is that it allows us to express $\vec{x}$ as the sum of its orthogonal projection, $\vec{w}$, onto $W$ and a vector orthogonal to $\vec{w}$, called $\vec{w}^\perp$.  
Definition \ref{def:projOntoSubspace} and the subsequent diagram summarize this discussion.


\begin{definition}[Projection onto a Subspace of $\RR^n$]\label{def:projOntoSubspace}
Let $W$ be a subspace of $\RR^n$ with orthogonal basis $\{\vec{f}_{1}, \vec{f}_{2}, \dots, \vec{f}_{m}\}$. If $\vec{x}$ is in $\RR^n$, the vector
\begin{equation}\label{def:projectontoWeasy}
\vec{w}=\mbox{proj}_W(\vec{x}) = \mbox{proj}_{\vec{f}_1}\vec{x} + \mbox{proj}_{\vec{f}_2}\vec{x} + \dots + \mbox{proj}_{\vec{f}_m}\vec{x}
\end{equation}
is called the \dfn{orthogonal projection} of $\vec{x}$ onto $W$.  %The vector $\vec{x}-\vec{w}$ is called the \dfn{component of $\vec{x}$ orthogonal to $\vec{w}$}.  We will write $\vec{w}^\perp = \vec{x}-\vec{w}$.
\end{definition}

An illustration of Definition \ref{def:projOntoSubspace} for a two-dimensional subspace $W$ with orthogonal basis $\{\vec{f}_1,\vec{f}_2\}$ is shown below.
\begin{center}
\tdplotsetmaincoords{70}{130}
	\begin{tikzpicture}[scale=0.8]
\filldraw[blue, opacity=0.2] (0,0,0)--(5,0,0)--(5,0,5)--(0,0,5)--cycle;
\draw[->,line width=0.4mm, -stealth, blue](0,0,0)--(6,0,0) ;
\draw[->,line width=0.4mm, -stealth, blue](0,0,0)--(0,0,6) ;
\node[label={above:$\vec{f}_1$}] at (6,0,0) {};
\node[label={above:$W$}] at (5.2,0,5) {};
\node[label={left:$\vec{f}_2$}] at (0,0,6) {};
\node[label={above:$\vec{x}$}] at (2,1.5,2) {};
\node[label={above:$\vec{w}$}] at (2.2,0,3.5) {};
\draw[-,line width=0.2mm, dashed](4,3,4)--(4,0,4) ;
    \draw[-,line width=0.2mm, dashed](0,0,4)--(4,0,4) ;
    \draw[-,line width=0.2mm, dashed](4,0,4)--(4,0,0) ;
    \draw[->,line width=1.5mm, -stealth, black,opacity=0.4](0,0,0)--(0,0,4) ;
    \draw[->,line width=1.5mm, -stealth, black,opacity=0.4](0,0,0)--(4,0,0) ;
   \draw[->,line width=0.8mm, -stealth, red](0,0,0)--(4,3,4) ;
    \draw[->,line width=1.5mm, -stealth, red,opacity=0.2](0,0,0)--(4,0,4) ;
    \node[label={below:$\vec{w}=\mbox{proj}_W\vec{x}=\mbox{proj}_{\vec{f}_1}\vec{x}+\mbox{proj}_{\vec{f}_2}\vec{x}$}] at (3,-1,3) {};
      \end{tikzpicture}
\end{center}

Using equation (\ref{eq:orthProj}) multiple times, we can also express $\vec{w}$ in Definition \ref{def:projOntoSubspace} using the following formula.

\begin{formula}\label{form:orthProjOntoW}
\begin{equation}\label{def:projectontoW}
\vec{w} = \mbox{proj}_W(\vec{x}) =\frac{\vec{x} \dotp \vec{f}_{1}}{\norm{\vec{f}_{1}}^2}\vec{f}_{1} + \frac{\vec{x} \dotp \vec{f}_{2}}{\norm{\vec{f}_{2}}^2}\vec{f}_{2}+ \dots +\frac{\vec{x} \dotp \vec{f}_{m}}{\norm{\vec{f}_{m}}^2}\vec{f}_{m}
\end{equation}
\end{formula}

\subsection*{Orthogonal Decomposition of $\vec{x}$}
Definition \ref{def:projOntoSubspace} allows us to express $\vec{x}$ as the sum of its orthogonal projection, $\vec{w}=\mbox{proj}_W\vec{x}$, located in $W$, and a vector we will call $\vec{w}^\perp$ (pronounced ``W-perp"), given by $\vec{w}^\perp=\vec{x}-\vec{w}$. This decomposition of $\vec{x}$ is shown in the diagram below.  
\begin{center}
\tdplotsetmaincoords{70}{130}
	\begin{tikzpicture}[scale=0.8]
\filldraw[blue, opacity=0.2] (0,0,0)--(5,0,0)--(5,0,5)--(0,0,5)--cycle;
\node[label={above:$\vec{x}$}] at (2,1.5,2) {};
\node[label={above:$\vec{w}$}] at (2.2,0,3.5) {};
\node[label={above:$\vec{w}^{\perp}$}] at (5.3,2.2,6) {};
\draw[->,line width=0.8mm, -stealth, red](0,0,0)--(4,3,4) ;
    \draw[->,line width=1.5mm, -stealth, red,opacity=0.2](0,0,0)--(4,0,4) ;
\draw[->,line width=1.5mm, -stealth, red,opacity=0.2](4,0,4)--(4,3,4) ;
    \node[label={below:$\vec{x}=\vec{w}+\vec{w}^{\perp}$}] at (3,-1,3) {};
    \node[label={above:$W$}] at (5,0,1.9) {};
      \end{tikzpicture}
          
     \end{center}
You have already met $\vec{w}^\perp$, under the name of $\vec{p}$ in Exploration \ref{exp:orthProjSub}, and observed that this vector is orthogonal to $W$. We will now prove that $\vec{w}^\perp$ is orthogonal to every vector in $W$.  This will be accomplished in two steps.  First, in Theorem \ref{th:orthDecompX} we will prove that $\vec{w}^\perp$ is orthogonal to all of the basis elements of $W$. Next, you will use this result to demonstrate that $\vec{w}^\perp$ is orthogonal to every vector in $W$.

\begin{theorem}\label{th:orthDecompX}
Let $W$ be a subspace of $\RR^n$ with orthogonal basis $\{\vec{f}_{1}, \vec{f}_{2}, \dots, \vec{f}_{m}\}$. Let $\vec{x}$ be in $\RR^n$, and define $\vec{w}^\perp$ as
\begin{equation*}
\vec{w}^\perp=\vec{x}-\mbox{proj}_W\vec{x} = \vec{x}-(\mbox{proj}_{\vec{f}_1}\vec{x} + \mbox{proj}_{\vec{f}_2}\vec{x} + \dots + \mbox{proj}_{\vec{f}_m}\vec{x})
\end{equation*}
Then $\vec{w}^\perp$ is orthogonal to $\vec{f}_i$ for $1\leq i\leq m$.
\end{theorem}
\begin{proof}
We will use Formula \ref{form:orthProjOntoW} to show that $\vec{w}^\perp\cdot \vec{f}_i$=0.  Recall that $\{\vec{f}_{1}, \vec{f}_{2}, \dots, \vec{f}_{m}\}$ is an orthogonal basis.  Therefore $\vec{f}_j\dotp\vec{f}_i=0$ for $i\neq j$.  This observation enables us to compute as follows.
\begin{eqnarray*}
\vec{w}^\perp\cdot \vec{f}_i&=&\left[\vec{x}-\left(\frac{\vec{x} \dotp \vec{f}_{1}}{\norm{\vec{f}_{1}}^2}\vec{f}_{1} +\dots + \frac{\vec{x} \dotp \vec{f}_{i}}{\norm{\vec{f}_{i}}^2}\vec{f}_{i}+ \dots +\frac{\vec{x} \dotp \vec{f}_{m}}{\norm{\vec{f}_{m}}^2}\vec{f}_{m}\right)\right]\cdot \vec{f}_i\\
& =& \vec{x}\dotp \vec{f}_i- \frac{\vec{x} \dotp \vec{f}_{i}}{\norm{\vec{f}_{i}}^2}(\vec{f}_{i}\dotp\vec{f}_i)\\
&=& \vec{x}\dotp \vec{f}_i- \frac{\vec{x} \dotp \vec{f}_{i}}{\norm{\vec{f}_{i}}^2}\norm{\vec{f}_{i}}^2=\vec{x}\dotp \vec{f}_i-\vec{x}\dotp \vec{f}_i=0
\end{eqnarray*}
\end{proof}

We leave the proof of the following Corollary as Practice Problem \ref{prob:proofCor}
\begin{corollary}\label{cor:orthProjOntoW}
Let $W$ be a subspace of $\RR^n$ with orthogonal basis $\{\vec{f}_{1}, \vec{f}_{2}, \dots, \vec{f}_{m}\}$. Let $\vec{x}$ be in $\RR^n$, and define $\vec{w}^\perp$ as
\begin{equation*}
\vec{w}^\perp=\vec{x}-\mbox{proj}_W\vec{x} = \vec{x}-(\mbox{proj}_{\vec{f}_1}\vec{x} + \mbox{proj}_{\vec{f}_2}\vec{x} + \dots + \mbox{proj}_{\vec{f}_m}\vec{x})
\end{equation*}
Then $\vec{w}^\perp$ is orthogonal to every vector in $W$.
\end{corollary}



The fact that the decomposition of $\vec{x}$ into the sum of $\vec{w}$ and $\vec{w}^\perp$ is unique is the subject of the Orthogonal Decomposition Theorem which we will prove in \href{https://ximera.osu.edu/oerlinalg/LinearAlgebra/RTH-0020/main}{Orthogonal Complements and Decompositions}.

Throughout this section we have worked with orthogonal bases of subspaces.  Does every subspace of $\RR^n$ have an orthogonal basis?  If so, how do we find one?  These questions will be addressed in \href{https://ximera.osu.edu/oerlinalg/LinearAlgebra/RTH-0015/main}{Gram-Schmidt Orthogonalization}.

\section*{Practice Problems}
\begin{problem}\label{prob:rref_way}
Retry Example~\ref{fourier} using Gaussian elimination.  Which method seems easier to you?
\end{problem}

\begin{problem}\label{prob:vec_eq_0}
    Let $\vec{x}_1, \vec{x}_2, \ldots, \vec{x}_k\in\RR^n$ and
suppose $\mbox{span}\{\vec{x}_1, \vec{x}_2, \ldots, \vec{x}_k\}=\RR^n$.
Furthermore, suppose that there exists a vector $\vec{v}\in\RR^n$ for which $\vec{v}\dotp \vec{x}_j=0$ for all $j$, $1\leq j\leq k$.
Show that $\vec{v}=\vec{0}$.
\end{problem}

\emph{Problems \ref{OrthoProj1.1}-\ref{OrthoProj1.3}}

Let $\vec{x} = \begin{bmatrix}1\\ -2\\ 1\\ 6\end{bmatrix}$ in $\RR^4$, and let $W = \mbox{span}\left(\begin{bmatrix}2\\ 1\\ 3\\ -4\end{bmatrix}, \begin{bmatrix}1\\ 2\\ 0\\ 1\end{bmatrix}\right)$.

\begin{problem}\label{OrthoProj1.1}
Compute $\mbox{proj}_W(\vec{x})$.

Answer:  $$\frac{1}{10}\begin{bmatrix}\answer{-9}\\\answer{3}\\\answer{-21}\\\answer{33}\end{bmatrix}$$
\end{problem}

\begin{problem}\label{OrthoProj1.2}
Show that $\left\{\begin{bmatrix}1\\ 0\\ 2\\ -3\end{bmatrix}, \begin{bmatrix}4\\ 7\\ 1\\ 2\end{bmatrix}\right\}$ is another orthogonal basis of $W$.
\end{problem}

\begin{problem}\label{OrthoProj1.3}
Use the basis in Problem \ref{OrthoProj1.2} to compute $\mbox{proj}_W(\vec{x})$.

Answer:  $$\frac{1}{70}\begin{bmatrix}\answer{-63}\\\answer{21}\\\answer{-147}\\\answer{231}\end{bmatrix}$$
\end{problem}


\begin{problem}\label{prob:proofCor}
Prove Corollary \ref{cor:orthProjOntoW}
\end{problem}
  
\section*{Text Source}
A portion of the text in this section is an adaptation of Section 4.11.1 of Ken Kuttler's \href{https://open.umn.edu/opentextbooks/textbooks/a-first-course-in-linear-algebra-2017}{\it A First Course in Linear Algebra}. (CC-BY)

Ken Kuttler, {\it  A First Course in Linear Algebra}, Lyryx 2017, Open Edition, p. 233-238.  

\end{document}
