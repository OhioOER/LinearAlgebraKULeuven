\documentclass{ximera}
%%% Begin Laad packages

\makeatletter
\@ifclassloaded{xourse}{%
    \typeout{Start loading preamble.tex (in a XOURSE)}%
    \def\isXourse{true}   % automatically defined; pre 112022 it had to be set 'manually' in a xourse
}{%
    \typeout{Start loading preamble.tex (NOT in a XOURSE)}%
}
\makeatother

\def\isEn\true 

\pgfplotsset{compat=1.16}

\usepackage{currfile}

% 201908/202301: PAS OP: babel en doclicense lijken problemen te veroorzaken in .jax bestand
% (wegens syntax error met toegevoegde \newcommands ...)
\pdfOnly{
    \usepackage[type={CC},modifier={by-nc-sa},version={4.0}]{doclicense}
    %\usepackage[hyperxmp=false,type={CC},modifier={by-nc-sa},version={4.0}]{doclicense}
    %%% \usepackage[dutch]{babel}
}



\usepackage[utf8]{inputenc}
\usepackage{morewrites}   % nav zomercursus (answer...?)
\usepackage{multirow}
\usepackage{multicol}
\usepackage{tikzsymbols}
\usepackage{ifthen}
%\usepackage{animate} BREAKS HTML STRUCTURE USED BY XIMERA
\usepackage{relsize}

\usepackage{eurosym}    % \euro  (€ werkt niet in xake ...?)
\usepackage{fontawesome} % smileys etc

% Nuttig als ook interactieve beamer slides worden voorzien:
\providecommand{\p}{} % default nothing ; potentially usefull for slides: redefine as \pause
%providecommand{\p}{\pause}

    % Layout-parameters voor het onderschrift bij figuren
\usepackage[margin=10pt,font=small,labelfont=bf, labelsep=endash,format=hang]{caption}
%\usepackage{caption} % captionof
%\usepackage{pdflscape}    % landscape environment

% Met "\newcommand\showtodonotes{}" kan je todonotes tonen (in pdf/online)
% 201908: online werkt het niet (goed)
\providecommand\showtodonotes{disable}
\providecommand\todo[1]{\typeout{TODO #1}}
%\usepackage[\showtodonotes]{todonotes}
%\usepackage{todonotes}

%
% Poging tot aanpassen layout
%
\usepackage{tcolorbox}
\tcbuselibrary{theorems}

%%% Einde laad packages

%%% Begin Ximera specifieke zaken

\graphicspath{
	{../../}
	{../}
	{./}
  	{../../pictures/}
   	{../pictures/}
   	{./pictures/}
	{./explog/}    % M05 in groeimodellen       
}

%%% Einde Ximera specifieke zaken

%
% define softer blue/red/green, use KU Leuven base colors for blue (and dark orange for red ?)
%
% todo: rather redefine blue/red/green ...?
%\definecolor{xmblue}{rgb}{0.01, 0.31, 0.59}
%\definecolor{xmred}{rgb}{0.89, 0.02, 0.17}
\definecolor{xmdarkblue}{rgb}{0.122, 0.671, 0.835}   % KU Leuven Blauw
\definecolor{xmblue}{rgb}{0.114, 0.553, 0.69}        % KU Leuven Blauw
\definecolor{xmgreen}{rgb}{0.13, 0.55, 0.13}         % No KULeuven variant for green found ...

\definecolor{xmaccent}{rgb}{0.867, 0.541, 0.18}      % KU Leuven Accent (orange ...)
\definecolor{kuaccent}{rgb}{0.867, 0.541, 0.18}      % KU Leuven Accent (orange ...)

\colorlet{xmred}{xmaccent!50!black}                  % Darker version of KU Leuven Accent

\providecommand{\blue}[1]{{\color{blue}#1}}    
\providecommand{\red}[1]{{\color{red}#1}}

\renewcommand\CancelColor{\color{xmaccent!50!black}}

% werkt in math en text mode om MATH met oranje (of grijze...)  achtergond te tonen (ook \important{\text{blabla}} lijkt te werken)
%\newcommand{\important}[1]{\ensuremath{\colorbox{xmaccent!50!white}{$#1$}}}   % werkt niet in Mathjax
%\newcommand{\important}[1]{\ensuremath{\colorbox{lightgray}{$#1$}}}
\newcommand{\important}[1]{\ensuremath{\colorbox{orange}{$#1$}}}   % TODO: kleur aanpassen voor mathjax; wordt overschreven infra!


% Uitzonderlijk kan met \pdfnl in de PDF een newline worden geforceerd, die online niet nodig/nuttig is omdat daar de regellengte hoe dan ook niet gekend is.
\ifdefined\HCode%
\providecommand{\pdfnl}{}%
\else%
\providecommand{\pdfnl}{%
  \\%
}%
\fi

% Uitzonderlijk kan met \handoutnl in de handout-PDF een newline worden geforceerd, die noch online noch in de PDF-met-antwoorden nuttig is.
\ifdefined\HCode
\providecommand{\handoutnl}{}
\else
\providecommand{\handoutnl}{%
\ifhandout%
  \nl%
\fi%
}
\fi



% \cellcolor IGNORED by tex4ht ?
% \begin{center} seems not to wordk
    % (missing margin-left: auto;   on tabular-inside-center ???)
%\newcommand{\importantcell}[1]{\ensuremath{\cellcolor{lightgray}#1}}  %  in tabular; usablility to be checked
\providecommand{\importantcell}[1]{\ensuremath{#1}}     % no mathjax2 support for colloring array cells

\pdfOnly{
  \renewcommand{\important}[1]{\ensuremath{\colorbox{kuaccent!50!white}{$#1$}}}
  \renewcommand{\importantcell}[1]{\ensuremath{\cellcolor{kuaccent!40!white}#1}}   
}

%%% Tikz styles


\pgfplotsset{compat=1.16}

\usetikzlibrary{trees,positioning,arrows,fit,shapes,math,calc,decorations.markings,through,intersections,patterns,matrix}

\usetikzlibrary{decorations.pathreplacing,backgrounds}    % 5/2023: from experimental


\usetikzlibrary{angles,quotes}

\usepgfplotslibrary{fillbetween} % bepaalde_integraal
\usepgfplotslibrary{polar}    % oa voor poolcoordinaten.tex

\pgfplotsset{ownstyle/.style={axis lines = center, axis equal image, xlabel = $x$, ylabel = $y$, enlargelimits}} 

\pgfplotsset{
	plot/.style={no marks,samples=50}
}

\newcommand{\xmPlotsColor}{
	\pgfplotsset{
		plot1/.style={darkgray,no marks,samples=100},
		plot2/.style={lightgray,no marks,samples=100},
		plotresult/.style={blue,no marks,samples=100},
		plotblue/.style={blue,no marks,samples=100},
		plotred/.style={red,no marks,samples=100},
		plotgreen/.style={green,no marks,samples=100},
		plotpurple/.style={purple,no marks,samples=100}
	}
}
\newcommand{\xmPlotsBlackWhite}{
	\pgfplotsset{
		plot1/.style={black,loosely dashed,no marks,samples=100},
		plot2/.style={black,loosely dotted,no marks,samples=100},
		plotresult/.style={black,no marks,samples=100},
		plotblue/.style={black,no marks,samples=100},
		plotred/.style={black,dotted,no marks,samples=100},
		plotgreen/.style={black,dashed,no marks,samples=100},
		plotpurple/.style={black,dashdotted,no marks,samples=100}
	}
}


\newcommand{\xmPlotsColorAndStyle}{
	\pgfplotsset{
		plot1/.style={darkgray,no marks,samples=100},
		plot2/.style={lightgray,no marks,samples=100},
		plotresult/.style={blue,no marks,samples=100},
		plotblue/.style={xmblue,no marks,samples=100},
		plotred/.style={xmred,dashed,thick,no marks,samples=100},
		plotgreen/.style={xmgreen,dotted,very thick,no marks,samples=100},
		plotpurple/.style={purple,no marks,samples=100}
	}
}


%\iftikzexport
\xmPlotsColorAndStyle
%\else
%\xmPlotsBlackWhite
%\fi
%%%


%
% Om venndiagrammen te arceren ...
%
\makeatletter
\pgfdeclarepatternformonly[\hatchdistance,\hatchthickness]{north east hatch}% name
{\pgfqpoint{-1pt}{-1pt}}% below left
{\pgfqpoint{\hatchdistance}{\hatchdistance}}% above right
{\pgfpoint{\hatchdistance-1pt}{\hatchdistance-1pt}}%
{
	\pgfsetcolor{\tikz@pattern@color}
	\pgfsetlinewidth{\hatchthickness}
	\pgfpathmoveto{\pgfqpoint{0pt}{0pt}}
	\pgfpathlineto{\pgfqpoint{\hatchdistance}{\hatchdistance}}
	\pgfusepath{stroke}
}
\pgfdeclarepatternformonly[\hatchdistance,\hatchthickness]{north west hatch}% name
{\pgfqpoint{-\hatchthickness}{-\hatchthickness}}% below left
{\pgfqpoint{\hatchdistance+\hatchthickness}{\hatchdistance+\hatchthickness}}% above right
{\pgfpoint{\hatchdistance}{\hatchdistance}}%
{
	\pgfsetcolor{\tikz@pattern@color}
	\pgfsetlinewidth{\hatchthickness}
	\pgfpathmoveto{\pgfqpoint{\hatchdistance+\hatchthickness}{-\hatchthickness}}
	\pgfpathlineto{\pgfqpoint{-\hatchthickness}{\hatchdistance+\hatchthickness}}
	\pgfusepath{stroke}
}
%\makeatother

\tikzset{
    hatch distance/.store in=\hatchdistance,
    hatch distance=10pt,
    hatch thickness/.store in=\hatchthickness,
   	hatch thickness=2pt
}

\colorlet{circle edge}{black}
\colorlet{circle area}{blue!20}


\tikzset{
    filled/.style={fill=green!30, draw=circle edge, thick},
    arceerl/.style={pattern=north east hatch, pattern color=blue!50, draw=circle edge},
    arceerr/.style={pattern=north west hatch, pattern color=yellow!50, draw=circle edge},
    outline/.style={draw=circle edge, thick}
}




%%% Updaten commando's
\def\hoofding #1#2#3{\maketitle}     % OBSOLETE ??

% we willen (bijna) altijd \geqslant ipv \geq ...!
\newcommand{\geqnoslant}{\geq}
\renewcommand{\geq}{\geqslant}
\newcommand{\leqnoslant}{\leq}
\renewcommand{\leq}{\leqslant}

% Todo: (201908) waarom komt er (soms) underlined voor emph ...?
\renewcommand{\emph}[1]{\textit{#1}}

% API commando's

\newcommand{\ds}{\displaystyle}
\newcommand{\ts}{\textstyle}  % tegenhanger van \ds   (Ximera zet PER  DEFAULT \ds!)

% uit Zomercursus-macro's: 
\newcommand{\bron}[1]{\begin{scriptsize} \emph{#1} \end{scriptsize}}     % deprecated ...?


%definities nieuwe commando's - afkortingen veel gebruikte symbolen
\newcommand{\R}{\ensuremath{\mathbb{R}}}
\newcommand{\Rnul}{\ensuremath{\mathbb{R}_0}}
\newcommand{\Reen}{\ensuremath{\mathbb{R}\setminus\{1\}}}
\newcommand{\Rnuleen}{\ensuremath{\mathbb{R}\setminus\{0,1\}}}
\newcommand{\Rplus}{\ensuremath{\mathbb{R}^+}}
\newcommand{\Rmin}{\ensuremath{\mathbb{R}^-}}
\newcommand{\Rnulplus}{\ensuremath{\mathbb{R}_0^+}}
\newcommand{\Rnulmin}{\ensuremath{\mathbb{R}_0^-}}
\newcommand{\Rnuleenplus}{\ensuremath{\mathbb{R}^+\setminus\{0,1\}}}
\newcommand{\N}{\ensuremath{\mathbb{N}}}
\newcommand{\Nnul}{\ensuremath{\mathbb{N}_0}}
\newcommand{\Z}{\ensuremath{\mathbb{Z}}}
\newcommand{\Znul}{\ensuremath{\mathbb{Z}_0}}
\newcommand{\Zplus}{\ensuremath{\mathbb{Z}^+}}
\newcommand{\Zmin}{\ensuremath{\mathbb{Z}^-}}
\newcommand{\Znulplus}{\ensuremath{\mathbb{Z}_0^+}}
\newcommand{\Znulmin}{\ensuremath{\mathbb{Z}_0^-}}
\newcommand{\C}{\ensuremath{\mathbb{C}}}
\newcommand{\Cnul}{\ensuremath{\mathbb{C}_0}}
\newcommand{\Cplus}{\ensuremath{\mathbb{C}^+}}
\newcommand{\Cmin}{\ensuremath{\mathbb{C}^-}}
\newcommand{\Cnulplus}{\ensuremath{\mathbb{C}_0^+}}
\newcommand{\Cnulmin}{\ensuremath{\mathbb{C}_0^-}}
\newcommand{\Q}{\ensuremath{\mathbb{Q}}}
\newcommand{\Qnul}{\ensuremath{\mathbb{Q}_0}}
\newcommand{\Qplus}{\ensuremath{\mathbb{Q}^+}}
\newcommand{\Qmin}{\ensuremath{\mathbb{Q}^-}}
\newcommand{\Qnulplus}{\ensuremath{\mathbb{Q}_0^+}}
\newcommand{\Qnulmin}{\ensuremath{\mathbb{Q}_0^-}}

\newcommand{\perdef}{\overset{\mathrm{def}}{=}}
\newcommand{\pernot}{\overset{\mathrm{notatie}}{=}}
\newcommand\perinderdaad{\overset{!}{=}}     % voorlopig gebruikt in limietenrekenregels
\newcommand\perhaps{\overset{?}{=}}          % voorlopig gebruikt in limietenrekenregels

\newcommand{\degree}{^\circ}


\DeclareMathOperator{\dom}{dom}     % domein
\DeclareMathOperator{\codom}{codom} % codomein
\DeclareMathOperator{\bld}{bld}     % beeld
\DeclareMathOperator{\graf}{graf}   % grafiek
\DeclareMathOperator{\rico}{rico}   % richtingcoëfficient
\DeclareMathOperator{\co}{co}       % coordinaat
\DeclareMathOperator{\gr}{gr}       % graad

\newcommand{\func}[5]{\ensuremath{#1: #2 \rightarrow #3: #4 \mapsto #5}} % Easy to write a function


% Operators
\DeclareMathOperator{\bgsin}{bgsin}
\DeclareMathOperator{\bgcos}{bgcos}
\DeclareMathOperator{\bgtan}{bgtan}
\DeclareMathOperator{\bgcot}{bgcot}
\DeclareMathOperator{\bgsinh}{bgsinh}
\DeclareMathOperator{\bgcosh}{bgcosh}
\DeclareMathOperator{\bgtanh}{bgtanh}
\DeclareMathOperator{\bgcoth}{bgcoth}

% Oude \Bgsin etc deprecated: gebruik \bgsin, en herdefinieer dat als je Bgsin wil!
%\DeclareMathOperator{\cosec}{cosec}    % not used? gebruik \csc en herdefinieer

% operatoren voor differentialen: to be verified; 1/2020: inconsequent gebruik bij afgeleiden/integralen
\renewcommand{\d}{\mathrm{d}}
\newcommand{\dx}{\d x}
\newcommand{\dd}[1]{\frac{\mathrm{d}}{\mathrm{d}#1}}
\newcommand{\ddx}{\dd{x}}

% om in voorbeelden/oefeningen de notatie voor afgeleiden te kunnen kiezen
% Usage: \afg{(2\sin(x))}  (en wordt d/dx, of accent, of D )
%\newcommand{\afg}[1]{{#1}'}
\newcommand{\afg}[1]{\left(#1\right)'}
%\renewcommand{\afg}[1]{\frac{\mathrm{d}#1}{\mathrm{d}x}}   % include in relevant exercises ...
%\renewcommand{\afg}[1]{D{#1}}

%
% \xmxxx commands: Extra KU Leuven functionaliteit van, boven of naast Ximera
%   ( Conventie 8/2019: xm+nederlandse omschrijving, maar is niet consequent gevolgd, en misschien ook niet erg handig !)
%
% (Met een minimale ximera.cls en preamble.tex zou een bruikbare .pdf moeten kunnen worden gemaakt van eender welke ximera)
%
% Usage: \xmtitle[Mijn korte abstract]{Mijn titel}{Mijn abstract}
% Eerste command na \begin{document}:
%  -> definieert de \title
%  -> definieert de abstract
%  -> doet \maketitle ( dus: print de hoofding als 'chapter' of 'sectie')
% Optionele parameter geeft eenn kort abstract (die met de globale setting \xmshortabstract{} al dan niet kan worden geprint.
% De optionele korte abstract kan worden gebruikt voor pseudo-grappige abtsarts, dus dus globaal al dan niet kunnen worden gebuikt...
% Globale settings:
%  de (optionele) 'korte abstract' wordt enkele getoond als \xmshortabstract is gezet
\providecommand\xmshortabstract{} % default: print (only!) short abstract if present
\newcommand{\xmtitle}[3][]{
	\title{#2}
	\begin{abstract}
		\ifdefined\xmshortabstract
		\ifstrempty{#1}{%
			#3
		}{%
			#1
		}%
		\else
		#3
		\fi
	\end{abstract}
	\maketitle
}

% 
% Kleine grapjes: moeten zonder verder gevolg kunnen worden verwijderd
%
%\newcommand{\xmopje}[1]{{\small#1{\reversemarginpar\marginpar{\Smiley}}}}   % probleem in floats!!
\newtoggle{showxmopje}
\toggletrue{showxmopje}

\newcommand{\xmopje}[1]{%
   \iftoggle{showxmopje}{#1}{}%
}


% -> geef een abstracte-formule-met-rechts-een-concreet-voorbeeld
% VB:  \formulevb{a^2+b^2=c^2}{3^2+4^2=5^2}
%
\ifdefined\HCode
\NewEnviron{xmdiv}[1]{\HCode{\Hnewline<div class="#1">\Hnewline}\BODY{\HCode{\Hnewline</div>\Hnewline}}}
\else
\NewEnviron{xmdiv}[1]{\BODY}
\fi

\providecommand{\formulevb}[2]{
	{\centering

    \begin{xmdiv}{xmformulevb}    % zie css voor online layout !!!
	\begin{tabular}{lcl}
		\important{#1}
		&  &
		Vb: $#2$
		\end{tabular}
	\end{xmdiv}

	}
}

\ifdefined\HCode
\providecommand{\vb}[1]{%
    \HCode{\Hnewline<span class="xmvb">}#1\HCode{</span>\Hnewline}%
}
\else
\providecommand{\vb}[1]{
    \colorbox{blue!10}{#1}
}
\fi

\ifdefined\HCode
\providecommand{\xmcolorbox}[2]{
	\HCode{\Hnewline<div class="xmcolorbox">\Hnewline}#2\HCode{\Hnewline</div>\Hnewline}
}
\else
\providecommand{\xmcolorbox}[2]{
  \cellcolor{#1}#2
}
\fi


\ifdefined\HCode
\providecommand{\xmopmerking}[1]{
 \HCode{\Hnewline<div class="xmopmerking">\Hnewline}#1\HCode{\Hnewline</div>\Hnewline}
}
\else
\providecommand{\xmopmerking}[1]{
	{\footnotesize #1}
}
\fi
% \providecommand{\voorbeeld}[1]{
% 	\colorbox{blue!10}{$#1$}
% }



% Hernoem Proof naar Bewijs, nodig voor HTML versie
\renewcommand*{\proofname}{Bewijs}

% Om opgave van oefening (wordt niet geprint bij oplossingenblad)
% (to be tested test)
\NewEnviron{statement}{\BODY}

% Environment 'oplossing' en 'uitkomst'
% voor resp. volledige 'uitwerking' dan wel 'enkel eindresultaat'
% geimplementeerd via feedback, omdat er in de ximera-server adhoc feedback-code is toegevoegd
%% Niet tonen indien handout
%% Te gebruiken om volledige oplossingen/uitwerkingen van oefeningen te tonen
%% \begin{oplossing}        De optelling is commutatief \end{oplossing}  : verschijnt online enkel 'op vraag'
%% \begin{oplossing}[toon]  De optelling is commutatief \end{oplossing}  : verschijnt steeds onmiddellijk online (bv te gebruiken bij voorbeelden) 

\ifhandout%
    \NewEnviron{oplossing}[1][onzichtbaar]%
    {%
    \ifthenelse{\equal{\detokenize{#1}}{\detokenize{toon}}}
    {
    \def\PH@Command{#1}% Use PH@Command to hold the content and be a target for "\expandafter" to expand once.

    \begin{trivlist}% Begin the trivlist to use formating of the "Feedback" label.
    \item[\hskip \labelsep\small\slshape\bfseries Oplossing% Format the "Feedback" label. Don't forget the space.
    %(\texttt{\detokenize\expandafter{\PH@Command}}):% Format (and detokenize) the condition for feedback to trigger
    \hspace{2ex}]\small%\slshape% Insert some space before the actual feedback given.
    \BODY
    \end{trivlist}
    }
    {  % \begin{feedback}[solution]   \BODY     \end{feedback}  }
    }
    }    
\else
% ONLY for HTML; xmoplossing is styled with css, and is not, and need not be a LaTeX environment
% THUS: it does NOT use feedback anymore ...
%    \NewEnviron{oplossing}{\begin{expandable}{xmoplossing}{\nlen{Toon uitwerking}{Show solution}}{\BODY}\end{expandable}}
    \newenvironment{oplossing}[1][onzichtbaar]
   {%
       \begin{expandable}{xmoplossing}{}
   }
   {%
   	   \end{expandable}
   } 
%     \newenvironment{oplossing}[1][onzichtbaar]
%    {%
%        \begin{feedback}[solution]   	
%    }
%    {%
%    	   \end{feedback}
%    } 
\fi

\ifhandout%
    \NewEnviron{uitkomst}[1][onzichtbaar]%
    {%
    \ifthenelse{\equal{\detokenize{#1}}{\detokenize{toon}}}
    {
    \def\PH@Command{#1}% Use PH@Command to hold the content and be a target for "\expandafter" to expand once.

    \begin{trivlist}% Begin the trivlist to use formating of the "Feedback" label.
    \item[\hskip \labelsep\small\slshape\bfseries Uitkomst:% Format the "Feedback" label. Don't forget the space.
    %(\texttt{\detokenize\expandafter{\PH@Command}}):% Format (and detokenize) the condition for feedback to trigger
    \hspace{2ex}]\small%\slshape% Insert some space before the actual feedback given.
    \BODY
    \end{trivlist}
    }
    {  % \begin{feedback}[solution]   \BODY     \end{feedback}  }
    }
    }    
\else
\ifdefined\HCode
   \newenvironment{uitkomst}[1][onzichtbaar]
    {%
        \begin{expandable}{xmuitkomst}{}%
    }
    {%
    	\end{expandable}%
    } 
\else
  % Do NOT print 'uitkomst' in non-handout
  %  (presumably, there is also an 'oplossing' ??)
  \newenvironment{uitkomst}[1][onzichtbaar]{}{}
\fi
\fi

%
% Uitweidingen zijn extra's die niet redelijkerwijze tot de leerstof behoren
% Uitbreidingen zijn extra's die wel redelijkerwijze tot de leerstof van bv meer geavanceerde versies kunnen behoren (B-programma/Wiskundestudenten/...?)
% Nog niet voorzien: design voor verschillende versies (A/B programma, BIO, voorkennis/ ...)
% Voor 'uitweidingen' is er een environment die online per default is ingeklapt, en in pdf al dan niet kan worden geincluded  (via \xmnouitweiding) 
%
% in een xourse, per default GEEN uitweidingen, tenzij \xmuitweiding expliciet ergens is gezet ...
\ifdefined\isXourse
   \ifdefined\xmuitweiding
   \else
       \def\xmnouitweiding{true}
   \fi
\fi

\ifdefined\xmnouitweiding
\newcounter{xmuitweiding}  % anders error undefined ...  
\excludecomment{xmuitweiding}
\else
\newtheoremstyle{dotless}{}{}{}{}{}{}{ }{}
\theoremstyle{dotless}
\newtheorem*{xmuitweidingnofrills}{}   % nofrills = no accordion; gebruikt dus de dotless theoremstyle!

\newcounter{xmuitweiding}
\newenvironment{xmuitweiding}[1][ ]%
{% 
	\refstepcounter{xmuitweiding}%
    \begin{expandable}{xmuitweiding}{\nlentext{Uitweiding \arabic{xmuitweiding}: #1}{Digression \arabic{xmuitweiding}: #1}}%
	\begin{xmuitweidingnofrills}%
}
{%
    \end{xmuitweidingnofrills}%
    \end{expandable}%
}   
% \newenvironment{xmuitweiding}[1][ ]%
% {% 
% 	\refstepcounter{xmuitweiding}
% 	\begin{accordion}\begin{accordion-item}[Uitweiding \arabic{xmuitweiding}: #1]%
% 			\begin{xmuitweidingnofrills}%
% 			}
% 			{\end{xmuitweidingnofrills}\end{accordion-item}\end{accordion}}   
\fi


\newenvironment{xmexpandable}[1][]{
	\begin{accordion}\begin{accordion-item}[#1]%
		}{\end{accordion-item}\end{accordion}}


% Command that gives a selection box online, but just prints the right answer in pdf
\newcommand{\xmonlineChoice}[1]{\pdfOnly{\wordchoicegiventrue}\wordChoice{#1}\pdfOnly{\wordchoicegivenfalse}}   % deprecated, gebruik onlineChoice ...
\newcommand{\onlineChoice}[1]{\pdfOnly{\wordchoicegiventrue}\wordChoice{#1}\pdfOnly{\wordchoicegivenfalse}}

\newcommand{\TJa}{\nlentext{ Ja }{ Yes }}
\newcommand{\TNee}{\nlentext{ Nee }{ No }}
\newcommand{\TJuist}{\nlentext{ Juist }{ True }}
\newcommand{\TFout}{\nlentext{ Fout }{ False }}

\newcommand{\choiceTrue }{{\renewcommand{\choiceminimumhorizontalsize}{4em}\wordChoice{\choice[correct]{\TJuist}\choice{\TFout}}}}
\newcommand{\choiceFalse}{{\renewcommand{\choiceminimumhorizontalsize}{4em}\wordChoice{\choice{\TJuist}\choice[correct]{\TFout}}}}

\newcommand{\choiceYes}{{\renewcommand{\choiceminimumhorizontalsize}{3em}\wordChoice{\choice[correct]{\TJa}\choice{\TNee}}}}
\newcommand{\choiceNo }{{\renewcommand{\choiceminimumhorizontalsize}{3em}\wordChoice{\choice{\TJa}\choice[correct]{\TNee}}}}

% Optional nicer formatting for wordChoice in PDF

\let\inlinechoiceorig\inlinechoice

%\makeatletter
%\providecommand{\choiceminimumverticalsize}{\vphantom{$\frac{\sqrt{2}}{2}$}}   % minimum height of boxes (cfr infra)
\providecommand{\choiceminimumverticalsize}{\vphantom{$\tfrac{2}{2}$}}   % minimum height of boxes (cfr infra)
\providecommand{\choiceminimumhorizontalsize}{1em}   % minimum width of boxes (cfr infra)

\newcommand{\inlinechoicesquares}[2][]{%
		\setkeys{choice}{#1}%
		\ifthenelse{\boolean{\choice@correct}}%
		{%
            \ifhandout%
               \fbox{\choiceminimumverticalsize #2}\allowbreak\ignorespaces%
            \else%
               \fcolorbox{blue}{blue!20}{\choiceminimumverticalsize #2}\allowbreak\ignorespaces\setkeys{choice}{correct=false}\ignorespaces%
            \fi%
		}%
		{% else
			\fbox{\choiceminimumverticalsize #2}\allowbreak\ignorespaces%  HACK: wat kleiner, zodat fits on line ... 	
		}%
}

\newcommand{\inlinechoicesquareX}[2][]{%
		\setkeys{choice}{#1}%
		\ifthenelse{\boolean{\choice@correct}}%
		{%
            \ifhandout%
               \framebox[\ifdim\choiceminimumhorizontalsize<\width\width\else\choiceminimumhorizontalsize\fi]{\choiceminimumverticalsize\ #2\ }\allowbreak\ignorespaces\setkeys{choice}{correct=false}\ignorespaces%
            \else%
               \fcolorbox{blue}{blue!20}{\makebox[\ifdim\choiceminimumhorizontalsize<\width\width\else\choiceminimumhorizontalsize\fi]{\choiceminimumverticalsize #2}}\allowbreak\ignorespaces\setkeys{choice}{correct=false}\ignorespaces%
            \fi%
		}%
		{% else
        \ifhandout%
			\framebox[\ifdim\choiceminimumhorizontalsize<\width\width\else\choiceminimumhorizontalsize\fi]{\choiceminimumverticalsize\ #2\ }\allowbreak\ignorespaces%  HACK: wat kleiner, zodat fits on line ... 	
        \fi
		}%
}


\newcommand{\inlinechoicepuntjes}[2][]{%
		\setkeys{choice}{#1}%
		\ifthenelse{\boolean{\choice@correct}}%
		{%
            \ifhandout%
               \dots\ldots\ignorespaces\setkeys{choice}{correct=false}\ignorespaces
            \else%
               \fcolorbox{blue}{blue!20}{\choiceminimumverticalsize #2}\allowbreak\ignorespaces\setkeys{choice}{correct=false}\ignorespaces%
            \fi%
		}%
		{% else
			%\fbox{\choiceminimumverticalsize #2}\allowbreak\ignorespaces%  HACK: wat kleiner, zodat fits on line ... 	
		}%
}

% print niets, maar definieer globale variable \myanswer
%  (gebruikt om oplossingsbladen te printen) 
\newcommand{\inlinechoicedefanswer}[2][]{%
		\setkeys{choice}{#1}%
		\ifthenelse{\boolean{\choice@correct}}%
		{%
               \gdef\myanswer{#2}\setkeys{choice}{correct=false}

		}%
		{% else
			%\fbox{\choiceminimumverticalsize #2}\allowbreak\ignorespaces%  HACK: wat kleiner, zodat fits on line ... 	
		}%
}



%\makeatother

\newcommand{\setchoicedefanswer}{
\ifdefined\HCode
\else
%    \renewenvironment{multipleChoice@}[1][]{}{} % remove trailing ')'
    \let\inlinechoice\inlinechoicedefanswer
\fi
}

\newcommand{\setchoicepuntjes}{
\ifdefined\HCode
\else
    \renewenvironment{multipleChoice@}[1][]{}{} % remove trailing ')'
    \let\inlinechoice\inlinechoicepuntjes
\fi
}
\newcommand{\setchoicesquares}{
\ifdefined\HCode
\else
    \renewenvironment{multipleChoice@}[1][]{}{} % remove trailing ')'
    \let\inlinechoice\inlinechoicesquares
\fi
}
%
\newcommand{\setchoicesquareX}{
\ifdefined\HCode
\else
    \renewenvironment{multipleChoice@}[1][]{}{} % remove trailing ')'
    \let\inlinechoice\inlinechoicesquareX
\fi
}
%
\newcommand{\setchoicelist}{
\ifdefined\HCode
\else
    \renewenvironment{multipleChoice@}[1][]{}{)}% re-add trailing ')'
    \let\inlinechoice\inlinechoiceorig
\fi
}

\setchoicesquareX  % by default list-of-squares with onlineChoice in PDF

% Omdat multicols niet werkt in html: enkel in pdf  (in html zijn langere pagina's misschien ook minder storend)
\newenvironment{xmmulticols}[1][2]{
 \pdfOnly{\begin{multicols}{#1}}%
}{ \pdfOnly{\end{multicols}}}

%
% Te gebruiken in plaats van \section\subsection
%  (in een printstyle kan dan het level worden aangepast
%    naargelang \chapter vs \section style )
% 3/2021: DO NOT USE \xmsubsection !
\newcommand\xmsection\subsection
\newcommand\xmsubsection\subsubsection

% Aanpassen printversie
%  (hier gedefinieerd, zodat ze in xourse kunnen worden gezet/overschreven)
\providebool{parttoc}
\providebool{printpartfrontpage}
\providebool{printactivitytitle}
\providebool{printactivityqrcode}
\providebool{printactivityurl}
\providebool{printcontinuouspagenumbers}
\providebool{numberactivitiesbysubpart}
\providebool{addtitlenumber}
\providebool{addsectiontitlenumber}
\addtitlenumbertrue
\addsectiontitlenumbertrue

% The following three commands are hardcoded in xake, you can't create other commands like these, without adding them to xake as well
%  ( gebruikt in xourses om juiste soort titelpagina te krijgen voor verschillende ximera's )
\newcommand{\activitychapter}[2][]{
    {    
    \ifstrequal{#1}{notnumbered}{
        \addtitlenumberfalse
    }{}
    \typeout{ACTIVITYCHAPTER #2}   % logging
	\chapterstyle
	\activity{#2}
    }
}
\newcommand{\activitysection}[2][]{
    {
    \ifstrequal{#1}{notnumbered}{
        \addsectiontitlenumberfalse
    }{}
	\typeout{ACTIVITYSECTION #2}   % logging
	\sectionstyle
	\activity{#2}
    }
}
% Practices worden als activity getoond om de grote blokken te krijgen online
\newcommand{\practicesection}[2][]{
    {
    \ifstrequal{#1}{notnumbered}{
        \addsectiontitlenumberfalse
    }{}
    \typeout{PRACTICESECTION #2}   % logging
	\sectionstyle
	\activity{#2}
    }
}
\newcommand{\activitychapterlink}[3][]{
    {
    \ifstrequal{#1}{notnumbered}{
        \addtitlenumberfalse
    }{}
    \typeout{ACTIVITYCHAPTERLINK #3}   % logging
	\chapterstyle
	\activitylink[#1]{#2}{#3}
    }
}

\newcommand{\activitysectionlink}[3][]{
    {
    \ifstrequal{#1}{notnumbered}{
        \addsectiontitlenumberfalse
    }{}
    \typeout{ACTIVITYSECTIONLINK #3}   % logging
	\sectionstyle
	\activitylink[#1]{#2}{#3}
    }
}


% Commando om de printstyle toe te voegen in ximera's. Zorgt ervoor dat er geen problemen zijn als je de xourses compileert
% hack om onhandige relative paden in TeX te omzeilen
% should work both in xourse and ximera (pre-112022 only in ximera; thus obsoletes adhoc setup in xourses)
% loads global.sty if present (cfr global.css for online settings!)
% use global.sty to overwrite settings in printstyle.sty ...
\newcommand{\addPrintStyle}[1]{
\iftikzexport\else   % only in PDF
  \makeatletter
  \ifx\@onlypreamble\@notprerr\else   % ONLY if in tex-preamble   (and e.g. not when included from xourse)
    \typeout{Loading printstyle}   % logging
    \usepackage{#1/printstyle} % mag enkel geinclude worden als je die apart compileert
    \IfFileExists{#1/global.sty}{
        \typeout{Loading printstyle-folder #1/global.sty}   % logging
        \usepackage{#1/global}
        }{
        \typeout{Info: No extra #1/global.sty}   % logging
    }   % load global.sty if present
    \IfFileExists{global.sty}{
        \typeout{Loading local-folder global.sty (or TEXINPUTPATH..)}   % logging
        \usepackage{global}
    }{
        \typeout{Info: No folder/global.sty}   % logging
    }   % load global.sty if present
    \IfFileExists{\currfilebase.sty}
    {
        \typeout{Loading \currfilebase.sty}
        \input{\currfilebase.sty}
    }{
        \typeout{Info: No local \currfilebase.sty}
    }
    \fi
  \makeatother
\fi
}

%
%  
% references: Ximera heeft adhoc logica	 om online labels te doen werken over verschillende files heen
% met \hyperref kan de getoonde tekst toch worden opgegeven, in plaats van af te hangen van de label-text
\ifdefined\HCode
% Link to standard \labels, but give your own description
% Usage:  Volg \hyperref[my_very_verbose_label]{deze link} voor wat tijdverlies
%   (01/2020: Ximera-server aangepast om bij class reference-keeptext de link-text NIET te vervangen door de label-text !!!) 
\renewcommand{\hyperref}[2][]{\HCode{<a class="reference reference-keeptext" href="\##1">}#2\HCode{</a>}}
%
%  Link to specific targets  (not tested ?)
\renewcommand{\hypertarget}[1]{\HCode{<a class="ximera-label" id="#1"></a>}}
\renewcommand{\hyperlink}[2]{\HCode{<a class="reference reference-keeptext" href="\##1">}#2\HCode{</a>}}
\fi

% Mmm, quid English ... (for keyword #1 !) ?
\newcommand{\wikilink}[2]{
    \hyperlink{https://nl.wikipedia.org/wiki/#1}{#2}
    \pdfOnly{\footnote{See \url{https://nl.wikipedia.org/wiki/#1}}
    }
}

\renewcommand{\figurename}{Figuur}
\renewcommand{\tablename}{Tabel}

%
% Gedoe om verschillende versies van xourse/ximera te maken afhankelijk van settings
%
% default: versie met antwoorden
% handout: versie voor de studenten, zonder antwoorden/oplossingen
% full: met alles erop en eraan, dus geschikt voor auteurs en/of lesgevers  (bevat in de pdf ook de 'online-only' stukken!)
%
%
% verder kunnen ook opties/variabele worden gezet voor hints/auteurs/uitweidingen/ etc
%
% 'Full' versie
\newtoggle{showonline}
\ifdefined\HCode   % zet default showOnline
    \toggletrue{showonline} 
\else
    \togglefalse{showonline}
\fi

% Full versie   % deprecated: see infra
\newcommand{\printFull}{
    \hintstrue
    \handoutfalse
    \toggletrue{showonline} 
}

\ifdefined\shouldPrintFull   % deprecated: see infra
    \printFull
\fi



% Overschrijf onlineOnly  (zoals gedefinieerd in ximera.cls)
\ifhandout   % in handout: gebruik de oorspronkelijke ximera.cls implementatie  (is dit wel nodig/nuttig?)
\else
    \iftoggle{showonline}{%
        \ifdefined\HCode
          \RenewEnviron{onlineOnly}{\bgroup\BODY\egroup}   % showOnline, en we zijn  online, dus toon de tekst
        \else
          \RenewEnviron{onlineOnly}{\bgroup\color{red!50!black}\BODY\egroup}   % showOnline, maar we zijn toch niet online: kleur de tekst rood 
        \fi
    }{%
      \RenewEnviron{onlineOnly}{}  % geen showOnline
    }
\fi

% hack om na hoofding van definition/proposition/... als dan niet op een nieuwe lijn te starten
% soms is dat goed en mooi, en soms niet; en in HTML is het nu (2/2020) anders dan in pdf
% vandaar suggestie om 
%     \begin{definition}[Nieuw concept] \nl
% te gebruiken als je zeker een newline wil na de hoofdig en titel
% (in het bijzonder itemize zonder \nl is 'lelijk' ...)
\ifdefined\HCode
\newcommand{\nl}{}
\else
\newcommand{\nl}{\ \par} % newline (achter heading van definition etc.)
\fi


% \nl enkel in handoutmode (ihb voor \wordChoice, die dan typisch veeeel langer wordt)
\ifdefined\HCode
\providecommand{\handoutnl}{}
\else
\providecommand{\handoutnl}{%
\ifhandout%
  \nl%
\fi%
}
\fi

% Could potentially replace \pdfOnline/\begin{onlineOnly} : 
% Usage= \ifonline{Hallo surfer}{Hallo PDFlezer}
\providecommand{\ifonline}[2]%
{
\begin{onlineOnly}#1\end{onlineOnly}%
\pdfOnly{#2}
}%


%
% Maak optionele 'basic' en 'extended' versies van een activity
%  met environment basicOnly, basicSkip en extendedOnly
%
%  (
%   Dit werkt ENKEL in de PDF; de online versies tonen (minstens voorklopig) steeds 
%   het default geval met printbasicversion en printextendversion beide FALSE
%  )
%
\providebool{printbasicversion}
\providebool{printextendedversion}   % not properly implemented
\providebool{printfullversion}       % presumably print everything (debug/auteur)
%
% only set these in xourses, and BEFORE loading this preamble
%
%\newif\ifshowbasic     \showbasictrue        % use this line in xourse to show 'basic' sections
%\newif\ifshowextended  \showextendedtrue     % use this line in xourse to show 'extended' sections
%
%
%\ifbool{showbasic}
%      { \NewEnviron{basicOnly}{\BODY} }    % if yes: just print contents
%      { \NewEnviron{basicOnly}{}      }    % if no:  completely ignore contents
%
%\ifbool{showbasic}
%      { \NewEnviron{basicSkip}{}      }
%      { \NewEnviron{basicSkip}{\BODY} }
%

\ifbool{printextendedversion}
      { \NewEnviron{extendedOnly}{\BODY} }
      { \NewEnviron{extendedOnly}{}      }
      


\ifdefined\HCode    % in html: always print
      {\newenvironment*{basicOnly}{}{}}    % if yes: just print contents
      {\newenvironment*{basicSkip}{}{}}    % if yes: just print contents
\else

\ifbool{printbasicversion}
      {\newenvironment*{basicOnly}{}{}}    % if yes: just print contents
      {\NewEnviron{basicOnly}{}      }    % if no:  completely ignore contents

\ifbool{printbasicversion}
      {\NewEnviron{basicSkip}{}      }
      {\newenvironment*{basicSkip}{}{}}

\fi

\usepackage{float}
\usepackage[rightbars,color]{changebar}

% Full versie
\ifbool{printfullversion}{
    \hintstrue
    \handoutfalse
    \toggletrue{showonline}
    \printbasicversionfalse
    \cbcolor{red}
    \renewenvironment*{basicOnly}{\cbstart}{\cbend}
    \renewenvironment*{basicSkip}{\cbstart}{\cbend}
    \def\xmtoonprintopties{FULL}   % will be printed in footer
}
{}
      
%
% Evalueer \ifhints IN de environment
%  
%
%\RenewEnviron{hint}
%{
%\ifhandout
%\ifhints\else\setbox0\vbox\fi%everything in een emty box
%\bgroup 
%\stepcounter{hintLevel}
%\BODY
%\egroup\ignorespacesafterend
%\addtocounter{hintLevel}{-1}
%\else
%\ifhints
%\begin{trivlist}\item[\hskip \labelsep\small\slshape\bfseries Hint:\hspace{2ex}]
%\small\slshape
%\stepcounter{hintLevel}
%\BODY
%\end{trivlist}
%\addtocounter{hintLevel}{-1}
%\fi
%\fi
%}

% Onafhankelijk van \ifhandout ...? TO BE VERIFIED
\RenewEnviron{hint}
{
\ifhints
\begin{trivlist}\item[\hskip \labelsep\small\bfseries Hint:\hspace{2ex}]
\small%\slshape
\stepcounter{hintLevel}
\BODY
\end{trivlist}
\addtocounter{hintLevel}{-1}
\else
\iftikzexport   % anders worden de tikz tekeningen in hints niet gegenereerd ?
\setbox0\vbox\bgroup
\stepcounter{hintLevel}
\BODY
\egroup\ignorespacesafterend
\addtocounter{hintLevel}{-1}
\fi % ifhandout
\fi %ifhints
}

%
% \tab sets typewriter-tabs (e.g. to format questions)
% (Has no effect in HTML :-( ))
%
\usepackage{tabto}
\ifdefined\HCode
  \renewcommand{\tab}{\quad}    % otherwise dummy .png's are generated ...?
\fi


% Also redefined in  preamble to get correct styling 
% for tikz images for (\tikzexport)
%

\theoremstyle{definition} % Bold titels
\makeatletter
\let\proposition\relax
\let\c@proposition\relax
\let\endproposition\relax
\makeatother
\newtheorem{proposition}{Eigenschap}


%\instructornotesfalse

% logic with \ifhandoutin ximera.cls unclear;so overwrite ...
\makeatletter
\@ifundefined{ifinstructornotes}{%
  \newif\ifinstructornotes
  \instructornotesfalse
  \newenvironment{instructorNotes}{}{}
}{}
\makeatother
\ifinstructornotes
\else
\renewenvironment{instructorNotes}%
{%
    \setbox0\vbox\bgroup
}
{%
    \egroup
}
\fi

% \RedeclareMathOperator
% from https://tex.stackexchange.com/questions/175251/how-to-redefine-a-command-using-declaremathoperator
\makeatletter
\newcommand\RedeclareMathOperator{%
    \@ifstar{\def\rmo@s{m}\rmo@redeclare}{\def\rmo@s{o}\rmo@redeclare}%
}
% this is taken from \renew@command
\newcommand\rmo@redeclare[2]{%
    \begingroup \escapechar\m@ne\xdef\@gtempa{{\string#1}}\endgroup
    \expandafter\@ifundefined\@gtempa
    {\@latex@error{\noexpand#1undefined}\@ehc}%
    \relax
    \expandafter\rmo@declmathop\rmo@s{#1}{#2}}
% This is just \@declmathop without \@ifdefinable
\newcommand\rmo@declmathop[3]{%
    \DeclareRobustCommand{#2}{\qopname\newmcodes@#1{#3}}%
}
\@onlypreamble\RedeclareMathOperator
\makeatother


%
% Engelse vertaling, vooral in mathmode
%
% 1. Algemene procedure
%
\ifdefined\isEn
 \newcommand{\nlen}[2]{#2}
 \newcommand{\nlentext}[2]{\text{#2}}
 \newcommand{\nlentextbf}[2]{\textbf{#2}}
\else
 \newcommand{\nlen}[2]{#1}
 \newcommand{\nlentext}[2]{\text{#1}}
 \newcommand{\nlentextbf}[2]{\textbf{#1}}
\fi

%
% 2. Lijst van erg veel gebruikte uitdrukkingen
%

% Ja/Nee/Fout/Juits etc
%\newcommand{\TJa}{\nlentext{ Ja }{ and }}
%\newcommand{\TNee}{\nlentext{ Nee }{ No }}
%\newcommand{\TJuist}{\nlentext{ Juist }{ Correct }
%\newcommand{\TFout}{\nlentext{ Fout }{ Wrong }
\newcommand{\TWaar}{\nlentext{ Waar }{ True }}
\newcommand{\TOnwaar}{\nlentext{ Vals }{ False }}
% Korte bindwoorden en, of, dus, ...
\newcommand{\Ten}{\nlentext{ en }{ and }}
\newcommand{\Tof}{\nlentext{ of }{ or }}
\newcommand{\Tdus}{\nlentext{ dus }{ so }}
\newcommand{\Tendus}{\nlentext{ en dus }{ and thus }}
\newcommand{\Tvooralle}{\nlentext{ voor alle }{ for all }}
\newcommand{\Took}{\nlentext{ ook }{ also }}
\newcommand{\Tals}{\nlentext{ als }{ when }} %of if?
\newcommand{\Twant}{\nlentext{ want }{ as }}
\newcommand{\Tmaal}{\nlentext{ maal }{ times }}
\newcommand{\Toptellen}{\nlentext{ optellen }{ add }}
\newcommand{\Tde}{\nlentext{ de }{ the }}
\newcommand{\Thet}{\nlentext{ het }{ the }}
\newcommand{\Tis}{\nlentext{ is }{ is }} %zodat is in text staat in mathmode (geen italics)
\newcommand{\Tmet}{\nlentext{ met }{ where }} % in situaties e.g met p < n --> where p < n
\newcommand{\Tnooit}{\nlentext{ nooit }{ never }}
\newcommand{\Tmaar}{\nlentext{ maar }{ but }}
\newcommand{\Tniet}{\nlentext{ niet }{ not }}
\newcommand{\Tuit}{\nlentext{ uit }{ from }}
\newcommand{\Ttov}{\nlentext{ t.o.v. }{ w.r.t. }}
\newcommand{\Tzodat}{\nlentext{ zodat }{ such that }}
\newcommand{\Tdeth}{\nlentext{de }{th }}
\newcommand{\Tomdat}{\nlentext{omdat }{because }} 


%
% Overschrijf addhoc commando's
%
\ifdefined\isEn
\renewcommand{\pernot}{\overset{\mathrm{notation}}{=}}
\RedeclareMathOperator{\bld}{im}     % beeld
\RedeclareMathOperator{\graf}{graph}   % grafiek
\RedeclareMathOperator{\rico}{slope}   % richtingcoëfficient
\RedeclareMathOperator{\co}{co}       % coordinaat
\RedeclareMathOperator{\gr}{deg}       % graad

% Operators
\RedeclareMathOperator{\bgsin}{arcsin}
\RedeclareMathOperator{\bgcos}{arccos}
\RedeclareMathOperator{\bgtan}{arctan}
\RedeclareMathOperator{\bgcot}{arccot}
\RedeclareMathOperator{\bgsinh}{arcsinh}
\RedeclareMathOperator{\bgcosh}{arccosh}
\RedeclareMathOperator{\bgtanh}{arctanh}
\RedeclareMathOperator{\bgcoth}{arccoth}

\fi


% HACK: use 'oplossing' for 'explanation' ...
\let\explanation\relax
\let\endexplanation\relax
% \newenvironment{explanation}{\begin{oplossing}}{\end{oplossing}}
\newcounter{explanation}

\ifhandout%
    \NewEnviron{explanation}[1][toon]%
    {%
    \RenewEnviron{verbatim}{ \red{VERBATIM CONTENT MISSING IN THIS PDF}} %% \expandafter\verb|\BODY|}

    \ifthenelse{\equal{\detokenize{#1}}{\detokenize{toon}}}
    {
    \def\PH@Command{#1}% Use PH@Command to hold the content and be a target for "\expandafter" to expand once.

    \begin{trivlist}% Begin the trivlist to use formating of the "Feedback" label.
    \item[\hskip \labelsep\small\slshape\bfseries Explanation:% Format the "Feedback" label. Don't forget the space.
    %(\texttt{\detokenize\expandafter{\PH@Command}}):% Format (and detokenize) the condition for feedback to trigger
    \hspace{2ex}]\small%\slshape% Insert some space before the actual feedback given.
    \BODY
    \end{trivlist}
    }
    {  % \begin{feedback}[solution]   \BODY     \end{feedback}  }
    }
    }    
\else
% ONLY for HTML; xmoplossing is styled with css, and is not, and need not be a LaTeX environment
% THUS: it does NOT use feedback anymore ...
%    \NewEnviron{oplossing}{\begin{expandable}{xmoplossing}{\nlen{Toon uitwerking}{Show solution}}{\BODY}\end{expandable}}
    \newenvironment{explanation}[1][toon]
   {%
       \begin{expandable}{xmoplossing}{}
   }
   {%
   	   \end{expandable}
   } 
\fi

\title{Determinants, Areas, and Volumes} \license{CC BY-NC-SA 4.0}

\begin{document}

\begin{abstract}
 \end{abstract}
\maketitle

\begin{onlineOnly}
\section*{Determinants, Areas, and Volumes}
\end{onlineOnly}

\subsection*{$2\times 2$ Determinant and the Area of a Parallelogram}

Consider the parallelogram determined by vectors $\vec{u}$ and $\vec{v}$ in $\RR^3$.

\begin{center}
\begin{tikzpicture}[scale=1.4]

  \filldraw[blue, opacity=0.3](0,0)--(4,2)--(5,4)--(1,2)--cycle;

\draw[line width=2pt,red,-stealth](0,0)--(4,2) node[below right]{$\vec{u}$};
 
  \draw[line width=2pt,blue, -stealth](0,0)--(1,2) node[above left]{$\vec{v}$}; 
\end{tikzpicture}
\end{center}

Recall that the area of a parallelogram is given by the product of the length of the base and the height.
As shown in the diagram below, the length of the base is the magnitude of $\vec{u}$. The height, $h$, can be found using trigonometry $$h=\norm{\vec{v}}\sin\theta$$ 
\begin{center}
\begin{tikzpicture}[scale=1.4]

  \filldraw[blue, opacity=0.3](0,0)--(4,2)--(5,4)--(1,2)--cycle;

\draw[line width=0.5pt, dashed](1,2)--(1.6,0.8);
\node[] at (2, 1.6)   (b) {$h=\norm{\vec{v}}\sin\theta$};
\draw[line width=2pt,red](0,0)--(4,2);
 \node[red] at (2, 0.6)   (b) {$\norm{\vec{u}}$};
 \node[blue] at (0.2, 1.2)   (b) {$\norm{\vec{v}}$};
 \node[] at (0.3, 0.3)   (b) {$\theta$};
  \draw[line width=2pt,blue](0,0)--(1,2); 
\end{tikzpicture}
\end{center}
Using the area of a parallelogram formula together with Theorem \ref{th:crossproductsin} we get
$$\mbox{Area}=(\text{base})(\text{height})=\norm{\vec{u}}h=\norm{\vec{u}}\norm{\vec{v}}\sin\theta=\norm{\vec{u}\times\vec{v}}$$
We have established the following formula.

\begin{formula}\label{form:areaofparallelogram} The area of a parallelogram determined by vectors $\vec{u}$ and $\vec{v}$ in $\RR^3$ is given by
$$\mbox{Area}=\norm{\vec{u}\times\vec{v}}$$
\end{formula}

\begin{example}\label{ex:areaOfParFormula}
    Use Formula \ref{form:areaofparallelogram} to find the area of a parallelogram determined by vectors 
    $$\vec{u}=\begin{bmatrix}-2\\1\\3\end{bmatrix},\quad \vec{v}=\begin{bmatrix}1\\6\\2\end{bmatrix}$$
\begin{explanation}
    We can start by visualizeing the parallelogram in GeoGebra.  RIGHT-CLICK and DRAG to rotate the image below.  The area of the parallelogram, rounded to two decimal places, is displayed inside the parallelogram.

\pdfOnly{
Access GeoGebra interactives through the online version of this text at 

\href{https://ximera.osu.edu/oerlinalg}{https://ximera.osu.edu/oerlinalg}.
}


\begin{onlineOnly}    
\begin{center}
\geogebra{g7g6kjqm}{600}{400}
\end{center}
\end{onlineOnly}

    To find the exact area we compute
    $$\vec{u}\times \vec{v}=\begin{bmatrix}-2\\1\\3\end{bmatrix}\times \begin{bmatrix}1\\6\\2\end{bmatrix}=\begin{vmatrix}\vec{i} &\vec{j} &\vec{k}\\-2 & 1 & 3\\1 & 6 & 2\end{vmatrix}=\begin{bmatrix}\answer{-16}\\\answer{7}\\\answer{-13}\end{bmatrix}$$

    $$\mbox{Area}=\norm{\vec{u}\times\vec{v}}=\sqrt{\answer{474}}$$
\end{explanation}
\end{example}




Formula \ref{form:areaofparallelogram} can be easily adapted to parallelograms determined by vectors in $\RR^2$, as illustrated by the following example.

\begin{example}\label{ex:areaofparallelogram}
Find the area of the parallelogram in the diagram.
\begin{center}
\begin{tikzpicture}[scale=1]
\draw[thin,gray!40] (-3,-1) grid (4,4);
  \draw[<->] (-3,0)--(4,0);
  \draw[<->] (0,-1)--(0,4);
  
  \filldraw[blue, opacity=0.3](0,0)--(-2,2)--(2,4)--(4,2)--cycle;

\draw[line width=2pt,red,-stealth](0,0)--(4,2);


 \draw[line width=2pt,blue,-stealth](0,0)--(-2,2);
 
\end{tikzpicture}
\end{center}
\begin{explanation}
The vectors that determine the parallelogram are 
$$\begin{bmatrix}4\\2\end{bmatrix}\quad\text{and}\quad\begin{bmatrix}-2\\2\end{bmatrix}$$
The problem we run into is that these vectors are in $\RR^2$, whereas the cross product is defined only for vectors in $\RR^3$.  We will get around this difficulty by ``padding" our vectors with zeros on the bottom.  In other words, we will consider them as vectors sitting in the $xy$-coordinate plane in $\RR^3$.  This allows us to compute the cross product 
$$\begin{bmatrix}4\\2\\0\end{bmatrix}\times\begin{bmatrix}-2\\2\\0\end{bmatrix}=\begin{vmatrix}\vec{i}&\vec{j}&\vec{k}\\4&2&0\\-2&2&0\end{vmatrix}=\vec{k}\Big((4)(2)-(2)(-2)\Big)=12\vec{k}=\begin{bmatrix}0\\0\\12\end{bmatrix}$$
The area of the parallelogram is then given by
$$\mbox{Area}=\norm{\begin{bmatrix}0\\0\\12\end{bmatrix}}=12$$
\end{explanation}
\end{example}

Example \ref{ex:areaofparallelogram} illustrates an important phenomenon.  Observe that the zeros in the last column of the determinant ensure that the $\vec{i}$ and $\vec{j}$ components of the cross product are zero, while the last component is the determinant of the $2\times 2$ matrix whose rows (or columns) are the two vectors that determine the parallelogram in $\RR^2$.  In general, if the parallelogram is determined by vectors 
$$\begin{bmatrix}a\\b\end{bmatrix}\quad\text{and}\quad\begin{bmatrix}c\\d\end{bmatrix}$$
then the area of the parallelogram can be computed as follows:
$$\begin{bmatrix}a\\b\\0\end{bmatrix}\times\begin{bmatrix}c\\d\\0\end{bmatrix}=\begin{vmatrix}\vec{i}&\vec{j}&\vec{k}\\a&b&0\\c&d&0\end{vmatrix}=\vec{k}\begin{vmatrix}a&b\\c&d\end{vmatrix}=\vec{k}\Big((a)(d)-(b)(c)\Big)=\begin{bmatrix}0\\0\\ad-bc\end{bmatrix}$$

$$\mbox{Area}=\norm{\begin{bmatrix}0\\0\\ad-bc\end{bmatrix}}=|ad-bc|=\Big|\det\begin{bmatrix}a&b\\c&d\end{bmatrix}\Big|=\Big|\det\begin{bmatrix}a&c\\b&d\end{bmatrix}\Big|$$

So the area of the parallelogram turns out to be the absolute value of the determinant of the matrix whose rows (or columns) are the two vectors that determine the parallelogram. 
The following formula summarizes our discussion.

\begin{formula}\label{form:areaofparallelogramdeterminant} Let $\vec{u}=\begin{bmatrix}a\\b\end{bmatrix}$ and $\vec{v}=\begin{bmatrix}c\\d\end{bmatrix}$ be vectors of $\RR^2$.  The area of the parallelogram determined by $\vec{u}$ and $\vec{v}$ is given by
$$\mbox{Area}=\Big|{\det\begin{bmatrix}a&b\\c&d\end{bmatrix}}\Big|=\Big|\det\begin{bmatrix}a&c\\b&d\end{bmatrix}\Big|$$
\end{formula}

\begin{example}\label{exp:polyArea}
    Use Formula \ref{form:areaofparallelogramdeterminant} to find the area of the polygon shown below.
  \begin{center}
\begin{tikzpicture}[scale=1]
\draw[thin,gray!40] (-3,-2) grid (3,4);
  \draw[<->] (-3,0)--(3,0);
  \draw[<->] (0,-2)--(0,4);
  \filldraw[blue, opacity=0.3](-1,-1)--(-2,2)--(-1,3)--(1,2)--(2,0)--cycle;
%\draw[line width=2pt,red,-stealth](0,0)--(4,2);
% \draw[line width=2pt,blue,-stealth](0,0)--(-2,2);
 \end{tikzpicture}
\end{center}  
\begin{explanation}
    We will start by splitting this region into triangles.
    \begin{center}
\begin{tikzpicture}[scale=1]
\draw[thin,gray!40] (-3,-2) grid (3,4);
  \draw[<->] (-3,0)--(3,0);
  \draw[<->] (0,-2)--(0,4);
  \filldraw[blue, opacity=0.3](-1,-1)--(-2,2)--(-1,3)--(1,2)--(2,0)--cycle;
\draw[line width=1pt,red,-stealth](0,0)--(-1,-1);
\draw[line width=1pt,red,-stealth](0,0)--(-2,2);
\draw[line width=1pt,red,-stealth](0,0)--(-1,3);
\draw[line width=1pt,red,-stealth](0,0)--(1,2);
\draw[line width=1pt,red,-stealth](0,0)--(2,0);
\node[] at (1, 0.6)   (b) {$A_1$};
\node[] at (0.2, 1.9)   (b) {$A_2$};
\node[] at (-1, 1.5)   (b) {$A_3$};
\node[] at (-1, 0.2)   (b) {$A_4$};
\node[] at (0.5, -0.4)   (b) {$A_5$};
 \end{tikzpicture}
\end{center} 
    We can find the total area of the polygon by finding the area of each triangle.  The area of each triangle is half of the area of the corresponding parallelogram.  For instance, $A_1$ is half of the area of the parallelogram depicted below. 
\begin{center}
\begin{tikzpicture}[scale=1]
\draw[thin,gray!40] (-3,-2) grid (3,4);
  \draw[<->] (-3,0)--(3,0);
  \draw[<->] (0,-2)--(0,4);
  \filldraw[blue, opacity=0.3](-1,-1)--(-2,2)--(-1,3)--(1,2)--(2,0)--cycle;
\draw[line width=1pt,red,-stealth](0,0)--(-1,-1);
\draw[line width=1pt,red,-stealth](0,0)--(-2,2);
\draw[line width=1pt,red,-stealth](0,0)--(-1,3);
\draw[line width=1pt,red,-stealth](0,0)--(1,2);
\draw[line width=1pt,red,-stealth](0,0)--(2,0);
\filldraw[orange, opacity=0.3](2,0)--(0,0)--(1,2)--(3,2)--cycle;
 \end{tikzpicture}
\end{center} 
We compute
$$A_1=\frac{1}{2}\left|\det\begin{bmatrix}2 & 1\\0 & 2\end{bmatrix}\right |=\answer{2}$$
$$A_2=\frac{1}{2}\left|\det\begin{bmatrix}1 & -1\\2 & 3\end{bmatrix}\right |=\answer{2.5}$$
$$A_3=\frac{1}{2}\left|\det\begin{bmatrix}-1 & -2\\3 & 2\end{bmatrix}\right |=\answer{2}$$
$$A_4=\frac{1}{2}\left|\det\begin{bmatrix}-2 & -1\\2 & -1\end{bmatrix}\right |=\answer{2}$$
$$A_5=\frac{1}{2}\left|\det\begin{bmatrix}-1 & 2\\-1 & 0\end{bmatrix}\right |=\answer{1}$$
The total area of the polygon is $\answer{9.5}$.
\end{explanation}
\end{example}


\subsection*{$3\times 3$ Determinant and the Volume of a Parallelepiped}

Our next goal is to find the volume of a three-dimensional figure called a \dfn{parallelepiped}.  A parallelepiped is a six-faced figure whose opposite faces are congruent parallelograms located in parallel planes.  A parallelepiped is a three-dimensional counterpart of a parallelogram, and is determined by three non-coplanar vectors in $\RR^3$.  The figure below shows a parallelepiped determined by three vectors.

\begin{center}
\tdplotsetmaincoords{70}{130}
\begin{tikzpicture}
	\draw[->](-2,0,0)--(5,0,0) node[below left]{$y$};
    \draw[->](0,-2,0)--(0,6,0) node[below left]{$z$};
    \draw[->](0,0,-2)--(0,0,5) node[below left]{$x$};
    
    \draw[line width=0.5pt, dashed](3,5,1)--(5,1,-2);
    \draw[line width=0.5pt, dashed](3,5,1)--(-2,4,3);
    \draw[line width=0.5pt, dashed](3,5,1)--(7,9,6);
    
    \filldraw[blue, opacity=0.4] (0,0,0)--(-2, 4, 3)--(2,8,8)--(4,4,5)--cycle;
    \filldraw[blue, opacity=0.5] (2,8,8)--(4,4,5)--(9,5,3)--(7,9,6)--cycle;
    \filldraw[blue, opacity=0.6] (0,0,0)--(4,4,5)--(9,5,3)--(5,1,-2)--cycle;
    
    \draw[->, line width=2pt,blue, -stealth](0,0,0)--(-2,4,3);
    \draw[->, line width=2pt,red, -stealth](0,0,0)--(4,4,5);
    \draw[->, line width=2pt, -stealth](0,0,0)--(5,1,-2);
    
\end{tikzpicture}
\end{center}

Consider a parallelepiped determined by vectors $\vec{u}$, $\vec{v}$ and $\vec{w}$, as shown below.  

\begin{center}
\tdplotsetmaincoords{70}{130}
\begin{tikzpicture}
    \draw[->, line width=2pt, -stealth,orange,dashed](0,0,0)--(4,0,1)node[above left]{$\vec{v}$};
    
    \draw[line width=0.5pt, dashed](4,0,1)--(5,0,6);
    \draw[line width=0.5pt, dashed](4,0,1)--(9,5,2);
    
    \draw[line width=2pt, gray](1,0,5)--(6,5,6);
    \draw[line width=2pt, gray](10,5,7)--(6,5,6);
    \draw[line width=2pt, gray](10,5,7)--(5,0,6);
    \draw[line width=2pt, gray](1,0,5)--(5,0,6);
    \draw[line width=2pt, gray](6,5,6)--(5,5,1);
    \draw[line width=2pt, gray](9,5,2)--(5,5,1);
    \draw[line width=2pt, gray](10,5,7)--(9,5,2);
    
    \draw[->, line width=2pt,red, -stealth](0,0,0)--(1,0,5)node[below left]{$\vec{u}$};
    \draw[->, line width=2pt, blue, -stealth](0,0,0)--(5,5,1)node[above left]{$\vec{w}$} ;
  
\end{tikzpicture}
\end{center}

The volume of a parallelepiped is given by 
$$\mbox{Volume}=(\text{area of base})(\text{height})$$
We will consider the parallelogram determined by $\vec{u}$ and $\vec{v}$ to be the base of the parallelepiped.  Thus, the area of the base is given by 
$$\mbox{Area of Base}=\norm{\vec{u}\times\vec{v}}$$
\begin{center}
\tdplotsetmaincoords{70}{130}
\begin{tikzpicture}
    \draw[->, line width=2pt, -stealth,orange,dashed](0,0,0)--(4,0,1)node[above left]{$\vec{v}$};
    
    \draw[line width=0.5pt, dashed](4,0,1)--(5,0,6);
    \draw[line width=0.5pt, dashed](4,0,1)--(9,5,2);
    
    \draw[line width=0.5pt, dashed](5,5,1)--(0,5,0);
    
    \draw[line width=2pt, gray](1,0,5)--(6,5,6);
    \draw[line width=2pt, gray](10,5,7)--(6,5,6);
    \draw[line width=2pt, gray](10,5,7)--(5,0,6);
    \draw[line width=2pt, gray](1,0,5)--(5,0,6);
    \draw[line width=2pt, gray](6,5,6)--(5,5,1);
    \draw[line width=2pt, gray](9,5,2)--(5,5,1);
    \draw[line width=2pt, gray](10,5,7)--(9,5,2);
    
    \draw[->, line width=2pt,red, -stealth](0,0,0)--(1,0,5)node[below left]{$\vec{u}$};
    \draw[->, line width=2pt, blue, -stealth](0,0,0)--(5,5,1)node[above left]{$\vec{w}$} ;
    \node[blue] at (0.4, 1.1)   (b) {$\theta$};
    \draw[->, line width=1pt, -stealth](0,0,0)--(0,6,0) node[above]{$\vec{u}\times\vec{v}$};
    \draw[->, gray, line width=0.5pt, -stealth](-0.25,2.5,0)node[left, black]{$h$}--(-0.25,0,0);
    \draw[->, gray, line width=0.5pt, -stealth](-0.25,2.5,0)--(-0.25,5,0);
\end{tikzpicture}
\end{center}

The height of the parallelepiped is measured along a line perpendicular to the base.  By Theorem \ref{th:crossproductorthtouandv}, $\vec{u}\times\vec{v}$ lies on such a line.  Let $\theta$ be the angle between $\vec{w}$ and $\vec{u}\times\vec{v}$, $0\leq \theta\leq\pi$.  Then the height, $h$, of the parallelepiped is given by 
$$h=\norm{\vec{w}}|\cos\theta |$$

It may be difficult to visualize this in two dimensions.  Below is a replica of of the above diagram in GeoGebra.  RIGHT-CLICK and DRAG to rotate the image.

\pdfOnly{
Access GeoGebra interactives through the online version of this text at 

\href{https://ximera.osu.edu/oerlinalg}{https://ximera.osu.edu/oerlinalg}.
}

\begin{onlineOnly}
\begin{center}
\geogebra{tfuzeqwr}{900}{700}
\end{center}
\end{onlineOnly}

This gives us the following formula for the volume of the parallelepiped
$$\mbox{Volume}=\norm{\vec{u}\times\vec{v}}\norm{\vec{w}}|\cos\theta |=|(\vec{u}\times\vec{v})\dotp\vec{w}|$$

We have established the following formula.

\begin{formula}\label{form:volumeparallelepiped}
The volume of a parallelepiped determined by vectors $\vec{u}$, $\vec{v}$ and $\vec{w}$ in $\RR^3$ is given by\\
$$\mbox{Volume}=|(\vec{u}\times\vec{v})\dotp\vec{w}|$$
\end{formula}

Our next goal is to show that this expression for the volume is equal to the determinant of a $3\times 3$ matrix whose rows are the vectors that determine the parallelepiped.

Let 
$$\vec{u}=\begin{bmatrix}u_1\\u_2\\u_3\end{bmatrix},\quad\vec{v}=\begin{bmatrix}v_1\\v_2\\v_3\end{bmatrix},\quad\vec{w}=\begin{bmatrix}w_1\\w_2\\w_3\end{bmatrix}$$
then
\begin{align}\label{eq:boxproduct}(\vec{u}\times\vec{v})\dotp\vec{w}=\begin{vmatrix}\vec{i}&\vec{j}&\vec{k}\\u_1&u_2&u_3\\v_1&v_2&v_3\end{vmatrix}\dotp\begin{bmatrix}w_1\\w_2\\w_3\end{bmatrix}=\begin{vmatrix}w_1&w_2&w_3\\u_1&u_2&u_3\\v_1&v_2&v_3\end{vmatrix}
\end{align}
The expression in (\ref{eq:boxproduct}) is sometimes referred to as the \dfn{box product} or the \dfn{scalar triple product}.

Recall that $\det(A)=\det(A^T)$ (Theorem \ref{th:detoftrans}).  Therefore, the three vectors that determine the parallelogram can be used to form rows or columns of the determinant on the right side of (\ref{eq:boxproduct}).  This gives us the following formula.

\begin{formula}\label{form:boxproduct}
Let $\vec{u}=\begin{bmatrix}u_1\\u_2\\u_3\end{bmatrix},\quad\vec{v}=\begin{bmatrix}v_1\\v_2\\v_3\end{bmatrix},\quad\vec{w}=\begin{bmatrix}w_1\\w_2\\w_3\end{bmatrix}$ be vectors in $\RR^3$.  Then the volume of the parallelepiped determined by $\vec{u}$, $\vec{v}$ and $\vec{w}$ is given by 
$$\mbox{Volume}=\Big|\det\begin{bmatrix}w_1&w_2&w_3\\u_1&u_2&u_3\\v_1&v_2&v_3\end{bmatrix}\Big|=\Big|\det\begin{bmatrix}w_1&u_1&v_1\\w_2&u_2&v_2\\w_3&u_3&v_3\end{bmatrix}\Big|$$
\end{formula}

\subsection*{Determinants and Linear Transformations}
We will now turn our attention to the determinant of a matrix of a linear transformation.  

\begin{exploration}\label{exp:LinTransAreaDet}
The following GeoGebra interactive shows a polygon $P$ located in the domain of a linear transformation $T$ induced by the matrix $M$.  The right-hand side shows the image of $P$ under $T$.  The number inside each polygon indicates its area.

\pdfOnly{
Access GeoGebra interactives through the online version of this text at 

\href{https://ximera.osu.edu/oerlinalg}{https://ximera.osu.edu/oerlinalg}.
}

\begin{onlineOnly}
\begin{center}
\geogebra{nr8jsz4w}{950}{700}
\end{center}
\end{onlineOnly}

\begin{question}
Let $M=\begin{bmatrix}1&1\\-1&2\end{bmatrix}$.  Find the determinant of $M$.
$$\det{M}=\answer{3}$$
Drag the vertices of $P$ to change the polygon.  Make a note of how the area of $P$ and the area of the image change.  How are the areas related to each other?
$$\mbox{Area}(T(P))=\answer{3}\mbox{Area}(P)$$
\end{question}
\begin{question}
Change the matrix $M$ to a matrix whose determinant is 1.  Compare the areas of $P$ and $T(P)$.  Try matrices whose determinant is 0 or negative.  What do you observe about the areas?

Formulate a conjecture about the relationship between the area of the polygon and the area of its image under a linear transformation.
\end{question}
\end{exploration}
We will not prove your conjecture in Exploration \ref{exp:LinTransAreaDet} for arbitrary figures as it is beyond the scope of this text.  However, we can tackle the problem of how linear transformations affect areas of parallelograms.  This is the topic of our next example.

\begin{example}\label{ex:detLinTransArea}
    Let $T:\RR^2\longrightarrow\RR^2$ be a linear transformation induced by matrix $M$.  Suppose $\vec{u}$ and $\vec{v}$ are vectors in $\RR^2$.  Let $P$ be a parallelogram determined by $\vec{u}$ and $\vec{v}$.  Show that 
    $$\mbox{Area}(T(P))=\left|\det{M}\right|\mbox{Area}(P)$$
    \begin{explanation}
        Let $\vec{u}=\begin{bmatrix}a\\b\end{bmatrix}$ and $\vec{v}=\begin{bmatrix}c\\d\end{bmatrix}$, and let $M=\begin{bmatrix}m & n\\p & q\end{bmatrix}$.  By Formula \ref{form:areaofparallelogramdeterminant}, $\mbox{Area}(P)=ad-bc$.  Applying $M$ to $\vec{u}$ and $\vec{v}$ we get 
        $$T(\vec{u})=\begin{bmatrix}am+bn\\ap+bq\end{bmatrix}\quad\mbox{and}\quad T(\vec{v})=\begin{bmatrix}cm+dn\\cp+dq\end{bmatrix}$$
        Using Formula \ref{form:areaofparallelogramdeterminant}, we compute
        \begin{align*}
        \mbox{Area}(T(P))=&|(am+bn)(cp+dq)-(ap+bq)(cm+dn)|\\
        =&|(mq-np)(ad-bc)|\\
        =&|\det{M}|\mbox{Area}(P)
        \end{align*}
    \end{explanation}
\end{example}

\section*{Practice Problems}

\begin{problem}\label{prob:areasquareandparal} Let $S$ be a square determined by $\begin{bmatrix}2\\0\end{bmatrix}$ and $\begin{bmatrix}0\\2\end{bmatrix}$.  Let $P$ be a parallelogram determined by vectors $\begin{bmatrix}2\\5\end{bmatrix}$ and $\begin{bmatrix}0\\2\end{bmatrix}$.  
\begin{enumerate}
\item Sketch both figures in the same coordinate plane, and use geometry to explain why $S$ and $P$ have the same area.  Compute the area of $P$ using Formula \ref{form:areaofparallelogramdeterminant}.

$$\text{Area of }P=\answer{4}$$
\item Suppose $M$ is the standard matrix of a linear transformation $T:\RR^2\rightarrow\RR^2$ such that $T(S)=P$.  Find $\det M$.
$$\det M=\answer{1}$$
\end{enumerate}
\end{problem}

\begin{problem}\label{prob:boxprductproof}
Supply the intermediate steps in (\ref{eq:boxproduct}).
\end{problem}

\begin{problem}\label{prob:volparallelepiped}
Find the volume of a parallelepiped determined by 
$$\begin{bmatrix}0\\5\\4\end{bmatrix}\quad\begin{bmatrix}3\\1\\2\end{bmatrix}\quad\begin{bmatrix}1\\1\\6\end{bmatrix}$$
Answer: $\mbox{Volume}=\answer{72}$
\end{problem}

\begin{problem}\label{prob:volparallelepiped0}
Find the volume of a parallelepiped determined by 
$$\begin{bmatrix}1\\4\\-1\end{bmatrix}\quad\begin{bmatrix}3\\-2\\4\end{bmatrix}\quad\begin{bmatrix}5\\6\\2\end{bmatrix}$$
Explain your result geometrically.

Answer: $\mbox{Volume}=\answer{0}$
\end{problem}


\end{document} 