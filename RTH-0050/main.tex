\documentclass{ximera}
%%% Begin Laad packages

\makeatletter
\@ifclassloaded{xourse}{%
    \typeout{Start loading preamble.tex (in a XOURSE)}%
    \def\isXourse{true}   % automatically defined; pre 112022 it had to be set 'manually' in a xourse
}{%
    \typeout{Start loading preamble.tex (NOT in a XOURSE)}%
}
\makeatother

\def\isEn\true 

\pgfplotsset{compat=1.16}

\usepackage{currfile}

% 201908/202301: PAS OP: babel en doclicense lijken problemen te veroorzaken in .jax bestand
% (wegens syntax error met toegevoegde \newcommands ...)
\pdfOnly{
    \usepackage[type={CC},modifier={by-nc-sa},version={4.0}]{doclicense}
    %\usepackage[hyperxmp=false,type={CC},modifier={by-nc-sa},version={4.0}]{doclicense}
    %%% \usepackage[dutch]{babel}
}



\usepackage[utf8]{inputenc}
\usepackage{morewrites}   % nav zomercursus (answer...?)
\usepackage{multirow}
\usepackage{multicol}
\usepackage{tikzsymbols}
\usepackage{ifthen}
%\usepackage{animate} BREAKS HTML STRUCTURE USED BY XIMERA
\usepackage{relsize}

\usepackage{eurosym}    % \euro  (€ werkt niet in xake ...?)
\usepackage{fontawesome} % smileys etc

% Nuttig als ook interactieve beamer slides worden voorzien:
\providecommand{\p}{} % default nothing ; potentially usefull for slides: redefine as \pause
%providecommand{\p}{\pause}

    % Layout-parameters voor het onderschrift bij figuren
\usepackage[margin=10pt,font=small,labelfont=bf, labelsep=endash,format=hang]{caption}
%\usepackage{caption} % captionof
%\usepackage{pdflscape}    % landscape environment

% Met "\newcommand\showtodonotes{}" kan je todonotes tonen (in pdf/online)
% 201908: online werkt het niet (goed)
\providecommand\showtodonotes{disable}
\providecommand\todo[1]{\typeout{TODO #1}}
%\usepackage[\showtodonotes]{todonotes}
%\usepackage{todonotes}

%
% Poging tot aanpassen layout
%
\usepackage{tcolorbox}
\tcbuselibrary{theorems}

%%% Einde laad packages

%%% Begin Ximera specifieke zaken

\graphicspath{
	{../../}
	{../}
	{./}
  	{../../pictures/}
   	{../pictures/}
   	{./pictures/}
	{./explog/}    % M05 in groeimodellen       
}

%%% Einde Ximera specifieke zaken

%
% define softer blue/red/green, use KU Leuven base colors for blue (and dark orange for red ?)
%
% todo: rather redefine blue/red/green ...?
%\definecolor{xmblue}{rgb}{0.01, 0.31, 0.59}
%\definecolor{xmred}{rgb}{0.89, 0.02, 0.17}
\definecolor{xmdarkblue}{rgb}{0.122, 0.671, 0.835}   % KU Leuven Blauw
\definecolor{xmblue}{rgb}{0.114, 0.553, 0.69}        % KU Leuven Blauw
\definecolor{xmgreen}{rgb}{0.13, 0.55, 0.13}         % No KULeuven variant for green found ...

\definecolor{xmaccent}{rgb}{0.867, 0.541, 0.18}      % KU Leuven Accent (orange ...)
\definecolor{kuaccent}{rgb}{0.867, 0.541, 0.18}      % KU Leuven Accent (orange ...)

\colorlet{xmred}{xmaccent!50!black}                  % Darker version of KU Leuven Accent

\providecommand{\blue}[1]{{\color{blue}#1}}    
\providecommand{\red}[1]{{\color{red}#1}}

\renewcommand\CancelColor{\color{xmaccent!50!black}}

% werkt in math en text mode om MATH met oranje (of grijze...)  achtergond te tonen (ook \important{\text{blabla}} lijkt te werken)
%\newcommand{\important}[1]{\ensuremath{\colorbox{xmaccent!50!white}{$#1$}}}   % werkt niet in Mathjax
%\newcommand{\important}[1]{\ensuremath{\colorbox{lightgray}{$#1$}}}
\newcommand{\important}[1]{\ensuremath{\colorbox{orange}{$#1$}}}   % TODO: kleur aanpassen voor mathjax; wordt overschreven infra!


% Uitzonderlijk kan met \pdfnl in de PDF een newline worden geforceerd, die online niet nodig/nuttig is omdat daar de regellengte hoe dan ook niet gekend is.
\ifdefined\HCode%
\providecommand{\pdfnl}{}%
\else%
\providecommand{\pdfnl}{%
  \\%
}%
\fi

% Uitzonderlijk kan met \handoutnl in de handout-PDF een newline worden geforceerd, die noch online noch in de PDF-met-antwoorden nuttig is.
\ifdefined\HCode
\providecommand{\handoutnl}{}
\else
\providecommand{\handoutnl}{%
\ifhandout%
  \nl%
\fi%
}
\fi



% \cellcolor IGNORED by tex4ht ?
% \begin{center} seems not to wordk
    % (missing margin-left: auto;   on tabular-inside-center ???)
%\newcommand{\importantcell}[1]{\ensuremath{\cellcolor{lightgray}#1}}  %  in tabular; usablility to be checked
\providecommand{\importantcell}[1]{\ensuremath{#1}}     % no mathjax2 support for colloring array cells

\pdfOnly{
  \renewcommand{\important}[1]{\ensuremath{\colorbox{kuaccent!50!white}{$#1$}}}
  \renewcommand{\importantcell}[1]{\ensuremath{\cellcolor{kuaccent!40!white}#1}}   
}

%%% Tikz styles


\pgfplotsset{compat=1.16}

\usetikzlibrary{trees,positioning,arrows,fit,shapes,math,calc,decorations.markings,through,intersections,patterns,matrix}

\usetikzlibrary{decorations.pathreplacing,backgrounds}    % 5/2023: from experimental


\usetikzlibrary{angles,quotes}

\usepgfplotslibrary{fillbetween} % bepaalde_integraal
\usepgfplotslibrary{polar}    % oa voor poolcoordinaten.tex

\pgfplotsset{ownstyle/.style={axis lines = center, axis equal image, xlabel = $x$, ylabel = $y$, enlargelimits}} 

\pgfplotsset{
	plot/.style={no marks,samples=50}
}

\newcommand{\xmPlotsColor}{
	\pgfplotsset{
		plot1/.style={darkgray,no marks,samples=100},
		plot2/.style={lightgray,no marks,samples=100},
		plotresult/.style={blue,no marks,samples=100},
		plotblue/.style={blue,no marks,samples=100},
		plotred/.style={red,no marks,samples=100},
		plotgreen/.style={green,no marks,samples=100},
		plotpurple/.style={purple,no marks,samples=100}
	}
}
\newcommand{\xmPlotsBlackWhite}{
	\pgfplotsset{
		plot1/.style={black,loosely dashed,no marks,samples=100},
		plot2/.style={black,loosely dotted,no marks,samples=100},
		plotresult/.style={black,no marks,samples=100},
		plotblue/.style={black,no marks,samples=100},
		plotred/.style={black,dotted,no marks,samples=100},
		plotgreen/.style={black,dashed,no marks,samples=100},
		plotpurple/.style={black,dashdotted,no marks,samples=100}
	}
}


\newcommand{\xmPlotsColorAndStyle}{
	\pgfplotsset{
		plot1/.style={darkgray,no marks,samples=100},
		plot2/.style={lightgray,no marks,samples=100},
		plotresult/.style={blue,no marks,samples=100},
		plotblue/.style={xmblue,no marks,samples=100},
		plotred/.style={xmred,dashed,thick,no marks,samples=100},
		plotgreen/.style={xmgreen,dotted,very thick,no marks,samples=100},
		plotpurple/.style={purple,no marks,samples=100}
	}
}


%\iftikzexport
\xmPlotsColorAndStyle
%\else
%\xmPlotsBlackWhite
%\fi
%%%


%
% Om venndiagrammen te arceren ...
%
\makeatletter
\pgfdeclarepatternformonly[\hatchdistance,\hatchthickness]{north east hatch}% name
{\pgfqpoint{-1pt}{-1pt}}% below left
{\pgfqpoint{\hatchdistance}{\hatchdistance}}% above right
{\pgfpoint{\hatchdistance-1pt}{\hatchdistance-1pt}}%
{
	\pgfsetcolor{\tikz@pattern@color}
	\pgfsetlinewidth{\hatchthickness}
	\pgfpathmoveto{\pgfqpoint{0pt}{0pt}}
	\pgfpathlineto{\pgfqpoint{\hatchdistance}{\hatchdistance}}
	\pgfusepath{stroke}
}
\pgfdeclarepatternformonly[\hatchdistance,\hatchthickness]{north west hatch}% name
{\pgfqpoint{-\hatchthickness}{-\hatchthickness}}% below left
{\pgfqpoint{\hatchdistance+\hatchthickness}{\hatchdistance+\hatchthickness}}% above right
{\pgfpoint{\hatchdistance}{\hatchdistance}}%
{
	\pgfsetcolor{\tikz@pattern@color}
	\pgfsetlinewidth{\hatchthickness}
	\pgfpathmoveto{\pgfqpoint{\hatchdistance+\hatchthickness}{-\hatchthickness}}
	\pgfpathlineto{\pgfqpoint{-\hatchthickness}{\hatchdistance+\hatchthickness}}
	\pgfusepath{stroke}
}
%\makeatother

\tikzset{
    hatch distance/.store in=\hatchdistance,
    hatch distance=10pt,
    hatch thickness/.store in=\hatchthickness,
   	hatch thickness=2pt
}

\colorlet{circle edge}{black}
\colorlet{circle area}{blue!20}


\tikzset{
    filled/.style={fill=green!30, draw=circle edge, thick},
    arceerl/.style={pattern=north east hatch, pattern color=blue!50, draw=circle edge},
    arceerr/.style={pattern=north west hatch, pattern color=yellow!50, draw=circle edge},
    outline/.style={draw=circle edge, thick}
}




%%% Updaten commando's
\def\hoofding #1#2#3{\maketitle}     % OBSOLETE ??

% we willen (bijna) altijd \geqslant ipv \geq ...!
\newcommand{\geqnoslant}{\geq}
\renewcommand{\geq}{\geqslant}
\newcommand{\leqnoslant}{\leq}
\renewcommand{\leq}{\leqslant}

% Todo: (201908) waarom komt er (soms) underlined voor emph ...?
\renewcommand{\emph}[1]{\textit{#1}}

% API commando's

\newcommand{\ds}{\displaystyle}
\newcommand{\ts}{\textstyle}  % tegenhanger van \ds   (Ximera zet PER  DEFAULT \ds!)

% uit Zomercursus-macro's: 
\newcommand{\bron}[1]{\begin{scriptsize} \emph{#1} \end{scriptsize}}     % deprecated ...?


%definities nieuwe commando's - afkortingen veel gebruikte symbolen
\newcommand{\R}{\ensuremath{\mathbb{R}}}
\newcommand{\Rnul}{\ensuremath{\mathbb{R}_0}}
\newcommand{\Reen}{\ensuremath{\mathbb{R}\setminus\{1\}}}
\newcommand{\Rnuleen}{\ensuremath{\mathbb{R}\setminus\{0,1\}}}
\newcommand{\Rplus}{\ensuremath{\mathbb{R}^+}}
\newcommand{\Rmin}{\ensuremath{\mathbb{R}^-}}
\newcommand{\Rnulplus}{\ensuremath{\mathbb{R}_0^+}}
\newcommand{\Rnulmin}{\ensuremath{\mathbb{R}_0^-}}
\newcommand{\Rnuleenplus}{\ensuremath{\mathbb{R}^+\setminus\{0,1\}}}
\newcommand{\N}{\ensuremath{\mathbb{N}}}
\newcommand{\Nnul}{\ensuremath{\mathbb{N}_0}}
\newcommand{\Z}{\ensuremath{\mathbb{Z}}}
\newcommand{\Znul}{\ensuremath{\mathbb{Z}_0}}
\newcommand{\Zplus}{\ensuremath{\mathbb{Z}^+}}
\newcommand{\Zmin}{\ensuremath{\mathbb{Z}^-}}
\newcommand{\Znulplus}{\ensuremath{\mathbb{Z}_0^+}}
\newcommand{\Znulmin}{\ensuremath{\mathbb{Z}_0^-}}
\newcommand{\C}{\ensuremath{\mathbb{C}}}
\newcommand{\Cnul}{\ensuremath{\mathbb{C}_0}}
\newcommand{\Cplus}{\ensuremath{\mathbb{C}^+}}
\newcommand{\Cmin}{\ensuremath{\mathbb{C}^-}}
\newcommand{\Cnulplus}{\ensuremath{\mathbb{C}_0^+}}
\newcommand{\Cnulmin}{\ensuremath{\mathbb{C}_0^-}}
\newcommand{\Q}{\ensuremath{\mathbb{Q}}}
\newcommand{\Qnul}{\ensuremath{\mathbb{Q}_0}}
\newcommand{\Qplus}{\ensuremath{\mathbb{Q}^+}}
\newcommand{\Qmin}{\ensuremath{\mathbb{Q}^-}}
\newcommand{\Qnulplus}{\ensuremath{\mathbb{Q}_0^+}}
\newcommand{\Qnulmin}{\ensuremath{\mathbb{Q}_0^-}}

\newcommand{\perdef}{\overset{\mathrm{def}}{=}}
\newcommand{\pernot}{\overset{\mathrm{notatie}}{=}}
\newcommand\perinderdaad{\overset{!}{=}}     % voorlopig gebruikt in limietenrekenregels
\newcommand\perhaps{\overset{?}{=}}          % voorlopig gebruikt in limietenrekenregels

\newcommand{\degree}{^\circ}


\DeclareMathOperator{\dom}{dom}     % domein
\DeclareMathOperator{\codom}{codom} % codomein
\DeclareMathOperator{\bld}{bld}     % beeld
\DeclareMathOperator{\graf}{graf}   % grafiek
\DeclareMathOperator{\rico}{rico}   % richtingcoëfficient
\DeclareMathOperator{\co}{co}       % coordinaat
\DeclareMathOperator{\gr}{gr}       % graad

\newcommand{\func}[5]{\ensuremath{#1: #2 \rightarrow #3: #4 \mapsto #5}} % Easy to write a function


% Operators
\DeclareMathOperator{\bgsin}{bgsin}
\DeclareMathOperator{\bgcos}{bgcos}
\DeclareMathOperator{\bgtan}{bgtan}
\DeclareMathOperator{\bgcot}{bgcot}
\DeclareMathOperator{\bgsinh}{bgsinh}
\DeclareMathOperator{\bgcosh}{bgcosh}
\DeclareMathOperator{\bgtanh}{bgtanh}
\DeclareMathOperator{\bgcoth}{bgcoth}

% Oude \Bgsin etc deprecated: gebruik \bgsin, en herdefinieer dat als je Bgsin wil!
%\DeclareMathOperator{\cosec}{cosec}    % not used? gebruik \csc en herdefinieer

% operatoren voor differentialen: to be verified; 1/2020: inconsequent gebruik bij afgeleiden/integralen
\renewcommand{\d}{\mathrm{d}}
\newcommand{\dx}{\d x}
\newcommand{\dd}[1]{\frac{\mathrm{d}}{\mathrm{d}#1}}
\newcommand{\ddx}{\dd{x}}

% om in voorbeelden/oefeningen de notatie voor afgeleiden te kunnen kiezen
% Usage: \afg{(2\sin(x))}  (en wordt d/dx, of accent, of D )
%\newcommand{\afg}[1]{{#1}'}
\newcommand{\afg}[1]{\left(#1\right)'}
%\renewcommand{\afg}[1]{\frac{\mathrm{d}#1}{\mathrm{d}x}}   % include in relevant exercises ...
%\renewcommand{\afg}[1]{D{#1}}

%
% \xmxxx commands: Extra KU Leuven functionaliteit van, boven of naast Ximera
%   ( Conventie 8/2019: xm+nederlandse omschrijving, maar is niet consequent gevolgd, en misschien ook niet erg handig !)
%
% (Met een minimale ximera.cls en preamble.tex zou een bruikbare .pdf moeten kunnen worden gemaakt van eender welke ximera)
%
% Usage: \xmtitle[Mijn korte abstract]{Mijn titel}{Mijn abstract}
% Eerste command na \begin{document}:
%  -> definieert de \title
%  -> definieert de abstract
%  -> doet \maketitle ( dus: print de hoofding als 'chapter' of 'sectie')
% Optionele parameter geeft eenn kort abstract (die met de globale setting \xmshortabstract{} al dan niet kan worden geprint.
% De optionele korte abstract kan worden gebruikt voor pseudo-grappige abtsarts, dus dus globaal al dan niet kunnen worden gebuikt...
% Globale settings:
%  de (optionele) 'korte abstract' wordt enkele getoond als \xmshortabstract is gezet
\providecommand\xmshortabstract{} % default: print (only!) short abstract if present
\newcommand{\xmtitle}[3][]{
	\title{#2}
	\begin{abstract}
		\ifdefined\xmshortabstract
		\ifstrempty{#1}{%
			#3
		}{%
			#1
		}%
		\else
		#3
		\fi
	\end{abstract}
	\maketitle
}

% 
% Kleine grapjes: moeten zonder verder gevolg kunnen worden verwijderd
%
%\newcommand{\xmopje}[1]{{\small#1{\reversemarginpar\marginpar{\Smiley}}}}   % probleem in floats!!
\newtoggle{showxmopje}
\toggletrue{showxmopje}

\newcommand{\xmopje}[1]{%
   \iftoggle{showxmopje}{#1}{}%
}


% -> geef een abstracte-formule-met-rechts-een-concreet-voorbeeld
% VB:  \formulevb{a^2+b^2=c^2}{3^2+4^2=5^2}
%
\ifdefined\HCode
\NewEnviron{xmdiv}[1]{\HCode{\Hnewline<div class="#1">\Hnewline}\BODY{\HCode{\Hnewline</div>\Hnewline}}}
\else
\NewEnviron{xmdiv}[1]{\BODY}
\fi

\providecommand{\formulevb}[2]{
	{\centering

    \begin{xmdiv}{xmformulevb}    % zie css voor online layout !!!
	\begin{tabular}{lcl}
		\important{#1}
		&  &
		Vb: $#2$
		\end{tabular}
	\end{xmdiv}

	}
}

\ifdefined\HCode
\providecommand{\vb}[1]{%
    \HCode{\Hnewline<span class="xmvb">}#1\HCode{</span>\Hnewline}%
}
\else
\providecommand{\vb}[1]{
    \colorbox{blue!10}{#1}
}
\fi

\ifdefined\HCode
\providecommand{\xmcolorbox}[2]{
	\HCode{\Hnewline<div class="xmcolorbox">\Hnewline}#2\HCode{\Hnewline</div>\Hnewline}
}
\else
\providecommand{\xmcolorbox}[2]{
  \cellcolor{#1}#2
}
\fi


\ifdefined\HCode
\providecommand{\xmopmerking}[1]{
 \HCode{\Hnewline<div class="xmopmerking">\Hnewline}#1\HCode{\Hnewline</div>\Hnewline}
}
\else
\providecommand{\xmopmerking}[1]{
	{\footnotesize #1}
}
\fi
% \providecommand{\voorbeeld}[1]{
% 	\colorbox{blue!10}{$#1$}
% }



% Hernoem Proof naar Bewijs, nodig voor HTML versie
\renewcommand*{\proofname}{Bewijs}

% Om opgave van oefening (wordt niet geprint bij oplossingenblad)
% (to be tested test)
\NewEnviron{statement}{\BODY}

% Environment 'oplossing' en 'uitkomst'
% voor resp. volledige 'uitwerking' dan wel 'enkel eindresultaat'
% geimplementeerd via feedback, omdat er in de ximera-server adhoc feedback-code is toegevoegd
%% Niet tonen indien handout
%% Te gebruiken om volledige oplossingen/uitwerkingen van oefeningen te tonen
%% \begin{oplossing}        De optelling is commutatief \end{oplossing}  : verschijnt online enkel 'op vraag'
%% \begin{oplossing}[toon]  De optelling is commutatief \end{oplossing}  : verschijnt steeds onmiddellijk online (bv te gebruiken bij voorbeelden) 

\ifhandout%
    \NewEnviron{oplossing}[1][onzichtbaar]%
    {%
    \ifthenelse{\equal{\detokenize{#1}}{\detokenize{toon}}}
    {
    \def\PH@Command{#1}% Use PH@Command to hold the content and be a target for "\expandafter" to expand once.

    \begin{trivlist}% Begin the trivlist to use formating of the "Feedback" label.
    \item[\hskip \labelsep\small\slshape\bfseries Oplossing% Format the "Feedback" label. Don't forget the space.
    %(\texttt{\detokenize\expandafter{\PH@Command}}):% Format (and detokenize) the condition for feedback to trigger
    \hspace{2ex}]\small%\slshape% Insert some space before the actual feedback given.
    \BODY
    \end{trivlist}
    }
    {  % \begin{feedback}[solution]   \BODY     \end{feedback}  }
    }
    }    
\else
% ONLY for HTML; xmoplossing is styled with css, and is not, and need not be a LaTeX environment
% THUS: it does NOT use feedback anymore ...
%    \NewEnviron{oplossing}{\begin{expandable}{xmoplossing}{\nlen{Toon uitwerking}{Show solution}}{\BODY}\end{expandable}}
    \newenvironment{oplossing}[1][onzichtbaar]
   {%
       \begin{expandable}{xmoplossing}{}
   }
   {%
   	   \end{expandable}
   } 
%     \newenvironment{oplossing}[1][onzichtbaar]
%    {%
%        \begin{feedback}[solution]   	
%    }
%    {%
%    	   \end{feedback}
%    } 
\fi

\ifhandout%
    \NewEnviron{uitkomst}[1][onzichtbaar]%
    {%
    \ifthenelse{\equal{\detokenize{#1}}{\detokenize{toon}}}
    {
    \def\PH@Command{#1}% Use PH@Command to hold the content and be a target for "\expandafter" to expand once.

    \begin{trivlist}% Begin the trivlist to use formating of the "Feedback" label.
    \item[\hskip \labelsep\small\slshape\bfseries Uitkomst:% Format the "Feedback" label. Don't forget the space.
    %(\texttt{\detokenize\expandafter{\PH@Command}}):% Format (and detokenize) the condition for feedback to trigger
    \hspace{2ex}]\small%\slshape% Insert some space before the actual feedback given.
    \BODY
    \end{trivlist}
    }
    {  % \begin{feedback}[solution]   \BODY     \end{feedback}  }
    }
    }    
\else
\ifdefined\HCode
   \newenvironment{uitkomst}[1][onzichtbaar]
    {%
        \begin{expandable}{xmuitkomst}{}%
    }
    {%
    	\end{expandable}%
    } 
\else
  % Do NOT print 'uitkomst' in non-handout
  %  (presumably, there is also an 'oplossing' ??)
  \newenvironment{uitkomst}[1][onzichtbaar]{}{}
\fi
\fi

%
% Uitweidingen zijn extra's die niet redelijkerwijze tot de leerstof behoren
% Uitbreidingen zijn extra's die wel redelijkerwijze tot de leerstof van bv meer geavanceerde versies kunnen behoren (B-programma/Wiskundestudenten/...?)
% Nog niet voorzien: design voor verschillende versies (A/B programma, BIO, voorkennis/ ...)
% Voor 'uitweidingen' is er een environment die online per default is ingeklapt, en in pdf al dan niet kan worden geincluded  (via \xmnouitweiding) 
%
% in een xourse, per default GEEN uitweidingen, tenzij \xmuitweiding expliciet ergens is gezet ...
\ifdefined\isXourse
   \ifdefined\xmuitweiding
   \else
       \def\xmnouitweiding{true}
   \fi
\fi

\ifdefined\xmnouitweiding
\newcounter{xmuitweiding}  % anders error undefined ...  
\excludecomment{xmuitweiding}
\else
\newtheoremstyle{dotless}{}{}{}{}{}{}{ }{}
\theoremstyle{dotless}
\newtheorem*{xmuitweidingnofrills}{}   % nofrills = no accordion; gebruikt dus de dotless theoremstyle!

\newcounter{xmuitweiding}
\newenvironment{xmuitweiding}[1][ ]%
{% 
	\refstepcounter{xmuitweiding}%
    \begin{expandable}{xmuitweiding}{\nlentext{Uitweiding \arabic{xmuitweiding}: #1}{Digression \arabic{xmuitweiding}: #1}}%
	\begin{xmuitweidingnofrills}%
}
{%
    \end{xmuitweidingnofrills}%
    \end{expandable}%
}   
% \newenvironment{xmuitweiding}[1][ ]%
% {% 
% 	\refstepcounter{xmuitweiding}
% 	\begin{accordion}\begin{accordion-item}[Uitweiding \arabic{xmuitweiding}: #1]%
% 			\begin{xmuitweidingnofrills}%
% 			}
% 			{\end{xmuitweidingnofrills}\end{accordion-item}\end{accordion}}   
\fi


\newenvironment{xmexpandable}[1][]{
	\begin{accordion}\begin{accordion-item}[#1]%
		}{\end{accordion-item}\end{accordion}}


% Command that gives a selection box online, but just prints the right answer in pdf
\newcommand{\xmonlineChoice}[1]{\pdfOnly{\wordchoicegiventrue}\wordChoice{#1}\pdfOnly{\wordchoicegivenfalse}}   % deprecated, gebruik onlineChoice ...
\newcommand{\onlineChoice}[1]{\pdfOnly{\wordchoicegiventrue}\wordChoice{#1}\pdfOnly{\wordchoicegivenfalse}}

\newcommand{\TJa}{\nlentext{ Ja }{ Yes }}
\newcommand{\TNee}{\nlentext{ Nee }{ No }}
\newcommand{\TJuist}{\nlentext{ Juist }{ True }}
\newcommand{\TFout}{\nlentext{ Fout }{ False }}

\newcommand{\choiceTrue }{{\renewcommand{\choiceminimumhorizontalsize}{4em}\wordChoice{\choice[correct]{\TJuist}\choice{\TFout}}}}
\newcommand{\choiceFalse}{{\renewcommand{\choiceminimumhorizontalsize}{4em}\wordChoice{\choice{\TJuist}\choice[correct]{\TFout}}}}

\newcommand{\choiceYes}{{\renewcommand{\choiceminimumhorizontalsize}{3em}\wordChoice{\choice[correct]{\TJa}\choice{\TNee}}}}
\newcommand{\choiceNo }{{\renewcommand{\choiceminimumhorizontalsize}{3em}\wordChoice{\choice{\TJa}\choice[correct]{\TNee}}}}

% Optional nicer formatting for wordChoice in PDF

\let\inlinechoiceorig\inlinechoice

%\makeatletter
%\providecommand{\choiceminimumverticalsize}{\vphantom{$\frac{\sqrt{2}}{2}$}}   % minimum height of boxes (cfr infra)
\providecommand{\choiceminimumverticalsize}{\vphantom{$\tfrac{2}{2}$}}   % minimum height of boxes (cfr infra)
\providecommand{\choiceminimumhorizontalsize}{1em}   % minimum width of boxes (cfr infra)

\newcommand{\inlinechoicesquares}[2][]{%
		\setkeys{choice}{#1}%
		\ifthenelse{\boolean{\choice@correct}}%
		{%
            \ifhandout%
               \fbox{\choiceminimumverticalsize #2}\allowbreak\ignorespaces%
            \else%
               \fcolorbox{blue}{blue!20}{\choiceminimumverticalsize #2}\allowbreak\ignorespaces\setkeys{choice}{correct=false}\ignorespaces%
            \fi%
		}%
		{% else
			\fbox{\choiceminimumverticalsize #2}\allowbreak\ignorespaces%  HACK: wat kleiner, zodat fits on line ... 	
		}%
}

\newcommand{\inlinechoicesquareX}[2][]{%
		\setkeys{choice}{#1}%
		\ifthenelse{\boolean{\choice@correct}}%
		{%
            \ifhandout%
               \framebox[\ifdim\choiceminimumhorizontalsize<\width\width\else\choiceminimumhorizontalsize\fi]{\choiceminimumverticalsize\ #2\ }\allowbreak\ignorespaces\setkeys{choice}{correct=false}\ignorespaces%
            \else%
               \fcolorbox{blue}{blue!20}{\makebox[\ifdim\choiceminimumhorizontalsize<\width\width\else\choiceminimumhorizontalsize\fi]{\choiceminimumverticalsize #2}}\allowbreak\ignorespaces\setkeys{choice}{correct=false}\ignorespaces%
            \fi%
		}%
		{% else
        \ifhandout%
			\framebox[\ifdim\choiceminimumhorizontalsize<\width\width\else\choiceminimumhorizontalsize\fi]{\choiceminimumverticalsize\ #2\ }\allowbreak\ignorespaces%  HACK: wat kleiner, zodat fits on line ... 	
        \fi
		}%
}


\newcommand{\inlinechoicepuntjes}[2][]{%
		\setkeys{choice}{#1}%
		\ifthenelse{\boolean{\choice@correct}}%
		{%
            \ifhandout%
               \dots\ldots\ignorespaces\setkeys{choice}{correct=false}\ignorespaces
            \else%
               \fcolorbox{blue}{blue!20}{\choiceminimumverticalsize #2}\allowbreak\ignorespaces\setkeys{choice}{correct=false}\ignorespaces%
            \fi%
		}%
		{% else
			%\fbox{\choiceminimumverticalsize #2}\allowbreak\ignorespaces%  HACK: wat kleiner, zodat fits on line ... 	
		}%
}

% print niets, maar definieer globale variable \myanswer
%  (gebruikt om oplossingsbladen te printen) 
\newcommand{\inlinechoicedefanswer}[2][]{%
		\setkeys{choice}{#1}%
		\ifthenelse{\boolean{\choice@correct}}%
		{%
               \gdef\myanswer{#2}\setkeys{choice}{correct=false}

		}%
		{% else
			%\fbox{\choiceminimumverticalsize #2}\allowbreak\ignorespaces%  HACK: wat kleiner, zodat fits on line ... 	
		}%
}



%\makeatother

\newcommand{\setchoicedefanswer}{
\ifdefined\HCode
\else
%    \renewenvironment{multipleChoice@}[1][]{}{} % remove trailing ')'
    \let\inlinechoice\inlinechoicedefanswer
\fi
}

\newcommand{\setchoicepuntjes}{
\ifdefined\HCode
\else
    \renewenvironment{multipleChoice@}[1][]{}{} % remove trailing ')'
    \let\inlinechoice\inlinechoicepuntjes
\fi
}
\newcommand{\setchoicesquares}{
\ifdefined\HCode
\else
    \renewenvironment{multipleChoice@}[1][]{}{} % remove trailing ')'
    \let\inlinechoice\inlinechoicesquares
\fi
}
%
\newcommand{\setchoicesquareX}{
\ifdefined\HCode
\else
    \renewenvironment{multipleChoice@}[1][]{}{} % remove trailing ')'
    \let\inlinechoice\inlinechoicesquareX
\fi
}
%
\newcommand{\setchoicelist}{
\ifdefined\HCode
\else
    \renewenvironment{multipleChoice@}[1][]{}{)}% re-add trailing ')'
    \let\inlinechoice\inlinechoiceorig
\fi
}

\setchoicesquareX  % by default list-of-squares with onlineChoice in PDF

% Omdat multicols niet werkt in html: enkel in pdf  (in html zijn langere pagina's misschien ook minder storend)
\newenvironment{xmmulticols}[1][2]{
 \pdfOnly{\begin{multicols}{#1}}%
}{ \pdfOnly{\end{multicols}}}

%
% Te gebruiken in plaats van \section\subsection
%  (in een printstyle kan dan het level worden aangepast
%    naargelang \chapter vs \section style )
% 3/2021: DO NOT USE \xmsubsection !
\newcommand\xmsection\subsection
\newcommand\xmsubsection\subsubsection

% Aanpassen printversie
%  (hier gedefinieerd, zodat ze in xourse kunnen worden gezet/overschreven)
\providebool{parttoc}
\providebool{printpartfrontpage}
\providebool{printactivitytitle}
\providebool{printactivityqrcode}
\providebool{printactivityurl}
\providebool{printcontinuouspagenumbers}
\providebool{numberactivitiesbysubpart}
\providebool{addtitlenumber}
\providebool{addsectiontitlenumber}
\addtitlenumbertrue
\addsectiontitlenumbertrue

% The following three commands are hardcoded in xake, you can't create other commands like these, without adding them to xake as well
%  ( gebruikt in xourses om juiste soort titelpagina te krijgen voor verschillende ximera's )
\newcommand{\activitychapter}[2][]{
    {    
    \ifstrequal{#1}{notnumbered}{
        \addtitlenumberfalse
    }{}
    \typeout{ACTIVITYCHAPTER #2}   % logging
	\chapterstyle
	\activity{#2}
    }
}
\newcommand{\activitysection}[2][]{
    {
    \ifstrequal{#1}{notnumbered}{
        \addsectiontitlenumberfalse
    }{}
	\typeout{ACTIVITYSECTION #2}   % logging
	\sectionstyle
	\activity{#2}
    }
}
% Practices worden als activity getoond om de grote blokken te krijgen online
\newcommand{\practicesection}[2][]{
    {
    \ifstrequal{#1}{notnumbered}{
        \addsectiontitlenumberfalse
    }{}
    \typeout{PRACTICESECTION #2}   % logging
	\sectionstyle
	\activity{#2}
    }
}
\newcommand{\activitychapterlink}[3][]{
    {
    \ifstrequal{#1}{notnumbered}{
        \addtitlenumberfalse
    }{}
    \typeout{ACTIVITYCHAPTERLINK #3}   % logging
	\chapterstyle
	\activitylink[#1]{#2}{#3}
    }
}

\newcommand{\activitysectionlink}[3][]{
    {
    \ifstrequal{#1}{notnumbered}{
        \addsectiontitlenumberfalse
    }{}
    \typeout{ACTIVITYSECTIONLINK #3}   % logging
	\sectionstyle
	\activitylink[#1]{#2}{#3}
    }
}


% Commando om de printstyle toe te voegen in ximera's. Zorgt ervoor dat er geen problemen zijn als je de xourses compileert
% hack om onhandige relative paden in TeX te omzeilen
% should work both in xourse and ximera (pre-112022 only in ximera; thus obsoletes adhoc setup in xourses)
% loads global.sty if present (cfr global.css for online settings!)
% use global.sty to overwrite settings in printstyle.sty ...
\newcommand{\addPrintStyle}[1]{
\iftikzexport\else   % only in PDF
  \makeatletter
  \ifx\@onlypreamble\@notprerr\else   % ONLY if in tex-preamble   (and e.g. not when included from xourse)
    \typeout{Loading printstyle}   % logging
    \usepackage{#1/printstyle} % mag enkel geinclude worden als je die apart compileert
    \IfFileExists{#1/global.sty}{
        \typeout{Loading printstyle-folder #1/global.sty}   % logging
        \usepackage{#1/global}
        }{
        \typeout{Info: No extra #1/global.sty}   % logging
    }   % load global.sty if present
    \IfFileExists{global.sty}{
        \typeout{Loading local-folder global.sty (or TEXINPUTPATH..)}   % logging
        \usepackage{global}
    }{
        \typeout{Info: No folder/global.sty}   % logging
    }   % load global.sty if present
    \IfFileExists{\currfilebase.sty}
    {
        \typeout{Loading \currfilebase.sty}
        \input{\currfilebase.sty}
    }{
        \typeout{Info: No local \currfilebase.sty}
    }
    \fi
  \makeatother
\fi
}

%
%  
% references: Ximera heeft adhoc logica	 om online labels te doen werken over verschillende files heen
% met \hyperref kan de getoonde tekst toch worden opgegeven, in plaats van af te hangen van de label-text
\ifdefined\HCode
% Link to standard \labels, but give your own description
% Usage:  Volg \hyperref[my_very_verbose_label]{deze link} voor wat tijdverlies
%   (01/2020: Ximera-server aangepast om bij class reference-keeptext de link-text NIET te vervangen door de label-text !!!) 
\renewcommand{\hyperref}[2][]{\HCode{<a class="reference reference-keeptext" href="\##1">}#2\HCode{</a>}}
%
%  Link to specific targets  (not tested ?)
\renewcommand{\hypertarget}[1]{\HCode{<a class="ximera-label" id="#1"></a>}}
\renewcommand{\hyperlink}[2]{\HCode{<a class="reference reference-keeptext" href="\##1">}#2\HCode{</a>}}
\fi

% Mmm, quid English ... (for keyword #1 !) ?
\newcommand{\wikilink}[2]{
    \hyperlink{https://nl.wikipedia.org/wiki/#1}{#2}
    \pdfOnly{\footnote{See \url{https://nl.wikipedia.org/wiki/#1}}
    }
}

\renewcommand{\figurename}{Figuur}
\renewcommand{\tablename}{Tabel}

%
% Gedoe om verschillende versies van xourse/ximera te maken afhankelijk van settings
%
% default: versie met antwoorden
% handout: versie voor de studenten, zonder antwoorden/oplossingen
% full: met alles erop en eraan, dus geschikt voor auteurs en/of lesgevers  (bevat in de pdf ook de 'online-only' stukken!)
%
%
% verder kunnen ook opties/variabele worden gezet voor hints/auteurs/uitweidingen/ etc
%
% 'Full' versie
\newtoggle{showonline}
\ifdefined\HCode   % zet default showOnline
    \toggletrue{showonline} 
\else
    \togglefalse{showonline}
\fi

% Full versie   % deprecated: see infra
\newcommand{\printFull}{
    \hintstrue
    \handoutfalse
    \toggletrue{showonline} 
}

\ifdefined\shouldPrintFull   % deprecated: see infra
    \printFull
\fi



% Overschrijf onlineOnly  (zoals gedefinieerd in ximera.cls)
\ifhandout   % in handout: gebruik de oorspronkelijke ximera.cls implementatie  (is dit wel nodig/nuttig?)
\else
    \iftoggle{showonline}{%
        \ifdefined\HCode
          \RenewEnviron{onlineOnly}{\bgroup\BODY\egroup}   % showOnline, en we zijn  online, dus toon de tekst
        \else
          \RenewEnviron{onlineOnly}{\bgroup\color{red!50!black}\BODY\egroup}   % showOnline, maar we zijn toch niet online: kleur de tekst rood 
        \fi
    }{%
      \RenewEnviron{onlineOnly}{}  % geen showOnline
    }
\fi

% hack om na hoofding van definition/proposition/... als dan niet op een nieuwe lijn te starten
% soms is dat goed en mooi, en soms niet; en in HTML is het nu (2/2020) anders dan in pdf
% vandaar suggestie om 
%     \begin{definition}[Nieuw concept] \nl
% te gebruiken als je zeker een newline wil na de hoofdig en titel
% (in het bijzonder itemize zonder \nl is 'lelijk' ...)
\ifdefined\HCode
\newcommand{\nl}{}
\else
\newcommand{\nl}{\ \par} % newline (achter heading van definition etc.)
\fi


% \nl enkel in handoutmode (ihb voor \wordChoice, die dan typisch veeeel langer wordt)
\ifdefined\HCode
\providecommand{\handoutnl}{}
\else
\providecommand{\handoutnl}{%
\ifhandout%
  \nl%
\fi%
}
\fi

% Could potentially replace \pdfOnline/\begin{onlineOnly} : 
% Usage= \ifonline{Hallo surfer}{Hallo PDFlezer}
\providecommand{\ifonline}[2]%
{
\begin{onlineOnly}#1\end{onlineOnly}%
\pdfOnly{#2}
}%


%
% Maak optionele 'basic' en 'extended' versies van een activity
%  met environment basicOnly, basicSkip en extendedOnly
%
%  (
%   Dit werkt ENKEL in de PDF; de online versies tonen (minstens voorklopig) steeds 
%   het default geval met printbasicversion en printextendversion beide FALSE
%  )
%
\providebool{printbasicversion}
\providebool{printextendedversion}   % not properly implemented
\providebool{printfullversion}       % presumably print everything (debug/auteur)
%
% only set these in xourses, and BEFORE loading this preamble
%
%\newif\ifshowbasic     \showbasictrue        % use this line in xourse to show 'basic' sections
%\newif\ifshowextended  \showextendedtrue     % use this line in xourse to show 'extended' sections
%
%
%\ifbool{showbasic}
%      { \NewEnviron{basicOnly}{\BODY} }    % if yes: just print contents
%      { \NewEnviron{basicOnly}{}      }    % if no:  completely ignore contents
%
%\ifbool{showbasic}
%      { \NewEnviron{basicSkip}{}      }
%      { \NewEnviron{basicSkip}{\BODY} }
%

\ifbool{printextendedversion}
      { \NewEnviron{extendedOnly}{\BODY} }
      { \NewEnviron{extendedOnly}{}      }
      


\ifdefined\HCode    % in html: always print
      {\newenvironment*{basicOnly}{}{}}    % if yes: just print contents
      {\newenvironment*{basicSkip}{}{}}    % if yes: just print contents
\else

\ifbool{printbasicversion}
      {\newenvironment*{basicOnly}{}{}}    % if yes: just print contents
      {\NewEnviron{basicOnly}{}      }    % if no:  completely ignore contents

\ifbool{printbasicversion}
      {\NewEnviron{basicSkip}{}      }
      {\newenvironment*{basicSkip}{}{}}

\fi

\usepackage{float}
\usepackage[rightbars,color]{changebar}

% Full versie
\ifbool{printfullversion}{
    \hintstrue
    \handoutfalse
    \toggletrue{showonline}
    \printbasicversionfalse
    \cbcolor{red}
    \renewenvironment*{basicOnly}{\cbstart}{\cbend}
    \renewenvironment*{basicSkip}{\cbstart}{\cbend}
    \def\xmtoonprintopties{FULL}   % will be printed in footer
}
{}
      
%
% Evalueer \ifhints IN de environment
%  
%
%\RenewEnviron{hint}
%{
%\ifhandout
%\ifhints\else\setbox0\vbox\fi%everything in een emty box
%\bgroup 
%\stepcounter{hintLevel}
%\BODY
%\egroup\ignorespacesafterend
%\addtocounter{hintLevel}{-1}
%\else
%\ifhints
%\begin{trivlist}\item[\hskip \labelsep\small\slshape\bfseries Hint:\hspace{2ex}]
%\small\slshape
%\stepcounter{hintLevel}
%\BODY
%\end{trivlist}
%\addtocounter{hintLevel}{-1}
%\fi
%\fi
%}

% Onafhankelijk van \ifhandout ...? TO BE VERIFIED
\RenewEnviron{hint}
{
\ifhints
\begin{trivlist}\item[\hskip \labelsep\small\bfseries Hint:\hspace{2ex}]
\small%\slshape
\stepcounter{hintLevel}
\BODY
\end{trivlist}
\addtocounter{hintLevel}{-1}
\else
\iftikzexport   % anders worden de tikz tekeningen in hints niet gegenereerd ?
\setbox0\vbox\bgroup
\stepcounter{hintLevel}
\BODY
\egroup\ignorespacesafterend
\addtocounter{hintLevel}{-1}
\fi % ifhandout
\fi %ifhints
}

%
% \tab sets typewriter-tabs (e.g. to format questions)
% (Has no effect in HTML :-( ))
%
\usepackage{tabto}
\ifdefined\HCode
  \renewcommand{\tab}{\quad}    % otherwise dummy .png's are generated ...?
\fi


% Also redefined in  preamble to get correct styling 
% for tikz images for (\tikzexport)
%

\theoremstyle{definition} % Bold titels
\makeatletter
\let\proposition\relax
\let\c@proposition\relax
\let\endproposition\relax
\makeatother
\newtheorem{proposition}{Eigenschap}


%\instructornotesfalse

% logic with \ifhandoutin ximera.cls unclear;so overwrite ...
\makeatletter
\@ifundefined{ifinstructornotes}{%
  \newif\ifinstructornotes
  \instructornotesfalse
  \newenvironment{instructorNotes}{}{}
}{}
\makeatother
\ifinstructornotes
\else
\renewenvironment{instructorNotes}%
{%
    \setbox0\vbox\bgroup
}
{%
    \egroup
}
\fi

% \RedeclareMathOperator
% from https://tex.stackexchange.com/questions/175251/how-to-redefine-a-command-using-declaremathoperator
\makeatletter
\newcommand\RedeclareMathOperator{%
    \@ifstar{\def\rmo@s{m}\rmo@redeclare}{\def\rmo@s{o}\rmo@redeclare}%
}
% this is taken from \renew@command
\newcommand\rmo@redeclare[2]{%
    \begingroup \escapechar\m@ne\xdef\@gtempa{{\string#1}}\endgroup
    \expandafter\@ifundefined\@gtempa
    {\@latex@error{\noexpand#1undefined}\@ehc}%
    \relax
    \expandafter\rmo@declmathop\rmo@s{#1}{#2}}
% This is just \@declmathop without \@ifdefinable
\newcommand\rmo@declmathop[3]{%
    \DeclareRobustCommand{#2}{\qopname\newmcodes@#1{#3}}%
}
\@onlypreamble\RedeclareMathOperator
\makeatother


%
% Engelse vertaling, vooral in mathmode
%
% 1. Algemene procedure
%
\ifdefined\isEn
 \newcommand{\nlen}[2]{#2}
 \newcommand{\nlentext}[2]{\text{#2}}
 \newcommand{\nlentextbf}[2]{\textbf{#2}}
\else
 \newcommand{\nlen}[2]{#1}
 \newcommand{\nlentext}[2]{\text{#1}}
 \newcommand{\nlentextbf}[2]{\textbf{#1}}
\fi

%
% 2. Lijst van erg veel gebruikte uitdrukkingen
%

% Ja/Nee/Fout/Juits etc
%\newcommand{\TJa}{\nlentext{ Ja }{ and }}
%\newcommand{\TNee}{\nlentext{ Nee }{ No }}
%\newcommand{\TJuist}{\nlentext{ Juist }{ Correct }
%\newcommand{\TFout}{\nlentext{ Fout }{ Wrong }
\newcommand{\TWaar}{\nlentext{ Waar }{ True }}
\newcommand{\TOnwaar}{\nlentext{ Vals }{ False }}
% Korte bindwoorden en, of, dus, ...
\newcommand{\Ten}{\nlentext{ en }{ and }}
\newcommand{\Tof}{\nlentext{ of }{ or }}
\newcommand{\Tdus}{\nlentext{ dus }{ so }}
\newcommand{\Tendus}{\nlentext{ en dus }{ and thus }}
\newcommand{\Tvooralle}{\nlentext{ voor alle }{ for all }}
\newcommand{\Took}{\nlentext{ ook }{ also }}
\newcommand{\Tals}{\nlentext{ als }{ when }} %of if?
\newcommand{\Twant}{\nlentext{ want }{ as }}
\newcommand{\Tmaal}{\nlentext{ maal }{ times }}
\newcommand{\Toptellen}{\nlentext{ optellen }{ add }}
\newcommand{\Tde}{\nlentext{ de }{ the }}
\newcommand{\Thet}{\nlentext{ het }{ the }}
\newcommand{\Tis}{\nlentext{ is }{ is }} %zodat is in text staat in mathmode (geen italics)
\newcommand{\Tmet}{\nlentext{ met }{ where }} % in situaties e.g met p < n --> where p < n
\newcommand{\Tnooit}{\nlentext{ nooit }{ never }}
\newcommand{\Tmaar}{\nlentext{ maar }{ but }}
\newcommand{\Tniet}{\nlentext{ niet }{ not }}
\newcommand{\Tuit}{\nlentext{ uit }{ from }}
\newcommand{\Ttov}{\nlentext{ t.o.v. }{ w.r.t. }}
\newcommand{\Tzodat}{\nlentext{ zodat }{ such that }}
\newcommand{\Tdeth}{\nlentext{de }{th }}
\newcommand{\Tomdat}{\nlentext{omdat }{because }} 


%
% Overschrijf addhoc commando's
%
\ifdefined\isEn
\renewcommand{\pernot}{\overset{\mathrm{notation}}{=}}
\RedeclareMathOperator{\bld}{im}     % beeld
\RedeclareMathOperator{\graf}{graph}   % grafiek
\RedeclareMathOperator{\rico}{slope}   % richtingcoëfficient
\RedeclareMathOperator{\co}{co}       % coordinaat
\RedeclareMathOperator{\gr}{deg}       % graad

% Operators
\RedeclareMathOperator{\bgsin}{arcsin}
\RedeclareMathOperator{\bgcos}{arccos}
\RedeclareMathOperator{\bgtan}{arctan}
\RedeclareMathOperator{\bgcot}{arccot}
\RedeclareMathOperator{\bgsinh}{arcsinh}
\RedeclareMathOperator{\bgcosh}{arccosh}
\RedeclareMathOperator{\bgtanh}{arctanh}
\RedeclareMathOperator{\bgcoth}{arccoth}

\fi


% HACK: use 'oplossing' for 'explanation' ...
\let\explanation\relax
\let\endexplanation\relax
% \newenvironment{explanation}{\begin{oplossing}}{\end{oplossing}}
\newcounter{explanation}

\ifhandout%
    \NewEnviron{explanation}[1][toon]%
    {%
    \RenewEnviron{verbatim}{ \red{VERBATIM CONTENT MISSING IN THIS PDF}} %% \expandafter\verb|\BODY|}

    \ifthenelse{\equal{\detokenize{#1}}{\detokenize{toon}}}
    {
    \def\PH@Command{#1}% Use PH@Command to hold the content and be a target for "\expandafter" to expand once.

    \begin{trivlist}% Begin the trivlist to use formating of the "Feedback" label.
    \item[\hskip \labelsep\small\slshape\bfseries Explanation:% Format the "Feedback" label. Don't forget the space.
    %(\texttt{\detokenize\expandafter{\PH@Command}}):% Format (and detokenize) the condition for feedback to trigger
    \hspace{2ex}]\small%\slshape% Insert some space before the actual feedback given.
    \BODY
    \end{trivlist}
    }
    {  % \begin{feedback}[solution]   \BODY     \end{feedback}  }
    }
    }    
\else
% ONLY for HTML; xmoplossing is styled with css, and is not, and need not be a LaTeX environment
% THUS: it does NOT use feedback anymore ...
%    \NewEnviron{oplossing}{\begin{expandable}{xmoplossing}{\nlen{Toon uitwerking}{Show solution}}{\BODY}\end{expandable}}
    \newenvironment{explanation}[1][toon]
   {%
       \begin{expandable}{xmoplossing}{}
   }
   {%
   	   \end{expandable}
   } 
\fi

\begin{document}

\begin{abstract}
\end{abstract}
\maketitle

If $A$ is an $n \times n$ matrix, the characteristic polynomial $c_{A}(x)$ is a polynomial of degree $n$ and the eigenvalues of $A$ are just the roots of $c_{A}(x)$. In most of our examples these roots have been \textit{real} numbers (in fact, the examples have been carefully chosen so this will be the case!); but it need not happen, even when the characteristic polynomial has real coefficients. For example, if $A = \left[ \begin{array}{rr}
0 & 1 \\
-1 & 0
\end{array}\right]$ then $c_{A}(x) = x^{2} + 1$ has roots $i$ and $-i$, where $i$ is a complex number satisfying $i^{2} = -1$. Therefore, we have to deal with the possibility that the eigenvalues of a (real) square matrix might be complex numbers.

In fact, nearly everything in this book would remain true if the phrase \textit{real number}\index{real numbers} were replaced by \textit{complex number}
 wherever it occurs. Then we would deal with matrices with complex
entries, systems of linear equations with complex coefficients (and
complex solutions), determinants of complex matrices, and vector spaces
with scalar multiplication by any complex number allowed. Moreover, the
proofs of most theorems about (the real version of) these concepts
extend easily to the complex case. It is not our intention here to give a
 full treatment of complex linear algebra. However, we will carry the
theory far enough to give another proof of the Real Spectral Theorem (\ref{th:PrinAxes}).

The set of complex numbers is denoted $\mathbb{C}$ . We will use only the most basic properties of these numbers (mainly conjugation and absolute values), and the reader can find this material in Appendix~\ref{chap:appacomplexnumbers}.

If $n \ge 1$, we denote the set of all $n$-tuples of complex numbers by $\mathbb{C}^n$. As with $\RR^n$, these $n$-tuples will be written either as row or column matrices and will be referred to as \dfn{vectors}. We define vector operations on $\mathbb{C}^n$ as follows:
\begin{align*}
[v_{1},  v_{2}, \ldots, v_{n}] + [w_{1}, w_{2}, \ldots, w_{n}] &= [v_{1} + w_{1}, v_{2} + w_{2}, \ldots, v_{n} + w_{n}] \\
u[v_{1}, v_{2}, \ldots, v_{n}] &= [uv_{1}, uv_{2}, \ldots, uv_{n}] \quad \mbox{ for } u \mbox{ in } \mathbb{C}
\end{align*}
With these definitions, $\mathbb{C}^n$ satisfies the axioms for a vector space (with complex scalars) given in \href{https://ximera.osu.edu/oerlinalg/LinearAlgebra/VSP-0050/main}{Abstract Vector Spaces}. Thus we can speak of spanning sets for $\mathbb{C}^n$, of linearly independent subsets, and of bases. In all cases, the definitions are identical to the real case, except that the scalars are allowed to be complex numbers. In particular, the standard basis of $\RR^n$ remains a basis of $\mathbb{C}^n$, called the \dfn{standard basis} of $\mathbb{C}^n$.


\subsection*{The Standard Inner Product}

There is a natural generalization to $\mathbb{C}^n$ of the dot product in $\RR^n$.

\begin{definition}{Standard Inner Product in $\RR^n$}\label{def:025549}
Given $\vec{z} = (z_{1}, z_{2}, \ldots, z_{n})$ and $\vec{w} = (w_{1}, w_{2}, \ldots, w_{n})$ in $\mathbb{C}^n$, define their \dfn{standard inner product} $\langle \vec{z}, \vec{w} \rangle$ by
\begin{equation*}
\langle \vec{z}, \vec{w} \rangle = z_{1}\overline{w}_{1} + z_{2}\overline{w}_{2} + \ldots + z_{n}\overline{w}_{n}
\end{equation*}
where $\overline{w}$ is the conjugate of the complex number $w$.
\end{definition}

Clearly, if $\vec{z}$ and $\vec{w}$ actually lie in $\RR^n$, then $\langle \vec{z}, \vec{w} \rangle = \vec{z} \dotp \vec{w}$ is the usual dot product.


\begin{example}\label{ex:025563}
If $\vec{z} = (2, 1 - i, 2i, 3 - i)$ and $\vec{w} = (1 - i, -1, -i, 3 + 2i)$, then
\begin{align*}
\langle \vec{z}, \vec{w} \rangle &= 2(1 + i) + (1 - i)(-1) + (2i)(i) + (3 - i)(3 - 2i) = 6 -6i \\
\langle \vec{z}, \vec{z} \rangle &= 2 \cdot 2 + (1 - i)(1 + i) + (2i)(-2i) + (3 - i)(3 + i) = 20
\end{align*}
\end{example}

Note that $\langle \vec{z}, \vec{w} \rangle$ is a complex number in general. However, if $\vec{w} = \vec{z} = (z_{1}, z_{2}, \ldots, z_{n})$, the definition gives $\langle \vec{z}, \vec{z} \rangle = |z_{1}|^{2} + \ldots  + |z_{n}|^{2}$ which is a nonnegative real number, equal to $0$ if and only if $\vec{z} = \vec{0}$. This explains the conjugation in the definition of $\langle \vec{z}, \vec{w} \rangle$, and it gives \ref{th:025575d} of the following theorem.


\begin{theorem}\label{th:025575}
Let $\vec{z}$, $\vec{z}_{1}$, $\vec{w}$, and $\vec{w}_{1}$ denote vectors in $\mathbb{C}^n$, and let $\lambda$ denote a complex number.
\begin{enumerate}
\item\label{th:025575a} $\langle \vec{z} + \vec{z}_{1}, \vec{w}\rangle = \langle \vec{z}, \vec{w} \rangle + \langle \vec{z}_{1}, \vec{w} \rangle$ \quad and \quad
$\langle \vec{z}, \vec{w} + \vec{w}_{1} \rangle = \langle \vec{z}, \vec{w} \rangle + \langle \vec{z}, \vec{w}_{1} \rangle$.

\item\label{th:025575b} $\langle \lambda \vec{z}, \vec{w} \rangle = \lambda \langle \vec{z}, \vec{w} \rangle$ \quad and \quad $\langle \vec{z}, \lambda \vec{w} \rangle = \overline{\lambda} \langle \vec{z}, \vec{w} \rangle$.

\item\label{th:025575c} $\langle \vec{z}, \vec{w} \rangle = \overline{\langle \vec{w}, \vec{z} \rangle}$.

\item\label{th:025575d} $\langle \vec{z}, \vec{z} \rangle \ge 0$, \quad and \quad $\langle \vec{z}, \vec{z} \rangle = 0$ if and only if $\vec{z} = \vec{0}$.

\end{enumerate}
\end{theorem}

\begin{proof}
We leave \ref{th:025575a} and \ref{th:025575b} to the reader (Practice Problem \ref{prb:complex_matrices10}), and \ref{th:025575d} has already been proved. To prove \ref{th:025575c}, write $\vec{z} = (z_{1}, z_{2}, \ldots, z_{n})$ and $\vec{w} = (w_{1}, w_{2}, \ldots, w_{n})$. Then
\begin{align*}
\overline{\langle \vec{w}, \vec{z} \rangle} = (\overline{w_{1}\overline{z}_{1} + \ldots + w_{n}\overline{z}_{n}}) &= \overline{w}_{1}\overline{\overline{z}}_{1} + \ldots + \overline{w}_{n}\overline{\overline{z}}_{n} \\
&= z_{1}\overline{w}_{1} + \ldots + z_{n}\overline{w}_{n} = \langle \vec{z}, \vec{w} \rangle
\end{align*}
\end{proof}

\begin{definition}\label{def:025606}
As for the dot product on $\RR^n$, property \ref{th:025575d} enables us to define the \dfn{norm} or \dfn{length} $\norm{\vec{z}}$ of a vector $\vec{z} = (z_{1}, z_{2}, \ldots, z_{n})$ in $\mathbb{C}^n$:
\begin{equation*}
\norm{\vec{z}} = \sqrt{\langle \vec{z}, \vec{z} \rangle} = \sqrt{|z_{1}|^2 + |z_{2}|^2 + \ldots + |z_{n}|^2}
\end{equation*}
\end{definition}

The only properties of the norm function we will need are the following (the proofs are left to the reader):


\begin{theorem}\label{th:025616}
If $\vec{z}$ is any vector in $\mathbb{C}^n$, then


\begin{enumerate}
\item\label{th:025616a} $\norm{\vec{z}} \ge 0$ and $\norm{\vec{z}} = 0$ if and only if $\vec{z} = \vec{0}$.

\item\label{th:025616b} $\norm{\lambda\vec{z}} = |\lambda| \norm{\vec{z}}$ for all complex numbers $\lambda$.

\end{enumerate}
\end{theorem}

A vector $\vec{u}$ in $\mathbb{C}^n$ is called a \dfn{unit vector} if $\norm{\vec{u}} = 1$. Property \ref{th:025616b} in Theorem~\ref{th:025616} then shows that if $\vec{z} \neq \vec{0}$ is any nonzero vector in $\mathbb{C}^n$, then $\vec{u} = \frac{1}{\norm{\vec{z}}}\vec{z}$ is a unit vector.


\begin{example}\label{ex:025631}
In $\mathbb{C}^4$, find a unit vector $\vec{u}$ that is a positive real multiple of $\vec{z} = \begin{bmatrix} 1 - i\\ i\\ 2\\ 3 + 4i \end{bmatrix}$.


\begin{explanation}
$\norm{\vec{z}} = \sqrt{2+1+4+25} = \sqrt{32} = 4\sqrt{2}$, so take $\vec{u} = \frac{1}{4\sqrt{2}}\vec{z}$.
\end{explanation}
\end{example}

A matrix $A = \left[ a_{ij} \right]$ is called a \textbf{complex matrix}\index{complex matrix!defined} if every entry $a_{ij}$ is a complex number. The notion of conjugation for complex numbers extends to matrices as follows: Define the \textbf{conjugate}\index{complex matrix!conjugate}\index{conjugate} of $A = \left[ a_{ij} \right]$ to be the matrix
\begin{equation*}
\overline{A} = \left[ \begin{array}{c} \overline{a}_{ij} \end{array}\right]
\end{equation*}
obtained from $A$ by conjugating every entry. Then (using Appendix~\ref{chap:appacomplexnumbers})
\begin{equation*}
\overline{A + B} = \overline{A} + \overline{B} \quad \mbox{ and } \quad \overline{AB} = \overline{A} \; \overline{B}
\end{equation*}
holds for all (complex) matrices of appropriate size.

Transposition of complex matrices is defined just as in the real case, and the following notion is fundamental.


\begin{definition}\label{dfn:conjtrans}
The \dfn{conjugate transpose} $A^{H}$ of a complex matrix $A$ is defined by
\begin{equation*}
A^H = (\overline{A})^T = \overline{(A^T)}
\end{equation*}
\end{definition}

\noindent Observe that $A^{H} = A^{T}$ when $A$ is real.\footnote{Other notations for $A^{H}$ are $A^\ast$ and $A^\dagger$.}

\begin{example}\label{ex:025654}
\begin{equation*}
\left[ \begin{array}{ccr}
3 & 1 - i & 2 + i \\
2i & 5 + 2i & -i
\end{array}\right]^H = \left[ \begin{array}{cc}
3 & -2i \\
1 + i & 5 - 2i \\
2 - i & i
\end{array}\right]
\end{equation*}
\end{example}

The following properties of $A^{H}$ follow easily from the rules for transposition of real matrices and extend these rules to complex matrices. Note the conjugate in property \ref{th:025659c}.


\begin{theorem}\label{th:025659}
Let $A$ and $B$ denote complex matrices, and let $\lambda$ be a complex number.

\begin{enumerate}
\item\label{th:025659a} $(A^{H})^{H} = A$.

\item\label{th:025659b} $(A + B)^{H} = A^{H} + B^{H}$.

\item\label{th:025659c} $(\lambda A)^H = \overline{\lambda}A^H$.

\item\label{th:025659d} $(AB)^{H} = B^{H}A^{H}$.

\end{enumerate}
\end{theorem}

\subsection*{Hermitian and Unitary Matrices}


If $A$ is a real symmetric matrix, it is clear that $A^{H} = A$. The complex matrices that satisfy this condition turn out to be the
most natural generalization of the real symmetric matrices:

\begin{definition}\label{def:Hermitian}
A square complex matrix $A$ is called \dfn{Hermitian} if $A^{H} = A$, equivalently $\overline{A} = A^T$.
\end{definition}
\begin{remark}
The name Hermitian honours Charles Hermite (1822--1901), a French
mathematician who worked primarily in analysis and is remembered as the
first to show that the number $e$ from calculus is transcendental---that is, $e$ is not a root of any polynomial with integer coefficients.
\end{remark}

Hermitian matrices are easy to
recognize because the entries on the main diagonal must be real, and the
 ``reflection'' of each off-diagonal entry in the main diagonal must be the
 conjugate of that entry.

\begin{example}\label{ex:025690}
$\left[ \begin{array}{ccc}
3 & i & 2 + i \\
-i & -2 & -7 \\
2 - i & -7 & 1
\end{array}\right]$
 is Hermitian, whereas $\left[ \begin{array}{rr}
 1 & i \\
 i & -2
 \end{array}\right]$ and $\left[ \begin{array}{rr}
 1 & i \\
 -i & i
 \end{array}\right]$ are not.
\end{example}

The following theorem extends Theorem~\ref{th:024396}, and gives a very useful characterization of Hermitian matrices in terms of the standard inner product in $\mathbb{C}^n$.


\begin{theorem}\label{th:025697}
An $n \times n$ complex matrix $A$ is Hermitian if and only if
\begin{equation*}
\langle A\vec{z}, \vec{w} \rangle = \langle \vec{z}, A\vec{w} \rangle
\end{equation*}
for all $n$-tuples $\vec{z}$ and $\vec{w}$ in $\mathbb{C}^n$.
\end{theorem}

\begin{proof}
If $A$ is Hermitian, we have $A^T = \overline{A}$. If $\vec{z}$ and $\vec{w}$ are columns in $\mathbb{C}^n$, then $\langle \vec{z}, \vec{w} \rangle = \vec{z}^T\overline{\vec{w}}$, so
\begin{equation*}
\langle A\vec{z}, \vec{w} \rangle =(A\vec{z})^T\overline{\vec{w}} = \vec{z}^TA^T\overline{\vec{w}} = \vec{z}^T\overline{A}\overline{\vec{w}} = \vec{z}^T(\overline{A\vec{w}}) = \langle \vec{z}, A\vec{w} \rangle
\end{equation*}
To prove the converse, let $\vec{e}_{j}$ denote column $j$ of the identity matrix. If $A = \left[ a_{ij} \right]$, the condition gives
\begin{equation*}
\overline{a}_{ij} = \langle \vec{e}_{i}, A\vec{e}_{j} \rangle = \langle A\vec{e}_{i}, \vec{e}_{j} \rangle = {a}_{ij}
\end{equation*}
Hence $\overline{A} = A^T$, so $A$ is Hermitian.
\end{proof}

Let $A$ be an $n \times n$ complex matrix. As in the real case, a complex number $\lambda$ is called an \textbf{eigenvalue}\index{eigenvalues!complex matrix}\index{complex matrix!eigenvalues} of $A$ if $A\vec{x} = \lambda \vec{x}$ holds for some column $\vec{x} \neq \vec{0}$ in $\mathbb{C}^n$. In this case $\vec{x}$ is called an \dfn{eigenvector} of $A$ corresponding to $\lambda$. The \dfn{characteristic polynomial} $c_{A}(x)$ is defined by
\begin{equation*}
c_{A}(x) = \mbox{det}(xI - A)
\end{equation*}
This polynomial has complex coefficients (possibly nonreal). However, the proof of Theorem~\ref{th:009033} goes through to show that the eigenvalues of $A$ are the roots (possibly complex) of $c_{A}(x)$.

It is at this point that the advantage
of working with complex numbers becomes apparent. The real numbers are
incomplete in the sense that the characteristic polynomial of a real
matrix may fail to have all its roots real. However, this difficulty
does not occur for the complex numbers. The so-called fundamental
theorem of algebra ensures that \textit{every} polynomial of positive degree with complex coefficients has a complex root. Hence every square complex matrix $A$ has a (complex) eigenvalue. Indeed (Appendix~\ref{chap:appacomplexnumbers}), $c_{A}(x)$ factors completely as follows:
\begin{equation*}
c_{A}(x) = (x -\lambda_{1})(x -\lambda_{2}) \cdots (x -\lambda_{n})
\end{equation*}
where $\lambda_{1}, \lambda_{2}, \ldots, \lambda_{n}$ are the eigenvalues of $A$ (with possible repetitions due to multiple roots).

The next result shows that, for
Hermitian matrices, the eigenvalues are actually real. Because symmetric
 real matrices are Hermitian, this re-proves Theorem~\ref{th:PrinAxes}. It also extends Theorem~\ref{th:symmetric_has_ortho_ev},
 which asserts that eigenvectors of a symmetric real matrix
corresponding to distinct eigenvalues are actually orthogonal. In the
complex context, two $n$-tuples $\vec{z}$ and $\vec{w}$ in $\mathbb{C}^n$ are said to be \textbf{orthogonal} if $\langle \vec{z}, \vec{w} \rangle = 0$.


\begin{theorem}\label{th:025729}
Suppose $A$ is a Hermitian matrix.

\begin{enumerate}
\item\label{th:025729a} The eigenvalues of $A$ are real.

\item\label{th:025729b} Eigenvectors of $A$ corresponding to distinct eigenvalues are orthogonal.

\end{enumerate}
\end{theorem}

\begin{proof}
Let $\lambda$ and $\mu$ be eigenvalues of $A$ with (nonzero) eigenvectors $\vec{z}$ and $\vec{w}$. Then $A\vec{z} = \lambda \vec{z}$ and $A\vec{w} = \mu \vec{w}$, so Theorem~\ref{thm:025697} gives
\begin{equation} \label{eigenvalEq}
\lambda \langle \vec{z}, \vec{w} \rangle = \langle \lambda \vec{z}, \vec{w} \rangle = \langle A\vec{z}, \vec{w} \rangle = \langle \vec{z}, A\vec{w} \rangle = \langle \vec{z}, \mu \vec{w} \rangle = \overline{\mu} \langle \vec{z}, \vec{w} \rangle
\end{equation}
If $\mu = \lambda$ and $\vec{w} = \vec{z}$, this becomes $\lambda \langle \vec{z}, \vec{z} \rangle  = \overline{\lambda} \langle \vec{z}, \vec{z} \rangle$. Because $\langle \vec{z}, \vec{z} \rangle = \norm{\vec{z}}^{2} \neq 0$, this implies $\lambda = \overline{\lambda}$. Thus $\lambda$ is real, proving (1). Similarly, $\mu$ is real, so equation (\ref{eigenvalEq}) gives $\lambda \langle \vec{z}, \vec{w} \rangle = \mu \langle \vec{z}, \vec{w} \rangle$. If $\lambda \neq \mu$, this implies $\langle \vec{z}, \vec{w} \rangle = 0$, proving (2).
\end{proof}

The Real Spectral Theorem (\ref{th:PrinAxes}) asserts that every real symmetric matrix $A$ is orthogonally diagonalizable---that is $U^{T}AU$ is diagonal where $U$ is an orthogonal matrix $(U^{-1} = U^{T})$. The next theorem identifies the complex analogs of these orthogonal real matrices.

\begin{definition}\label{def:025749}
As in the real case, a set of nonzero vectors $\{\vec{z}_{1}, \vec{z}_{2}, \ldots, \vec{z}_{m}\}$ in $\mathbb{C}^n$ is called \dfn{orthogonal} if $\langle \vec{z}_{i}, \vec{z}_{j}\rangle = 0$ whenever $i \neq j$, and it is \dfn{orthonormal} if, in addition, $\norm{\vec{z}_{i} } = 1$ for each $i$.
\end{definition}

\begin{theorem}\label{th:025759}
The following are equivalent for an $n \times n$ complex matrix $A$.

\begin{enumerate}
\item\label{th:025759a} $A$ is invertible and $A^{-1} = A^{H}$.

\item\label{th:025759b} The rows of $A$ are an orthonormal set in $\mathbb{C}^n$.

\item\label{th:025759c} The columns of $A$ are an orthonormal set in $\mathbb{C}^n$.

\end{enumerate}
\end{theorem}

\begin{proof}
If $A = \left[ \begin{array}{cccc}
\vec{c}_{1} & \vec{c}_{2} & \cdots & \vec{c}_{n}
\end{array}\right]$ is a complex matrix with $j$th column $\vec{c}_{j}$, then $A^T\overline{A} = \left[ \langle \vec{c}_{i}, \vec{c}_{j}\rangle \right]$, as in Theorem~\ref{thm:024227}. Now \ref{th:025759a} $\Leftrightarrow$ \ref{th:025759b} follows, and \ref{th:025759a} $\Leftrightarrow$ \ref{th:025759c} is proved in the same way.
\end{proof}

\begin{definition}\label{def:Unitary}
A square complex matrix $U$ is called \dfn{unitary} if $U^{-1} = U^{H}$.
\end{definition}

Thus a real matrix is unitary if and only if it is orthogonal.


\begin{example}\label{ex:025787}
The matrix $A = \left[ \begin{array}{rr}
1 + i & 1 \\
1 - i & i
\end{array}\right]$ has orthogonal columns, but the rows are not orthogonal. Normalizing the columns gives the unitary matrix $\frac{1}{2}\left[ \begin{array}{rr}
	1 + i & \sqrt{2} \\
	1 - i & \sqrt{2}i
\end{array}\right]$.
\end{example}

Given a real symmetric matrix $A$, we saw in RTH-0035 a procedure for finding an orthogonal matrix $U$ such that $U^{T}AU$ is diagonal (see Example~\ref{exa:024374}). The following example illustrates Theorem~\ref{th:025729} and shows that the technique works for complex matrices.

\begin{example}\label{ex:025794}
Consider the Hermitian matrix $A = \left[ \begin{array}{cc}
3 & 2 + i \\
2 - i & 7
\end{array}\right]$. Find the eigenvalues of $A$, find two orthonormal eigenvectors, and so find a unitary matrix $U$ such that $U^{H}AU$ is diagonal.


\begin{explanation}
  The characteristic polynomial of $A$ is
\begin{equation*}
c_{A}(x) = \mbox{det}(xI - A) = \mbox{det}\left[ \begin{array}{rr}
x - 3 & -2 - i \\
-2 + i & x - 7
\end{array}\right] = (x-2)(x-8)
\end{equation*}
Hence the eigenvalues are $2$ and $8$ (both real as expected), and corresponding eigenvectors are $\left[ \begin{array}{cc}
2 + i \\
-1
\end{array}\right]$ and $\left[ \begin{array}{cc}
1 \\
2 - i
\end{array}\right]$ (orthogonal as expected). Each has length $\sqrt{6}$,
 so let $U = \frac{1}{\sqrt{6}}\left[ \begin{array}{cc}
 2 + i & 1 \\
 -1 & 2 - i
 \end{array}\right]$ be the unitary matrix with the normalized eigenvectors as columns.

Then $U^HAU = \left[ \begin{array}{rr}
2 & 0 \\
0 & 8
\end{array}\right]$ is diagonal.
\end{explanation}
\end{example}

\subsection*{Unitary Diagonalization}

An $n \times n$ complex matrix $A$ is called \dfn{unitarily diagonalizable} if $U^{H}AU$ is diagonal for some unitary matrix $U$. As Example~\ref{ex:025794} suggests, we are going to prove that every Hermitian matrix is unitarily diagonalizable. However, with only a little extra effort, we can get a very important theorem that has this result as an easy consequence.

A complex matrix is called \textbf{upper triangular} if every entry below the main diagonal is zero. We owe the following theorem to Issai Schur.
\begin{remark}
Issai
 Schur (1875--1941) was a German mathematician who did fundamental work
in the theory of representations of groups as matrices.
\end{remark} 

\begin{theorem}[Schur's Theorem]\label{th:025814}
If $A$ is any $n \times n$ complex matrix, there exists a unitary matrix $U$ such that
\begin{equation*}
U^HAU = T
\end{equation*}
is upper triangular. Moreover, the entries on the main diagonal of $T$ are the eigenvalues $\lambda_{1}, \lambda_{2}, \ldots, \lambda_{n}$ of $A$ (including multiplicities).
\end{theorem}

\begin{proof}
We use induction on $n$. If $n = 1$, $A$ is already upper triangular. If $n > 1$, assume the theorem is valid for $(n - 1) \times (n - 1)$ complex matrices. Let $\lambda_{1}$ be an eigenvalue of $A$, and let $\vec{y}_{1}$ be an eigenvector with $\norm{\vec{y}_{1}} = 1$. Then $\vec{y}_{1}$ is part of a basis of $\mathbb{C}^n$ (by the analog of Theorem~\ref{thm:019430}), so the (complex analog of the) Gram-Schmidt process provides $\vec{y}_{2}, \ldots, \vec{y}_{n}$ such that $\{\vec{y}_{1}, \vec{y}_{2}, \ldots, \vec{y}_{n}\}$ is an orthonormal basis of $\mathbb{C}^n$. If $U_{1} = \left[ \begin{array}{cccc}
\vec{y}_{1} & \vec{y}_{2} & \cdots & \vec{y}_{n}
\end{array}\right]$ is the matrix with these vectors as its columns, then (see )
\begin{equation*}
U_{1}^HAU_{1} = \left[ \begin{array}{cc}
\lambda_{1} & X_{1} \\
0 & A_{1}
\end{array}\right]
\end{equation*}
in block form. Now apply induction to find a unitary $(n - 1) \times (n - 1)$ matrix $W_{1}$ such that $W_{1}^HA_{1}W_{1} = T_{1}$
 is upper triangular. Then $U_{2} = \left[ \begin{array}{cc}
 1 & 0 \\
 0 & W_{1}
 \end{array}\right]$
 is a unitary $n \times n$ matrix. Hence $U = U_{1}U_{2}$ is unitary (using Theorem~\ref{th:025759}), and
\begin{align*}
U^HAU &= U_{2}^H(U_{1}^HAU_{1})U_{2} \\
	   &= \left[ \begin{array}{cc}
	   1 & 0 \\
	   0 & W_{1}^H
	   \end{array}\right] \left[ \begin{array}{cc}
   	   \lambda_{1} & X_{1} \\
	   0 & A_{1}
	   \end{array}\right] \left[ \begin{array}{cc}
	   1 & 0 \\
	   0 & W_{1}
	   \end{array}\right] = \left[ \begin{array}{cc}
	   \lambda_{1} & X_{1}W_{1} \\
	   0 & T_{1}
	   \end{array}\right]
\end{align*}
is upper triangular. Finally, $A$ and $U^{H}AU = T$ have the same eigenvalues by (the complex version of) Theorem~\ref{th:016008}, and they are the diagonal entries of $T$ because $T$ is upper triangular.
\end{proof}

The fact that similar matrices have the same traces and determinants gives the following consequence of Schur's theorem.

\begin{corollary}\label{cor:025850}
Let $A$ be an $n \times n$ complex matrix, and let $\lambda_{1}, \lambda_{2}, \ldots, \lambda_{n}$ denote the eigenvalues of $A$, including multiplicities. Then
\begin{equation*}
\mbox{det }A = \lambda_1\lambda_2 \cdots \lambda_n \quad \mbox{and} \quad \mbox{tr }A = \lambda_1 + \lambda_2 + \cdots + \lambda_n
\end{equation*}
\end{corollary}

Schur's theorem asserts that every
complex matrix can be ``unitarily triangularized.'' However, we cannot
substitute ``unitarily diagonalized'' here. In fact, if $A = \left[ \begin{array}{cc}
1 & 1 \\
0 & 1
\end{array}\right]$, there is no invertible complex matrix $U$ at all such that $U^{-1}AU$ is diagonal. However, the situation is much better for Hermitian matrices.


\begin{theorem}\label{th:Spectral Theorem}
If $A$ is Hermitian, there is a unitary matrix $U$ such that $U^{H}AU$ is diagonal.
\end{theorem}

\begin{proof}
By Schur's theorem, let $U^{H}AU = T$ be upper triangular where $U$ is unitary. Since $A$ is Hermitian, this gives
\begin{equation*}
T^H = (U^HAU)^H = U^HA^HU^{HH} = U^HAU = T
\end{equation*}
This means that $T$ is both upper and lower triangular. Hence $T$ is actually diagonal.
\end{proof}

The Real Spectral Theorem asserts that a real matrix $A$ is symmetric if and only if it is orthogonally diagonalizable (that is, $P^{T}AP$ is diagonal for some real orthogonal matrix $P$). Theorem~\ref{th:Spectral Theorem}
 is the complex analog of half of this result. However, the converse is
false for complex matrices: There exist unitarily diagonalizable
matrices that are not Hermitian.

\begin{example}\label{ex:025874}
Show that the non-Hermitian matrix $A = \left[ \begin{array}{rr}
0 & 1 \\
-1 & 0
\end{array}\right]$ is unitarily diagonalizable.


\begin{explanation}
  The characteristic polynomial is $c_{A}(x) = x^{2} + 1$. Hence the eigenvalues are $i$ and $-i$, and it is easy to verify that $\left[ \begin{array}{r}
  i \\
  -1
  \end{array}\right]$ and $\left[ \begin{array}{r}
  -1 \\
  i
  \end{array}\right]$ are corresponding eigenvectors. Moreover, these eigenvectors are orthogonal and both have length $\sqrt{2}$, so $U = \frac{1}{\sqrt{2}}\left[ \begin{array}{rr}
  i & -1 \\
  -1 & i
  \end{array}\right]$ is a unitary matrix such that $U^HAU = \left[ \begin{array}{rr}
  i & 0 \\
  0 & -i
  \end{array}\right]$ is diagonal.
\end{explanation}
\end{example}

There is a very simple way to characterize those complex matrices that are unitarily diagonalizable. To this end, an $n \times n$ complex matrix $N$ is called \dfn{normal} if $NN^{H} = N^{H}N$. It is clear that every Hermitian or unitary matrix is normal, as is the matrix $\left[ \begin{array}{rr}
	0 & 1 \\
	-1 & 0
\end{array}\right]$ in Example~\ref{exa:025874}. In fact we have the following result.

\begin{theorem}{}{025890}
An $n \times n$ complex matrix $A$ is unitarily diagonalizable if and only if $A$ is normal.
\end{theorem}

\begin{proof}
Assume first that $U^{H}AU = D$, where $U$ is unitary and $D$ is diagonal. Then $DD^{H} = D^{H}D$ as is easily verified. Because $DD^{H} = U^{H}(AA^{H})U$ and $D^{H}D = U^{H}(A^{H}A)U$, it follows by cancellation that $AA^{H} = A^{H}A$.

Conversely, assume $A$ is normal---that is, $AA^{H} = A^{H}A$. By Schur's theorem, let $U^{H}AU = T$, where $T$ is upper triangular and $U$ is unitary. Then $T$ is normal too:
\begin{equation*}
TT^H = U^H(AA^H)U = U^H(A^HA)U = T^HT
\end{equation*}
Hence it suffices to show that a normal $n \times n$ upper triangular matrix $T$ must be diagonal. We induct on $n$; it is clear if $n = 1$. If $n > 1$ and $T = \left[ t_{ij} \right]$, then equating $(1, 1)$-entries in $TT^{H}$ and $T^{H}T$ gives
\begin{equation*}
|t_{11}|^2 + |t_{12}|^2 + \ldots + |t_{1n}|^2 = |t_{11}|^2
\end{equation*}
This implies $t_{12} = t_{13} = \ldots = t_{1n} = 0$, so $T = \left[ \begin{array}{cc}
t_{11} & 0 \\
0 & T_{1}
\end{array}\right]$ in block form. Hence $T = \left[ \begin{array}{cc}
\overline{t}_{11} & 0 \\
0 & T_{1}^H
\end{array}\right]$ so $TT^{H} = T^{H}T$ implies $T_{1}T_{1}^H = T_{1}T_{1}^H$. Thus $T_{1}$ is diagonal by induction, and the proof is complete.
\end{proof}

We conclude this section by using Schur's theorem (Theorem~\ref{th:025814}) to prove a famous theorem about matrices. Recall that the characteristic polynomial of a square matrix $A$ is defined by $c_{A}(x) = \mbox{det}(xI - A)$, and that the eigenvalues of $A$ are just the roots of $c_{A}(x)$.

\begin{theorem}[Cayley-Hamilton Theorem]\label{th:Cayley_Hamilton}
If $A$ is an $n \times n$ complex matrix, then $c_{A}(A) = 0$; that is, $A$ is a root of its characteristic polynomial.
\end{theorem}
\begin{remark}
Named after the English mathematician Arthur Cayley (1821--1895) and William Rowan Hamilton (1805--1865), an Irish mathematician famous for his work on physical dynamics.
\end{remark}

\begin{proof}
If $p(x)$ is any polynomial with complex coefficients, then $p(P^{-1}AP) = P^{-1}p(A)P$ for any invertible complex matrix $P$. Hence, by Schur's theorem, we may assume that $A$ is upper triangular. Then the eigenvalues $\lambda_{1}, \lambda_{2}, \ldots, \lambda_{n}$ of $A$ appear along the main diagonal, so
\begin{equation*}
c_{A}(x) = (x - \lambda_{1})(x - \lambda_{2})(x - \lambda_{3}) \cdots (x -\lambda_{n})
\end{equation*}
Thus
\begin{equation*}
c_{A}(A) = (A - \lambda_{1}I)(A - \lambda_{2}I)(A - \lambda_{3}I) \cdots (A - \lambda_{n}I)
\end{equation*}
Note that each matrix $A - \lambda_{i}I$ is upper triangular. Now observe:
\begin{enumerate}
\item $A - \lambda_{1}I$ has zero first column because column 1 of $A$ is $(\lambda_{1}, 0, 0, \ldots, 0)^{T}$.
\item Then $(A - \lambda_{1}I)(A - \lambda_{2}I)$ has the first two columns zero because the second column of $(A - \lambda_{2}I)$ is $(b, 0, 0, \ldots, 0)^{T}$ for some constant $b$.
\item Next $(A - \lambda_{1}I)(A - \lambda_{2}I)(A - \lambda_{3}I)$ has the first three columns zero because column 3 of $(A -\lambda_{3}I)$ is $(c, d, 0, \ldots, 0)^{T}$ for some constants $c$ and $d$.
\end{enumerate}
Continuing in this way we see that $(A - \lambda_{1}I)(A - \lambda_{2}I)(A - \lambda_{3}I) \cdots (A - \lambda_{n}I)$ has all $n$ columns zero; that is, $c_{A}(A) = 0$.
\end{proof}

\section*{Practice Problems}


\begin{problem}\label{prb:complex_matrices1}
In each case, compute the norm of the complex vector.


\begin{enumerate}
\item $(1, 1 - i, -2, i)$
$$\answer{\sqrt{6}}$$
\item $(1 - i, 1 + i, 1, -1)$

\item $(2 + i, 1 - i, 2, 0, -i)$
$$\answer{\sqrt{13}}$$
\item $(-2, -i, 1 + i, 1 - i, 2i)$

\end{enumerate}
\end{problem}

\begin{problem}\label{prb:complex_matrices2}
In each case, determine whether the two vectors are orthogonal.

\begin{enumerate}
\item $\begin{bmatrix}
4\\ -3i\\ 2 + i\end{bmatrix}$, $\begin{bmatrix} i\\ 2\\ 2 - 4i\end{bmatrix}$
\wordChoice{\choice{Yes}\choice[correct]{No}}
\item $\begin{bmatrix} i\\ -i\\ 2 + i \end{bmatrix}$, $\begin{bmatrix} i\\ i\\ 2 -i \end{bmatrix}$
\wordChoice{\choice{Yes}\choice[correct]{No}}
\item $\begin{bmatrix} 1\\ 1\\ i\\ i\end{bmatrix}$, $\begin{bmatrix} 1\\ i\\ -i\\ 1\end{bmatrix}$
\wordChoice{\choice[correct]{Yes}\choice{No}}
\item $\begin{bmatrix} 4 + 4i\\ 2 + i\\ 2i\end{bmatrix}$, $\begin{bmatrix} -1 + i\\ 2\\ 3 - 2i\end{bmatrix}$
\wordChoice{\choice[correct]{Yes}\choice{No}}
\end{enumerate}

\end{problem}

\begin{problem}\label{prb:complex_matrices3}
A subset $U$ of $\mathbb{C}^n$ is called a \textbf{complex subspace} of $\mathbb{C}^n$ if it contains $0$ and if, given $\vec{v}$ and $\vec{w}$ in $U$, both $\vec{v} + \vec{w}$ and $z\vec{v}$ lie in $U$ ($z$ any complex number). In each case, determine whether $U$ is a complex subspace of $\mathbb{C}^3$.


\begin{enumerate}
\item $U = \{(w, \overline{w}, 0) \mid w \mbox{ in } \mathbb{C}\}$
\begin{hint}
Not a subspace. For example, $i(0, 0, 1) = (0, 0, i)$ is not in $U$.
\end{hint}

\item $U = \{(w, 2w, a) \mid w \mbox{ in } \mathbb{C}, a \mbox{ in } \RR\}$

\item $U = \RR^3$
\wordChoice{\choice[correct]{Subspace}\choice{Not a Subspace}}

\item $U = \{(v + w, v - 2w, v) \mid v, w \mbox{ in } \mathbb{C}\}$

\end{enumerate}
\end{problem}

\begin{problem}\label{prb:complex_matrices4}
In each case, find a basis over $\mathbb{C}$, and determine the dimension of the complex subspace $U$ of $\mathbb{C}^3$ (see the previous exercise).

\begin{enumerate}
\item $U = \{(w, v + w, v - iw) \mid v, w \mbox{ in } \mathbb{C}\}$
\begin{hint}
Basis $\{(i, 0, 2), (1, 0, -1)\}$; dimension $2$
\end{hint}

\item $U = \{(iv + w, 0, 2v - w) \mid v, w \mbox{ in } \mathbb{C}\}$

\item $U = \{(u, v, w) \mid  iu - 3v + (1 - i)w = 0;\  \\ u, v, w \mbox{ in } \mathbb{C}\}$
\begin{hint}
Basis $\{(1, 0, -2i), (0, 1, 1 - i)\}$; dimension $2$
\end{hint}

\item $U = \{(u, v, w) \mid 2u + (1 + i)v - iw = 0;\ \\ u, v, w \mbox{ in } \mathbb{C}\}$
\end{enumerate}
\end{problem}

\begin{problem}\label{prb:complex_matrices5}
In each case, determine whether the given matrix is Hermitian, unitary, or normal.

\begin{enumerate}
\item $\left[ \begin{array}{rr}
1 & -i \\
i & i
\end{array}\right]$
\begin{hint}
Normal only
\end{hint}

\item $\left[ \begin{array}{rr}
2 & 3 \\
-3 & 2
\end{array}\right]$

\item $\left[ \begin{array}{rr}
1 & i \\
-i & 2
\end{array}\right]$
\begin{hint}
Hermitian (and normal), not unitary
\end{hint}

\item $\left[ \begin{array}{rr}
1 & -i \\
i & -1
\end{array}\right]$

\item $\frac{1}{\sqrt{2}} \left[ \begin{array}{rr}
1 & -1 \\
1 & 1
\end{array}\right]$

\item $\left[ \begin{array}{cc}
1 & 1 + i \\
1 + i & i
\end{array}\right]$
\begin{hint}
None of these adjectives apply.
\end{hint}

\item $\left[ \begin{array}{cc}
1 + i & 1 \\
-i & -1 + i
\end{array}\right]$

\item $\frac{1}{\sqrt{2}|z|}\left[ \begin{array}{rr}
z & z \\
\overline{z} & -\overline{z}
\end{array}\right]$, $z \neq 0$
\begin{hint}
Unitary (and normal); Hermitian if and only if $z$ is real
\end{hint}

\end{enumerate}
\end{problem}

\begin{problem}\label{prb:complex_matrices6}
Show that a matrix $N$ is normal if and only if $\overline{N}N^T = N^T\overline{N}$.
\end{problem}

\begin{problem}\label{prb:complex_matrices7}
Let $A = \left[ \begin{array}{cc}
z & \overline{v} \\
v & w
\end{array}\right]$
 where $v$, $w$, and $z$ are complex numbers. Characterize in terms of $v$, $w$, and $z$ when $A$ is

\begin{enumerate}
\item Hermitian
\item unitary
\item normal.
\end{enumerate}
\end{problem}

\begin{problem}\label{prb:complex_matrices8}
In each case, find a unitary matrix $U$ such that $U^{H}AU$ is diagonal.

\begin{enumerate}
\item $A = \left[ \begin{array}{rr}
1 & i\\
-i & 1
\end{array}\right]$


\item $A = \left[ \begin{array}{cc}
4 & 3 - i \\
3 + i & 1
\end{array}\right]$
\begin{hint}
$U = \frac{1}{\sqrt{14}}\left[ \begin{array}{cc}
-2 & 3 - i \\
3 + i & 2
\end{array}\right]$, $U^HAU = \left[ \begin{array}{rr}
-1 & 0 \\
0 & 6
\end{array}\right]$
\end{hint}

\item $A = \left[ \begin{array}{rr}
a & b\\
-b & a
\end{array}\right]$;  $a$, $b$, real

\item $A = \left[ \begin{array}{cc}
2 & 1 + i\\
1 - i & 3
\end{array}\right]$
\begin{hint}
$U = \frac{1}{\sqrt{3}}\left[ \begin{array}{cc}
1 + i & 1 \\
-1 & 1 - i
\end{array}\right]$, $U^HAU = \left[ \begin{array}{rr}
1 & 0 \\
0 & 4
\end{array}\right]$
\end{hint}

\item $A = \left[ \begin{array}{ccc}
1 & 0 &  1 + i\\
0 & 2 & 0 \\
1 - i & 0 & 0
\end{array}\right]$

\item $A = \left[ \begin{array}{ccc}
1 & 0 & 0\\
0 & 1 & 1 + i\\
0 & 1 - i & 2
\end{array}\right]$
\begin{hint}
$U = \frac{1}{\sqrt{3}}\left[ \begin{array}{ccc}
\sqrt{3} & 0 & 0 \\
0 & 1 + i & 1 \\
0 & -1 & 1 - i
\end{array}\right]$, $U^HAU = \left[ \begin{array}{rrr}
1 & 0 & 0 \\
0 & 0 & 0 \\
0 & 0 & 3
\end{array}\right]$
\end{hint}

\end{enumerate}
\end{problem}

\begin{problem}\label{prb:complex_matrices9}
Show that $\langle A \vec{x}, \vec{y} \rangle = \langle \vec{x}, A^{H}\vec{y}\rangle$ holds for all $n \times n$ matrices $A$ and for all $n$-tuples $\vec{x}$ and $\vec{y}$ in $\mathbb{C}^n$.
\end{problem}

\begin{problem}\label{prb:complex_matrices10}
\begin{enumerate}
\item Prove \ref{th:025575a} and \ref{th:025575b} of Theorem~\ref{th:025575}.

\item Prove Theorem~\ref{th:025616}.
\begin{hint}
$\norm{ \lambda Z }^2 = \langle \lambda Z, \lambda Z \rangle = \lambda\overline{\lambda} \langle Z, Z \rangle = |\lambda|^2 \norm{ Z }^2$
\end{hint}

\item Prove Theorem~\ref{th:025659}.

\end{enumerate}
\end{problem}

\begin{problem}\label{prb:complex_matrices11}
\begin{enumerate}[label={\alph*.}]
\item Show that $A$ is Hermitian if and only if $\overline{A} = A^T$.

\item Show that the diagonal entries of any Hermitian matrix are real.

\begin{hint}
If the $(k, k)$-entry of $A$ is $a_{kk}$, then the $(k, k)$-entry of $\overline{A}$ is $\overline{a}_{kk}$ so the $(k, k)$-entry of $(\overline{A})^T = A^{H}$ is $\overline{a}_{kk}$. This equals $a$, so $a_{kk}$ is real.
\end{hint}
\end{enumerate}
\end{problem}

\begin{problem}\label{prb:complex_matrices12}
\begin{enumerate}
\item Show that every complex matrix $Z$ can be written uniquely in the form $Z = A + iB$, where $A$ and $B$ are real matrices.

\item If $Z = A + iB$ as in (a), show that $Z$ is Hermitian if and only if $A$ is symmetric, and $B$ is skew-symmetric (that is, $B^{T} = -B$).

\end{enumerate}
\end{problem}

\begin{problem}\label{prb:complex_matrices13}
If $Z$ is any complex $n \times n$ matrix, show that $ZZ^{H}$ and $Z + Z^{H}$ are Hermitian.
\end{problem}

\begin{problem}\label{prb:complex_matrices14}
A complex matrix $B$ is called \textbf{skew-Hermitian}\index{skew-Hermitian}\index{complex matrix!skew-Hermitian} if $B^{H} = -B$.

\begin{enumerate}[label={\alph*.}]
\item Show that $Z - Z^{H}$ is skew-Hermitian for any square complex matrix $Z$.

\item If $B$ is skew-Hermitian, show that $B^{2}$ and $iB$ are Hermitian.
\begin{hint}
Show that $(B^2)^H = B^HB^H = (-B)(-B) = B^2$; $(iB)^H = \overline{i}B^H = (-i)(-B) = iB$.
\end{hint}

\item If $B$ is skew-Hermitian, show that the eigenvalues of $B$ are pure imaginary ($i \lambda$ for real $\lambda$).

\item Show that every $n \times n$ complex matrix $Z$ can be written uniquely as $Z = A + B$, where $A$ is Hermitian and $B$ is skew-Hermitian.
\begin{hint}
If $Z = A + B$, as given, first show that $Z^{H} = A - B$, and hence that $A = \frac{1}{2}(Z + Z^{H})$ and $B = \frac{1}{2}(Z - Z^{H})$.
\end{hint}
\end{enumerate}
\end{problem}

\begin{problem}\label{prb:complex_matrices15}
Let $U$ be a unitary matrix. Show that:

\begin{enumerate}
\item $\norm{ U\vec{x} } = \norm{ \vec{x} }$ for all columns $\vec{x}$ in $\mathbb{C}^n$.

\item $|\lambda| = 1$ for every eigenvalue $\lambda$ of $U$.

\end{enumerate}
\end{problem}


\begin{problem}\label{prb:complex_matrices16}
\begin{enumerate}
\item If $Z$ is an invertible complex matrix, show that $Z^{H}$ is invertible and that $(Z^{H})^{-1} = (Z^{-1})^{H}$.

\item Show that the inverse of a unitary matrix is again unitary.
\begin{hint}
If $U$ is unitary, $(U^{-1})^{-1} = (U^{H})^{-1} = (U^{-1})^{H}$, so $U^{-1}$ is unitary.
\end{hint}

\item If $U$ is unitary, show that $U^{H}$ is unitary.

\end{enumerate}
\end{problem}

\begin{problem}\label{prb:complex_matrices17}
Let $Z$ be an $m \times n$ matrix such that $Z^{H}Z = I_{n}$ (for example, $Z$ is a unit column in $\mathbb{C}^n$).

\begin{enumerate}
\item Show that $V = ZZ^{H}$ is Hermitian and satisfies \\ $V^{2} = V$.

\item Show that $U = I - 2ZZ^{H}$ is both unitary and Hermitian (so $U^{-1} = U^{H} = U$).

\end{enumerate}
\end{problem}

\begin{problem}\label{prb:complex_matrices18}
\begin{enumerate}
\item If $N$ is normal, show that $zN$ is also normal for all complex numbers $z$.

\item Show that (a) fails if \textit{normal} is replaced by \textit{Hermitian}.
\begin{hint}
$H = \left[ \begin{array}{rr}
1 & i \\
-i & 0
\end{array}\right]$ is Hermitian but $iH = \left[ \begin{array}{rr}
i & -1 \\
1 & 0
\end{array}\right]$ is not.
\end{hint}
\end{enumerate}
\end{problem}

\begin{problem}\label{prb:complex_matrices19}
Show that a real $2 \times 2$ normal matrix is either symmetric or has the form $\left[ \begin{array}{rr}
a & b \\
-b & a
\end{array}\right]$.
\end{problem}

\begin{problem}\label{prb:complex_matrices20}
If $A$ is Hermitian, show that all the coefficients of $c_{A}(x)$ are real numbers.
\end{problem}

\begin{problem}\label{prb:complex_matrices21}
\begin{enumerate}
\item If $A = \left[ \begin{array}{rr}
1 & 1 \\
0 & 1
\end{array}\right]$, show that $U^{-1}AU$ is not diagonal for any invertible complex matrix $U$.

\item If $A = \left[ \begin{array}{rr}
0 & 1 \\
-1 & 0
\end{array}\right]$, show that $U^{-1}AU$ is not upper triangular for any \textit{real} invertible matrix $U$.
\begin{hint}
Let $U = \left[ \begin{array}{rr}
a & b \\
c & d
\end{array}\right]$ be real and invertible, and assume that $U^{-1}AU = \left[ \begin{array}{rr}
\lambda & \mu \\
0 & v
\end{array}\right]$.
 Then $AU = U\left[ \begin{array}{rr}
 \lambda & \mu \\
 0 & v
 \end{array}\right]$, and first column entries are $c = a\lambda$ and $-a = c\lambda$. Hence $\lambda$ is real ($c$ and $a$ are both real and are not both $0$), and $(1 + \lambda^{2})a = 0$. Thus $a = 0$, $c = a\lambda = 0$, a contradiction.
\end{hint}
\end{enumerate}
\end{problem}

\begin{problem}\label{prb:complex_matrices22}
If $A$ is any $n \times n$ matrix, show that $U^{H}AU$ is lower triangular for some unitary matrix $U$.
\end{problem}

\begin{problem}\label{prb:complex_matrices23}
If $A$ is a $3 \times 3$ matrix, show that $A^{2} = 0$ if and only if there exists a unitary matrix $U$ such that $U^{H}AU$ has the form $\left[ \begin{array}{rrr}
0 & 0 & u \\
0 & 0 & v \\
0 & 0 & 0
\end{array}\right]$
or the form $\left[ \begin{array}{rrr}
0 & u & v \\
0 & 0 & 0 \\
0 & 0 & 0
\end{array}\right]$.
\end{problem}

\begin{problem}\label{prb:complex_matrices24}
If $A^{2} = A$, show that rank $A = \mbox{tr}A$. [\textit{Hint}: Use Schur's theorem.]
\end{problem}

\section*{Text Source} This section was adapted from Section 8.6 of Keith Nicholson's \href{https://open.umn.edu/opentextbooks/textbooks/linear-algebra-with-applications}{\it Linear Algebra with Applications}. (CC-BY-NC-SA)

W. Keith Nicholson, {\it Linear Algebra with Applications}, Lyryx 2018, Open Edition, pp. 445--456.
\end{document}