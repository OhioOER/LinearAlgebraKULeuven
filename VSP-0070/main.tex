\documentclass{ximera}
%%% Begin Laad packages

\makeatletter
\@ifclassloaded{xourse}{%
    \typeout{Start loading preamble.tex (in a XOURSE)}%
    \def\isXourse{true}   % automatically defined; pre 112022 it had to be set 'manually' in a xourse
}{%
    \typeout{Start loading preamble.tex (NOT in a XOURSE)}%
}
\makeatother

\def\isEn\true 

\pgfplotsset{compat=1.16}

\usepackage{currfile}

% 201908/202301: PAS OP: babel en doclicense lijken problemen te veroorzaken in .jax bestand
% (wegens syntax error met toegevoegde \newcommands ...)
\pdfOnly{
    \usepackage[type={CC},modifier={by-nc-sa},version={4.0}]{doclicense}
    %\usepackage[hyperxmp=false,type={CC},modifier={by-nc-sa},version={4.0}]{doclicense}
    %%% \usepackage[dutch]{babel}
}



\usepackage[utf8]{inputenc}
\usepackage{morewrites}   % nav zomercursus (answer...?)
\usepackage{multirow}
\usepackage{multicol}
\usepackage{tikzsymbols}
\usepackage{ifthen}
%\usepackage{animate} BREAKS HTML STRUCTURE USED BY XIMERA
\usepackage{relsize}

\usepackage{eurosym}    % \euro  (€ werkt niet in xake ...?)
\usepackage{fontawesome} % smileys etc

% Nuttig als ook interactieve beamer slides worden voorzien:
\providecommand{\p}{} % default nothing ; potentially usefull for slides: redefine as \pause
%providecommand{\p}{\pause}

    % Layout-parameters voor het onderschrift bij figuren
\usepackage[margin=10pt,font=small,labelfont=bf, labelsep=endash,format=hang]{caption}
%\usepackage{caption} % captionof
%\usepackage{pdflscape}    % landscape environment

% Met "\newcommand\showtodonotes{}" kan je todonotes tonen (in pdf/online)
% 201908: online werkt het niet (goed)
\providecommand\showtodonotes{disable}
\providecommand\todo[1]{\typeout{TODO #1}}
%\usepackage[\showtodonotes]{todonotes}
%\usepackage{todonotes}

%
% Poging tot aanpassen layout
%
\usepackage{tcolorbox}
\tcbuselibrary{theorems}

%%% Einde laad packages

%%% Begin Ximera specifieke zaken

\graphicspath{
	{../../}
	{../}
	{./}
  	{../../pictures/}
   	{../pictures/}
   	{./pictures/}
	{./explog/}    % M05 in groeimodellen       
}

%%% Einde Ximera specifieke zaken

%
% define softer blue/red/green, use KU Leuven base colors for blue (and dark orange for red ?)
%
% todo: rather redefine blue/red/green ...?
%\definecolor{xmblue}{rgb}{0.01, 0.31, 0.59}
%\definecolor{xmred}{rgb}{0.89, 0.02, 0.17}
\definecolor{xmdarkblue}{rgb}{0.122, 0.671, 0.835}   % KU Leuven Blauw
\definecolor{xmblue}{rgb}{0.114, 0.553, 0.69}        % KU Leuven Blauw
\definecolor{xmgreen}{rgb}{0.13, 0.55, 0.13}         % No KULeuven variant for green found ...

\definecolor{xmaccent}{rgb}{0.867, 0.541, 0.18}      % KU Leuven Accent (orange ...)
\definecolor{kuaccent}{rgb}{0.867, 0.541, 0.18}      % KU Leuven Accent (orange ...)

\colorlet{xmred}{xmaccent!50!black}                  % Darker version of KU Leuven Accent

\providecommand{\blue}[1]{{\color{blue}#1}}    
\providecommand{\red}[1]{{\color{red}#1}}

\renewcommand\CancelColor{\color{xmaccent!50!black}}

% werkt in math en text mode om MATH met oranje (of grijze...)  achtergond te tonen (ook \important{\text{blabla}} lijkt te werken)
%\newcommand{\important}[1]{\ensuremath{\colorbox{xmaccent!50!white}{$#1$}}}   % werkt niet in Mathjax
%\newcommand{\important}[1]{\ensuremath{\colorbox{lightgray}{$#1$}}}
\newcommand{\important}[1]{\ensuremath{\colorbox{orange}{$#1$}}}   % TODO: kleur aanpassen voor mathjax; wordt overschreven infra!


% Uitzonderlijk kan met \pdfnl in de PDF een newline worden geforceerd, die online niet nodig/nuttig is omdat daar de regellengte hoe dan ook niet gekend is.
\ifdefined\HCode%
\providecommand{\pdfnl}{}%
\else%
\providecommand{\pdfnl}{%
  \\%
}%
\fi

% Uitzonderlijk kan met \handoutnl in de handout-PDF een newline worden geforceerd, die noch online noch in de PDF-met-antwoorden nuttig is.
\ifdefined\HCode
\providecommand{\handoutnl}{}
\else
\providecommand{\handoutnl}{%
\ifhandout%
  \nl%
\fi%
}
\fi



% \cellcolor IGNORED by tex4ht ?
% \begin{center} seems not to wordk
    % (missing margin-left: auto;   on tabular-inside-center ???)
%\newcommand{\importantcell}[1]{\ensuremath{\cellcolor{lightgray}#1}}  %  in tabular; usablility to be checked
\providecommand{\importantcell}[1]{\ensuremath{#1}}     % no mathjax2 support for colloring array cells

\pdfOnly{
  \renewcommand{\important}[1]{\ensuremath{\colorbox{kuaccent!50!white}{$#1$}}}
  \renewcommand{\importantcell}[1]{\ensuremath{\cellcolor{kuaccent!40!white}#1}}   
}

%%% Tikz styles


\pgfplotsset{compat=1.16}

\usetikzlibrary{trees,positioning,arrows,fit,shapes,math,calc,decorations.markings,through,intersections,patterns,matrix}

\usetikzlibrary{decorations.pathreplacing,backgrounds}    % 5/2023: from experimental


\usetikzlibrary{angles,quotes}

\usepgfplotslibrary{fillbetween} % bepaalde_integraal
\usepgfplotslibrary{polar}    % oa voor poolcoordinaten.tex

\pgfplotsset{ownstyle/.style={axis lines = center, axis equal image, xlabel = $x$, ylabel = $y$, enlargelimits}} 

\pgfplotsset{
	plot/.style={no marks,samples=50}
}

\newcommand{\xmPlotsColor}{
	\pgfplotsset{
		plot1/.style={darkgray,no marks,samples=100},
		plot2/.style={lightgray,no marks,samples=100},
		plotresult/.style={blue,no marks,samples=100},
		plotblue/.style={blue,no marks,samples=100},
		plotred/.style={red,no marks,samples=100},
		plotgreen/.style={green,no marks,samples=100},
		plotpurple/.style={purple,no marks,samples=100}
	}
}
\newcommand{\xmPlotsBlackWhite}{
	\pgfplotsset{
		plot1/.style={black,loosely dashed,no marks,samples=100},
		plot2/.style={black,loosely dotted,no marks,samples=100},
		plotresult/.style={black,no marks,samples=100},
		plotblue/.style={black,no marks,samples=100},
		plotred/.style={black,dotted,no marks,samples=100},
		plotgreen/.style={black,dashed,no marks,samples=100},
		plotpurple/.style={black,dashdotted,no marks,samples=100}
	}
}


\newcommand{\xmPlotsColorAndStyle}{
	\pgfplotsset{
		plot1/.style={darkgray,no marks,samples=100},
		plot2/.style={lightgray,no marks,samples=100},
		plotresult/.style={blue,no marks,samples=100},
		plotblue/.style={xmblue,no marks,samples=100},
		plotred/.style={xmred,dashed,thick,no marks,samples=100},
		plotgreen/.style={xmgreen,dotted,very thick,no marks,samples=100},
		plotpurple/.style={purple,no marks,samples=100}
	}
}


%\iftikzexport
\xmPlotsColorAndStyle
%\else
%\xmPlotsBlackWhite
%\fi
%%%


%
% Om venndiagrammen te arceren ...
%
\makeatletter
\pgfdeclarepatternformonly[\hatchdistance,\hatchthickness]{north east hatch}% name
{\pgfqpoint{-1pt}{-1pt}}% below left
{\pgfqpoint{\hatchdistance}{\hatchdistance}}% above right
{\pgfpoint{\hatchdistance-1pt}{\hatchdistance-1pt}}%
{
	\pgfsetcolor{\tikz@pattern@color}
	\pgfsetlinewidth{\hatchthickness}
	\pgfpathmoveto{\pgfqpoint{0pt}{0pt}}
	\pgfpathlineto{\pgfqpoint{\hatchdistance}{\hatchdistance}}
	\pgfusepath{stroke}
}
\pgfdeclarepatternformonly[\hatchdistance,\hatchthickness]{north west hatch}% name
{\pgfqpoint{-\hatchthickness}{-\hatchthickness}}% below left
{\pgfqpoint{\hatchdistance+\hatchthickness}{\hatchdistance+\hatchthickness}}% above right
{\pgfpoint{\hatchdistance}{\hatchdistance}}%
{
	\pgfsetcolor{\tikz@pattern@color}
	\pgfsetlinewidth{\hatchthickness}
	\pgfpathmoveto{\pgfqpoint{\hatchdistance+\hatchthickness}{-\hatchthickness}}
	\pgfpathlineto{\pgfqpoint{-\hatchthickness}{\hatchdistance+\hatchthickness}}
	\pgfusepath{stroke}
}
%\makeatother

\tikzset{
    hatch distance/.store in=\hatchdistance,
    hatch distance=10pt,
    hatch thickness/.store in=\hatchthickness,
   	hatch thickness=2pt
}

\colorlet{circle edge}{black}
\colorlet{circle area}{blue!20}


\tikzset{
    filled/.style={fill=green!30, draw=circle edge, thick},
    arceerl/.style={pattern=north east hatch, pattern color=blue!50, draw=circle edge},
    arceerr/.style={pattern=north west hatch, pattern color=yellow!50, draw=circle edge},
    outline/.style={draw=circle edge, thick}
}




%%% Updaten commando's
\def\hoofding #1#2#3{\maketitle}     % OBSOLETE ??

% we willen (bijna) altijd \geqslant ipv \geq ...!
\newcommand{\geqnoslant}{\geq}
\renewcommand{\geq}{\geqslant}
\newcommand{\leqnoslant}{\leq}
\renewcommand{\leq}{\leqslant}

% Todo: (201908) waarom komt er (soms) underlined voor emph ...?
\renewcommand{\emph}[1]{\textit{#1}}

% API commando's

\newcommand{\ds}{\displaystyle}
\newcommand{\ts}{\textstyle}  % tegenhanger van \ds   (Ximera zet PER  DEFAULT \ds!)

% uit Zomercursus-macro's: 
\newcommand{\bron}[1]{\begin{scriptsize} \emph{#1} \end{scriptsize}}     % deprecated ...?


%definities nieuwe commando's - afkortingen veel gebruikte symbolen
\newcommand{\R}{\ensuremath{\mathbb{R}}}
\newcommand{\Rnul}{\ensuremath{\mathbb{R}_0}}
\newcommand{\Reen}{\ensuremath{\mathbb{R}\setminus\{1\}}}
\newcommand{\Rnuleen}{\ensuremath{\mathbb{R}\setminus\{0,1\}}}
\newcommand{\Rplus}{\ensuremath{\mathbb{R}^+}}
\newcommand{\Rmin}{\ensuremath{\mathbb{R}^-}}
\newcommand{\Rnulplus}{\ensuremath{\mathbb{R}_0^+}}
\newcommand{\Rnulmin}{\ensuremath{\mathbb{R}_0^-}}
\newcommand{\Rnuleenplus}{\ensuremath{\mathbb{R}^+\setminus\{0,1\}}}
\newcommand{\N}{\ensuremath{\mathbb{N}}}
\newcommand{\Nnul}{\ensuremath{\mathbb{N}_0}}
\newcommand{\Z}{\ensuremath{\mathbb{Z}}}
\newcommand{\Znul}{\ensuremath{\mathbb{Z}_0}}
\newcommand{\Zplus}{\ensuremath{\mathbb{Z}^+}}
\newcommand{\Zmin}{\ensuremath{\mathbb{Z}^-}}
\newcommand{\Znulplus}{\ensuremath{\mathbb{Z}_0^+}}
\newcommand{\Znulmin}{\ensuremath{\mathbb{Z}_0^-}}
\newcommand{\C}{\ensuremath{\mathbb{C}}}
\newcommand{\Cnul}{\ensuremath{\mathbb{C}_0}}
\newcommand{\Cplus}{\ensuremath{\mathbb{C}^+}}
\newcommand{\Cmin}{\ensuremath{\mathbb{C}^-}}
\newcommand{\Cnulplus}{\ensuremath{\mathbb{C}_0^+}}
\newcommand{\Cnulmin}{\ensuremath{\mathbb{C}_0^-}}
\newcommand{\Q}{\ensuremath{\mathbb{Q}}}
\newcommand{\Qnul}{\ensuremath{\mathbb{Q}_0}}
\newcommand{\Qplus}{\ensuremath{\mathbb{Q}^+}}
\newcommand{\Qmin}{\ensuremath{\mathbb{Q}^-}}
\newcommand{\Qnulplus}{\ensuremath{\mathbb{Q}_0^+}}
\newcommand{\Qnulmin}{\ensuremath{\mathbb{Q}_0^-}}

\newcommand{\perdef}{\overset{\mathrm{def}}{=}}
\newcommand{\pernot}{\overset{\mathrm{notatie}}{=}}
\newcommand\perinderdaad{\overset{!}{=}}     % voorlopig gebruikt in limietenrekenregels
\newcommand\perhaps{\overset{?}{=}}          % voorlopig gebruikt in limietenrekenregels

\newcommand{\degree}{^\circ}


\DeclareMathOperator{\dom}{dom}     % domein
\DeclareMathOperator{\codom}{codom} % codomein
\DeclareMathOperator{\bld}{bld}     % beeld
\DeclareMathOperator{\graf}{graf}   % grafiek
\DeclareMathOperator{\rico}{rico}   % richtingcoëfficient
\DeclareMathOperator{\co}{co}       % coordinaat
\DeclareMathOperator{\gr}{gr}       % graad

\newcommand{\func}[5]{\ensuremath{#1: #2 \rightarrow #3: #4 \mapsto #5}} % Easy to write a function


% Operators
\DeclareMathOperator{\bgsin}{bgsin}
\DeclareMathOperator{\bgcos}{bgcos}
\DeclareMathOperator{\bgtan}{bgtan}
\DeclareMathOperator{\bgcot}{bgcot}
\DeclareMathOperator{\bgsinh}{bgsinh}
\DeclareMathOperator{\bgcosh}{bgcosh}
\DeclareMathOperator{\bgtanh}{bgtanh}
\DeclareMathOperator{\bgcoth}{bgcoth}

% Oude \Bgsin etc deprecated: gebruik \bgsin, en herdefinieer dat als je Bgsin wil!
%\DeclareMathOperator{\cosec}{cosec}    % not used? gebruik \csc en herdefinieer

% operatoren voor differentialen: to be verified; 1/2020: inconsequent gebruik bij afgeleiden/integralen
\renewcommand{\d}{\mathrm{d}}
\newcommand{\dx}{\d x}
\newcommand{\dd}[1]{\frac{\mathrm{d}}{\mathrm{d}#1}}
\newcommand{\ddx}{\dd{x}}

% om in voorbeelden/oefeningen de notatie voor afgeleiden te kunnen kiezen
% Usage: \afg{(2\sin(x))}  (en wordt d/dx, of accent, of D )
%\newcommand{\afg}[1]{{#1}'}
\newcommand{\afg}[1]{\left(#1\right)'}
%\renewcommand{\afg}[1]{\frac{\mathrm{d}#1}{\mathrm{d}x}}   % include in relevant exercises ...
%\renewcommand{\afg}[1]{D{#1}}

%
% \xmxxx commands: Extra KU Leuven functionaliteit van, boven of naast Ximera
%   ( Conventie 8/2019: xm+nederlandse omschrijving, maar is niet consequent gevolgd, en misschien ook niet erg handig !)
%
% (Met een minimale ximera.cls en preamble.tex zou een bruikbare .pdf moeten kunnen worden gemaakt van eender welke ximera)
%
% Usage: \xmtitle[Mijn korte abstract]{Mijn titel}{Mijn abstract}
% Eerste command na \begin{document}:
%  -> definieert de \title
%  -> definieert de abstract
%  -> doet \maketitle ( dus: print de hoofding als 'chapter' of 'sectie')
% Optionele parameter geeft eenn kort abstract (die met de globale setting \xmshortabstract{} al dan niet kan worden geprint.
% De optionele korte abstract kan worden gebruikt voor pseudo-grappige abtsarts, dus dus globaal al dan niet kunnen worden gebuikt...
% Globale settings:
%  de (optionele) 'korte abstract' wordt enkele getoond als \xmshortabstract is gezet
\providecommand\xmshortabstract{} % default: print (only!) short abstract if present
\newcommand{\xmtitle}[3][]{
	\title{#2}
	\begin{abstract}
		\ifdefined\xmshortabstract
		\ifstrempty{#1}{%
			#3
		}{%
			#1
		}%
		\else
		#3
		\fi
	\end{abstract}
	\maketitle
}

% 
% Kleine grapjes: moeten zonder verder gevolg kunnen worden verwijderd
%
%\newcommand{\xmopje}[1]{{\small#1{\reversemarginpar\marginpar{\Smiley}}}}   % probleem in floats!!
\newtoggle{showxmopje}
\toggletrue{showxmopje}

\newcommand{\xmopje}[1]{%
   \iftoggle{showxmopje}{#1}{}%
}


% -> geef een abstracte-formule-met-rechts-een-concreet-voorbeeld
% VB:  \formulevb{a^2+b^2=c^2}{3^2+4^2=5^2}
%
\ifdefined\HCode
\NewEnviron{xmdiv}[1]{\HCode{\Hnewline<div class="#1">\Hnewline}\BODY{\HCode{\Hnewline</div>\Hnewline}}}
\else
\NewEnviron{xmdiv}[1]{\BODY}
\fi

\providecommand{\formulevb}[2]{
	{\centering

    \begin{xmdiv}{xmformulevb}    % zie css voor online layout !!!
	\begin{tabular}{lcl}
		\important{#1}
		&  &
		Vb: $#2$
		\end{tabular}
	\end{xmdiv}

	}
}

\ifdefined\HCode
\providecommand{\vb}[1]{%
    \HCode{\Hnewline<span class="xmvb">}#1\HCode{</span>\Hnewline}%
}
\else
\providecommand{\vb}[1]{
    \colorbox{blue!10}{#1}
}
\fi

\ifdefined\HCode
\providecommand{\xmcolorbox}[2]{
	\HCode{\Hnewline<div class="xmcolorbox">\Hnewline}#2\HCode{\Hnewline</div>\Hnewline}
}
\else
\providecommand{\xmcolorbox}[2]{
  \cellcolor{#1}#2
}
\fi


\ifdefined\HCode
\providecommand{\xmopmerking}[1]{
 \HCode{\Hnewline<div class="xmopmerking">\Hnewline}#1\HCode{\Hnewline</div>\Hnewline}
}
\else
\providecommand{\xmopmerking}[1]{
	{\footnotesize #1}
}
\fi
% \providecommand{\voorbeeld}[1]{
% 	\colorbox{blue!10}{$#1$}
% }



% Hernoem Proof naar Bewijs, nodig voor HTML versie
\renewcommand*{\proofname}{Bewijs}

% Om opgave van oefening (wordt niet geprint bij oplossingenblad)
% (to be tested test)
\NewEnviron{statement}{\BODY}

% Environment 'oplossing' en 'uitkomst'
% voor resp. volledige 'uitwerking' dan wel 'enkel eindresultaat'
% geimplementeerd via feedback, omdat er in de ximera-server adhoc feedback-code is toegevoegd
%% Niet tonen indien handout
%% Te gebruiken om volledige oplossingen/uitwerkingen van oefeningen te tonen
%% \begin{oplossing}        De optelling is commutatief \end{oplossing}  : verschijnt online enkel 'op vraag'
%% \begin{oplossing}[toon]  De optelling is commutatief \end{oplossing}  : verschijnt steeds onmiddellijk online (bv te gebruiken bij voorbeelden) 

\ifhandout%
    \NewEnviron{oplossing}[1][onzichtbaar]%
    {%
    \ifthenelse{\equal{\detokenize{#1}}{\detokenize{toon}}}
    {
    \def\PH@Command{#1}% Use PH@Command to hold the content and be a target for "\expandafter" to expand once.

    \begin{trivlist}% Begin the trivlist to use formating of the "Feedback" label.
    \item[\hskip \labelsep\small\slshape\bfseries Oplossing% Format the "Feedback" label. Don't forget the space.
    %(\texttt{\detokenize\expandafter{\PH@Command}}):% Format (and detokenize) the condition for feedback to trigger
    \hspace{2ex}]\small%\slshape% Insert some space before the actual feedback given.
    \BODY
    \end{trivlist}
    }
    {  % \begin{feedback}[solution]   \BODY     \end{feedback}  }
    }
    }    
\else
% ONLY for HTML; xmoplossing is styled with css, and is not, and need not be a LaTeX environment
% THUS: it does NOT use feedback anymore ...
%    \NewEnviron{oplossing}{\begin{expandable}{xmoplossing}{\nlen{Toon uitwerking}{Show solution}}{\BODY}\end{expandable}}
    \newenvironment{oplossing}[1][onzichtbaar]
   {%
       \begin{expandable}{xmoplossing}{}
   }
   {%
   	   \end{expandable}
   } 
%     \newenvironment{oplossing}[1][onzichtbaar]
%    {%
%        \begin{feedback}[solution]   	
%    }
%    {%
%    	   \end{feedback}
%    } 
\fi

\ifhandout%
    \NewEnviron{uitkomst}[1][onzichtbaar]%
    {%
    \ifthenelse{\equal{\detokenize{#1}}{\detokenize{toon}}}
    {
    \def\PH@Command{#1}% Use PH@Command to hold the content and be a target for "\expandafter" to expand once.

    \begin{trivlist}% Begin the trivlist to use formating of the "Feedback" label.
    \item[\hskip \labelsep\small\slshape\bfseries Uitkomst:% Format the "Feedback" label. Don't forget the space.
    %(\texttt{\detokenize\expandafter{\PH@Command}}):% Format (and detokenize) the condition for feedback to trigger
    \hspace{2ex}]\small%\slshape% Insert some space before the actual feedback given.
    \BODY
    \end{trivlist}
    }
    {  % \begin{feedback}[solution]   \BODY     \end{feedback}  }
    }
    }    
\else
\ifdefined\HCode
   \newenvironment{uitkomst}[1][onzichtbaar]
    {%
        \begin{expandable}{xmuitkomst}{}%
    }
    {%
    	\end{expandable}%
    } 
\else
  % Do NOT print 'uitkomst' in non-handout
  %  (presumably, there is also an 'oplossing' ??)
  \newenvironment{uitkomst}[1][onzichtbaar]{}{}
\fi
\fi

%
% Uitweidingen zijn extra's die niet redelijkerwijze tot de leerstof behoren
% Uitbreidingen zijn extra's die wel redelijkerwijze tot de leerstof van bv meer geavanceerde versies kunnen behoren (B-programma/Wiskundestudenten/...?)
% Nog niet voorzien: design voor verschillende versies (A/B programma, BIO, voorkennis/ ...)
% Voor 'uitweidingen' is er een environment die online per default is ingeklapt, en in pdf al dan niet kan worden geincluded  (via \xmnouitweiding) 
%
% in een xourse, per default GEEN uitweidingen, tenzij \xmuitweiding expliciet ergens is gezet ...
\ifdefined\isXourse
   \ifdefined\xmuitweiding
   \else
       \def\xmnouitweiding{true}
   \fi
\fi

\ifdefined\xmnouitweiding
\newcounter{xmuitweiding}  % anders error undefined ...  
\excludecomment{xmuitweiding}
\else
\newtheoremstyle{dotless}{}{}{}{}{}{}{ }{}
\theoremstyle{dotless}
\newtheorem*{xmuitweidingnofrills}{}   % nofrills = no accordion; gebruikt dus de dotless theoremstyle!

\newcounter{xmuitweiding}
\newenvironment{xmuitweiding}[1][ ]%
{% 
	\refstepcounter{xmuitweiding}%
    \begin{expandable}{xmuitweiding}{\nlentext{Uitweiding \arabic{xmuitweiding}: #1}{Digression \arabic{xmuitweiding}: #1}}%
	\begin{xmuitweidingnofrills}%
}
{%
    \end{xmuitweidingnofrills}%
    \end{expandable}%
}   
% \newenvironment{xmuitweiding}[1][ ]%
% {% 
% 	\refstepcounter{xmuitweiding}
% 	\begin{accordion}\begin{accordion-item}[Uitweiding \arabic{xmuitweiding}: #1]%
% 			\begin{xmuitweidingnofrills}%
% 			}
% 			{\end{xmuitweidingnofrills}\end{accordion-item}\end{accordion}}   
\fi


\newenvironment{xmexpandable}[1][]{
	\begin{accordion}\begin{accordion-item}[#1]%
		}{\end{accordion-item}\end{accordion}}


% Command that gives a selection box online, but just prints the right answer in pdf
\newcommand{\xmonlineChoice}[1]{\pdfOnly{\wordchoicegiventrue}\wordChoice{#1}\pdfOnly{\wordchoicegivenfalse}}   % deprecated, gebruik onlineChoice ...
\newcommand{\onlineChoice}[1]{\pdfOnly{\wordchoicegiventrue}\wordChoice{#1}\pdfOnly{\wordchoicegivenfalse}}

\newcommand{\TJa}{\nlentext{ Ja }{ Yes }}
\newcommand{\TNee}{\nlentext{ Nee }{ No }}
\newcommand{\TJuist}{\nlentext{ Juist }{ True }}
\newcommand{\TFout}{\nlentext{ Fout }{ False }}

\newcommand{\choiceTrue }{{\renewcommand{\choiceminimumhorizontalsize}{4em}\wordChoice{\choice[correct]{\TJuist}\choice{\TFout}}}}
\newcommand{\choiceFalse}{{\renewcommand{\choiceminimumhorizontalsize}{4em}\wordChoice{\choice{\TJuist}\choice[correct]{\TFout}}}}

\newcommand{\choiceYes}{{\renewcommand{\choiceminimumhorizontalsize}{3em}\wordChoice{\choice[correct]{\TJa}\choice{\TNee}}}}
\newcommand{\choiceNo }{{\renewcommand{\choiceminimumhorizontalsize}{3em}\wordChoice{\choice{\TJa}\choice[correct]{\TNee}}}}

% Optional nicer formatting for wordChoice in PDF

\let\inlinechoiceorig\inlinechoice

%\makeatletter
%\providecommand{\choiceminimumverticalsize}{\vphantom{$\frac{\sqrt{2}}{2}$}}   % minimum height of boxes (cfr infra)
\providecommand{\choiceminimumverticalsize}{\vphantom{$\tfrac{2}{2}$}}   % minimum height of boxes (cfr infra)
\providecommand{\choiceminimumhorizontalsize}{1em}   % minimum width of boxes (cfr infra)

\newcommand{\inlinechoicesquares}[2][]{%
		\setkeys{choice}{#1}%
		\ifthenelse{\boolean{\choice@correct}}%
		{%
            \ifhandout%
               \fbox{\choiceminimumverticalsize #2}\allowbreak\ignorespaces%
            \else%
               \fcolorbox{blue}{blue!20}{\choiceminimumverticalsize #2}\allowbreak\ignorespaces\setkeys{choice}{correct=false}\ignorespaces%
            \fi%
		}%
		{% else
			\fbox{\choiceminimumverticalsize #2}\allowbreak\ignorespaces%  HACK: wat kleiner, zodat fits on line ... 	
		}%
}

\newcommand{\inlinechoicesquareX}[2][]{%
		\setkeys{choice}{#1}%
		\ifthenelse{\boolean{\choice@correct}}%
		{%
            \ifhandout%
               \framebox[\ifdim\choiceminimumhorizontalsize<\width\width\else\choiceminimumhorizontalsize\fi]{\choiceminimumverticalsize\ #2\ }\allowbreak\ignorespaces\setkeys{choice}{correct=false}\ignorespaces%
            \else%
               \fcolorbox{blue}{blue!20}{\makebox[\ifdim\choiceminimumhorizontalsize<\width\width\else\choiceminimumhorizontalsize\fi]{\choiceminimumverticalsize #2}}\allowbreak\ignorespaces\setkeys{choice}{correct=false}\ignorespaces%
            \fi%
		}%
		{% else
        \ifhandout%
			\framebox[\ifdim\choiceminimumhorizontalsize<\width\width\else\choiceminimumhorizontalsize\fi]{\choiceminimumverticalsize\ #2\ }\allowbreak\ignorespaces%  HACK: wat kleiner, zodat fits on line ... 	
        \fi
		}%
}


\newcommand{\inlinechoicepuntjes}[2][]{%
		\setkeys{choice}{#1}%
		\ifthenelse{\boolean{\choice@correct}}%
		{%
            \ifhandout%
               \dots\ldots\ignorespaces\setkeys{choice}{correct=false}\ignorespaces
            \else%
               \fcolorbox{blue}{blue!20}{\choiceminimumverticalsize #2}\allowbreak\ignorespaces\setkeys{choice}{correct=false}\ignorespaces%
            \fi%
		}%
		{% else
			%\fbox{\choiceminimumverticalsize #2}\allowbreak\ignorespaces%  HACK: wat kleiner, zodat fits on line ... 	
		}%
}

% print niets, maar definieer globale variable \myanswer
%  (gebruikt om oplossingsbladen te printen) 
\newcommand{\inlinechoicedefanswer}[2][]{%
		\setkeys{choice}{#1}%
		\ifthenelse{\boolean{\choice@correct}}%
		{%
               \gdef\myanswer{#2}\setkeys{choice}{correct=false}

		}%
		{% else
			%\fbox{\choiceminimumverticalsize #2}\allowbreak\ignorespaces%  HACK: wat kleiner, zodat fits on line ... 	
		}%
}



%\makeatother

\newcommand{\setchoicedefanswer}{
\ifdefined\HCode
\else
%    \renewenvironment{multipleChoice@}[1][]{}{} % remove trailing ')'
    \let\inlinechoice\inlinechoicedefanswer
\fi
}

\newcommand{\setchoicepuntjes}{
\ifdefined\HCode
\else
    \renewenvironment{multipleChoice@}[1][]{}{} % remove trailing ')'
    \let\inlinechoice\inlinechoicepuntjes
\fi
}
\newcommand{\setchoicesquares}{
\ifdefined\HCode
\else
    \renewenvironment{multipleChoice@}[1][]{}{} % remove trailing ')'
    \let\inlinechoice\inlinechoicesquares
\fi
}
%
\newcommand{\setchoicesquareX}{
\ifdefined\HCode
\else
    \renewenvironment{multipleChoice@}[1][]{}{} % remove trailing ')'
    \let\inlinechoice\inlinechoicesquareX
\fi
}
%
\newcommand{\setchoicelist}{
\ifdefined\HCode
\else
    \renewenvironment{multipleChoice@}[1][]{}{)}% re-add trailing ')'
    \let\inlinechoice\inlinechoiceorig
\fi
}

\setchoicesquareX  % by default list-of-squares with onlineChoice in PDF

% Omdat multicols niet werkt in html: enkel in pdf  (in html zijn langere pagina's misschien ook minder storend)
\newenvironment{xmmulticols}[1][2]{
 \pdfOnly{\begin{multicols}{#1}}%
}{ \pdfOnly{\end{multicols}}}

%
% Te gebruiken in plaats van \section\subsection
%  (in een printstyle kan dan het level worden aangepast
%    naargelang \chapter vs \section style )
% 3/2021: DO NOT USE \xmsubsection !
\newcommand\xmsection\subsection
\newcommand\xmsubsection\subsubsection

% Aanpassen printversie
%  (hier gedefinieerd, zodat ze in xourse kunnen worden gezet/overschreven)
\providebool{parttoc}
\providebool{printpartfrontpage}
\providebool{printactivitytitle}
\providebool{printactivityqrcode}
\providebool{printactivityurl}
\providebool{printcontinuouspagenumbers}
\providebool{numberactivitiesbysubpart}
\providebool{addtitlenumber}
\providebool{addsectiontitlenumber}
\addtitlenumbertrue
\addsectiontitlenumbertrue

% The following three commands are hardcoded in xake, you can't create other commands like these, without adding them to xake as well
%  ( gebruikt in xourses om juiste soort titelpagina te krijgen voor verschillende ximera's )
\newcommand{\activitychapter}[2][]{
    {    
    \ifstrequal{#1}{notnumbered}{
        \addtitlenumberfalse
    }{}
    \typeout{ACTIVITYCHAPTER #2}   % logging
	\chapterstyle
	\activity{#2}
    }
}
\newcommand{\activitysection}[2][]{
    {
    \ifstrequal{#1}{notnumbered}{
        \addsectiontitlenumberfalse
    }{}
	\typeout{ACTIVITYSECTION #2}   % logging
	\sectionstyle
	\activity{#2}
    }
}
% Practices worden als activity getoond om de grote blokken te krijgen online
\newcommand{\practicesection}[2][]{
    {
    \ifstrequal{#1}{notnumbered}{
        \addsectiontitlenumberfalse
    }{}
    \typeout{PRACTICESECTION #2}   % logging
	\sectionstyle
	\activity{#2}
    }
}
\newcommand{\activitychapterlink}[3][]{
    {
    \ifstrequal{#1}{notnumbered}{
        \addtitlenumberfalse
    }{}
    \typeout{ACTIVITYCHAPTERLINK #3}   % logging
	\chapterstyle
	\activitylink[#1]{#2}{#3}
    }
}

\newcommand{\activitysectionlink}[3][]{
    {
    \ifstrequal{#1}{notnumbered}{
        \addsectiontitlenumberfalse
    }{}
    \typeout{ACTIVITYSECTIONLINK #3}   % logging
	\sectionstyle
	\activitylink[#1]{#2}{#3}
    }
}


% Commando om de printstyle toe te voegen in ximera's. Zorgt ervoor dat er geen problemen zijn als je de xourses compileert
% hack om onhandige relative paden in TeX te omzeilen
% should work both in xourse and ximera (pre-112022 only in ximera; thus obsoletes adhoc setup in xourses)
% loads global.sty if present (cfr global.css for online settings!)
% use global.sty to overwrite settings in printstyle.sty ...
\newcommand{\addPrintStyle}[1]{
\iftikzexport\else   % only in PDF
  \makeatletter
  \ifx\@onlypreamble\@notprerr\else   % ONLY if in tex-preamble   (and e.g. not when included from xourse)
    \typeout{Loading printstyle}   % logging
    \usepackage{#1/printstyle} % mag enkel geinclude worden als je die apart compileert
    \IfFileExists{#1/global.sty}{
        \typeout{Loading printstyle-folder #1/global.sty}   % logging
        \usepackage{#1/global}
        }{
        \typeout{Info: No extra #1/global.sty}   % logging
    }   % load global.sty if present
    \IfFileExists{global.sty}{
        \typeout{Loading local-folder global.sty (or TEXINPUTPATH..)}   % logging
        \usepackage{global}
    }{
        \typeout{Info: No folder/global.sty}   % logging
    }   % load global.sty if present
    \IfFileExists{\currfilebase.sty}
    {
        \typeout{Loading \currfilebase.sty}
        \input{\currfilebase.sty}
    }{
        \typeout{Info: No local \currfilebase.sty}
    }
    \fi
  \makeatother
\fi
}

%
%  
% references: Ximera heeft adhoc logica	 om online labels te doen werken over verschillende files heen
% met \hyperref kan de getoonde tekst toch worden opgegeven, in plaats van af te hangen van de label-text
\ifdefined\HCode
% Link to standard \labels, but give your own description
% Usage:  Volg \hyperref[my_very_verbose_label]{deze link} voor wat tijdverlies
%   (01/2020: Ximera-server aangepast om bij class reference-keeptext de link-text NIET te vervangen door de label-text !!!) 
\renewcommand{\hyperref}[2][]{\HCode{<a class="reference reference-keeptext" href="\##1">}#2\HCode{</a>}}
%
%  Link to specific targets  (not tested ?)
\renewcommand{\hypertarget}[1]{\HCode{<a class="ximera-label" id="#1"></a>}}
\renewcommand{\hyperlink}[2]{\HCode{<a class="reference reference-keeptext" href="\##1">}#2\HCode{</a>}}
\fi

% Mmm, quid English ... (for keyword #1 !) ?
\newcommand{\wikilink}[2]{
    \hyperlink{https://nl.wikipedia.org/wiki/#1}{#2}
    \pdfOnly{\footnote{See \url{https://nl.wikipedia.org/wiki/#1}}
    }
}

\renewcommand{\figurename}{Figuur}
\renewcommand{\tablename}{Tabel}

%
% Gedoe om verschillende versies van xourse/ximera te maken afhankelijk van settings
%
% default: versie met antwoorden
% handout: versie voor de studenten, zonder antwoorden/oplossingen
% full: met alles erop en eraan, dus geschikt voor auteurs en/of lesgevers  (bevat in de pdf ook de 'online-only' stukken!)
%
%
% verder kunnen ook opties/variabele worden gezet voor hints/auteurs/uitweidingen/ etc
%
% 'Full' versie
\newtoggle{showonline}
\ifdefined\HCode   % zet default showOnline
    \toggletrue{showonline} 
\else
    \togglefalse{showonline}
\fi

% Full versie   % deprecated: see infra
\newcommand{\printFull}{
    \hintstrue
    \handoutfalse
    \toggletrue{showonline} 
}

\ifdefined\shouldPrintFull   % deprecated: see infra
    \printFull
\fi



% Overschrijf onlineOnly  (zoals gedefinieerd in ximera.cls)
\ifhandout   % in handout: gebruik de oorspronkelijke ximera.cls implementatie  (is dit wel nodig/nuttig?)
\else
    \iftoggle{showonline}{%
        \ifdefined\HCode
          \RenewEnviron{onlineOnly}{\bgroup\BODY\egroup}   % showOnline, en we zijn  online, dus toon de tekst
        \else
          \RenewEnviron{onlineOnly}{\bgroup\color{red!50!black}\BODY\egroup}   % showOnline, maar we zijn toch niet online: kleur de tekst rood 
        \fi
    }{%
      \RenewEnviron{onlineOnly}{}  % geen showOnline
    }
\fi

% hack om na hoofding van definition/proposition/... als dan niet op een nieuwe lijn te starten
% soms is dat goed en mooi, en soms niet; en in HTML is het nu (2/2020) anders dan in pdf
% vandaar suggestie om 
%     \begin{definition}[Nieuw concept] \nl
% te gebruiken als je zeker een newline wil na de hoofdig en titel
% (in het bijzonder itemize zonder \nl is 'lelijk' ...)
\ifdefined\HCode
\newcommand{\nl}{}
\else
\newcommand{\nl}{\ \par} % newline (achter heading van definition etc.)
\fi


% \nl enkel in handoutmode (ihb voor \wordChoice, die dan typisch veeeel langer wordt)
\ifdefined\HCode
\providecommand{\handoutnl}{}
\else
\providecommand{\handoutnl}{%
\ifhandout%
  \nl%
\fi%
}
\fi

% Could potentially replace \pdfOnline/\begin{onlineOnly} : 
% Usage= \ifonline{Hallo surfer}{Hallo PDFlezer}
\providecommand{\ifonline}[2]%
{
\begin{onlineOnly}#1\end{onlineOnly}%
\pdfOnly{#2}
}%


%
% Maak optionele 'basic' en 'extended' versies van een activity
%  met environment basicOnly, basicSkip en extendedOnly
%
%  (
%   Dit werkt ENKEL in de PDF; de online versies tonen (minstens voorklopig) steeds 
%   het default geval met printbasicversion en printextendversion beide FALSE
%  )
%
\providebool{printbasicversion}
\providebool{printextendedversion}   % not properly implemented
\providebool{printfullversion}       % presumably print everything (debug/auteur)
%
% only set these in xourses, and BEFORE loading this preamble
%
%\newif\ifshowbasic     \showbasictrue        % use this line in xourse to show 'basic' sections
%\newif\ifshowextended  \showextendedtrue     % use this line in xourse to show 'extended' sections
%
%
%\ifbool{showbasic}
%      { \NewEnviron{basicOnly}{\BODY} }    % if yes: just print contents
%      { \NewEnviron{basicOnly}{}      }    % if no:  completely ignore contents
%
%\ifbool{showbasic}
%      { \NewEnviron{basicSkip}{}      }
%      { \NewEnviron{basicSkip}{\BODY} }
%

\ifbool{printextendedversion}
      { \NewEnviron{extendedOnly}{\BODY} }
      { \NewEnviron{extendedOnly}{}      }
      


\ifdefined\HCode    % in html: always print
      {\newenvironment*{basicOnly}{}{}}    % if yes: just print contents
      {\newenvironment*{basicSkip}{}{}}    % if yes: just print contents
\else

\ifbool{printbasicversion}
      {\newenvironment*{basicOnly}{}{}}    % if yes: just print contents
      {\NewEnviron{basicOnly}{}      }    % if no:  completely ignore contents

\ifbool{printbasicversion}
      {\NewEnviron{basicSkip}{}      }
      {\newenvironment*{basicSkip}{}{}}

\fi

\usepackage{float}
\usepackage[rightbars,color]{changebar}

% Full versie
\ifbool{printfullversion}{
    \hintstrue
    \handoutfalse
    \toggletrue{showonline}
    \printbasicversionfalse
    \cbcolor{red}
    \renewenvironment*{basicOnly}{\cbstart}{\cbend}
    \renewenvironment*{basicSkip}{\cbstart}{\cbend}
    \def\xmtoonprintopties{FULL}   % will be printed in footer
}
{}
      
%
% Evalueer \ifhints IN de environment
%  
%
%\RenewEnviron{hint}
%{
%\ifhandout
%\ifhints\else\setbox0\vbox\fi%everything in een emty box
%\bgroup 
%\stepcounter{hintLevel}
%\BODY
%\egroup\ignorespacesafterend
%\addtocounter{hintLevel}{-1}
%\else
%\ifhints
%\begin{trivlist}\item[\hskip \labelsep\small\slshape\bfseries Hint:\hspace{2ex}]
%\small\slshape
%\stepcounter{hintLevel}
%\BODY
%\end{trivlist}
%\addtocounter{hintLevel}{-1}
%\fi
%\fi
%}

% Onafhankelijk van \ifhandout ...? TO BE VERIFIED
\RenewEnviron{hint}
{
\ifhints
\begin{trivlist}\item[\hskip \labelsep\small\bfseries Hint:\hspace{2ex}]
\small%\slshape
\stepcounter{hintLevel}
\BODY
\end{trivlist}
\addtocounter{hintLevel}{-1}
\else
\iftikzexport   % anders worden de tikz tekeningen in hints niet gegenereerd ?
\setbox0\vbox\bgroup
\stepcounter{hintLevel}
\BODY
\egroup\ignorespacesafterend
\addtocounter{hintLevel}{-1}
\fi % ifhandout
\fi %ifhints
}

%
% \tab sets typewriter-tabs (e.g. to format questions)
% (Has no effect in HTML :-( ))
%
\usepackage{tabto}
\ifdefined\HCode
  \renewcommand{\tab}{\quad}    % otherwise dummy .png's are generated ...?
\fi


% Also redefined in  preamble to get correct styling 
% for tikz images for (\tikzexport)
%

\theoremstyle{definition} % Bold titels
\makeatletter
\let\proposition\relax
\let\c@proposition\relax
\let\endproposition\relax
\makeatother
\newtheorem{proposition}{Eigenschap}


%\instructornotesfalse

% logic with \ifhandoutin ximera.cls unclear;so overwrite ...
\makeatletter
\@ifundefined{ifinstructornotes}{%
  \newif\ifinstructornotes
  \instructornotesfalse
  \newenvironment{instructorNotes}{}{}
}{}
\makeatother
\ifinstructornotes
\else
\renewenvironment{instructorNotes}%
{%
    \setbox0\vbox\bgroup
}
{%
    \egroup
}
\fi

% \RedeclareMathOperator
% from https://tex.stackexchange.com/questions/175251/how-to-redefine-a-command-using-declaremathoperator
\makeatletter
\newcommand\RedeclareMathOperator{%
    \@ifstar{\def\rmo@s{m}\rmo@redeclare}{\def\rmo@s{o}\rmo@redeclare}%
}
% this is taken from \renew@command
\newcommand\rmo@redeclare[2]{%
    \begingroup \escapechar\m@ne\xdef\@gtempa{{\string#1}}\endgroup
    \expandafter\@ifundefined\@gtempa
    {\@latex@error{\noexpand#1undefined}\@ehc}%
    \relax
    \expandafter\rmo@declmathop\rmo@s{#1}{#2}}
% This is just \@declmathop without \@ifdefinable
\newcommand\rmo@declmathop[3]{%
    \DeclareRobustCommand{#2}{\qopname\newmcodes@#1{#3}}%
}
\@onlypreamble\RedeclareMathOperator
\makeatother


%
% Engelse vertaling, vooral in mathmode
%
% 1. Algemene procedure
%
\ifdefined\isEn
 \newcommand{\nlen}[2]{#2}
 \newcommand{\nlentext}[2]{\text{#2}}
 \newcommand{\nlentextbf}[2]{\textbf{#2}}
\else
 \newcommand{\nlen}[2]{#1}
 \newcommand{\nlentext}[2]{\text{#1}}
 \newcommand{\nlentextbf}[2]{\textbf{#1}}
\fi

%
% 2. Lijst van erg veel gebruikte uitdrukkingen
%

% Ja/Nee/Fout/Juits etc
%\newcommand{\TJa}{\nlentext{ Ja }{ and }}
%\newcommand{\TNee}{\nlentext{ Nee }{ No }}
%\newcommand{\TJuist}{\nlentext{ Juist }{ Correct }
%\newcommand{\TFout}{\nlentext{ Fout }{ Wrong }
\newcommand{\TWaar}{\nlentext{ Waar }{ True }}
\newcommand{\TOnwaar}{\nlentext{ Vals }{ False }}
% Korte bindwoorden en, of, dus, ...
\newcommand{\Ten}{\nlentext{ en }{ and }}
\newcommand{\Tof}{\nlentext{ of }{ or }}
\newcommand{\Tdus}{\nlentext{ dus }{ so }}
\newcommand{\Tendus}{\nlentext{ en dus }{ and thus }}
\newcommand{\Tvooralle}{\nlentext{ voor alle }{ for all }}
\newcommand{\Took}{\nlentext{ ook }{ also }}
\newcommand{\Tals}{\nlentext{ als }{ when }} %of if?
\newcommand{\Twant}{\nlentext{ want }{ as }}
\newcommand{\Tmaal}{\nlentext{ maal }{ times }}
\newcommand{\Toptellen}{\nlentext{ optellen }{ add }}
\newcommand{\Tde}{\nlentext{ de }{ the }}
\newcommand{\Thet}{\nlentext{ het }{ the }}
\newcommand{\Tis}{\nlentext{ is }{ is }} %zodat is in text staat in mathmode (geen italics)
\newcommand{\Tmet}{\nlentext{ met }{ where }} % in situaties e.g met p < n --> where p < n
\newcommand{\Tnooit}{\nlentext{ nooit }{ never }}
\newcommand{\Tmaar}{\nlentext{ maar }{ but }}
\newcommand{\Tniet}{\nlentext{ niet }{ not }}
\newcommand{\Tuit}{\nlentext{ uit }{ from }}
\newcommand{\Ttov}{\nlentext{ t.o.v. }{ w.r.t. }}
\newcommand{\Tzodat}{\nlentext{ zodat }{ such that }}
\newcommand{\Tdeth}{\nlentext{de }{th }}
\newcommand{\Tomdat}{\nlentext{omdat }{because }} 


%
% Overschrijf addhoc commando's
%
\ifdefined\isEn
\renewcommand{\pernot}{\overset{\mathrm{notation}}{=}}
\RedeclareMathOperator{\bld}{im}     % beeld
\RedeclareMathOperator{\graf}{graph}   % grafiek
\RedeclareMathOperator{\rico}{slope}   % richtingcoëfficient
\RedeclareMathOperator{\co}{co}       % coordinaat
\RedeclareMathOperator{\gr}{deg}       % graad

% Operators
\RedeclareMathOperator{\bgsin}{arcsin}
\RedeclareMathOperator{\bgcos}{arccos}
\RedeclareMathOperator{\bgtan}{arctan}
\RedeclareMathOperator{\bgcot}{arccot}
\RedeclareMathOperator{\bgsinh}{arcsinh}
\RedeclareMathOperator{\bgcosh}{arccosh}
\RedeclareMathOperator{\bgtanh}{arctanh}
\RedeclareMathOperator{\bgcoth}{arccoth}

\fi


% HACK: use 'oplossing' for 'explanation' ...
\let\explanation\relax
\let\endexplanation\relax
% \newenvironment{explanation}{\begin{oplossing}}{\end{oplossing}}
\newcounter{explanation}

\ifhandout%
    \NewEnviron{explanation}[1][toon]%
    {%
    \RenewEnviron{verbatim}{ \red{VERBATIM CONTENT MISSING IN THIS PDF}} %% \expandafter\verb|\BODY|}

    \ifthenelse{\equal{\detokenize{#1}}{\detokenize{toon}}}
    {
    \def\PH@Command{#1}% Use PH@Command to hold the content and be a target for "\expandafter" to expand once.

    \begin{trivlist}% Begin the trivlist to use formating of the "Feedback" label.
    \item[\hskip \labelsep\small\slshape\bfseries Explanation:% Format the "Feedback" label. Don't forget the space.
    %(\texttt{\detokenize\expandafter{\PH@Command}}):% Format (and detokenize) the condition for feedback to trigger
    \hspace{2ex}]\small%\slshape% Insert some space before the actual feedback given.
    \BODY
    \end{trivlist}
    }
    {  % \begin{feedback}[solution]   \BODY     \end{feedback}  }
    }
    }    
\else
% ONLY for HTML; xmoplossing is styled with css, and is not, and need not be a LaTeX environment
% THUS: it does NOT use feedback anymore ...
%    \NewEnviron{oplossing}{\begin{expandable}{xmoplossing}{\nlen{Toon uitwerking}{Show solution}}{\BODY}\end{expandable}}
    \newenvironment{explanation}[1][toon]
   {%
       \begin{expandable}{xmoplossing}{}
   }
   {%
   	   \end{expandable}
   } 
\fi
\title{Inner Product Spaces} \license{CC BY-NC-SA 4.0}
\begin{document}
\begin{abstract}
 \end{abstract}
\maketitle

\section*{Inner Product Spaces}

We have used the dot product in $\RR^n$ to compute the length of vectors (\ref{cor:length_via_dotprod}) and also the \href{https://ximera.osu.edu/oerlinalg/LinearAlgebra/VEC-0060/main}{angle between vectors}. The plan in this section is to define an \dfn{inner product} on an arbitrary real vector space $V$ (of which the dot product is an example in $\RR^n$) and use it to introduce these concepts in $V$.

\begin{definition}\label{def:innerproductspace}
An \dfn{inner product} on a real vector space $V$ is a function that assigns a real number $\langle\vec{v}, \vec{w}\rangle$ to every pair $\vec{v}$, $\vec{w}$ of vectors in $V$ in such a way that the following axioms are satisfied.
\end{definition}
\begin{itemize}
\item[\textit{P1.}]\label{prop:inner_prod_1}  $\langle\vec{v}, \vec{w}\rangle$ \textit{is a real number for all} $\vec{v}$ \textit{and} $\vec{w}$ \textit{in} $V$.

\item[\textit{P2.}]\label{prop:inner_prod_2}  $\langle\vec{v}, \vec{w}\rangle = \langle\vec{w}, \vec{v}\rangle$ \textit{for all} $\vec{v}$ \textit{and} $\vec{w}$ \textit{in} $V$.

\item[\textit{P3.}]\label{prop:inner_prod_3}  $\langle\vec{v} + \vec{w}, \vec{u}\rangle = \langle\vec{v}, \vec{u}\rangle + \langle\vec{w}, \vec{u}\rangle$ \textit{for all} $\vec{u}$, $\vec{v}$, \textit{and} $\vec{w}$ \textit{in} $V$.

\item[\textit{P4.}]\label{prop:inner_prod_4} $\langle r\vec{v}, \vec{w}\rangle = r\langle\vec{v}, \vec{w}\rangle$ \textit{for all} $\vec{v}$ \textit{and} $\vec{w}$ \textit{in} $V$ \textit{and all} $r$ \textit{in} $\RR$.

\item[\textit{P5.}]\label{prop:inner_prod_5}  $\langle\vec{v}, \vec{v}\rangle > 0$ \textit{for all} $\vec{v} \neq \vec{0}$ \textit{in} $V$.

\end{itemize}

A real vector space $V$ with an inner product $\langle\ , \rangle$ will be called an \dfn{inner product space}.  Note that every subspace of an inner product space is again an inner product space using the same inner product.
\begin{remark}
    If we regard $\mathbb{C}^n$ as a vector space over the field $\mathbb{C}$  of complex numbers, then the ``standard inner product'' on $\mathbb{C}^n$ defined in Section~\ref{sec:8_6} does not satisfy Axiom P4 (see Theorem~\ref{thm:025575}(3)).
\end{remark}

\begin{example}\label{exa:030303}
$\RR^n$ is an inner product space with the dot product as inner product:
\begin{equation*}
\langle \vec{v}, \vec{w} \rangle = \vec{v} \dotp \vec{w} \quad \mbox{ for all } \vec{v},  \vec{w} \in \RR^n
\end{equation*}
See Theorem~\ref{th:dotproductproperties}. This is also called the
\dfn{euclidean} inner product, and
$\RR^n$, equipped with the dot product, is called \dfn{euclidean}
$n$-\dfn{space}.
\end{example}

\begin{example}\label{exa:030310}
If $A$ and $B$ are $m \times n$ matrices, define $\langle A, B\rangle = \mbox{tr}(AB^{T})$ where $\mbox{tr}(X)$ is the trace of the square matrix $X$. Show that $\langle\ , \rangle$ is an inner product in $\mathbb{M}_{mn}$.

\begin{explanation}
P1 is clear. Since $\mbox{tr}(P) = \mbox{tr}(P^{T})$ for every square matrix $P$, we have P2:
\begin{equation*}
\langle A, B \rangle = \mbox{tr}(AB^T) = \mbox{tr}[(AB^T)^T] = \mbox{tr}(BA^T) = \langle B, A \rangle
\end{equation*}
Next, P3 and P4 follow because trace is a linear transformation $\mathbb{M}_{mn} \to \RR$ (Practice Problem \ref{ex:10_1_19}). Turning to P5, let $\vec{r}_{1}, \vec{r}_{2}, \dots, \vec{r}_{m}$ denote the rows of the matrix $A$. Then the $(i, j)$-entry of $AA^{T}$ is $\vec{r}_{i} \dotp \vec{r}_{j}$, so
\begin{equation*}
\langle A, A \rangle = \mbox{tr}(AA^T) =
\vec{r}_1 \dotp \vec{r}_1 +
\vec{r}_2 \dotp \vec{r}_2 + \dots +
\vec{r}_m \dotp \vec{r}_m
\end{equation*}
But $\vec{r}_{j} \dotp \vec{r}_{j}$ is the sum of the squares of
the entries of $\vec{r}_{j}$, so this shows that $\langle A, A\rangle$ is the sum of the squares of all $nm$ entries of $A$. Axiom P5 follows.
\end{explanation}
\end{example}

The next example is important in analysis.

\begin{example}\label{exa:030334}
Let $\mathcal{C}[a, b]$ denote the vector space of
continuous functions from $[a, b]$ to $\RR$, a subspace of $\mathcal{F}[a, b]$. Show that
\begin{equation*}
\langle f, g \rangle = \int_{a}^{b} f(x)g(x)dx
\end{equation*}
defines an inner product on $\mathcal{C}[a, b]$.

\begin{remark}
    This example (and others later that refer to it) can be omitted with no loss of continuity by students with no calculus background.
\end{remark}

\begin{explanation}
Axioms P1 and P2 are clear. As to axiom P4,
\begin{equation*}
\langle rf, g \rangle = \int_{a}^{b} rf(x)g(x)dx = r\int_{a}^{b} f(x)g(x)dx =
r\langle f, g \rangle
\end{equation*}
Axiom P3 is similar. Finally, theorems of calculus show that $\langle f, f \rangle = \int_{a}^{b} f(x)^2dx \geq 0$ and, if $f$ is continuous, that this is zero if and only if $f$ is the zero function. This gives axiom P5.
\end{explanation}
\end{example}


If $\vec{v}$ is any vector, then, using axiom P3, we get
\begin{equation*}
\langle \vec{0}, \vec{v} \rangle = \langle \vec{0} + \vec{0}, \vec{v} \rangle =
\langle \vec{0}, \vec{v} \rangle + \langle \vec{0}, \vec{v} \rangle
\end{equation*}
and it follows that the number $\langle\vec{0}, \vec{v}\rangle$ must be zero. This observation is recorded for reference in the following theorem, along with several other properties of inner products. The other proofs are left as Practice Problem \ref{ex:10_1_20}.

\begin{theorem}\label{thm:030346}
Let $\langle\ , \rangle$ be an inner product on a space $V$; let $\vec{v}$, $\vec{u}$, and $\vec{w}$ denote vectors in $V$; and let $r$ denote a real number.

\begin{enumerate}
\item\label{030346a} $\langle \vec{u}, \vec{v} + \vec{w}\rangle =
  \langle\vec{u}, \vec{v}\rangle + \langle\vec{u}, \vec{w}\rangle$

\item\label{030346b} $\langle\vec{v}, r\vec{w}\rangle =  r\langle\vec{v}, \vec{w}\rangle = \langle r\vec{v}, \vec{w}\rangle$

\item\label{030346c} $\langle\vec{v}, \vec{0}\rangle = 0 = \langle\vec{0}, \vec{v} \rangle $

\item\label{030346d} $\langle\vec{v}, \vec{v}\rangle = 0$ if and only if $\vec{v} = \vec{0}$

\end{enumerate}
\end{theorem}

\begin{remark}
    If $r$ is a complex number, then \ref{030346b} must be modified.  See Theorem \ref{th:025575}.
\end{remark}

If $\langle\ , \rangle$ is an inner product on a space $V$, then, given $\vec{u}$, $\vec{v}$, and $\vec{w}$ in $V$,
\begin{equation*}
\langle r\vec{u} + s\vec{v}, \vec{w} \rangle = \langle r\vec{u}, \vec{w} \rangle + \langle s\vec{v}, \vec{w} \rangle = r\langle \vec{u}, \vec{w} \rangle + s\langle \vec{v}, \vec{w} \rangle
\end{equation*}
for all $r$ and $s$ in $\RR$ by axioms P3 and P4. Moreover, there is nothing special about the fact that there are two terms in the linear combination or that it is in the first component:
\begin{equation*}
\langle r_1\vec{v}_1 + r_2\vec{v}_2 + \dots + r_n\vec{v}_n, \vec{w} \rangle =
r_1\langle \vec{v}_1, \vec{w} \rangle +
r_2\langle \vec{v}_2, \vec{w} \rangle + \dots +
r_n\langle \vec{v}_n, \vec{w} \rangle
\end{equation*}
and
\begin{equation*}
\langle \vec{v}, s_1\vec{w}_1 + s_2\vec{w}_2 + \dots + s_m\vec{w}_m \rangle =
s_1\langle \vec{v}, \vec{w}_1 \rangle +
s_2\langle \vec{v}, \vec{w}_2 \rangle + \dots +
s_m\langle \vec{v}, \vec{w}_m \rangle
\end{equation*}
hold for all $r_{i}$ and $s_{i}$ in $\RR$ and all $\vec{v}$, $\vec{w}$, $\vec{v}_{i}$, and $\vec{w}_{j}$ in $V$. These results are described by saying that inner products ``preserve'' linear combinations. For example,
\begin{align*}
\langle 2\vec{u} - \vec{v}, 3\vec{u} + 2\vec{v} \rangle &=
\langle 2\vec{u}, 3\vec{u} \rangle + \langle 2\vec{u}, 2\vec{v} \rangle + \langle -\vec{v}, 3\vec{u} \rangle + \langle -\vec{v}, 2\vec{v} \rangle \\
&= 6 \langle \vec{u}, \vec{u} \rangle + 4 \langle \vec{u}, \vec{v} \rangle -3 \langle \vec{v}, \vec{u} \rangle - 2 \langle \vec{v}, \vec{v} \rangle \\
&= 6 \langle \vec{u}, \vec{u} \rangle + \langle \vec{u}, \vec{v} \rangle - 2 \langle \vec{v}, \vec{v} \rangle
\end{align*}

If $A$ is a symmetric $n \times n$ matrix and $\vec{x}$ and $\vec{y}$ are columns in $\RR^n$, we regard the $1 \times 1$ matrix $\vec{x}^{T}A\vec{y}$ as a number. If we write
\begin{equation*}
\langle \vec{x}, \vec{y} \rangle = \vec{x}^TA\vec{y} \quad \mbox{ for all columns } \vec{x}, \vec{y} \mbox{ in } \RR^n
\end{equation*}
then axioms P1--P4 follow from matrix arithmetic (only P2 requires that $A$ is symmetric). Axiom P5 reads
\begin{equation*}
\vec{x}^TA \vec{x} > 0 \quad \mbox{ for all columns } \vec{x} \neq \vec{0} \mbox{ in } \RR^n
\end{equation*}
and this condition characterizes the positive definite matrices  (Theorem~\ref{thm:024830}). This proves the first assertion in the next theorem.

\begin{theorem}\label{thm:030372}
If $A$ is any $n \times n$ positive definite matrix, then
\begin{equation*}
\langle \vec{x}, \vec{y} \rangle = \vec{x}^TA\vec{y} \mbox{ for all columns } \vec{x}, \vec{y} \mbox{ in } \RR^n
\end{equation*}
defines an inner product on $\RR^n$, and every inner product on $\RR^n$ arises in this way.
\end{theorem}

\begin{proof}
Given an inner product $\langle\ , \rangle$ on $\RR^n$, let $\{\vec{e}_{1}, \vec{e}_{2}, \dots, \vec{e}_{n}\}$ be the standard basis of $\RR^n$. If $\vec{x} = \displaystyle \sum_{i = 1}^{n} x_i\vec{e}_i$ and $\vec{y} = \displaystyle \sum_{j = 1}^{n} y_j\vec{e}_j$ are two vectors in $\RR^n$, compute $\langle\vec{x}, \vec{y}\rangle$ by adding the inner product of each term $x_{i}\vec{e}_{i}$ to each term $y_{j}\vec{e}_{j}$. The result is a double sum.
\begin{equation*}
\langle \vec{x}, \vec{y} \rangle = \displaystyle \sum_{i = 1}^{n} \sum_{j = 1}^{n} \langle x_i \vec{e}_i, y_j\vec{e}_j \rangle =
\displaystyle \sum_{i = 1}^{n} \sum_{j = 1}^{n} x_i \langle \vec{e}_i, \vec{e}_j \rangle y_j
\end{equation*}
As the reader can verify, this is a matrix product:
\begin{equation*}
\langle \vec{x}, \vec{y} \rangle =
\left[ \begin{array}{cccc}
x_1 & x_2 & \cdots & x_n \\
\end{array} \right]
\left[ \begin{array}{cccc}
\langle \vec{e}_1, \vec{e}_1 \rangle & \langle \vec{e}_1, \vec{e}_2 \rangle & \cdots & \langle \vec{e}_1, \vec{e}_n \rangle \\
\langle \vec{e}_2, \vec{e}_1 \rangle & \langle \vec{e}_2, \vec{e}_2 \rangle & \cdots & \langle \vec{e}_2, \vec{e}_n \rangle \\
\vdots & \vdots & \ddots & \vdots \\
\langle \vec{e}_n, \vec{e}_1 \rangle & \langle \vec{e}_n, \vec{e}_2 \rangle & \cdots & \langle \vec{e}_n, \vec{e}_n \rangle \\
\end{array} \right]
\left[ \begin{array}{c}
	y_1 \\
	y_2 \\
	\vdots \\
	y_n
\end{array} \right]
\end{equation*}
Hence $\langle\vec{x}, \vec{y}\rangle = \vec{x}^{T}A\vec{y}$, where $A$ is the $n \times n$ matrix whose $(i, j)$-entry is $\langle\vec{e}_{i}, \vec{e}_{j} \rangle$. The fact that
\begin{equation*}
\langle\vec{e}_{i}, \vec{e}_{j}\rangle = \langle\vec{e}_{j}, \vec{e}_{i}\rangle\end{equation*}
shows that $A$ is symmetric. Finally, $A$ is positive definite by Theorem~\ref{thm:024830}.
\end{proof}

Thus, just as every linear operator $\RR^n \to \RR^n$ corresponds to an $n \times n$ matrix, every inner product on $\RR^n$ corresponds to a positive definite $n \times n$ matrix. In particular, the dot product corresponds to the identity matrix $I_{n}$.

\begin{remark}
    If we refer to the inner product space $\RR^n$ without specifying the inner product, we mean that the dot product is to be used.
\end{remark}

\begin{example}\label{exa:030413}
Let the inner product $\langle\ , \rangle$ be defined on $\RR^2$ by
\begin{equation*}
\left \langle
\left[ \begin{array}{c}
v_1 \\
v_2
\end{array} \right], \left[ \begin{array}{c}
w_1 \\
w_2
\end{array} \right]
\right \rangle
= 2v_1w_1 - v_1w_2 - v_2w_1 + v_2w_2
\end{equation*}
Find a symmetric $2 \times 2$ matrix $A$ such that $\langle\vec{x}, \vec{y}\rangle = \vec{x}^{T}A\vec{y}$ for all $\vec{x}$, $\vec{y}$ in $\RR^2$.

\begin{explanation}
The $(i, j)$-entry of the matrix $A$ is the coefficient of $v_{i}w_{j}$ in the expression, so
$ A =
\left[ \begin{array}{rr}
2 & -1 \\
-1 & 1
\end{array} \right]$. Incidentally, if
$\vec{x} =
\left[ \begin{array}{r}
x \\
y
\end{array} \right]$, then
\begin{equation*}
\langle \vec{x}, \vec{x} \rangle = 2x^2 - 2xy + y^2 = x^2 +(x - y)^2 \geq 0
\end{equation*}

for all $\vec{x}$, so $\langle\vec{x}, \vec{x}\rangle = 0$ implies
$\vec{x} = \vec{0}$. Hence $\langle\ , \rangle$ is indeed an inner product, so $A$ is positive definite.
\end{explanation}
\end{example}

Let $\langle\ , \rangle$ be an inner product on $\RR^n$ given as in Theorem~\ref{thm:030372} by a positive definite matrix $A$. If $\vec{x} =
\left[ \begin{array}{cccc}
x_1 & x_2 & \cdots & x_n
\end{array} \right]^T $, then $\langle\vec{x}, \vec{x}\rangle = \vec{x}^{T}A\vec{x}$ is an expression in the variables
$x_{1}, x_{2}, \dots, x_{n}$ called a \dfn{quadratic form}. For more on quadratic forms, see Section~\ref{sec:8_8} of [Nicholson], pp. 472--482.

\subsection*{Norm and Distance}

\begin{definition}\label{def:030438}
As in $\RR^n$, if $\langle\ , \rangle$ is an inner product on a space $V$, the \dfn{norm}
 $\norm{\vec{v}}$ of a vector $\vec{v}$ in $V$ is defined by
\begin{equation*}
\norm{ \vec{v} } = \sqrt{\langle \vec{v}, \vec{v} \rangle}
\end{equation*}
We define the \dfn{distance} between vectors $\vec{v}$ and $\vec{w}$ in an inner product space $V$ to be
\begin{equation*}
\mbox{d}(\vec{v}, \vec{w}) = \norm{ \vec{v} - \vec{w} }
\end{equation*}
\end{definition}
\begin{remark}
 If the dot product is used in $\RR^n$, the norm $\norm{\vec{x}}$ of a vector $\vec{x}$ is usually called the \dfn{length} of $\vec{x}$.   
\end{remark}


Note that axiom P5 guarantees that
$\langle\vec{v}, \vec{v}\rangle \geq 0$, so $\norm{\vec{v}}$ is a real number.

\begin{example}\label{exa:030446}
%NEED FIGURE

The norm of a continuous function $f = f(x)$ in $\mathcal{C}[a, b]$ (with the inner product from Example~\ref{exa:030334}) is given by
\begin{equation*}
\norm{ f } = \sqrt{\int_{a}^{b} f(x)^2dx}
\end{equation*}
Hence $\norm{ f}^{2}$ is the area beneath the graph of $y = f(x)^{2}$ between $x = a$ and $x = b$ (shaded in the diagram).
\end{example}

\begin{example}\label{030454}
Show that $\langle\vec{u} + \vec{v}, \vec{u} - \vec{v}\rangle = \norm{\vec{u}}^{2} - \norm{\vec{v}}^{2}$ in any inner product space.

\begin{explanation}
\begin{align*}
\langle \vec{u} + \vec{v}, \vec{u} - \vec{v} \rangle &= \langle \vec{u}, \vec{u} \rangle - \langle \vec{u}, \vec{v} \rangle + \langle \vec{v}, \vec{u} \rangle - \langle \vec{v}, \vec{v} \rangle \\
&= \norm{ \vec{u} }^2 - \langle \vec{u}, \vec{v} \rangle + \langle \vec{u}, \vec{v} \rangle - \norm{ \vec{v} }^2 \\
&= \norm{ \vec{u} }^2 - \norm{ \vec{v} }^2
\end{align*}
\end{explanation}
\end{example}

A vector $\vec{v}$ in an inner product space $V$ is called a \dfn{unit vector} if $\norm{\vec{v}} = 1$. The set of all unit vectors in $V$ is called the \dfn{unit ball} in $V$. For example, if $V = \RR^2$ (with the dot product) and $\vec{v} = (x, y)$, then
\begin{equation*}
\norm{ \vec{v} }^2 = 1 \quad \mbox{ if and only if } \quad x^2 + y^2 = 1
\end{equation*}
Hence the unit ball in $\RR^2$ is the \dfn{unit circle} $x^{2} + y^{2} = 1$ with centre at the origin and radius $1$. However, the shape of the unit ball varies with the choice of inner product.

\begin{example}\label{030469}
%NEED FIGURE


Let $a > 0$ and $b > 0$. If $\vec{v} = (x, y)$ and $\vec{w} = (x_{1}, y_{1})$, define an inner product on $\RR^2$ by
\begin{equation*}
\langle \vec{v}, \vec{w} \rangle = \frac{xx_1}{a^2} + \frac{yy_1}{b^2}
\end{equation*}
The reader can verify (Practice Problem \ref{ex:10_1_5}) that this is indeed an inner product. In this case
\begin{equation*}
\norm{ \vec{v} }^2 = 1 \quad \mbox{ if and only if } \quad \frac{x^2}{a^2} + \frac{y^2}{b^2} = 1
\end{equation*}
so the unit ball is the ellipse shown in the diagram.
\end{example}

\noindent Example~\ref{exa:030469} graphically illustrates the fact that norms and distances in an inner product space $V$ vary with the choice of inner product in $V$.

\begin{theorem}\label{030480}
If $\vec{v} \neq \vec{0}$ is any vector in an inner product space $V$, then $\frac{1}{\norm{ \vec{v} }} \vec{v}$ is the unique unit vector that is a positive multiple of $\vec{v}$.
\end{theorem}

The next theorem reveals an important and useful fact about the relationship between norms and inner products, extending the Cauchy inequality for $\RR^n$ (Theorem~\ref{thm:014907}).

\begin{theorem}[Cauchy-Schwarz Inequality]\label{030486}
If $\vec{v}$ and $\vec{w}$ are two vectors in an inner product space $V$, then
\begin{equation*}
\langle \vec{v}, \vec{w} \rangle^2 \leq \norm{ \vec{v} }^2 \norm{ \vec{w} }^2
\end{equation*}
Moreover, equality occurs if and only if one of $\vec{v}$ and $\vec{w}$ is a scalar multiple of the other.
\end{theorem}
\begin{remark}
   Hermann Amandus Schwarz (1843--1921) was a German mathematician at the University of Berlin. He had strong geometric intuition, which he applied with great ingenuity to particular problems. A version of the inequality appeared in 1885. 
\end{remark}

\begin{proof}
Write $\norm{\vec{v}} = a$ and $\norm{\vec{w}} = b$. Using Theorem~\ref{thm:030346} we compute:
\begin{equation}
\label{eq:thm10_1_4}
\begin{split}
	\norm{ b\vec{v} - a \vec{w} }^2 &= b^2 \norm{ \vec{v} }^2 - 2ab \langle \vec{v}, \vec{w} \rangle + a^2\norm{ \vec{w} }^2 = 2ab(ab - \langle \vec{v}, \vec{w} \rangle) \\
	\norm{ b\vec{v} + a \vec{w} }^2 &= b^2 \norm{ \vec{v} }^2 + 2ab \langle \vec{v}, \vec{w} \rangle + a^2\norm{ \vec{w} }^2 = 2ab(ab + \langle \vec{v}, \vec{w} \rangle) \\
\end{split}
\end{equation}
It follows that $ab - \langle\vec{v}, \vec{w}\rangle \geq 0$ and
$ab + \langle\vec{v}, \vec{w}\rangle \geq 0$, and hence that $-ab \leq \langle\vec{v}, \vec{w}\rangle \leq ab$. But then $| \langle\vec{v}, \vec{w}\rangle | \leq ab = \norm{\vec{v}} \norm{ \vec{w}}$, as desired.

Conversely, if $|\langle \vec{v},  \vec{w}\rangle | =
\norm{\vec{v}} \norm{ \vec{w} } = ab$
then $\langle\vec{v}, \vec{w}\rangle = \pm ab$. Hence (\ref{eq:thm10_1_4}) shows that $b\vec{v} - a\vec{w} = \vec{0}$ or $b\vec{v} + a\vec{w} = \vec{0}$. It follows that one of $\vec{v}$ and $\vec{w}$ is a scalar multiple of the other, even if $a = 0$ or $b = 0$.
\end{proof}

\begin{example}\label{030499}
If $f$ and $g$ are continuous functions on the interval $[a, b]$, then (see Example~\ref{exa:030334})
\begin{equation*}
\left(\int_{a}^{b} f(x)g(x)dx \right) ^2 \leq \int_{a}^{b} f(x)^2 dx \int_{a}^{b} g(x)^2 dx
\end{equation*}
\end{example}

Another famous inequality, the so-called \dfn{triangle inequality} (See \href{https://ximera.osu.edu/oerlinalg/LinearAlgebra/APX-0010/main}{Triangle Inequality} in the Appendix), also comes from the Cauchy-Schwarz inequality. It is included in the following list of basic properties of the norm of a vector.

\begin{theorem}\label{thm:030504}
If $V$ is an inner product space, the norm $\norm{ \dotp }$ has the following properties.

\begin{enumerate}
\item\label{thm:030504a} $\norm{\vec{v}} \geq 0$ for every vector $\vec{v}$ in $V$.

\item\label{thm:030504b} $\norm{\vec{v}} = 0$ if and only if $\vec{v} = \vec{0}$.

\item\label{thm:030504c} $\norm{ r \vec{v}} = |r|\norm{\vec{v}}$ for every $\vec{v}$ in $V$ and every $r$ in $\RR$.

\item\label{thm:030504d} $\norm{\vec{v} + \vec{w}} \leq \norm{\vec{v}} + \norm{\vec{w}}$ for all $\vec{v}$ and $\vec{w}$ in $V$ (\dfn{triangle inequality}).

\end{enumerate}
\end{theorem}

\begin{proof}
Because $\norm{ \vec{v} } = \sqrt{\langle \vec{v}, \vec{v} \rangle}$, properties \ref{thm:030504a} and \ref{thm:030504b} follow immediately from \ref{thm:030346c} and \ref{thm:030346d} of Theorem~\ref{thm:030346}. As to \ref{thm:030504c}, compute
\begin{equation*}
\norm{ r\vec{v} } ^2 = \langle r\vec{v}, r\vec{v} \rangle = r^2\langle \vec{v}, \vec{v} \rangle = r^2\norm{ \vec{v} }^2
\end{equation*}
Hence \ref{thm:030504c} follows by taking positive square roots. Finally, the fact that $\langle\vec{v}, \vec{w}\rangle \leq \norm{\vec{v}}\norm{\vec{w}}$ by the Cauchy-Schwarz inequality gives
\begin{align*}
\norm{ \vec{v} + \vec{w} } ^2 =
\langle \vec{v} + \vec{w}, \vec{v} + \vec{w} \rangle &=
\norm{ \vec{v} } ^2 + 2 \langle \vec{v}, \vec{w} \rangle +
\norm{ \vec{w} } ^2 \\
&\leq \norm{ \vec{v} } ^2 +
2 \norm{ \vec{v} } \norm{ \vec{w} } +
\norm{ \vec{w} } ^2 \\
&= (\norm{ \vec{v} } + \norm{ \vec{w} })^2
\end{align*}
Hence \ref{thm:030504d} follows by taking positive square roots.
\end{proof}

It is worth noting that the usual triangle inequality for absolute values,
\begin{equation*}
| r + s | \leq |r| + |s| \mbox{ for all real numbers } r \mbox{ and } s
\end{equation*}
is a special case of \ref{thm:030504d} where $V = \RR = \RR^1$ and the dot product $\langle r, s \rangle = rs$ is used.

In many calculations in an inner product space, it is required to show that some vector $\vec{v}$ is zero. This is often accomplished most easily by showing that its norm $\norm{\vec{v}}$ is zero. Here is an example.

\begin{example}\label{030528}
Let $\{\vec{v}_{1}, \dots, \vec{v}_{n}\}$ be a spanning set for an inner product space $V$. If $\vec{v}$ in $V$ satisfies $\langle\vec{v}, \vec{v}_{i}\rangle = 0$ for each $i = 1, 2, \dots, n$, show that $\vec{v} = \vec{0}$.

\begin{explanation}
Write $\vec{v} = r_{1}\vec{v}_{1} + \dots + r_{n}\vec{v}_{n}$, $r_{i}$ in $\RR$. To show that $\vec{v} = \vec{0}$, we show that $\norm{\vec{v}}^{2} = \langle\vec{v}, \vec{v}\rangle = 0$. Compute:
\begin{equation*}
\langle \vec{v}, \vec{v} \rangle
= \langle \vec{v}, r_1\vec{v}_1 + \dots + r_n\vec{v}_n \rangle
= r_1\langle \vec{v}, \vec{v}_1 \rangle + \dots + r_n \langle \vec{v}, \vec{v}_n \rangle
= 0
\end{equation*}
by hypothesis, and the result follows.
\end{explanation}
\end{example}

The norm properties in Theorem~\ref{thm:030504} translate to the following properties of distance familiar from geometry. The proof is Practice Problem \ref{ex:10_1_21}.

\begin{theorem}\label{030545}
Let $V$ be an inner product space.

\begin{enumerate}
\item $\mbox{d}(\vec{v}, \vec{w}) \geq 0$ for all $\vec{v}$, $\vec{w}$ in $V$.

\item $\mbox{d}(\vec{v}, \vec{w}) = 0$ if and only if $\vec{v} = \vec{w}$.

\item $\mbox{d}(\vec{v}, \vec{w}) = \mbox{d}(\vec{w}, \vec{v})$ for all $\vec{v}$ and $\vec{w}$ in $V$.

\item $\mbox{d}(\vec{v}, \vec{w}) \leq \mbox{d}(\vec{v}, \vec{u}) + \mbox{d}(\vec{u}, \vec{w})$ for all $\vec{v}$, $\vec{u}$, and $\vec{w}$ in $V$.

\end{enumerate}
\end{theorem}

\section*{Practice Problems}

\begin{problem}\label{prob:inner_prod_1}
In each case, determine which of axioms P1--P5 in Definition \ref{def:innerproductspace} fail to hold.

\begin{enumerate} 
\item $V = \RR^2$, $\langle (x_1, y_1), (x_2, y_2) \rangle = x_1y_1x_2y_2$

\item $V = \RR^3$, \\$\langle (x_1, x_2, x_3), (y_1, y_2, y_3) \rangle = x_1y_1 - x_2y_2 + x_3y_3$

Click the arrow to see the answer.
\begin{expandable}
P5 fails.
\end{expandable}

\item $V = \mathbb{C}$, $\langle z, w \rangle = z\overline{w}$, where $\overline{w}$ is complex
conjugation

Click the arrow to see the answer.
\begin{expandable}
P1 fails, as sometimes we get a complex number.  However, we will return to this definition of $\langle\ , \rangle$ in \href{https://ximera.osu.edu/oerlinalg/LinearAlgebra/RTH-0050/main}{Complex Matrices} -- see \ref{def:025549}
\end{expandable}

\item $V = \mathbb{P}^3$, $\langle p(x), q(x) \rangle = p(1)q(1)$

Click the arrow to see the answer.
\begin{expandable}
P5 fails.
\end{expandable}

\item $V = \mathbb{M}_{22}$, $\langle A, B \rangle = \mbox{det}(AB)$

\item $V = \mathcal{F}[0, 1]$, $\langle f, g \rangle = f(1)g(0) + f(0)g(1)$

Click the arrow to see the answer.
\begin{expandable}
P5 fails.
\end{expandable}

\end{enumerate}
\end{problem}

\begin{problem}\label{prob:inner_prod_2}
Let $V$ be an inner product space. If $U \subseteq V$ is a subspace, show that $U$ is an inner product space using the same inner product.

\begin{hint}
Axioms P1--P5 hold in $U$ because they hold in $V$.
\end{hint}
\end{problem}

\begin{problem}\label{prob:inner_prod_3}
In each case, find a scalar multiple of $\vec{v}$ that is a unit vector.

\begin{enumerate} 
\item $\vec{v} = f$ in $\mathcal{C}[0, 1]$ where
$f(x) = x^2$  \\ $\langle f, g \rangle \int_{0}^{1} f(x)g(x)dx$

\item $\vec{v} = f$ in $\mathcal{C}[-\pi, \pi]$ where
$f(x) = \cos x$ \\ $\langle f, g \rangle \int_{-\pi}^{\pi} f(x)g(x)dx$

Click the arrow to see the answer.
\begin{expandable}
$\frac{1}{\sqrt{\pi}}f$
\end{expandable}

\item $\vec{v} =
\left[ \begin{array}{r}
1 \\
3
\end{array} \right]$
in $\RR^2$ where $\langle \vec{v}, \vec{w} \rangle = \vec{v}^T
\left[ \begin{array}{rr}
1 & 1 \\
1 & 2
\end{array} \right]
\vec{w}$

\item $ \vec{v} =
\left[ \begin{array}{r}
3 \\
-1
\end{array} \right]$
in $\RR^2$, $\langle \vec{v}, \vec{w} \rangle = \vec{v}^T
\left[ \begin{array}{rr}
1 & -1 \\
-1 & 2
\end{array} \right]
\vec{w}$

Click the arrow to see the answer.
\begin{expandable}
$\frac{1}{\sqrt{17}}
\left[ \begin{array}{r}
3 \\
-1
\end{array} \right]$
\end{expandable}

\end{enumerate}
\end{problem}

\begin{problem}\label{prob:inner_prod_4}
In each case, find the distance between $\vec{u}$ and $\vec{v}$.

\begin{enumerate} 
\item $\vec{u} = (3, -1, 2, 0)$, $\vec{v} = (1, 1, 1, 3);
\langle \vec{u}, \vec{v} \rangle = \vec{u} \dotp \vec{v}$

\item $\vec{u} = (1,  2, -1, 2)$, $\vec{v} = (2, 1, -1, 3);
\langle \vec{u}, \vec{v} \rangle = \vec{u} \dotp \vec{v}$

Click the arrow to see the answer.
\begin{expandable}
$\sqrt{3}$
\end{expandable}

\item $\vec{u} = f$, $\vec{v} = g $ in $\mathcal{C}[0, 1]$ where $f(x) = x^2 $ and $g(x) = 1 - x$; $\langle f, g \rangle = \int_{0}^{1} f(x)g(x)dx$

\item $\vec{u} = f$, $\vec{v} = g $ in $\mathcal{C}[-\pi, \pi]$ where $f(x) = 1$ and $g(x) = \cos x$; $\langle f, g \rangle = \int_{-\pi}^{\pi} f(x)g(x)dx$

Click the arrow to see the answer.
\begin{expandable}
$\sqrt{3\pi}$
\end{expandable}

\end{enumerate}
\end{problem}

\begin{problem}\label{ex:10_1_5}
Let $a_{1}, a_{2}, \dots, a_{n}$ be positive numbers. Given $\vec{v} = (v_{1}, v_{2}, \dots, v_{n})$ and $\vec{w} = (w_{1}, w_{2}, \dots, w_{n})$, define $\langle\vec{v}, \vec{w}\rangle = a_{1}v_{1}w_{1} + \dots + a_{n}v_{n}w_{n}$. Show that this is an inner product on $\RR^n$.
\end{problem}

\begin{problem}\label{prob:inner_prod_6}
If $\{\vec{b}_{1}, \dots, \vec{b}_{n}\}$ is a basis of $V$ and if
$\vec{v} = v_1\vec{b}_1 + \dots + v_n\vec{b}_n$ and
$\vec{w} = w_1\vec{b}_1 + \dots + w_n\vec{b}_n$ are vectors in $V$, define
\begin{equation*}
\langle \vec{v}, \vec{w} \rangle = v_1w_1 + \dots + v_nw_n .
\end{equation*}
Show that this is an inner product on $V$.
\end{problem}


\begin{problem}\label{prob:inner_prod_9}
Let $\mbox{re}(z)$ denote the real part of the complex number
$z$. Show that $\langle\ , \rangle$ is an inner product on $\mathbb{C}$ if $\langle\vec{z}, \vec{w}\rangle = \mbox{re}(z\overline{w})$.
\end{problem}

\begin{problem}\label{prob:inner_prod_10}
If $T : V \to V$ is an isomorphism of the inner product space $V$, show that
\begin{equation*}
\langle \vec{v}, \vec{w} \rangle_1 = \langle T(\vec{v}), T(\vec{w}) \rangle
\end{equation*}
defines a new inner product $\langle\ , \rangle_{1}$ on $V$.
\end{problem}

\begin{problem}\label{prob:inner_prod_11}
Show that every inner product $\langle\ , \rangle$ on $\RR^n$ has the form $\langle\vec{x}, \vec{y}\rangle = (U\vec{x}) \dotp (U\vec{y})$ for some upper triangular matrix $U$ with positive diagonal entries. [\textit{Hint}: Theorem~\ref{thm:024907}.]
\end{problem}

\begin{problem}\label{prob:inner_prod_12}
In each case, show that $\langle\vec{v}, \vec{w}\rangle = \vec{v}^{T}A\vec{w}$ defines an inner product on $\RR^2$ and hence show that $A$ is positive definite.
\begin{enumerate}
\item
$A =
\left[ \begin{array}{rr}
2 & 1 \\
1 & 1
\end{array} \right]$
\item
$A =
\left[ \begin{array}{rr}
5 & -3 \\
-3 & 2
\end{array} \right]$
\item
$A =
\left[ \begin{array}{rr}
3 & 2 \\
2 & 3
\end{array} \right]$
\item
$A =
\left[ \begin{array}{rr}
3 & 4 \\
4 & 6
\end{array} \right]$
\end{enumerate}
\begin{hint}
\begin{enumerate} 
 
\item  $  \langle \vec{v}, \vec{v} \rangle = 5v_1^2 - 6v_1v_2 + 2v_2^2 =
\frac{1}{5}[(5v_1 - 3v_2)^2 + v_2^2] $


\item  $ \langle \vec{v}, \vec{v} \rangle = 3v_1^2 + 8v_1v_2 + 6v_2^2 =
\frac{1}{3}[(3v_1 + 4v_2)^2 + 2v_2^2] $

\end{enumerate}
\end{hint}
\end{problem}

\begin{problem}\label{prob:inner_prod_13}
In each case, find a symmetric matrix $A$ such that $\langle\vec{v}, \vec{w}\rangle = \vec{v}^{T}A\vec{w}$.


\begin{enumerate} 
\item
$  \left\langle
\left[ \begin{array}{r}
v_1 \\
v_2
\end{array} \right], \left[ \begin{array}{r}
w_1 \\
w_2
\end{array} \right]
\right\rangle
= v_1w_1 + 2v_1w_2 + 2v_2w_1 + 5v_2w_2$

\item
$\left\langle
\left[ \begin{array}{r}
v_1 \\
v_2
\end{array} \right], \left[ \begin{array}{r}
w_1 \\
w_2
\end{array} \right]
\right\rangle
= v_1w_1 - v_1w_2 - v_2w_1 + 2v_2w_2$

\item
$\left\langle
\left[ \begin{array}{r}
v_1 \\
v_2 \\
v_3
\end{array} \right], \left[ \begin{array}{r}
w_1 \\
w_2 \\
w_3
\end{array} \right]
\right\rangle
= 2v_1w_1 + v_2w_2 + v_3w_3 - v_1w_2 \\ -v_2w_1 + v_2w_3 + v_3w_2$

\item
$\left\langle
\left[ \begin{array}{r}
v_1 \\
v_2 \\
v_3
\end{array} \right], \left[ \begin{array}{r}
w_1 \\
w_2 \\
w_3
\end{array} \right]
\right\rangle
= v_1w_1 + 2v_2w_2 + 5v_3w_3 \\ - 2v_1w_3 - 2v_3w_1$

\end{enumerate}
\begin{hint}
\begin{enumerate} 
 
\item $\left[ \begin{array}{rr}
1 & -2 \\
-2 & 1
\end{array} \right]$


\item $\left[ \begin{array}{rrr}
1 & 0 & -2 \\
0 & 2 & 0 \\
-2 & 0 & 5
\end{array} \right]$

\end{enumerate}
\end{hint}
\end{problem}

\begin{problem}\label{prob:inner_prod_14}
If $A$ is symmetric and $\vec{x}^{T}A\vec{x} = 0$ for all columns $\vec{x}$ in $\RR^n$, show that $A = 0$. [\textit{Hint}: Consider
$\langle \vec{x} + \vec{y}, \vec{x} + \vec{y} \rangle$ where $\langle \vec{x}, \vec{y} \rangle = \vec{x}^TA\vec{y}$.]

\begin{hint}
By the condition, $\langle \vec{x}, \vec{y} \rangle = \frac{1}{2} \langle \vec{x} + \vec{y}, \vec{x} + \vec{y} \rangle = 0$ for all $\vec{x}$, $\vec{y}$. Let $\vec{e}_{i}$ denote column $i$ of $I$. If $A = \left[ a_{ij} \right]$, then $a_{ij} = \vec{e}_{i}^{T}A\vec{e}_{j} = \{\vec{e}_{i}, \vec{e}_{j}\} = 0$ for all $i$ and $j$.
\end{hint}
\end{problem}

\begin{problem}\label{prob:inner_prod_15}
Show that the sum of two inner products on $V$ is again an inner product.
\end{problem}

\begin{problem} \label{ex:10_1_16}
Let $ \norm{ \vec{u} } = 1$, $\norm{ \vec{v} } = 2$, $\norm{ \vec{w} } = \sqrt{3} $, $\langle \vec{u}, \vec{v} \rangle = -1$, $\langle\vec{u}, \vec{w}\rangle = 0$ and $\langle\vec{v}, \vec{w}\rangle = 3$. Compute:
\begin{enumerate}
\item $\langle \vec{v} + \vec{w}, 2\vec{u} - \vec{v} \rangle$
\item $\langle \vec{u} - 2 \vec{v} - \vec{w}, 3\vec{w} - \vec{v} \rangle$
\end{enumerate}
\begin{hint}
\begin{enumerate} 

\item  $-15$

\end{enumerate}
\end{hint}
\end{problem}

\begin{problem}\label{prob:inner_prod_17}
Given the data in Practice Problem \ref{ex:10_1_16}, show that $\vec{u} + \vec{v} = \vec{w}$.
\end{problem}

\begin{problem}\label{prob:inner_prod_18}
Show that no vectors exist such that $\norm{\vec{u}} = 1$, $\norm{\vec{v}} = 2$, and $\langle\vec{u}, \vec{v}\rangle = -3$.
\end{problem}

\begin{problem} \label{ex:10_1_19}
Complete Example~\ref{exa:030310}.
\end{problem}

\begin{problem} \label{ex:10_1_20}
Prove Theorem~\ref{thm:030346}.

\begin{hint}
\dfn{1.} Using P2:
$ \langle \vec{u}, \vec{v} + \vec{w} \rangle =
\langle \vec{v} + \vec{w}, \vec{u} \rangle =
\langle \vec{v}, \vec{u} \rangle + \langle \vec{w}, \vec{u} \rangle =
\langle \vec{u}, \vec{v} \rangle + \langle \vec{u}, \vec{w} \rangle$.

\dfn{2.} Using P2 and P4:
$\langle \vec{v}, r\vec{w} \rangle =
\langle r\vec{w}, \vec{v} \rangle =
r \langle \vec{w}, \vec{v} \rangle =
r \langle \vec{v}, \vec{w} \rangle$.

\dfn{3.} Using P3:
$\langle \vec{0}, \vec{v} \rangle =
\langle \vec{0} + \vec{0}, \vec{v} \rangle =
\langle \vec{0}, \vec{v} \rangle + \langle \vec{0}, \vec{v} \rangle
$, so $ \langle \vec{0}, \vec{v} \rangle = 0$. The rest is P2.

\dfn{4.} Assume that $\langle \vec{v}, \vec{v} \rangle = 0$. If $\vec{v} \neq \vec{0}$ this contradicts P5, so $\vec{v} = \vec{0}$. Conversely, if $\vec{v} = \vec{0}$, then $\langle \vec{v}, \vec{v} \rangle = 0$ by Part 3 of this theorem.
\end{hint}
\end{problem}

\begin{problem} \label{ex:10_1_21}
Prove Theorem~\ref{thm:030545}.
\end{problem}

\begin{problem}\label{prob:inner_prod_22}
Let $\vec{u}$ and $\vec{v}$ be vectors in an inner product space $V$.

\begin{enumerate} 
\item Expand $\langle2\vec{u} - 7\vec{v}, 3\vec{u} + 5\vec{v} \rangle$.

\item Expand $\langle3\vec{u} - 4\vec{v}, 5\vec{u} + \vec{v} \rangle$.

\item Show that $\norm{ \vec{u} + \vec{v} } ^2 = \norm{ \vec{u} } ^2 + 2 \langle \vec{u}, \vec{v} \rangle + \norm{ \vec{v} } ^2 $.

\item Show that $\norm{ \vec{u} - \vec{v} } ^2 = \norm{ \vec{u} } ^2 - 2 \langle \vec{u}, \vec{v} \rangle +
\norm{ \vec{v} } ^2$.

\end{enumerate}
\begin{hint}
\begin{enumerate} 
 
\item  $15\norm{\vec{u}}^{2} - 17 \langle \vec{u}, \vec{v} \rangle - 4\norm{\vec{v}}^{2}$

\item  $\norm{\vec{u} + \vec{v}}^{2} = \langle \vec{u} + \vec{v}, \vec{u} + \vec{v} \rangle = \norm{\vec{u}}^{2} + 2\langle \vec{u}, \vec{v}\rangle + \norm{\vec{v}}^{2}$

\end{enumerate}
\end{hint}
\end{problem}

\begin{problem}\label{prob:inner_prod_23}
Show that
\begin{equation*}
\norm{ \vec{v} } ^2 +
\norm{ \vec{w} } ^2 = \frac{1}{2} \{
\norm{ \vec{v} + \vec{w} } ^2 +
\norm{ \vec{v} - \vec{w} } ^2\}
\end{equation*}
for any $\vec{v}$ and $\vec{w}$ in an inner product space.
\end{problem}

\begin{problem}\label{prob:inner_prod_24}
Let $\langle\ , \rangle$ be an inner product on a vector space $V$. Show that the corresponding distance function is translation invariant. That is, show that \newline $\mbox{d}(\vec{v}, \vec{w}) = \mbox{d}(\vec{v} + \vec{u}, \vec{w} + \vec{u})$ for all $\vec{v}$, $\vec{w}$, and $\vec{u}$ in $V$.
\end{problem}

\begin{problem}\label{prob:inner_prod_25}
\begin{enumerate} 
\item Show that $\langle \vec{u}, \vec{v} \rangle = \frac{1}{4}[\norm{ \vec{u} + \vec{v} } ^2 - \norm{ \vec{u} - \vec{v} } ^2]$ for all $\vec{u}$, $\vec{v}$ in an inner product space $V$.

\item If $\langle\ , \rangle$ and $\langle\ , \rangle^\prime$ are two inner products on $V$ that have equal associated norm functions, show that $\langle\vec{u}, \vec{v}\rangle = \langle\vec{u}, \vec{v}\rangle^\prime$ holds for all $\vec{u}$ and $\vec{v}$.

\end{enumerate}
\end{problem}

\begin{problem}\label{prob:inner_prod_26}
Let $\vec{v}$ denote a vector in an inner product space $V$.

\begin{enumerate} 
\item Show that $W = \{\vec{w} \mid \vec{w} \mbox{ in } V, \langle\vec{v}, \vec{w} = 0\}$ is a subspace of $V$.

\item Let $W$ be as in (a). If $V = \RR^3$ with the dot product, and if $\vec{v} = (1, -1, 2)$, find a basis for $W$.

\end{enumerate}
\begin{hint}
\begin{enumerate} 
 
\item  $\{(1, 1, 0), (0, 2, 1)\}$

\end{enumerate}
\end{hint}
\end{problem}

\begin{problem} \label{ex:10_1_27}
Given vectors $\vec{w}_{1}, \vec{w}_{2}, \dots, \vec{w}_{n}$ and $\vec{v}$, assume that $\langle\vec{v}, \vec{w}_{i}\rangle = 0$ for each $i$. Show that $\langle\vec{v}, \vec{w}\rangle = 0$ for all $\vec{w}$ in $\mbox{span}\{\vec{w}_{1}, \vec{w}_{2}, \dots, \vec{w}_{n}\}$.
\end{problem}

\begin{problem}\label{prob:inner_prod_28}
If $V = \mbox{span}\{\vec{v}_{1}, \vec{v}_{2}, \dots, \vec{v}_{n}\}$ and $\langle\vec{v}, \vec{v}_{i}\rangle = \langle\vec{w}, \vec{v}_i\rangle$ holds for each $i$. Show that $\vec{v} = \vec{w}$.

\begin{hint}
$\langle \vec{v} - \vec{w}, \vec{v}_{i} \rangle = \langle \vec{v}, \vec{v}_{i} \rangle - \langle \vec{w}, \vec{v}_{i} \rangle = 0$ for each $i$, so $\vec{v} = \vec{w}$ by Practice Problem \ref{ex:10_1_27}.
\end{hint}
\end{problem}

\begin{problem}\label{prob:inner_prod_29}
Use the Cauchy-Schwarz inequality in an inner product space to show that:

\begin{enumerate} 
\item If $\norm{\vec{u}} \leq 1$, then $\langle\vec{u}, \vec{v}\rangle^{2} \leq \norm{\vec{v}}^{2}$ for all $\vec{v}$ in $V$.

\item $(x \cos \theta + y \sin \theta)^{2} \leq x^{2} + y^{2}$ for all real $x$, $y$, and $\theta$.

\item $\norm{ r_1\vec{v}_1 + \dots + r_n\vec{v}_n } ^2 \leq [r_1 \norm{ \vec{v}_1 } + \dots + r_n \norm{ \vec{v}_n } ]^2$
for all vectors $\vec{v}_{i}$, and all $r_{i} > 0$ in $\RR$.

\end{enumerate}
\begin{hint}
\begin{enumerate} 
 
\item  If $\vec{u} = (\cos \theta, \sin \theta)$ in $\RR^2$ (with the dot product) then $\norm{\vec{u}} = 1$. Use \dfn{(a)} with $\vec{v} = (x, y)$.

\end{enumerate}
\end{hint}
\end{problem}

\begin{problem}\label{prob:inner_prod_30}
If $A$ is a $2 \times n$ matrix, let $\vec{u}$ and $\vec{v}$ denote the rows of $A$.

\begin{enumerate} 
\item Show that
$AA^T = \left[ \begin{array}{rr}
\norm{ \vec{u} } ^2 & \vec{u} \dotp \vec{v} \\
\vec{u} \dotp \vec{v} & \norm{ \vec{v} } ^2
\end{array} \right]$.

\item Show that $\mbox{det}(AA^{T}) \geq 0$.

\end{enumerate}
\end{problem}

\begin{problem} \label{ex:10_1_31}
\begin{enumerate} 
\item If $\vec{v}$ and $\vec{w}$ are nonzero vectors in an inner product space $V$, show that
$-1 \leq \frac{\langle \vec{v}, \vec{w} \rangle}{\norm{ \vec{v} } \norm{ \vec{w} }} \leq 1$, and hence that a unique angle $\theta$ exists such that \newline $\frac{\langle \vec{v}, \vec{w} \rangle}{\norm{ \vec{v} } \norm{ \vec{w} }} = \cos \theta$ and $0 \leq \theta \leq \pi$. This angle $\theta$ is called the
\dfn{angle between} $\vec{v}$ and $\vec{w}$.

\item Find the angle between $\vec{v} = (1, 2, -1, 1\, 3)$ and $\vec{w} = (2, 1, 0, 2, 0)$ in $\RR^5$ with the dot product.

\item If $\theta$ is the angle between $\vec{v}$ and $\vec{w}$, show that the \dfn{law of cosines} is valid:
\begin{equation*}
\norm{ \vec{v} - \vec{w} } = \norm{ \vec{v} } ^2 + \norm{ \vec{w} } ^2 - 2\norm{ \vec{v} } \norm{ \vec{w} } \cos \theta.
\end{equation*}
\end{enumerate}
\end{problem}

\begin{problem}\label{prob:inner_prod_32}
If $V = \RR^2$, define $\norm{(x, y)} = |x| + |y|$.

\begin{enumerate} 
\item Show that $\norm{\dotp}$ satisfies the conditions in Theorem~\ref{thm:030504}.

\item Show that $\norm{\dotp}$ does not arise from an inner product on $\RR^2$ given by a matrix $A$. 
\begin{hint}
    If it did, use Theorem~\ref{thm:030372} to find numbers $a$, $b$, and $c$ such that $\norm{(x, y)}^{2} = ax^{2} + bxy + cy^{2}$ for all $x$ and $y$.
\end{hint}

\end{enumerate}
\end{problem}

\section*{Text Source} This section was adapted from Section 10.1 of Keith Nicholson's \href{https://open.umn.edu/opentextbooks/textbooks/linear-algebra-with-applications}{\it Linear Algebra with Applications}. (CC-BY-NC-SA)

W. Keith Nicholson, {\it Linear Algebra with Applications}, Lyryx 2018, Open Edition, pp. 521--530.

\end{document}