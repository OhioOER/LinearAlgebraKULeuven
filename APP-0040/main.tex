\documentclass{ximera}
%%% Begin Laad packages

\makeatletter
\@ifclassloaded{xourse}{%
    \typeout{Start loading preamble.tex (in a XOURSE)}%
    \def\isXourse{true}   % automatically defined; pre 112022 it had to be set 'manually' in a xourse
}{%
    \typeout{Start loading preamble.tex (NOT in a XOURSE)}%
}
\makeatother

\def\isEn\true 

\pgfplotsset{compat=1.16}

\usepackage{currfile}

% 201908/202301: PAS OP: babel en doclicense lijken problemen te veroorzaken in .jax bestand
% (wegens syntax error met toegevoegde \newcommands ...)
\pdfOnly{
    \usepackage[type={CC},modifier={by-nc-sa},version={4.0}]{doclicense}
    %\usepackage[hyperxmp=false,type={CC},modifier={by-nc-sa},version={4.0}]{doclicense}
    %%% \usepackage[dutch]{babel}
}



\usepackage[utf8]{inputenc}
\usepackage{morewrites}   % nav zomercursus (answer...?)
\usepackage{multirow}
\usepackage{multicol}
\usepackage{tikzsymbols}
\usepackage{ifthen}
%\usepackage{animate} BREAKS HTML STRUCTURE USED BY XIMERA
\usepackage{relsize}

\usepackage{eurosym}    % \euro  (€ werkt niet in xake ...?)
\usepackage{fontawesome} % smileys etc

% Nuttig als ook interactieve beamer slides worden voorzien:
\providecommand{\p}{} % default nothing ; potentially usefull for slides: redefine as \pause
%providecommand{\p}{\pause}

    % Layout-parameters voor het onderschrift bij figuren
\usepackage[margin=10pt,font=small,labelfont=bf, labelsep=endash,format=hang]{caption}
%\usepackage{caption} % captionof
%\usepackage{pdflscape}    % landscape environment

% Met "\newcommand\showtodonotes{}" kan je todonotes tonen (in pdf/online)
% 201908: online werkt het niet (goed)
\providecommand\showtodonotes{disable}
\providecommand\todo[1]{\typeout{TODO #1}}
%\usepackage[\showtodonotes]{todonotes}
%\usepackage{todonotes}

%
% Poging tot aanpassen layout
%
\usepackage{tcolorbox}
\tcbuselibrary{theorems}

%%% Einde laad packages

%%% Begin Ximera specifieke zaken

\graphicspath{
	{../../}
	{../}
	{./}
  	{../../pictures/}
   	{../pictures/}
   	{./pictures/}
	{./explog/}    % M05 in groeimodellen       
}

%%% Einde Ximera specifieke zaken

%
% define softer blue/red/green, use KU Leuven base colors for blue (and dark orange for red ?)
%
% todo: rather redefine blue/red/green ...?
%\definecolor{xmblue}{rgb}{0.01, 0.31, 0.59}
%\definecolor{xmred}{rgb}{0.89, 0.02, 0.17}
\definecolor{xmdarkblue}{rgb}{0.122, 0.671, 0.835}   % KU Leuven Blauw
\definecolor{xmblue}{rgb}{0.114, 0.553, 0.69}        % KU Leuven Blauw
\definecolor{xmgreen}{rgb}{0.13, 0.55, 0.13}         % No KULeuven variant for green found ...

\definecolor{xmaccent}{rgb}{0.867, 0.541, 0.18}      % KU Leuven Accent (orange ...)
\definecolor{kuaccent}{rgb}{0.867, 0.541, 0.18}      % KU Leuven Accent (orange ...)

\colorlet{xmred}{xmaccent!50!black}                  % Darker version of KU Leuven Accent

\providecommand{\blue}[1]{{\color{blue}#1}}    
\providecommand{\red}[1]{{\color{red}#1}}

\renewcommand\CancelColor{\color{xmaccent!50!black}}

% werkt in math en text mode om MATH met oranje (of grijze...)  achtergond te tonen (ook \important{\text{blabla}} lijkt te werken)
%\newcommand{\important}[1]{\ensuremath{\colorbox{xmaccent!50!white}{$#1$}}}   % werkt niet in Mathjax
%\newcommand{\important}[1]{\ensuremath{\colorbox{lightgray}{$#1$}}}
\newcommand{\important}[1]{\ensuremath{\colorbox{orange}{$#1$}}}   % TODO: kleur aanpassen voor mathjax; wordt overschreven infra!


% Uitzonderlijk kan met \pdfnl in de PDF een newline worden geforceerd, die online niet nodig/nuttig is omdat daar de regellengte hoe dan ook niet gekend is.
\ifdefined\HCode%
\providecommand{\pdfnl}{}%
\else%
\providecommand{\pdfnl}{%
  \\%
}%
\fi

% Uitzonderlijk kan met \handoutnl in de handout-PDF een newline worden geforceerd, die noch online noch in de PDF-met-antwoorden nuttig is.
\ifdefined\HCode
\providecommand{\handoutnl}{}
\else
\providecommand{\handoutnl}{%
\ifhandout%
  \nl%
\fi%
}
\fi



% \cellcolor IGNORED by tex4ht ?
% \begin{center} seems not to wordk
    % (missing margin-left: auto;   on tabular-inside-center ???)
%\newcommand{\importantcell}[1]{\ensuremath{\cellcolor{lightgray}#1}}  %  in tabular; usablility to be checked
\providecommand{\importantcell}[1]{\ensuremath{#1}}     % no mathjax2 support for colloring array cells

\pdfOnly{
  \renewcommand{\important}[1]{\ensuremath{\colorbox{kuaccent!50!white}{$#1$}}}
  \renewcommand{\importantcell}[1]{\ensuremath{\cellcolor{kuaccent!40!white}#1}}   
}

%%% Tikz styles


\pgfplotsset{compat=1.16}

\usetikzlibrary{trees,positioning,arrows,fit,shapes,math,calc,decorations.markings,through,intersections,patterns,matrix}

\usetikzlibrary{decorations.pathreplacing,backgrounds}    % 5/2023: from experimental


\usetikzlibrary{angles,quotes}

\usepgfplotslibrary{fillbetween} % bepaalde_integraal
\usepgfplotslibrary{polar}    % oa voor poolcoordinaten.tex

\pgfplotsset{ownstyle/.style={axis lines = center, axis equal image, xlabel = $x$, ylabel = $y$, enlargelimits}} 

\pgfplotsset{
	plot/.style={no marks,samples=50}
}

\newcommand{\xmPlotsColor}{
	\pgfplotsset{
		plot1/.style={darkgray,no marks,samples=100},
		plot2/.style={lightgray,no marks,samples=100},
		plotresult/.style={blue,no marks,samples=100},
		plotblue/.style={blue,no marks,samples=100},
		plotred/.style={red,no marks,samples=100},
		plotgreen/.style={green,no marks,samples=100},
		plotpurple/.style={purple,no marks,samples=100}
	}
}
\newcommand{\xmPlotsBlackWhite}{
	\pgfplotsset{
		plot1/.style={black,loosely dashed,no marks,samples=100},
		plot2/.style={black,loosely dotted,no marks,samples=100},
		plotresult/.style={black,no marks,samples=100},
		plotblue/.style={black,no marks,samples=100},
		plotred/.style={black,dotted,no marks,samples=100},
		plotgreen/.style={black,dashed,no marks,samples=100},
		plotpurple/.style={black,dashdotted,no marks,samples=100}
	}
}


\newcommand{\xmPlotsColorAndStyle}{
	\pgfplotsset{
		plot1/.style={darkgray,no marks,samples=100},
		plot2/.style={lightgray,no marks,samples=100},
		plotresult/.style={blue,no marks,samples=100},
		plotblue/.style={xmblue,no marks,samples=100},
		plotred/.style={xmred,dashed,thick,no marks,samples=100},
		plotgreen/.style={xmgreen,dotted,very thick,no marks,samples=100},
		plotpurple/.style={purple,no marks,samples=100}
	}
}


%\iftikzexport
\xmPlotsColorAndStyle
%\else
%\xmPlotsBlackWhite
%\fi
%%%


%
% Om venndiagrammen te arceren ...
%
\makeatletter
\pgfdeclarepatternformonly[\hatchdistance,\hatchthickness]{north east hatch}% name
{\pgfqpoint{-1pt}{-1pt}}% below left
{\pgfqpoint{\hatchdistance}{\hatchdistance}}% above right
{\pgfpoint{\hatchdistance-1pt}{\hatchdistance-1pt}}%
{
	\pgfsetcolor{\tikz@pattern@color}
	\pgfsetlinewidth{\hatchthickness}
	\pgfpathmoveto{\pgfqpoint{0pt}{0pt}}
	\pgfpathlineto{\pgfqpoint{\hatchdistance}{\hatchdistance}}
	\pgfusepath{stroke}
}
\pgfdeclarepatternformonly[\hatchdistance,\hatchthickness]{north west hatch}% name
{\pgfqpoint{-\hatchthickness}{-\hatchthickness}}% below left
{\pgfqpoint{\hatchdistance+\hatchthickness}{\hatchdistance+\hatchthickness}}% above right
{\pgfpoint{\hatchdistance}{\hatchdistance}}%
{
	\pgfsetcolor{\tikz@pattern@color}
	\pgfsetlinewidth{\hatchthickness}
	\pgfpathmoveto{\pgfqpoint{\hatchdistance+\hatchthickness}{-\hatchthickness}}
	\pgfpathlineto{\pgfqpoint{-\hatchthickness}{\hatchdistance+\hatchthickness}}
	\pgfusepath{stroke}
}
%\makeatother

\tikzset{
    hatch distance/.store in=\hatchdistance,
    hatch distance=10pt,
    hatch thickness/.store in=\hatchthickness,
   	hatch thickness=2pt
}

\colorlet{circle edge}{black}
\colorlet{circle area}{blue!20}


\tikzset{
    filled/.style={fill=green!30, draw=circle edge, thick},
    arceerl/.style={pattern=north east hatch, pattern color=blue!50, draw=circle edge},
    arceerr/.style={pattern=north west hatch, pattern color=yellow!50, draw=circle edge},
    outline/.style={draw=circle edge, thick}
}




%%% Updaten commando's
\def\hoofding #1#2#3{\maketitle}     % OBSOLETE ??

% we willen (bijna) altijd \geqslant ipv \geq ...!
\newcommand{\geqnoslant}{\geq}
\renewcommand{\geq}{\geqslant}
\newcommand{\leqnoslant}{\leq}
\renewcommand{\leq}{\leqslant}

% Todo: (201908) waarom komt er (soms) underlined voor emph ...?
\renewcommand{\emph}[1]{\textit{#1}}

% API commando's

\newcommand{\ds}{\displaystyle}
\newcommand{\ts}{\textstyle}  % tegenhanger van \ds   (Ximera zet PER  DEFAULT \ds!)

% uit Zomercursus-macro's: 
\newcommand{\bron}[1]{\begin{scriptsize} \emph{#1} \end{scriptsize}}     % deprecated ...?


%definities nieuwe commando's - afkortingen veel gebruikte symbolen
\newcommand{\R}{\ensuremath{\mathbb{R}}}
\newcommand{\Rnul}{\ensuremath{\mathbb{R}_0}}
\newcommand{\Reen}{\ensuremath{\mathbb{R}\setminus\{1\}}}
\newcommand{\Rnuleen}{\ensuremath{\mathbb{R}\setminus\{0,1\}}}
\newcommand{\Rplus}{\ensuremath{\mathbb{R}^+}}
\newcommand{\Rmin}{\ensuremath{\mathbb{R}^-}}
\newcommand{\Rnulplus}{\ensuremath{\mathbb{R}_0^+}}
\newcommand{\Rnulmin}{\ensuremath{\mathbb{R}_0^-}}
\newcommand{\Rnuleenplus}{\ensuremath{\mathbb{R}^+\setminus\{0,1\}}}
\newcommand{\N}{\ensuremath{\mathbb{N}}}
\newcommand{\Nnul}{\ensuremath{\mathbb{N}_0}}
\newcommand{\Z}{\ensuremath{\mathbb{Z}}}
\newcommand{\Znul}{\ensuremath{\mathbb{Z}_0}}
\newcommand{\Zplus}{\ensuremath{\mathbb{Z}^+}}
\newcommand{\Zmin}{\ensuremath{\mathbb{Z}^-}}
\newcommand{\Znulplus}{\ensuremath{\mathbb{Z}_0^+}}
\newcommand{\Znulmin}{\ensuremath{\mathbb{Z}_0^-}}
\newcommand{\C}{\ensuremath{\mathbb{C}}}
\newcommand{\Cnul}{\ensuremath{\mathbb{C}_0}}
\newcommand{\Cplus}{\ensuremath{\mathbb{C}^+}}
\newcommand{\Cmin}{\ensuremath{\mathbb{C}^-}}
\newcommand{\Cnulplus}{\ensuremath{\mathbb{C}_0^+}}
\newcommand{\Cnulmin}{\ensuremath{\mathbb{C}_0^-}}
\newcommand{\Q}{\ensuremath{\mathbb{Q}}}
\newcommand{\Qnul}{\ensuremath{\mathbb{Q}_0}}
\newcommand{\Qplus}{\ensuremath{\mathbb{Q}^+}}
\newcommand{\Qmin}{\ensuremath{\mathbb{Q}^-}}
\newcommand{\Qnulplus}{\ensuremath{\mathbb{Q}_0^+}}
\newcommand{\Qnulmin}{\ensuremath{\mathbb{Q}_0^-}}

\newcommand{\perdef}{\overset{\mathrm{def}}{=}}
\newcommand{\pernot}{\overset{\mathrm{notatie}}{=}}
\newcommand\perinderdaad{\overset{!}{=}}     % voorlopig gebruikt in limietenrekenregels
\newcommand\perhaps{\overset{?}{=}}          % voorlopig gebruikt in limietenrekenregels

\newcommand{\degree}{^\circ}


\DeclareMathOperator{\dom}{dom}     % domein
\DeclareMathOperator{\codom}{codom} % codomein
\DeclareMathOperator{\bld}{bld}     % beeld
\DeclareMathOperator{\graf}{graf}   % grafiek
\DeclareMathOperator{\rico}{rico}   % richtingcoëfficient
\DeclareMathOperator{\co}{co}       % coordinaat
\DeclareMathOperator{\gr}{gr}       % graad

\newcommand{\func}[5]{\ensuremath{#1: #2 \rightarrow #3: #4 \mapsto #5}} % Easy to write a function


% Operators
\DeclareMathOperator{\bgsin}{bgsin}
\DeclareMathOperator{\bgcos}{bgcos}
\DeclareMathOperator{\bgtan}{bgtan}
\DeclareMathOperator{\bgcot}{bgcot}
\DeclareMathOperator{\bgsinh}{bgsinh}
\DeclareMathOperator{\bgcosh}{bgcosh}
\DeclareMathOperator{\bgtanh}{bgtanh}
\DeclareMathOperator{\bgcoth}{bgcoth}

% Oude \Bgsin etc deprecated: gebruik \bgsin, en herdefinieer dat als je Bgsin wil!
%\DeclareMathOperator{\cosec}{cosec}    % not used? gebruik \csc en herdefinieer

% operatoren voor differentialen: to be verified; 1/2020: inconsequent gebruik bij afgeleiden/integralen
\renewcommand{\d}{\mathrm{d}}
\newcommand{\dx}{\d x}
\newcommand{\dd}[1]{\frac{\mathrm{d}}{\mathrm{d}#1}}
\newcommand{\ddx}{\dd{x}}

% om in voorbeelden/oefeningen de notatie voor afgeleiden te kunnen kiezen
% Usage: \afg{(2\sin(x))}  (en wordt d/dx, of accent, of D )
%\newcommand{\afg}[1]{{#1}'}
\newcommand{\afg}[1]{\left(#1\right)'}
%\renewcommand{\afg}[1]{\frac{\mathrm{d}#1}{\mathrm{d}x}}   % include in relevant exercises ...
%\renewcommand{\afg}[1]{D{#1}}

%
% \xmxxx commands: Extra KU Leuven functionaliteit van, boven of naast Ximera
%   ( Conventie 8/2019: xm+nederlandse omschrijving, maar is niet consequent gevolgd, en misschien ook niet erg handig !)
%
% (Met een minimale ximera.cls en preamble.tex zou een bruikbare .pdf moeten kunnen worden gemaakt van eender welke ximera)
%
% Usage: \xmtitle[Mijn korte abstract]{Mijn titel}{Mijn abstract}
% Eerste command na \begin{document}:
%  -> definieert de \title
%  -> definieert de abstract
%  -> doet \maketitle ( dus: print de hoofding als 'chapter' of 'sectie')
% Optionele parameter geeft eenn kort abstract (die met de globale setting \xmshortabstract{} al dan niet kan worden geprint.
% De optionele korte abstract kan worden gebruikt voor pseudo-grappige abtsarts, dus dus globaal al dan niet kunnen worden gebuikt...
% Globale settings:
%  de (optionele) 'korte abstract' wordt enkele getoond als \xmshortabstract is gezet
\providecommand\xmshortabstract{} % default: print (only!) short abstract if present
\newcommand{\xmtitle}[3][]{
	\title{#2}
	\begin{abstract}
		\ifdefined\xmshortabstract
		\ifstrempty{#1}{%
			#3
		}{%
			#1
		}%
		\else
		#3
		\fi
	\end{abstract}
	\maketitle
}

% 
% Kleine grapjes: moeten zonder verder gevolg kunnen worden verwijderd
%
%\newcommand{\xmopje}[1]{{\small#1{\reversemarginpar\marginpar{\Smiley}}}}   % probleem in floats!!
\newtoggle{showxmopje}
\toggletrue{showxmopje}

\newcommand{\xmopje}[1]{%
   \iftoggle{showxmopje}{#1}{}%
}


% -> geef een abstracte-formule-met-rechts-een-concreet-voorbeeld
% VB:  \formulevb{a^2+b^2=c^2}{3^2+4^2=5^2}
%
\ifdefined\HCode
\NewEnviron{xmdiv}[1]{\HCode{\Hnewline<div class="#1">\Hnewline}\BODY{\HCode{\Hnewline</div>\Hnewline}}}
\else
\NewEnviron{xmdiv}[1]{\BODY}
\fi

\providecommand{\formulevb}[2]{
	{\centering

    \begin{xmdiv}{xmformulevb}    % zie css voor online layout !!!
	\begin{tabular}{lcl}
		\important{#1}
		&  &
		Vb: $#2$
		\end{tabular}
	\end{xmdiv}

	}
}

\ifdefined\HCode
\providecommand{\vb}[1]{%
    \HCode{\Hnewline<span class="xmvb">}#1\HCode{</span>\Hnewline}%
}
\else
\providecommand{\vb}[1]{
    \colorbox{blue!10}{#1}
}
\fi

\ifdefined\HCode
\providecommand{\xmcolorbox}[2]{
	\HCode{\Hnewline<div class="xmcolorbox">\Hnewline}#2\HCode{\Hnewline</div>\Hnewline}
}
\else
\providecommand{\xmcolorbox}[2]{
  \cellcolor{#1}#2
}
\fi


\ifdefined\HCode
\providecommand{\xmopmerking}[1]{
 \HCode{\Hnewline<div class="xmopmerking">\Hnewline}#1\HCode{\Hnewline</div>\Hnewline}
}
\else
\providecommand{\xmopmerking}[1]{
	{\footnotesize #1}
}
\fi
% \providecommand{\voorbeeld}[1]{
% 	\colorbox{blue!10}{$#1$}
% }



% Hernoem Proof naar Bewijs, nodig voor HTML versie
\renewcommand*{\proofname}{Bewijs}

% Om opgave van oefening (wordt niet geprint bij oplossingenblad)
% (to be tested test)
\NewEnviron{statement}{\BODY}

% Environment 'oplossing' en 'uitkomst'
% voor resp. volledige 'uitwerking' dan wel 'enkel eindresultaat'
% geimplementeerd via feedback, omdat er in de ximera-server adhoc feedback-code is toegevoegd
%% Niet tonen indien handout
%% Te gebruiken om volledige oplossingen/uitwerkingen van oefeningen te tonen
%% \begin{oplossing}        De optelling is commutatief \end{oplossing}  : verschijnt online enkel 'op vraag'
%% \begin{oplossing}[toon]  De optelling is commutatief \end{oplossing}  : verschijnt steeds onmiddellijk online (bv te gebruiken bij voorbeelden) 

\ifhandout%
    \NewEnviron{oplossing}[1][onzichtbaar]%
    {%
    \ifthenelse{\equal{\detokenize{#1}}{\detokenize{toon}}}
    {
    \def\PH@Command{#1}% Use PH@Command to hold the content and be a target for "\expandafter" to expand once.

    \begin{trivlist}% Begin the trivlist to use formating of the "Feedback" label.
    \item[\hskip \labelsep\small\slshape\bfseries Oplossing% Format the "Feedback" label. Don't forget the space.
    %(\texttt{\detokenize\expandafter{\PH@Command}}):% Format (and detokenize) the condition for feedback to trigger
    \hspace{2ex}]\small%\slshape% Insert some space before the actual feedback given.
    \BODY
    \end{trivlist}
    }
    {  % \begin{feedback}[solution]   \BODY     \end{feedback}  }
    }
    }    
\else
% ONLY for HTML; xmoplossing is styled with css, and is not, and need not be a LaTeX environment
% THUS: it does NOT use feedback anymore ...
%    \NewEnviron{oplossing}{\begin{expandable}{xmoplossing}{\nlen{Toon uitwerking}{Show solution}}{\BODY}\end{expandable}}
    \newenvironment{oplossing}[1][onzichtbaar]
   {%
       \begin{expandable}{xmoplossing}{}
   }
   {%
   	   \end{expandable}
   } 
%     \newenvironment{oplossing}[1][onzichtbaar]
%    {%
%        \begin{feedback}[solution]   	
%    }
%    {%
%    	   \end{feedback}
%    } 
\fi

\ifhandout%
    \NewEnviron{uitkomst}[1][onzichtbaar]%
    {%
    \ifthenelse{\equal{\detokenize{#1}}{\detokenize{toon}}}
    {
    \def\PH@Command{#1}% Use PH@Command to hold the content and be a target for "\expandafter" to expand once.

    \begin{trivlist}% Begin the trivlist to use formating of the "Feedback" label.
    \item[\hskip \labelsep\small\slshape\bfseries Uitkomst:% Format the "Feedback" label. Don't forget the space.
    %(\texttt{\detokenize\expandafter{\PH@Command}}):% Format (and detokenize) the condition for feedback to trigger
    \hspace{2ex}]\small%\slshape% Insert some space before the actual feedback given.
    \BODY
    \end{trivlist}
    }
    {  % \begin{feedback}[solution]   \BODY     \end{feedback}  }
    }
    }    
\else
\ifdefined\HCode
   \newenvironment{uitkomst}[1][onzichtbaar]
    {%
        \begin{expandable}{xmuitkomst}{}%
    }
    {%
    	\end{expandable}%
    } 
\else
  % Do NOT print 'uitkomst' in non-handout
  %  (presumably, there is also an 'oplossing' ??)
  \newenvironment{uitkomst}[1][onzichtbaar]{}{}
\fi
\fi

%
% Uitweidingen zijn extra's die niet redelijkerwijze tot de leerstof behoren
% Uitbreidingen zijn extra's die wel redelijkerwijze tot de leerstof van bv meer geavanceerde versies kunnen behoren (B-programma/Wiskundestudenten/...?)
% Nog niet voorzien: design voor verschillende versies (A/B programma, BIO, voorkennis/ ...)
% Voor 'uitweidingen' is er een environment die online per default is ingeklapt, en in pdf al dan niet kan worden geincluded  (via \xmnouitweiding) 
%
% in een xourse, per default GEEN uitweidingen, tenzij \xmuitweiding expliciet ergens is gezet ...
\ifdefined\isXourse
   \ifdefined\xmuitweiding
   \else
       \def\xmnouitweiding{true}
   \fi
\fi

\ifdefined\xmnouitweiding
\newcounter{xmuitweiding}  % anders error undefined ...  
\excludecomment{xmuitweiding}
\else
\newtheoremstyle{dotless}{}{}{}{}{}{}{ }{}
\theoremstyle{dotless}
\newtheorem*{xmuitweidingnofrills}{}   % nofrills = no accordion; gebruikt dus de dotless theoremstyle!

\newcounter{xmuitweiding}
\newenvironment{xmuitweiding}[1][ ]%
{% 
	\refstepcounter{xmuitweiding}%
    \begin{expandable}{xmuitweiding}{\nlentext{Uitweiding \arabic{xmuitweiding}: #1}{Digression \arabic{xmuitweiding}: #1}}%
	\begin{xmuitweidingnofrills}%
}
{%
    \end{xmuitweidingnofrills}%
    \end{expandable}%
}   
% \newenvironment{xmuitweiding}[1][ ]%
% {% 
% 	\refstepcounter{xmuitweiding}
% 	\begin{accordion}\begin{accordion-item}[Uitweiding \arabic{xmuitweiding}: #1]%
% 			\begin{xmuitweidingnofrills}%
% 			}
% 			{\end{xmuitweidingnofrills}\end{accordion-item}\end{accordion}}   
\fi


\newenvironment{xmexpandable}[1][]{
	\begin{accordion}\begin{accordion-item}[#1]%
		}{\end{accordion-item}\end{accordion}}


% Command that gives a selection box online, but just prints the right answer in pdf
\newcommand{\xmonlineChoice}[1]{\pdfOnly{\wordchoicegiventrue}\wordChoice{#1}\pdfOnly{\wordchoicegivenfalse}}   % deprecated, gebruik onlineChoice ...
\newcommand{\onlineChoice}[1]{\pdfOnly{\wordchoicegiventrue}\wordChoice{#1}\pdfOnly{\wordchoicegivenfalse}}

\newcommand{\TJa}{\nlentext{ Ja }{ Yes }}
\newcommand{\TNee}{\nlentext{ Nee }{ No }}
\newcommand{\TJuist}{\nlentext{ Juist }{ True }}
\newcommand{\TFout}{\nlentext{ Fout }{ False }}

\newcommand{\choiceTrue }{{\renewcommand{\choiceminimumhorizontalsize}{4em}\wordChoice{\choice[correct]{\TJuist}\choice{\TFout}}}}
\newcommand{\choiceFalse}{{\renewcommand{\choiceminimumhorizontalsize}{4em}\wordChoice{\choice{\TJuist}\choice[correct]{\TFout}}}}

\newcommand{\choiceYes}{{\renewcommand{\choiceminimumhorizontalsize}{3em}\wordChoice{\choice[correct]{\TJa}\choice{\TNee}}}}
\newcommand{\choiceNo }{{\renewcommand{\choiceminimumhorizontalsize}{3em}\wordChoice{\choice{\TJa}\choice[correct]{\TNee}}}}

% Optional nicer formatting for wordChoice in PDF

\let\inlinechoiceorig\inlinechoice

%\makeatletter
%\providecommand{\choiceminimumverticalsize}{\vphantom{$\frac{\sqrt{2}}{2}$}}   % minimum height of boxes (cfr infra)
\providecommand{\choiceminimumverticalsize}{\vphantom{$\tfrac{2}{2}$}}   % minimum height of boxes (cfr infra)
\providecommand{\choiceminimumhorizontalsize}{1em}   % minimum width of boxes (cfr infra)

\newcommand{\inlinechoicesquares}[2][]{%
		\setkeys{choice}{#1}%
		\ifthenelse{\boolean{\choice@correct}}%
		{%
            \ifhandout%
               \fbox{\choiceminimumverticalsize #2}\allowbreak\ignorespaces%
            \else%
               \fcolorbox{blue}{blue!20}{\choiceminimumverticalsize #2}\allowbreak\ignorespaces\setkeys{choice}{correct=false}\ignorespaces%
            \fi%
		}%
		{% else
			\fbox{\choiceminimumverticalsize #2}\allowbreak\ignorespaces%  HACK: wat kleiner, zodat fits on line ... 	
		}%
}

\newcommand{\inlinechoicesquareX}[2][]{%
		\setkeys{choice}{#1}%
		\ifthenelse{\boolean{\choice@correct}}%
		{%
            \ifhandout%
               \framebox[\ifdim\choiceminimumhorizontalsize<\width\width\else\choiceminimumhorizontalsize\fi]{\choiceminimumverticalsize\ #2\ }\allowbreak\ignorespaces\setkeys{choice}{correct=false}\ignorespaces%
            \else%
               \fcolorbox{blue}{blue!20}{\makebox[\ifdim\choiceminimumhorizontalsize<\width\width\else\choiceminimumhorizontalsize\fi]{\choiceminimumverticalsize #2}}\allowbreak\ignorespaces\setkeys{choice}{correct=false}\ignorespaces%
            \fi%
		}%
		{% else
        \ifhandout%
			\framebox[\ifdim\choiceminimumhorizontalsize<\width\width\else\choiceminimumhorizontalsize\fi]{\choiceminimumverticalsize\ #2\ }\allowbreak\ignorespaces%  HACK: wat kleiner, zodat fits on line ... 	
        \fi
		}%
}


\newcommand{\inlinechoicepuntjes}[2][]{%
		\setkeys{choice}{#1}%
		\ifthenelse{\boolean{\choice@correct}}%
		{%
            \ifhandout%
               \dots\ldots\ignorespaces\setkeys{choice}{correct=false}\ignorespaces
            \else%
               \fcolorbox{blue}{blue!20}{\choiceminimumverticalsize #2}\allowbreak\ignorespaces\setkeys{choice}{correct=false}\ignorespaces%
            \fi%
		}%
		{% else
			%\fbox{\choiceminimumverticalsize #2}\allowbreak\ignorespaces%  HACK: wat kleiner, zodat fits on line ... 	
		}%
}

% print niets, maar definieer globale variable \myanswer
%  (gebruikt om oplossingsbladen te printen) 
\newcommand{\inlinechoicedefanswer}[2][]{%
		\setkeys{choice}{#1}%
		\ifthenelse{\boolean{\choice@correct}}%
		{%
               \gdef\myanswer{#2}\setkeys{choice}{correct=false}

		}%
		{% else
			%\fbox{\choiceminimumverticalsize #2}\allowbreak\ignorespaces%  HACK: wat kleiner, zodat fits on line ... 	
		}%
}



%\makeatother

\newcommand{\setchoicedefanswer}{
\ifdefined\HCode
\else
%    \renewenvironment{multipleChoice@}[1][]{}{} % remove trailing ')'
    \let\inlinechoice\inlinechoicedefanswer
\fi
}

\newcommand{\setchoicepuntjes}{
\ifdefined\HCode
\else
    \renewenvironment{multipleChoice@}[1][]{}{} % remove trailing ')'
    \let\inlinechoice\inlinechoicepuntjes
\fi
}
\newcommand{\setchoicesquares}{
\ifdefined\HCode
\else
    \renewenvironment{multipleChoice@}[1][]{}{} % remove trailing ')'
    \let\inlinechoice\inlinechoicesquares
\fi
}
%
\newcommand{\setchoicesquareX}{
\ifdefined\HCode
\else
    \renewenvironment{multipleChoice@}[1][]{}{} % remove trailing ')'
    \let\inlinechoice\inlinechoicesquareX
\fi
}
%
\newcommand{\setchoicelist}{
\ifdefined\HCode
\else
    \renewenvironment{multipleChoice@}[1][]{}{)}% re-add trailing ')'
    \let\inlinechoice\inlinechoiceorig
\fi
}

\setchoicesquareX  % by default list-of-squares with onlineChoice in PDF

% Omdat multicols niet werkt in html: enkel in pdf  (in html zijn langere pagina's misschien ook minder storend)
\newenvironment{xmmulticols}[1][2]{
 \pdfOnly{\begin{multicols}{#1}}%
}{ \pdfOnly{\end{multicols}}}

%
% Te gebruiken in plaats van \section\subsection
%  (in een printstyle kan dan het level worden aangepast
%    naargelang \chapter vs \section style )
% 3/2021: DO NOT USE \xmsubsection !
\newcommand\xmsection\subsection
\newcommand\xmsubsection\subsubsection

% Aanpassen printversie
%  (hier gedefinieerd, zodat ze in xourse kunnen worden gezet/overschreven)
\providebool{parttoc}
\providebool{printpartfrontpage}
\providebool{printactivitytitle}
\providebool{printactivityqrcode}
\providebool{printactivityurl}
\providebool{printcontinuouspagenumbers}
\providebool{numberactivitiesbysubpart}
\providebool{addtitlenumber}
\providebool{addsectiontitlenumber}
\addtitlenumbertrue
\addsectiontitlenumbertrue

% The following three commands are hardcoded in xake, you can't create other commands like these, without adding them to xake as well
%  ( gebruikt in xourses om juiste soort titelpagina te krijgen voor verschillende ximera's )
\newcommand{\activitychapter}[2][]{
    {    
    \ifstrequal{#1}{notnumbered}{
        \addtitlenumberfalse
    }{}
    \typeout{ACTIVITYCHAPTER #2}   % logging
	\chapterstyle
	\activity{#2}
    }
}
\newcommand{\activitysection}[2][]{
    {
    \ifstrequal{#1}{notnumbered}{
        \addsectiontitlenumberfalse
    }{}
	\typeout{ACTIVITYSECTION #2}   % logging
	\sectionstyle
	\activity{#2}
    }
}
% Practices worden als activity getoond om de grote blokken te krijgen online
\newcommand{\practicesection}[2][]{
    {
    \ifstrequal{#1}{notnumbered}{
        \addsectiontitlenumberfalse
    }{}
    \typeout{PRACTICESECTION #2}   % logging
	\sectionstyle
	\activity{#2}
    }
}
\newcommand{\activitychapterlink}[3][]{
    {
    \ifstrequal{#1}{notnumbered}{
        \addtitlenumberfalse
    }{}
    \typeout{ACTIVITYCHAPTERLINK #3}   % logging
	\chapterstyle
	\activitylink[#1]{#2}{#3}
    }
}

\newcommand{\activitysectionlink}[3][]{
    {
    \ifstrequal{#1}{notnumbered}{
        \addsectiontitlenumberfalse
    }{}
    \typeout{ACTIVITYSECTIONLINK #3}   % logging
	\sectionstyle
	\activitylink[#1]{#2}{#3}
    }
}


% Commando om de printstyle toe te voegen in ximera's. Zorgt ervoor dat er geen problemen zijn als je de xourses compileert
% hack om onhandige relative paden in TeX te omzeilen
% should work both in xourse and ximera (pre-112022 only in ximera; thus obsoletes adhoc setup in xourses)
% loads global.sty if present (cfr global.css for online settings!)
% use global.sty to overwrite settings in printstyle.sty ...
\newcommand{\addPrintStyle}[1]{
\iftikzexport\else   % only in PDF
  \makeatletter
  \ifx\@onlypreamble\@notprerr\else   % ONLY if in tex-preamble   (and e.g. not when included from xourse)
    \typeout{Loading printstyle}   % logging
    \usepackage{#1/printstyle} % mag enkel geinclude worden als je die apart compileert
    \IfFileExists{#1/global.sty}{
        \typeout{Loading printstyle-folder #1/global.sty}   % logging
        \usepackage{#1/global}
        }{
        \typeout{Info: No extra #1/global.sty}   % logging
    }   % load global.sty if present
    \IfFileExists{global.sty}{
        \typeout{Loading local-folder global.sty (or TEXINPUTPATH..)}   % logging
        \usepackage{global}
    }{
        \typeout{Info: No folder/global.sty}   % logging
    }   % load global.sty if present
    \IfFileExists{\currfilebase.sty}
    {
        \typeout{Loading \currfilebase.sty}
        \input{\currfilebase.sty}
    }{
        \typeout{Info: No local \currfilebase.sty}
    }
    \fi
  \makeatother
\fi
}

%
%  
% references: Ximera heeft adhoc logica	 om online labels te doen werken over verschillende files heen
% met \hyperref kan de getoonde tekst toch worden opgegeven, in plaats van af te hangen van de label-text
\ifdefined\HCode
% Link to standard \labels, but give your own description
% Usage:  Volg \hyperref[my_very_verbose_label]{deze link} voor wat tijdverlies
%   (01/2020: Ximera-server aangepast om bij class reference-keeptext de link-text NIET te vervangen door de label-text !!!) 
\renewcommand{\hyperref}[2][]{\HCode{<a class="reference reference-keeptext" href="\##1">}#2\HCode{</a>}}
%
%  Link to specific targets  (not tested ?)
\renewcommand{\hypertarget}[1]{\HCode{<a class="ximera-label" id="#1"></a>}}
\renewcommand{\hyperlink}[2]{\HCode{<a class="reference reference-keeptext" href="\##1">}#2\HCode{</a>}}
\fi

% Mmm, quid English ... (for keyword #1 !) ?
\newcommand{\wikilink}[2]{
    \hyperlink{https://nl.wikipedia.org/wiki/#1}{#2}
    \pdfOnly{\footnote{See \url{https://nl.wikipedia.org/wiki/#1}}
    }
}

\renewcommand{\figurename}{Figuur}
\renewcommand{\tablename}{Tabel}

%
% Gedoe om verschillende versies van xourse/ximera te maken afhankelijk van settings
%
% default: versie met antwoorden
% handout: versie voor de studenten, zonder antwoorden/oplossingen
% full: met alles erop en eraan, dus geschikt voor auteurs en/of lesgevers  (bevat in de pdf ook de 'online-only' stukken!)
%
%
% verder kunnen ook opties/variabele worden gezet voor hints/auteurs/uitweidingen/ etc
%
% 'Full' versie
\newtoggle{showonline}
\ifdefined\HCode   % zet default showOnline
    \toggletrue{showonline} 
\else
    \togglefalse{showonline}
\fi

% Full versie   % deprecated: see infra
\newcommand{\printFull}{
    \hintstrue
    \handoutfalse
    \toggletrue{showonline} 
}

\ifdefined\shouldPrintFull   % deprecated: see infra
    \printFull
\fi



% Overschrijf onlineOnly  (zoals gedefinieerd in ximera.cls)
\ifhandout   % in handout: gebruik de oorspronkelijke ximera.cls implementatie  (is dit wel nodig/nuttig?)
\else
    \iftoggle{showonline}{%
        \ifdefined\HCode
          \RenewEnviron{onlineOnly}{\bgroup\BODY\egroup}   % showOnline, en we zijn  online, dus toon de tekst
        \else
          \RenewEnviron{onlineOnly}{\bgroup\color{red!50!black}\BODY\egroup}   % showOnline, maar we zijn toch niet online: kleur de tekst rood 
        \fi
    }{%
      \RenewEnviron{onlineOnly}{}  % geen showOnline
    }
\fi

% hack om na hoofding van definition/proposition/... als dan niet op een nieuwe lijn te starten
% soms is dat goed en mooi, en soms niet; en in HTML is het nu (2/2020) anders dan in pdf
% vandaar suggestie om 
%     \begin{definition}[Nieuw concept] \nl
% te gebruiken als je zeker een newline wil na de hoofdig en titel
% (in het bijzonder itemize zonder \nl is 'lelijk' ...)
\ifdefined\HCode
\newcommand{\nl}{}
\else
\newcommand{\nl}{\ \par} % newline (achter heading van definition etc.)
\fi


% \nl enkel in handoutmode (ihb voor \wordChoice, die dan typisch veeeel langer wordt)
\ifdefined\HCode
\providecommand{\handoutnl}{}
\else
\providecommand{\handoutnl}{%
\ifhandout%
  \nl%
\fi%
}
\fi

% Could potentially replace \pdfOnline/\begin{onlineOnly} : 
% Usage= \ifonline{Hallo surfer}{Hallo PDFlezer}
\providecommand{\ifonline}[2]%
{
\begin{onlineOnly}#1\end{onlineOnly}%
\pdfOnly{#2}
}%


%
% Maak optionele 'basic' en 'extended' versies van een activity
%  met environment basicOnly, basicSkip en extendedOnly
%
%  (
%   Dit werkt ENKEL in de PDF; de online versies tonen (minstens voorklopig) steeds 
%   het default geval met printbasicversion en printextendversion beide FALSE
%  )
%
\providebool{printbasicversion}
\providebool{printextendedversion}   % not properly implemented
\providebool{printfullversion}       % presumably print everything (debug/auteur)
%
% only set these in xourses, and BEFORE loading this preamble
%
%\newif\ifshowbasic     \showbasictrue        % use this line in xourse to show 'basic' sections
%\newif\ifshowextended  \showextendedtrue     % use this line in xourse to show 'extended' sections
%
%
%\ifbool{showbasic}
%      { \NewEnviron{basicOnly}{\BODY} }    % if yes: just print contents
%      { \NewEnviron{basicOnly}{}      }    % if no:  completely ignore contents
%
%\ifbool{showbasic}
%      { \NewEnviron{basicSkip}{}      }
%      { \NewEnviron{basicSkip}{\BODY} }
%

\ifbool{printextendedversion}
      { \NewEnviron{extendedOnly}{\BODY} }
      { \NewEnviron{extendedOnly}{}      }
      


\ifdefined\HCode    % in html: always print
      {\newenvironment*{basicOnly}{}{}}    % if yes: just print contents
      {\newenvironment*{basicSkip}{}{}}    % if yes: just print contents
\else

\ifbool{printbasicversion}
      {\newenvironment*{basicOnly}{}{}}    % if yes: just print contents
      {\NewEnviron{basicOnly}{}      }    % if no:  completely ignore contents

\ifbool{printbasicversion}
      {\NewEnviron{basicSkip}{}      }
      {\newenvironment*{basicSkip}{}{}}

\fi

\usepackage{float}
\usepackage[rightbars,color]{changebar}

% Full versie
\ifbool{printfullversion}{
    \hintstrue
    \handoutfalse
    \toggletrue{showonline}
    \printbasicversionfalse
    \cbcolor{red}
    \renewenvironment*{basicOnly}{\cbstart}{\cbend}
    \renewenvironment*{basicSkip}{\cbstart}{\cbend}
    \def\xmtoonprintopties{FULL}   % will be printed in footer
}
{}
      
%
% Evalueer \ifhints IN de environment
%  
%
%\RenewEnviron{hint}
%{
%\ifhandout
%\ifhints\else\setbox0\vbox\fi%everything in een emty box
%\bgroup 
%\stepcounter{hintLevel}
%\BODY
%\egroup\ignorespacesafterend
%\addtocounter{hintLevel}{-1}
%\else
%\ifhints
%\begin{trivlist}\item[\hskip \labelsep\small\slshape\bfseries Hint:\hspace{2ex}]
%\small\slshape
%\stepcounter{hintLevel}
%\BODY
%\end{trivlist}
%\addtocounter{hintLevel}{-1}
%\fi
%\fi
%}

% Onafhankelijk van \ifhandout ...? TO BE VERIFIED
\RenewEnviron{hint}
{
\ifhints
\begin{trivlist}\item[\hskip \labelsep\small\bfseries Hint:\hspace{2ex}]
\small%\slshape
\stepcounter{hintLevel}
\BODY
\end{trivlist}
\addtocounter{hintLevel}{-1}
\else
\iftikzexport   % anders worden de tikz tekeningen in hints niet gegenereerd ?
\setbox0\vbox\bgroup
\stepcounter{hintLevel}
\BODY
\egroup\ignorespacesafterend
\addtocounter{hintLevel}{-1}
\fi % ifhandout
\fi %ifhints
}

%
% \tab sets typewriter-tabs (e.g. to format questions)
% (Has no effect in HTML :-( ))
%
\usepackage{tabto}
\ifdefined\HCode
  \renewcommand{\tab}{\quad}    % otherwise dummy .png's are generated ...?
\fi


% Also redefined in  preamble to get correct styling 
% for tikz images for (\tikzexport)
%

\theoremstyle{definition} % Bold titels
\makeatletter
\let\proposition\relax
\let\c@proposition\relax
\let\endproposition\relax
\makeatother
\newtheorem{proposition}{Eigenschap}


%\instructornotesfalse

% logic with \ifhandoutin ximera.cls unclear;so overwrite ...
\makeatletter
\@ifundefined{ifinstructornotes}{%
  \newif\ifinstructornotes
  \instructornotesfalse
  \newenvironment{instructorNotes}{}{}
}{}
\makeatother
\ifinstructornotes
\else
\renewenvironment{instructorNotes}%
{%
    \setbox0\vbox\bgroup
}
{%
    \egroup
}
\fi

% \RedeclareMathOperator
% from https://tex.stackexchange.com/questions/175251/how-to-redefine-a-command-using-declaremathoperator
\makeatletter
\newcommand\RedeclareMathOperator{%
    \@ifstar{\def\rmo@s{m}\rmo@redeclare}{\def\rmo@s{o}\rmo@redeclare}%
}
% this is taken from \renew@command
\newcommand\rmo@redeclare[2]{%
    \begingroup \escapechar\m@ne\xdef\@gtempa{{\string#1}}\endgroup
    \expandafter\@ifundefined\@gtempa
    {\@latex@error{\noexpand#1undefined}\@ehc}%
    \relax
    \expandafter\rmo@declmathop\rmo@s{#1}{#2}}
% This is just \@declmathop without \@ifdefinable
\newcommand\rmo@declmathop[3]{%
    \DeclareRobustCommand{#2}{\qopname\newmcodes@#1{#3}}%
}
\@onlypreamble\RedeclareMathOperator
\makeatother


%
% Engelse vertaling, vooral in mathmode
%
% 1. Algemene procedure
%
\ifdefined\isEn
 \newcommand{\nlen}[2]{#2}
 \newcommand{\nlentext}[2]{\text{#2}}
 \newcommand{\nlentextbf}[2]{\textbf{#2}}
\else
 \newcommand{\nlen}[2]{#1}
 \newcommand{\nlentext}[2]{\text{#1}}
 \newcommand{\nlentextbf}[2]{\textbf{#1}}
\fi

%
% 2. Lijst van erg veel gebruikte uitdrukkingen
%

% Ja/Nee/Fout/Juits etc
%\newcommand{\TJa}{\nlentext{ Ja }{ and }}
%\newcommand{\TNee}{\nlentext{ Nee }{ No }}
%\newcommand{\TJuist}{\nlentext{ Juist }{ Correct }
%\newcommand{\TFout}{\nlentext{ Fout }{ Wrong }
\newcommand{\TWaar}{\nlentext{ Waar }{ True }}
\newcommand{\TOnwaar}{\nlentext{ Vals }{ False }}
% Korte bindwoorden en, of, dus, ...
\newcommand{\Ten}{\nlentext{ en }{ and }}
\newcommand{\Tof}{\nlentext{ of }{ or }}
\newcommand{\Tdus}{\nlentext{ dus }{ so }}
\newcommand{\Tendus}{\nlentext{ en dus }{ and thus }}
\newcommand{\Tvooralle}{\nlentext{ voor alle }{ for all }}
\newcommand{\Took}{\nlentext{ ook }{ also }}
\newcommand{\Tals}{\nlentext{ als }{ when }} %of if?
\newcommand{\Twant}{\nlentext{ want }{ as }}
\newcommand{\Tmaal}{\nlentext{ maal }{ times }}
\newcommand{\Toptellen}{\nlentext{ optellen }{ add }}
\newcommand{\Tde}{\nlentext{ de }{ the }}
\newcommand{\Thet}{\nlentext{ het }{ the }}
\newcommand{\Tis}{\nlentext{ is }{ is }} %zodat is in text staat in mathmode (geen italics)
\newcommand{\Tmet}{\nlentext{ met }{ where }} % in situaties e.g met p < n --> where p < n
\newcommand{\Tnooit}{\nlentext{ nooit }{ never }}
\newcommand{\Tmaar}{\nlentext{ maar }{ but }}
\newcommand{\Tniet}{\nlentext{ niet }{ not }}
\newcommand{\Tuit}{\nlentext{ uit }{ from }}
\newcommand{\Ttov}{\nlentext{ t.o.v. }{ w.r.t. }}
\newcommand{\Tzodat}{\nlentext{ zodat }{ such that }}
\newcommand{\Tdeth}{\nlentext{de }{th }}
\newcommand{\Tomdat}{\nlentext{omdat }{because }} 


%
% Overschrijf addhoc commando's
%
\ifdefined\isEn
\renewcommand{\pernot}{\overset{\mathrm{notation}}{=}}
\RedeclareMathOperator{\bld}{im}     % beeld
\RedeclareMathOperator{\graf}{graph}   % grafiek
\RedeclareMathOperator{\rico}{slope}   % richtingcoëfficient
\RedeclareMathOperator{\co}{co}       % coordinaat
\RedeclareMathOperator{\gr}{deg}       % graad

% Operators
\RedeclareMathOperator{\bgsin}{arcsin}
\RedeclareMathOperator{\bgcos}{arccos}
\RedeclareMathOperator{\bgtan}{arctan}
\RedeclareMathOperator{\bgcot}{arccot}
\RedeclareMathOperator{\bgsinh}{arcsinh}
\RedeclareMathOperator{\bgcosh}{arccosh}
\RedeclareMathOperator{\bgtanh}{arctanh}
\RedeclareMathOperator{\bgcoth}{arccoth}

\fi


% HACK: use 'oplossing' for 'explanation' ...
\let\explanation\relax
\let\endexplanation\relax
% \newenvironment{explanation}{\begin{oplossing}}{\end{oplossing}}
\newcounter{explanation}

\ifhandout%
    \NewEnviron{explanation}[1][toon]%
    {%
    \RenewEnviron{verbatim}{ \red{VERBATIM CONTENT MISSING IN THIS PDF}} %% \expandafter\verb|\BODY|}

    \ifthenelse{\equal{\detokenize{#1}}{\detokenize{toon}}}
    {
    \def\PH@Command{#1}% Use PH@Command to hold the content and be a target for "\expandafter" to expand once.

    \begin{trivlist}% Begin the trivlist to use formating of the "Feedback" label.
    \item[\hskip \labelsep\small\slshape\bfseries Explanation:% Format the "Feedback" label. Don't forget the space.
    %(\texttt{\detokenize\expandafter{\PH@Command}}):% Format (and detokenize) the condition for feedback to trigger
    \hspace{2ex}]\small%\slshape% Insert some space before the actual feedback given.
    \BODY
    \end{trivlist}
    }
    {  % \begin{feedback}[solution]   \BODY     \end{feedback}  }
    }
    }    
\else
% ONLY for HTML; xmoplossing is styled with css, and is not, and need not be a LaTeX environment
% THUS: it does NOT use feedback anymore ...
%    \NewEnviron{oplossing}{\begin{expandable}{xmoplossing}{\nlen{Toon uitwerking}{Show solution}}{\BODY}\end{expandable}}
    \newenvironment{explanation}[1][toon]
   {%
       \begin{expandable}{xmoplossing}{}
   }
   {%
   	   \end{expandable}
   } 
\fi

\title{Application to Computer Graphics} \license{CC BY-NC-SA 4.0}

\begin{document}

\begin{abstract}
\end{abstract}
\maketitle

\section*{Application to Computer Graphics}
Computer graphics deals with images
displayed on a computer screen, and so arises in a variety of
applications, ranging from word processors, to \textit{Star Wars}
animations, to video games, to wire-frame images of an airplane.  These
images consist of a number of points on the screen, together with
instructions on how to fill in areas bounded by lines and curves. Often
curves are approximated by a set of short straight-line segments, so
that the curve is specified by a series of points on the screen at the
end of these segments. Matrix transformations are important here because
 matrix images of straight line segments are again line segments.
 % \begin{remark}
 %     If $\vec{v}_{0}$ and $\vec{v}_{1}$ are vectors, the vector from $\vec{v}_{0}$ to $\vec{v}_{1}$ is $\vec{d} = \vec{v}_{1} - \vec{v}_{0}$. So a vector $\vec{v}$ lies on the line segment between $\vec{v}_{0}$ and $\vec{v}_{1}$ if and only if $\vec{v} = \vec{v}_{0} + t\vec{d}$ for some number $t$ in the range $0 \leq t \leq 1$. Thus the image of this segment is the set of vectors $A\vec{v} = A\vec{v}_{0} + tA \vec{d}$ with $0 \leq t \leq 1$, that is the image is the segment between $A\vec{v}_{0}$ and $A\vec{v}_{1}$.
 % \end{remark}
 
 Note that a color image requires that three images are sent, one to
each of the red, green, and blue phosphorus dots on the screen, in
varying intensities.\index{computer graphics}\index{linear transformations!in computer graphics}\index{vector geometry!computer graphics}

Consider displaying the letter $A$. In reality, it is depicted on the screen, as in the figure below, by specifying the coordinates of the 11 corners and filling in the interior.

  \begin{center}
\begin{tikzpicture}[scale=0.4]
\draw[thin,gray!40] (0,0) grid (9,13);
  \filldraw[white, opacity=1](1,1)--(2.36,6)--(6.64,6)--(8,1)--(7,1)--(5.91,5)--(3.09,5)--(2,1)--cycle;
  \filldraw[white, opacity=1](2.36,6)--(4,12)--(5,12)--(6.64,6)--(5.64,6)--(4.5,10.17)--(3.36,6)--cycle;
  \draw[line width=2pt,blue](1,1)--(4,12);
  \draw[line width=2pt,blue](5,12)--(4,12);
  \draw[line width=2pt,blue](5,12)--(8,1);
  \draw[line width=2pt,blue](8,1)--(7,1);
  \draw[line width=2pt,blue](7,1)--(5.91,5);
  \draw[line width=2pt,blue](5.91,5)--(3.09,5);
  \draw[line width=2pt,blue](3.09,5)--(2,1);
  \draw[line width=2pt,blue](1,1)--(2,1);
  \draw[line width=2pt,blue](4.5,10.17)--(3.36,6);
  \draw[line width=2pt,blue](4.5,10.17)--(5.64,6);
  \draw[line width=2pt,blue](3.36,6)--(5.64,6);
  \fill[] (1,1) circle (0.2cm);
  \fill[] (4,12) circle (0.2cm);
  \fill[] (5,12) circle (0.2cm);
  \fill[] (8,1) circle (0.2cm);
  \fill[] (7,1) circle (0.2cm);
  \fill[] (5.91,5) circle (0.2cm);
  \fill[] (3.09,5) circle (0.2cm);
  \fill[] (2,1) circle (0.2cm);
  \fill[] (4.5,10.17) circle (0.2cm);
  \fill[] (3.36,6) circle (0.2cm);
  \fill[] (5.64,6) circle (0.2cm);
 \end{tikzpicture}
\end{center}  

 For simplicity, we will disregard the thickness of the letter,
 so we require only five coordinates as in the figure below.

   \begin{center}
\begin{tikzpicture}[scale=0.4]
\draw[thin,gray!40] (-1,-1) grid (7,10);
  \draw[line width=2pt,red](0,0)--(3,9);
  \draw[line width=2pt,red](6,0)--(3,9);
  \draw[line width=2pt,red](5,3)--(1,3);
  \fill[] (0,0)node[above=2mm, left=1mm]{$(0,0)$} circle (0.2cm);
  \fill[] (6,0)node[above=2mm, right=1mm]{$(6,0)$} circle (0.2cm);
  \fill[] (5,3)node[above=2mm, right=1mm]{$(5,3)$} circle (0.2cm);
  \fill[] (1,3)node[above=2mm, left=1mm]{$(1,3)$} circle (0.2cm);
  \fill[] (3,9)node[above=2mm, right=1mm]{$(3,9)$} circle (0.2cm);
 \end{tikzpicture}
\end{center}  

This simplified letter can then be stored as a data matrix

$$D =\begin{bmatrix}
0 & 6 & 5 & 1 & 3\\
0 & 0 & 3 & 3 & 9
\end{bmatrix}$$

where the columns are the coordinates of the vertices. Then if we want to transform the letter by a $2 \times 2$ matrix $A$, we left-multiply this data matrix by $A$ (the effect is to multiply each column by $A$ and so transform each vertex).

For example, we can slant the letter to the right by multiplying by a horizontal shear matrix $A = \left[
\begin{array}{ll}
1 & 0.2\\
0 & 1\\
\end{array}
\right]$
 (See Formula \ref{form:shears}.) The result is the letter with data matrix
\begin{equation*}
A = \left[
\begin{array}{ll}
1 & 0.2\\
0 & 1
\end{array}
\right] \left[
\begin{array}{rrrrr}
0 & 6 & 5 & 1 & 3\\
0 & 0 & 3 & 3 & 9
\end{array}
\right] =
 \left[
\begin{array}{lllll}
0 & 6 & 5.6 & 1.6 & 4.8\\
0 & 0 & 3 & 3 & 9
\end{array}
\right]
\end{equation*}
which is shown in the figure below.

 \begin{center}
\begin{tikzpicture}[scale=0.4]
\draw[thin,gray!40] (-1,-1) grid (7,10);
  \draw[line width=2pt,red](0,0)--(4.8,9);
  \draw[line width=2pt,red](6,0)--(4.8,9);
  \draw[line width=2pt,red](5.6,3)--(1.6,3);
  \fill[] (0,0)node[above=2mm, left=1mm]{$(0,0)$} circle (0.2cm);
  \fill[] (6,0)node[above=2mm, right=1mm]{$(6,0)$} circle (0.2cm);
  \fill[] (5.6,3)node[above=2mm, right=1mm]{$(5.6,3)$} circle (0.2cm);
  \fill[] (1.6,3)node[above=2mm, left=1mm]{$(1.6,3)$} circle (0.2cm);
  \fill[] (4.8,9)node[above=2mm, right=1mm]{$(4.8,9)$} circle (0.2cm);
 \end{tikzpicture}
\end{center}  


 If we want to make this slanted matrix narrower, we can now apply a scaling matrix $B = \left[
\begin{array}{ll}
	0.8 & 0\\
	0 & 1
\end{array}
\right]$ that shrinks the $x$-coordinate by $0.8$. (See Formula \ref{form:horvertscaling}.) The result is the composite transformation
\begin{align*}
BAD &= \left[
\begin{array}{ll}
0.8 & 0\\
0 & 1
\end{array}
\right] \left[
\begin{array}{ll}
1 & 0.2\\
0 & 1
\end{array}
\right] \left[
\begin{array}{rrrrr}
0 & 6 & 5 & 1 & 3\\
0 & 0 & 3 & 3 & 9
\end{array}
\right] \\ &=
\left[
\begin{array}{lllll}
0 & 4.8 & 4.48 & 1.28 & 3.84\\
0 & 0 & 3 & 3 & 9
\end{array}
\right]
\end{align*}
which is drawn in the figure below.

 \begin{center}
\begin{tikzpicture}[scale=0.4]
\draw[thin,gray!40] (-1,-1) grid (7,10);
  \draw[line width=2pt,red](0,0)--(3.84,9);
  \draw[line width=2pt,red](4.8,0)--(3.84,9);
  \draw[line width=2pt,red](4.48,3)--(1.28,3);
  \fill[] (0,0)node[above=2mm, left=1mm]{$(0,0)$} circle (0.2cm);
  \fill[] (4.8,0)node[above=2mm, right=1mm]{$(4.8,0)$} circle (0.2cm);
  \fill[] (4.48,3)node[above=2mm, right=1mm]{$(4.48,3)$} circle (0.2cm);
  \fill[] (1.28,3)node[above=2mm, left=1mm]{$(1.28,3)$} circle (0.2cm);
  \fill[] (3.84,9)node[above=2mm, right=1mm]{$(3.84,9)$} circle (0.2cm);
 \end{tikzpicture}
\end{center}  

On the other hand, we can rotate the letter about the origin through $\frac{\pi}{6}$ (or $30^\circ$) by multiplying by the rotation matrix  $$R_{\frac{\pi}{2}} = \left[
\def\arraystretch{1.5}\begin{array}{rr}
\cos(\frac{\pi}{6}) & -\sin(\frac{\pi}{6})\\
\sin(\frac{\pi}{6}) & \cos(\frac{\pi}{6})
\end{array}
\right] = \left[
\begin{array}{ll}
0.866 & -0.5\\
0.5 & 0.866
\end{array}
\right]$$
(See Formula \ref{form:rotation}.)  This gives
\begin{align*}
R_{\frac{\pi}{2}} &= \left[
\begin{array}{ll}
0.866 & -0.5\\
0.5 & 0.866
\end{array}
\right] \left[
\begin{array}{rrrrr}
0 & 6 & 5 & 1 & 3\\
0 & 0 & 3 & 3 & 9
\end{array}
\right] \\ &=
\left[
\begin{array}{lllrr}
0 & 5.196 & 2.83 & -0.634 & -1.902\\
0 & 3 & 5.098 & 3.098 & 9.294
\end{array}
\right]
\end{align*}
and is plotted in the figure below.

 \begin{center}
\begin{tikzpicture}[scale=0.4]
\draw[thin,gray!40] (-1,-1) grid (7,10);
  \draw[line width=2pt,red](0,0)--(-1.902,9.294);
  \draw[line width=2pt,red](-0.634,3.098)--(2.83,5.098);
  \draw[line width=2pt,red](5.196,3)--(-1.902,9.294);
  \fill[] (0,0)node[above=2mm, left=1mm]{$(0,0)$} circle (0.2cm);
  \fill[] (2.83,5.098)node[above=2mm, right=1mm]{$(2.83,5.098)$} circle (0.2cm);
  \fill[] (5.196,3)node[above=2mm, right=1mm]{$(5.196,3)$} circle (0.2cm);
  \fill[] (-0.634,3.098)node[above=2mm, left=1mm]{$(-0.634,3.098)$} circle (0.2cm);
  \fill[] (-1.902,9.294)node[above=2mm, right=1mm]{$(-1.902,9.294)$} circle (0.2cm);
 \end{tikzpicture}
\end{center}  

This poses a problem: How do we rotate
about a point other than the origin? It turns out that we can do this when
we have solved another more basic problem. It is clearly important to be
 able to translate a screen image by a fixed vector $\vec{w}$, that is apply the transformation $T_{w} : \RR^2 \to \RR^2$ given by $T_{w}(\vec{v}) = \vec{v} + \vec{w}$ for all $\vec{v}$ in $\RR^2$. The problem is that these translations are not matrix transformations $\RR^2 \to \RR^2$ because they do not carry $\vec{0}$ to $\vec{0}$ (unless $\vec{w} = \vec{0}$). (See Practice Problem \ref{prob:translation_does_not_work}.) However, there is a clever way around this.

The idea is to represent a point $\vec{v} = \left[
\begin{array}{r}
x\\
y
\end{array}
\right]$
 as a $3 \times 1$ column $\left[
 \begin{array}{r}
 x\\
 y\\
 1
 \end{array}
 \right]$, called the \dfn{homogeneous coordinates}\index{homogeneous coordinates} of $\vec{v}$. Then translation by $\vec{w} = \left[
 \begin{array}{r}
 p\\
 q
 \end{array}
 \right]$ can be achieved by multiplying by a $3 \times 3$ matrix:
\begin{equation*}
\left[
\begin{array}{rrr}
1 & 0 & p\\
0 & 1 & q\\
0 & 0 & 1
\end{array}
\right] \left[
\begin{array}{r}
x\\
y\\
1
\end{array}
\right]
=
\left[
\begin{array}{c}
x + p\\
y + q\\
1
\end{array}
\right]
=
\left[
\begin{array}{c}
T_{\vec{w}}(\vec{v})\\
1
\end{array}
\right]
\end{equation*}
Thus, by using homogeneous coordinates we can implement the translation $T_{w}$ in the top two coordinates. At the same time, matrix transformations we performed earlier can still be accomplished using homogeneous coordinates.  For instance, the matrix transformation induced by $A = \left[
\begin{array}{rr}
a & b\\
c & d
\end{array}
\right]$ is given by a $3 \times 3$ matrix:
\begin{equation*}
\left[
\begin{array}{rrr}
a & b & 0\\
c & d & 0\\
0 & 0 & 1
\end{array}
\right] \left[
\begin{array}{r}
x\\
y\\
1
\end{array}
\right]
=
\left[
\begin{array}{c}
ax + by\\
cx + dy\\
1
\end{array}
\right]
=
\left[
\begin{array}{c}
A\vec{v}\\
1
\end{array}
\right]
\end{equation*}
So everything can be accomplished at the expense of using $3 \times 3$ matrices and homogeneous coordinates.

It is interesting to examine the geometric implications of homogeneous coordinates.  All points of the form $\begin{bmatrix}x\\y\end{bmatrix}$, originally located in the $xy$-plane in $\RR^2$, now have the form $\begin{bmatrix}x\\y\\1\end{bmatrix}$ and are located in a horizontal plane $z=1$ in $\RR^3$.  What we perceive as a translation in the horizontal plane $z=1$, is actually a shear in $\RR^3$.  This is similar to how horizontal shears in $\RR^2$ affect points along the line $y=1$: all such points move a fixed distance to the left or to the right. 


\begin{example}\label{exa:013322}
Rotate the letter $A$ through $\frac{\pi}{6}$ about the point $\left[
\begin{array}{c}
4\\
5
\end{array}
\right]$.

\begin{center}
\begin{tikzpicture}[scale=0.4]
\draw[thin,gray!40] (-1,-1) grid (7,10);
  \draw[line width=2pt,red](0,0)--(3,9);
  \draw[line width=2pt,red](6,0)--(3,9);
  \draw[line width=2pt,red](5,3)--(1,3);
  \fill[] (0,0)node[above=2mm, left=1mm]{$(0,0)$} circle (0.2cm);
  \fill[] (6,0)node[above=2mm, right=1mm]{$(6,0)$} circle (0.2cm);
  \fill[] (5,3)node[above=2mm, right=1mm]{$(5,3)$} circle (0.2cm);
  \fill[] (1,3)node[above=2mm, left=1mm]{$(1,3)$} circle (0.2cm);
  \fill[] (3,9)node[above=2mm, right=1mm]{$(3,9)$} circle (0.2cm);
 \end{tikzpicture}
\end{center} 

\begin{explanation}
Using homogenous coordinates for the vertices of the letter results in a data matrix with three rows:
\begin{equation*}
K_{d} =  \left[
\begin{array}{rrrrr}
0 & 6 & 5 & 1 & 3\\
0 & 0 & 3 & 3 & 9 \\
1 & 1 & 1 & 1 & 1
\end{array}
\right]
\end{equation*}

%\begin{wrapfigure}[6]{l}{5cm}
%\vspace*{-5em}
%\centering
%\input{4-vector-geometry/figures/5-an-application-to-computer-graphics/figure4.5.6}
%\caption{\label{fig:013335}}
%\end{wrapfigure}


If we write $\vec{w} = \left[
\begin{array}{c}
4\\
5
\end{array}
\right]$, the idea is to use a composite of transformations: First translate the letter by $-\vec{w}$ so that the point $\vec{w}$ moves to the origin, then rotate this translated letter, and then translate it by $\vec{w}$ back to its original position. The following GeoGebra interactive illustrates this sequence.

\begin{center}
\geogebra{yd5agbdg}{950}{650}
\end{center}

The matrix arithmetic is as follows (remember the order of composition!):
\begin{align*}
& \left[
\begin{array}{rrr}
1 & 0 & 4\\
0 & 1 & 5\\
0 & 0 & 1
\end{array}
\right]
\left[
\begin{array}{lrr}
0.866 & -0.5 & 0\\
0.5 & 0.866 & 0\\
0 & 0 & 1
\end{array}
\right]
\left[
\begin{array}{rrr}
1 & 0 & -4\\
0 & 1 & -5\\
0 & 0 & 1
\end{array}
\right]
\left[
\begin{array}{rrrrr}
0 & 6 & 5 & 1 & 3\\
0 & 0 & 3 & 3 & 9 \\
1 & 1 & 1 & 1 & 1
\end{array}
\right] \\
&= \left[
\begin{array}{lllll}
3.036 & 8.232 & 5.866 & 2.402 & 1.134\\
-1.33 & 1.67 & 3.768 & 1.768 & 7.964 \\
1 & 1 & 1 & 1 & 1
\end{array}
\right]
\end{align*}

\end{explanation}
\end{example}

This discussion merely touches the
surface of computer graphics, and the reader is referred to specialized
books on the subject. Realistic graphic rendering requires an enormous
number of matrix calculations. In fact, matrix multiplication algorithms
 are now embedded in microchip circuits, and can perform over 100
million matrix multiplications per second. This is particularly
important in the field of three-dimensional graphics where the
homogeneous coordinates have four components and $4 \times 4$ matrices are
required.


\section*{Practice Problems}

\begin{problem}\label{prob:app0040_1}
Consider the letter $A$ below. 
\begin{center}
\begin{tikzpicture}[scale=0.4]
\draw[thin,gray!40] (-1,-1) grid (7,10);
  \draw[line width=2pt,red](0,0)--(3,9);
  \draw[line width=2pt,red](6,0)--(3,9);
  \draw[line width=2pt,red](5,3)--(1,3);
  \fill[] (0,0)node[above=2mm, left=1mm]{$(0,0)$} circle (0.2cm);
  \fill[] (6,0)node[above=2mm, right=1mm]{$(6,0)$} circle (0.2cm);
  \fill[] (5,3)node[above=2mm, right=1mm]{$(5,3)$} circle (0.2cm);
  \fill[] (1,3)node[above=2mm, left=1mm]{$(1,3)$} circle (0.2cm);
  \fill[] (3,9)node[above=2mm, right=1mm]{$(3,9)$} circle (0.2cm);
 \end{tikzpicture}
\end{center}  

Find the data matrix for the letter obtained by:


\begin{enumerate} 
\item Rotating the letter through $\frac{\pi}{4}$ about the origin.

\item Rotating the letter through $\frac{\pi}{4}$
 about the point $\left[
 \begin{array}{c}
 1\\
 2
 \end{array}
 \right]$.

Click the arrow to see the answer.
\begin{expandable}
${\frac{1}{2}\left[
\begin{array}{ccccc}
\sqrt{2} + 2 & 7\sqrt{2} + 2 & 3\sqrt{2} + 2 & -\sqrt{2} + 2 & -5\sqrt{2} + 2\\
-3\sqrt{2} + 4 & 3\sqrt{2} + 4 & 5\sqrt{2} + 4 & \sqrt{2} + 4 & 9\sqrt{2} + 4 \\
2 & 2 & 2 & 2 & 2
\end{array}
\right]}$
\end{expandable}
\end{enumerate}
\end{problem}

% \begin{problem}\label{prob:app0040_2}
% Find the matrix for turning the letter $A$ in Figure~\ref{fig:013290} upside-down in place.
% \end{problem}

\begin{problem}\label{prob:app0040_3}
Find the $3 \times 3$ matrix for reflecting in the line $y = mx + b$. Use $\left[
\begin{array}{c}
1\\
m
\end{array}
\right]$ as direction vector for the line.
\end{problem}

\begin{problem}\label{prob:app0040_4}
Find the $3 \times 3$ matrix for rotating through the angle $\theta$ about the point $P(a, b)$.
\end{problem}

\begin{problem}\label{prob:app0040_5}
Find the reflection of the point $P$ in the line $y = 1 + 2x$ in $\RR^2$ if:


\begin{enumerate} 
\item $P = P(1, 1)$

\item $P = P(1, 4)$

Click the arrow to see the answer.
\begin{expandable}
$P(\frac{9}{5}, \frac{18}{5})$
\end{expandable}

\item What about $P = P(1, 3)$? Explain.
\begin{hint}
    See Example~\ref{exa:013322}.
\end{hint}
\end{enumerate}
\end{problem}

\end{document}