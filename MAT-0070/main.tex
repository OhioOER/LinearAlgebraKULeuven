\documentclass{ximera}
%%% Begin Laad packages

\makeatletter
\@ifclassloaded{xourse}{%
    \typeout{Start loading preamble.tex (in a XOURSE)}%
    \def\isXourse{true}   % automatically defined; pre 112022 it had to be set 'manually' in a xourse
}{%
    \typeout{Start loading preamble.tex (NOT in a XOURSE)}%
}
\makeatother

\def\isEn\true 

\pgfplotsset{compat=1.16}

\usepackage{currfile}

% 201908/202301: PAS OP: babel en doclicense lijken problemen te veroorzaken in .jax bestand
% (wegens syntax error met toegevoegde \newcommands ...)
\pdfOnly{
    \usepackage[type={CC},modifier={by-nc-sa},version={4.0}]{doclicense}
    %\usepackage[hyperxmp=false,type={CC},modifier={by-nc-sa},version={4.0}]{doclicense}
    %%% \usepackage[dutch]{babel}
}



\usepackage[utf8]{inputenc}
\usepackage{morewrites}   % nav zomercursus (answer...?)
\usepackage{multirow}
\usepackage{multicol}
\usepackage{tikzsymbols}
\usepackage{ifthen}
%\usepackage{animate} BREAKS HTML STRUCTURE USED BY XIMERA
\usepackage{relsize}

\usepackage{eurosym}    % \euro  (€ werkt niet in xake ...?)
\usepackage{fontawesome} % smileys etc

% Nuttig als ook interactieve beamer slides worden voorzien:
\providecommand{\p}{} % default nothing ; potentially usefull for slides: redefine as \pause
%providecommand{\p}{\pause}

    % Layout-parameters voor het onderschrift bij figuren
\usepackage[margin=10pt,font=small,labelfont=bf, labelsep=endash,format=hang]{caption}
%\usepackage{caption} % captionof
%\usepackage{pdflscape}    % landscape environment

% Met "\newcommand\showtodonotes{}" kan je todonotes tonen (in pdf/online)
% 201908: online werkt het niet (goed)
\providecommand\showtodonotes{disable}
\providecommand\todo[1]{\typeout{TODO #1}}
%\usepackage[\showtodonotes]{todonotes}
%\usepackage{todonotes}

%
% Poging tot aanpassen layout
%
\usepackage{tcolorbox}
\tcbuselibrary{theorems}

%%% Einde laad packages

%%% Begin Ximera specifieke zaken

\graphicspath{
	{../../}
	{../}
	{./}
  	{../../pictures/}
   	{../pictures/}
   	{./pictures/}
	{./explog/}    % M05 in groeimodellen       
}

%%% Einde Ximera specifieke zaken

%
% define softer blue/red/green, use KU Leuven base colors for blue (and dark orange for red ?)
%
% todo: rather redefine blue/red/green ...?
%\definecolor{xmblue}{rgb}{0.01, 0.31, 0.59}
%\definecolor{xmred}{rgb}{0.89, 0.02, 0.17}
\definecolor{xmdarkblue}{rgb}{0.122, 0.671, 0.835}   % KU Leuven Blauw
\definecolor{xmblue}{rgb}{0.114, 0.553, 0.69}        % KU Leuven Blauw
\definecolor{xmgreen}{rgb}{0.13, 0.55, 0.13}         % No KULeuven variant for green found ...

\definecolor{xmaccent}{rgb}{0.867, 0.541, 0.18}      % KU Leuven Accent (orange ...)
\definecolor{kuaccent}{rgb}{0.867, 0.541, 0.18}      % KU Leuven Accent (orange ...)

\colorlet{xmred}{xmaccent!50!black}                  % Darker version of KU Leuven Accent

\providecommand{\blue}[1]{{\color{blue}#1}}    
\providecommand{\red}[1]{{\color{red}#1}}

\renewcommand\CancelColor{\color{xmaccent!50!black}}

% werkt in math en text mode om MATH met oranje (of grijze...)  achtergond te tonen (ook \important{\text{blabla}} lijkt te werken)
%\newcommand{\important}[1]{\ensuremath{\colorbox{xmaccent!50!white}{$#1$}}}   % werkt niet in Mathjax
%\newcommand{\important}[1]{\ensuremath{\colorbox{lightgray}{$#1$}}}
\newcommand{\important}[1]{\ensuremath{\colorbox{orange}{$#1$}}}   % TODO: kleur aanpassen voor mathjax; wordt overschreven infra!


% Uitzonderlijk kan met \pdfnl in de PDF een newline worden geforceerd, die online niet nodig/nuttig is omdat daar de regellengte hoe dan ook niet gekend is.
\ifdefined\HCode%
\providecommand{\pdfnl}{}%
\else%
\providecommand{\pdfnl}{%
  \\%
}%
\fi

% Uitzonderlijk kan met \handoutnl in de handout-PDF een newline worden geforceerd, die noch online noch in de PDF-met-antwoorden nuttig is.
\ifdefined\HCode
\providecommand{\handoutnl}{}
\else
\providecommand{\handoutnl}{%
\ifhandout%
  \nl%
\fi%
}
\fi



% \cellcolor IGNORED by tex4ht ?
% \begin{center} seems not to wordk
    % (missing margin-left: auto;   on tabular-inside-center ???)
%\newcommand{\importantcell}[1]{\ensuremath{\cellcolor{lightgray}#1}}  %  in tabular; usablility to be checked
\providecommand{\importantcell}[1]{\ensuremath{#1}}     % no mathjax2 support for colloring array cells

\pdfOnly{
  \renewcommand{\important}[1]{\ensuremath{\colorbox{kuaccent!50!white}{$#1$}}}
  \renewcommand{\importantcell}[1]{\ensuremath{\cellcolor{kuaccent!40!white}#1}}   
}

%%% Tikz styles


\pgfplotsset{compat=1.16}

\usetikzlibrary{trees,positioning,arrows,fit,shapes,math,calc,decorations.markings,through,intersections,patterns,matrix}

\usetikzlibrary{decorations.pathreplacing,backgrounds}    % 5/2023: from experimental


\usetikzlibrary{angles,quotes}

\usepgfplotslibrary{fillbetween} % bepaalde_integraal
\usepgfplotslibrary{polar}    % oa voor poolcoordinaten.tex

\pgfplotsset{ownstyle/.style={axis lines = center, axis equal image, xlabel = $x$, ylabel = $y$, enlargelimits}} 

\pgfplotsset{
	plot/.style={no marks,samples=50}
}

\newcommand{\xmPlotsColor}{
	\pgfplotsset{
		plot1/.style={darkgray,no marks,samples=100},
		plot2/.style={lightgray,no marks,samples=100},
		plotresult/.style={blue,no marks,samples=100},
		plotblue/.style={blue,no marks,samples=100},
		plotred/.style={red,no marks,samples=100},
		plotgreen/.style={green,no marks,samples=100},
		plotpurple/.style={purple,no marks,samples=100}
	}
}
\newcommand{\xmPlotsBlackWhite}{
	\pgfplotsset{
		plot1/.style={black,loosely dashed,no marks,samples=100},
		plot2/.style={black,loosely dotted,no marks,samples=100},
		plotresult/.style={black,no marks,samples=100},
		plotblue/.style={black,no marks,samples=100},
		plotred/.style={black,dotted,no marks,samples=100},
		plotgreen/.style={black,dashed,no marks,samples=100},
		plotpurple/.style={black,dashdotted,no marks,samples=100}
	}
}


\newcommand{\xmPlotsColorAndStyle}{
	\pgfplotsset{
		plot1/.style={darkgray,no marks,samples=100},
		plot2/.style={lightgray,no marks,samples=100},
		plotresult/.style={blue,no marks,samples=100},
		plotblue/.style={xmblue,no marks,samples=100},
		plotred/.style={xmred,dashed,thick,no marks,samples=100},
		plotgreen/.style={xmgreen,dotted,very thick,no marks,samples=100},
		plotpurple/.style={purple,no marks,samples=100}
	}
}


%\iftikzexport
\xmPlotsColorAndStyle
%\else
%\xmPlotsBlackWhite
%\fi
%%%


%
% Om venndiagrammen te arceren ...
%
\makeatletter
\pgfdeclarepatternformonly[\hatchdistance,\hatchthickness]{north east hatch}% name
{\pgfqpoint{-1pt}{-1pt}}% below left
{\pgfqpoint{\hatchdistance}{\hatchdistance}}% above right
{\pgfpoint{\hatchdistance-1pt}{\hatchdistance-1pt}}%
{
	\pgfsetcolor{\tikz@pattern@color}
	\pgfsetlinewidth{\hatchthickness}
	\pgfpathmoveto{\pgfqpoint{0pt}{0pt}}
	\pgfpathlineto{\pgfqpoint{\hatchdistance}{\hatchdistance}}
	\pgfusepath{stroke}
}
\pgfdeclarepatternformonly[\hatchdistance,\hatchthickness]{north west hatch}% name
{\pgfqpoint{-\hatchthickness}{-\hatchthickness}}% below left
{\pgfqpoint{\hatchdistance+\hatchthickness}{\hatchdistance+\hatchthickness}}% above right
{\pgfpoint{\hatchdistance}{\hatchdistance}}%
{
	\pgfsetcolor{\tikz@pattern@color}
	\pgfsetlinewidth{\hatchthickness}
	\pgfpathmoveto{\pgfqpoint{\hatchdistance+\hatchthickness}{-\hatchthickness}}
	\pgfpathlineto{\pgfqpoint{-\hatchthickness}{\hatchdistance+\hatchthickness}}
	\pgfusepath{stroke}
}
%\makeatother

\tikzset{
    hatch distance/.store in=\hatchdistance,
    hatch distance=10pt,
    hatch thickness/.store in=\hatchthickness,
   	hatch thickness=2pt
}

\colorlet{circle edge}{black}
\colorlet{circle area}{blue!20}


\tikzset{
    filled/.style={fill=green!30, draw=circle edge, thick},
    arceerl/.style={pattern=north east hatch, pattern color=blue!50, draw=circle edge},
    arceerr/.style={pattern=north west hatch, pattern color=yellow!50, draw=circle edge},
    outline/.style={draw=circle edge, thick}
}




%%% Updaten commando's
\def\hoofding #1#2#3{\maketitle}     % OBSOLETE ??

% we willen (bijna) altijd \geqslant ipv \geq ...!
\newcommand{\geqnoslant}{\geq}
\renewcommand{\geq}{\geqslant}
\newcommand{\leqnoslant}{\leq}
\renewcommand{\leq}{\leqslant}

% Todo: (201908) waarom komt er (soms) underlined voor emph ...?
\renewcommand{\emph}[1]{\textit{#1}}

% API commando's

\newcommand{\ds}{\displaystyle}
\newcommand{\ts}{\textstyle}  % tegenhanger van \ds   (Ximera zet PER  DEFAULT \ds!)

% uit Zomercursus-macro's: 
\newcommand{\bron}[1]{\begin{scriptsize} \emph{#1} \end{scriptsize}}     % deprecated ...?


%definities nieuwe commando's - afkortingen veel gebruikte symbolen
\newcommand{\R}{\ensuremath{\mathbb{R}}}
\newcommand{\Rnul}{\ensuremath{\mathbb{R}_0}}
\newcommand{\Reen}{\ensuremath{\mathbb{R}\setminus\{1\}}}
\newcommand{\Rnuleen}{\ensuremath{\mathbb{R}\setminus\{0,1\}}}
\newcommand{\Rplus}{\ensuremath{\mathbb{R}^+}}
\newcommand{\Rmin}{\ensuremath{\mathbb{R}^-}}
\newcommand{\Rnulplus}{\ensuremath{\mathbb{R}_0^+}}
\newcommand{\Rnulmin}{\ensuremath{\mathbb{R}_0^-}}
\newcommand{\Rnuleenplus}{\ensuremath{\mathbb{R}^+\setminus\{0,1\}}}
\newcommand{\N}{\ensuremath{\mathbb{N}}}
\newcommand{\Nnul}{\ensuremath{\mathbb{N}_0}}
\newcommand{\Z}{\ensuremath{\mathbb{Z}}}
\newcommand{\Znul}{\ensuremath{\mathbb{Z}_0}}
\newcommand{\Zplus}{\ensuremath{\mathbb{Z}^+}}
\newcommand{\Zmin}{\ensuremath{\mathbb{Z}^-}}
\newcommand{\Znulplus}{\ensuremath{\mathbb{Z}_0^+}}
\newcommand{\Znulmin}{\ensuremath{\mathbb{Z}_0^-}}
\newcommand{\C}{\ensuremath{\mathbb{C}}}
\newcommand{\Cnul}{\ensuremath{\mathbb{C}_0}}
\newcommand{\Cplus}{\ensuremath{\mathbb{C}^+}}
\newcommand{\Cmin}{\ensuremath{\mathbb{C}^-}}
\newcommand{\Cnulplus}{\ensuremath{\mathbb{C}_0^+}}
\newcommand{\Cnulmin}{\ensuremath{\mathbb{C}_0^-}}
\newcommand{\Q}{\ensuremath{\mathbb{Q}}}
\newcommand{\Qnul}{\ensuremath{\mathbb{Q}_0}}
\newcommand{\Qplus}{\ensuremath{\mathbb{Q}^+}}
\newcommand{\Qmin}{\ensuremath{\mathbb{Q}^-}}
\newcommand{\Qnulplus}{\ensuremath{\mathbb{Q}_0^+}}
\newcommand{\Qnulmin}{\ensuremath{\mathbb{Q}_0^-}}

\newcommand{\perdef}{\overset{\mathrm{def}}{=}}
\newcommand{\pernot}{\overset{\mathrm{notatie}}{=}}
\newcommand\perinderdaad{\overset{!}{=}}     % voorlopig gebruikt in limietenrekenregels
\newcommand\perhaps{\overset{?}{=}}          % voorlopig gebruikt in limietenrekenregels

\newcommand{\degree}{^\circ}


\DeclareMathOperator{\dom}{dom}     % domein
\DeclareMathOperator{\codom}{codom} % codomein
\DeclareMathOperator{\bld}{bld}     % beeld
\DeclareMathOperator{\graf}{graf}   % grafiek
\DeclareMathOperator{\rico}{rico}   % richtingcoëfficient
\DeclareMathOperator{\co}{co}       % coordinaat
\DeclareMathOperator{\gr}{gr}       % graad

\newcommand{\func}[5]{\ensuremath{#1: #2 \rightarrow #3: #4 \mapsto #5}} % Easy to write a function


% Operators
\DeclareMathOperator{\bgsin}{bgsin}
\DeclareMathOperator{\bgcos}{bgcos}
\DeclareMathOperator{\bgtan}{bgtan}
\DeclareMathOperator{\bgcot}{bgcot}
\DeclareMathOperator{\bgsinh}{bgsinh}
\DeclareMathOperator{\bgcosh}{bgcosh}
\DeclareMathOperator{\bgtanh}{bgtanh}
\DeclareMathOperator{\bgcoth}{bgcoth}

% Oude \Bgsin etc deprecated: gebruik \bgsin, en herdefinieer dat als je Bgsin wil!
%\DeclareMathOperator{\cosec}{cosec}    % not used? gebruik \csc en herdefinieer

% operatoren voor differentialen: to be verified; 1/2020: inconsequent gebruik bij afgeleiden/integralen
\renewcommand{\d}{\mathrm{d}}
\newcommand{\dx}{\d x}
\newcommand{\dd}[1]{\frac{\mathrm{d}}{\mathrm{d}#1}}
\newcommand{\ddx}{\dd{x}}

% om in voorbeelden/oefeningen de notatie voor afgeleiden te kunnen kiezen
% Usage: \afg{(2\sin(x))}  (en wordt d/dx, of accent, of D )
%\newcommand{\afg}[1]{{#1}'}
\newcommand{\afg}[1]{\left(#1\right)'}
%\renewcommand{\afg}[1]{\frac{\mathrm{d}#1}{\mathrm{d}x}}   % include in relevant exercises ...
%\renewcommand{\afg}[1]{D{#1}}

%
% \xmxxx commands: Extra KU Leuven functionaliteit van, boven of naast Ximera
%   ( Conventie 8/2019: xm+nederlandse omschrijving, maar is niet consequent gevolgd, en misschien ook niet erg handig !)
%
% (Met een minimale ximera.cls en preamble.tex zou een bruikbare .pdf moeten kunnen worden gemaakt van eender welke ximera)
%
% Usage: \xmtitle[Mijn korte abstract]{Mijn titel}{Mijn abstract}
% Eerste command na \begin{document}:
%  -> definieert de \title
%  -> definieert de abstract
%  -> doet \maketitle ( dus: print de hoofding als 'chapter' of 'sectie')
% Optionele parameter geeft eenn kort abstract (die met de globale setting \xmshortabstract{} al dan niet kan worden geprint.
% De optionele korte abstract kan worden gebruikt voor pseudo-grappige abtsarts, dus dus globaal al dan niet kunnen worden gebuikt...
% Globale settings:
%  de (optionele) 'korte abstract' wordt enkele getoond als \xmshortabstract is gezet
\providecommand\xmshortabstract{} % default: print (only!) short abstract if present
\newcommand{\xmtitle}[3][]{
	\title{#2}
	\begin{abstract}
		\ifdefined\xmshortabstract
		\ifstrempty{#1}{%
			#3
		}{%
			#1
		}%
		\else
		#3
		\fi
	\end{abstract}
	\maketitle
}

% 
% Kleine grapjes: moeten zonder verder gevolg kunnen worden verwijderd
%
%\newcommand{\xmopje}[1]{{\small#1{\reversemarginpar\marginpar{\Smiley}}}}   % probleem in floats!!
\newtoggle{showxmopje}
\toggletrue{showxmopje}

\newcommand{\xmopje}[1]{%
   \iftoggle{showxmopje}{#1}{}%
}


% -> geef een abstracte-formule-met-rechts-een-concreet-voorbeeld
% VB:  \formulevb{a^2+b^2=c^2}{3^2+4^2=5^2}
%
\ifdefined\HCode
\NewEnviron{xmdiv}[1]{\HCode{\Hnewline<div class="#1">\Hnewline}\BODY{\HCode{\Hnewline</div>\Hnewline}}}
\else
\NewEnviron{xmdiv}[1]{\BODY}
\fi

\providecommand{\formulevb}[2]{
	{\centering

    \begin{xmdiv}{xmformulevb}    % zie css voor online layout !!!
	\begin{tabular}{lcl}
		\important{#1}
		&  &
		Vb: $#2$
		\end{tabular}
	\end{xmdiv}

	}
}

\ifdefined\HCode
\providecommand{\vb}[1]{%
    \HCode{\Hnewline<span class="xmvb">}#1\HCode{</span>\Hnewline}%
}
\else
\providecommand{\vb}[1]{
    \colorbox{blue!10}{#1}
}
\fi

\ifdefined\HCode
\providecommand{\xmcolorbox}[2]{
	\HCode{\Hnewline<div class="xmcolorbox">\Hnewline}#2\HCode{\Hnewline</div>\Hnewline}
}
\else
\providecommand{\xmcolorbox}[2]{
  \cellcolor{#1}#2
}
\fi


\ifdefined\HCode
\providecommand{\xmopmerking}[1]{
 \HCode{\Hnewline<div class="xmopmerking">\Hnewline}#1\HCode{\Hnewline</div>\Hnewline}
}
\else
\providecommand{\xmopmerking}[1]{
	{\footnotesize #1}
}
\fi
% \providecommand{\voorbeeld}[1]{
% 	\colorbox{blue!10}{$#1$}
% }



% Hernoem Proof naar Bewijs, nodig voor HTML versie
\renewcommand*{\proofname}{Bewijs}

% Om opgave van oefening (wordt niet geprint bij oplossingenblad)
% (to be tested test)
\NewEnviron{statement}{\BODY}

% Environment 'oplossing' en 'uitkomst'
% voor resp. volledige 'uitwerking' dan wel 'enkel eindresultaat'
% geimplementeerd via feedback, omdat er in de ximera-server adhoc feedback-code is toegevoegd
%% Niet tonen indien handout
%% Te gebruiken om volledige oplossingen/uitwerkingen van oefeningen te tonen
%% \begin{oplossing}        De optelling is commutatief \end{oplossing}  : verschijnt online enkel 'op vraag'
%% \begin{oplossing}[toon]  De optelling is commutatief \end{oplossing}  : verschijnt steeds onmiddellijk online (bv te gebruiken bij voorbeelden) 

\ifhandout%
    \NewEnviron{oplossing}[1][onzichtbaar]%
    {%
    \ifthenelse{\equal{\detokenize{#1}}{\detokenize{toon}}}
    {
    \def\PH@Command{#1}% Use PH@Command to hold the content and be a target for "\expandafter" to expand once.

    \begin{trivlist}% Begin the trivlist to use formating of the "Feedback" label.
    \item[\hskip \labelsep\small\slshape\bfseries Oplossing% Format the "Feedback" label. Don't forget the space.
    %(\texttt{\detokenize\expandafter{\PH@Command}}):% Format (and detokenize) the condition for feedback to trigger
    \hspace{2ex}]\small%\slshape% Insert some space before the actual feedback given.
    \BODY
    \end{trivlist}
    }
    {  % \begin{feedback}[solution]   \BODY     \end{feedback}  }
    }
    }    
\else
% ONLY for HTML; xmoplossing is styled with css, and is not, and need not be a LaTeX environment
% THUS: it does NOT use feedback anymore ...
%    \NewEnviron{oplossing}{\begin{expandable}{xmoplossing}{\nlen{Toon uitwerking}{Show solution}}{\BODY}\end{expandable}}
    \newenvironment{oplossing}[1][onzichtbaar]
   {%
       \begin{expandable}{xmoplossing}{}
   }
   {%
   	   \end{expandable}
   } 
%     \newenvironment{oplossing}[1][onzichtbaar]
%    {%
%        \begin{feedback}[solution]   	
%    }
%    {%
%    	   \end{feedback}
%    } 
\fi

\ifhandout%
    \NewEnviron{uitkomst}[1][onzichtbaar]%
    {%
    \ifthenelse{\equal{\detokenize{#1}}{\detokenize{toon}}}
    {
    \def\PH@Command{#1}% Use PH@Command to hold the content and be a target for "\expandafter" to expand once.

    \begin{trivlist}% Begin the trivlist to use formating of the "Feedback" label.
    \item[\hskip \labelsep\small\slshape\bfseries Uitkomst:% Format the "Feedback" label. Don't forget the space.
    %(\texttt{\detokenize\expandafter{\PH@Command}}):% Format (and detokenize) the condition for feedback to trigger
    \hspace{2ex}]\small%\slshape% Insert some space before the actual feedback given.
    \BODY
    \end{trivlist}
    }
    {  % \begin{feedback}[solution]   \BODY     \end{feedback}  }
    }
    }    
\else
\ifdefined\HCode
   \newenvironment{uitkomst}[1][onzichtbaar]
    {%
        \begin{expandable}{xmuitkomst}{}%
    }
    {%
    	\end{expandable}%
    } 
\else
  % Do NOT print 'uitkomst' in non-handout
  %  (presumably, there is also an 'oplossing' ??)
  \newenvironment{uitkomst}[1][onzichtbaar]{}{}
\fi
\fi

%
% Uitweidingen zijn extra's die niet redelijkerwijze tot de leerstof behoren
% Uitbreidingen zijn extra's die wel redelijkerwijze tot de leerstof van bv meer geavanceerde versies kunnen behoren (B-programma/Wiskundestudenten/...?)
% Nog niet voorzien: design voor verschillende versies (A/B programma, BIO, voorkennis/ ...)
% Voor 'uitweidingen' is er een environment die online per default is ingeklapt, en in pdf al dan niet kan worden geincluded  (via \xmnouitweiding) 
%
% in een xourse, per default GEEN uitweidingen, tenzij \xmuitweiding expliciet ergens is gezet ...
\ifdefined\isXourse
   \ifdefined\xmuitweiding
   \else
       \def\xmnouitweiding{true}
   \fi
\fi

\ifdefined\xmnouitweiding
\newcounter{xmuitweiding}  % anders error undefined ...  
\excludecomment{xmuitweiding}
\else
\newtheoremstyle{dotless}{}{}{}{}{}{}{ }{}
\theoremstyle{dotless}
\newtheorem*{xmuitweidingnofrills}{}   % nofrills = no accordion; gebruikt dus de dotless theoremstyle!

\newcounter{xmuitweiding}
\newenvironment{xmuitweiding}[1][ ]%
{% 
	\refstepcounter{xmuitweiding}%
    \begin{expandable}{xmuitweiding}{\nlentext{Uitweiding \arabic{xmuitweiding}: #1}{Digression \arabic{xmuitweiding}: #1}}%
	\begin{xmuitweidingnofrills}%
}
{%
    \end{xmuitweidingnofrills}%
    \end{expandable}%
}   
% \newenvironment{xmuitweiding}[1][ ]%
% {% 
% 	\refstepcounter{xmuitweiding}
% 	\begin{accordion}\begin{accordion-item}[Uitweiding \arabic{xmuitweiding}: #1]%
% 			\begin{xmuitweidingnofrills}%
% 			}
% 			{\end{xmuitweidingnofrills}\end{accordion-item}\end{accordion}}   
\fi


\newenvironment{xmexpandable}[1][]{
	\begin{accordion}\begin{accordion-item}[#1]%
		}{\end{accordion-item}\end{accordion}}


% Command that gives a selection box online, but just prints the right answer in pdf
\newcommand{\xmonlineChoice}[1]{\pdfOnly{\wordchoicegiventrue}\wordChoice{#1}\pdfOnly{\wordchoicegivenfalse}}   % deprecated, gebruik onlineChoice ...
\newcommand{\onlineChoice}[1]{\pdfOnly{\wordchoicegiventrue}\wordChoice{#1}\pdfOnly{\wordchoicegivenfalse}}

\newcommand{\TJa}{\nlentext{ Ja }{ Yes }}
\newcommand{\TNee}{\nlentext{ Nee }{ No }}
\newcommand{\TJuist}{\nlentext{ Juist }{ True }}
\newcommand{\TFout}{\nlentext{ Fout }{ False }}

\newcommand{\choiceTrue }{{\renewcommand{\choiceminimumhorizontalsize}{4em}\wordChoice{\choice[correct]{\TJuist}\choice{\TFout}}}}
\newcommand{\choiceFalse}{{\renewcommand{\choiceminimumhorizontalsize}{4em}\wordChoice{\choice{\TJuist}\choice[correct]{\TFout}}}}

\newcommand{\choiceYes}{{\renewcommand{\choiceminimumhorizontalsize}{3em}\wordChoice{\choice[correct]{\TJa}\choice{\TNee}}}}
\newcommand{\choiceNo }{{\renewcommand{\choiceminimumhorizontalsize}{3em}\wordChoice{\choice{\TJa}\choice[correct]{\TNee}}}}

% Optional nicer formatting for wordChoice in PDF

\let\inlinechoiceorig\inlinechoice

%\makeatletter
%\providecommand{\choiceminimumverticalsize}{\vphantom{$\frac{\sqrt{2}}{2}$}}   % minimum height of boxes (cfr infra)
\providecommand{\choiceminimumverticalsize}{\vphantom{$\tfrac{2}{2}$}}   % minimum height of boxes (cfr infra)
\providecommand{\choiceminimumhorizontalsize}{1em}   % minimum width of boxes (cfr infra)

\newcommand{\inlinechoicesquares}[2][]{%
		\setkeys{choice}{#1}%
		\ifthenelse{\boolean{\choice@correct}}%
		{%
            \ifhandout%
               \fbox{\choiceminimumverticalsize #2}\allowbreak\ignorespaces%
            \else%
               \fcolorbox{blue}{blue!20}{\choiceminimumverticalsize #2}\allowbreak\ignorespaces\setkeys{choice}{correct=false}\ignorespaces%
            \fi%
		}%
		{% else
			\fbox{\choiceminimumverticalsize #2}\allowbreak\ignorespaces%  HACK: wat kleiner, zodat fits on line ... 	
		}%
}

\newcommand{\inlinechoicesquareX}[2][]{%
		\setkeys{choice}{#1}%
		\ifthenelse{\boolean{\choice@correct}}%
		{%
            \ifhandout%
               \framebox[\ifdim\choiceminimumhorizontalsize<\width\width\else\choiceminimumhorizontalsize\fi]{\choiceminimumverticalsize\ #2\ }\allowbreak\ignorespaces\setkeys{choice}{correct=false}\ignorespaces%
            \else%
               \fcolorbox{blue}{blue!20}{\makebox[\ifdim\choiceminimumhorizontalsize<\width\width\else\choiceminimumhorizontalsize\fi]{\choiceminimumverticalsize #2}}\allowbreak\ignorespaces\setkeys{choice}{correct=false}\ignorespaces%
            \fi%
		}%
		{% else
        \ifhandout%
			\framebox[\ifdim\choiceminimumhorizontalsize<\width\width\else\choiceminimumhorizontalsize\fi]{\choiceminimumverticalsize\ #2\ }\allowbreak\ignorespaces%  HACK: wat kleiner, zodat fits on line ... 	
        \fi
		}%
}


\newcommand{\inlinechoicepuntjes}[2][]{%
		\setkeys{choice}{#1}%
		\ifthenelse{\boolean{\choice@correct}}%
		{%
            \ifhandout%
               \dots\ldots\ignorespaces\setkeys{choice}{correct=false}\ignorespaces
            \else%
               \fcolorbox{blue}{blue!20}{\choiceminimumverticalsize #2}\allowbreak\ignorespaces\setkeys{choice}{correct=false}\ignorespaces%
            \fi%
		}%
		{% else
			%\fbox{\choiceminimumverticalsize #2}\allowbreak\ignorespaces%  HACK: wat kleiner, zodat fits on line ... 	
		}%
}

% print niets, maar definieer globale variable \myanswer
%  (gebruikt om oplossingsbladen te printen) 
\newcommand{\inlinechoicedefanswer}[2][]{%
		\setkeys{choice}{#1}%
		\ifthenelse{\boolean{\choice@correct}}%
		{%
               \gdef\myanswer{#2}\setkeys{choice}{correct=false}

		}%
		{% else
			%\fbox{\choiceminimumverticalsize #2}\allowbreak\ignorespaces%  HACK: wat kleiner, zodat fits on line ... 	
		}%
}



%\makeatother

\newcommand{\setchoicedefanswer}{
\ifdefined\HCode
\else
%    \renewenvironment{multipleChoice@}[1][]{}{} % remove trailing ')'
    \let\inlinechoice\inlinechoicedefanswer
\fi
}

\newcommand{\setchoicepuntjes}{
\ifdefined\HCode
\else
    \renewenvironment{multipleChoice@}[1][]{}{} % remove trailing ')'
    \let\inlinechoice\inlinechoicepuntjes
\fi
}
\newcommand{\setchoicesquares}{
\ifdefined\HCode
\else
    \renewenvironment{multipleChoice@}[1][]{}{} % remove trailing ')'
    \let\inlinechoice\inlinechoicesquares
\fi
}
%
\newcommand{\setchoicesquareX}{
\ifdefined\HCode
\else
    \renewenvironment{multipleChoice@}[1][]{}{} % remove trailing ')'
    \let\inlinechoice\inlinechoicesquareX
\fi
}
%
\newcommand{\setchoicelist}{
\ifdefined\HCode
\else
    \renewenvironment{multipleChoice@}[1][]{}{)}% re-add trailing ')'
    \let\inlinechoice\inlinechoiceorig
\fi
}

\setchoicesquareX  % by default list-of-squares with onlineChoice in PDF

% Omdat multicols niet werkt in html: enkel in pdf  (in html zijn langere pagina's misschien ook minder storend)
\newenvironment{xmmulticols}[1][2]{
 \pdfOnly{\begin{multicols}{#1}}%
}{ \pdfOnly{\end{multicols}}}

%
% Te gebruiken in plaats van \section\subsection
%  (in een printstyle kan dan het level worden aangepast
%    naargelang \chapter vs \section style )
% 3/2021: DO NOT USE \xmsubsection !
\newcommand\xmsection\subsection
\newcommand\xmsubsection\subsubsection

% Aanpassen printversie
%  (hier gedefinieerd, zodat ze in xourse kunnen worden gezet/overschreven)
\providebool{parttoc}
\providebool{printpartfrontpage}
\providebool{printactivitytitle}
\providebool{printactivityqrcode}
\providebool{printactivityurl}
\providebool{printcontinuouspagenumbers}
\providebool{numberactivitiesbysubpart}
\providebool{addtitlenumber}
\providebool{addsectiontitlenumber}
\addtitlenumbertrue
\addsectiontitlenumbertrue

% The following three commands are hardcoded in xake, you can't create other commands like these, without adding them to xake as well
%  ( gebruikt in xourses om juiste soort titelpagina te krijgen voor verschillende ximera's )
\newcommand{\activitychapter}[2][]{
    {    
    \ifstrequal{#1}{notnumbered}{
        \addtitlenumberfalse
    }{}
    \typeout{ACTIVITYCHAPTER #2}   % logging
	\chapterstyle
	\activity{#2}
    }
}
\newcommand{\activitysection}[2][]{
    {
    \ifstrequal{#1}{notnumbered}{
        \addsectiontitlenumberfalse
    }{}
	\typeout{ACTIVITYSECTION #2}   % logging
	\sectionstyle
	\activity{#2}
    }
}
% Practices worden als activity getoond om de grote blokken te krijgen online
\newcommand{\practicesection}[2][]{
    {
    \ifstrequal{#1}{notnumbered}{
        \addsectiontitlenumberfalse
    }{}
    \typeout{PRACTICESECTION #2}   % logging
	\sectionstyle
	\activity{#2}
    }
}
\newcommand{\activitychapterlink}[3][]{
    {
    \ifstrequal{#1}{notnumbered}{
        \addtitlenumberfalse
    }{}
    \typeout{ACTIVITYCHAPTERLINK #3}   % logging
	\chapterstyle
	\activitylink[#1]{#2}{#3}
    }
}

\newcommand{\activitysectionlink}[3][]{
    {
    \ifstrequal{#1}{notnumbered}{
        \addsectiontitlenumberfalse
    }{}
    \typeout{ACTIVITYSECTIONLINK #3}   % logging
	\sectionstyle
	\activitylink[#1]{#2}{#3}
    }
}


% Commando om de printstyle toe te voegen in ximera's. Zorgt ervoor dat er geen problemen zijn als je de xourses compileert
% hack om onhandige relative paden in TeX te omzeilen
% should work both in xourse and ximera (pre-112022 only in ximera; thus obsoletes adhoc setup in xourses)
% loads global.sty if present (cfr global.css for online settings!)
% use global.sty to overwrite settings in printstyle.sty ...
\newcommand{\addPrintStyle}[1]{
\iftikzexport\else   % only in PDF
  \makeatletter
  \ifx\@onlypreamble\@notprerr\else   % ONLY if in tex-preamble   (and e.g. not when included from xourse)
    \typeout{Loading printstyle}   % logging
    \usepackage{#1/printstyle} % mag enkel geinclude worden als je die apart compileert
    \IfFileExists{#1/global.sty}{
        \typeout{Loading printstyle-folder #1/global.sty}   % logging
        \usepackage{#1/global}
        }{
        \typeout{Info: No extra #1/global.sty}   % logging
    }   % load global.sty if present
    \IfFileExists{global.sty}{
        \typeout{Loading local-folder global.sty (or TEXINPUTPATH..)}   % logging
        \usepackage{global}
    }{
        \typeout{Info: No folder/global.sty}   % logging
    }   % load global.sty if present
    \IfFileExists{\currfilebase.sty}
    {
        \typeout{Loading \currfilebase.sty}
        \input{\currfilebase.sty}
    }{
        \typeout{Info: No local \currfilebase.sty}
    }
    \fi
  \makeatother
\fi
}

%
%  
% references: Ximera heeft adhoc logica	 om online labels te doen werken over verschillende files heen
% met \hyperref kan de getoonde tekst toch worden opgegeven, in plaats van af te hangen van de label-text
\ifdefined\HCode
% Link to standard \labels, but give your own description
% Usage:  Volg \hyperref[my_very_verbose_label]{deze link} voor wat tijdverlies
%   (01/2020: Ximera-server aangepast om bij class reference-keeptext de link-text NIET te vervangen door de label-text !!!) 
\renewcommand{\hyperref}[2][]{\HCode{<a class="reference reference-keeptext" href="\##1">}#2\HCode{</a>}}
%
%  Link to specific targets  (not tested ?)
\renewcommand{\hypertarget}[1]{\HCode{<a class="ximera-label" id="#1"></a>}}
\renewcommand{\hyperlink}[2]{\HCode{<a class="reference reference-keeptext" href="\##1">}#2\HCode{</a>}}
\fi

% Mmm, quid English ... (for keyword #1 !) ?
\newcommand{\wikilink}[2]{
    \hyperlink{https://nl.wikipedia.org/wiki/#1}{#2}
    \pdfOnly{\footnote{See \url{https://nl.wikipedia.org/wiki/#1}}
    }
}

\renewcommand{\figurename}{Figuur}
\renewcommand{\tablename}{Tabel}

%
% Gedoe om verschillende versies van xourse/ximera te maken afhankelijk van settings
%
% default: versie met antwoorden
% handout: versie voor de studenten, zonder antwoorden/oplossingen
% full: met alles erop en eraan, dus geschikt voor auteurs en/of lesgevers  (bevat in de pdf ook de 'online-only' stukken!)
%
%
% verder kunnen ook opties/variabele worden gezet voor hints/auteurs/uitweidingen/ etc
%
% 'Full' versie
\newtoggle{showonline}
\ifdefined\HCode   % zet default showOnline
    \toggletrue{showonline} 
\else
    \togglefalse{showonline}
\fi

% Full versie   % deprecated: see infra
\newcommand{\printFull}{
    \hintstrue
    \handoutfalse
    \toggletrue{showonline} 
}

\ifdefined\shouldPrintFull   % deprecated: see infra
    \printFull
\fi



% Overschrijf onlineOnly  (zoals gedefinieerd in ximera.cls)
\ifhandout   % in handout: gebruik de oorspronkelijke ximera.cls implementatie  (is dit wel nodig/nuttig?)
\else
    \iftoggle{showonline}{%
        \ifdefined\HCode
          \RenewEnviron{onlineOnly}{\bgroup\BODY\egroup}   % showOnline, en we zijn  online, dus toon de tekst
        \else
          \RenewEnviron{onlineOnly}{\bgroup\color{red!50!black}\BODY\egroup}   % showOnline, maar we zijn toch niet online: kleur de tekst rood 
        \fi
    }{%
      \RenewEnviron{onlineOnly}{}  % geen showOnline
    }
\fi

% hack om na hoofding van definition/proposition/... als dan niet op een nieuwe lijn te starten
% soms is dat goed en mooi, en soms niet; en in HTML is het nu (2/2020) anders dan in pdf
% vandaar suggestie om 
%     \begin{definition}[Nieuw concept] \nl
% te gebruiken als je zeker een newline wil na de hoofdig en titel
% (in het bijzonder itemize zonder \nl is 'lelijk' ...)
\ifdefined\HCode
\newcommand{\nl}{}
\else
\newcommand{\nl}{\ \par} % newline (achter heading van definition etc.)
\fi


% \nl enkel in handoutmode (ihb voor \wordChoice, die dan typisch veeeel langer wordt)
\ifdefined\HCode
\providecommand{\handoutnl}{}
\else
\providecommand{\handoutnl}{%
\ifhandout%
  \nl%
\fi%
}
\fi

% Could potentially replace \pdfOnline/\begin{onlineOnly} : 
% Usage= \ifonline{Hallo surfer}{Hallo PDFlezer}
\providecommand{\ifonline}[2]%
{
\begin{onlineOnly}#1\end{onlineOnly}%
\pdfOnly{#2}
}%


%
% Maak optionele 'basic' en 'extended' versies van een activity
%  met environment basicOnly, basicSkip en extendedOnly
%
%  (
%   Dit werkt ENKEL in de PDF; de online versies tonen (minstens voorklopig) steeds 
%   het default geval met printbasicversion en printextendversion beide FALSE
%  )
%
\providebool{printbasicversion}
\providebool{printextendedversion}   % not properly implemented
\providebool{printfullversion}       % presumably print everything (debug/auteur)
%
% only set these in xourses, and BEFORE loading this preamble
%
%\newif\ifshowbasic     \showbasictrue        % use this line in xourse to show 'basic' sections
%\newif\ifshowextended  \showextendedtrue     % use this line in xourse to show 'extended' sections
%
%
%\ifbool{showbasic}
%      { \NewEnviron{basicOnly}{\BODY} }    % if yes: just print contents
%      { \NewEnviron{basicOnly}{}      }    % if no:  completely ignore contents
%
%\ifbool{showbasic}
%      { \NewEnviron{basicSkip}{}      }
%      { \NewEnviron{basicSkip}{\BODY} }
%

\ifbool{printextendedversion}
      { \NewEnviron{extendedOnly}{\BODY} }
      { \NewEnviron{extendedOnly}{}      }
      


\ifdefined\HCode    % in html: always print
      {\newenvironment*{basicOnly}{}{}}    % if yes: just print contents
      {\newenvironment*{basicSkip}{}{}}    % if yes: just print contents
\else

\ifbool{printbasicversion}
      {\newenvironment*{basicOnly}{}{}}    % if yes: just print contents
      {\NewEnviron{basicOnly}{}      }    % if no:  completely ignore contents

\ifbool{printbasicversion}
      {\NewEnviron{basicSkip}{}      }
      {\newenvironment*{basicSkip}{}{}}

\fi

\usepackage{float}
\usepackage[rightbars,color]{changebar}

% Full versie
\ifbool{printfullversion}{
    \hintstrue
    \handoutfalse
    \toggletrue{showonline}
    \printbasicversionfalse
    \cbcolor{red}
    \renewenvironment*{basicOnly}{\cbstart}{\cbend}
    \renewenvironment*{basicSkip}{\cbstart}{\cbend}
    \def\xmtoonprintopties{FULL}   % will be printed in footer
}
{}
      
%
% Evalueer \ifhints IN de environment
%  
%
%\RenewEnviron{hint}
%{
%\ifhandout
%\ifhints\else\setbox0\vbox\fi%everything in een emty box
%\bgroup 
%\stepcounter{hintLevel}
%\BODY
%\egroup\ignorespacesafterend
%\addtocounter{hintLevel}{-1}
%\else
%\ifhints
%\begin{trivlist}\item[\hskip \labelsep\small\slshape\bfseries Hint:\hspace{2ex}]
%\small\slshape
%\stepcounter{hintLevel}
%\BODY
%\end{trivlist}
%\addtocounter{hintLevel}{-1}
%\fi
%\fi
%}

% Onafhankelijk van \ifhandout ...? TO BE VERIFIED
\RenewEnviron{hint}
{
\ifhints
\begin{trivlist}\item[\hskip \labelsep\small\bfseries Hint:\hspace{2ex}]
\small%\slshape
\stepcounter{hintLevel}
\BODY
\end{trivlist}
\addtocounter{hintLevel}{-1}
\else
\iftikzexport   % anders worden de tikz tekeningen in hints niet gegenereerd ?
\setbox0\vbox\bgroup
\stepcounter{hintLevel}
\BODY
\egroup\ignorespacesafterend
\addtocounter{hintLevel}{-1}
\fi % ifhandout
\fi %ifhints
}

%
% \tab sets typewriter-tabs (e.g. to format questions)
% (Has no effect in HTML :-( ))
%
\usepackage{tabto}
\ifdefined\HCode
  \renewcommand{\tab}{\quad}    % otherwise dummy .png's are generated ...?
\fi


% Also redefined in  preamble to get correct styling 
% for tikz images for (\tikzexport)
%

\theoremstyle{definition} % Bold titels
\makeatletter
\let\proposition\relax
\let\c@proposition\relax
\let\endproposition\relax
\makeatother
\newtheorem{proposition}{Eigenschap}


%\instructornotesfalse

% logic with \ifhandoutin ximera.cls unclear;so overwrite ...
\makeatletter
\@ifundefined{ifinstructornotes}{%
  \newif\ifinstructornotes
  \instructornotesfalse
  \newenvironment{instructorNotes}{}{}
}{}
\makeatother
\ifinstructornotes
\else
\renewenvironment{instructorNotes}%
{%
    \setbox0\vbox\bgroup
}
{%
    \egroup
}
\fi

% \RedeclareMathOperator
% from https://tex.stackexchange.com/questions/175251/how-to-redefine-a-command-using-declaremathoperator
\makeatletter
\newcommand\RedeclareMathOperator{%
    \@ifstar{\def\rmo@s{m}\rmo@redeclare}{\def\rmo@s{o}\rmo@redeclare}%
}
% this is taken from \renew@command
\newcommand\rmo@redeclare[2]{%
    \begingroup \escapechar\m@ne\xdef\@gtempa{{\string#1}}\endgroup
    \expandafter\@ifundefined\@gtempa
    {\@latex@error{\noexpand#1undefined}\@ehc}%
    \relax
    \expandafter\rmo@declmathop\rmo@s{#1}{#2}}
% This is just \@declmathop without \@ifdefinable
\newcommand\rmo@declmathop[3]{%
    \DeclareRobustCommand{#2}{\qopname\newmcodes@#1{#3}}%
}
\@onlypreamble\RedeclareMathOperator
\makeatother


%
% Engelse vertaling, vooral in mathmode
%
% 1. Algemene procedure
%
\ifdefined\isEn
 \newcommand{\nlen}[2]{#2}
 \newcommand{\nlentext}[2]{\text{#2}}
 \newcommand{\nlentextbf}[2]{\textbf{#2}}
\else
 \newcommand{\nlen}[2]{#1}
 \newcommand{\nlentext}[2]{\text{#1}}
 \newcommand{\nlentextbf}[2]{\textbf{#1}}
\fi

%
% 2. Lijst van erg veel gebruikte uitdrukkingen
%

% Ja/Nee/Fout/Juits etc
%\newcommand{\TJa}{\nlentext{ Ja }{ and }}
%\newcommand{\TNee}{\nlentext{ Nee }{ No }}
%\newcommand{\TJuist}{\nlentext{ Juist }{ Correct }
%\newcommand{\TFout}{\nlentext{ Fout }{ Wrong }
\newcommand{\TWaar}{\nlentext{ Waar }{ True }}
\newcommand{\TOnwaar}{\nlentext{ Vals }{ False }}
% Korte bindwoorden en, of, dus, ...
\newcommand{\Ten}{\nlentext{ en }{ and }}
\newcommand{\Tof}{\nlentext{ of }{ or }}
\newcommand{\Tdus}{\nlentext{ dus }{ so }}
\newcommand{\Tendus}{\nlentext{ en dus }{ and thus }}
\newcommand{\Tvooralle}{\nlentext{ voor alle }{ for all }}
\newcommand{\Took}{\nlentext{ ook }{ also }}
\newcommand{\Tals}{\nlentext{ als }{ when }} %of if?
\newcommand{\Twant}{\nlentext{ want }{ as }}
\newcommand{\Tmaal}{\nlentext{ maal }{ times }}
\newcommand{\Toptellen}{\nlentext{ optellen }{ add }}
\newcommand{\Tde}{\nlentext{ de }{ the }}
\newcommand{\Thet}{\nlentext{ het }{ the }}
\newcommand{\Tis}{\nlentext{ is }{ is }} %zodat is in text staat in mathmode (geen italics)
\newcommand{\Tmet}{\nlentext{ met }{ where }} % in situaties e.g met p < n --> where p < n
\newcommand{\Tnooit}{\nlentext{ nooit }{ never }}
\newcommand{\Tmaar}{\nlentext{ maar }{ but }}
\newcommand{\Tniet}{\nlentext{ niet }{ not }}
\newcommand{\Tuit}{\nlentext{ uit }{ from }}
\newcommand{\Ttov}{\nlentext{ t.o.v. }{ w.r.t. }}
\newcommand{\Tzodat}{\nlentext{ zodat }{ such that }}
\newcommand{\Tdeth}{\nlentext{de }{th }}
\newcommand{\Tomdat}{\nlentext{omdat }{because }} 


%
% Overschrijf addhoc commando's
%
\ifdefined\isEn
\renewcommand{\pernot}{\overset{\mathrm{notation}}{=}}
\RedeclareMathOperator{\bld}{im}     % beeld
\RedeclareMathOperator{\graf}{graph}   % grafiek
\RedeclareMathOperator{\rico}{slope}   % richtingcoëfficient
\RedeclareMathOperator{\co}{co}       % coordinaat
\RedeclareMathOperator{\gr}{deg}       % graad

% Operators
\RedeclareMathOperator{\bgsin}{arcsin}
\RedeclareMathOperator{\bgcos}{arccos}
\RedeclareMathOperator{\bgtan}{arctan}
\RedeclareMathOperator{\bgcot}{arccot}
\RedeclareMathOperator{\bgsinh}{arcsinh}
\RedeclareMathOperator{\bgcosh}{arccosh}
\RedeclareMathOperator{\bgtanh}{arctanh}
\RedeclareMathOperator{\bgcoth}{arccoth}

\fi


% HACK: use 'oplossing' for 'explanation' ...
\let\explanation\relax
\let\endexplanation\relax
% \newenvironment{explanation}{\begin{oplossing}}{\end{oplossing}}
\newcounter{explanation}

\ifhandout%
    \NewEnviron{explanation}[1][toon]%
    {%
    \RenewEnviron{verbatim}{ \red{VERBATIM CONTENT MISSING IN THIS PDF}} %% \expandafter\verb|\BODY|}

    \ifthenelse{\equal{\detokenize{#1}}{\detokenize{toon}}}
    {
    \def\PH@Command{#1}% Use PH@Command to hold the content and be a target for "\expandafter" to expand once.

    \begin{trivlist}% Begin the trivlist to use formating of the "Feedback" label.
    \item[\hskip \labelsep\small\slshape\bfseries Explanation:% Format the "Feedback" label. Don't forget the space.
    %(\texttt{\detokenize\expandafter{\PH@Command}}):% Format (and detokenize) the condition for feedback to trigger
    \hspace{2ex}]\small%\slshape% Insert some space before the actual feedback given.
    \BODY
    \end{trivlist}
    }
    {  % \begin{feedback}[solution]   \BODY     \end{feedback}  }
    }
    }    
\else
% ONLY for HTML; xmoplossing is styled with css, and is not, and need not be a LaTeX environment
% THUS: it does NOT use feedback anymore ...
%    \NewEnviron{oplossing}{\begin{expandable}{xmoplossing}{\nlen{Toon uitwerking}{Show solution}}{\BODY}\end{expandable}}
    \newenvironment{explanation}[1][toon]
   {%
       \begin{expandable}{xmoplossing}{}
   }
   {%
   	   \end{expandable}
   } 
\fi

 \title{LU Factorization} \license{CC BY-NC-SA 4.0}

\begin{document}
\begin{abstract}

\end{abstract}
\maketitle
\section*{$LU$ factorization}

\begin{exploration}\label{init:LUfactorization}
Consider the equation:
$$\begin{bmatrix}1&0&0\\3&1&0\\-2&2&1\end{bmatrix}\begin{bmatrix}2&-1&2\\0&4&-1\\0&0&3\end{bmatrix}\vec{x}=\begin{bmatrix}1\\8\\5\end{bmatrix}$$
This equation is unusual in that it involves two matrices on the left-hand side.  If we multiply the matrices together, we get
$$\begin{bmatrix}2&-1&2\\6&1&5\\-4&10&-3\end{bmatrix}\vec{x}=\begin{bmatrix}1\\8\\5\end{bmatrix}$$
Gaussian elimination yields:
$$\left[\begin{array}{ccc|c}
2&-1&2&1\\6&1&5&8\\-4&10&-3&5
\end{array}\right]
\quad\rightsquigarrow\quad
\left[\begin{array}{ccc|c}
 1&0&0&2\\0&1&0&1\\0&0&1&-1
\end{array}\right]
$$
We conclude that
$$\vec{x}=\begin{bmatrix}2\\1\\-1\end{bmatrix}.$$

Now that we know the answer, we will turn to the original question and consider the advantages of the format of the original problem.  Observe that the two matrices have a distinct pattern.  The matrix on the left has zeros above the main diagonal, while the matrix on the right has zeros below the main diagonal.  Matrices that follow such a pattern are called \dfn{lower triangular} and \dfn{upper triangular} matrices, respectively.

Let $\vec{y}=\begin{bmatrix}2&-1&2\\0&4&-1\\0&0&3\end{bmatrix}\vec{x}$.  Then we can write the original equation as
$$\begin{bmatrix}1&0&0\\3&1&0\\-2&2&1\end{bmatrix}\left(\begin{bmatrix}2&-1&2\\0&4&-1\\0&0&3\end{bmatrix}\vec{x}\right)=\begin{bmatrix}1&0&0\\3&1&0\\-2&2&1\end{bmatrix}\vec{y}=\begin{bmatrix}1\\8\\5\end{bmatrix}$$
The equation 
$$\begin{bmatrix}1&0&0\\3&1&0\\-2&2&1\end{bmatrix}\vec{y}=\begin{bmatrix}1&0&0\\3&1&0\\-2&2&1\end{bmatrix}\begin{bmatrix}y_1\\y_2\\y_3\end{bmatrix}=\begin{bmatrix}1\\8\\5\end{bmatrix}$$ 
corresponds to the system
$$\begin{matrix}
	 y_1&&&=&1\\
	   3y_1 & +y_2&&= &8 \\
      -2y_1&+2y_2&+y_3&=&5
    \end{matrix}$$
    
This system can be easily solved by substitution, giving us
$$y_1=1$$
$$3(1)+y_2=8 \rightsquigarrow y_2=5$$
$$-2(1)+2(5)+y_3=5 \rightsquigarrow y_3=-3$$
So,
$$\vec{y}=\begin{bmatrix}y_1\\y_2\\y_3\end{bmatrix}=\begin{bmatrix}1\\5\\-3\end{bmatrix}$$
    
Recall that we let $\vec{y}=\begin{bmatrix}2&-1&2\\0&4&-1\\0&0&3\end{bmatrix}\vec{x}$.  
The equation
$$\begin{bmatrix}2&-1&2\\0&4&-1\\0&0&3\end{bmatrix}\vec{x}=\begin{bmatrix}1\\5\\-3\end{bmatrix}$$
corresponds to the system
$$\begin{matrix}
	 2x_1& -&x_2&+&2x_3&=&1\\
	   & &4x_2&-&x_3&= &5 \\
      & &&&3x_3&=&-3
    \end{matrix}$$
    which can be easily solved by back-substitution:
    $$x_3=\answer{-1}$$
    $$x_2=\answer{1}$$
    $$x_1=\answer{2}$$
    
    Thus, $\vec{x}=\begin{bmatrix}2\\1\\-1\end{bmatrix}$.  Observe that this is the same answer that we obtained in the beginning of the problem using Gaussian elimination.  
\end{exploration}

Given a matrix equation such as 
$$\begin{bmatrix}2&-1&2\\6&1&5\\-4&10&-3\end{bmatrix}\vec{x}=\begin{bmatrix}1\\8\\5\end{bmatrix}$$
of Exploration \ref{init:LUfactorization}, often in practice we express the matrix on the left as a product of an upper triangular matrix $U$ and a lower triangular matrix $L$ and use substitution to solve the equation instead of using Gaussian elimination to find the solution.  In Exploration \ref{init:LUfactorization}, 

$$\begin{bmatrix}2&-1&2\\6&1&5\\-4&10&-3\end{bmatrix}=LU=\begin{bmatrix}1&0&0\\3&1&0\\-2&2&1\end{bmatrix}\begin{bmatrix}2&-1&2\\0&4&-1\\0&0&3\end{bmatrix}$$

The process of taking a matrix $A$ and expressing it as a product $A=LU$ of a lower triangular matrix $L$ and an upper triangular matrix $U$ is called \dfn{$LU$ factorization}.
%An $LU$ factorization of a matrix $A$ is when we express the given matrix as the
%product
%$$A=LU$$
%where  $L$ is a lower triangular square matrix and $U$ is an upper triangular matrix.  
We also adopt the (common) convention that $L$ is a matrix with a diagonal consisting entirely of $1$'s, which is often called a \dfn{lower unit triangular} matrix.

Not every matrix has an $LU$ factorization, but we will see that we can correct for that.

\begin{example}\label{ex:usingLU}
Consider the system
$$\begin{array}{ccccccccc}
        & &12x_2&-&18x_3&= &6 \\
	 x_1 &+ &2x_2&-&3x_3&= &1\\
      x_1&- &2x_2&+&x_3&=&1
    \end{array}.$$
We can express this system as the matrix equation $A\vec{x}=\vec{b}$, and we get $$\begin{bmatrix}0&12&-18\\1&2&-3\\1&-2&1\end{bmatrix}\vec{x}=\begin{bmatrix}6\\1\\1\end{bmatrix}.$$

Unfortunately, for reasons we will explain below, it is not possible to find an $LU$ factorization of this coefficient matrix.  However, if we simply swap the first two equations:
$$\begin{array}{ccccccccc}
	 x_1 &+ &2x_2&-&3x_3&= &1\\
	  & &12x_2&-&18x_3&= &6 \\
      x_1&- &2x_2&+&x_3&=&1
    \end{array},$$
we are able to find an $LU$ factorization of the coefficient matrix.
\[
\begin{bmatrix}
1 & 2 & -3 \\
0 & 12 & -18 \\
1 & -2 & 1\end{bmatrix}
=
\begin{bmatrix}
1 & 0 & 0 \\
0 & 1 & 0 \\
1  & -1/3  & 1
\end{bmatrix} 
\begin{bmatrix}
1 & 2 & -3 \\
0 & 12  & -18 \\
0 & 0 & -2
\end{bmatrix}
\]
(How we actually come up with an  $LU$ factorization of $A$ will be discussed after this example.)

To understand how an $LU$ factorization can be used, it is helpful to think of this system of equations as a matrix equation.
$$A\vec{x}=\vec{b}$$
$$(LU)\vec{x}=\vec{b}$$
$$L(U\vec{x})=\vec{b}$$
So if we let $\vec{y}=U\vec{x}$ and also write $\vec{y}=\begin{bmatrix} y_1 \\ y_2 \\ y_3 \end{bmatrix}$, then we are able to solve for $ y_1, y_2, \text{ and } y_3$ using \emph{forward-substitution}.
$$L\vec{y}=\vec{b}$$
\[
\begin{bmatrix}
1 & 0 & 0 \\
0 & 1 & 0 \\
1  & -1/3  & 1
\end{bmatrix} 
\begin{bmatrix} y_1 \\ y_2 \\ y_3 \end{bmatrix}
=
\begin{bmatrix} 1 \\ 6 \\ 1 \end{bmatrix}
\]
See if you can do it!
$$y_1=\answer{1},\quad y_2=\answer{6},\quad y_3=\answer{2}$$
Once we have the values of $\vec{y}$, since $\vec{y}=U\vec{x}$, all that remains is to solve the following matrix equation using back-substitution.
$$U\vec{x}=\vec{y}$$

\begin{expandable}
$$\begin{bmatrix}
1 & 2 & -3 \\
0 & 12  & -18 \\
0 & 0 & -2
\end{bmatrix} 
\begin{bmatrix} x_1 \\ x_2 \\ x_3 \end{bmatrix}
=
\begin{bmatrix} 1 \\ 6 \\ 2 \end{bmatrix}$$


By now you are quite used to solving this kind of problem...
$$x_1=\answer{0},\quad x_2=\answer{-1},\quad x_3=\answer{-1}$$
\end{expandable}
\end{example}


\subsection*{Finding an $LU$ factorization by Inspection}

\begin{example}\label{ex:LU1}
Find an $LU$ factorization of $A=
\begin{bmatrix}
1 & 2 & 0 & 2 \\
1 & 3 & 2 & 1 \\
2 & 3 & 4 & 0
\end{bmatrix}
.$
\end{example}

One way to find the $LU$ factorization%
 is to simply look for it directly.
We need
\[
\begin{bmatrix}
1 & 2 & 0 & 2 \\
1 & 3 & 2 & 1 \\
2 & 3 & 4 & 0
\end{bmatrix}
=
\begin{bmatrix}
1 & 0 & 0 \\
x & 1 & 0 \\
y & z & 1
\end{bmatrix} 
\begin{bmatrix}
a & d & h & j \\
0 & b & e & i \\
0 & 0 & c & f
\end{bmatrix}
\]
By multiplying the latter matrices we get
\[
\begin{bmatrix}
a & d & h & j \\
xa & xd+b & xh+e & xj+i \\
ya & yd+zb & yh+ze+c & yj+iz+f
\end{bmatrix}
\]
and from this, it is a simple exercise to determine each of the unknowns. For example, from the first
column, we get $a=1$ and then $x=1,y=2.$ 

% Now go to the second column. You need $
% d=2,xd+b=3$ so $b=1,yd+zb=3$ so $z=-1.$ From the third column, $h=0,e=2,c=6.$
% Now from the fourth column, $j=2,i=-1,f=-5.$ 

See if you can continue to determine the entire $LU$
factorization of $A$.
\[
\begin{bmatrix}
1 & 2 & 0 & 2 \\
1 & 3 & 2 & 1 \\
2 & 3 & 4 & 0
\end{bmatrix}
=
\begin{bmatrix}
1 & 0 & 0 \\
1 & 1 & 0 \\
2 & \answer{-1} & 1
\end{bmatrix} 
\begin{bmatrix}
1 & \answer{2} & \answer{0} & \answer{2} \\
0 & \answer{1} & \answer{2} & \answer{-1} \\
0 & 0 & \answer{6} & \answer{-5}
\end{bmatrix}
\]

\subsection*{Finding an $LU$ factorization by the Multiplier Method}

In the following example we demonstrate a method for computing an $LU$ factorization known as the multiplier method.

\begin{example}\label{ex:multiplierLU}
Find an $LU$ factorization for
\[
\begin{bmatrix}
1 & 2 & 3 \\
2 & 3 & 1 \\
-2 & 3 & -2
\end{bmatrix} 
\]

\begin{explanation}
Write the matrix as the following product.
\[
\begin{bmatrix}
1 & 0 & 0 \\
0 & 1 & 0 \\
0 & 0 & 1
\end{bmatrix} 
\begin{bmatrix}
1 & 2 & 3 \\
2 & 3 & 1 \\
-2 & 3 & -2
\end{bmatrix} 
\]
In the matrix on the right, we begin with the left row and zero
out the entries below the top using the row operation which involves adding a multiple of a row to another row. You do this and also update the matrix on the left so that the product will be unchanged. 

Here is the first step.  We take $-2$ times the top row and add to the second. Then take $2$ times the top row and add to the second in the matrix on the left.  This give us
\[
\begin{bmatrix}
1 & 0 & 0 \\
2 & 1 & 0 \\
0 & 0 & 1
\end{bmatrix} 
\begin{bmatrix}
1 & 2 & 3 \\
0 & -1 & -5 \\
-2 & 3 & -2
\end{bmatrix} 
\]

The next step is to take $2$ times the top row and add to the bottom in the matrix on the right. To ensure that the product is unchanged, we place a $-2$ in the bottom left corner in the matrix on the left. Thus the second step yields
\[
\begin{bmatrix}
1 & 0 & 0 \\
2 & 1 & 0 \\
-2 & 0 & 1
\end{bmatrix} 
\begin{bmatrix}
1 & 2 & 3 \\
0 & -1 & -5 \\
0 & 7 & 4
\end{bmatrix} 
\]

For our final step, we take $7$ times the middle row on right and add to bottom row. Updating the matrix on the left in the manner we did earlier, we get
\[
\begin{bmatrix}
1 & 0 & 0 \\
2 & 1 & 0 \\
-2 & -7 & 1
\end{bmatrix} 
\begin{bmatrix}
1 & 2 & 3 \\
0 & -1 & -5 \\
0 & 0 & -31
\end{bmatrix} 
\]
At this point, we can stop. We have an $LU$ factorization of the original matrix.  We can always multiply the matrices to check, if we wish.
\end{explanation}
\end{example}

Notice that when we perform the multiplier method, we are making repeated use of a certain type of elementary row operation, namely, we are adding a scalar multiple of one row to a row below it.  The reason this helps to create an $LU$ factorization depends upon the fact that the elementary matrices corresponding to such operations are lower triangular.  To understand how this works, we begin with a lemma. 

\begin{lemma}\label{lem:multipliermethodtriangularmatrices}
Let $L$ be a lower (upper) triangular matrix $m\times m$
which has ones down the main diagonal. Then $L^{-1}$ also is a lower (upper)
triangular matrix which has ones down the main diagonal. In the case that $L$
is of the form
\begin{equation}\label{4nove1h}
L=
\begin{bmatrix}
1 &  &  &  \\
a_{1} & 1 &  &  \\
\vdots &  & \ddots &  \\
a_{n} &  &  & 1
\end{bmatrix}, 
\end{equation}
where all entries are zero except for the left column and main diagonal, it
is also the case that $L^{-1}$ is obtained from $L$ by simply multiplying each entry below the main diagonal in $L$ by $-1$. The same is true if the single nonzero column is in another position.
\end{lemma}

\begin{proof}Consider the usual setup for finding the inverse $\begin{bmatrix}
L &|& I
\end{bmatrix}$.  Each row operation used on $L$ to transform this matrix to reduced row echelon form changes only the entries in $I$ below the main diagonal. Whether we have the special case of $L$ given in \ref{4nove1h} where the nonzero nondiagonal entries are in the left column, or if the single
nonzero column is in another position, it is clear that multiplication by $-1$ as described in the lemma gives us $L^{-1}$.
\end{proof}

For a simple illustration of the lemma, observe:
\begin{equation*}
\left[\begin{array}{ccc|ccc}
1 & 0 & 0 & 1 & 0 & 0 \\
0 & 1 & 0 & 0 & 1 & 0 \\
0 & a & 1 & 0 & 0 & 1
\end{array}\right]
\rightsquigarrow 
\left[\begin{array}{ccc|ccc}
1 & 0 & 0 & 1 & 0 & 0 \\
0 & 1 & 0 & 0 & 1 & 0 \\
0 & 0 & 1 & 0 & -a & 1
\end{array}\right]
\end{equation*}

Now let $A$ be an $m\times n$ matrix, say
\begin{equation*}
A=
\begin{bmatrix}
a_{11} & a_{12} & \cdots & a_{1n} \\
a_{21} & a_{22} & \cdots & a_{2n} \\
\vdots & \vdots &  & \vdots \\
a_{m1} & a_{m2} & \cdots & a_{mn}
\end{bmatrix},
\end{equation*}
and assume $A$ can be row reduced to an upper triangular form using only row
operation 3. Thus, in particular, $a_{11}\neq 0$. Multiply on the left by 
\begin{equation*}
E_{1}=
\begin{bmatrix}
1 & 0 & \cdots & 0 \\
-
\frac{a_{21}}{a_{11}} & 1 & \cdots & 0 \\
\vdots & \vdots & \ddots & \vdots \\
-\frac{a_{m1}}{a_{11}} & 0 & \cdots & 1
\end{bmatrix}
\end{equation*}
This is the product of elementary matrices which make modifications in the
first column only. It is equivalent to taking $-a_{21}/a_{11}$ times the
first row and adding to the second. Then taking $-a_{31}/a_{11}$ times the
first row and adding to the third and so forth. The quotients in the first
column of the above matrix are the multipliers. Thus the result is of the
form
\begin{equation*}
E_{1}A=
\begin{bmatrix}
a_{11} & a_{12} & \cdots & a_{1n}^{\prime } \\
0 & a_{22}^{\prime } & \cdots & a_{2n}^{\prime } \\
\vdots & \vdots &  & \vdots \\
0 & a_{m2}^{\prime } & \cdots & a_{mn}^{\prime }
\end{bmatrix}
\end{equation*}
By assumption, $a_{22}^{\prime }\neq 0$ and so it is possible to use this
entry to zero out all the entries below it in the matrix on the right by
multiplication by a matrix of the form 
\begin{equation*}
E_{2}=
\begin{bmatrix}
1 & \vec{0} \\
\vec{0} & E
\end{bmatrix}
\end{equation*}
where $E$ is an $( m-1)\times ( m-1)$
matrix of the form
\begin{equation*}
E=\begin{bmatrix}
1 & 0 & \cdots & 0 \\
-\frac{a_{32}^{\prime }}{a_{22}^{\prime }} & 1 & \cdots & 0 \\
\vdots & \vdots & \ddots & \vdots \\
-\frac{a_{m2}^{\prime }}{a_{22}^{\prime }} & 0 & \cdots & 1
\end{bmatrix}
\end{equation*}
Again, the entries in the first column below the 1 are the multipliers.
Continuing this way, zeroing out the entries below the diagonal entries, finally leads us to
\begin{equation*}
E_{m-1}E_{n-2}\cdots E_{1}A=U
\end{equation*}
where $U$ is upper triangular. Each $E_{j}$ has all ones down the main diagonal and is lower triangular. Now we can multiply both sides by the inverses of the $E_{j}$ in the reverse order. This yields
\begin{equation*}
A=E_{1}^{-1}E_{2}^{-1}\cdots E_{m-1}^{-1}U
\end{equation*}

By Lemma \ref{lem:multipliermethodtriangularmatrices}, this implies that the product of those $E_{j}^{-1}$
is a lower triangular matrix having all ones down the main diagonal.

The above discussion and lemma explain how the multiplier
method works. The expressions
\begin{equation*}
-\frac{a_{21}}{a_{11}},-\frac{a_{31}}{a_{11}},\cdots, -\frac{a_{m1}}{a_{11}},
\end{equation*}
which we obtained in building $E_{1}$, and which we denote respectively by $M_{21},\cdots ,M_{m1}$ to save notation, are the multipliers.
\index{multipliers} Therefore, according to the lemma, to find $E_{1}^{-1}$ we simply write
\begin{equation*}
\begin{bmatrix}
1 & 0 & \cdots & 0 \\
-M_{21} & 1 & \cdots & 0 \\
\vdots & \vdots & \ddots & \vdots \\
-M_{m1} & 0 & \cdots & 1
\end{bmatrix}
\end{equation*}
Similar considerations apply to the other $E_{j}^{-1}.$ Thus $L$ is a
product of the form
\begin{equation*}
\begin{bmatrix}
1 & 0 & \cdots & 0 \\
-M_{21} & 1 & \cdots & 0 \\
\vdots & \vdots & \ddots & \vdots \\
-M_{m1} & 0 & \cdots & 1
\end{bmatrix}
\cdots 
\begin{bmatrix}
1 & 0 & \cdots & 0 \\
0 & 1 & \cdots & 0 \\
\vdots & 0 & \ddots & \vdots \\
0 & \cdots & -M_{m(m-1)} & 1
\end{bmatrix}, 
\end{equation*}
where each factor has at most one nonzero column, the position of which moves from left to right as we scan the above product of matrices from left to right. It follows from what we know  about the effect of multiplying on the left by an elementary matrix that the above product is of the form
\begin{equation*}
\begin{bmatrix}
1 & 0 & \cdots & 0 & 0 \\
-M_{21} & 1 & \cdots & 0 & 0 \\
\vdots & -M_{32} & \ddots & \vdots & \vdots \\
-M_{(M-1)1} & \vdots & \cdots & 1 & 0 \\
-M_{M1} & -M_{M2} & \cdots & -M_{MM-1} & 1
\end{bmatrix}
\end{equation*}

To sum up the procedure, beginning at the left column and moving toward the right, you
simply insert, into the corresponding position in the identity matrix, $-1$
times the multiplier which was used to zero out an entry in that position
below the main diagonal in $A,$ while retaining the main diagonal which
consists entirely of ones. This will give us $L$ as we create $U$ using row operation 3.

We now return to Example \ref{ex:usingLU}, to understand why we could not use the multiplier method to find an $LU$ factorization of the coefficient matrix.  Suppose we had written 
$$\begin{bmatrix}1&0&0\\0&1&0\\0&0&1\end{bmatrix}\begin{bmatrix}0&12&-18\\1&2&-3\\1&-2&1\end{bmatrix}.$$
Our first step of Gaussian elimination would require a row swap to get a nonzero entry into the top left corner (we swapped rows 1 and 2 in Example \ref{ex:usingLU}).  Unfortunately, the elementary matrix that accomplishes this, $P=\begin{bmatrix}0&1&0\\1&0&0\\0&0&1\end{bmatrix}$ is NOT lower triangular.  So we cannot apply Lemma \ref{lem:multipliermethodtriangularmatrices} to generate a lower triangular $L$.

If the elementary matrices used to reduce our matrix to row-echelon form are all lower triangular, then we can find an $LU$ factorization.  But what about the general case?

\begin{theorem}\label{th:LUPA}
Suppose an $m \times n$ matrix $A$ is transformed to a row-echelon matrix $U$ using Gaussian elimination. Let $P_1, P_2, \ldots, P_s$ be the elementary matrices corresponding (in order) to the row interchanges used,
and write $P=P_s \cdots P_2 P_1$. (If no interchanges are used take $P = I_M$.) Then:
\begin{enumerate}
\item $PA$ is the matrix obtained from $A$ by doing these interchanges (in order) to A.
\item $PA$ has an $LU$ factorization.
\end{enumerate}
\end{theorem}

For a proof of the above theorem, the reader is referred to [Nicholson].

The matrix $P$ in the above theorem is called a \dfn{permutation matrix}.  These matrices have other important applications, as we will see later.


\section*{Practice Problems}

\begin{problem}\label{prob:LU1}
Use the given $LU$ factorization of the coefficient matrix to solve the system of equations.
$$\begin{array}{ccccccc}
      x & +&2y&+&3z&= &5 \\
	 2x&+&3y&+&z&=&6\\
     3x& +&5y&+&4z&=&11
    \end{array}$$
Observe that an $LU$ factorization of the coefficient matrix is
\[
\begin{bmatrix}
1 & 2 & 3 \\
2 & 3 & 1 \\
3 & 5 & 4\end{bmatrix}
=
\begin{bmatrix}
1 & 0 & 0 \\
2 & 1 & 0 \\
3 & 1 & 1
\end{bmatrix} 
\begin{bmatrix}
1 & 2 & 3 \\
0 & -1 & -5 \\
0 & 0 & 0
\end{bmatrix}
\]
\end{problem}

\begin{problem}
Find the $LU$ factorization of the coefficient matrix and use it to solve the system of equations.
\begin{problem}\label{prob:LU2a}

$$
\begin{array}{ccccc}
      x& +&2y&=&5\\
      2x & +&3y&= &6 
    \end{array}
$$
$$L=\begin{bmatrix}\answer{1}&0\\\answer{2}&\answer{1}\end{bmatrix},\quad U=\begin{bmatrix}\answer{1}&\answer{2}\\0&\answer{-1}\end{bmatrix}$$
$$x=\answer{-3}, y=\answer{4}$$
\end{problem}

\begin{problem}\label{prob:LU2b}
$$\begin{array}{ccccccc}
      x & +&2y&+&z&= &1 \\
	 & &y&+&3z&=&2\\
     2x& +&3y&&&=&6
    \end{array}$$
    $$L=\begin{bmatrix}\answer{1}&0&0\\\answer{0}&\answer{1}&0\\\answer{2}&\answer{-1}&\answer{1}\end{bmatrix},\quad U=\begin{bmatrix} \answer{1}&\answer{2}&\answer{1}\\0&\answer{1}&\answer{3}\\0&0&\answer{1}\end{bmatrix}$$
    $$x=\answer{27},y=\answer{-16}, z=\answer{6}$$
\end{problem}

%\begin{problem}
%\begin{equation*}
%\begin{array}{c}
%x+2y+3z=5 \\
%2x+3y+z=6 \\
%x-y+z=2
%\end{array}
%\end{equation*}
%\end{problem}
\end{problem}

\begin{problem}\label{prob:LU4}
Is there only one $LU$ factorization for a given matrix?

\begin{hint}Consider the equation
$$
\begin{bmatrix}0 & 1 \\0 & 1\end{bmatrix}=\begin{bmatrix}1& 0 \\1 & 1\end{bmatrix} \begin{bmatrix}0 & 1 \\0 & 0\end{bmatrix}
$$

Look for all possible $LU$ factorizations.
\end{hint}
\end{problem}

\begin{problem}\label{prob:LU5}
Can you show that every permutation matrix is invertible?  If so, What does the inverse of a permutation matrix look like? (Recall that a permutation matrix is matrix $P$ of Theorem \ref{th:LUPA}.)
\end{problem}

\section*{Text Source}
The text in this section is an adaptation of Section 2.2 of Ken Kuttler's \href{https://open.umn.edu/opentextbooks/textbooks/a-first-course-in-linear-algebra-2017}{\it A First Course in Linear Algebra}. (CC-BY)

Ken Kuttler, {\it  A First Course in Linear Algebra}, Lyryx 2017, Open Edition, p. 99-106.

\section*{Bibliography}

[Nicholson] W. Keith Nicholson, {\it Linear Algebra with Applications}, Lyryx 2018, Open Edition, pp. 123--127.

\end{document}
